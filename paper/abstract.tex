% This paper presents the first compositional definition of sequential
% composition that applies to a relaxed memory model weak enough to allow
% efficient implementation on Arm.  We extend the denotational model of pomsets
% with preconditions with predicate transformers. Previous work has shown that
% pomsets with preconditions are a model of concurrent composition, and that
% predicate transformers are a model of sequential composition.  This paper
% show how they can be combined.

% Program logics and semantics tell us that in order to derive
% \begin{math}
%   ((S_1\SEMI S_2), \sigma_0) \Downarrow \sigma_2,
% \end{math}
% we derive 
% \begin{math}
%   (S_1, \sigma_0) \Downarrow \sigma_1
% \end{math}
% and then
% \begin{math}
%   (S_2, \sigma_1) \Downarrow \sigma_2.
% \end{math}
Program logics and semantics tell us that when executing $(S_1; S_2)$ starting in state $s_0$,
we execute $S_1$ in $s_0$ to arrive at $s_1$, then execute $S_2$ in $s_1$ to
arrive at the final state $s_2$.  This is, of course, an abstraction.  Processors
execute instructions out of order, due to pipelines and caches, and compilers
reorder programs even more dramatically.  All of this reordering is meant to
be unobservable in single-threaded code, but is observable in multi-threaded code.
A formal attempt to understand the resulting mess is known
as a ``relaxed memory model.''  The relaxed memory models that have been
proposed to date either fail to address sequential composition directly, or overly
restrict processors and compilers.

To support sequential composition while targeting modern hardware, we propose adding families of predicate
transformers to the existing model of ``Pomsets with Preconditions,'' which
already supports parallel composition.  When composing $(S_1;S_2)$, the predicate
transformers used to validate the preconditions of events in $S_2$ are chosen
based on the semantic dependencies from events in $S_1$ to events in $S_2$.  Our model
retains the good properties of prior work, including efficient
implementation on Arm8, support for compiler optimizations, support for
logics that prove the absence of thin-air behaviors, and a local data-race-freedom theorem.


\endinput

Program logics and semantics tell us that when executing (C1; C2) starting in
state s0, we execute C1 in s0 to arrive at s1, then execute C2 in s1 to
arrive at the final state s2. This is, of course, an abstraction. Processors
execute instructions out of order, due to pipelines and caches, and compilers
reorder programs even more dramatically. All of this reordering is meant to
be unobservable in single-threaded code, but is observable in multi-threaded
code. A formal attempt to understand the resulting mess is known as a
"relaxed memory model." The relaxed memory models that have been proposed to
date either fail to address sequential composition, or overly restrict
processors and compilers.

To support sequential composition while targeting modern hardware, we propose
adding families of predicate transformers to the existing model of "Pomsets
with Preconditions," which already supports parallel composition.  When
composing (C1;C2), the predicate transformers used to validate the
preconditions of events in $C2$ are chosen based on the semantic
dependencies from events in $C1$ to events in $C2$.  Our model retains the
good properties of the prior work, including efficient implementation on
Arm8, support for compiler optimizations, support for logics that prove the
absence of thin-air behaviors, and a local data race freedom theorem.
