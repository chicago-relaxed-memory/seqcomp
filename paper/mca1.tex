\section{\PwTmcaTITLE{}: Pomsets with Predicate Transformers for MCA}
\label{sec:mca}

We derive a model of concurrent computation by adding \emph{parallel
  composition} and \emph{reads-from} to \reffig{fig:seq}.  To model coherence
and synchronization, we add \emph{delay} to the rule for sequential
composition.  For \mca{} architectures, it is sufficient to encode delay in
the pomset order.  The resulting model, \PwTmca{1}, supports optimal lowering
for relaxed access on \armeight{}, but requires extra synchronization for
acquiring reads.

A variant, \PwTmca{2}, supports optimal lowering for all access modes on
\armeight{}.  To achieve this, \PwTmca{2} drops the global requirement that
\emph{reads-from} implies pomset order \eqref{pom-rf-le}.  The models are the
same, except for \emph{internal reads}, where a thread reads its own write.

The lowering proofs can be found in \textsection\ref{sec:arm}.  The proofs
use recent alternative characterizations of \armeight{} \cite{armed}.

\subsection{\PwTmcaTITLE{1}}
\label{sec:mca1}

We define \PwTmca{1} by extending \refdef{def:pomset} and \reffig{fig:seq}.
The definition uses several relations over actions---$\rmatches$, $\rblocks$
and $\rdelays$---as well a distinguished set of $\sread$ actions; see
\textsection\ref{sec:actions}.
\begin{definition}
  \label{def:pwt:mca1}
  A \PwTmca{1} is a \PwT{} (\refdef{def:pomset}) equipped
  with a relation $\rrfx$ such that 
  \begin{enumerate}[,label=(\textsc{m}\arabic*),ref=\textsc{m}\arabic*]
    \setcounter{enumi}{\value{Brf}}
  \item \label{pom-rf} \makecounter{rf} ${\rrfx} \subseteq \aEvs\times\aEvs$
    is an injective relation capturing \emph{reads-from}, such that
    % \end{enumerate}
    % A pomset is a \emph{candidate} if there is an injective relation
    % ${\rrfx} : \aEvs\times\aEvs$, capturing \emph{reads-from}, such that
    \begin{enumerate}
      % \begin{enumerate}[,label=(\textsc{i}\arabic*),ref=\textsc{i}\arabic*]
      % \item \label{rf-injective}
      %   if $\bEv\xrfx\aEv$ and $\cEv\xrfx\aEv$ then $\bEv=\cEv$, that is,
      %   ${\rrfx}$ is injective,
    \item \label{pom-rf-match} if $\bEv\xrfx\aEv$ then
      $\labeling(\bEv) \rmatches \labeling(\aEv)$,
    \item \label{pom-rf-block} if $\bEv\xrfx\aEv$ and
      $\labeling(\cEv) \rblocks \labeling(\aEv)$ then either $\cEv\le\bEv$ or
      $\aEv\le\cEv$,
    \item \label{pom-rf-le} if $\bEv\xrfx\aEv$ then $\bEv\lt\aEv$.
    \end{enumerate}
  \end{enumerate}

  \noindent
  \label{def:pwt:mca:complete}
  A \PwTmca{} is \emph{complete} if %it is a complete \PwT{} (\refdef{def:pomset}) and
  \begin{multicols}{2}
    \begin{enumerate}[,label=(\textsc{c}\arabic*),ref=\textsc{c}\arabic*]
    \item[\eqref{top-kappa}]
      \eqref{top-term}\;
      as in \refdef{def:pomset},
      \setcounter{enumi}{\value{Brf}}
    \item \label{top-rf}
      if $\labeling(\aEv)$ is a $\sread$ then there is some $\bEv\xrfx\aEv$.
    \end{enumerate}
  \end{multicols}
  \bigskip
  
  %The semantic rules are as follows:
  % \begin{figure}
%   \bookmark{\PwTmcaTITLE{1} Semantics}
%   \raggedright
\noindent
\begin{minipage}{1.0\linewidth}
  If $\aPS \in \sLPAR{\aPSS_1}{\aPSS_2}$ then  
  $(\exists\aPS_1\in\aPSS_1)$ $(\exists\aPS_2\in\aPSS_2)$
  \begin{multicols}{2}
    \begin{enumerate}[topsep=0pt,label=(\textsc{p}\arabic*),ref=\textsc{p}\arabic*]
    \item \label{par-E}
      $\aEvs = (\aEvs_1\uplus\aEvs_2)$,
    \item \label{par-lambda}
      ${\labeling}=\PBR{{\labeling_1}\cup {\labeling_2}}$, 
      \stepcounter{enumi}
    \item[] \labeltext[\textsc{p}3]{}{par-kappa}
      \begin{enumerate}[leftmargin=0pt]
      \item \label{par-kappa1}
        if $\aEv\in\aEvs_1$ then $\labelingForm(\aEv) \riff \labelingForm_1(\aEv)$,
      \item \label{par-kappa2}
        if $\aEv\in\aEvs_2$ then $\labelingForm(\aEv) \riff \labelingForm_2(\aEv)$,
      \end{enumerate}
    \item \label{par-tau}
      $\aTr{\bEvs}{\bForm} \riff \aTr[1]{\bEvs}{\bForm}$,
    \item \label{par-term}
      $\aTerm \riff \aTerm[1]\land\aTerm[2]$,
    \item \label{par-le}
        ${\le}\rsupset{\le_1}{\le_2}$,
    \item \label{par-rf}
        ${\rrfx}\rsupset{\rrfx_1}{\rrfx_2}$.
      \stepcounter{enumi}
    \end{enumerate}
  \end{multicols}
\end{minipage}
\smallskip

  \noindent
  If $\aPS \in \sSEMI{\aPSS_1}{\aPSS_2}$ then $(\exists\aPS_1\in\aPSS_1)$
  $(\exists\aPS_2\in\aPSS_2)$
  % let $\aTerm[1](\aEv) = \aTerm[1]$ if $\labeling_2(\aEv)$ is a $\srelease$
  % and $\aTerm[1](\aEv) = \TRUE$ otherwise,\\
  % let $\labelingForm'_2(\aEv)=\aTr[1]{\Cdown{\aEv}}{\labelingForm_2(\aEv})$,
  % where
  % $\Cdown{\aEv}=\{ \cEv \mid \cEv \lt \aEv \}$%, if $\labeling(\aEv)$ is a write, and $\Cdown{\aEv}=\aEvs_1$, otherwise
  %
  % \begin{math}
  %   \Cdown{\aEv}=
  %   \begin{cases}
  %     \{ \cEv \mid \cEv \lt \aEv \} & \textif \labeling(\aEv) \;\text{is a write}
  %     \\
  %     \aEvs_1 & \textotherwise
  %   \end{cases}
  % \end{math}
  % let
  % \begin{math}
  %   \smash{
  %   \downarrow\aEv=\begin{cases}
  %     \{ \cEv \mid \cEv \lt \aEv \} &\textif \labeling(\aEv) \;\text{is a write}\\
  %     \aEvs_1 &\textotherwise
  %   \end{cases}}
  % \end{math}
  \begin{multicols}{2}
    \begin{enumerate}[topsep=0pt,label=(\textsc{s}\arabic*),ref=\textsc{s}\arabic*]
    \item[\eqref{seq-E}]
      \eqref{seq-lambda}\;
      \eqref{seq-kappa}\;
      \eqref{seq-tau}\;
      \eqref{seq-term}\;
      \eqref{seq-le}\;
      as in \reffig{fig:seq},
      \setcounter{enumi}{\value{le}}
    % \item \label{seq-E}
    %   $\aEvs = (\aEvs_1\cup\aEvs_2)$,
    % \item \label{seq-lambda}
    %   ${\labeling}=\PBR{{\labeling_1}\cup {\labeling_2}}$, 
    %   \stepcounter{enumi}
    % \item[] \labeltext[\textsc{s}3]{}{seq-kappa}
    %   \begin{enumerate}[leftmargin=0pt]
    %   \item \label{seq-kappa1}
    %     if $\aEv\in\aEvs_1\setminus\aEvs_2$ then $\labelingForm(\aEv) \riff \labelingForm_1(\aEv)$,
    %   \item \label{seq-kappa2}
    %     if $\aEv\in\aEvs_2\setminus\aEvs_1$ then $\labelingForm(\aEv) \riff \labelingForm'_2(\aEv) \land \aTerm[1](\aEv)$,
    %   \item \label{seq-kappa12}
    %     % if $\aEv{\in}\aEvs_1{\cap}\aEvs_2$ then $\labelingForm(\aEv) \riff \labelingForm_1(\aEv){\lor}\labelingForm'_2(\aEv)$,\\
    %     if $\aEv\in\aEvs_1\cap\aEvs_2$ then $\labelingForm(\aEv) \riff (\labelingForm_1(\aEv)\lor\labelingForm'_2(\aEv)) \land \aTerm[1](\aEv)$,
    %     % where
    %     % $\labelingForm'_2(\aEv)=\aTr[1]{\Cdown{\aEv}}{\labelingForm_2(\aEv})$,
    %     % where $\Cdown{\aEv}=\{ \cEv \mid \cEv \lt \aEv \}$ if $\labeling(\aEv)$
    %     % is a write, and $\Cdown{\aEv}=\aEvs_1$, otherwise,
    %   % \item \label{seq-kappa-release}
    %   %   if $\labeling_2(\aEv)$ is a $\srelease$ then $\labelingForm(\aEv) \riff \aTerm[1]$,
    %   \end{enumerate}
    % \item \label{seq-tau}
    %   $\aTr{\bEvs}{\bForm} \riff \aTr[1]{\bEvs}{\aTr[2]{\bEvs}{\bForm}}$,
    % \item \label{seq-term}
    %   $\aTerm \riff \aTerm[1]\land\aTr[1]{}{\aTerm[2]}$,
    %   \stepcounter{enumi}
    \item[]
      % \labeltext[\textsc{s}6]{}{seq-le}
      \begin{enumerate}[leftmargin=0pt]
      % \item \label{seq-le-extends}
      %   ${\le}\rsupset{\le_1}{\le_2}$, 
      \item \label{seq-le-delays}
        if $\labeling_1(\bEv) \rdelays \labeling_2(\aEv)$ then
        %$\labelingForm_1(\bEv) \land \labelingForm_2(\aEv)$ is satisfiable then
        $\bEv\le\aEv$,
        % either $\bEv\le\aEv$ or $\labelingForm(\bEv)\land\labelingForm(\aEv)$
        % is unsatisfiable.
      \end{enumerate}
    %   \stepcounter{enumi}
      \item \label{seq-rf}
        ${\rrfx}\rsupset{\rrfx_1}{\rrfx_2}$.
    % \item[] \labeltext[\textsc{s}7]{}{seq-rf}
    %   \begin{enumerate}[leftmargin=0pt]
    %   \item \label{seq-rf-extends}
    %     ${\rrfx}\rextends{\rrfx_1}{\rrfx_2}$,
    %   \item \label{seq-rf-le}
    %     if $\bEv\in\aEvs_1$, $\aEv\in\aEvs_2$ and $\bEv\xrfx\aEv$ then $\bEv\le\aEv$.
    %   \end{enumerate}
      %   % if $\cEv\in\aEvs_1$ and $(\bEv,\aEv)\in\PBR{{\rrfx}\cap(\aEvs_1\times \aEvs_2)}$ then 
      %   % either $\labeling(\cEv) \reorderca \labeling(\bEv)$
      %   % or $\cEv\le\bEv$,
    \end{enumerate}
  \end{multicols}
  \smallskip

  \noindent
  If $\aPS \in \sIF{\aForm}\sTHEN\aPSS_1\sELSE\aPSS_2\sFI$ then
  $(\exists\aPS_1\in\aPSS_1)$ $(\exists\aPS_2\in\aPSS_2)$
  \begin{multicols}{2}
    \begin{enumerate}[topsep=0pt,label=(\textsc{i}\arabic*),ref=\textsc{i}\arabic*]
    \item[\eqref{if-E}]
      \eqref{if-lambda}\;
      \eqref{if-kappa}\;
      \eqref{if-tau}\;
      \eqref{if-term}\; 
      \eqref{if-le}\; as in \reffig{fig:seq},
      \setcounter{enumi}{\value{le}}
    % \item \label{if-E}
    %   $\aEvs = (\aEvs_1\cup\aEvs_2)$,
    % \item \label{if-lambda}
    %   ${\labeling}=\PBR{{\labeling_1}\cup {\labeling_2}}$, 
    %   % \item[(\ref{par-E}--\ref{par-le})] as for $\sLPAR{}{}$,
    %   \stepcounter{enumi}
    % \item[] \labeltext[\textsc{i}3]{}{if-kappa}
    %   \begin{enumerate}[leftmargin=0pt]
    %   \item \label{if-kappa1}
    %     if $\aEv\in\aEvs_1\setminus\aEvs_2$ then $\labelingForm(\aEv) \riff \aForm\land\labelingForm_1(\aEv)$,
    %   \item \label{if-kappa2}
    %     if $\aEv\in\aEvs_2\setminus\aEvs_1$ then $\labelingForm(\aEv) \riff \neg\aForm\land\labelingForm_2(\aEv)$, 
    %   \item \label{if-kappa12}
    %     if $\aEv\in\aEvs_1\cap\aEvs_2$\\ then
    %     % $\labelingForm(\aEv) \riff (\aForm\limplies\labelingForm_1(\aEv))\land(\neg\aForm\limplies\labelingForm_2(\aEv))$,
    %     $\labelingForm(\aEv) \riff (\aForm\land\labelingForm_1(\aEv))\lor(\neg\aForm\land\labelingForm_2(\aEv))$,
    %   \end{enumerate}
    % \item \label{if-tau}
    %   % $\aTr{\bEvs}{\bForm} \riff (\aForm\limplies\aTr[1]{\bEvs}{\bForm})\land(\neg\aForm\limplies\aTr[2]{\bEvs}{\bForm})$,
    %   $\aTr{\bEvs}{\bForm} \riff (\aForm\land\aTr[1]{\bEvs}{\bForm})\lor(\neg\aForm\land\aTr[2]{\bEvs}{\bForm})$,
    % \item \label{if-term}
    %   % $\aTerm \riff (\aForm\limplies\aTerm[1])\land(\neg\aForm\limplies\aTerm[2])$.
    %   $\aTerm \riff (\aForm\land\aTerm[1])\lor(\neg\aForm\land\aTerm[2])$.
    %   \stepcounter{enumi}
    % \item[]
    %   % \labeltext[\textsc{i}6]{}{if-le}
    %   \begin{enumerate}[leftmargin=0pt]
    %     % \item \label{if-le-rf}
    %     %   if $\bEv\in\aEvs_1$ and $\aEv\in\aEvs_2$ and $\bEv\xrfx\aEv$ then $\bEv\le\aEv$,
    %   \item \label{if-le-extends}
    %     ${\le}\rextends{\le_1}{\le_2}$,
    %   \item \label{if-le-subset}
    %     ${\le}\rsubset{\le_1}{\le_2}$,
    %   \end{enumerate}
    %   \stepcounter{enumi}
      \item \label{if-rf}
        ${\rrfx}\rsupset{\rrfx_1}{\rrfx_2}$.
    % \item[] \labeltext[\textsc{i}7]{}{if-rf}
    %   \begin{enumerate}[leftmargin=0pt]
    %   \item \label{if-rf-extends}
    %     ${\rrfx}\rextends{\rrfx_1}{\rrfx_2}$,
    %   \item \label{if-rf-le}
    %     ${\rrfx}\rsubset{\rrfx_1}{\rrfx_2}$.
    %   \end{enumerate}
    \end{enumerate}
  \end{multicols}
  \smallskip

  % \noindent
  % If $\aPS\in\sLET{\aReg}{\aExp}$ then $\aEvs = \emptyset$ and
  % $\aTr{\bEvs}{\bForm} \riff \bForm[\aExp/\aReg]$.
  % \smallskip

  % \noindent
  % If $\aPS \in \sLOAD[\amode]{\aReg}[\ascope]{\aLoc}[\aThrd]$ then
  % $(\exists\aVal\in\Val)$
  % \begin{multicols}{2}
  %   \begin{enumerate}[topsep=0pt,label=(\textsc{r}\arabic*),ref=\textsc{r}\arabic*]
  %   \item \label{read-E}
  %     if $\bEv,\aEv\in\aEvs$ then $\bEv=\aEv$,
  %   \item \label{read-lambda}
  %     $\labelingAct(\aEv) = \DR[\amode]{\aLoc}[\ascope]{\aVal}[\aThrd]$,
  %   \item \label{read-kappa}
  %     $\labelingForm(\aEv) \riff \Q{\aLoc}$,
  %     %$\labelingForm(\aEv) \riff \TRUE$,
  %     \stepcounter{enumi}
  %   \item[] \labeltext[\textsc{r}4]{}{read-tau}
  %     \begin{enumerate}[leftmargin=0pt]
  %     \item \label{read-tau-dep}
  %       if $\aEvs\neq\emptyset$ and $(\aEvs\cap\bEvs)\neq\emptyset$ then
  %       \begin{math}
  %         \aTr{\bEvs}{\bForm} \riff
  %         \aVal{=}\aReg
  %         \limplies \bForm
  %       \end{math},    
  %     \item \label{read-tau-ind}
  %       if $\aEvs\neq\emptyset$ and $(\aEvs\cap\bEvs)=\emptyset$ then
  %       \begin{math}
  %         \aTr{\bEvs}{\bForm} \riff
  %         \PBR{\aVal{=}\aReg \lor \aLoc{=}\aReg} \limplies
  %         \bForm,
  %       \end{math}
  %     \item \label{read-tau-empty}
  %       if $\aEvs=\emptyset$ then
  %       \begin{math}
  %         \aTr{\bEvs}{\bForm} \riff
  %         % \PBR{\aVal{=}\aReg \lor \aLoc{=}\aReg} \limplies
  %         \bForm,
  %       \end{math}
  %     \end{enumerate}
  %     \stepcounter{enumi}
  %   \item[] \labeltext[\textsc{r}5]{}{read-term}
  %     \begin{enumerate}[leftmargin=0pt]
  %     \item \label{read-term-nonempty}
  %       if $\aEvs\neq\emptyset$ or $\amode\lemode\mRLX$ then $\aTerm \riff \TRUE$. 
  %     \item \label{read-term-empty}
  %       if $\aEvs=\emptyset$ and $\amode\gemode\mACQ$ then $\aTerm \riff \FALSE$. 
  %     \end{enumerate}      
  %   % \item \label{read-term}
  %   %   if $\aEvs=\emptyset$ and $\amode\gemode\mACQ$ then $\aTerm \riff \FALSE$. 
  %   \end{enumerate}
  % \end{multicols}
  % \smallskip

  % \noindent
  % If $\aPS \in \sSTORE[\amode]{\aLoc}[\ascope]{\aExp}[\aThrd]$ then
  % $(\exists\aVal\in\Val)$
  % \begin{multicols}{2}
  %   \begin{enumerate}[topsep=0pt,label=(\textsc{w}\arabic*),ref=\textsc{w}\arabic*]
  %   \item \label{write-E}
  %     if $\bEv,\aEv\in\aEvs$ then $\bEv=\aEv$,
  %   \item \label{write-lambda}
  %     $\labelingAct(\aEv) = \DW[\amode]{\aLoc}[\ascope]{\aVal}[\aThrd]$,
  %   \item \label{write-kappa}
  %     \begin{math}
  %       \labelingForm(\aEv) \riff
  %       \aExp{=}\aVal
  %     \end{math},    
  %   % \item \label{write-tau}
  %   %   % if $(\aEvs\cap\bEvs)\neq\emptyset$ then
  %   %   \begin{math}
  %   %     \aTr{\bEvs}{\bForm} \riff 
  %   %     \bForm
  %   %     [\aExp/\aLoc]
  %   %     % \land\aExp{=}\aVal
  %   %   \end{math},
  %     \stepcounter{enumi}      
  %   \item[] \labeltext[\textsc{w}5]{}{write-tau}
  %     \begin{enumerate}[leftmargin=0pt]
  %     \item \label{write-tau-nonempty}
  %       if $\aEvs\neq\emptyset$ then 
  %       \makebox[0cm][l]{%
  %       \begin{math}
  %         \aTr{\bEvs}{\bForm} \riff 
  %         \bForm
  %         [\aExp/\aLoc][\aExp{=}\aVal/\Q{\aLoc}]
  %       \end{math}}
  %     \item \label{write-tau-empty}
  %       if $\aEvs=\emptyset$ then 
  %       \begin{math}
  %         \aTr{\bEvs}{\bForm} \riff 
  %         \bForm
  %         [\aExp/\aLoc][\FALSE/\Q{\aLoc}]
  %       \end{math}
  %     \end{enumerate}
  %     \stepcounter{enumi}      
  %   \item[] \labeltext[\textsc{w}5]{}{write-term}
  %     \begin{enumerate}[leftmargin=0pt]
  %     \item \label{write-term-nonempty}
  %       if $\aEvs\neq\emptyset$ then $\aTerm \riff \aExp{=}\aVal$,
  %     \item \label{write-term-empty}
  %       if $\aEvs=\emptyset$ then $\aTerm \riff \FALSE$.
  %     \end{enumerate}
  %   \end{enumerate}
  % \end{multicols}
  % \smallskip

  % \noindent
  % If $\aPS \in \sFENCE[\ascope]{\amode}[\aThrd]$ then
  % % $(\exists\aLocs\subseteq\Loc)$
  % % $(\exists\bmode\in\{\amode,\mRLX\})$
  % \begin{multicols}{2}
  %   \begin{enumerate}[topsep=0pt,label=(\textsc{f}\arabic*),ref=\textsc{f}\arabic*]
  %   \item \label{fence-E}
  %     % $\aEvs\neq\emptyset$ and
  %     if $\bEv,\aEv\in\aEvs$ then $\bEv=\aEv$,
  %   \item \label{fence-lambda}
  %     $\labelingAct(\aEv) = \DF[\ascope]{\amode}[\aThrd]$,
  %   \item \label{fence-kappa}
  %     $\labelingForm(\aEv) \riff \TRUE$,
  %   \item \label{fence-tau}
  %     $\aTr{\bEvs}{\bForm} \riff \bForm$,
  %     \stepcounter{enumi}      
  %   \item[] \labeltext[\textsc{f}5]{}{fence-term}
  %     \begin{enumerate}[leftmargin=0pt]
  %     \item \label{fence-term-nonempty}
  %       if $\aEvs\neq\emptyset$ then $\aTerm \riff \TRUE$,
  %     \item \label{fence-term-empty}
  %       if $\aEvs=\emptyset$ then $\aTerm \riff \FALSE$.
  %     \end{enumerate}
  %   % \item \label{fence-term}
  %   %   if $\aEvs=\emptyset$ then $\aTerm \riff \FALSE$.
  %   \end{enumerate}
  % \end{multicols}

%   \caption{\PwTmcaTITLE{1} Semantics}
%   \label{fig:mca1}
% \end{figure}

\end{definition}
Let
\begin{math}
  \sem{\aCmd_1 \RPAR \aCmd_2} = \sPAR{\sem{\aCmd_1}}{\sem{\aCmd_2}}.
\end{math}
We write $\semmca{1}{}$ for the semantic function when it is unclear from context.

In complete pomsets, $\rrfx$ must pair every read with a matching write
\eqref{top-rf}.  The requirements \ref{pom-rf-match}, \ref{pom-rf-block}, and
\ref{pom-rf-le} guarantee that reads are \emph{fulfilled}, as in
\cite[\textsection 2.7]{DBLP:journals/pacmpl/JagadeesanJR20}.

The semantic rules are mostly straightforward: Parallel composition is disjoint
union, and all constructs respect reads-from.  The monoid laws
(\reflem{lem:monoid}) extend to parallel composition, with $\SKIP$ as right
unit only due to the asymmetry of \ref{par-tau}.

Only \ref{seq-le-delays} requires explanation.
% \begin{lemma}
%   \label{lem:rf:implies:le}
%   For any $\aPS$ in the range of $\sembase{}$, $\bEv\xrfx\aEv$ implies
%   $\bEv\le\aEv$.
%   \vspace{-.5\baselineskip}
%   \begin{proof}
%     Induction on the definition of $\sembase{}$, using \ref{par-rf-le1}, \ref{par-rf-le2}, and \ref{pom-rf-le}.
%   \end{proof}
% \end{lemma}
% \footnote{The
%   basic model would be the same if we move $\rrfx$ from the model itself to
%   be existentially quantified in the definition of top-level pomset, along
%   with \ref{pom-rf-match} and \ref{pom-rf-block}.  This was the approach of
%   \citeauthor{DBLP:journals/pacmpl/JagadeesanJR20} We include $\rrfx$
%   explicitly for use in \textsection\ref{sec:arm2}, where we introduce a
%   variant semantics $\frf{\semrr{}}$ where \ref{pom-rf-le} is not required.}
From \refdef{def:actions}, recall that $\aAct \rdelaysdef \bAct$ if
$\aAct\eqreorderco\bAct$ or $\aAct\reorderra\bAct$ or
$\aAct\eqreordersc\bAct$.  \ref{seq-le-delays} guarantees that sequential
order is enforced between conflicting accesses of the same location
($\eqreorderco$), into a release and out of an acquire ($\reorderra$), and
between SC accesses ($\eqreordersc$).  Combined with the fulfillment
requirements (\ref{pom-rf-match}, \ref{pom-rf-block} and \ref{pom-rf-le}),
these ensure coherence, publication, subscription and other idioms.  For
example, consider the following:\footnote{We use different colors for arrows representing order:
  \begin{multicols}{2}
    \begin{itemize}  
    \item \makebox{$\bEv\xwki\aEv$} arises from $\eqreorderco$ \eqref{seq-le-delays},
    \item \makebox{$\bEv\xsync\aEv$} arises from $\reorderra$ or $\eqreordersc$ \eqref{seq-le-delays},
    \item \makebox{$\bEv\xpo\aEv$} arises from \makebox[0pt][l]{control/data/address \emph{dependency} (\ref{seq-kappa}, definition of $\labelingForm'_2(\bEv)$),}
    \item \makebox{$\bEv\xrf\aEv$} arises from \emph{reads-from} \eqref{pom-rf-match},
      % \item \makebox{$\bEv\xsyncsc\aEv$} arises from \emph{strong fencing} \eqref{cand-lesync-sc},
    \item \makebox{$\bEv\xwk\aEv$} arises from \emph{blocking} \eqref{pom-rf-block}.
    \end{itemize}    
  \end{multicols}
  In \PwTmca{2}, it is possible for $\rrf$ to contradict
  $\lt$.  In this case, we use a dotted arrow for $\rrf$: $\bEv\xrfint\aEv$
  indicates that $\aEv\lt\bEv$.}
\begin{gather*}
  \taglabel{pub}
  \begin{gathered}    
    \PW{x}{0}\SEMI 
    \PW{x}{1}\SEMI \PW[\mREL]{y}{1} \PAR \PR[\mACQ]{y}{r}\SEMI \PR{x}{s}
    \\[-.4ex]
    \nonumber
    \hbox{\begin{tikzinline}[node distance=1.5em]
        \event{wx0}{\DW{x}{0}}{}
        \event{wx1}{\DW{x}{1}}{right=of wx0}
        \event{wy1}{\DW[\mREL]{y}{1}}{right=of wx1}
        \event{ry1}{\DR[\mACQ]{y}{1}}{right=2.5em of wy1}
        \event{rx0}{\DR{x}{0}}{right=of ry1}
        \sync{wx1}{wy1}
        \sync{ry1}{rx0}
        \rf{wy1}{ry1}
        \wk[out=170,in=10]{rx0}{wx1}
        \wki{wx0}{wx1}
      \end{tikzinline}}
  \end{gathered}
\end{gather*}
The execution is disallowed due to the cycle.  All of the order shown is
required at top-level: The intra-thread order comes from \ref{seq-le-delays}:
$\DWP{x}{0}\xwki \DWP{x}{1}$ is required by $\eqreorderco$.
$\DWP{x}{1}\xsync \DWP[\mREL]{y}{1}$ and $\DRP[\mACQ]{y}{1}\xsync \DRP{x}{0}$ are
required by $\reorderra$.  The cross-thread order is required by fulfillment:
\ref{top-rf} requires that all top-level reads are in the image of $\xrfx$.
\ref{pom-rf-match} ensures that $\DWP[\mREL]{y}{1}\xrfx \DRP[\mACQ]{y}{1}$, and
\ref{pom-rf-le} subsequently ensures that
$\DWP[\mREL]{y}{1}\lt \DRP[\mACQ]{y}{1}$.  The \emph{antidependency}
$\DRP{x}{0}\xwk \DWP{x}{1}$ is required by \ref{pom-rf-block}.  (Alternatively,
we could have $\DWP{x}{1}\xwk \DWP{x}{0}$, again resulting in a cycle.)

The semantics gives the expected results for store buffering and load buffering,
as well as litmus tests involving fences and SC access.  The model of
coherence is weaker than C11, in order to support common subexpression
elimination, and stronger than Java, in order to support local reasoning
about data races.  For further examples, see \textsection\ref{sec:extras} and
\cite[\textsection3.1]{DBLP:journals/pacmpl/JagadeesanJR20}.

Lemmas \ref{lem:monoid} and \ref{lem:if} hold for \PwTmca{1}.  For further
discussion of items \eqref{lem:ifelse:if:if1} and \eqref{lem:ifelse:if:if2},
see \textsection\ref{sec:delay}.


% To include \RMW{}s, also add
% \begin{enumerate}
% \item[(\textsc{p}10)]\;
%   ${\rrmw}=\PBR{{\rrmw_1}\cup{\rrmw_2}}$,
% \end{enumerate}


%The semantic rules are given in \reffig{fig:mca1}.

% From the data model, we require an additional binary relation 
% ${\rdelays}\subseteq(\Act\times\Act)$.  For the actions in this paper, we
% define this as follows.

% \begin{scope}
% \end{scope}

