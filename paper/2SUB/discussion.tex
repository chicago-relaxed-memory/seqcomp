% \begin{figure*}[t]
%   \showRAtrue
%   \begin{center}
%     \begin{minipage}{.91\textwidth}
%       \renewcommand{\cEvs}{D}
\renewcommand{\dEvs}{D}
\noindent
If $\aPS \in \sSTORE[\amode]{\aLoc}{\aExp}$ then
$(\exists\aVal:\aEvs\fun\Val)$
$(\exists\cForm:\aEvs\fun\Formulae)$
\begin{enumerate}
\item[{\labeltext[S1]{S1)}{S1no-q-or-addr}}] 
  if $\cForm_\bEv\land\cForm_\aEv$ is satisfiable then $\bEv=\aEv$,
\item[{\labeltext[S2]{S2)}{S2no-q-or-addr}}] 
  $\labelingAct(\aEv) = \DW[\amode]{\aLoc}{\aVal_\aEv}$,
\item[{\labeltext[S3]{S3)}{S3no-q-or-addr}}] 
  $\labelingForm(\aEv)$ implies
  \begin{math}
    \cForm_\aEv
    \land \QS{}{\amode}
    \land \aExp{=}\aVal_\aEv
  \end{math},
  
  
\item[{\labeltext[S4]{S4)}{S4no-q-or-addr}}] 
  \begin{math}
    (\forall\aEv\in\aEvs\cap\bEvs)
  \end{math}
  $\aTr{\bEvs}{\bForm}$ implies 
  \begin{math}
    \cForm_\aEv
    \limplies {
      \bForm
      [\aExp/\aLoc]
      \DS{\aLoc}{\amode}
      [(\Q{}\land\aExp{=}\aVal_\aEv)/\Q{}]
    }
  \end{math},
\item[{\labeltext[S5]{S5)}{S5no-q-or-addr}}] 
  \begin{math}    
    (\forall\aEv\in\aEvs\setminus\cEvs)
  \end{math}
  $\aTr{\cEvs}{\bForm}$ implies
  \begin{math}
    \cForm_\aEv
    \limplies {
      \bForm
      [\aExp/\aLoc]
      \DS{\aLoc}{\amode}
      [\FALSE/\Q{}]
    }.
  \end{math}
% \item[{\labeltext[S6]{S6)}{S6no-q-or-addr}}] 
%   $\aTr{\dEvs}{\bForm}$ implies
%   \begin{math}
%     (\!\not\exists\aEv\in\aEvs \suchthat \cForm_\aEv)
%     \limplies {
%       \bForm
%       [\aExp/\aLoc]
%       \DS{\aLoc}{\amode}
%       [\FALSE/\Q{}]
%     }.
%   \end{math}
\end{enumerate}

\noindent
If $\aPS \in \sLOAD[\amode]{\aReg}{\aLoc}$ then
$(\exists\aVal:\aEvs\fun\Val)$
$(\exists\cForm:\aEvs\fun\Formulae)$
$(\exists\bmode\in\{\amode,\mRLX\})$

\begin{enumerate}
\item[{\labeltext[L1]{L1)}{L1no-q-or-addr}}] 
  if $\cForm_\bEv\land\cForm_\aEv$ is satisfiable then $\bEv=\aEv$,
\item[{\labeltext[L2]{L2)}{L2no-q-or-addr}}] 
  $\labelingAct(\aEv) = \DR[\bmode]{\aLoc}{\aVal_\aEv}$
\item[{\labeltext[L3]{L3)}{L3no-q-or-addr}}] 
  $\labelingForm(\aEv)$ implies
  \begin{math}
    \cForm_\aEv
    \land \QL{}{\amode}
  \end{math},
    
\item[{\labeltext[L4]{L4)}{L4no-q-or-addr}}] 
  \begin{math}
    (\forall\aEv\in\aEvs\cap\bEvs)
  \end{math}
  $\aTr{\bEvs}{\bForm}$ implies
  \begin{math}
    \cForm_\aEv
    \limplies \aVal_\aEv{=}\uReg{\aEv}
    \limplies \bForm[\uReg{\aEv}/\aReg]
  \end{math},
  
\item[{\labeltext[L5]{L5)}{L5no-q-or-addr}}] 
  \begin{math}
    (\forall\aEv\in\aEvs\setminus\cEvs)
  \end{math}
  $\aTr{\cEvs}{\bForm}$ implies
  \begin{math}
    \cForm_\aEv 
    \limplies
    \DLX{\aLoc}{\amode}{\bmode}
    \land
    \PBRbig{
      \ABRbig{
        \aVal_\aEv{=}\uReg{\aEv}
        \lor
        \PBR{
          \RW\land
          \aLoc{=}\uReg{\aEv}
        }
      }
      \limplies
      \bForm
      [\uReg{\aEv}/\aReg]
      [\FALSE/\Q{}]
    }    
  \end{math},
\item[{\labeltext[L6]{L6)}{L6no-q-or-addr}}] 
  \begin{math}
    (\forall\bReg)
  \end{math}
  $\aTr{\dEvs}{\bForm}$  implies 
  \begin{math}
    (\!\not\exists\aEv\in\aEvs \suchthat \cForm_\aEv)
    \limplies \PBR{        
      \DLX{\aLoc}{\amode}{\bmode} \land
      \bForm
      [\bReg/\aReg]
      [\FALSE/\Q{}]
    }.
  \end{math}  
\end{enumerate}  





















































%     \end{minipage}
%   \end{center}
%   \caption{Simplified Quiescence Semantics w/o Address Calculation
%     (See %\refdef{def:QSx} for $\QS{}{\amode}$, $\QL{}{\amode}$, and
%     \refdef{def:dlx} for $\DLX{\aLoc}{\amode}{\bmode}$, $\DS{\aLoc}{\amode}$)
%   } 
%   \label{fig:no-q-or-addr}
% \end{figure*}    
% \begin{figure*}
%   \begin{center}
%     \begin{minipage}{.91\textwidth}
%       \noindent
If $\aPS \in \sSTORE[\amode]{\cExp}{\aExp}$ then
$(\exists\cVal:\aEvs\fun\Val)$
$(\exists\aVal:\aEvs\fun\Val)$
$(\exists\cForm:\aEvs\fun\Formulae)$
\begin{enumerate}
\item[\ref{S1})] if $\cForm_\bEv\land\cForm_\aEv$ is satisfiable then $\bEv=\aEv$,
\item[\ref{S2})] $\labelingAct(\aEv) = \DWREF{\cVal_\aEv}{\aVal_\aEv}$,
\item[\ref{S3})] 
  $\labelingForm(\aEv)$ implies
  \begin{math}
    \cForm_\aEv
    \land \QS{\REF{\cVal_\aEv}}{\amode}
    % \land \RW
    \land \cExp{=}\cVal_\aEv
    \land \aExp{=}\aVal_\aEv
  \end{math},
  % where
  % $\QS{}{\mRLX}=\QxREF{\cVal_\aEv}$ and otherwise $\QS{}{\amode}=\Q{\amode}$, % for $\amode\neq\mRLX$,
\item[\ref{S4})]
  \begin{math}
    (\forall\dVal)
    (\forall\aEv\in\aEvs\cap\bEvs)
  \end{math}
  $\aTr{\bEvs}{\bForm}$ \;implies \,
  \begin{math}
    \cForm_\aEv
    \limplies (\cExp{=}\dVal)
    \limplies \PBR{
      %(\QwREF{\dVal} \limplies \aExp{=}\aVal_\aEv) \land
      \bForm
      [\aExp/\REF{\dVal}]
      \DS{\REF{\dVal}}{\amode}
      [(\QwREF{\dVal}\land\aExp{=}\aVal)/\QwREF{\dVal}]
    }
  \end{math},
\item[\ref{S5})] %if 
  % \begin{math}
  %   (\forall\aEv\in\bEvs)(\cForm \textimplies
  %   \lnot\cForm_\aEv)
  % \end{math}
  % then
  \begin{math}
    (\forall\dVal)
  \end{math}
  $\aTr{\cEvs}{\bForm}$ implies
  \begin{math}
    (\!\not\exists\aEv\in\aEvs\cap\cEvs \suchthat \cForm_\aEv)
    \limplies (\cExp{=}\dVal)
    \limplies \PBR{
      % \lnot\QwREF{\dVal} \land
      \bForm
      [\aExp/\REF{\dVal}]
      \DS{\REF{\dVal}}{\amode}
      [\FALSE/\QS{\REF{\dVal}}{\amode}]
    }.
  \end{math}
  % \\ where 
  % $\DS{}{\mRLX}{}=[\TRUE/\DxREF{\dVal}]$ and otherwise
  % $\DS{}{\amode}{}=[\FALSE/\D]$. % for $\amode\neq\mRLX$.
\end{enumerate}
% \item if $\amode=\mRLX$ then
%   $\labelingForm(\aEv)$ implies
%   \begin{math}
%     \cForm_\aEv
%     \land \cExp{=}\cVal_\aEv
%     \land \aExp{=}\aVal_\aEv
%     \land \RW
%     \land \QxREF{\cVal_\aEv},
%   \end{math}
% \item if $\amode\neq\mRLX$ then
%   $\labelingForm(\aEv)$ implies
%   \begin{math}
%     \cForm_\aEv
%     \land \cExp{=}\cVal_\aEv
%     \land \aExp{=}\aVal_\aEv
%     \land \RW
%     \land \Q{},
%   \end{math}
% \item if
%   $\aEv\in\bEvs$
%   and
%   $\amode=\mRLX$ then
%   \begin{math}
%     (\forall\dVal)
%   \end{math}
%   $\aTr{\bEvs}{\bForm}$ implies 
%   \begin{math}
%     \cForm_\aEv
%     \limplies (\cExp{=}\dVal)
%     \limplies \PBRbig{
%     (\QwREF{\dVal} \limplies \aExp{=}\aVal_\aEv)
%     \land \bForm[\aExp/\REF{\dVal}][\TRUE/\DxREF{\dVal}]
%   }
%   \end{math}
% \item if
%   $\aEv\in\bEvs$
%   and
%   $\amode\neq\mRLX$ then
%   \begin{math}
%     (\forall\dVal)
%   \end{math}
%   $\aTr{\bEvs}{\bForm}$ implies 
%   \begin{math}
%     \cForm_\aEv
%     \limplies (\cExp{=}\dVal)
%     \limplies \PBRbig{
%     (\QwREF{\dVal} \limplies \aExp{=}\aVal_\aEv)
%     \land \bForm[\aExp/\REF{\dVal}][\FALSE/\D]
%   }
%   \end{math}
% \item if 
%   \begin{math}
%     (\forall\aEv\in\bEvs)(\cForm \textimplies
%     \lnot\cForm_\aEv)
%   \end{math}
%   and $\amode=\mRLX$ 
%   then
%   \begin{math}
%     (\forall\dVal)
%   \end{math}
%   $\aTr{\bEvs}{\bForm}$ implies 
%   \begin{math}
%     \cForm
%     \limplies (\cExp{=}\dVal)
%     \limplies \PBRbig{
%     \lnot\QwREF{\dVal}
%     \land \bForm[\aExp/\REF{\dVal}][\TRUE/\DxREF{\dVal}]
%   }
%   \end{math}
% \item if 
%   \begin{math}
%     (\forall\aEv\in\bEvs)
%     (\cForm \textimplies \lnot\cForm_\aEv)
%   \end{math}
%   and $\amode\neq\mRLX$ 
%   then
%   \begin{math}
%     (\forall\dVal)
%   \end{math}
%   $\aTr{\bEvs}{\bForm}$ implies 
%   \begin{math}
%     \cForm
%     \limplies (\cExp{=}\dVal)
%     \limplies \PBRbig{
%     \lnot\QwREF{\dVal}
%     \land \bForm[\aExp/\REF{\dVal}][\FALSE/\D]
%   }
%   \end{math}

\noindent
If $\aPS \in \sLOAD[\amode]{\aReg}{\cExp}$ then
$(\exists\cVal:\aEvs\fun\Val)$
$(\exists\aVal:\aEvs\fun\Val)$
$(\exists\cForm:\aEvs\fun\Formulae)$
% $(\forall\uReg{\aEv}\in\uRegs{\aEvs})$
\begin{enumerate}
\item[\ref{L1})] if $\cForm_\bEv\land\cForm_\aEv$ is satisfiable then $\bEv=\aEv$,
\item[\ref{L2})] $\labelingAct(\aEv) = \DRREF{\cVal_\aEv}{\aVal_\aEv}$,
\item[\ref{L3})] $\labelingForm(\aEv)$ implies
  \begin{math}
    \cForm_\aEv
    \land \QL{\REF{\cVal_\aEv}}{\amode}
    % \land \RO
    \land \cExp{=}\cVal_\aEv
  \end{math},
  % where    
  % $\QL{}{\mSC}=\Q{\mSC}$ and otherwise $\QL{}{\amode}=\QwREF{\cVal_\aEv}$, % for $\amode\neq\mRLX$,
\item[\ref{L4})]
  \begin{math}
    (\forall\dVal)
    (\forall\aEv\in\aEvs\cap\bEvs)
  \end{math}
  $\aTr{\bEvs}{\bForm}$ implies
  \begin{math}
    \cForm_\aEv
    \limplies (\cExp{=}\dVal)
    \limplies (\aVal_\aEv{=}\uReg{\aEv})
    \limplies \bForm[\uReg{\aEv}/\aReg]%[\uReg{\aEv}/\REF{\dVal}]
  \end{math},
  \makebox[5.75cm]{}
\item[\ref{L5})] 
  \begin{math}
    (\forall\dVal)
    (\forall\aEv\in\aEvs\setminus\cEvs)
  \end{math}
  $\aTr{\cEvs}{\bForm}$ implies
  \begin{math}
    \cForm_\aEv
    \limplies (\cExp{=}\dVal)
    \limplies \PBR{        
      %\lnot\QxREF{\dVal}\land
      \DL{\REF{\dVal}}{\amode} \land
      (\RW
      \limplies (\aVal_\aEv{=}\uReg{\aEv}\lor\REF{\dVal}{=}\uReg{\aEv}) 
      \limplies
      \bForm
      [\uReg{\aEv}/\aReg]%[\uReg{\aEv}/\REF{\dVal}]
      [\FALSE/\QL{\REF{\dVal}}{\amode}]
      )
    }      
  \end{math},
\item[\ref{L6})] % if 
  % \begin{math}
  %   (\forall\aEv\in\bEvs)(\cForm \textimplies
  %   \lnot\cForm_\aEv)
  % \end{math}
  % then
  \begin{math}
    (\forall\dVal)
    (\forall\bReg)
  \end{math}
  $\aTr{\dEvs}{\bForm}$  implies 
  \begin{math}
    (\!\not\exists\aEv\in\aEvs \suchthat \cForm_\aEv)
    \limplies (\cExp{=}\dVal)
    \limplies \PBR{        
      %\lnot\QxREF{\dVal} \land
      \DL{\REF{\dVal}}{\amode} \land
      \bForm
      [\bReg/\aReg]%[\bReg/\REF{\dVal}]
      [\FALSE/\QL{\REF{\dVal}}{\amode}]
    }.
  \end{math}
  % \\ where $\DL{}{\mRLX}=\TRUE$ and otherwise $\DL{}{\amode}=\DxREF{\dVal}$.
  % Recall that $\uRegs{\bEvs}=\{\uReg{\aEv}\mid\aEv\in\bEvs\}$.
\end{enumerate}  
% \item if $\amode=\mRLX$ and $\bEv\notin\bEvs$ then
%   \begin{math}
%     (\forall\dVal)
%   \end{math}
%   $\aTr{\bEvs}{\bForm}$ implies
%   \begin{math}
%     \cForm_\bEv
%     \limplies (\cExp{=}\dVal)
%     \limplies \PBRbig{
%     (
%     \RW
%     \limplies (\aVal{=}\uReg{\bEv}\lor\aLoc{=}\uReg{\bEv}) 
%     \limplies \bForm[\uReg{\bEv}/\aReg][\uReg{\bEv}/\REF{\dVal}]
%     )
%     \land \lnot\QxREF{\dVal}
%   }
%     \phantom{\land\; \Dx{\dVal}}
%   \end{math}
% \item if $\amode\neq\mRLX$ and $\bEv\notin\bEvs$ then
%   \begin{math}
%     (\forall\dVal)
%   \end{math}
%   $\aTr{\bEvs}{\bForm}$ implies
%   \begin{math}
%     \cForm_\bEv
%     \limplies (\cExp{=}\dVal)
%     \limplies \PBRbig{
%     (
%     \RW
%     \limplies (\aVal{=}\uReg{\bEv}\lor\aLoc{=}\uReg{\bEv}) 
%     \limplies \bForm[\uReg{\bEv}/\aReg][\uReg{\bEv}/\REF{\dVal}]
%     )
%     \land \lnot\QxREF{\dVal}
%     \land \Dx{\dVal}
%   }
%   \end{math}

\noindent
If $\aPS \in \sTHREAD{\aPSS}$ then
$(\exists\aPS_1\in\aPSS)$
\begin{enumerate}
\item[\ref{T1})]
  $\aEvs=\aEvs_1$,
\item[\ref{T2})]
  $\labelingAct(\aEv) = \labelingAct_1(\aEv)$,
\item[\ref{T3})]
  $\labelingForm(\aEv)$ implies
  $\labelingForm_1(\aEv) [\TRUE/\Qr{*}][\TRUE/\Qw{*}][\TRUE/\Qsc][\TRUE/\RW]$ if $\labelingAct_1(\aEv)$ is a write,
  \\
  $\labelingForm(\aEv)$ implies
  $\labelingForm_1(\aEv) [\TRUE/\Qr{*}][\TRUE/\Qw{*}][\TRUE/\Qsc][\FALSE/\RW]$ otherwise.
\end{enumerate}  

%       % \noindent
If $\aPS \in \sSTORE[\amode]{\cExp}{\aExp}$ then
$(\exists\cVal:\aEvs\fun\Val)$
$(\exists\aVal:\aEvs\fun\Val)$
$(\exists\cForm:\aEvs\fun\Formulae)$
\begin{enumerate}
\item if $\cForm_\bEv\land\cForm_\aEv$ is satisfiable then $\bEv=\aEv$,
\item $\labelingAct(\aEv) = \DWREFP{\cVal_\aEv}{\aVal_\aEv}$,
\item 
  $\labelingForm(\aEv)$ implies
  \begin{math}
    \cForm_\aEv
    \land \cExp{=}\cVal_\aEv
    \land \aExp{=}\aVal_\aEv
    \land \RW
    \land \Qmode{\amode}
  \end{math},
  where
  $\Qmode{\mRLX}=\QxREF{\cVal_\aEv}$ and otherwise $\Qmode{\amode}=\Q{\amode}$, % for $\amode\neq\mRLX$,
\item
  \begin{math}
    (\forall\dVal)
  \end{math}
  if
  $\bEv\in\bEvs$
  then
  $\aTr{\bEvs}{\aForm}$ implies 
  \begin{math}
    \cForm_\bEv
    \limplies (\cExp{=}\dVal)
    \limplies \PBRbig{
      (\QwREF{\dVal} \limplies \aExp{=}\aVal_\bEv)
      \land \aForm [\aExp/\REF{\dVal}]\Dmode{\amode}
    }
  \end{math},
\item %if 
  % \begin{math}
  %   (\forall\bEv\in\bEvs)(\cForm \textimplies
  %   \lnot\cForm_\bEv)
  % \end{math}
  % then
  \begin{math}
    (\forall\dVal)
  \end{math}
  $\aTr{\bEvs}{\aForm}$ implies 
  \begin{math}
    (\not\exists\bEv\in\bEvs.\; \cForm_\bEv)
    \limplies (\cExp{=}\dVal)
    \limplies \PBR{
      \lnot\QwREF{\dVal}
      \land \aForm [\aExp/\REF{\dVal}]\Dmode{\amode}
    }
  \end{math},
  \\ where 
  $\Dmode{\mRLX}=[\TRUE/\DxREF{\dVal}]$ and otherwise
  $\Dmode{\amode}=[\FALSE/\D]$. % for $\amode\neq\mRLX$.
\end{enumerate}
% \item if $\amode=\mRLX$ then
%   $\labelingForm(\aEv)$ implies
%   \begin{math}
%     \cForm_\aEv
%     \land \cExp{=}\cVal_\aEv
%     \land \aExp{=}\aVal_\aEv
%     \land \RW
%     \land \QxREF{\cVal_\aEv},
%   \end{math}
% \item if $\amode\neq\mRLX$ then
%   $\labelingForm(\aEv)$ implies
%   \begin{math}
%     \cForm_\aEv
%     \land \cExp{=}\cVal_\aEv
%     \land \aExp{=}\aVal_\aEv
%     \land \RW
%     \land \Q{},
%   \end{math}
% \item if
%   $\bEv\in\bEvs$
%   and
%   $\amode=\mRLX$ then
%   \begin{math}
%     (\forall\dVal)
%   \end{math}
%   $\aTr{\bEvs}{\aForm}$ implies 
%   \begin{math}
%     \cForm_\bEv
%     \limplies (\cExp{=}\dVal)
%     \limplies \PBRbig{
%     (\QwREF{\dVal} \limplies \aExp{=}\aVal_\bEv)
%     \land \aForm[\aExp/\REF{\dVal}][\TRUE/\DxREF{\dVal}]
%   }
%   \end{math}
% \item if
%   $\bEv\in\bEvs$
%   and
%   $\amode\neq\mRLX$ then
%   \begin{math}
%     (\forall\dVal)
%   \end{math}
%   $\aTr{\bEvs}{\aForm}$ implies 
%   \begin{math}
%     \cForm_\bEv
%     \limplies (\cExp{=}\dVal)
%     \limplies \PBRbig{
%     (\QwREF{\dVal} \limplies \aExp{=}\aVal_\bEv)
%     \land \aForm[\aExp/\REF{\dVal}][\FALSE/\D]
%   }
%   \end{math}
% \item if 
%   \begin{math}
%     (\forall\bEv\in\bEvs)(\cForm \textimplies
%     \lnot\cForm_\bEv)
%   \end{math}
%   and $\amode=\mRLX$ 
%   then
%   \begin{math}
%     (\forall\dVal)
%   \end{math}
%   $\aTr{\bEvs}{\aForm}$ implies 
%   \begin{math}
%     \cForm
%     \limplies (\cExp{=}\dVal)
%     \limplies \PBRbig{
%     \lnot\QwREF{\dVal}
%     \land \aForm[\aExp/\REF{\dVal}][\TRUE/\DxREF{\dVal}]
%   }
%   \end{math}
% \item if 
%   \begin{math}
%     (\forall\bEv\in\bEvs)
%     (\cForm \textimplies \lnot\cForm_\bEv)
%   \end{math}
%   and $\amode\neq\mRLX$ 
%   then
%   \begin{math}
%     (\forall\dVal)
%   \end{math}
%   $\aTr{\bEvs}{\aForm}$ implies 
%   \begin{math}
%     \cForm
%     \limplies (\cExp{=}\dVal)
%     \limplies \PBRbig{
%     \lnot\QwREF{\dVal}
%     \land \aForm[\aExp/\REF{\dVal}][\FALSE/\D]
%   }
%   \end{math}

\noindent
If $\aPS \in \sLOAD[\amode]{\aReg}{\cExp}$ then
$(\exists\cVal:\aEvs\fun\Val)$
$(\exists\aVal:\aEvs\fun\Val)$
$(\exists\cForm:\aEvs\fun\Formulae)$
% $(\forall\uReg{\aEv}\in\uRegs{\aEvs})$
\begin{enumerate}
\item if $\cForm_\bEv\land\cForm_\aEv$ is satisfiable then $\bEv=\aEv$,
\item $\labelingAct(\aEv) = \DRREFP{\cVal_\aEv}{\aVal_\aEv}$,
\item $\labelingForm(\aEv)$ implies
  \begin{math}
    \cForm_\aEv
    \land \cExp{=}\cVal_\aEv
    \land \RO
    \land \Qmode{\amode}
  \end{math},
  where    
  $\Qmode{\mSC}=\Q{\mSC}$ and otherwise $\Qmode{\amode}=\QwREF{\cVal_\aEv}$, % for $\amode\neq\mRLX$,
\item
  \begin{math}
    (\forall\dVal)
  \end{math}
  if $\bEv\in\bEvs$ then
  $\aTr{\bEvs}{\aForm}$ implies
  \begin{math}
    \cForm_\bEv
    \limplies (\cExp{=}\dVal)
    \limplies (\aVal{=}\uReg{\bEv})
    \limplies \aForm[\uReg{\bEv}/\aReg][\uReg{\bEv}/\REF{\dVal}]
  \end{math},
  \makebox[4.4cm]{}
\item 
  \begin{math}
    (\forall\dVal)
  \end{math}
  if $\bEv\notin\bEvs$ then
  $\aTr{\bEvs}{\aForm}$ implies
  \begin{math}
    \cForm_\bEv
    \limplies (\cExp{=}\dVal)
    \limplies \PBRbig{        
      \Dmode{\amode}
      \land \lnot\QxREF{\dVal}
      \land
      (\RW
      \limplies (\aVal{=}\uReg{\bEv}\lor\aLoc{=}\uReg{\bEv}) 
      \limplies \aForm[\uReg{\bEv}/\aReg][\uReg{\bEv}/\REF{\dVal}]
      )
    }      
  \end{math},
\item % if 
  % \begin{math}
  %   (\forall\bEv\in\bEvs)(\cForm \textimplies
  %   \lnot\cForm_\bEv)
  % \end{math}
  % then
  \begin{math}
    (\forall\dVal)
    (\forall\bReg)
  \end{math}
  $\aTr{\bEvs}{\aForm}$ implies 
  \begin{math}
    (\not\exists\bEv\in\bEvs.\; \cForm_\bEv)
    \limplies (\cExp{=}\dVal)
    \limplies \PBR{        
      \Dmode{\amode}
      \land \lnot\QxREF{\dVal}
      \land
      \limplies \aForm[\bReg/\aReg][\bReg/\REF{\dVal}]
    }      
  \end{math},
  \\ where $\Dmode{\mRLX}=\TRUE$ and otherwise $\Dmode{\amode}=\Dx{\dVal}$.
  Recall that $\uRegs{\bEvs}=\{\uReg{\bEv}\mid\bEv\in\bEvs\}$.
\end{enumerate}  
% \item if $\amode=\mRLX$ and $\bEv\notin\bEvs$ then
%   \begin{math}
%     (\forall\dVal)
%   \end{math}
%   $\aTr{\bEvs}{\aForm}$ implies
%   \begin{math}
%     \cForm_\bEv
%     \limplies (\cExp{=}\dVal)
%     \limplies \PBRbig{
%     (
%     \RW
%     \limplies (\aVal{=}\uReg{\bEv}\lor\aLoc{=}\uReg{\bEv}) 
%     \limplies \aForm[\uReg{\bEv}/\aReg][\uReg{\bEv}/\REF{\dVal}]
%     )
%     \land \lnot\QxREF{\dVal}
%   }
%     \phantom{\land\; \Dx{\dVal}}
%   \end{math}
% \item if $\amode\neq\mRLX$ and $\bEv\notin\bEvs$ then
%   \begin{math}
%     (\forall\dVal)
%   \end{math}
%   $\aTr{\bEvs}{\aForm}$ implies
%   \begin{math}
%     \cForm_\bEv
%     \limplies (\cExp{=}\dVal)
%     \limplies \PBRbig{
%     (
%     \RW
%     \limplies (\aVal{=}\uReg{\bEv}\lor\aLoc{=}\uReg{\bEv}) 
%     \limplies \aForm[\uReg{\bEv}/\aReg][\uReg{\bEv}/\REF{\dVal}]
%     )
%     \land \lnot\QxREF{\dVal}
%     \land \Dx{\dVal}
%   }
%   \end{math}

%       % \noindent
If $\aPS \in \sSTORE[\amode]{\cExp}{\aExp}$ then
$(\exists\cVal:\aEvs\fun\Val)$
$(\exists\aVal:\aEvs\fun\Val)$
$(\exists\bForm:\aEvs\fun\Formulae)$
\begin{enumerate}
\item if $\bForm_\bEv\land\bForm_\aEv$ is satisfiable then $\bEv=\aEv$,
\item $\labelingAct(\aEv) = \DWREFP{\cVal_\aEv}{\aVal_\aEv}$,
\item 
  $\labelingForm(\aEv)$ implies
  \begin{math}
    \bForm_\aEv
    \land \cExp{=}\cVal_\aEv
    \land \aExp{=}\aVal_\aEv
    \land \RW
    \land \QS{}{\amode}
  \end{math},
\item
  \begin{math}
    (\forall\dVal)
  \end{math}
  if
  $\bEv\in\bEvs$
  then
  $\aTr[\bEvs](\aForm)$ implies 
  \begin{math}
    \bForm_\bEv
    \limplies (\cExp{=}\dVal)
    \limplies \PBRbig{
      \aExp{=}\aVal_\bEv
      \land \DS{\REF{\dVal}}{\amode}{\aForm[\aExp/\REF{\dVal}]}
    }
  \end{math},
\item 
  \begin{math}
    (\forall\dVal)
  \end{math}
  $\aTr[\bEvs](\aForm)$ implies 
  \begin{math}
    (\not\exists\bEv\in\bEvs.\; \bForm_\bEv)
    \limplies (\cExp{=}\dVal)
    \limplies \PBR{
      \lnot\Q{\mRA}
      \land \DS{\REF{\dVal}}{\amode}{\aForm[\aExp/\REF{\dVal}]}
    }.
  \end{math}
\end{enumerate}

\noindent
If $\aPS \in \sLOAD[\amode]{\aReg}{\cExp}$ then
$(\exists\cVal:\aEvs\fun\Val)$
$(\exists\aVal:\aEvs\fun\Val)$
$(\exists\bForm:\aEvs\fun\Formulae)$
\begin{enumerate}
\item if $\bForm_\bEv\land\bForm_\aEv$ is satisfiable then $\bEv=\aEv$,
\item $\labelingAct(\aEv) = \DRREFP{\cVal_\aEv}{\aVal_\aEv}$,
\item $\labelingForm(\aEv)$ implies
  \begin{math}
    \bForm_\aEv
    \land \cExp{=}\cVal_\aEv
    \land \RO
    \land \QL{}{\amode}
  \end{math},
\item
  \begin{math}
    (\forall\dVal)
  \end{math}
  if $\bEv\in\bEvs$ then
  $\aTr[\bEvs](\aForm)$ implies
  \begin{math}
    \bForm_\bEv
    \limplies (\cExp{=}\dVal)
    \limplies (\aVal{=}\uReg{\bEv})
    \limplies \aForm[\uReg{\bEv}/\aReg][\uReg{\bEv}/\REF{\dVal}]
  \end{math},
  \makebox[4.8cm]{}
\item 
  \begin{math}
    (\forall\dVal)
  \end{math}
  if $\bEv\notin\bEvs$ then
  $\aTr[\bEvs](\aForm)$ implies
  \begin{math}
    \bForm_\bEv
    \limplies (\cExp{=}\dVal)
    \limplies \PBRbig{        
      \DL{\REF{\dVal}}{\amode}
      \land \lnot\Q{\mRA}
      \land
      (\RW
      \limplies (\aVal{=}\uReg{\bEv}\lor\aLoc{=}\uReg{\bEv}) 
      \limplies \aForm[\uReg{\bEv}/\aReg][\uReg{\bEv}/\REF{\dVal}]
      )
    }      
  \end{math},
\item 
  \begin{math}
    (\forall\dVal)
    (\forall\bReg)
  \end{math}
  $\aTr[\bEvs](\aForm)$ implies 
  \begin{math}
    (\not\exists\bEv\in\bEvs.\; \bForm_\bEv)
    \limplies (\cExp{=}\dVal)
    \limplies \PBR{        
      \DL{\REF{\dVal}}{\amode}
      \land \lnot\Q{\mRA}
      \land
      \limplies \aForm[\bReg/\aReg][\bReg/\REF{\dVal}]
    }.
  \end{math}
\end{enumerate}  

%     \end{minipage}
%   \end{center}
%   \caption{Full Semantics with Address Calculation
%     (See \refdef{def:QS} for $\QS{\aLoc}{\amode}$, $\QL{\aLoc}{\amode}$
%     and \refdef{def:DS} for $\DL{\aLoc}{\amode}$, $\DS{\aLoc}{\amode}$)
%   }
%   \label{fig:full}
% \end{figure*}    

%\section{Discussion}
\section{Downset Closure}
\label{sec:downset}

% We would like the semantics to be closed with respect to \emph{augments} and
% \emph{downsets}.

% Augments include more order and stronger formulae; in examples, we typically
% consider pomsets that are augment-minimal.  One intuitive reading of augment
% closure is that adding order can only cause preconditions to weaken.
% \begin{definition}
%   \label{def:augment}
%   $\aPS_2$ is an \emph{augment} of $\aPS_1$ if
%   \begin{enumerate}
%   \item $\aEvs_2=\aEvs_1$,
%   \item $\labelingAct_2(\aEv)=\labelingAct_1(\aEv)$,
%   \item $\labelingForm_2(\aEv) \rimplies \labelingForm_1(\aEv)$,
%   \item $\aTr[2]{\bEvs}{\aEv} \rimplies \aTr[1]{\bEvs}{\aEv}$,
%   \item if $\bEv\le_2\aEv$ then $\bEv\le_1\aEv$.
%   \end{enumerate}
% \end{definition}

% \begin{proposition}
%   %   Suppose $\aPS_1\in\sem{\aCmd}$.
%   If $\aPS_1\in\sem{\aCmd}$ and $\aPS_2$  augments $\aPS_1$ then $\aPS_2\in\sem{\aCmd}$.
%   % \item If $\aPS_2$ is a downset of $\aPS_1$ then $\aPS_2\in\sem{\aCmd}$.
%   % \end{enumerate}
% \end{proposition}

We would like the semantics to be closed with respect to \emph{downsets}.
Downsets include a subset of initial events, similar to \emph{prefixes} for
strings.
\begin{definition}
  \label{def:downset}
  $\aPS_2$ is an \emph{downset} of $\aPS_1$ if
  \begin{multicols}{2}
    \begin{enumerate}
    \item $\aEvs_2\subseteq\aEvs_1$,
    \item $(\forall \aEv\in\aEvs_2)$ $\labelingAct_2(\aEv)=\labelingAct_1(\aEv)$,
    \item $(\forall \aEv\in\aEvs_2)$ $\labelingForm_2(\aEv)\riff\labelingForm_1(\aEv)$,
    \item $(\forall \aEv\in\aEvs_2)$ $\aTr[2]{\bEvs}{\aEv}\riff\aTr[1]{\bEvs}{\aEv}$,
    \item $\aTerm[2] \rimplies \aTerm[1]$,
    \item $(\forall \bEv\in\aEvs_2)$ $(\forall \aEv\in\aEvs_2)$ $\bEv\rrfx_2\aEv$ iff $\bEv\rrfx_1\aEv$,, 
      \stepcounter{enumi}
    \item[] 
      \begin{enumerate}[leftmargin=0pt]
      \item $(\forall \bEv\in\aEvs_2)$ $(\forall \aEv\in\aEvs_2)$ $\bEv\le_2\aEv$ iff $\bEv\le_1\aEv$,
      \item $(\forall \bEv\in\aEvs_1)$ $(\forall \aEv\in\aEvs_2)$ if
        $\bEv\le_1\aEv$ then $\bEv\in\aEvs_2$.
      \end{enumerate}
    \end{enumerate}
  \end{multicols}
\end{definition}

Downset closure fails due to for two reasons.  The key property is that the
empty set transformer should behave the same as the independent transformer.

First, downset closure fails for \refdef{def:semrr}, because it does not
enforce read-read dependencies.
  % For \xRRD{}, \refdef{def:pomsets-rr} states:
  % \begin{enumerate}
  % \item[\ref{L4})]
  %   $\aTr{\bEvs}{\bForm} \rimplies \aVal{=}\aReg\limplies\bForm$, 
  % \item[\ref{L5})]
  %   $\aTr{\cEvs}{\bForm} \rimplies (\aVal{=}\aReg\lor\RW)\limplies\bForm$,
  % \item[\ref{L6})] 
  %   $\aTr{\dEvs}{\bForm} \rimplies \bForm$, when $\aEvs=\emptyset$.
  % \end{enumerate}
  % This semantics is not downset closed due to the lack of read-read dependencies.
  % In both cases, for subsequent writes, \ref{L5} is the same as \ref{L6}.  For
  % subsequent reads, \ref{L5} is the same as \ref{L4}.
Consider
\begin{gather*}
  \begin{gathered}[t]
    \PR{x}{r}\SEMI\IF{\BANG r}\THEN\PR{y}{s}\FI
    \\
    \hbox{\begin{tikzinline}[node distance=.5em and 1.5em]
        \event{a}{\DR{x}{0}}{}
        \event{b}{\DR{y}{0}}{right=of a}
      \end{tikzinline}}
  \end{gathered}    
\end{gather*}
The semantics of this program includes the singleton pomset $\DRP{x}{0}$,
but not the singleton pomset $\DRP{y}{0}$.
To get $\DRP{x}{0}$, we combine:
\begin{align*}
  \begin{gathered}[t]
    \PR{x}{r}
    \\
    \hbox{\begin{tikzinline}[node distance=.5em and 1.5em]
        \event{a}{\DR{x}{0}}{}
      \end{tikzinline}}
  \end{gathered}    
  &&
  \begin{gathered}[t]
    \IF{\BANG r}\THEN\PR{y}{s}\FI
    \\
    \emptyset
  \end{gathered}    
\end{align*}
Attempting to get $\DRP{y}{0}$, we instead get:
\begin{align*}
  \begin{gathered}[t]
    \PR{x}{r}
    \\
    \emptyset
  \end{gathered}    
  &&
  \begin{gathered}[t]
    \IF{\BANG r}\THEN\PR{y}{s}\FI
    \\
    \hbox{\begin{tikzinline}[node distance=.5em and 1.5em]
        \event{b}{r\EQ0\mid\DR{y}{0}}{}
      \end{tikzinline}}
  \end{gathered}    
\end{align*}
Since $r$ appears only once in the program, this pomset cannot contribute
to a top-level pomset.


Second, the semantics is not downset closed because the independency reasoning of
\ref{read-tau-ind} is only applicable for pomsets where the ignored read is present!
Revisiting JMM causality test case 1 from the end of \textsection\ref{sec:ex:control}:
\begin{align*}
  \begin{gathered}[t]
    \PW{x}{0} 
    \\
    \hbox{\begin{tikzinline}[node distance=.5em and 1.5em]
        \event{a0}{\DW{x}{0}}{}
        \xform{xi}{\bForm[0/x]}{below=of a0}
      \end{tikzinline}}    
  \end{gathered}
  &&
  \begin{gathered}[t]
    \PR{x}{r} 
    \\
    \hbox{\begin{tikzinline}[node distance=.5em and 1.5em]
        \event{a1}{\DR{x}{1}}{}
        \xform{xi}{(1{=}r\lor x{=}r)\limplies\bForm}{below=of a1}
      \end{tikzinline}}    
  \end{gathered}
  &&
  \begin{gathered}[t]
    \IF{r{\geq}0}\THEN \PW{y}{1} \FI
    \SEMI
    \PW{z}{r}
    \\
    \hbox{\begin{tikzinline}[node distance=.5em and 1.5em]
        \event{a2}{r{\geq}0\mid\DW{y}{1}}{}      
        \event{a3}{r{=}1\mid\DW{z}{1}}{right=of a2}      
      \end{tikzinline}}    
  \end{gathered}
\end{align*}
% Composing:
\begin{align*}
  \begin{gathered}[t]
    \PW{x}{0} 
    \SEMI\PR{x}{r} 
    \SEMI\IF{r{\geq}0}\THEN \PW{y}{1} \FI
    \SEMI
    \PW{z}{r}
    \\
    \hbox{\begin{tikzinline}[node distance=.5em and 1.5em]
        \event{a0}{\DW{x}{0}}{}
        \event{a1}{\DR{x}{1}}{right=of a0}
        \event{a2}{(1{=}r\lor 0{=}r)\limplies r{\geq}0\mid\DW{y}{1}}{right=of a1}      
        \event{a3}{1{=}r\limplies r{=}1\mid\DW{z}{1}}{right=of a2}
        \po[out=-15,in=-165]{a1}{a3}
        \wki{a0}{a1}
      \end{tikzinline}}    
  \end{gathered}
\end{align*}
The precondition of $\DWP{y}{1}$ is a tautology.

Taking the empty set for the read, however,
the precondition of $\DWP{y}{1}$ is not a tautology:
\begin{align*}
  \begin{gathered}[t]
    \PW{x}{0} 
    \SEMI\PR{x}{r} 
    \SEMI\IF{r{\geq}0}\THEN \PW{y}{1} \FI
    \SEMI
    \PW{z}{r}
    \\
    \hbox{\begin{tikzinline}[node distance=.5em and 1.5em]
        \event{a0}{\DW{x}{0}}{}
        % \event{a1}{\DR{x}{1}}{right=of a0}
        \event{a2}{r{\geq}0\mid\DW{y}{1}}{right=6em of a0}      
        \event{a3}{r{=}1\mid\DW{z}{1}}{right=of a2}
        % \wk{a0}{a1}
      \end{tikzinline}}    
  \end{gathered}
\end{align*}
The second issue goes away if one allows general access elimination to merge
$\DWP{x}{0}$ and $\DRP{x}{0}$, as in \textsection\ref{sec:access:elimination}.
\begin{align*}
  \begin{gathered}[t]
    \PW{x}{0} 
    \SEMI\PR{x}{r} 
    \SEMI\IF{r{\geq}0}\THEN \PW{y}{1} \FI
    \SEMI
    \PW{z}{r}
    \\
    \hbox{\begin{tikzinline}[node distance=.5em and 1.5em]
        \event{a0}{\DW{x}{0}}{}
        %\event{a1}{\DR{x}{1}}{right=of a0}
        \event{a2}{(0{=}r\lor 0{=}r)\limplies r{\geq}0\mid\DW{y}{1}}{right=6em of a0}      
        \event{a3}{r{=}1\mid\DW{z}{1}}{right=of a2}
        %\po[out=-15,in=-165]{a1}{a3}
        %\wki{a0}{a1}
      \end{tikzinline}}    
  \end{gathered}
\end{align*}

\begin{comment}
  if in L6 we said [x/r], that says we know read the local version...  ignoring
  the value read...  Perhaps there is some intervening stuff that stops you
  from seeing the local state, such as release-acquire

  We could potentially get rid of [x/r] If you do two reads, your not allowed
  to be independent of the second based on the value that was read in the first
  read.

  x=0; r=x; if (r=1) { s=x; if (s=?) {y=1}}
  read 1 then 2.


  In order for the write to be independent of second read what does its
  precondition have to be.
  [r/x] then s==1
  no sub then s==0

  (s=? | Wy1)

  if (phi) z=1
  phi = s is even
  phi = s < 2

  With substitution you are saying you know that the ``local copy'' of x is the
  same as r.  Sitting in the local cache.  Read might have gone to main
  memory, but if it did it has updated the cache line so that the local copy is
  what I just read.

  If second read is a cache hit, then I know that I am seeing the same value.

  If we take substitution out then 
\end{comment}


\section{Comments on Case Analysis, etc}

Case analysis gives very weak results when combined with thread inlining.
See \cite[\textsection B.1]{DBLP:journals/pacmpl/ChakrabortyV19appendix}.
These happen by performing transformations that: 
(1) introduce conditionals,
(2) inline two threads on both sides of the introduced conditional,
(3) choose different orders for the two threads for the two sides of the conditional.

Case analysis gives very weak results when combined with read introduction.
See \cite{promising-ldrf}.
These happen by performing transformations that: 
(1) introduce reads,
(2) introduce conditionals,
(3) choose different values for the reads on the two sides of the conditional.


The fact that the semantics is not verifiable a posteriori is something it
shares with \weakestmo{}, where the justification relation must be built
inductively.

\weakestmo{} admits FADD, but \PS{} does not.
\PS{} CohCYC, but \weakestmo{} does not.


\section{Additional Examples}

\subsection{Arm}
The following execution is allowed by Arm.
\begin{gather*}
  {
    \PW{x}{1}
    \SEMI
    \PW[\mREL]{y}{1}
  }\PAR{
    \PR{y}{r}
    \SEMI
    \PW{y}{2}
    \SEMI
    \PR[\mACQ]{y}{s}
     \SEMI
    \PR{x}{t}
  }
  \\
  \hbox{\begin{tikzinline}[node distance=1.5em]
      \event{a}{\DW{x}{1}}{}
      \raevent{b}{\DW[\mREL]{y}{1}}{right=of a}
      \event{c}{\DR{y}{1}}{right=3em of b}
      \event{d}{\DW{y}{2}}{right=of c}
      \raevent{e}{\DR[\mACQ]{y}{2}}{right=of d}
      \event{f}{\DR{x}{0}}{right=of e}
      \lob{a}{b}
      \rfx{b}{c}
      %\sync[out=15,in=165]{c}{e}
      \lob{c}{d}
      \rfx{d}{e}
      \lob{e}{f}
      \fr[out=-165,in=-15]{f}[above,pos=.45]{a}
      %\close[out=-15,in=-165]{b}{e}
    \end{tikzinline}}
  \\
  \tag{$\rgcb$}
  \hbox{\begin{tikzinline}[node distance=1.5em]
      \event{a}{\DW{x}{1}}{}
      \raevent{b}{\DW[\mREL]{y}{1}}{right=of a}
      \event{c}{\DR{y}{1}}{right=3em of b}
      \event{d}{\DW{y}{2}}{right=of c}
      \raevent{e}{\DR[\mACQ]{y}{2}}{right=of d}
      \event{f}{\DR{x}{0}}{right=of e}
      \gcbz{a}{b}
      \gcbz{b}{c}
      \gcbz{c}{d}
      \gcbz{d}{e}
      %\gcbz{e}{f}
      \gcbz[out=-165,in=-15]{f}{a}
    \end{tikzinline}}
  \\
  \tag{$\rcb$}
  \hbox{\begin{tikzinline}[node distance=1.5em]
      \event{a}{\DW{x}{1}}{}
      \raevent{b}{\DW[\mREL]{y}{1}}{right=of a}
      \event{c}{\DR{y}{1}}{right=3em of b}
      \event{d}{\DW{y}{2}}{right=of c}
      \raevent{e}{\DR[\mACQ]{y}{2}}{right=of d}
      \event{f}{\DR{x}{0}}{right=of e}
      \cbz{a}{b}
      \cbz{b}{c}
      \cbz{c}{d}
      %\cbz{d}{e}
      \cbz{e}{f}
      \cbz[out=-165,in=-15]{f}{a}
    \end{tikzinline}}
\end{gather*}

\subsection{RMWs}
It is not possible for two \RMW{}s to see the same write.
\begin{gather*}
  \begin{gathered}
    \PW{x}{0} \SEMI \bigl(\PFADD[\mRLX][\mRLX]{x}{}{1} \PAR \PFADD[\mRLX][\mRLX]{x}{}{1}\bigr)
    \\
    \hbox{\begin{tikzinline}[node distance=2em]
        \event{a0}{\DW{x}{0}}{}
        \event{a1}{\DR{x}{0}}{right=3em of a0}
        \event{a2}{\DW{x}{1}}{right=of a1}
        \event{b1}{\DR{x}{0}}{right=3em of a2}
        \event{b2}{\DW{x}{1}}{right=of b1}
        \rmw{a1}{a2}
        \rf{a0}{a1}
        \rf[out=-15,in=-165]{a0}{b1}
        \wk[out=-15,in=-165]{a1}{b2}
        \wk{b1}{a2}
        \graywk[bend left]{a2}{b1}
        \rmw{b1}{b2}
      \end{tikzinline}}
  \end{gathered}
  \taglabel{rmw0}
\end{gather*}
The gray arrow is required the \RMW{} atomicity axioms.

\citet{DBLP:conf/pldi/LeeCPCHLV20} introduce \PS{2.0} to refine the treatment of
\RMW{}s in the promising semantics (\PS{}).  Their examples have the expected
results here, with far less work.  First they recall that \PS{} requires
quantification over multiple futures in order to disallow executions such as
\ref{CDRF}:
\begin{gather*}
  \taglabel{CDRF}
    \begin{gathered}
      \PFADD[\mACQ][\mREL]{x}{r}{1}\SEMI \IF{r{=}0}\THEN \PW{y}{1} \FI
      \PAR
      \PFADD[\mACQ][\mREL]{x}{r}{1}\SEMI \IF{r{=}0}\THEN \IF{y}\THEN \PW{x}{0} \FI \FI
      \\
      \hbox{\begin{tikzinline}[node distance=2em]
          \event{a1}{\DR[\mACQ]{x}{0}}{}
          \event{a1b}{\DW[\mREL]{x}{1}}{below=1em of a1}
          \event{a2}{\DW{y}{1}}{right=of a1}
          \event{b0}{\DR[\mACQ]{x}{0}}{right=3em of a2}
          \event{b0b}{\DW[\mREL]{x}{1}}{below=1em of b0}
          \event{b1}{\DR{y}{1}}{right=of b0}
          \event{b2}{\DW{x}{0}}{right=of b1}
          \rmw{a1}{a1b}
          \rmw{b0}{b0b}
          \rf[out=-13,in=-163]{a2}{b1}
          \po{a1}{a2}
          \sync{b0}{b1}
          \po{b1}{b2}
          \rf[out=-165,in=-12]{b2}{a1}
        \end{tikzinline}}
    \end{gathered}
  \end{gather*}
This execution is clearly impossible, due to the cycle above.  In this
diagram, we have not drawn order adjacent to the writes of the \RMW{}s, since
this is not necessary to produce the cycle.
If \ref{CDRF} is allowed then \drfra{} fails.


  
\PS{} does not support global value range analysis, as modeled by \ref{GA+E} below.  Our
semantics permits \ref{GA+E}:
\begin{gather*}
  \taglabel{GA+E}
    \begin{gathered}
      \PW{x}{0} \SEMI
      \bigl(
        \PCAS[\mRLX][\mRLX]{x}{r}{0}{1}\SEMI \IF{r{<}10}\THEN \PW{y}{1} \FI
        \PAR
        \PW{x}{42}\SEMI \PW{x}{y}
      \bigr)
      \\
      \hbox{\begin{tikzinline}[node distance=2em]
          \event{a0}{\DW{x}{0}}{}
          \event{a1}{\DR{x}{1}}{right=3em of a0}
          \event{a2}{0{<}10\mid\DW{y}{1}}{right=of a1}
          \event{b0}{\DW{x}{42}}{right=3em of a2}
          \event{b1}{\DR{y}{1}}{right=of b0}
          \event{b2}{\DW{x}{1}}{right=of b1}
          %\rmw{a1}{a2}
          \rf[out=-15,in=-160]{a2}{b1}
          \po{b1}{b2}
          \rf[out=-165,in=-15]{b2}{a1}
          \wk[out=10,in=170]{a0}{b0}
          \wk[out=15,in=165]{b0}{b2}
        \end{tikzinline}}
    \end{gathered}
\end{gather*}
\PS{} also does not support register promotion, as modeled by \ref{RP} below.    Our
semantics permits \ref{RP}:
\begin{gather*}
  \taglabel{RP}
    \begin{gathered}
      \PR{x}{r}\SEMI
      \PFADD[\mRLX][\mRLX]{z}{s}{r}\SEMI \PW{y}{s{+}1}
      \PAR
      \PW{x}{y}
      \\
      \hbox{\begin{tikzinline}[node distance=2em]
          \event{a0}{\DR{x}{1}}{}
          \event{a1}{\DR{z}{0}}{right=of a0}
          \event{a1b}{\DW{z}{1}}{right=of a1}
          \event{a2}{\DW{y}{1}}{right=of a1b}
          \event{b0}{\DR{y}{1}}{right=3em of a2}
          \event{b1}{\DW{x}{1}}{right=of b0}
          \rmw{a1}{a1b}
          \po[out=20,in=160]{a0}{a1b}
          \po[out=20,in=160]{a1}{a2}
          \po{b0}{b1}
          \rf{a2}{b0}
          \rf[out=-165,in=-15]{b1}{a0}
        \end{tikzinline}}
    \end{gathered}
\end{gather*}



These following examples are from ``Modular Data-Race-Freedom Guarantees in
the Promising Semantics'' to appear in PLDI21.

\ref{CDRF} shows that our semantics is not too permissive for $\mRA$-\RMW{}s.
But what about $\mRLX$-\RMW{}s.  The following execution is allowed by \armeight,
and \PS{2.0}, but disallowed by \PS{2.1}.
\begin{gather*}
  \taglabel{RMW-W}
  \begin{gathered}
    \PFADD[\mRLX][\mRLX]{x}{r}{1}\SEMI \PW{y}{1}
    \PAR
    \PR{y}{r}\SEMI \PFADD[\mRLX][\mRLX]{x}{s}{r}
    \\
    \hbox{\begin{tikzinline}[node distance=2em]
        \event{a1}{\DR{x}{1}}{}
        \event{a1b}{\DW{x}{2}}{below=1em of a1}
        \event{a2}{\DW{y}{1}}{right=of a1}
        \event{b1}{\DR{y}{1}}{right=3em of a2}
        \event{b2}{\DR{x}{0}}{right=of b1}
        \event{b2b}{\DW{x}{1}}{below=1em of b2}
        \rmw{a1}{a1b}
        \rmw{b2}{b2b}
        \rf{a2}{b1}
        \po{b1}{b2b}
        \rf[out=-175,in=-20]{b2b}{a1}
      \end{tikzinline}}
  \end{gathered}
\end{gather*}

If this $\ldrfra{z}$?
\begin{gather*}
  \taglabel{Naive-LDRF-RA-Fail}
  \begin{gathered}
    \IF{y}\THEN \PW{x}{z} \ELSE \PW{x}{1} \FI
    \PAR
    \PR{x}{r}\SEMI \PW{z}{1}\SEMI \PW{y}{r}
    \\
    \hbox{\begin{tikzinline}[node distance=2em]
        \event{a1}{\DR{y}{1}}{}
        \event{a2}{\DR{z}{1}}{right=of a1}
        \event{a3}{\DW{x}{1}}{right=of a2}
        \event{b1}{\DR{x}{1}}{right=3em of a3}
        \event{b2}{\DW{z}{1}}{right=of b1}
        \event{b3}{\DW{y}{1}}{right=of b2}
        \po{a2}{a3}
        \po[in=165,out=15]{b1}{b3}
        \rf[out=-165,in=-15]{b2}{a2}
        \rf[out=-165,in=-15]{b3}{a1}
        \rf{a3}{b1}
      \end{tikzinline}}
  \end{gathered}
\intertext{Interpreting $\{z\}$ as $\mRA$:}
    \\
  \begin{gathered}
    \hbox{\begin{tikzinline}[node distance=2em]
        \event{a1}{\DR{y}{1}}{}
        \event{a2}{\DR[\mACQ]{z}{1}}{right=of a1}
        \event{a3}{\DW{x}{1}}{right=of a2}
        \event{b1}{\DR{x}{1}}{right=3em of a3}
        \event{b2}{\DW[\mREL]{z}{1}}{right=of b1}
        \event{b3}{\DW{y}{1}}{right=of b2}
        \po{a2}{a3}
        \po[in=165,out=15]{b1}{b3}
        \rf[out=-165,in=-15]{b2}{a2}
        \rf[out=-165,in=-15]{b3}{a1}
        \rf{a3}{b1}
        \sync{a1}{a2}
        \sync{b2}{b3}
      \end{tikzinline}}
  \end{gathered}
\end{gather*}

Our semantics already disallows \ref{LDRF-Fail-PS}, which is similar to \ref{OOTA4}.
\begin{gather*}  
  \taglabel{LDRF-Fail-PS}
  \begin{gathered}
  \IF{x}\THEN
    \PFADD{w}{}{1}\SEMI
    \PW{y}{1}\SEMI
    \PW{z}{1}
  \FI
  \PAR
  \IF{\BANG z}\THEN
    \PW{x}{1}
  \ELSE
    \IF{\BANG \PFADD{w}{}{1}}\THEN
      \PW{x}{\PR{y}{}}
    \FI
  \FI
    \\
    \hbox{\begin{tikzinline}[node distance=2em]
        \event{a1}{\DR{x}{1}}{}
        \event{a2}{\DR{w}{1}}{right=of a1}
        \event{a3}{\DW{w}{2}}{right=of a2}
        \event{a4}{\DW{y}{1}}{right=of a3}
        \event{a5}{\DW{z}{1}}{right=of a4}
        \event{b1}{\DR{z}{1}}{right=5em of a5}
        \event{b2}{\DR{w}{0}}{right=of b1}
        \event{b3}{\DW{w}{1}}{right=of b2}
        \event{b4}{\DR{y}{1}}{right=of b3}
        \event{b5}{\DW{x}{1}}{right=of b4}
        \rmw{a2}{a3}
        \po[out=15,in=165]{a1}{a3}
        \po[out=15,in=165]{a1}{a4}
        \po[out=15,in=165]{a1}{a5}        
        \rmw{b2}{b3}
        \po{b4}{b5}
        \po[out=15,in=165]{b2}{b5}        
        \po[out=15,in=165]{b1}{b3}
        \rf{a5}{b1}
        \rf[out=15,in=165]{a4}{b4}
        \rf[out=-165,in=-15]{b3}{a2}
        \rf[out=-165,in=-15]{b5}{a1}
      \end{tikzinline}}
  \end{gathered}
\end{gather*}
\begin{gather}
  \taglabel{OOTA4}
  \begin{gathered}
    \PW{y}{x}
    \PAR
    \PR{y}{r} \SEMI \IF{b}\THEN  \PW{x}{r} \SEMI \PW{z}{r} \ELSE \PW{x}{1} \FI
    \PAR
    \PW{b}{1}
    \\[-1ex]
    \hbox{\begin{tikzinline}[node distance=1.5em]
        \event{rx}{\DR{x}{1}}{}
        \event{wy}{\DW{y}{1}}{right=of rx}
        \po{rx}{wy}
        \event{ry}{\DR{y}{1}}{right=3em of wy} 
        \event{wx}{\DW{x}{1}}{right=of ry}
        \event{wz}{\DW{z}{1}}{right=of wx}
        \event{rb}{\DR{b}{1}}{right=of wz}
        \event{wb1}{\DW{b}{1}}{right=3em of rb}
        \po{ry}{wx}
        \rf{wb1}{rb}
        \rf{wy}{ry}
        \rf[out=-170,in=-10]{wx}{rx}
        \po{rb}{wz}
        \po[out=15,in=165]{ry}{wz}
      \end{tikzinline}}
  \end{gathered}  
\end{gather}
\begin{example}
  This definition ensures atomicity, disallowing executions such as
  \cite[Ex.~3.2]{DBLP:journals/pacmpl/PodkopaevLV19}:
  \begin{gather*}
    % \taglabel{RMW1}
    \begin{gathered}
      \PW{x}{0}\SEMI \PINC[\mRLX][\mRLX]{x}{}
      \PAR
      \PW{x}{2}\SEMI \PR{x}{r}
      \\
      \hbox{\begin{tikzinline}[node distance=1.5em]
          \event{a2}{\DR{x}{0}}{}
          \event{a1}{\DW{x}{0}}{left=of a2}
          \rf{a1}{a2}
          \event{a3}{\DW{x}{2}}{right=of a2}
          \wk{a2}{a3}
          \event{b2}{\DW{x}{1}}{right=of a3}
          \event{b3}{\DR{x}{1}}{right=of b2}
          \rmw[out=-15,in=-165]{a2}[below]{b2}
          \wk{a3}{b2}
          \rf{b2}{b3}
          \liftrmw[out=165,in=15]{a3}{a2}
        \end{tikzinline}}
    \end{gathered}
  \end{gather*}
  By \ref{pom-rmw-atomic1}, since $\DWP{x}{2}\xwk\DWP{x}{1}$, it must be that
  $\DWP{x}{2}\xwk\DRP{x}{0}$, creating a cycle.
\end{example}

\begin{example}
  \label{ex:rmw-33}
  Two successful \RMW{}s cannot see the same write:
  \begin{gather*}
    \begin{gathered}
      \PW{x}{0}\SEMI (\PINC[\mRLX][\mRLX]{x}{} \PAR \PINC[\mRLX][\mRLX]{x}{})
      \\
      \hbox{\begin{tikzinline}[node distance=1.5em]
          \event{i}{\DW{x}{0}}{}
          \event{a1}{a{:}\DR{x}{0}}{right=3em of i}
          \event{a2}{b{:}\DW{x}{1}}{right=of a1}
          \event{b1}{c{:}\DR{x}{0}}{right=3em of a2}
          \event{b2}{d{:}\DW{x}{1}}{right=of b1}
          \rmw{a1}{a2}
          \rmw{b1}{b2}
          \rf{i}{a1}
          \rf[out=15,in=165]{i}{b1}
          \wk[out=-15,in=-165]{a1}{b2}
          \liftrmw[out=-15,in=-165]{a2}{b1}
          % \wk{a1}{b2}
          \wk{b1}{a2}
        \end{tikzinline}}
    \end{gathered}
  \end{gather*}
  The order from read-to-write is required by fulfillment.  
  Apply \ref{pom-rmw-atomic1} of the second \RMW{} to $a\xwk d$, we have that $a\xwk c$.  Subsequently
  applying \ref{pom-rmw-atomic2} of the first \RMW{}, we have $b \xwk c$, creating a cycle.
\end{example}

\begin{example}
  By using two actions rather than one, the definition allows examples such as the
  following, which is allowed by \armeight{} 
  \cite[Ex.~3.10]{DBLP:journals/pacmpl/PodkopaevLV19}:
  \begin{gather*}
    % \taglabel{RMW2}
    \begin{gathered}
      \PR{z}{r}\SEMI
      % \PW{x}{0}\SEMI
      \PINC[\mRLX][\mREL]{x}{s} \SEMI
      \PW{y}{s}{+}1
      \PAR
      \PR{y}{r}\SEMI
      \PW{z}{r}
      \\[-1ex]
      \hbox{\begin{tikzinline}[node distance=1.5em]
          \event{b1}{\DR{z}{1}}{}
          % \event{b2}{\DW{x}{0}}{right=of b1}
          \event{b3}{\DR{x}{0}}{right=of b1}
          %\rf{b2}{b3}
          \event{b4}{\DWRel{x}{1}}{right=2em of b3}
          \rmw{b3}{b4}
          \event{b5}{\DW{y}{1}}{right=of b4}
          \sync[out=-20,in=-160]{b1}{b4}
          \po[out=-20,in=-160]{b3}{b5}
          \event{a1}{\DR{y}{1}}{right=3em of b5}
          \event{a2}{\DW{z}{1}}{right=of a1}
          \po{a1}{a2}
          \rf{b5}{a1}
          \rf[out=170,in=10]{a2}{b1}
        \end{tikzinline}}
    \end{gathered}
  \end{gather*}
  A similar example, also allowed by \armeight{}
  \cite[Fig.~6]{DBLP:journals/pacmpl/ChakrabortyV19}:
  \begin{gather*}
    % \taglabel{RMW2}
    \begin{gathered}
      \PR{z}{r}\SEMI
      % \PW{x}{0}\SEMI
      \PFADD[\mRLX][\mRLX]{x}{s}{r} \SEMI
      \PW{y}{s}{+}1
      \PAR
      \PR{y}{r}\SEMI
      \PW{z}{r}
      \\[-1ex]
      \hbox{\begin{tikzinline}[node distance=1.5em]
          \event{b1}{\DR{z}{1}}{}
          %\event{b2}{\DW{x}{0}}{right=of b1}
          \event{b3}{\DR{x}{0}}{right=of b1}
          %\rf{b2}{b3}
          \event{b4}{\DW{x}{1}}{right=2em of b3}
          \rmw{b3}{b4}
          \event{b5}{\DW{y}{1}}{right=of b4}
          \po[out=-20,in=-160]{b1}{b4}
          \po[out=-20,in=-160]{b3}{b5}
          \event{a1}{\DR{y}{1}}{right=3em of b5}
          \event{a2}{\DW{z}{1}}{right=of a1}
          \po{a1}{a2}
          \rf{b5}{a1}
          \rf[out=170,in=10]{a2}{b1}
        \end{tikzinline}}
    \end{gathered}
  \end{gather*}
\end{example}
This is allowed by \weakestmo{}, but not \PS{}.

\begin{example}
  Consider the \textsc{cdrf} example from \cite{DBLP:conf/pldi/LeeCPCHLV20}:
  \begin{gather*}
    \begin{gathered}
      \begin{aligned}
        &\PINC[\mACQ][\mREL]{x}{r}\SEMI \IF{r{=}0}\THEN \PW{y}{1} \FI
        \\\PAR\;\;&
        \PINC[\mACQ][\mREL]{x}{r}\SEMI \IF{r{=}0}\THEN \IF{y}\THEN \PW{x}{0} \FI \FI
      \end{aligned}
      \\
      \hbox{\footnotesize\begin{tikzinline}[node distance=1.5em]
          \raevent{a1}{\DR[\mACQ]{x}{0}}{}
          \raevent{a1b}{\DW[\mREL]{x}{1}}{right=of a1}
          \event{a2}{\DW{y}{1}}{right=of a1b}
          \raevent{b0}{\DR[\mACQ]{x}{0}}{right=3em of a2}
          \raevent{b0b}{\DW[\mREL]{x}{1}}{right=of b0}
          \event{b1}{\DR{y}{1}}{right=of b0b}
          \event{b2}{\DW{x}{0}}{right=of b1}
          \rmw{a1}{a1b}
          \rmw{b0}{b0b}
          \rf[out=-13,in=-163]{a2}{b1}
          \sync[out=20,in=160]{a1}{a2}
          \sync[out=20,in=160]{b0}{b1}
          \po{b1}{b2}
          \rf[out=-165,in=-12]{b2}{a1}
        \end{tikzinline}}
    \end{gathered}
  \end{gather*}
\end{example}

\begin{example}
  Consider this example from \cite[\textsection C]{DBLP:conf/pldi/LeeCPCHLV20}:
  \begin{gather*}
    \begin{gathered}
      \begin{aligned}
        &\PCAS[\mRLX][\mRLX]{x}{r}{0}{1}\SEMI \IF{r{\leq}1}\THEN \PW{y}{1} \FI
        \\\PAR\;\;&
        \PCAS[\mRLX][\mRLX]{x}{r}{0}{2}\SEMI \IF{r{=}0}\THEN \IF{y}\THEN \PW{x}{0} \FI \FI
      \end{aligned}
      \\
      \hbox{\footnotesize\begin{tikzinline}[node distance=1.5em]
          \event{a1}{\DR{x}{0}}{}
          \event{a1b}{\DW{x}{1}}{right=of a1}
          \event{a2}{\DW{y}{1}}{right=of a1b}
          \event{b0}{\DR{x}{0}}{right=3em of a2}
          \event{b0b}{\DW{x}{2}}{right=of b0}
          \event{b1}{\DR{y}{1}}{right=of b0b}
          \event{b2}{\DW{x}{0}}{right=of b1}
          \rmw{a1}{a1b}
          \rmw{b0}{b0b}
          \rf[out=-13,in=-163]{a2}{b1}
          \po[out=20,in=160]{a1}{a2}
          \po[out=20,in=160]{b0}{b1}
          \po{b1}{b2}
          \rf[out=-165,in=-12]{b2}{a1}
        \end{tikzinline}}
    \end{gathered}
  \end{gather*}
\end{example}

\subsection{Coherence}

The following execution is disallowed by fulfillment.
\begin{gather*}
  \tag{\textsc{coh}}
  \begin{gathered}
    \PW{x}{1}\SEMI
    \PR{x}{r}
    \PAR
    \PW{x}{2}\SEMI
    \PR{x}{s}
    \\\nonumber
    \hbox{\begin{tikzinline}[node distance=1.5em]
        \event{a1}{\DW{x}{1}}{}
        \event{a2}{\DR{x}{2}}{right=of a1}
        \event{b1}{\DW{x}{2}}{right=3em of a2}
        \event{b2}{\DR{x}{1}}{right=of b1}
        \wki{a1}{a2}
        \wki{b1}{b2}
        \rf{b1}{a2}
        \rf[out=20,in=160]{a1}{b2}
        \wk[out=15,in=155]{a1}{b1}
        \wk[out=-155,in=-15]{b1}{a1}
      \end{tikzinline}}
  \end{gathered}
\end{gather*}

Our model is more coherent than Java, which permits the following:
\begin{gather*}
  \taglabel{TC16}
  \begin{gathered}
    \PR{x}{r}\SEMI \PW{x}{1}
    \PAR
    \PR{x}{s}\SEMI \PW{x}{2}
    \\[-1ex]
    \hbox{\begin{tikzinline}[node distance=1.5em]
        \event{a1}{\DR{x}{2}}{}
        \event{a2}{\DW{x}{1}}{right=of a1}
        \wki{a1}{a2}
        \event{b1}{\DR{x}{1}}{right=3em of a2}
        \event{b2}{\DW{x}{2}}{right=of b1}
        \wki{b1}{b2}
        \rf{a2}{b1}
        \rf[out=-165,in=-15]{b2}{a1}
      \end{tikzinline}}
  \end{gathered}
\end{gather*}
We also forbid the following, which Java allows:
\begin{gather*}
  \taglabel{Co3}
  \begin{gathered}
    \PW{x}{1}\SEMI \PW[\mRA]{y}{1}
    \PAR
    \PW{x}{2}\SEMI \PW[\mRA]{z}{1}
    \PAR
    \PR[\mRA]{z}{r} \SEMI 
    \PR[\mRA]{y}{r} \SEMI 
    \PR{x}{r} \SEMI 
    \PR{x}{r}
    \\[-1ex]
    \hbox{\begin{tikzinline}[node distance=1.5em]
        \event{a1}{\DW{x}{1}}{}
        \event{a2}{\DW[\mRA]{y}{1}}{right=of a1}
        \sync{a1}{a2}
        \event{b1}{\DW{x}{2}}{right=3em of a2}
        \event{b2}{\DW[\mRA]{\,z}{1}}{right=of b1}
        \sync{b1}{b2}
        \event{c1}{\DR[\mRA]{\,z}{1}}{right=3em of b2}
        \event{c2}{\DR[\mRA]{y}{1}}{right=of c1}
        \event{c3}{\DR{x}{2}}{right=of c2}
        \event{c4}{\DR{x}{1}}{right=of c3}
        \sync{c1}{c2}
        \sync{c2}{c3}
        \sync[out=20,in=160]{c2}{c4}
        \rf[out=8,in=172]{a2}{c2}
        \rf{b2}{c1}
        \wk[out=19,in=161]{a1}{b1}
        \wk[out=-172,in=-8]{c4}{b1}
      \end{tikzinline}}
  \end{gathered}
\end{gather*}


The following outcome is allowed by the promising semantics
\cite{DBLP:conf/popl/KangHLVD17}, but not in \weakestmo{}
\cite[Fig.~3]{DBLP:journals/pacmpl/ChakrabortyV19} nor in our semantics, due
to the cycle:
\begin{gather*}
  \tag{\textsc{coh-cyc}}
  \begin{gathered}
    x\GETS 2\SEMI
    \IF{x\NOTEQ2}\THEN y\GETS 1 \FI
    \PAR
    x\GETS 1\SEMI
    r\GETS x\SEMI
    \IF{y}\THEN x\GETS 3 \FI
    \\\nonumber
    \hbox{\begin{tikzinline}[node distance=1.5em]
        \event{wx2}{\DW{x}{2}}{}
        \event{rx3}{\DR{x}{3}}{right=of wx2}
        \wki{wx2}{rx3}
        \event{wy1}{\DW{y}{1}}{right=of rx3}
        \po{rx3}{wy1}
        \event{wx1}{\DW{x}{1}}{right=2em of wy1}
        \event{rx2}{\DR{x}{2}}{right=of wx1}
        \wki{wx1}{rx2}
        \event{ry1}{\DR{y}{1}}{right=of rx2}
        \event{wx3}{\DW{x}{3}}{right=of ry1}
        \po{ry1}{wx3}
        \wki[in=165,out=15]{rx2}{wx3}
        \rf[in=-170,out=-10]{wy1}{ry1}
        \rf[in=170,out=10]{wx2}{rx2}
        \rf[out=-170,in=-10]{wx3}{rx3}
        \wk[out=-170,in=-10]{wx1}{wx2}
      \end{tikzinline}}
  \end{gathered}
\end{gather*}

Since reads are not ordered by intra-thread coherence,
we {allow} the following unintuitive behavior. C11 includes read-read
coherence between relaxed atomics in order to forbid this:
\begin{gather*}
  \taglabel{Co2}
  \begin{gathered}
    \PW{x}{1}\SEMI \PW{x}{2}
    \PAR
    \PW{y}{x} \SEMI \PW{z}{x}
    \\[-1ex]
    \hbox{\begin{tikzinline}[node distance=1.5em]
        \event{a}{\DW{x}{1}}{}
        \event{b}{\DW{x}{2}}{right=of a}
        \wki{a}{b}
        \event{c}{\DR{x}{2}}{right=3em of b}
        \event{d}{\DW{y}{2}}{right=of c}
        \po{c}{d}
        \event{e}{\DR{x}{1}}{right=of d}
        \event{f}{\DW{z}{1}}{right=of e}
        \po{e}{f}
        \rf{b}{c}
        \rf[out=10,in=170]{a}{e}
        \wk[out=-165,in=-15]{e}{b}
      \end{tikzinline}}
  \end{gathered}
\end{gather*}
Here, the reader sees $2$ then $1$, although they are written in the reverse
order.
This behavior is allowed by Java in order to validate CSE without requiring
aliasing analysis.

\subsection{MCA}

\begin{gather*}
  \taglabel{MCA1}
  \begin{gathered}
    \IF{z}\THEN \PW{x}{0} \FI \SEMI \PW{x}{1}
    {\PAR}
    \IF{x}\THEN \PW{y}{0} \FI \SEMI \PW{y}{1}
    {\PAR}
    \IF{y}\THEN \PW{z}{0} \FI \SEMI \PW{z}{1}
    \\[-1ex]
    \hbox{\begin{tikzinline}[node distance=1.5em]
        \event{a1}{\DR{z}{1}}{}
        \event{a2}{\DW{x}{0}}{right=of a1}
        \po{a1}{a2}
        \event{a3}{\DW{x}{1}}{right=of a2}
        \wki{a2}{a3}
        \event{b1}{\DR{x}{1}}{right=3em of a3}
        \event{b2}{\DW{y}{0}}{right=of b1}
        \po{b1}{b2}
        \event{b3}{\DW{y}{1}}{right=of b2}
        \wki{b2}{b3}
        \event{c1}{\DR{y}{1}}{right=3em of b3}
        \event{c2}{\DW{z}{0}}{right=of c1}
        \po{c1}{c2}
        \event{c3}{\DW{z}{1}}{right=of c2}
        \wki{c2}{c3}
        \rf{a3}{b1}
        \rf{b3}{c1}
        \rf[out=173,in=7]{c3}{a1}  
      \end{tikzinline}}
  \end{gathered}
  \\[1ex]
  \taglabel{MCA2}
  \begin{gathered}
    \PW{x}{0}\SEMI \PW{x}{1}
    \PAR
    \PW{y}{x}
    \PAR
    \PR[\mRA]{y}{r} \SEMI \PR{x}{s}
    \\[-1ex]
    \hbox{\begin{tikzinline}[node distance=1.5em]
        \event{wx0}{\DW{x}{0}}{}
        \event{wx1}{\DW{x}{1}}{right=of wx0}
        \wki{wx0}{wx1}
        \event{rx1}{\DR{x}{1}}{right=3em of wx1}
        \event{wy1}{\DW{y}{1}}{right=of rx1}
        \po{rx1}{wy1}
        \event{ry1}{\DRAcq{y}{1}}{right=3em of wy1}
        \event{rx0}{\DR{x}{0}}{right=of ry1}
        \rf{wx1}{rx1}
        \rf{wy1}{ry1}
        \sync{ry1}{rx0}
        \wk[out=170,in=10]{rx0}{wx1}
      \end{tikzinline}}
  \end{gathered}
\end{gather*}

These candidate executions are invalid, due to cycles.

\subsection{IRIW}
Status of IRIW is unclear in our model, since we allow everything allowed by
power...
\begin{gather*}
  \begin{gathered}
    % \PW{x}{0}\SEMI
    \PW{x}{1}
    \PAR
    \PR[\mRA]{x}{r}\SEMI \PR{y}{s}
    \PAR
    % \PW{y}{0}\SEMI
    \PW{y}{1}
    \PAR
    \PR[\mRA]{y}{s} \SEMI \PR{x}{r}
    \\
    %\smash[b]{
      \hbox{\begin{tikzinline}[node distance=1.5em]
          % \event{wx0}{\DW{x}{0}}{}
          % \event{wx1}{\DW{x}{1}}{right=of wx0}
          % \event{wy0}{\DW{y}{0}}{below=4ex of wx0}
          % \event{wy1}{\DW{y}{1}}{right=of wy0}
          \event{wx1}{\DW{x}{1}}{}
          \event{rx1}{\DR[\mRA]{x}{1}}{right=3em of wx1}
          \event{ry0}{\DR{y}{0}}{right=of rx1}
          \event{wy1}{\DW{y}{1}}{right=3em of ry0}
          \event{ry1}{\DR[\mRA]{y}{1}}{right=3em of wy1}
          \event{rx0}{\DR{x}{0}}{right=of ry1}
          % \wk{wx0}{wx1}
          % \wk{wy0}{wy1}
          % \rf[bend left]{wy0}{ry0}
          % \rf[bend right]{wx0}{rx0}
          \sync{rx1}{ry0}
          \sync{ry1}{rx0}
          \rf{wx1}{rx1}
          \rf{wy1}{ry1}
          \wk[out=170,in=10]{rx0}{wx1}
          \wk{ry0}{wy1}
        \end{tikzinline}}
    %}
  \end{gathered}
\end{gather*}


\endinput
\section{More Stuff}

\subsection{If Closure and Address Dependencies}
\label{sec:addr}

An optimization ($p$/$q$ are registers):
\begin{displaymath}
  \PRREF{p}{r}\SEMI
  \PRREF{q}{s}
\end{displaymath}
vs
\begin{displaymath}
  \PRREF{p}{r}\SEMI
  \IF{p{=}q}\THEN \LET{s}{r} \ELSE \PRREF{q}{s}\FI
\end{displaymath}

\begin{displaymath}
  \LET{r}{\mathtt{new}}\SEMI
  \PW{\REF{r}}{42}\SEMI
  \PR{\REF{r}}{s}\SEMI
  \PW{x}{r}
  \PAR
  \PR{x}{r}\SEMI
  \PW{\REF{r}}{7}
\end{displaymath}

If closure is at odds with Java Final field semantics.

Do sequencing and if commute?

\subsection{About Arm}
Hypothesis: gcb cannot contradict (poloc minus RxR).


\subsection{Completed Pomsets and Fork}
\label{sec:fork}

It is sometimes useful to distinguish \emph{terminated} or \emph{completed}
executions from partial executions.  For example in
\begin{math}
  \sem{\PW{x}{1}\SEMI\PW{y}{1}},
\end{math}
we expect completed executions to include two write actions.  Note that this
is different from being downset-maximal.
\begin{gather}
  \nonumber
  \PW{x}{0} \SEMI \PW{x}{1}
  \PAR
  \PR{x}{r}\SEMI\PR{x}{s}\SEMI\IF{s}\THEN\PW{y}{1}\FI
  \\
  \label{down1}
  \hbox{\begin{tikzinline}[node distance=0.5em and 1.5em]
      \event{a}{\DW{x}{0}}{}
      \event{b}{\DW{x}{1}}{right=of a}
      \event{c}{\DR{x}{1}}{right=3em of b}
      \event{d}{\DR{x}{0}}{right=of c}
      \wki{a}{b}
      \rf{b}{c}
      \rf[out=-20,in=-160]{a}{d}
    \end{tikzinline}}
  \\
  \label{down2}
  \hbox{\begin{tikzinline}[node distance=0.5em and 1.5em]
      \event{a}{\DW{x}{0}}{}
      \event{b}{\DW{x}{1}}{right=of a}
      \event{c}{\DR{x}{0}}{right=3em of b}
      \event{d}{\DR{x}{1}}{right=of c}
      \event{e}{\DW{y}{1}}{right=of d}
      \po{d}{e}
      \wki{a}{b}
      \rf[out=-20,in=-160]{a}{c}
      \rf[out=-20,in=-160]{b}{d}
    \end{tikzinline}}
\end{gather}
\eqref{down1} is a downset of \eqref{down2}, but both are completed. 

For pomsets with predicate transformers, we identify \emph{completion} with
\emph{quiescence.}
\begin{definition}
  \label{def:completed}
  A pomset with predicate transformers $\aPS$ is \emph{completed} if
  $\aTr{\aEvs}{\aSym} \rimplies \aSym$, for every quiescence symbol $\aSym$.
\end{definition}
For example, there are no pomsets in $\sem{\ABORT}$ that are completed,
whereas the augment-minimal pomset of $\sem{\SKIP}$ (which has the identity
transformer) is completed.

% While this definition is sensible for single \emph{threads}, it is less
% satisfying for thread \emph{groups}.  To see why, consider that in
% $\sem{\FORK{\THREAD{\aCmd}}}$:
% \begin{itemize}
% \item by \ref{T3full}, quiescence symbols and the symbol $\RW$ have been
%   substituted out of preconditions $\labelingForm(\aEv)$,
% \item by \ref{F4}, every predicate transformer $\aTr{\bEvs}{}$ is the
%   identity function. %, for any $\bEvs$.
% \end{itemize}
% Every pomset in $\sem{\FORK{\aGrp}}$ is completed, by definition. As a
% result, in general, $\sem{\FORK{\THREAD{\aCmd}}}\neq\sem{\aCmd}$.

The $\FORK{}$ operation is asynchronous: In
$\sem{\aCmd_1\SEMI\aCmd_2\SEMI\FORK{\aGrp}\SEMI \aCmd_3}$, the only order enforced
between $\sem{\aCmd_3}$ and $\sem{\aGrp}$ comes from
quiescence preconditions in $\sem{\aGrp}$; the transformer of
$\sem{\FORK{\aGrp}}$ is the identity transformer, thus $\sem{\aGrp}$ runs
concurrently with $\sem{\aCmd_3}$.  In addition, if $\aCmd_2$ includes no
releases and the locations of $\aCmd_2$ are disjoint from those of $\aGrp$,
then $\sem{\aGrp}$ run concurrently with $\sem{\aCmd_2\SEMI\aCmd_3}$.
% $\FORK{}$ %(\textsection\ref{sec:pomsets-trans})
% does not introduce barriers:
\begin{gather*}
  \PW{x}{1}\SEMI \PW{y}{1}\SEMI\FORK{\THREAD{\PW{x}{2}}}
  \\
  \hbox{\begin{tikzinline}[node distance=0.5em and 1.5em]
      \event{a}{\DW{x}{1}}{}
      \event{b}{\DW{y}{1}}{right=of a}
      \event{c}{\DW{x}{2}}{right=3em of b}
      \wki[out=-20,in=-160]{a}{c}
    \end{tikzinline}}
\end{gather*}
% In fact, perhaps surprisingly,
% \begin{math}
%   \sem{\PR{x}{r}\SEMI\FORK{\THREAD{\PW{x}{1}}}} = \sem{\FORK{\THREAD{\PW{x}{1}}}\SEMI\PR{x}{r}}.
% \end{math}
% Order between the threads
% can be enforced using synchronization.  For example, the ``backwards'' read
% above is forbidden in:
% \begin{gather*}
%   \PR{x}{r}\SEMI\PW[\mREL]{z}{1}\SEMI\FORK{\THREAD{\IF{\PR[\mACQ]{z}{}}\THEN\PW{x}{1}\FI}}
%   \\
%   \hbox{\begin{tikzinline}[node distance=0.5em and 1.5em]
%     \event{a}{\DR{x}{1}}{}
%     \event{b}{\DW{z}{1}}{right=of a}
%     \event{c}{\DR{z}{1}}{right=3em of b}
%     \event{d}{\DW{x}{1}}{right=of c}
%     \rf{b}{c}
%     \sync{a}{b}
%     \sync{c}{d}
%     \rf[out=-160,in=-20]{d}{a}
%   \end{tikzinline}}
% \end{gather*}

% [Proposal: add level of syntax for full programs and do all thread
% substitutions there... have one level of syntax around for each semantic
% categories.]

\subsection{Fork-Join Parallelism}

In this subsection, we model a variant of our language that removes the
asynchronous $\FORK{}$ operation and adds a synchronous \emph{fork-join},
which we define using an asymmetric operator for parallel composition, as in
\cite{DBLP:conf/icfp/FerreiraHJ96}.

We remove thread groups as a separate category in the syntax and semantics.
The transformer for the \emph{left composition} $\aCmd_1\LPAR\aCmd_2$ takes
the register state from the left, but quiescence from both sides.
\begin{align*}
  \aCmd
  \BNFDEF& \ABORT
  \BNFSEP \SKIP
  \BNFSEP \LET{\aReg}{\aExp}
  % \BNFSEP \PR[\amode]{\aLoc}{\aReg}
  % \BNFSEP \PW[\amode]{\aLoc}{\aExp}
  \BNFSEP \PRREF[\amode]{\cExp}{\aReg}
  \BNFSEP \PWREF[\amode]{\cExp}{\aExp}
  % \BNFSEP \PA{\aLoc}{\aExp} 
  \\[-.5ex]
  \BNFSEP& \aCmd_1 \LPAR \aCmd_2
  \BNFSEP \aCmd_1 \SEMI \aCmd_2
  \BNFSEP \IF{\aExp} \THEN \aCmd_1 \ELSE \aCmd_2 \FI  
\end{align*}

\begin{definition}
  If $\aPS \in (\aPSS_1\sLPAR\aPSS_2)$ then
  $(\exists\aPS_1\in\aPSS_1)$ $(\exists\aPS_2\in\aPSS_2)$
  % there are $\aPS_1\in\aPSS_1$ and $\aPS_2\in\aPSS_2$ such that
  \begin{enumerate}
    % \setcounter{enumi}{\value{pomsetPreParCount}}
  \item[\ref{par-E}--\ref{par-kappa2})]
    as for $\sLPAR{}{}$ in \refdef{def:pomsets-pre},
  \item \label{par-tau1}
    $\aTr{\bEvs}{\bForm} \rimplies \aTr[1]{\bEvs}{\bForm}$,
  \item \label{par-tau2}
    $\aTr{\bEvs}{\aSym} \rimplies \aTr[2]{\bEvs}{\aSym}$,
    for every quiescence symbol $\aSym$.
  \end{enumerate}
\end{definition}

We now interpret pomsets with predicate transformers directly as
\emph{top-level} (rather than passing through the intermediate interpretation
of thread groups as pomsets with preconditions).  We only consider
\emph{completed} pomsets with predicate transformers to be top-level.
\begin{definition}
  \label{def:pomsets-top2}
  \labeltext[\ensuremath{\sTOP{}}]{}{sTOP2}
  \noindent
  If $\aPS \in \sTOP{\aPSS}$ then
  $(\exists\aPS_1\in\aPSS)$
  \begin{enumerate}
  \item[1-5)] as in \refdef{def:pomsets-top},
    % \item  \label{top2-E} % [{\labeltext[T1]{T1)}{T1}}]
    %   $\aEvs=\aEvs_1$,
    % \item  \label{top2-lambda} % [{\labeltext[T2]{T2)}{T2}}]
    %   $\labelingAct(\aEv) = \labelingAct_1(\aEv)$,
    % \item  \label{top2-le} % [{\labeltext[T2]{T2)}{T2}}]
    %   if $\bEv\le_1\aEv$ then $\bEv\le\aEv$, 
    % \item  \label{top2-kappa-write} % [{\labeltext[T3]{T3)}{T3}}]
    %   if $\labelingAct_1(\aEv)$ is a write, $\labelingForm_1(\aEv) [\TRUE/\Q{}][\TRUE/\RW]$ is a tautology,
    % \item  \label{top2-kappa-read} % [{\labeltext[T3]{T3)}{T3}}]
    %   if $\labelingAct_1(\aEv)$ is a read,
    %   $\labelingForm_1(\aEv) [\TRUE/\Q{}][\FALSE/\RW]$ is a tautology and
    %   $\aEv$ is fulfilled (\refdef{def:fulfilled}),
  \item[6)]  \label{top2-tau} % [{\labeltext[T3]{T3)}{T3}}]
    $\aTr[1]{\aEvs_1}{\aSym} \rimplies \aSym$, for every quiescence symbol $\aSym$.
  \end{enumerate}  
\end{definition}

\subsection{Fork-Join Parallelism 2}
\label{sec:join2}

In this subsection, we model a variant of our language that removes the
asynchronous $\FORK{}$ operation and adds a synchronous $\FORKJOIN{}$.
\begin{align*}
  \aCmd
  \BNFDEF& \ABORT
  \BNFSEP \SKIP
  \BNFSEP \LET{\aReg}{\aExp}
  % \BNFSEP \PR[\amode]{\aLoc}{\aReg}
  % \BNFSEP \PW[\amode]{\aLoc}{\aExp}
  \BNFSEP \PRREF[\amode]{\cExp}{\aReg}
  \BNFSEP \PWREF[\amode]{\cExp}{\aExp}
  % \BNFSEP \PA{\aLoc}{\aExp} 
  \\[-.5ex]
  \BNFSEP& \FORKJOIN{\aGrp}
  \BNFSEP \aCmd_1 \SEMI \aCmd_2
  \BNFSEP \IF{\aExp} \THEN \aCmd_1 \ELSE \aCmd_2 \FI
\end{align*}
In $\PBR{\aCmd_1\SEMI\FORKJOIN{\aGrp}\SEMI\aCmd_2}$, ${\aCmd_1}$ must
complete before ${\aGrp}$ begins, and threads in ${\aGrp}$ must complete
before ${\aCmd_2}$ begins.  Thus $\PBR{\FORKJOIN{\THREAD{\PR{x}{r}}}}$ acts
like a full fence.
% 
As modeled here, however, if $\aGrp$ is empty, no order is imposed between
${\aCmd_1}$ and ${\aCmd_2}$.  Thus
$\sem{\FORKJOIN{\THREAD{\SKIP}}}=\sem{\SKIP}$.

To model $\FORKJOIN{}$, we give the semantics of thread groups using pomsets
with preconditions \emph{and termination}.
\begin{definition}
  \label{def:pomsets-term}
  A \emph{pomset with preconditions and termination} is a pomset with
  preconditions (\refdef{def:pomsets-pre}) together with a termination
  predicate (notation $\TICK$).
\end{definition}

% The definition is a small change relative to that of
% \textsection\ref{sec:pomsets-trans}.

% Define $\sTHREAD{}$ to transform a pomset with predicate transformers into
% a pomset with preconditions and termination by dropping the predicate
% transformer and setting $\TICK$ to indicate whether the pomset was
% completed.

% Extend the definition of $\sNIL$ so that $\TICK$ is true.

% Extend the definition of $\sLPAR{}{}$ to handle for $\TICK$ by adding the
% following.
% \begin{enumerate}
%   \setcounter{enumi}{\value{pomsetPreParCount}}
% \item \label{par-tick} if $\TICK$ then $\TICK_1$ and $\TICK_2$.
% \end{enumerate}

% Similarly, $\sFORKJOIN{}$ extends $\sFORK{}$ by adding the following.
% % \noindent
% % If $\aPS \in \sFORKJOIN{\aPSS}$ then
% % $(\exists\aPS_1\in\aPSS)$
% \begin{enumerate}
%   \setcounter{enumi}{\value{pomsetXForkCount}}
% \item $\TICK_1$.
% \end{enumerate}

\begin{definition}%$\phantom{\;}$\par
  \label{def:pomsets-fj}
  % \noindent
  % If $\aPS\in\sNIL$ then $\aEvs = \emptyset$ and $\TICK$.
  \noindent
  If $\aPS \in \sTHREAD{\aPSS}$ then $(\exists\aPS_1\in\aPSS)$
  \begin{enumerate}
    % \setcounter{enumi}{\value{pomsetXThreadCount}}
  \item[\ref{thread-E}--\ref{thread-kappa})] as for $\sTHREAD{}$ in \refdef{def:pomsets-group}, %, repeating 3 below,% \reffig{fig:full},    
    % \item[\ref{T3full})]
    %   $\labelingForm(\aEv)$ implies
    %   $\labelingForm_1(\aEv) [\TRUE/\Q{}][\TRUE/\RW]$ if $\labelingAct_1(\aEv)$ is a write,
    %   \\
    %   $\labelingForm(\aEv)$ implies
    %   $\labelingForm_1(\aEv) [\TRUE/\Q{}][\FALSE/\RW]$ otherwise.
    
  \item %[{\labeltext[T4]{T4)}{T4}}]
    if $\TICK$ then $\aPS$ is completed (\refdef{def:completed}).
    % $\aTr{\aEvs}{\Q{}} \rimplies \Q{}$.
  \end{enumerate}

  \noindent
  If $\aPS \in \sLPAR{\aPSS_1}{\aPSS_2}$ then $(\exists\aPS_1\in\aPSS_1)$
  $(\exists\aPS_2\in\aPSS_2)$
  \begin{enumerate}
    % \setcounter{enumi}{\value{pomsetPreParCount}}
  \item[\ref{par-E}--\ref{par-kappa2})] as for $\sLPAR{}{}$ in
    \refdef{def:pomsets-pre},
  \item \label{par-tick} $\TICK \rimplies \TICK_1\land\TICK_2$.
  \end{enumerate}

  \noindent
  If $\aPS \in \sFORKJOIN{\aPSS}$ then $(\exists\aPS_1\in\aPSS)$
  \begin{enumerate}
    % \setcounter{enumi}{\value{pomsetXForkCount}}
  \item[\ref{fork-E}--\ref{fork-le})] as for $\sFORK{}$ in \refdef{def:pomsets-group},
  \item
    $\labelingForm(\aEv) \rimplies \labelingForm_1(\aEv)$,    
  \item 
    $\labelingForm(\aEv) \rimplies \aSym$, for every quiescence symbol $\aSym$,
  \item
    $\aTr{\bEvs}{\bForm} \rimplies \bForm$, if $\bEvs=\aEvs$ and $\TICK_1$,
  \item %[{\labeltext[F5]{F5)}{F5}}]
    $\aTr{\bEvs}{\bForm} \rimplies \bForm[\FALSE/\Q{}]$, otherwise.
  \end{enumerate}
\end{definition}
\begin{definition}
  Update \refdef{def:sem-funs} to include:
  \begin{align*}
    \sem{\FORKJOIN{\aGrp}} = \sFORKJOIN{}\sem{\aGrp}  
  \end{align*}
\end{definition}

We embed pomsets with predicate transformers into pomsets with preconditions
and termination using {completion}.  The rules for thread groups keep track
of the termination predicate.
As noted in \textsection\ref{sec:fork}, every pomset in $\sem{\FORK{\aGrp}}$ is
completed.  In contrast, a pomset in $\sem{\FORKJOIN{\aGrp}}$ is completed
only if every thread in $\aGrp$ is completed.

Top-level thread groups do not need quiescence symbols; thus, $\sTHREAD{}$
removes all quiescence symbols by substitution.  However, $\sFORKJOIN{\aPSS}$
adds every possible quiescence symbol as a precondition to the events of
$\aPSS$.  For example, the preconditions of $\sem{\THREAD{\aCmd}\PAR\NIL}$ do
not contain quiescence symbols.  Instead, the preconditions of
$\sem{\FORKJOIN{\THREAD{\aCmd}\PAR\NIL}}$ are saturated with them.  As a
result, in completed top-level pomsets of
$\sem{\aCmd_1\SEMI\FORKJOIN{\aGrp}}$, all of the events from
$\sem{\aCmd_1}$ must precede those of $\sem{\aGrp}$.

A similar thing happens with predicate transformers.  Thread groups in
$\sem{\THREAD{\aCmd}\PAR\NIL}$ do not contain predicate transformers.
Instead, all of the independent predicate transformers of
$\sem{\FORKJOIN{\THREAD{\aCmd}\PAR\NIL}}$ take $\bForm$ to
$\bForm[\FALSE/\Q{}]$.  As a result, in completed top-level pomsets of
$\sem{\FORKJOIN{\aGrp}\SEMI\aCmd_2}$, all of the events from $\sem{\aGrp}$
must precede those of $\sem{\aCmd_2}$.


% The $\JOIN$ operation requires a full synchronization, but $\FORK{}$ does
% not.  The following execution is allowed.
% \begin{gather*}
%   \PR{x}{r}\SEMI\FORKJOIN{\THREAD{\PW{x}{1}}}\SEMI\PW{y}{1}
%   \\
%   \hbox{\begin{tikzinline}[node distance=0.5em and 1.5em]      
%     \event{a}{\DW{x}{1}}{}
%     \event{b}{\DW{y}{1}}{right=3em of a}
%     \event{c}{\DR{x}{1}}{left=3em of a}
%     \rf{a}{c}
%     \sync{a}{b}
%     \sync[out=-20,in=-160]{c}{b}
%   \end{tikzinline}}
% \end{gather*}
% Synchronization can be added to 
% This asymmetry arises naturally when using pomsets with preconditions to
% model thread groups.


\subsection{Coherence/Synchronization via Reordering}
\label{sec:independency-ra}
\begin{scope}
  \showRAtrue

  In \textsection\ref{sec:sync}, we encoded coherence and synchronized access
  using quiescence symbols.  Building on the language with $\FORKJOIN{}$, it
  is possible to model these using reorderability
  (\textsection\ref{sec:pomsets}), rather than encoding them in the logic.
  With synchronization, the relationship becomes asymmetric.

  To capture completion, we use a single quiescence symbol: $\Q{}$.


  Update actions to include access modes: $\DWP[\amode]{\aLoc}{\aVal}$ and
  $\DRP[\amode]{\aLoc}{\aVal}$.
  Reorderability of two sequential actions is determined, in part, by the modes
  of the two actions, capturing synchronization:
  \begin{center}
    \begin{tabular}{c|ccc|ccc}
      &  \multicolumn{6}{|c}{$2^{\text{nd}}$} \\
      \hline
      $1^{\text{st}}$
      & $\DR[\mRLX]{}{}$ & $\DR[\mACQ]{}{}$ & $\DR[\mSC]{}{}$ & $\DW[\mRLX]{}{}$ & $\DW[\mREL]{}{}$ & $\DW[\mSC]{}{}$\\
      \hline
      $\DR[\mRLX]{}{}$ & \cmark           & \cmark          & \cmark          & \cmark           & \xmark          & \xmark         \\
      $\DR[\mACQ]{}{}$  & \xmark           & \xmark          & \xmark          & \xmark           & \xmark          & \xmark         \\
      $\DR[\mSC]{}{}$  & \xmark           & \xmark          & \xmark          & \xmark           & \xmark          & \xmark         \\
      \hline
      $\DW[\mRLX]{}{}$ & \cmark           & \cmark          & \cmark          & \cmark           & \xmark          & \xmark         \\
      $\DW[\mREL]{}{}$  & \cmark           & \cmark          & \cmark          & \cmark           & \xmark          & \xmark         \\
      $\DW[\mSC]{}{}$  & \cmark           & \cmark          & \xmark          & \cmark           & \xmark          & \xmark 
    \end{tabular}
  \end{center}

  % Least permissive for fences:
  \begin{comment}
    \showRAtrue
    \setlength{\tabcolsep}{4pt}
    \begin{tabular}{c|cccc|cccc|c}
      &  \multicolumn{9}{|c}{$2^{\text{nd}}$} \\
      \hline
      $1^{\text{st}}$
      & $\DR[\mRLX]{}{}$ & $\DR[\mACQ]{}{}$ & $\DR[\mSC]{}{}$ & $\DF{\fACQ}$ & $\DW[\mRLX]{}{}$ & $\DW[\mREL]{}{}$ & $\DW[\mSC]{}{}$ & $\DF{\fREL}$&$\DF{\mSC}$\\
      \hline                                                                                                                                                      
      $\DR[\mRLX]{}{}$ & \cmark           & \cmark          & \cmark          & \xmark       & \cmark           & \xmark          & \xmark          & \xmark      & \xmark    \\
      $\DR[\mACQ]{}{}$  & \xmark           & \xmark          & \xmark          & \xmark       & \xmark           & \xmark          & \xmark          & \xmark      & \xmark    \\
      $\DR[\mSC]{}{}$  & \xmark           & \xmark          & \xmark          & \xmark       & \xmark           & \xmark          & \xmark          & \xmark      & \xmark    \\
      $\DF{\fACQ}$     & \xmark           & \xmark          & \xmark          & \xmark       & \xmark           & \xmark          & \xmark          & \xmark      & \xmark    \\
      \hline                                                                                                                                                      
      $\DW[\mRLX]{}{}$ & \cmark           & \cmark          & \cmark          & \cmark       & \cmark           & \xmark          & \xmark          & \xmark      & \xmark    \\
      $\DW[\mREL]{}{}$  & \cmark           & \cmark          & \cmark          & \cmark       & \cmark           & \xmark          & \xmark          & \xmark      & \xmark    \\
      $\DW[\mSC]{}{}$  & \cmark           & \cmark          & \xmark          & \cmark       & \cmark           & \xmark          & \xmark          & \xmark      & \xmark    \\
      $\DF{\fREL}$     & \cmark           & \cmark          & \cmark          & \cmark       & \xmark           & \xmark          & \xmark          & \xmark      & \xmark    \\
      \hline                                                                                                                                                      
      $\DF{\mSC}$      & \xmark           & \xmark          & \xmark          & \xmark       & \xmark           & \xmark          & \xmark          & \xmark      & \xmark 
    \end{tabular}
  \end{comment}
  It seems that fences generally do not commute, except for road-motel.
  \begin{center}
    \showRAtrue
    \setlength{\tabcolsep}{4pt}
    \begin{tabular}{c|ccc|ccc|ccc}
      &  \multicolumn{9}{|c}{$2^{\text{nd}}$} \\
      \hline
      $1^{\text{st}}$
      & $\DR[\mRLX]{}{}$ & $\DR[\mACQ]{}{}$ & $\DR[\mSC]{}{}$ & $\DW[\mRLX]{}{}$ & $\DW[\mREL]{}{}$ & $\DW[\mSC]{}{}$ & $\DF{\fREL}$&$\DF{\fACQ}$ &$\DF{\fSC}$\\
      \hline                                                                                                                                                     
      $\DR[\mRLX]{}{}$ & \cmark           & \cmark          & \cmark          & \cmark           & \xmark          & \xmark          & \xmark      &\xmark       & \xmark    \\
      $\DR[\mACQ]{}{}$  & \xmark           & \xmark          & \xmark          & \xmark           & \xmark          & \xmark          & \xmark      &\xmark       & \xmark    \\
      $\DR[\mSC]{}{}$  & \xmark           & \xmark          & \xmark          & \xmark           & \xmark          & \xmark          & \xmark      &\xmark       & \xmark    \\
      \hline                                                                                                                                                     
      $\DW[\mRLX]{}{}$ & \cmark           & \cmark          & \cmark          & \cmark           & \xmark          & \xmark          & \xmark      &\cmark       & \xmark    \\
      $\DW[\mREL]{}{}$  & \cmark           & \cmark          & \cmark          & \cmark           & \xmark          & \xmark          & \xmark      &\cmark       & \xmark    \\
      $\DW[\mSC]{}{}$  & \cmark           & \cmark          & \xmark          & \cmark           & \xmark          & \xmark          & \xmark      &\cmark       & \xmark    \\
      \hline                                                                                                                                                     
      $\DF{\fREL}$     & \cmark           & \cmark          & \cmark          & \xmark           & \xmark          & \xmark          & \xmark      &\cmark       & \xmark    \\
      $\DF{\fACQ}$     & \xmark           & \xmark          & \xmark          & \xmark           & \xmark          & \xmark          & \xmark      &\xmark       & \xmark    \\
      $\DF{\fSC}$     & \xmark           & \xmark          & \xmark          & \xmark           & \xmark          & \xmark          & \xmark      &\xmark       & \xmark 
    \end{tabular}
  \end{center}
  \begin{definition}
    \label{def:reorderra}
    Including coherence, \emph{reorderability} is defined:
    \begin{align*}
      {\reorderco}
      &=
      \{(\DW{\aLoc}{}, \DR{\bLoc}{}) \mid \aLoc\neq\bLoc\}
      \cup\{(\DW{\aLoc}{}, \DW{\bLoc}{}) \mid \aLoc\neq\bLoc\}
      \\&
      \cup\{(\DR{\aLoc}{}, \DW{\bLoc}{}) \mid \aLoc\neq\bLoc\}
      \cup\{(\DR{\aLoc}{}, \DR{\bLoc}{}) \}
      \\
      {\reorderra}
      &=
      \{(\DW[\amode]{}{}, \DR[\bmode]{}{}) \mid \amode\neq\mSC \lor \bmode\neq\mSC\}
      \cup\{(\DW[\amode]{}{}, \DW[\mRLX]{}{} \} %\mid \amode\neq\fREL \}
      \\&
      \cup\{(\DR[\amode]{}{}, \DW[\bmode]{}{}) \mid \amode=\mRLX \land \bmode=\mRLX\}
      \cup\{(\DR[\mRLX]{}{},  \DR[\bmode]{}{}) \} %\mid \bmode\neq\fACQ\}
      \\&
      \cup\{(\DF{\fREL},      \DF{\fACQ}    ) \}
      \cup\{(\DF{\fREL},      \DR[\bmode]{}{}) \}%\mid \amode=\fREL \}
      \cup\{(\DW[\amode]{}{}, \DF{\fACQ}     ) \}%\mid \bmode=\fACQ \}
      \\
      {\reorderlt}
      &=
      {\reorderra}\cap{\reorderco} %\{(\aAct, \bAct) \mid \aAct\reorderco\bAct \land \aAct\reorderra\bAct\}
    \end{align*}  
  \end{definition}
  Here is the version with generalized modes for read and write:  
  \begin{scope}
    \small
    \begin{align*}
      {\reorderra}
      &=
      \{(\DW[\amode]{}{}, \DR[\bmode]{}{}) \mid \amode\not\gemode\mSC \lor \bmode\not\gemode\mSC\}
      \cup\{(\DW[\amode]{}{}, \DW[\mRLX]{}{} \mid \amode\not\gemode\fREL \}
      \\&
      \cup\{(\DR[\amode]{}{}, \DW[\bmode]{}{}) \mid \amode=\mRLX \land \bmode=\mRLX\}
      \cup\{(\DR[\mRLX]{}{},  \DR[\bmode]{}{}) \mid \bmode\not\gemode\fACQ \}
      \\&
      \cup\{(\DF{\fREL},      \DF{\fACQ}    ) \}
      \cup\{(\DF{\fREL},      \DR[\bmode]{}{}) \}%\mid \amode=\fREL \}
      \cup\{(\DW[\amode]{}{}, \DF{\fACQ}     ) \}%\mid \bmode=\fACQ \}
    \end{align*}  
  \end{scope}
  \begin{center}
    \footnotesize
    \showRAtrue
    \setlength{\tabcolsep}{4pt}
    \begin{tabular}{c|ccccc|ccccc}
      &  \multicolumn{9}{|c}{$2^{\text{nd}}$} \\
      \hline
      $1^{\text{st}}$
      & $\DR[\mRLX]{}{}$  & $\DR[\mACQ]{}{}$ & $\DR[\fACQ]{}{}$& $\DR[\mSC]{}{}$  & $\DR[\fSC]{}{}$ & $\DW[\mRLX]{}{}$& $\DW[\mREL]{}{}$ & $\DW[\fREL]{}{}$& $\DW[\mSC]{}{}$& $\DW[\fSC]{}{}$\\% & $\DF{\fREL}$&$\DF{\fACQ}$ &$\DF{\fSC}$\\
      \hline                                                                                                                                                                                                                            
      $\DR[\mRLX]{}{}$ & \cmark            & \cmark          & \xmark          & \cmark           & \xmark           & \cmark          & \xmark          & \xmark          & \xmark         & \xmark         \\%  & \xmark      &\xmark       & \xmark    \\
      $\DR[\mACQ]{}{}$  & \xmark            & \xmark          & \xmark          & \xmark           & \xmark           & \xmark          & \xmark          & \xmark          & \xmark         & \xmark         \\%  & \xmark      &\xmark       & \xmark    \\
      $\DR[\fACQ]{}{}$ & \xmark            & \xmark          & \xmark          & \xmark           & \xmark           & \xmark          & \xmark          & \xmark          & \xmark         & \xmark         \\%  & \xmark      &\xmark       & \xmark    \\
      $\DR[\mSC]{}{}$  & \xmark            & \xmark          & \xmark          & \xmark           & \xmark           & \xmark          & \xmark          & \xmark          & \xmark         & \xmark         \\%  & \xmark      &\xmark       & \xmark    \\
      $\DR[\fSC]{}{}$ & \xmark            & \xmark          & \xmark          & \xmark           & \xmark           & \xmark          & \xmark          & \xmark          & \xmark         & \xmark         \\%  & \xmark      &\xmark       & \xmark    \\
      \hline                                                                                                                                                                                                                           
      $\DW[\mRLX]{}{}$ & \cmark            & \cmark          & \cmark          & \cmark           & \xmark           & \cmark          & \xmark          & \xmark          & \xmark         & \xmark         \\%  & \xmark      &\cmark       & \xmark    \\
      $\DW[\mREL]{}{}$  & \cmark            & \cmark          & \cmark          & \cmark           & \xmark           & \cmark          & \xmark          & \xmark          & \xmark         & \xmark         \\%  & \xmark      &\cmark       & \xmark    \\
      $\DW[\fREL]{}{}$ & \cmark            & \cmark          & \cmark          & \cmark           & \xmark           & \xmark          & \xmark          & \xmark          & \xmark         & \xmark         \\%  & \xmark      &\cmark       & \xmark    \\
      $\DW[\mSC]{}{}$  & \cmark            & \cmark          & \cmark          & \xmark           & \xmark           & \cmark          & \xmark          & \xmark          & \xmark         & \xmark         \\%  & \xmark      &\cmark       & \xmark    \\
      $\DW[\fSC]{}{}$ & \xmark            & \xmark          & \xmark          & \xmark           & \xmark           & \xmark          & \xmark          & \xmark          & \xmark         & \xmark         \\%  & \xmark      &\cmark       & \xmark    \\
      % \hline                                                                                                                                                                                       
      % $\DF{\fREL}$     & \cmark            & \cmark          & \cmark          & \cmark           & \xmark          & \xmark          & \xmark          & \xmark          & \xmark      &\cmark       & \xmark    \\
      % $\DF{\fACQ}$     & \xmark            & \xmark          & \xmark          & \xmark           & \xmark          & \xmark          & \xmark          & \xmark          & \xmark      &\xmark       & \xmark    \\
      % $\DF{\fSC}$     & \xmark            & \xmark          & \xmark          & \xmark           & \xmark          & \xmark          & \xmark          & \xmark          & \xmark      &\xmark       & \xmark 
    \end{tabular}
  \end{center}
  \begin{comment}
    \label{def:QSx}
    Let formulae $\QS{}{\amode}$ and $\QL{}{\amode}$ be defined:
    \begin{align*}
      \QS{}{\amode}&=
      \begin{cases}
        \Q{} & \textif \amode\neq\mRLX
        \\
        \TRUE & \textotherwise
      \end{cases}
      &
      \QL{}{\amode}&=
      \begin{cases}
        \Q{} & \textif \amode=\mSC
        \\
        \TRUE & \textotherwise
      \end{cases}
    \end{align*}
  \end{comment}
  
  \begin{definition}
    \label{def:independency-co}
    \noindent
    If $\aPS \in \sSEMI{\aPSS_1}{\aPSS_2}$ then $(\exists\aPS_1\in\aPSS_1)$
    $(\exists\aPS_2\in\aPSS_2)$
    % there are $\aPS_1\in\aPSS_1$ and $\aPS_2\in\aPSS_2$ such that
    % let
    % $\labelingForm'_2(\aEv)=\aTr[1]{{\downclose[0]{\aEv}}}{\labelingForm_2(\aEv)}$
    % let
    % $\labelingForm'_2(\aEv)=\aTr[1]{\{ \bEv \mid \bEv < \aEv
    % \}}{\labelingForm_2(\aEv})$
    \begin{enumerate}
      % \setcounter{enumi}{\value{pomsetXSemiCount}}
    \item[1--\ref{seq-tau})] as for $\sSEMI{}{}$ in \refdef{def:pomsets-trans},
    \item
      \label{seq-reorder} if $\bEv\in\aEvs_1$ and $\aEv\in\aEvs_2$ either
      $\bEv\le\aEv$ or $a\reorderlt\labeling_2(\aEv)$.
    \end{enumerate}
    Update the read and write rules of \refdef{def:pomsets-trans} to: %and \ref{def:pomsets-fj} to: % (\ref{S4}/\ref{L4} unchanged):
    \begin{enumerate}
    \item[\ref{S2})]
      $\labelingAct(\aEv) = \DW[\amode]{\aLoc}{\aVal}$,
    \item[\ref{S3})] 
      $\labelingForm(\aEv) \rimplies \aExp{=}\aVal$,
    \item[\ref{S4})]
      $\aTr{\bEvs}{\bForm} \rimplies \bForm \land\aExp{=}\aVal$,
    \item[\ref{S5})]
      $\aTr{\cEvs}{\bForm} \rimplies \bForm[\FALSE/\Q{}]$,
    \item[\ref{L2})]
      $\labelingAct(\aEv) = \DR[\amode]{\aLoc}{\aVal}$,
    \item[\ref{L3})] 
      $\labelingForm(\aEv) \rimplies \TRUE$.
    \item[\ref{L4})]
      $\aTr{\bEvs}{\bForm} \rimplies \aVal{=}\aReg\limplies\bForm$, 
    \item[\ref{L5})]
      $\aTr{\cEvs}{\bForm} \rimplies \bForm[\FALSE/\Q{}]$.
    \end{enumerate}
  \end{definition}

  \begin{comment}
    The logic of quiescence is greatly simplified.  Compare the following to
    \refex{ex:q1}.
    \begin{align*}
      \begin{gathered}
        \begin{gathered}[t]
          \PW[\mRLX]{x}{\aExp}
          \\
          \hbox{\begin{tikzinline}[node distance=.5em and 1.5em]
              \event{a}{\aExp{=}v\mid\DW[\mRLX]{x}{v}}{}
              \xform{xi}{\bForm[\FALSE/\Q{}]}{above=of a}
              \xform{xd}{\bForm\land\aExp{=}\aVal}{below=of a}
              \xo[xright]{a}{xd}
            \end{tikzinline}}
        \end{gathered}
        \\
        \begin{gathered}[t]
          \PW[\amode\in\{\mREL, \mSC \}]{x}{\aExp}
          \\
          \hbox{\begin{tikzinline}[node distance=.5em and 1.5em]
              \event{a}{\Q{} \land \aExp{=}v\mid\DW[\amode]{x}{v}}{}
              \xform{xi}{\bForm[\FALSE/\Q{}]}{above=of a}
              \xform{xd}{\bForm\land\aExp{=}\aVal}{below=of a}
              \xo[xright]{a}{xd}
            \end{tikzinline}}
        \end{gathered}
      \end{gathered}
      &&
      \begin{gathered}
        \begin{gathered}[t]
          \PR[\amode\in\{\mRLX, \mACQ \}]{x}{r}
          \\
          \hbox{\begin{tikzinline}[node distance=.5em and 1.5em]
              \event{a}{\DR[\amode]{x}{v}}{}
              \xform{xi}{\bForm[\FALSE/\Q{}]}{above=of a}
              \xform{xd}{v{=}r\limplies\bForm}{below=of a}
              \xo[xright]{a}{xd}
            \end{tikzinline}}
        \end{gathered}
        \\
        \begin{gathered}[t]
          \PR[\mSC]{x}{r}
          \\
          \hbox{\begin{tikzinline}[node distance=.5em and 1.5em]
              \event{a}{\Q{} \mid \DR[\mSC]{x}{v}}{}
              \xform{xi}{\bForm[\FALSE/\Q{}]}{above=of a}
              \xform{xd}{v{=}r\limplies\bForm}{below=of a}
              \xo[xright]{a}{xd}
            \end{tikzinline}}
        \end{gathered}
      \end{gathered}
    \end{align*}
  \end{comment}
  \begin{example}
    The logic of quiescence is greatly simplified.  Compare the following to
    \refex{ex:q1}.
    \begin{align*}
      \begin{gathered}[t]
        \PW[\amode]{x}{\aExp}
        \\
        \hbox{\begin{tikzinline}[node distance=.5em and 1.5em]
            \event{a}{\Q{} \land \aExp{=}v\mid\DW[\amode]{x}{v}}{}
            \xform{xi}{\bForm[\FALSE/\Q{}]}{above=of a}
            \xform{xd}{\bForm\land\aExp{=}\aVal}{below=of a}
            \xo[xright]{a}{xd}
          \end{tikzinline}}
      \end{gathered}
      &&
      \begin{gathered}[t]
        \PR[\amode]{x}{r}
        \\
        \hbox{\begin{tikzinline}[node distance=.5em and 1.5em]
            \event{a}{\DR[\amode]{x}{v}}{}
            \xform{xi}{\bForm[\FALSE/\Q{}]}{above=of a}
            \xform{xd}{v{=}r\limplies\bForm}{below=of a}
            \xo[xright]{a}{xd}
          \end{tikzinline}}
      \end{gathered}
    \end{align*}
  \end{example}

  \reffig{fig:no-q-or-addr} shows the resulting full semantics of read and
  write, without address calculation.  \reffig{fig:no-q} shows the resulting
  full semantics with address calculation.

  \begin{definition}
    \label{def:dlx}
    % Let $[\aForm/\Dx{*}]$ substitute $\aForm$ for every $\Dx{\bLoc}$.

    \noindent
    Let substitution $\DS{\aLoc}{\amode}$ be defined:
    \begin{displaymath}
      \DS{\aLoc}{\amode}=
      \begin{cases}
        [\TRUE/\Dx{\aLoc}] & \textif \amode=\mRLX
        \\
        [\FALSE/\Dx{*}] & \textotherwise
      \end{cases}
    \end{displaymath}
    Let formula $\DLX{\aLoc}{\amode}{\bmode}$ be defined:
    \begin{displaymath}
      \DLX{\aLoc}{\amode}{\bmode}=
      \begin{cases}
        \TRUE & \textif \amode=\mRLX \textor \amode=\bmode
        \\
        \Dx{\aLoc} & \textotherwise
      \end{cases}
    \end{displaymath}
  \end{definition}
\end{scope}

\begin{figure*}
  \showRAtrue
  \begin{center}
    \begin{minipage}{.91\textwidth}
      \renewcommand{\cEvs}{D}
\renewcommand{\dEvs}{D}
\noindent
If $\aPS \in \sSTORE[\amode]{\cExp}{\aExp}$ then
$(\exists\cVal:\aEvs\fun\Val)$
$(\exists\aVal:\aEvs\fun\Val)$
$(\exists\cForm:\aEvs\fun\Formulae)$
\begin{enumerate}
\item[{\labeltext[S1]{S1)}{S1no-q}}] % [\ref{S1})]
  if $\cForm_\bEv\land\cForm_\aEv$ is satisfiable then $\bEv=\aEv$,
\item[{\labeltext[S2]{S2)}{S2no-q}}] %[\ref{S2})]
  $\labelingAct(\aEv) = \DWREF[\amode]{\cVal_\aEv}{\aVal_\aEv}$,
\item[{\labeltext[S3]{S3)}{S3no-q}}] %[\ref{S3})] 
  $\labelingForm(\aEv)$ implies
  \begin{math}
    \cForm_\aEv
    % \land \RW
    \land \cExp{=}\cVal_\aEv
    \land \aExp{=}\aVal_\aEv
  \end{math},
\item[{\labeltext[S4]{S4)}{S4no-q}}] %[\ref{S4})]
  \begin{math}
    (\forall\aEv\in\aEvs\cap\bEvs)
  \end{math}
  $\aTr{\bEvs}{\bForm}$ implies 
  \begin{math}
    \cForm_\aEv
    %\limplies (\cExp{=}\cVal_\aEv)
    \limplies {
      \bForm
      [\aExp/\REF{\cVal_\aEv}]
      \DS{\REF{\cVal_\aEv}}{\amode}
      [(\Q{}\land\aExp{=}\aVal_\aEv\land\cExp{=}\cVal_\aEv)/\Q{}]
    }
  \end{math},
\item[{\labeltext[S5]{S5)}{S5no-q}}] %[\ref{S5})] 
  \begin{math}    
    (\forall\aEv\in\aEvs\setminus\cEvs)
  \end{math}
  $\aTr{\cEvs}{\bForm}$ implies
  \begin{math}
    \cForm_\aEv
    %\limplies (\cExp{=}\cVal_\aEv)
    \limplies {
      \bForm
      [\aExp/\REF{\cVal_\aEv}]
      \DS{\REF{\cVal_\aEv}}{\amode}
      [\FALSE/\Q{}]
    },
  \end{math}
\item[{\labeltext[S6]{S6)}{S6no-q}}] %[S6)]%\ref{S6})] 
  \begin{math}
    (\forall\dVal)
  \end{math}
  $\aTr{\dEvs}{\bForm}$ implies
  \begin{math}
    (\!\not\exists\aEv\in\aEvs \suchthat \cForm_\aEv)
    %(\!\not\exists\aEv\in\aEvs\cap\cEvs \suchthat \cForm_\aEv)
    \limplies (\cExp{=}\dVal)
    \limplies {
      \bForm
      [\aExp/\REF{\dVal}]
      \DS{\REF{\dVal}}{\amode}
      [\FALSE/\Q{}]
    }.
  \end{math}
% \item[S5-6)]%\ref{S6})] 
%   \begin{math}
%     (\forall\dVal)
%   \end{math}
%   $\aTr{\cEvs}{\bForm}$ implies
%   \begin{math}
%     %(\!\not\exists\aEv\in\aEvs \suchthat \cForm_\aEv)
%     (\!\not\exists\aEv\in\aEvs\cap\cEvs \suchthat \cForm_\aEv)
%     \limplies (\cExp{=}\dVal)
%     \limplies \PBR{
%       \bForm
%       [\aExp/\REF{\dVal}]
%       \DS{\REF{\dVal}}{\amode}
%       [\FALSE/\QS{\REF{\dVal}}{\amode}]
%     }.
%   \end{math}
  % \\ where 
  % $\DS{}{\mRLX}{}=[\TRUE/\DxREF{\dVal}]$ and otherwise
  % $\DS{}{\amode}{}=[\FALSE/\D]$. % for $\amode\neq\mRLX$.
\end{enumerate}
% \item if $\amode=\mRLX$ then
%   $\labelingForm(\aEv)$ implies
%   \begin{math}
%     \cForm_\aEv
%     \land \cExp{=}\cVal_\aEv
%     \land \aExp{=}\aVal_\aEv
%     \land \RW
%     \land \QxREF{\cVal_\aEv},
%   \end{math}
% \item if $\amode\neq\mRLX$ then
%   $\labelingForm(\aEv)$ implies
%   \begin{math}
%     \cForm_\aEv
%     \land \cExp{=}\cVal_\aEv
%     \land \aExp{=}\aVal_\aEv
%     \land \RW
%     \land \Q{},
%   \end{math}
% \item if
%   $\aEv\in\bEvs$
%   and
%   $\amode=\mRLX$ then
%   \begin{math}
%     (\forall\dVal)
%   \end{math}
%   $\aTr{\bEvs}{\bForm}$ implies 
%   \begin{math}
%     \cForm_\aEv
%     \limplies (\cExp{=}\dVal)
%     \limplies \PBRbig{
%     (\QwREF{\dVal} \limplies \aExp{=}\aVal_\aEv)
%     \land \bForm[\aExp/\REF{\dVal}][\TRUE/\DxREF{\dVal}]
%   }
%   \end{math}
% \item if
%   $\aEv\in\bEvs$
%   and
%   $\amode\neq\mRLX$ then
%   \begin{math}
%     (\forall\dVal)
%   \end{math}
%   $\aTr{\bEvs}{\bForm}$ implies 
%   \begin{math}
%     \cForm_\aEv
%     \limplies (\cExp{=}\dVal)
%     \limplies \PBRbig{
%     (\QwREF{\dVal} \limplies \aExp{=}\aVal_\aEv)
%     \land \bForm[\aExp/\REF{\dVal}][\FALSE/\D]
%   }
%   \end{math}
% \item if 
%   \begin{math}
%     (\forall\aEv\in\bEvs)(\cForm \textimplies
%     \lnot\cForm_\aEv)
%   \end{math}
%   and $\amode=\mRLX$ 
%   then
%   \begin{math}
%     (\forall\dVal)
%   \end{math}
%   $\aTr{\bEvs}{\bForm}$ implies 
%   \begin{math}
%     \cForm
%     \limplies (\cExp{=}\dVal)
%     \limplies \PBRbig{
%     \lnot\QwREF{\dVal}
%     \land \bForm[\aExp/\REF{\dVal}][\TRUE/\DxREF{\dVal}]
%   }
%   \end{math}
% \item if 
%   \begin{math}
%     (\forall\aEv\in\bEvs)
%     (\cForm \textimplies \lnot\cForm_\aEv)
%   \end{math}
%   and $\amode\neq\mRLX$ 
%   then
%   \begin{math}
%     (\forall\dVal)
%   \end{math}
%   $\aTr{\bEvs}{\bForm}$ implies 
%   \begin{math}
%     \cForm
%     \limplies (\cExp{=}\dVal)
%     \limplies \PBRbig{
%     \lnot\QwREF{\dVal}
%     \land \bForm[\aExp/\REF{\dVal}][\FALSE/\D]
%   }
%   \end{math}

\noindent
If $\aPS \in \sLOAD[\amode]{\aReg}{\cExp}$ then
$(\exists\cVal:\aEvs\fun\Val)$
$(\exists\aVal:\aEvs\fun\Val)$
$(\exists\cForm:\aEvs\fun\Formulae)$
$(\exists\bmode\in\{\amode,\mRLX\})$
% $(\forall\uReg{\aEv}\in\uRegs{\aEvs})$
\begin{enumerate}
\item[{\labeltext[L1]{L1)}{L1no-q}}] %[\ref{L1})]
  if $\cForm_\bEv\land\cForm_\aEv$ is satisfiable then $\bEv=\aEv$,
\item[{\labeltext[L2]{L2)}{L2no-q}}] %[\ref{L2})]
  $\labelingAct(\aEv) = \DRREF[\bmode]{\cVal_\aEv}{\aVal_\aEv}$,
\item[{\labeltext[L3]{L3)}{L3no-q}}] %[\ref{L3})]
  $\labelingForm(\aEv)$ implies
  \begin{math}
    \cForm_\aEv
    % \land \RO
    \land \cExp{=}\cVal_\aEv
  \end{math},
  % where    
  % $\QL{}{\mSC}=\Q{\mSC}$ and otherwise $\QL{}{\amode}=\QwREF{\cVal_\aEv}$, % for $\amode\neq\mRLX$,
\item[{\labeltext[L4]{L4)}{L4no-q}}] %[\ref{L4})]
  \begin{math}
    (\forall\aEv\in\aEvs\cap\bEvs)
  \end{math}
  $\aTr{\bEvs}{\bForm}$ implies
  \begin{math}
    \cForm_\aEv
    \limplies (\cExp{=}\cVal_\aEv\limplies\aVal_\aEv{=}\uReg{\aEv})
    \limplies \bForm[\uReg{\aEv}/\aReg]
  \end{math},
  %\makebox[6.2cm]{}
\item[{\labeltext[L5]{L5)}{L5no-q}}] %[\ref{L5})] 
  \begin{math}
    (\forall\aEv\in\aEvs\setminus\cEvs)
  \end{math}
  $\aTr{\cEvs}{\bForm}$ implies
  \begin{math}
    \cForm_\aEv 
    \limplies
    \DLX{\REF{\cVal_\aEv}}{\amode}{\bmode}
    \land
    \PBRbig{
      \ABRbig{
        \PBR{\cExp{=}\cVal_\aEv\limplies\aVal_\aEv{=}\uReg{\aEv}}
        \lor
        \PBR{
          \RW\land
          \PBR{\cExp{=}\cVal_\aEv\limplies\REF{\cVal_\aEv}{=}\uReg{\aEv}}
        }
      }
      \limplies
      \bForm
      [\uReg{\aEv}/\aReg]
      [\FALSE/\Q{}]
    }    
  \end{math},
\item[{\labeltext[L6]{L6)}{L6no-q}}] %[\ref{L6})] 
  \begin{math}
    (\forall\dVal)
    (\forall\bReg)
  \end{math}
  $\aTr{\dEvs}{\bForm}$  implies 
  \begin{math}
    (\!\not\exists\aEv\in\aEvs \suchthat \cForm_\aEv)
    \limplies (\cExp{=}\dVal)
    \limplies \PBR{        
      \DLX{\REF{\dVal}}{\amode}{\bmode} \land
      \bForm
      [\bReg/\aReg]
      [\FALSE/\Q{}]
    }.
  \end{math}
  % \\ where $\DL{}{\mRLX}=\TRUE$ and otherwise $\DL{}{\amode}=\DxREF{\dVal}$.
  % Recall that $\uRegs{\bEvs}=\{\uReg{\aEv}\mid\aEv\in\bEvs\}$.
\end{enumerate}  
% \item if $\amode=\mRLX$ and $\bEv\notin\bEvs$ then
%   \begin{math}
%     (\forall\dVal)
%   \end{math}
%   $\aTr{\bEvs}{\bForm}$ implies
%   \begin{math}
%     \cForm_\bEv
%     \limplies (\cExp{=}\dVal)
%     \limplies \PBRbig{
%     (
%     \RW
%     \limplies (\aVal{=}\uReg{\bEv}\lor\aLoc{=}\uReg{\bEv}) 
%     \limplies \bForm[\uReg{\bEv}/\aReg][\uReg{\bEv}/\REF{\dVal}]
%     )
%     \land \lnot\QxREF{\dVal}
%   }
%     \phantom{\land\; \Dx{\dVal}}
%   \end{math}
% \item if $\amode\neq\mRLX$ and $\bEv\notin\bEvs$ then
%   \begin{math}
%     (\forall\dVal)
%   \end{math}
%   $\aTr{\bEvs}{\bForm}$ implies
%   \begin{math}
%     \cForm_\bEv
%     \limplies (\cExp{=}\dVal)
%     \limplies \PBRbig{
%     (
%     \RW
%     \limplies (\aVal{=}\uReg{\bEv}\lor\aLoc{=}\uReg{\bEv}) 
%     \limplies \bForm[\uReg{\bEv}/\aReg][\uReg{\bEv}/\REF{\dVal}]
%     )
%     \land \lnot\QxREF{\dVal}
%     \land \Dx{\dVal}
%   }
%   \end{math}

% \noindent
% If $\aPS \in \sTHREAD{\aPSS}$ then
% $(\exists\aPS_1\in\aPSS)$
% \begin{enumerate}
% \item[{\labeltext[T1]{T1)}{T1no-q}}] %[\ref{T1})]
%   $\aEvs=\aEvs_1$,
% \item[{\labeltext[T2]{T2)}{T2no-q}}] %[\ref{T2})]
%   $\labelingAct(\aEv) = \labelingAct_1(\aEv)$,
% \item[{\labeltext[T3]{T3)}{T3no-q}}] %[\ref{T3})]
%   $\labelingForm(\aEv)$ implies
%   $\labelingForm_1(\aEv) [\TRUE/\Q{}][\TRUE/\RW]$ if $\labelingAct_1(\aEv)$ is a write,
%   \\
%   $\labelingForm(\aEv)$ implies
%   $\labelingForm_1(\aEv) [\TRUE/\Q{}][\FALSE/\RW]$ otherwise.
% \end{enumerate}  

    \end{minipage}
  \end{center}
  \caption{Simplified Quiescence Semantics with Address Calculation
    (See %\refdef{def:QSx} for $\QS{}{\amode}$, $\QL{}{\amode}$, and
    \refdef{def:dlx} for $\DLX{\aLoc}{\amode}{\bmode}$, $\DS{\aLoc}{\amode}$)
  } 
  \label{fig:no-q}
\end{figure*}    


\section{Additional Features}

\subsection{Read-Modify-Write Operations (\xRMW)}

Extend the syntax
\begin{align*}
  \aCmd
  \BNFDEF& \cdots 
  \BNFSEP \PCAS[\amode_1][\amode_2]{\REF{\cExp}}[\ascope]{\aReg}{\aExp}{\bExp}
  \BNFSEP \PFADD[\amode_1][\amode_2]{\REF{\cExp}}[\ascope]{\aReg}{\aExp}
  \BNFSEP \PEXCHG[\amode_1][\amode_2]{\REF{\cExp}}[\ascope]{\aReg}{\aExp}
\end{align*}
From the data model, we require an additional binary relation over
$\Act\times\Act$: $\roverlapsdef$.  For the actions in this paper, we say
$\aAct \roverlapsdef \bAct$ if they access the same location.


\RMW{} operations are formalized by adding a relation
${\xrmw}\subseteq\aEvs\times\aEvs$ that relates the read of a successful
\RMW{} to the succeeding write.
% Let two actions \emph{overlap} if they access the same location.
Extend the definition of a pomset as follows. % where two actions \emph{overlap} if they access the same location:
% \begin{enumerate}
% \item
%   ${\rrmw} \subseteq {\le}$ is a relation capturing
%   \emph{read-modify-write atomicity}, such that for any $\cEv$, $\bEv$, $\aEv\in\aEvs$,
%   where $\labeling(\cEv)$ and
%   $\labeling(\bEv)$ access the same location:
%   \begin{itemize}
%   \item if $\bEv \xrmw \aEv$ and $\cEv\le \aEv$ then $\cEv\le \bEv$,
%   \item if $\bEv \xrmw \aEv$ and $\bEv\le \cEv$ then $\aEv\le \cEv$.
%   \end{itemize}
% \end{enumerate}

\begin{enumerate}[,label=(\textsc{m}\arabic*),ref=\textsc{m}\arabic*]
  \setcounter{enumi}{-1}
\item \label{pom-rmw}
  ${\rrmw} : \aEvs\fun\aEvs$ is a partial function capturing
  read-modify-write \emph{atomicity}, such that
  \begin{enumerate}
  \item \label{pom-rmw-block}
    if $\bEv\xrmw\aEv$ then $\labeling(\aEv) \rblocks \labeling(\bEv)$,
  \item \label{pom-rmw-le}
    if $\bEv\xrmw\aEv$ then $\bEv \le \aEv$,    
  \item \label{pom-rmw-atomic}
    if $\labeling(\cEv) \roverlaps \labeling(\bEv)$ then
    \begin{enumerate}        
    \item \label{pom-rmw-atomic1}
      if $\bEv \xrmw \aEv$ then
      $\cEv\ledep \aEv$ implies $\cEv\le \bEv$,
    \item \label{pom-rmw-atomic2}
      if $\bEv \xrmw \aEv$ then
      $\bEv\ledep \cEv$ implies $\aEv\le \cEv$.
    \end{enumerate}
  \end{enumerate}
\end{enumerate}

Extend the definition of par, if, seq to include:
\begin{enumerate}
  \setcounter{enumi}{-1}
\item ${\rrmw}=\PBR{{\rrmw_1}\cup{\rrmw_2}}$,
\end{enumerate}


\begin{example}
  \label{ex:rmw-dep}
  For \RMW{} operations, the independent case for a read should be the same as
  the empty case.  To see why, consider the semantics of local invariant
  reasoning (\xLIR) from \refdef{def:pomsets-lir}:
  \begin{enumerate}
  \item[\ref{S4})]
    $\aTr{\bEvs}{\bForm} \rimplies \bForm[\aExp/\aLoc]\land\aExp{=}\aVal$,
  \item[\ref{S5})]
    $\aTr{\cEvs}{\bForm} \rimplies \bForm[\aExp/\aLoc]$,
  \item[\ref{L4})]
    $\aTr{\bEvs}{\bForm} \rimplies \aVal{=}\aReg\limplies\bForm$, 
  \item[\ref{L5})]
    $\aTr{\cEvs}{\bForm} \rimplies (\aVal{=}\aReg\lor\aLoc{=}\aReg)\limplies\bForm$, when $\aEvs\neq\emptyset$,
  \item[\ref{L6})] 
    $\aTr{\dEvs}{\bForm} \rimplies \bForm$, when $\aEvs=\emptyset$.
  \end{enumerate}
  Consider the relaxed variant of the \textsc{cdrf} example from
  \cite{DBLP:conf/pldi/LeeCPCHLV20}, using a semantics for $\FADD$ that
  simply composes the rules for load and store above.
  \begin{gather*}
    \begin{gathered}
      \PW{x}{0}\SEMI
      \begin{aligned}[t]
        (&\PFADD[\mRLX][\mRLX]{x}{r}{1}\SEMI \IF{\BANG r}\THEN \IF{y}\THEN \PW{x}{0} \FI \FI \;\;\PAR
        \\&
        \PFADD[\mRLX][\mRLX]{x}{r}{1}\SEMI \IF{\BANG r}\THEN \PW{y}{1} \FI)
      \end{aligned}
      \\
      \hbox{\begin{tikzinlinesmall}[node distance=1em]
          \event{i}{\DW{x}{0}}{}
          \event{b0}{\DR{x}{0}}{right=2em of i}
          \event{b0b}{\DW{x}{1}}{right=1.5em of b0}
          \event{b1}{\DR{y}{1}}{right=of b0b}
          \event{b2}{\DW{x}{0}}{right=of b1}
          \event{a1}{\DR{x}{0}}{right=2em of b2}
          \event{a1b}{\DW{x}{1}}{right=1.5em of a1}
          \event{a2}{\DW{y}{1}}{right=of a1b}
          \rmw{a1}{a1b}
          \rmw{b0}{b0b}
          \rf{i}{b0}
          \rf[out=-165,in=-12]{a2}{b1}
          \wki[out=20,in=160]{b0b}{b2}
          % \sync{a1}{a2}
          % \sync{b0}{b1}
          \po{b1}{b2}
          \rf{b2}{a1}
        \end{tikzinlinesmall}}
    \end{gathered}
  \end{gather*}
  Looking at the independent transformers of the second thread and
  initializer, we have:
  \begin{align*}
    \begin{gathered}[t]
      \PW{x}{0}
      \\
      \hbox{\begin{tikzinline}[node distance=.5em and 1.5em]
          \event{a}{\DW{x}{0}}{}      
          \xform{xi}{\bForm[0/x]}{below=of a}
        \end{tikzinline}}    
    \end{gathered}
    &&
    \begin{gathered}[t]
      \smash{\FADD^{\mRLX,\mRLX}(x,1)}
      \\
      \hbox{\begin{tikzinline}[node distance=1em and .75em]
          \event{a0}{\DR{x}{0}}{}
          \node(a)[right=of a0]{};
          \event{a1}{\DW{x}{1}}{right=of a}
          \rmw{a0}{a1}
          \xform{xi}{(0{=}r\lor x{=}r)\limplies\bForm[1/x]}{below=of a}
        \end{tikzinline}}    
    \end{gathered}
    &&
    \begin{gathered}[t]
      \IF{\BANG r}\THEN \PW{y}{1} \FI
      \\
      \hbox{\begin{tikzinline}[node distance=.5em and 1.5em]
          \event{a2}{r{=}0\mid\DW{y}{1}}{}      
          \xform{xi}{\bForm[1/y]}{below=of a2}
        \end{tikzinline}}    
    \end{gathered}
  \end{align*}
  After sequencing, the precondition of $\DWP{y}{1}$ is a tautology:
  $(0{=}r\lor 0{=}r)\limplies r{=}0$.

  Here, local invariant reasoning is using the initializing write to $x$ to
  justify the independence of the write to $y$.  But this write is made
  unavailable by the first thread's successful \RMW{}.
\end{example}
As a result, we disallow the use of \ref{L5} when treating the read event in
an \RMW{}.

[Todo: write out the rules.]

% \begin{example}
%   Let $\CAS$ return the value read, which is sufficient to determine whether
%   the $\CAS$ succeeded.
%   \begin{align*}
%     \begin{gathered}
%       \DW{x}{0}\SEMI(
%       \IF{\BANG \CAS(x,0,1)}\THEN \PW{y}{1} \FI
%       \PAR
%       \IF{\BANG \CAS(x,0,1)}\THEN \PW{z}{1} \FI
%       )
%       \\
%       \hbox{\begin{tikzinline}[node distance=1.5em]
%         \event{a1}{\DR{x}{0}}{}
%         \event{a2}{\DW{x}{1}}{right=of a1}
%         \event{a3}{\DW{y}{1}}{right=of a2}
%         \event{b1}{\DR{x}{1}}{right=4em of a3}
%         \event{b2}{\DW{z}{1}}{right=of b1}          
%         \event{i}{\DW{x}{0}}{left=4em of a1}          
%         \rmw{a1}{a2}
%         \rf{i}{a1}
%         \rf[out=-15,in=-165]{a2}{b1}
%       \end{tikzinline}}
%     \end{gathered}
%   \end{align*}
%   This clearly should not be allowed.
%   What's gone wrong here is that 
% \end{example}

\subsection{Fence Operations (\xFENCE)}
\label{sec:fence}
% Fence actions are pretty straightforward to handle in the semantics that uses
% independency for synchronization.  You just add dummy actions with the
% corresponding rules.  See \textsection\ref{sec:independency-ra}.

Syntactic fences $\FENCE^{\fmode}$ have corresponding actions:
$\DFP{\fmode}$.  The \emph{syntactic fence mode}
$(\fmode \!\!\BNFDEF\!\! \fREL \!\BNFSEP\! \fACQ \!\BNFSEP\! \fSC)$ is either
\emph{release}, \emph{acquire}, or \emph{sequentially-consistent}.



Formalizing this, $\QS{\aLoc}{\fREL}$ substitutes for $\Qw{*}$ in addition to
$\Qw{\aLoc}$, as in $\QS{\aLoc}{\mRA}$.  $\QL{\aLoc}{\fACQ}$ requires
$\Qr{*}$ in addition to $\Qw{\aLoc}$, as in $\QL{\aLoc}{\mRA}$.

\begin{definition}
  Extend \refdef{def:QS} and \refdef{def:DS}.
  \begin{align*}
    \QF{\aLoc}{\fREL}
    % =\QS{\aLoc}{\fREL}
    &=\Qr{*}\land\Qw{*} 
    &
    \QF{\aLoc}{\fACQ}
    % =\QL{\aLoc}{\fACQ}
    &=\Qw{\aLoc}\land \Qr{*}
    \\
    [\aForm/\QF{\aLoc}{\fREL}]
    % = [\aForm/\QS{\aLoc}{\fREL}]
    &= [\aForm/\Qw{*}]
    &{}
    [\aForm/\QF{\aLoc}{\fACQ}]
    % = [\aForm/\QL{\aLoc}{\fACQ}]
    &= [\aForm/\Qr{*},\aForm/\Qw{*}]
    % \\
    % \DS{\aLoc}{\fREL}&=[\FALSE/\D]
    % &\DL{\aLoc}{\fACQ}&=\Dx{\aLoc}%\land\D{}
  \end{align*}
  \begin{align*}
    \QF{\aLoc}{\fSC}
    % = \QS{\aLoc}{\fSC}
    % = \QL{\aLoc}{\fSC}
    &= \Qr{*}\land\Qw{*} \land\Qsc
    \\
    [\aForm/\QF{\aLoc}{\fSC}] 
    % = [\aForm/\QS{\aLoc}{\fSC}] 
    % = [\aForm/\QL{\aLoc}{\fACQ}]
    &= [\aForm/\Qr{*},\aForm/\Qw{*},\aForm/\Qsc]
  \end{align*}
\end{definition}

If $\aPS \in \sFENCE[\ascope]{\amode}$ then
$(\exists\aLocs\subseteq\Loc)$
% $(\exists\bmode\in\{\amode,\mRLX\})$
\begin{enumerate}[resume]
\item%[{\labeltext[F1]{(F1)}{F1}}]
  if $\bEv,\aEv\in\aEvs$ then $\bEv=\aEv$,
\item%[{\labeltext[F2]{(F2)}{F2}}]
  $\labelingAct(\aEv) = \DG[\ascope]{\amode}{\aLocs}$,
\item%[{\labeltext[F3]{(F3)}{F3}}]
  $\labelingForm(\aEv) \rimplies \bigwedge_{\aLoc\in (\Loc\setminus\aLocs)}\QF{\aLoc}{\amode}$,
\item%[{\labeltext[F4]{(F4)}{F4}}]
  $\aTr{\bEvs}{\bForm} \rimplies \bForm$, where $\bEvs\cap\aEvs\neq\emptyset$,
\item%[{\labeltext[F5]{(F5)}{F5}}]
  $\aTr{\cEvs}{\bForm} \rimplies \bigwedge_{\aLoc\in\aLocs}\Dx{\aLoc} \land\bForm[\FALSE/\QF{\aLoc}{}]$, where $\cEvs\cap\aEvs=\emptyset$.
\end{enumerate}

\begin{example}
  Extend \refex{ex:q1}.
  \begin{gather*}
    \begin{gathered}
      \begin{gathered}[t]
        \PW[\fREL]{x}{\aExp}
        \\
        \hbox{\begin{tikzinline}[node distance=.5em and 1.5em]
            \raevent{a}{\Qr{*}\land\Qw{*}\land\aExp{=}v\mid\DW[\mREL]{x}{v}}{}
            \xform{xi}{\bForm[\FALSE/\Qw{*}]}{above=of a}
            \xform{xd}{\bForm\land\aExp{=}\aVal}{below=of a}
            \xo[xright]{a}{xd}
          \end{tikzinline}}
      \end{gathered}
      \\
      \begin{gathered}[t]
        \PF{\fREL}
        \\
        \hbox{\begin{tikzinline}[node distance=.5em and 1.5em]
            \raevent{a}{\Qr{*}\land\Qw{*}\mid\DF{\fREL}}{}
            \xform{xi}{\bForm[\FALSE/\Qw{*}]}{above=of a}
            \xform{xd}{\bForm}{below=of a}
            \xo[xright]{a}{xd}
          \end{tikzinline}}
      \end{gathered}      
    \end{gathered}
    \qquad
    \begin{gathered}
      \begin{gathered}[t]
        \PR[\fACQ]{x}{r}
        \\
        \hbox{\begin{tikzinline}[node distance=.5em and 1.5em]
            \raevent{a}{\Qw{x}\land\Qr{*}\mid\DR[\fACQ]{x}{v}}{}
            \xform{xi}{\bForm[\FALSE/\Qr{*}][\FALSE/\Qw{*}]}{above=of a}
            \xform{xd}{v{=}r\limplies\bForm}{below=of a}
            \xo[xright]{a}{xd}
          \end{tikzinline}}
      \end{gathered}
      \\
      \begin{gathered}[t]
        \PF{\fACQ}
        \\
        \hbox{\begin{tikzinline}[node distance=.5em and 1.5em]
            \raevent{a}{\Qr{*}\mid\DF{\fACQ}}{}
            \xform{xi}{\bForm[\FALSE/\Qr{*}][\FALSE/\Qw{*}]}{above=of a}
            \xform{xd}{\bForm}{below=of a}
            \xo[xright]{a}{xd}
          \end{tikzinline}}
      \end{gathered}
    \end{gathered}
    \\
    \begin{gathered}[t]
      \PF{\fSC}
      \\
      \hbox{\begin{tikzinline}[node distance=.5em and 1.5em]
          \scevent{a}{\Qr{*}\land\Qw{*}\land\Qsc\mid\DF{\mSC}}{}
          \xform{xi}{\bForm[\FALSE/\Qr{*}][\FALSE/\Qw{*}][\FALSE/\Qsc]}{above=of a}
          \xform{xd}{\bForm}{below=of a}
          \xo[xright]{a}{xd}
        \end{tikzinline}}
    \end{gathered}
    \\
    \begin{gathered}[t]
      \PW[\fSC]{x}{\aExp}
      \\
      \hbox{\begin{tikzinline}[node distance=.5em and 1.5em]
          \scevent{a}{\Qr{*}\land\Qw{*}\land\Qsc\land\aExp{=}\aVal\mid\DW[\fSC]{x}{v}}{}
          \xform{xi}{\bForm[\FALSE/\Qr{*}][\FALSE/\Qw{*}][\FALSE/\Qsc]}{above=of a}
          \xform{xd}{\bForm\land\aExp{=}\aVal}{below=of a}
          \xo[xright]{a}{xd}
        \end{tikzinline}}
    \end{gathered}
    \;
    \begin{gathered}[t]
      \PR[\fSC]{x}{\aExp}
      \\
      \hbox{\begin{tikzinline}[node distance=.5em and 1.5em]
          \scevent{a}{\Qr{*}\land\Qw{*}\land\Qsc\mid\DR[\fSC]{x}{v}}{}
          \xform{xi}{\bForm[\FALSE/\Qr{*}][\FALSE/\Qw{*}][\FALSE/\Qsc]}{above=of a}
          \xform{xd}{v{=}r\limplies\bForm}{below=of a}
          \xo[xright]{a}{xd}
        \end{tikzinline}}
    \end{gathered}
  \end{gather*}
\end{example}



\subsection{Fence Actions with Downgrading Reads (\xFENCE/\xDGR)}
\label{sec:fence}

\begin{example}
  Revisiting \refex{ex:dgr2} using fences:
  % \begin{align*}
  %   \begin{gathered}[t]
  %     \PW{x}{2}
  %     \\
  %     \hbox{\begin{tikzinline}[node distance=0.5em and 1.5em]
  %       \event{a}{\DW{x}{2}}{}
  %       \xform{x1d}{\bForm[\TRUE/\Dx{\aLoc}]}{below=of a}
  %         %       \xform{x1i}{\lnot\Qw{x}\land\bForm[\TRUE/\Dx{x}]}{below=of x1d}
  %       \xo[xright]{a}{x1d}
  %     \end{tikzinline}}  
  %   \end{gathered}  
  %   &&
  %   \begin{gathered}[t]
  %     \PR[\mACQ]{x}{r}
  %     \\
  %     \hbox{\begin{tikzinline}[node distance=0.5em and 1.5em]
  %       \raevent{b}{\DR[\mACQ]{x}{2}}{}
  %         %       \xform{x1d}{2{=}r\limplies\bForm}{below=of b}
  %       \xform{x1i}{\Dx{x} \land\bForm[\FALSE/\Qr{x}]}{below=of b} %x1d}
  %         %       \xo[xright]{b}{x1d}
  %     \end{tikzinline}}  
  %   \end{gathered}  
  %   &&
  %   \begin{gathered}[t]
  %     \PW{y}{1}
  %     \\
  %     \hbox{\begin{tikzinline}[node distance=0.5em and 1.5em]
  %       \event{c}{\DW{y}{1}}{}
  %         %       \xform{x1d}{\bForm[\TRUE/\Dx{\aLoc}]}{below=of c}
  %         %       \xform{x1i}{\lnot\Qw{y}\land\bForm[\TRUE/\Dx{y}]}{below=of x1d}
  %         %       \xo[xright]{c}{x1d}
  %     \end{tikzinline}}  
  %   \end{gathered}  
  % \end{align*}
  % Associating right:
  % \begin{align*}
  %   \begin{gathered}[t]
  %     \PW{x}{2}
  %     \\
  %     \hbox{\begin{tikzinline}[node distance=0.5em and 1.5em]
  %       \event{a}{\DW{x}{2}}{}
  %       \xform{x1d}{\bForm[\TRUE/\Dx{\aLoc}]}{right=.7em of a}
  %         %       \xform{x1i}{\lnot\Qw{x}\land\bForm[\TRUE/\Dx{x}]}{below=of x1d}
  %       \xo{a}{x1d}
  %     \end{tikzinline}}  
  %   \end{gathered}  
  %   &&
  %   \begin{gathered}[t]
  %     \PR[\mACQ]{x}{r}
  %     \SEMI
  %     \PW{y}{1}
  %     \\
  %     \hbox{\begin{tikzinline}[node distance=0.5em and 1.5em]
  %       \raevent{b}{\DR[\mACQ]{x}{2}}{}
  %       \event{c}{\Dx{x}\mid\DW{y}{1}}{right=.7em of b}
  %     \end{tikzinline}}  
  %   \end{gathered}  
  % \end{align*}
  % Composing, we have, as desired:
  \begin{align*}
    \begin{gathered}[t]
      \PW{x}{2}
      \SEMI
      \PR{x}{r}
      \SEMI
      \PF{\fACQ}
      \SEMI
      \PW{y}{1}
      \\
      \hbox{\begin{tikzinline}[node distance=0.5em and 1.5em]
          \event{a}{\DW{x}{2}}{}
          \event{b}{\DR{x}{2}}{right=of a}
          \raevent{c}{\DF{\fACQ}}{right=of b}
          \event{d}{\DW{y}{1}}{right=of c}
          \rf{a}{b}
          \sync{b}{c}
          \sync{c}{d}
        \end{tikzinline}}  
    \end{gathered}  
  \end{align*}
  What we want is this:
  \begin{align*}
    \begin{gathered}[t]
      \hbox{\begin{tikzinline}[node distance=0.5em and 1.5em]
          \event{a}{\DW{x}{2}}{}
          \event{b}{\DR{x}{2}}{right=of a}
          \raevent{c}{\DF{\fACQ}}{right=of b}
          \event{d}{\DW{y}{1}}{right=of c}
          \rf{a}{b}
          \sync{c}{d}
        \end{tikzinline}}  
    \end{gathered}  
  \end{align*}
\end{example}

Let acquiring fence actions include a set: $\DGP{\fmode}{\aLocs}$.
The set is nondeterministically chosen by the semantics.
Idea is to downgrade the reads in $\aLocs$ and fence everything else.


\subsection{Extended Access Modes}

We can enrich read and write actions to use fence modes.
The resulting order is:
% Reads use the four point order:
% \begin{math}
%   \mRLX \ltmode \mRA \ltmode \fREL \ltmode \mSC.
% \end{math}
% Writes use the four point order:
% \begin{math}
%   \mRLX \ltmode \mRA \ltmode \fACQ \ltmode \mSC.
% \end{math}
\begin{displaymath}
  \begin{tikzpicture}
    \node (rlx) at (0, 0) {$\mathstrut\mRLX$};
    \node (ra)  at (1, 0) {$\mathstrut\mRA$};
    \node (sc)  at (2, 0) {$\mathstrut\mSC$};
    \node (fsc) at (3, 0) {$\mathstrut\fSC$};
    \node (rel) at (2, -0.3) {$\mathstrut\fREL$};
    \node (acq) at (2,  0.3) {$\mathstrut\fACQ$};
    \draw[->](rlx)to(ra);
    \draw[->](ra)to(sc);
    \draw[->](ra)to(rel);
    \draw[->](ra)to(acq);
    \draw[->](sc)to(fsc);
    \draw[->](rel)to(fsc);
    \draw[->](acq)to(fsc);
  \end{tikzpicture}
\end{displaymath}
We write $\amode\lemode\bmode$ for this order.
Let $\amode\lubmode\bmode$ denote the least upper bound of $\amode$ and $\bmode$.

Reads allow all annotations but $\fREL$.  Writes allow all annotations but
$\fACQ$.  Fences allow only the three annotations $\fREL$, $\fACQ$ and
$\fSC$.
% \begin{displaymath}
%   \begin{tikzcd}[column sep=.5em, row sep=.5ex]
%     &                                                                 & \fREL \arrow[rd,sqsubset] &        \\
%     \mRLX \arrow[r,sqsubset] & \mRA \arrow[r,sqsubset] \arrow[ru,sqsubset] \arrow[rd,sqsubset] & \mSC  \arrow[r,sqsubset]  & \fSC  \\
%     &                                                                 & \fACQ \arrow[ru,sqsubset] &          
%   \end{tikzcd}
% \end{displaymath}


\begin{definition}
  \showRAtrue
  \begin{align*}
    \fmerge{\DR[\amode]{\aLoc}{\aVal}}{\DR[\bmode]{\aLoc}{\aVal}}
    % = \fmerge{\DF{\amode}}{\DR[\bmode]{\aLoc}{\aVal}}
    % = \fmerge{\DR[\amode]{\aLoc}{\aVal}}{\DF{\bmode}}
    &= \{ \DR[\amode\lubmode\bmode]{\aLoc}{\aVal} \}
    \\
    \fmerge{\DW[\amode]{\aLoc}{\aVal}}{\DW[\bmode]{\aLoc}{\bVal}}
    % = \fmerge{\DF{\amode}}{\DW[\bmode]{\aLoc}{\bVal}}
    % = \fmerge{\DW[\amode]{\aLoc}{\bVal}}{\DF{\bmode}}
    &= \{ \DW[\amode\lubmode\bmode]{\aLoc}{\bVal} \}
    \\
    \fmerge{\DF{\amode}}{\DF{\bmode}} &= \{ \DF{\amode\lubmode\bmode} \}
    \\
    \fmerge{\DF{\amode}}{\DR[\bmode]{\aLoc}{\aVal}}
    = \fmerge{\DR[\amode]{\aLoc}{\aVal}}{\DF{\bmode}}
    &= \{ \DR[\amode\lubmode\bmode]{\aLoc}{\aVal} \}
    \\
    \fmerge{\DF{\amode}}{\DW[\bmode]{\aLoc}{\bVal}}
    = \fmerge{\DW[\amode]{\aLoc}{\bVal}}{\DF{\bmode}}
    &= \{ \DW[\amode\lubmode\bmode]{\aLoc}{\bVal} \}
    \\
    \fmerge{\aAct}{\bAct} &= \emptyset, \textotherwise
  \end{align*}
\end{definition}  


\subsection{Address Calculation (\xADDR)}
\reffig{fig:full} describes the full semantics with address calculation.
% \ref{S6full} is necessary because \ref{S5full} mentions $\cVal_\aEv$, and
% there is no such value when the set of events is empty.
\begin{definition}[\xADDR]
  \label{def:pomsets-addr}
  Update \refdef{def:pomsets-trans} to existentially quantify over $\cVal$
  in $\sSTORE{}{}$ and $\sLOAD{}{}$:
  \begin{enumerate}
  \item[\ref{S2})] $\labelingAct(\aEv) = \DW{\REF\cVal}{\aVal}$,
  \item[\ref{L2})] $\labelingAct(\aEv) = \DR{\REF{\cVal}}{\aVal}$.
  \end{enumerate}

  \begin{enumerate}
  \item[\ref{S3})] $\labelingForm(\aEv) \rimplies \cExp{=}\cVal \land \aExp{=}\aVal$,
  \item[\ref{L3})] $\labelingForm(\aEv) \rimplies \cExp{=}\cVal$.
  \end{enumerate}

  \begin{enumerate}
    % \item[\ref{S4})] $(\forall\dVal)$ $\aTr{\bEvs}{\bForm} \rimplies \cExp{=}\dVal \limplies \bForm\noSUB{[\aExp/\REF{\dVal}]}$,
    % \item[\ref{S5})] $(\forall\dVal)$ $\aTr{\cEvs}{\bForm} \rimplies \cExp{=}\dVal \limplies \bForm\noSUB{[\aExp/\REF{\dVal}]}$,
    % \item[\ref{L4})] $\phantom{(\forall\dVal)}$ $\aTr{\bEvs}{\bForm} \rimplies (\cExp{=}\cVal\limplies\aVal{=}\aReg)\limplies\bForm$, 
    % \item[\ref{L4})] $(\forall\dVal)$ $\aTr{\bEvs}{\bForm} \rimplies \cVal{=}\dVal \limplies (\cExp{=}\dVal\limplies\aVal{=}\aReg)\limplies\bForm$, 
  \item[\ref{L4})] $\aTr{\bEvs}{\bForm} \rimplies (\cExp{=}\cVal\limplies\aVal{=}\aReg)\limplies\bForm$, 
    % \item[\ref{L5})] $(\forall\dVal)$ $\aTr{\cEvs}{\bForm} \rimplies \cExp{=}\dVal \limplies (\aVal{=}\aReg\lor \REF{\dVal}{=}\aReg)\limplies\bForm$,
    % \item[\ref{L6})] $(\forall\dVal)$ $\aTr{\dEvs}{\bForm} \rimplies \cExp{=}\dVal \limplies \bForm$.
    % \item[\ref{L5})] $(\forall\dVal)$ $\aTr{\cEvs}{\bForm} \rimplies \cExp{=}\dVal \limplies \bForm$.
  \item[\ref{L5})] $\aTr{\cEvs}{\bForm} \rimplies \bForm$.
  \end{enumerate}  
\end{definition}


\subsection{Access Elimination}
For reads, get rid of $\FALSE/\Q{}$ in \ref{L6full}.

For writes, change the label rules of sequential composition to:
\begin{enumerate}
\item %\label{par-lambda1}
  if $\aEv\in\aEvs_1\setminus\aEvs_2$ then $\labeling(\aEv) = \labeling_1(\aEv)$, 
\item %\label{par-lambda2}
  if $\aEv\in\aEvs_2\setminus\aEvs_1$ then $\labeling(\aEv) = \labeling_2(\aEv)$,
\item %\label{par-lambda2}
  if $\aEv\in\aEvs_1\cap\aEvs_2$ then $\labeling(\aEv) \in \fmerge{\labeling_1(\aEv)}{\labeling_2(\aEv)}$.
\end{enumerate}

\begin{definition}
  % Fences use the three point order:
  % \begin{math}
  %   \fREL \ltmode \mSC
  % \end{math}
  % and 
  % \begin{math}
  %   \fACQ \ltmode \mSC.
  % \end{math}
  \noindent    
  \begin{align*}
    \fmerge{\DR[\amode]{\aLoc}{\aVal}}{\DR[\bmode]{\aLoc}{\aVal}} &= \{ \DR[\amode\lubmode\bmode]{\aLoc}{\aVal} \}
    \\
    \fmerge{\DW[\amode]{\aLoc}{\aVal}}{\DW[\bmode]{\aLoc}{\bVal}} &= \{ \DW[\amode\lubmode\bmode]{\aLoc}{\bVal} \}
    \\
    \fmerge{\DF{\amode}}{\DF{\bmode}} &= \{ \DF{\amode\lubmode\bmode} \}
    \\
    \fmerge{\aAct}{\bAct} &= \emptyset, \textotherwise
  \end{align*}
\end{definition}  

\subsection{Merging Different labels}

Reordering and Merging:
\cite[\textsection7.1]{Kang19}
\cite[\textsection E]{DBLP:conf/cgo/ChakrabortyV17}

Examples of Unsafe Reorderings
\cite[\textsection D]{DBLP:conf/cgo/ChakrabortyV17}
See the slides for this paper...


We combine access and fence modes into a single order:
\begin{align*}
  \begin{tikzcenter}
    \node (rlx) at (0, 0) {$\mathstrut\mRLX$};
    \node (ra)  at (1, 0) {$\mathstrut\mRA$};
    \node (sc)  at (2, 0) {$\mathstrut\mSC$};
    \draw[->](rlx)to(ra);
    \draw[->](ra)to(sc);
  \end{tikzcenter}
  &&
  \begin{tikzcenter}
    \node (fsc) at (3, 0) {$\mathstrut\fSC$};
    \node (rel) at (2, -0.3) {$\mathstrut\fREL$};
    \node (acq) at (2,  0.3) {$\mathstrut\fACQ$};
    \draw[->](rel)to(fsc);
    \draw[->](acq)to(fsc);
  \end{tikzcenter}
\end{align*}

Note that for associativity, you have to take the join of modes.
\begin{definition}
  \label{def:compat}
  Define $\fmerge{}{}:\Act\times\Act\fun2^{\Act}$ as follows.  If
  $\aAct_0\in\fmerge{\aAct_1}{\aAct_2}$, then $\aAct_1$ and $\aAct_2$ can
  coalesce, resulting in $\aAct_0$.  This is useful for replacing
  $(\PW{x}{1}\SEMI \PW{x}{2})$ by $(\PW{x}{2})$.
  \begin{align*}
    \fmerge{\DR[\amode]{\aLoc}{\aVal}}{\DR[\bmode]{\aLoc}{\aVal}}
    &= \{ \DR[\amode\lubmode\bmode]{\aLoc}{\aVal} \}
    \\
    \fmerge{\DW[\amode]{\aLoc}{\aVal}}{\DW[\bmode]{\aLoc}{\bVal}}
    &= \{ \DW[\amode\lubmode\bmode]{\aLoc}{\bVal} \}
    \\
    \fmerge{\DR[\mRLX]{\aLoc}{\aVal}}{\DW[\amode]{\aLoc}{\aVal}}
    &= \{ \DW[\amode]{\aLoc}{\aVal} \}
    \\
    \fmerge{\DR[\amode\neq\mRLX]{\aLoc}{\aVal}}{\DW[\bmode]{\aLoc}{\aVal}}
    &= \{ \DW[\mSC]{\aLoc}{\aVal} \}
    \\
    \fmerge{\DF{\amode}}{\DF{\bmode}} &= \{ \DF{\amode\lubmode\bmode} \}
    \\
    \fmerge{\aAct}{\bAct} &= \emptyset, \textotherwise
  \end{align*}
\end{definition}  

\begin{enumerate}
\item \label{seq-lambda1}
  if $\aEv\in\aEvs_1\setminus\aEvs_2$ then $\labeling(\aEv)=\labeling_1(\aEv)$,
\item \label{seq-lambda2}
  if $\aEv\in\aEvs_2\setminus\aEvs_1$ then $\labeling(\aEv)=\labeling_2(\aEv)$,
\item \label{seq-lambda12}
  if $\aEv\in\aEvs_1\cap\aEvs_2$ then $\labeling(\aEv)\in
  \fmerge{\labeling_1(\aEv)}{\labeling_2(\aEv)}$, the first has no rf,
\end{enumerate}


\subsection{Downgraded Reads}
\label{sec:dgr}

We allow downgrades in executions that are not be allowed by \armeight{}.
\begin{gather*}
  \PW{x}{2}\SEMI 
  \PR[\mACQ]{x}{r}\SEMI
  \PW{y}{1} \PAR
  \PW{y}{2}\SEMI
  \PW[\mREL]{x}{1} \PAR
  \PW{x}{3}
  \\
  \hbox{\begin{tikzinline}[node distance=1.5em]
      \event{a}{\DW{x}{2}}{}
      \raevent{b}{\DR[\mACQ]{x}{3}}{right=of a}
      \event{c}{\DW{y}{1}}{right=of b}
      \event{d}{\DW{y}{2}}{right=2.5em of c}
      \raevent{e}{\DW[\mREL]{x}{1}}{right=of d}
      \event{f}{\DW{x}{3}}{right=2.5em of e}
      \wk[out=15,in=165]{a}{f}
      \rf[out=-165,in=-15]{f}{b}
      % \sync{b}{c}
      \wk{c}{d}
      \sync{d}{e}
      \wk[out=-165,in=-15]{e}{a}
    \end{tikzinline}}  
\end{gather*}
\armeight{} disallows this because the acquiring read is fulfilled by an
external write.
\begin{gather*}
  % \taglabel{data-rfi-rfe-rfe}
  \PW{x}{\PR{z}{}} \SEMI
  \PR[\mACQ]{x}{r}\SEMI
  \PW{y}{1} \PAR
  \PW{z}{\PR{y}{}}
  \\
  % \tag{\xmark\armeight}
  \hbox{\begin{tikzinline}[node distance=1.5em]
      \event{a}{\DR{z}{1}}{}
      \event{b}{\DW{x}{1}}{right=of a}
      \raevent{c}{\DR[\mACQ]{x}{1}}{right=of b}
      \event{d}{\DW{y}{1}}{right=of c}
      \event{e}{\DW{y}{1}}{right=2.5em of d}
      \event{f}{\DW{z}{1}}{right=of e}
      \po{a}{b}
      \rf{b}{c}
      % \bob{c}{d}
      \po{e}{f}
      \rf[out=-165,in=-15]{f}{a}
      \rf{d}{e}
    \end{tikzinline}}
\end{gather*}
\armeight{} disallows this because data and control dependencies change  acquiring read is fulfilled by an
external write.


\subsection{Using Independency for Coherence}
\label{sec:independency-co}

It is also possible to use independency only to capture coherence, but the
results are less interesting.

\begin{figure*}[t]
  \begin{subfigure}{.5\textwidth}
    \centering
    \begin{align*}
  \begin{gathered}
    \begin{gathered}[t]
      \PW{x}{\aExp}
      \\
      \hbox{\begin{tikzinline}[node distance=.5em and 1.5em]
          \event{a}{\aExp{=}v\land\Qr{x}\land\Qw{x}\mid\DW{x}{v}}{}
          \xform{xi}{\bForm[\FALSE/\Qw{x}]}{above=of a}
          \xform{xd}{\bForm[(\Qw{x}\land\aExp{=}\aVal)/\Qw{x}]}{below=of a}
          \xo{a}{xd}
        \end{tikzinline}}
    \end{gathered}
    \\[1ex]
    \begin{gathered}[t]
      \PW[\mRA]{x}{\aExp}
      \\
      \hbox{\begin{tikzinline}[node distance=.5em and 1.5em]
          \raevent{a}{\aExp{=}v\land\Qr{*}\land\Qw{*}\mid\DW[\mRA]{x}{v}}{}
          \xform{xi}{\bForm[\FALSE/\Qw{x}]}{above=of a}
          \xform{xd}{\bForm[(\Qw{x}\land\aExp{=}\aVal)/\Qw{x}]}{below=of a}
          \xo{a}{xd}
        \end{tikzinline}}
    \end{gathered}
    \\[1ex]
    \begin{gathered}[t]
      \PW[\mSC]{x}{\aExp}
      \\
      \hbox{\begin{tikzinline}[node distance=.5em and 1.5em]
          \scevent{a}{\aExp{=}v\land\Qr{*}\land\Qw{*}\land\Qsc\mid\DW[\mSC]{x}{v}}{}
          \xform{xi}{\bForm[\FALSE/\Qw{x}][\FALSE/\Qsc]}{above=of a}
          \xform{xd}{\bForm[(\Qw{x}\land\aExp{=}\aVal)/\Qw{x}]}{below=of a}
          \xo{a}{xd}
        \end{tikzinline}}
    \end{gathered}
  \end{gathered}
  &&
  \begin{gathered}
    \begin{gathered}[t]
      \PR{x}{r}
      \\
      \hbox{\begin{tikzinline}[node distance=.5em and 1.5em]
          \event{a}{\Qw{x}\mid\DR{x}{v}}{}
          \xform{xi}{\bForm[\FALSE/\Qr{x}]}{above=of a}
          \xform{xd}{v{=}r\limplies\bForm}{below=of a}
          \xo{a}{xd}
        \end{tikzinline}}
    \end{gathered}
    \\[1ex]
    \begin{gathered}[t]
      \PR[\mRA]{x}{r}
      \\
      \hbox{\begin{tikzinline}[node distance=.5em and 1.5em]
          \raevent{a}{\Qw{x}\mid\DR[\mRA]{x}{v}}{}
          \xform{xi}{\bForm[\FALSE/\Qr{*}][\FALSE/\Qw{*}]}{above=of a}
          \xform{xd}{v{=}r\limplies\bForm}{below=of a}
          \xo{a}{xd}
        \end{tikzinline}}
    \end{gathered}
    \\[1ex]
    \begin{gathered}[t]
      \PR[\mSC]{x}{r}
      \\
      \hbox{\begin{tikzinline}[node distance=.5em and 1.5em]
          \scevent{a}{\Qw{x}\land\Qsc\mid\DR[\mSC]{x}{v}}{}
          \xform{xi}{\bForm[\FALSE/\Qr{*}][\FALSE/\Qw{*}][\FALSE/\Qsc]}{above=of a}
          \xform{xd}{v{=}r\limplies\bForm}{below=of a}
          \xo{a}{xd}
        \end{tikzinline}}
    \end{gathered}
  \end{gathered}
\end{align*}
    \caption{Quiescence Examples (\textsection\ref{sec:sync})}
    \label{fig:q1}
  \end{subfigure}  
  \begin{subfigure}{.5\textwidth}
    \centering
    \begin{align*}
  \begin{gathered}
    \begin{gathered}[t]
      \PW{x}{\aExp}
      \\
      \hbox{\begin{tikzinline}[node distance=.5em and 1.5em]
          \event{a}{\Qra \land\aExp{=}v\mid\DW{x}{v}}{}
          \xform{xi}{\bForm[\FALSE/\Qrlx]}{above=of a}
          \xform{xd}{\bForm\land \aExp{=}\aVal}{below=of a}
          %\xform{xd}{\bForm[(\Qrlx\land\aExp{=}\aVal)/\Qrlx]}{below=of a}
          \xo[xright]{a}{xd}
        \end{tikzinline}}
    \end{gathered}
    \\[1ex]
    \begin{gathered}[t]
      \PW[\mREL]{x}{\aExp}
      \\
      \hbox{\begin{tikzinline}[node distance=.5em and 1.5em]
          \raevent{a}{\Qra\land\Qrlx \land\aExp{=}v\mid\DW[\mREL]{x}{v}}{}
          \xform{xi}{\bForm[\FALSE/\Qrlx]}{above=of a}
          \xform{xd}{\bForm\land \aExp{=}\aVal}{below=of a}
          %\xform{xd}{\bForm[(\Qrlx\land\aExp{=}\aVal)/\Qrlx]}{below=of a}
          \xo[xright]{a}{xd}
        \end{tikzinline}}
    \end{gathered}
    \\[1ex]
    \begin{gathered}[t]
      \PW[\mSC]{x}{\aExp}
      \\
      \hbox{\begin{tikzinline}[node distance=.5em and 1.5em]
          \scevent{a}{\Qra\land\Qrlx\land\Qsc \land\aExp{=}v\mid\DW[\mSC]{x}{v}}{}
          \xform{xi}{\bForm[\FALSE/\Qrlx][\FALSE/\Qsc]}{above=of a}
          \xform{xd}{\bForm\land \aExp{=}\aVal}{below=of a}
          %\xform{xd}{\bForm[(\Qrlx\land\aExp{=}\aVal)/\Qrlx]}{below=of a}
          \xo[xright]{a}{xd}
        \end{tikzinline}}
    \end{gathered}
  \end{gathered}
  &&
  \begin{gathered}
    \begin{gathered}[t]
      \PR{x}{r}
      \\
      \hbox{\begin{tikzinline}[node distance=.5em and 1.5em]
          \event{a}{\Qra\mid\DR{x}{v}}{}
          \xform{xi}{\bForm[\FALSE/\Qrlx]}{above=of a}
          \xform{xd}{v{=}r\limplies\bForm}{below=of a}
          \xo[xright]{a}{xd}
        \end{tikzinline}}
    \end{gathered}
    \\[1ex]
    \begin{gathered}[t]
      \PR[\mACQ]{x}{r}
      \\
      \hbox{\begin{tikzinline}[node distance=.5em and 1.5em]
          \raevent{a}{\Qra\mid\DR[\mACQ]{x}{v}}{}
          \xform{xi}{\bForm[\FALSE/\Qrlx][\FALSE/\Qra]}{above=of a}
          \xform{xd}{v{=}r\limplies\bForm}{below=of a}
          \xo[xright]{a}{xd}
        \end{tikzinline}}
    \end{gathered}
    \\[1ex]
    \begin{gathered}[t]
      \PR[\mSC]{x}{r}
      \\
      \hbox{\begin{tikzinline}[node distance=.5em and 1.5em]
          \scevent{a}{\Qra\land\Qsc\mid\DR[\mSC]{x}{v}}{}
          \xform{xi}{\bForm[\FALSE/\Qrlx][\FALSE/\Qra][\FALSE/\Qsc]}{above=of a}
          \xform{xd}{v{=}r\limplies\bForm}{below=of a}
          \xo[xright]{a}{xd}
        \end{tikzinline}}
    \end{gathered}
  \end{gathered}
\end{align*}

    \caption{Quiescence Examples (\textsection\ref{sec:independency-co})}
    \label{fig:q2}
  \end{subfigure}
  \caption{Quiescence Examples for Coherence}
  \label{fig:q-co}
\end{figure*}
% \begin{figure}[t!]
%   \centering
%   \begin{align*}
  \begin{gathered}
    \begin{gathered}[t]
      \PW{x}{\aExp}
      \\
      \hbox{\begin{tikzinline}[node distance=.5em and 1.5em]
          \event{a}{\aExp{=}v\land\Qr{x}\land\Qw{x}\mid\DW{x}{v}}{}
          \xform{xi}{\bForm[\FALSE/\Qw{x}]}{above=of a}
          \xform{xd}{\bForm[(\Qw{x}\land\aExp{=}\aVal)/\Qw{x}]}{below=of a}
          \xo{a}{xd}
        \end{tikzinline}}
    \end{gathered}
    \\[1ex]
    \begin{gathered}[t]
      \PW[\mRA]{x}{\aExp}
      \\
      \hbox{\begin{tikzinline}[node distance=.5em and 1.5em]
          \raevent{a}{\aExp{=}v\land\Qr{*}\land\Qw{*}\mid\DW[\mRA]{x}{v}}{}
          \xform{xi}{\bForm[\FALSE/\Qw{x}]}{above=of a}
          \xform{xd}{\bForm[(\Qw{x}\land\aExp{=}\aVal)/\Qw{x}]}{below=of a}
          \xo{a}{xd}
        \end{tikzinline}}
    \end{gathered}
    \\[1ex]
    \begin{gathered}[t]
      \PW[\mSC]{x}{\aExp}
      \\
      \hbox{\begin{tikzinline}[node distance=.5em and 1.5em]
          \scevent{a}{\aExp{=}v\land\Qr{*}\land\Qw{*}\land\Qsc\mid\DW[\mSC]{x}{v}}{}
          \xform{xi}{\bForm[\FALSE/\Qw{x}][\FALSE/\Qsc]}{above=of a}
          \xform{xd}{\bForm[(\Qw{x}\land\aExp{=}\aVal)/\Qw{x}]}{below=of a}
          \xo{a}{xd}
        \end{tikzinline}}
    \end{gathered}
  \end{gathered}
  &&
  \begin{gathered}
    \begin{gathered}[t]
      \PR{x}{r}
      \\
      \hbox{\begin{tikzinline}[node distance=.5em and 1.5em]
          \event{a}{\Qw{x}\mid\DR{x}{v}}{}
          \xform{xi}{\bForm[\FALSE/\Qr{x}]}{above=of a}
          \xform{xd}{v{=}r\limplies\bForm}{below=of a}
          \xo{a}{xd}
        \end{tikzinline}}
    \end{gathered}
    \\[1ex]
    \begin{gathered}[t]
      \PR[\mRA]{x}{r}
      \\
      \hbox{\begin{tikzinline}[node distance=.5em and 1.5em]
          \raevent{a}{\Qw{x}\mid\DR[\mRA]{x}{v}}{}
          \xform{xi}{\bForm[\FALSE/\Qr{*}][\FALSE/\Qw{*}]}{above=of a}
          \xform{xd}{v{=}r\limplies\bForm}{below=of a}
          \xo{a}{xd}
        \end{tikzinline}}
    \end{gathered}
    \\[1ex]
    \begin{gathered}[t]
      \PR[\mSC]{x}{r}
      \\
      \hbox{\begin{tikzinline}[node distance=.5em and 1.5em]
          \scevent{a}{\Qw{x}\land\Qsc\mid\DR[\mSC]{x}{v}}{}
          \xform{xi}{\bForm[\FALSE/\Qr{*}][\FALSE/\Qw{*}][\FALSE/\Qsc]}{above=of a}
          \xform{xd}{v{=}r\limplies\bForm}{below=of a}
          \xo{a}{xd}
        \end{tikzinline}}
    \end{gathered}
  \end{gathered}
\end{align*}
%   \caption{Quiescence Examples (Old)}
%   \label{fig:q1}
% \end{figure}
% \begin{figure}[t!]
%   \centering
%   \begin{align*}
  \begin{gathered}
    \begin{gathered}[t]
      \PW{x}{\aExp}
      \\
      \hbox{\begin{tikzinline}[node distance=.5em and 1.5em]
          \event{a}{\Qra \land\aExp{=}v\mid\DW{x}{v}}{}
          \xform{xi}{\bForm[\FALSE/\Qrlx]}{above=of a}
          \xform{xd}{\bForm\land \aExp{=}\aVal}{below=of a}
          %\xform{xd}{\bForm[(\Qrlx\land\aExp{=}\aVal)/\Qrlx]}{below=of a}
          \xo[xright]{a}{xd}
        \end{tikzinline}}
    \end{gathered}
    \\[1ex]
    \begin{gathered}[t]
      \PW[\mREL]{x}{\aExp}
      \\
      \hbox{\begin{tikzinline}[node distance=.5em and 1.5em]
          \raevent{a}{\Qra\land\Qrlx \land\aExp{=}v\mid\DW[\mREL]{x}{v}}{}
          \xform{xi}{\bForm[\FALSE/\Qrlx]}{above=of a}
          \xform{xd}{\bForm\land \aExp{=}\aVal}{below=of a}
          %\xform{xd}{\bForm[(\Qrlx\land\aExp{=}\aVal)/\Qrlx]}{below=of a}
          \xo[xright]{a}{xd}
        \end{tikzinline}}
    \end{gathered}
    \\[1ex]
    \begin{gathered}[t]
      \PW[\mSC]{x}{\aExp}
      \\
      \hbox{\begin{tikzinline}[node distance=.5em and 1.5em]
          \scevent{a}{\Qra\land\Qrlx\land\Qsc \land\aExp{=}v\mid\DW[\mSC]{x}{v}}{}
          \xform{xi}{\bForm[\FALSE/\Qrlx][\FALSE/\Qsc]}{above=of a}
          \xform{xd}{\bForm\land \aExp{=}\aVal}{below=of a}
          %\xform{xd}{\bForm[(\Qrlx\land\aExp{=}\aVal)/\Qrlx]}{below=of a}
          \xo[xright]{a}{xd}
        \end{tikzinline}}
    \end{gathered}
  \end{gathered}
  &&
  \begin{gathered}
    \begin{gathered}[t]
      \PR{x}{r}
      \\
      \hbox{\begin{tikzinline}[node distance=.5em and 1.5em]
          \event{a}{\Qra\mid\DR{x}{v}}{}
          \xform{xi}{\bForm[\FALSE/\Qrlx]}{above=of a}
          \xform{xd}{v{=}r\limplies\bForm}{below=of a}
          \xo[xright]{a}{xd}
        \end{tikzinline}}
    \end{gathered}
    \\[1ex]
    \begin{gathered}[t]
      \PR[\mACQ]{x}{r}
      \\
      \hbox{\begin{tikzinline}[node distance=.5em and 1.5em]
          \raevent{a}{\Qra\mid\DR[\mACQ]{x}{v}}{}
          \xform{xi}{\bForm[\FALSE/\Qrlx][\FALSE/\Qra]}{above=of a}
          \xform{xd}{v{=}r\limplies\bForm}{below=of a}
          \xo[xright]{a}{xd}
        \end{tikzinline}}
    \end{gathered}
    \\[1ex]
    \begin{gathered}[t]
      \PR[\mSC]{x}{r}
      \\
      \hbox{\begin{tikzinline}[node distance=.5em and 1.5em]
          \scevent{a}{\Qra\land\Qsc\mid\DR[\mSC]{x}{v}}{}
          \xform{xi}{\bForm[\FALSE/\Qrlx][\FALSE/\Qra][\FALSE/\Qsc]}{above=of a}
          \xform{xd}{v{=}r\limplies\bForm}{below=of a}
          \xo[xright]{a}{xd}
        \end{tikzinline}}
    \end{gathered}
  \end{gathered}
\end{align*}

%   \caption{Quiescence Examples (New)}
%   \label{fig:q2}
% \end{figure}

In the logic, we remove the symbols $\Qw{\aLoc}$ and $\Qr{\aLoc}$.
Previously, we had given the semantics of $\mRA$ access using $\Qw{*}$ and
$\Qr{*}$, which were encoded using $\Qw{\aLoc}$ and $\Qr{\aLoc}$.  With these
gone, we introduce the quiescence symbol $\Qrlx$ and $\Qra$.  Thus, the only
quiescence symbols required are $\Qrlx$, $\Qra$ and $\Qsc$.
\reffig{fig:q-co} shows the difference with the semantics of \textsection\ref{sec:sync}.
% \reffig{fig:q1}.  In the new
% interpretation, see \reffig{fig:q2}.

% To see how the quiescence symbols are used, consider the following examples:

\begin{comment}
  Let formulae $\QS{\aLoc}{\amode}$ and $\QL{\aLoc}{\amode}$ be defined:
  \begin{align*}
    \QS{\aLoc}{\mRLX}&=\Qr{\aLoc}\land\Qw{\aLoc}
    &\QL{\aLoc}{\mRLX}&=\Qw{\aLoc}
    \\
    \QS{\aLoc}{\mRA}&=
    \Qr{*}\land\Qw{*} %\textstyle\bigwedge_\bLoc \Qr{\bLoc}\land\Qw{\bLoc}
    &\QL{\aLoc}{\mRA}&=\Qw{\aLoc}
    \\
    \QS{\aLoc}{\mSC}&=
    \Qr{*}\land\Qw{*} %\textstyle\bigwedge_\bLoc \Qr{\bLoc}\land\Qw{\bLoc}
    \land \Qsc
    &\QL{\aLoc}{\mSC}&=\Qw{\aLoc}\land\Qsc
  \end{align*}
  Let substitutions $[\aForm/\QS{\aLoc}{\amode}]$ and  $[\aForm/\QL{\aLoc}{\amode}]$ be defined:
  \begin{align*}
    [\aForm/\QS{\aLoc}{\mRLX}] &= [\aForm/\Qw{\aLoc}]
    &{} [\aForm/\QL{\aLoc}{\mRLX}] &= [\aForm/\Qr{\aLoc}]
    \\
    [\aForm/\QS{\aLoc}{\mRA}] &= [\aForm/\Qw{\aLoc}]
    &{} [\aForm/\QL{\aLoc}{\mRA}] &= [\aForm/\Qr{*},\aForm/\Qw{*}]
    \\
    [\aForm/\QS{\aLoc}{\mSC}] &= [\aForm/\Qw{\aLoc},\aForm/\Qsc]
    &{} [\aForm/\QL{\aLoc}{\mSC}] &= [\aForm/\Qr{*},\aForm/\Qw{*},\aForm/\Qsc]
  \end{align*}
  Update \refdef{def:pomsets-trans} from: % (\ref{S4}/\ref{L4} unchanged):
  \begin{enumerate}
  \item[\ref{S3})]
    $\labelingForm(\aEv) \rimplies \aExp{=}\aVal\land\QS{\aLoc}{\amode}$,
  \item[\ref{L3})]
    $\labelingForm(\aEv) \rimplies \QL{\aLoc}{\amode}$,
  \item[\ref{T3})]
    $\labelingForm(\aEv) \rimplies \labelingForm_1(\aEv)[\TRUE/\Qr{*}][\TRUE/\Qw{*}][\TRUE/\Qsc]$,
  \end{enumerate}
  \begin{enumerate}
  \item[\ref{S4})]
    $\aTr{\bEvs}{\bForm} \rimplies \bForm \land\aExp{=}\aVal$,
  \item[\ref{S5})]
    $\aTr{\cEvs}{\bForm} \rimplies \bForm[\FALSE/\QS{\aLoc}{\amode}]$,
  \item[\ref{L4})]
    $\aTr{\bEvs}{\bForm} \rimplies \aVal{=}\aReg\limplies\bForm$, 
  \item[\ref{L5})]
    $\aTr{\cEvs}{\bForm} \rimplies \bForm[\FALSE/\QL{\aLoc}{\amode}]$.
  \end{enumerate}
\end{comment}

\begin{definition}
  Let formulae $\QS{}{\amode}$ and $\QL{}{\amode}$ be defined:
  \begin{scope}
    \small
    \begin{align*}
      \QS{}{\mRLX}&=\Qra
      &\QL{}{\mRLX}&=\Qra
      \\
      \QS{}{\mRA}&=\Qra\land\Qrlx
      &\QL{}{\mRA}&=\Qra
      \\
      \QS{}{\mSC}&=\Qra\land \Qrlx \land \Qsc
      &\QL{}{\mSC}&=\Qra\land\Qsc
    \end{align*}
  \end{scope}
  Let substitutions $[\aForm/\QS{}{\amode}]$ and  $[\aForm/\QL{}{\amode}]$ be defined:
  \begin{scope}
    \small
    \begin{align*}
      [\aForm/\QS{}{\mRLX}] &= [\aForm/\Qrlx]
      &{} [\aForm/\QL{}{\mRLX}] &= [\aForm/\Qrlx]
      \\
      [\aForm/\QS{}{\mRA}] &= [\aForm/\Qrlx]
      &{} [\aForm/\QL{}{\mRA}] &= [\aForm/\Qrlx,\aForm/\Qra]
      \\
      [\aForm/\QS{}{\mSC}] &= [\aForm/\Qrlx,\aForm/\Qsc]
      &{} [\aForm/\QL{}{\mSC}] &= [\aForm/\Qrlx,\aForm/\Qra,\aForm/\Qsc]
    \end{align*}
  \end{scope}
\end{definition}
\begin{definition}%[\xCO/\xRASC]
  Update \refdef{def:pomsets-trans} to: %and \ref{def:pomsets-fj} to: % (\ref{S4}/\ref{L4} unchanged):
  \begin{enumerate}
  \item[\ref{S3})]
    $\labelingForm(\aEv) \rimplies \QS{}{\amode}\land\aExp{=}\aVal$,
  \item[\ref{L3})]
    $\labelingForm(\aEv) \rimplies \QL{}{\amode}$,
    % \item[\ref{F3})]
    %   $\labelingForm(\aEv) \rimplies \Qrlx\land\Qra\land\Qsc\land\labelingForm_1(\aEv)$, 
    % \item[\ref{T3})]
    %   $\labelingForm(\aEv) \rimplies \labelingForm_1(\aEv)[\TRUE/\Q{}]$, %[\TRUE/\Qrlx][\TRUE/\Qra][\TRUE/\Qsc]$,
  \end{enumerate}
  \begin{enumerate}
  \item[\ref{S4})]
    $\aTr{\bEvs}{\bForm} \rimplies \bForm \land\aExp{=}\aVal$,
  \item[\ref{S5})]
    $\aTr{\cEvs}{\bForm} \rimplies \bForm[\FALSE/\QS{}{\amode}]$,
  \item[\ref{L4})]
    $\aTr{\bEvs}{\bForm} \rimplies \aVal{=}\aReg\limplies\bForm$, 
  \item[\ref{L5})]
    $\aTr{\cEvs}{\bForm} \rimplies \bForm[\FALSE/\QL{}{\amode}]$.
  \end{enumerate}
\end{definition}

The most interesting examples in \reffig{fig:q2} concern $\mRA$ access.
Every independent transformer substitutes $[\FALSE/\Qrlx]$.  $\Qrlx$ is a
precondition for any releasing write $\aEv$, ensuring that all preceding
events must are ordered before $\aEv$.  Conversely, $\Qra$ is a precondition
of every event.  The independent transformer for any acquiring read $\aEv$
substitutes $[\FALSE/\Qra]$, ensuring that all following events must be
ordered after $\aEv$.

% As before, the substitution in \ref{S4} ensures that left merges are not
% quiescent (\refex{ex:left-merge}).

Item \ref{seq-reorder} of \refdef{def:independency-co} ensures
coherence.  This definition is incompatible with asynchronous $\FORK{}$
parallelism of \refdef{def:pomsets-group}, where we expect executions such
as:
\begin{gather*}
  \FORK{\THREAD{\PR{x}{r}}}\SEMI \PW{x}{1}
  \\
  \hbox{\begin{tikzinline}[node distance=0.5em and 1.5em]
      \event{a}{\DR{x}{1}}{}
      \event{b}{\DW{x}{1}}{right=3em of a}
      \rf{b}{a}
    \end{tikzinline}}
\end{gather*}
Item \ref{seq-reorder} would require $\DRP{x}{1} \xwki \DWP{x}{1}$, forbidding
this.

% , since \ref{T3full} substitutes
% $\TRUE$ for every quiescence symbol.  Preconditions of augment-minimal
% pomsets in $\sem{\FORK{\THREAD{\aCmd}}}$ contain no quiescence symbols.
% Instead, preconditions of augment-minimal pomsets in
% $\sem{\FORKJOIN{\THREAD{\aCmd}}}$ are saturated with quiescence symbols.

% One must be careful, however, due to \emph{inconsistency}.  Consider that
% \texttt{x=0;x=1} should not have completed pomset with only $\DWP{x}{0}$.

% \eqref{seq-reorder} does not do the right thing with fork either.  If you
% want to enforce coherence this way then you need to use fork-join as the
% sequential combinator, rather than fork.


% [We drop $\reorder$ because incompatible with $\sFORK{}$.  If you want to
% use $\reorder$, then you need to use fork-join as the sequential
% combinator, rather than fork.]

% We can then encode coherence as follows.
% \begin{enumerate}
%   \setcounter{enumi}{\value{pomsetXSemiCount}}
% \item if $\bEv\in\aEvs_1$ and $\aEv\in\aEvs_2$ either $\bEv<\aEv$ or
%   $a\reorder\labeling_2(\aEv)$.
% \end{enumerate}


% Access modes can be encoded in the independency relation, indexing labels by
% $\amode$, but the extra flexibility of the logic is necessary for \armeight{}
% (see \textsection\ref{sec:downgrade}).  Using independency, one would also
% need another way to define completed pomsets.  Finally, this use of
% independency is incompatible with fork (see \textsection\ref{sec:co}).


% If we move coherence to independency (and use fork-join), we have the
% following, assuming that each register occurs at most once.
% \begin{align*}
%   \QS{}{\mSC}&=\Q{\mSC}
%   &\QS{}{\mRA}&=\Q{\mRA}
%   &\QS{}{\mRLX}&=\Qx{\aLoc}
%   \\
%   \QL{}{\mSC}&=\Q{\mSC}
%   &\QL{}{\mRA}&=\Qw{\aLoc}
%   &\QL{}{\mRLX}&=\Qw{\aLoc}
%   \\
%   \DS{\aLoc}{\mSC}{\bForm}&=\bForm[\FALSE/\D]
%   &\DS{\aLoc}{\mRA}{\bForm}&=\bForm[\FALSE/\D]
%   &\DS{\aLoc}{\mRLX}{\bForm}&=\bForm[\TRUE/\Dx{\aLoc}] 
%   \\
%   \DL{\aLoc}{\mSC}&=\Dx{\aLoc}
%   &\DL{\aLoc}{\mRA}&=\Dx{\aLoc}
%   &\DL{\aLoc}{\mRLX}&=\TRUE
% \end{align*}

% % $\QS{}{\mRLX}=\TRUE$ and otherwise $\QS{}{\amode}=\Q{\amode}$.

% % $\QL{}{\mSC}=\Q{\mSC}$ and otherwise $\QL{}{\amode}=\TRUE$.

% % $\DS{\aLoc}{\mRLX}{\bForm}=\bForm[\TRUE/\Dx{\aLoc}]$ and otherwise
% % $\DS{\aLoc}{\amode}{\bForm}=\bForm[\FALSE/\D]$.

% % $\DL{\aLoc}{\mRLX}=\TRUE$ and otherwise $\DL{\aLoc}{\amode}=\Dx{\aLoc}$.

% % \begin{definition}$\phantom{\;}$\par
% %   $\QS{}{\mRLX}=\TRUE$ and otherwise $\QS{}{\amode}=\Q{\amode}$.

% %   $\QL{}{\mSC}=\Q{\mSC}$ and otherwise $\QL{}{\amode}=\TRUE$.

%   \noindent
%   \begin{enumerate}
%   \item[\ref{S3})] $\labelingForm(\aEv)$ implies
%     \begin{math}
%       \aExp{=}\aVal \land \RW \land \QS{}{\amode}
%     \end{math},
%   \item[\ref{S4})] $\aTr{\bEvs}{\bForm}$ implies
%     \begin{math}
%       \aExp{=}\aVal \land \DS{\aLoc}{\amode}{\bForm[\aExp/{\aLoc}]}
%     \end{math},
%   \item[\ref{S5})] $\aTr{\emptyset}{\bForm}$ implies
%     \begin{math}
%       \lnot\Q{\mRA} \land \DS{\aLoc}{\amode}{\bForm[\aExp/{\aLoc}]}
%     \end{math}
%   \end{enumerate}

%   \noindent
%   \begin{enumerate}
%   \item[\ref{L3})] $\labelingForm(\aEv)$ implies
%     \begin{math}
%       \RO \land \QL{}{\amode}
%     \end{math},
%   \item[\ref{L4})] $\aTr{\bEvs}{\bForm}$ implies
%     \begin{math}
%       (\aVal{=}\aReg) \limplies \bForm[\aReg/{\aLoc}]
%     \end{math}
%   \item[\ref{L5})] $\aTr{\emptyset}{\bForm}$ implies
%     \begin{math}
%       \DL{\aLoc}{\amode} \land \lnot\Q{\mRA} \land (\RW \limplies
%       (\aVal{=}\aReg\lor\aLoc{=}\aReg) \limplies \bForm[\aReg/{\aLoc}] ).
%     \end{math}
%   \end{enumerate}

