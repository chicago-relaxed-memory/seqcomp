\section{Quiescence}
\label{sec:q}

We introduce \emph{quiescence}, which captures \emph{coherence},
\emph{synchronized access}, and \emph{completion}.  Recall from
\textsection\ref{sec:prelim} that formulae include
symbols $\Q{\mSC}$, $\Qr{\aLoc}$, and $\Qw{\aLoc}$.
We refer to these collectively as \emph{quiescence symbols}.
In this section, we will show how these logical symbols can be used
to capture coherence and synchronization. This illustrates a feature
of our model, which is that many features of weak memory can be cautured
in the logic, not in the pomset model itself.

\subsection{Coherence ($\xCO$)}
\label{sec:co}

In the logic, the quiescence symbols are just uninterpreted formula, but
the semantics uses them as preconditions, to ensure appropriate causal
order.
For example, \emph{write-write coherence} enforces order
between writes to the same location in the same thread. We
model this by adding the precondition
$(\Qr{\aLoc}\land\Qw{\aLoc})$ to 
events that write to $x$, for example:
  \begin{gather*}
    \PW{x}{1}\SEMI \PW{x}{2}
    \\
    \hbox{\begin{tikzinline}[node distance=.5em and 1.5em]
        \event{wx1}{1{=}1\land \Qr{\aLoc}\land\Qw{\aLoc} \mid\DW{x}{1}}{}
        \event{wx2}{2{=}2\land \Qr{\aLoc}\land\Qw{\aLoc} \mid\DW{x}{2}}{right=of wx1}
        \wk{wx1}{wx2}
      \end{tikzinline}}
  \end{gather*}
These symbols are left alone in the dependent case, but
in the independent case we substitute $\FALSE$ for $\Qw{\aLoc}$:
  \begin{gather*}
    \PW{x}{1}\SEMI \PW{x}{2}
    \\
    \hbox{\begin{tikzinline}[node distance=.5em and 1.5em]
        \event{wx1}{1{=}1\land \Qr{\aLoc}\land\Qw{\aLoc} \mid\DW{x}{1}}{}
        \event{wx2}{2{=}2\land \Qr{\aLoc}\land\FALSE \mid\DW{x}{2}}{right=of wx1}
      \end{tikzinline}}
  \end{gather*}
This substitution is part of the predicate transformer for store:
  \begin{gather*}
      \PW{x}{1} 
      \\
      \hbox{\begin{tikzinline}[node distance=.5em and 1.5em]
        \event{wx1}{1{=}1\land \Qr{\aLoc}\land\Qw{\aLoc} \mid\DW{x}{1}}{}
        \xform{xd}{\bForm}{right=of wx1}
        \xform{xi}{\bForm[\FALSE/\Qw{\aLoc}]}{right=of xd}
        \xo{wx1}{xd}
      \end{tikzinline}}    
  \end{gather*}
We treat read-write and write-read coherence similarly:
  \begin{gather*}
      \PR{x}{r} 
      \\
      \hbox{\begin{tikzinline}[node distance=.5em and 1.5em]
        \event{rx1}{\Qw{\aLoc} \mid\DR{x}{1}}{}
        \xform{xd}{r{=}1 \limplies \bForm}{right=of rx1}
        \xform{xi}{\bForm[\FALSE/\Qr{\aLoc}]}{right=of xd}
        \xo{rx1}{xd}
      \end{tikzinline}}    
  \end{gather*}
In this model, there is no read-read coherence, but to restore it
we would identify $\Qr{\aLoc}$ with $\Qw{\aLoc}$.

When threads are initialized, we substitute every quiesence symbol
with $\TRUE$, so at top level there are no remaining quiescence
symbols, for example:
  \begin{gather*}
    \THREAD{\PW{x}{1}\SEMI \PW{x}{2}} \PAR
    \THREAD{\PR{x}{r}}
    \\
    \hbox{\begin{tikzinline}[node distance=.5em and 1.5em]
        \event{wx1}{1{=}1\land \TRUE \land \TRUE \mid\DW{x}{1}}{}
        \event{wx2}{2{=}2\land \TRUE \land \TRUE \mid\DW{x}{2}}{right=4em of wx1}
        \event{rx1}{\TRUE \mid \DR{x}{1}}{below right=1ex and 0em of wx1}
        \wk{wx1}{wx2}
        \rf[out=320,in=180]{wx1}{rx1}
        \wk[out=0,in=220]{rx1}{wx2}
      \end{tikzinline}}
  \end{gather*}

\begin{definition}
  Let $\QS{\aLoc}{}$ and $\QL{\aLoc}{}$ be defined:
  \begin{align*}
    \QS{\aLoc}{}&=\Qr{\aLoc}\land\Qw{\aLoc}
    &\QL{\aLoc}{}&=\Qw{\aLoc}
  \end{align*}
\end{definition}

\begin{definition}
  Let $[\TRUE/\Q{}]$ be the substitution that replaces all quiescence
  symbols by $\TRUE$.
\end{definition}

\begin{definition}[$\xCO$]
  \label{def:pomsets-co}
  Update \refdef{def:pomsets-trans} to:
  \begin{enumerate}
  \item[\ref{S3})]
    $\labelingForm(\aEv)$ implies $\QS{\aLoc}{}\land\aExp{=}\aVal$,
  \item[\ref{L3})]
    $\labelingForm(\aEv)$ implies $\QL{\aLoc}{}$,
  \item[\ref{T3})]
    $\labelingForm(\aEv)$ implies $\labelingForm_1(\aEv)[\TRUE/\Q{\mSC}]$,
  \end{enumerate}
  \begin{enumerate}
  \item[\ref{S4})]
    $\aTr{\bEvs}{\bForm}$ implies $\bForm[(\Qw{\aLoc}\land\aExp{=}\aVal)/\Qw{\aLoc}]$,
  \item[\ref{S5})]
    $\aTr{\cEvs}{\bForm}$ implies $\bForm[\FALSE/\Qw{\aLoc}]$,.
  \item[\ref{L4})]
    $\aTr{\bEvs}{\bForm}$ implies $\aVal{=}\aReg\limplies\bForm$, 
  \item[\ref{L5})]
  %   $\aTr{\cEvs}{\bForm}$ implies $\lnot\Qx{\aLoc} \land((\aVal{=}\aReg\lor\aLoc{=}\aReg)\limplies\bForm)$,
  % \item[\ref{L6})]
    $\aTr{\cEvs}{\bForm}$ implies $[\FALSE/\Qr{\aLoc}]$.
    % \end{enumerate}
    % Update thread as follows
    % \begin{enumerate}
  \end{enumerate}
\end{definition}


\begin{example}
  \refdef{def:pomsets-co} enforces coherence.  Consider:
  \begin{align*}
    \begin{gathered}
      \PW{x}{1}
      \\
      \hbox{\begin{tikzinline}[node distance=.5em and 1.5em]
          \event{a1}{1{=}1\land\Qr{\aLoc}\land\Qw{\aLoc}\mid\DW{x}{1}}{}
          \xform{x1d}{\bForm[(\Qw{\aLoc}\land1{=}1)/\Qw{\aLoc}]}{below=of a1}
          \xform{x1i}{\bForm[\FALSE/\Qw{\aLoc}]}{below=of x1d}
          \xo{a1}{x1d}
        \end{tikzinline}}
    \end{gathered}
    &&
    \begin{gathered}
      \PW{x}{2}
      \\
      \hbox{\begin{tikzinline}[node distance=.5em and 1.5em]
          \event{a2}{2{=}2\land\Qr{\aLoc}\land\Qw{\aLoc}\mid\DW{x}{2}}{}
          \xform{x2d}{\bForm[(\Qw{\aLoc}\land2{=}2)/\Qw{\aLoc}]}{below=of a2}
          \xform{x2i}{\bForm[\FALSE/\Qw{\aLoc}]}{below=of x2d}
          \xo{a2}{x2d}
        \end{tikzinline}}
    \end{gathered}
  \end{align*}
  Simplifying, we have:
  \begin{align*}
    \begin{gathered}
      \PW{x}{1}
      \\
      \hbox{\begin{tikzinline}[node distance=.5em and 1.5em]
          \event{a1}{\Qr{\aLoc}\land\Qw{\aLoc}\mid\DW{x}{1}}{}
          \xform{x1d}{\bForm}{below right=of a1}
          \xform{x1i}{\bForm[\FALSE/\Qw{\aLoc}]}{below=of a1}
          \xo{a1}{x1d}
        \end{tikzinline}}
    \end{gathered}
    &&
    \begin{gathered}
      \PW{x}{2}
      \\
      \hbox{\begin{tikzinline}[node distance=.5em and 1.5em]
          \event{a2}{\Qr{\aLoc}\land\Qw{\aLoc}\mid\DW{x}{2}}{}
          \xform{x2d}{\bForm}{below left=of a2}
          \xform{x2i}{\bForm[\FALSE/\Qw{\aLoc}]}{below=of a2}
          \xo{a2}{x2d}
        \end{tikzinline}}
    \end{gathered}
  \end{align*}
  If we attempt to put these together unordered, the precondition of
  $(\DW{x}{2})$ becomes unsatisfiable:
  \begin{gather*}
    \PW{x}{1}\SEMI \PW{x}{2}
    \\
    \hbox{\begin{tikzinline}[node distance=.5em and 1.5em]
          \event{a1}{\Qr{\aLoc}\land\Qw{\aLoc}\mid\DW{x}{1}}{}
          \event{a2}{\Qr{\aLoc}\land\FALSE\mid\DW{x}{2}}{above right=of x1d}
          \xform{x1d}{\bForm}{below right=of a1}
          \xform{x2i}{\bForm[\FALSE/\Qw{\aLoc}]}{below=of a1}
          \xform{x2d}{\bForm}{below left=of a2}
          \xform{x1i}{\bForm[\FALSE/\Qw{\aLoc}]}{below=of a2}
          \xform{xii}{\bForm[\FALSE/\Qw{\aLoc}]}{below right=of a2}
          \xo{a1}{x1d}
          \xo{a2}{x2d}
          \xo{a1}{x2i}
          \xo{a2}{x1i}
        \end{tikzinline}}
  \end{gather*}
  In order to get a satisfiable precondition for $\DWP{x}{2}$, we must
  introduce order:
  \begin{gather*}
    \PW{x}{1}\SEMI \PW{x}{2}
    \\
    \hbox{\begin{tikzinline}[node distance=.5em and 1.5em]
          \event{a1}{\Qr{\aLoc}\land\Qw{\aLoc}\mid\DW{x}{1}}{}
          \xform{x1d}{\bForm}{below right=of a1}
          \xform{x2i}{\bForm[\FALSE/\Qw{\aLoc}]}{below=of a1}
          \xo{a1}{x1d}
          \event{a2}{\Qr{\aLoc}\land\Qw{\aLoc}\mid\DW{x}{2}}{above right=of x1d}
          \xform{x2d}{\bForm}{below left=of a2}
          \xform{x1i}{\bForm[\FALSE/\Qw{\aLoc}]}{below=of a2}
          \xform{xii}{\bForm[\FALSE/\Qw{\aLoc}]}{below right=of a2}
          \xo{a2}{x2d}
          \xo{a1}{x2i}
          \xo{a2}{x1i}
          \wk{a1}{a2}
        \end{tikzinline}}
  \end{gather*}
\end{example}

\begin{example}
  \ref{S4} includes the substitution $\bForm[(\Qw{\aLoc}\land\aExp{=}\aVal)/\Qw{\aLoc}]$ to ensure that
  \emph{left merges} are not quiescent.  Consider the following.
  \begin{align*}
    \begin{gathered}
      \PW{x}{1}
      \\
      \hbox{\begin{tikzinline}[node distance=.5em and 1.5em]
          \event{a1}{1{=}1\land\Qr{\aLoc}\land\Qw{\aLoc}\mid\DW{x}{1}}{}
          \xform{x1d}{\bForm[(\Qw{\aLoc}\land1{=}1)/\Qw{\aLoc}]}{below=of a1}
          \xform{x1i}{\bForm[\FALSE/\Qw{\aLoc}]}{below=of x1d}
          \xo{a1}{x1d}
        \end{tikzinline}}
    \end{gathered}
    &&
    \begin{gathered}
      \PW{x}{2}
      \\
      \hbox{\begin{tikzinline}[node distance=.5em and 1.5em]
          \event{a2}{2{=}1\land\Qr{\aLoc}\land\Qw{\aLoc}\mid\DW{x}{1}}{}
          \xform{x2d}{\bForm[(\Qw{\aLoc}\land2{=}1)/\Qw{\aLoc}]}{below=of a2}
          \xform{x2i}{\bForm[\FALSE/\Qw{\aLoc}]}{below=of x2d}
          \xo{a2}{x2d}
        \end{tikzinline}}
    \end{gathered}
  \end{align*}
  Simplifying:
  %% \begin{align*}
  %%   \begin{gathered}
  %%     \PW{x}{1}
  %%     \\
  %%     \hbox{\begin{tikzinline}[node distance=.5em and 1.5em]
  %%         \event{a1}{\Qr{\aLoc}\land\Qw{\aLoc}\mid\DW{x}{1}}{}
  %%         \xform{x1d}{\bForm}{below=of a1}
  %%         \xform{x1i}{\bForm[\FALSE/\Qw{\aLoc}]}{below=of x1d}
  %%         \xo{a1}{x1d}
  %%       \end{tikzinline}}
  %%   \end{gathered}
  %%   &&
  %%   \begin{gathered}
  %%     \PW{x}{2}
  %%     \\
  %%     \hbox{\begin{tikzinline}[node distance=.5em and 1.5em]
  %%         \event{a2}{\FALSE\mid\DW{x}{1}}{}
  %%         \xform{x2d}{\bForm[\FALSE/\Qw{\aLoc}]}{below=of a2}
  %%         \xform{x2i}{\bForm[\FALSE/\Qw{\aLoc}]}{below=of x2d}
  %%         \xo{a2}{x2d}
  %%       \end{tikzinline}}
  %%   \end{gathered}
  %% \end{align*}
  and merging the actions, we have:
  \begin{gather*}
    \PW{x}{1}\SEMI \PW{x}{2}
    \\
    \hbox{\begin{tikzinline}[node distance=.5em and 1.5em]
        \event{a1}{\Qr{\aLoc}\land\Qw{\aLoc}\mid\DW{x}{1}}{}
        \xform{x1d}{\bForm[\FALSE/\Qw{\aLoc}]}{below=of a2}
        \xform{x1i}{\bForm[\FALSE/\Qw{\aLoc}]}{right=of a2}
        \xo{a1}{x1d}
      \end{tikzinline}}
  \end{gather*}
  which is what we would hope, that the program $\PW{x}{1}\SEMI \PW{x}{2}$ 
  should only be quiescent if there is a $(\DW{x}{2})$ event.
\end{example}


% \begin{comment}
%   Read to write dependency, first separately:
%   \begin{align*}
%     \begin{gathered}
%       \PR{x}{r} 
%       \\
%       \hbox{\begin{tikzinline}[node distance=.5em and 1.5em]
%           \xform{xd}{(1{=}r)\limplies\bForm}{}
%           \xform{xi}{((x{=}r\lor1{=}r)\limplies\bForm)\land \lnot\Q{}}{below=of xd}
%           \event{a1}{\DR{x}{1}}{above=of xd}
%           \xo{a1}{xd}
%         \end{tikzinline}}    
%     \end{gathered}
%     &&
%     \begin{gathered}
%       \PW{y}{r}
%       \\
%       \hbox{\begin{tikzinline}[node distance=.5em and 1.5em]
%           \xform{xd}{\bForm\noSUB{[r/y]}\land (\Q{}\limplies r{=}1)}{}
%           \xform{xi}{\bForm\noSUB{[r/y]}\land \lnot\Q{}}{below=of xd}
%           \event{a2}{r{=}1\mid\DW{y}{1}}{above=of xd}      
%           \xo{a2}{xd}
%         \end{tikzinline}}    
%     \end{gathered}
%   \end{align*}
%   Putting these together without order:
%   \begin{gather*}
%     \PR{x}{r} \SEMI
%     \PW{y}{r}
%     \\
%     \hbox{\begin{tikzinline}[node distance=.5em and 1.5em]
%         \xform{xdi}{((1{=}r)\limplies\bForm)[r/y]\land \lnot\Q{}}{}
%         \xform{xid}{(((x{=}r\lor1{=}r)\limplies\bForm)\land \lnot\Q{})[r/y]\land (\Q{}\limplies r{=}1)}{below right=.5em and -10em of xdi}
%         \event{a1}{\DR{x}{1}}{above=of xdi}
%         \event{a2}{((x{=}r\lor1{=}r)\limplies r{=}1) \land\lnot\Q{}\mid\DW{y}{1}}{above right=2.7em and -16em of xid}
%         \xform{xdd}{((1{=}r)\limplies\bForm)[r/y]\land (\Q{}\limplies r{=}1)}{above right=.5em and -1em of a1}
%         \xform{xii}{(((x{=}r\lor1{=}r)\limplies\bForm)\land \lnot\Q{})[r/y]\land \lnot\Q{}}{below=of xid}
%         \xo{a1}{xdi}
%         \xo{a2}{xid}
%         \xo{a1}{xdd}
%         \xo{a2}{xdd}
%       \end{tikzinline}}
%   \end{gather*}
%   Note that
%   \begin{math}
%     \lnot\Q{}\land (\Q{}\limplies r{=}1)
%   \end{math}
%   simplifies to 
%   \begin{math}
%     \lnot\Q{}.
%   \end{math}
%   \begin{gather*}
%     \PR{x}{r} \SEMI
%     \PW{y}{r}
%     \\
%     \hbox{\begin{tikzinline}[node distance=.5em and 1.5em]
%         \xform{xdi}{((1{=}r)\limplies\bForm\noSUB{[r/y]})\land \lnot\Q{}}{}
%         \xform{xid}{((x{=}r\lor1{=}r)\limplies\bForm\noSUB{[r/y]})\land \lnot\Q{}}{below right=.5em and -4em of xdi}
%         \event{a1}{\DR{x}{1}}{above=of xdi}
%         \event{a2}{((x{=}r\lor1{=}r)\limplies r{=}1) \land\lnot\Q{}\mid\DW{y}{1}}{above=2.7em of xid}
%         \xform{xdd}{((1{=}r)\limplies\bForm\noSUB{[r/y]})\land (\Q{}\limplies r{=}1)}{above right=.5em and -1em of a1}
%         \xform{xii}{((x{=}r\lor1{=}r)\limplies\bForm\noSUB{[r/y]})\land \lnot\Q{}}{below=of xid}
%         \xo{a1}{xdi}
%         \xo{a2}{xid}
%         \xo{a1}{xdd}
%         \xo{a2}{xdd}
%       \end{tikzinline}}
%   \end{gather*}
%   With order:
%   \begin{gather*}
%     % \PR{x}{r} \SEMI
%     % \PW{y}{r}
%     % \\
%     \hbox{\begin{tikzinline}[node distance=.5em and 1.5em]
%         \event{a1}{\DR{x}{1}}{}
%         \event{a2}{1{=}r\limplies r{=}1\mid\DW{y}{1}}{right=of a1}
%         \po{a1}{a2}
%       \end{tikzinline}}
%   \end{gather*}
% \end{comment}



\subsection{Synchronized Access (\xRASC)}
% Write $\Q{}$ as $\Q{\mSC}$ and introduce $\Q{\mRA}$.

% $\Q{\mSC}$ implies $\Q{\mRA}$.

\begin{example}
  \label{ex:pub2}
  The publication idiom requires that we disallow the execution below, which is
  allowed by \refdef{def:pomsets-co}.
  \begin{gather*}
    \begin{gathered}
      x\GETS0\SEMI %y\GETS0\SEMI
      x\GETS 1\SEMI y^\mRA\GETS1 \PAR r\GETS y^\mRA\SEMI s\GETS x
      \\[-.4ex]
      \nonumber
      \hbox{\begin{tikzinline}[node distance=1.5em]
          \event{wx0}{\DW{x}{0}}{}
          \event{wx1}{\DW{x}{1}}{right=of wx0}
          \raevent{wy1}{\DW[\mRA]{y}{1}}{right=of wx1}
          \raevent{ry1}{\DR[\mRA]{y}{1}}{right=2.5em of wy1}
          \event{rx0}{\DR{x}{0}}{right=of ry1}
          % \sync{wx1}{wy1}
          % \sync{ry1}{rx0}
          \rf{wy1}{ry1}
          \wk[in=-15,out=-165]{rx0}{wx1}
          \wk{wx0}{wx1}
          \rf[out=15,in=165]{wx0}{rx0}
        \end{tikzinline}}
    \end{gathered}
  \end{gather*}
  We disallow this by introducing order
  $\DWP{x}{1}\xsync \DWP[\mRA]{y}{1}$
  and
  $\DRP[\mRA]{y}{1}\xsync \DRP{x}{0}$.
  \begin{gather*}
    \hbox{\begin{tikzinline}[node distance=1.5em]
        \event{wx0}{\DW{x}{0}}{}
        \event{wx1}{\DW{x}{1}}{right=of wx0}
        \raevent{wy1}{\DW[\mRA]{y}{1}}{right=of wx1}
        \raevent{ry1}{\DR[\mRA]{y}{1}}{right=2.5em of wy1}
        \event{rx0}{\DR{x}{0}}{right=of ry1}
        \sync{wx1}{wy1}
        \sync{ry1}{rx0}
        \rf{wy1}{ry1}
        \wk[in=-15,out=-165]{rx0}{wx1}
        \wk{wx0}{wx1}
        % \rf[out=15,in=165]{wx0}{rx0}
      \end{tikzinline}}  
  \end{gather*}
\end{example}

In order to describe $\mRA$/$\mSC$ access, we use the uninterpreted logical
symbols $\Q{\mRA}$ and $\Q{\mSC}$, with the interpretation that
\begin{math}
  \Q{\mSC} \textimplies \Q{\mRA} \textimplies \Qx{\aLoc} \textimplies \Qw{\bLoc} \textwhen \aLoc=\bLoc.
\end{math}

\begin{definition}
  Let $\QS{\aLoc}{\amode}$ and $\QL{\aLoc}{\amode}$ be defined:
  \begin{align*}
    \QS{\aLoc}{\mRLX}&=\Qr{\aLoc}\land\Qw{\aLoc}
    &\QL{\aLoc}{\mRLX}&=\Qw{\aLoc}
    \\
    \QS{\aLoc}{\mRA}&=\textstyle\bigwedge_\bLoc \Qr{\bLoc}\land\Qw{\bLoc}
    &\QL{\aLoc}{\mRA}&=\Qw{\aLoc}
    \\
    \QS{\aLoc}{\mSC}&=\textstyle\bigwedge_\bLoc \Qr{\bLoc}\land\Qw{\bLoc} \land \Qsc
    &\QL{\aLoc}{\mSC}&=\Qw{\aLoc}\land\Qsc
  \end{align*}
\end{definition}

\begin{definition}[$\xCO$/$\xRASC$]
  \label{def:QS}
  \label{def:pomsets-ra}
  Update \refdef{def:pomsets-co} to:
  \begin{enumerate}
  \item[\ref{S3})]
    $\labelingForm(\aEv)$ implies $\QS{\aLoc}{\amode}\land\aExp{=}\aVal$,
  \item[\ref{L3})]
    $\labelingForm(\aEv)$ implies $\QL{\aLoc}{\amode}$.
  \end{enumerate}
\end{definition}



% $\QS{\aLoc}{\mRLX}=\Qx{\aLoc}$ and otherwise $\QS{\aLoc}{\amode}=\Q{\amode}$.

% $\QL{\aLoc}{\mSC}=\Q{\mSC}$ and otherwise $\QL{\aLoc}{\amode}=\Qw{\aLoc}$.

% $\DS{\aLoc}{\mRLX}{\bForm}=\bForm[\TRUE/\Dx{\aLoc}]$ and otherwise
% $\DS{\aLoc}{\amode}{\bForm}=\bForm[\FALSE/\D]$. 

% $\DL{\aLoc}{\mRLX}=\TRUE$ and otherwise $\DL{\aLoc}{\amode}=\Dx{\aLoc}$.

% \begin{definition}$\phantom{\;}$\par
%   \noindent
%   If $\aPS \in \sSTORE[\amode]{\aLoc}{\aExp}$ then
%   \begin{enumerate}
%   \item[\ref{S1}--\ref{S2})] as before,
%   \item[\ref{S3})]
%     $\labelingForm(\aEv)$ implies
%     \begin{math}
%       \aExp{=}\aVal
%       \land \RW
%       \land \QS{\aLoc}{\amode}
%     \end{math},
%   \item[\ref{S4})]
%     $\aTr{\bEvs}{\bForm}$ implies 
%     \begin{math}
%       (\Qw{\aLoc} \limplies \aExp{=}\aVal)
%       \land \DS{\aLoc}{\amode}{\bForm[\aExp/{\aLoc}]}
%     \end{math},
%   \item[\ref{S5})]
%     $\aTr{\emptyset}{\bForm}$ implies 
%     \begin{math}
%       \lnot\Qw{\aLoc}
%       \land \DS{\aLoc}{\amode}{\bForm[\aExp/{\aLoc}]}.
%     \end{math}
%   \end{enumerate}

%   \noindent
%   If $\aPS \in \sLOAD[\amode]{\aReg}{\aLoc}$ then
%   \begin{enumerate}
%   \item[\ref{L1}--\ref{L2})] as before,
%   \item[\ref{L3})]
%     $\labelingForm(\aEv)$ implies
%     \begin{math}
%       \RO
%       \land \QL{\aLoc}{\amode}
%     \end{math},
%   \item[\ref{L4})]
%     $\aTr{\bEvs}{\bForm}$ implies
%     \begin{math}
%       (\aVal{=}\aReg)
%       \limplies \bForm[\aReg/{\aLoc}]
%     \end{math}
%   \item[\ref{L5})] 
%     $\aTr{\emptyset}{\bForm}$ implies
%     \begin{math}
%       \DL{\aLoc}{\amode}
%       \land \lnot\Qx{\aLoc}
%       %     \end{math}
%       %     \\
%       %     \begin{math}
%       %       {}
%       \land 
%       (\RW
%       \limplies (\aVal{=}\aReg\lor\aLoc{=}\aReg) 
%       \limplies \bForm[\aReg/{\aLoc}]
%       ).
%     \end{math}
%   \end{enumerate}  
% \end{definition}


\subsection{Completed Pomsets}

\begin{definition}
  A pomset with predicate transformers $\aPS$ is \emph{completed} if
  $\aTr{\aEvs}{\Q{\mSC}}$ implies $\Q{\mSC}$.
\end{definition}


