\section{Quiescence}
\label{sec:q}

We introduce \emph{quiescence}, which captures \emph{coherence} and
\emph{synchronized access}.  Recall from
\textsection\ref{sec:prelim} that formulae include
symbols $\Q{\mSC}$, $\Qr{\aLoc}$, and $\Qw{\aLoc}$.
We refer to these collectively as \emph{quiescence symbols}.
In this section, we will show how these logical symbols can be used
to capture coherence and synchronization. This illustrates a feature
of our model, which is that many features of weak memory can be cautured
in the logic, not in the pomset model itself.

\subsection{Coherence (\xCO)}
\label{sec:co}

In the logic, the quiescence symbols are just uninterpreted formula, but
the semantics uses them as preconditions, to ensure appropriate causal
order.
For example, \emph{write-write coherence} enforces order
between writes to the same location in the same thread. We
model this by adding the precondition
$(\Qr{\aLoc}\land\Qw{\aLoc})$ to 
events that write to $x$, for example:
  \begin{gather*}
    \PW{x}{1}\SEMI \PW{x}{2}
    \\
    \hbox{\begin{tikzinline}[node distance=.5em and 1.5em]
        \event{wx1}{1{=}1\land \Qr{\aLoc}\land\Qw{\aLoc} \mid\DW{x}{1}}{}
        \event{wx2}{2{=}2\land \Qr{\aLoc}\land\Qw{\aLoc} \mid\DW{x}{2}}{right=of wx1}
        \wk{wx1}{wx2}
      \end{tikzinline}}
  \end{gather*}
These symbols are left alone in the dependent case, but
in the independent case we substitute $\FALSE$ for $\Qw{\aLoc}$:
  \begin{gather*}
    % \PW{x}{1}\SEMI \PW{x}{2}
    % \\
    \hbox{\begin{tikzinline}[node distance=.5em and 1.5em]
        \event{wx1}{1{=}1\land \Qr{\aLoc}\land\Qw{\aLoc} \mid\DW{x}{1}}{}
        \event{wx2}{2{=}2\land \Qr{\aLoc}\land\FALSE \mid\DW{x}{2}}{right=of wx1}
      \end{tikzinline}}
  \end{gather*}
This substitution is part of the predicate transformer:
  \begin{gather*}
      \PW{x}{1} 
      \\
      \hbox{\begin{tikzinline}[node distance=.5em and 1.5em]
        \event{wx1}{1{=}1\land \Qr{\aLoc}\land\Qw{\aLoc} \mid\DW{x}{1}}{}
        \xform{xd}{\bForm}{right=of wx1}
        \xform{xi}{\bForm[\FALSE/\Qw{\aLoc}]}{right=of xd}
        \xos{wx1}{xd}
      \end{tikzinline}}    
  \end{gather*}
We treat read-write and write-read coherence similarly:
  \begin{gather*}
      \PR{x}{r} 
      \\
      \hbox{\begin{tikzinline}[node distance=.5em and 1.5em]
        \event{rx1}{\Qw{\aLoc} \mid\DR{x}{1}}{}
        \xform{xd}{r{=}1 \limplies \bForm}{right=of rx1}
        \xform{xi}{\bForm[\FALSE/\Qr{\aLoc}]}{right=of xd}
        \xos{rx1}{xd}
      \end{tikzinline}}    
  \end{gather*}
In this model, there is no read-read coherence, but to restore it
we would identify $\Qr{\aLoc}$ with $\Qw{\aLoc}$.

When threads are initialized, we substitute every quiesence symbol
with $\TRUE$, so at top level there are no remaining quiescence
symbols, for example:
  \begin{gather*}
    \THREAD{\PW{x}{1}\SEMI \PW{x}{2}} \PAR
    \THREAD{\PR{x}{r}}
    \\
    \hbox{\begin{tikzinline}[node distance=.5em and 1.5em]
        \event{wx1}{1{=}1\land \TRUE \land \TRUE \mid\DW{x}{1}}{}
        \event{wx2}{2{=}2\land \TRUE \land \TRUE \mid\DW{x}{2}}{right=4em of wx1}
        \event{rx1}{\TRUE \mid \DR{x}{1}}{below right=1ex and 0em of wx1}
        \wk{wx1}{wx2}
        \rf[out=320,in=180]{wx1}{rx1}
        \wk[out=0,in=220]{rx1}{wx2}
      \end{tikzinline}}
  \end{gather*}

\begin{definition}[\xCO]
  \label{def:pomsets-co}
  Update \refdef{def:pomsets-trans}: % to (\ref{L4} unchanged):
  \begin{enumerate}
  \item[\ref{S3})]
    $\labelingForm(\aEv)$ implies $\Qr{\aLoc}\land\Qw{\aLoc}\land\aExp{=}\aVal$,
  \item[\ref{L3})]
    $\labelingForm(\aEv)$ implies $\Qw{\aLoc}$,
  % \item[\ref{T3})]
  %   $\labelingForm(\aEv)$ implies $\labelingForm_1(\aEv)[\TRUE/\Q{}]$,
  \end{enumerate}
  \begin{enumerate}
  % \item[\ref{S4})]
  %   $\aTr{\bEvs}{\bForm}$ implies $\aExp{=}\aVal\land\bForm$,
  \item[\ref{S5})]
    $\aTr{\cEvs}{\bForm}$ implies $\bForm[\FALSE/\Qw{\aLoc}]$,
  % \item[\ref{L4})]
  %   $\aTr{\bEvs}{\bForm}$ implies $\aVal{=}\aReg\limplies\bForm$, 
  \item[\ref{L5})]
  %   $\aTr{\cEvs}{\bForm}$ implies $\lnot\Qx{\aLoc} \land((\aVal{=}\aReg\lor\aLoc{=}\aReg)\limplies\bForm)$,
  % \item[\ref{L6})]
    $\aTr{\cEvs}{\bForm}$ implies $\bForm[\FALSE/\Qr{\aLoc}]$.
    % \end{enumerate}
    % Update thread as follows
    % \begin{enumerate}
  \end{enumerate}
\end{definition}

\begin{example}
  \refdef{def:pomsets-co} enforces coherence.  Consider:
  \begin{align*}
    \begin{gathered}
      \PW{x}{1}
      \\
      \hbox{\begin{tikzinline}[node distance=.5em and 1.5em]
          \event{a1}{1{=}1\land\Qr{\aLoc}\land\Qw{\aLoc}\mid\DW{x}{1}}{}
          \xform{x1d}{\bForm\land1{=}1}{below=of a1}
          \xform{x1i}{\bForm[\FALSE/\Qw{\aLoc}]}{below=of x1d}
          \xo{a1}{x1d}
        \end{tikzinline}}
    \end{gathered}
    &&
    \begin{gathered}
      \PW{x}{2}
      \\
      \hbox{\begin{tikzinline}[node distance=.5em and 1.5em]
          \event{a2}{2{=}2\land\Qr{\aLoc}\land\Qw{\aLoc}\mid\DW{x}{2}}{}
          \xform{x2d}{\bForm\land2{=}2}{below=of a2}
          \xform{x2i}{\bForm[\FALSE/\Qw{\aLoc}]}{below=of x2d}
          \xo{a2}{x2d}
        \end{tikzinline}}
    \end{gathered}
  \end{align*}
  Simplifying, we have:
  \begin{align*}
    \begin{gathered}
      % \PW{x}{1}
      % \\
      \hbox{\begin{tikzinline}[node distance=.5em and 1.5em]
          \event{a1}{\Qr{\aLoc}\land\Qw{\aLoc}\mid\DW{x}{1}}{}
          \xform{x1d}{\bForm}{below right=of a1}
          \xform{x1i}{\bForm[\FALSE/\Qw{\aLoc}]}{below=of a1}
          \xos{a1}{x1d}
        \end{tikzinline}}
    \end{gathered}
    &&
    \begin{gathered}
      % \PW{x}{2}
      % \\
      \hbox{\begin{tikzinline}[node distance=.5em and 1.5em]
          \event{a2}{\Qr{\aLoc}\land\Qw{\aLoc}\mid\DW{x}{2}}{}
          \xform{x2d}{\bForm}{below left=of a2}
          \xform{x2i}{\bForm[\FALSE/\Qw{\aLoc}]}{below=of a2}
          \xos{a2}{x2d}
        \end{tikzinline}}
    \end{gathered}
  \end{align*}
  If we attempt to put these together unordered, the precondition of
  $(\DW{x}{2})$ becomes unsatisfiable:
  \begin{gather*}
    \PW{x}{1}\SEMI \PW{x}{2}
    \\
    \hbox{\begin{tikzinline}[node distance=.5em and 1.5em]
          \event{a1}{\Qr{\aLoc}\land\Qw{\aLoc}\mid\DW{x}{1}}{}
          \event{a2}{\Qr{\aLoc}\land\FALSE\mid\DW{x}{2}}{above right=of x1d}
          \xform{x1d}{\bForm}{below right=of a1}
          \xform{x2i}{\bForm[\FALSE/\Qw{\aLoc}]}{below=of a1}
          %\xform{x2d}{\bForm}{below left=of a2}
          \xform{x1i}{\bForm[\FALSE/\Qw{\aLoc}]}{below=of a2}
          \xform{xii}{\bForm[\FALSE/\Qw{\aLoc}]}{below right=of a2}
          \xos{a1}{x1d}
          \xos{a2}{x1d}
          \xo[xleft]{a1}{x2i}
          \xo{a2}{x1i}
        \end{tikzinline}}
  \end{gather*}
  In order to get a satisfiable precondition for $\DWP{x}{2}$, we must
  introduce order:
  \begin{gather*}
    % \PW{x}{1}\SEMI \PW{x}{2}
    % \\
    \hbox{\begin{tikzinline}[node distance=.5em and 1.5em]
          \event{a1}{\Qr{\aLoc}\land\Qw{\aLoc}\mid\DW{x}{1}}{}
          \xform{x1d}{\bForm}{below right=of a1}
          \xform{x2i}{\bForm[\FALSE/\Qw{\aLoc}]}{below=of a1}
          \event{a2}{\Qr{\aLoc}\land\Qw{\aLoc}\mid\DW{x}{2}}{above right=of x1d}
          %\xform{x2d}{\bForm}{below left=of a2}
          \xform{x1i}{\bForm[\FALSE/\Qw{\aLoc}]}{below=of a2}
          \xform{xii}{\bForm[\FALSE/\Qw{\aLoc}]}{below right=of a2}
          \xos{a1}{x1d}
          \xos{a2}{x1d}
          \xo[xleft]{a1}{x2i}
          \xo{a2}{x1i}
          \wk{a1}{a2}
        \end{tikzinline}}
  \end{gather*}
\end{example}

% \begin{example}
%   \label{ex:left-merge}
%   \ref{S4} includes the substitution $\bForm[(\Qw{\aLoc}\land\aExp{=}\aVal)/\Qw{\aLoc}]$ to ensure that
%   \emph{left merges} are not quiescent.  Consider the following.
%   \begin{align*}
%     \begin{gathered}
%       \PW{x}{1}
%       \\
%       \hbox{\begin{tikzinline}[node distance=.5em and 1.5em]
%           \event{a1}{1{=}1\land\Qr{\aLoc}\land\Qw{\aLoc}\mid\DW{x}{1}}{}
%           \xform{x1d}{\bForm[(\Qw{\aLoc}\land1{=}1)/\Qw{\aLoc}]}{below=of a1}
%           \xform{x1i}{\bForm[\FALSE/\Qw{\aLoc}]}{below=of x1d}
%           \xo{a1}{x1d}
%         \end{tikzinline}}
%     \end{gathered}
%     &&
%     \begin{gathered}
%       \PW{x}{2}
%       \\
%       \hbox{\begin{tikzinline}[node distance=.5em and 1.5em]
%           \event{a2}{2{=}1\land\Qr{\aLoc}\land\Qw{\aLoc}\mid\DW{x}{1}}{}
%           \xform{x2d}{\bForm[(\Qw{\aLoc}\land2{=}1)/\Qw{\aLoc}]}{below=of a2}
%           \xform{x2i}{\bForm[\FALSE/\Qw{\aLoc}]}{below=of x2d}
%           \xo{a2}{x2d}
%         \end{tikzinline}}
%     \end{gathered}
%   \end{align*}
%   % Simplifying
%   %% \begin{align*}
%   %%   \begin{gathered}
%   %%     \PW{x}{1}
%   %%     \\
%   %%     \hbox{\begin{tikzinline}[node distance=.5em and 1.5em]
%   %%         \event{a1}{\Qr{\aLoc}\land\Qw{\aLoc}\mid\DW{x}{1}}{}
%   %%         \xform{x1d}{\bForm}{below=of a1}
%   %%         \xform{x1i}{\bForm[\FALSE/\Qw{\aLoc}]}{below=of x1d}
%   %%         \xo{a1}{x1d}
%   %%       \end{tikzinline}}
%   %%   \end{gathered}
%   %%   &&
%   %%   \begin{gathered}
%   %%     \PW{x}{2}
%   %%     \\
%   %%     \hbox{\begin{tikzinline}[node distance=.5em and 1.5em]
%   %%         \event{a2}{\FALSE\mid\DW{x}{1}}{}
%   %%         \xform{x2d}{\bForm[\FALSE/\Qw{\aLoc}]}{below=of a2}
%   %%         \xform{x2i}{\bForm[\FALSE/\Qw{\aLoc}]}{below=of x2d}
%   %%         \xo{a2}{x2d}
%   %%       \end{tikzinline}}
%   %%   \end{gathered}
%   %% \end{align*}
%   Merging the actions, since $2{=}1$ is unsatisfiable, we have:
%   \begin{gather*}
%     \PW{x}{1}\SEMI \PW{x}{2}
%     \\
%     \hbox{\begin{tikzinline}[node distance=.5em and 1.5em]
%         \event{a1}{\Qr{\aLoc}\land\Qw{\aLoc}\mid\DW{x}{1}}{}
%         \xform{x1d}{\bForm[\FALSE/\Qw{\aLoc}]}{right=of a2}
%         \xform{x1i}{\bForm[\FALSE/\Qw{\aLoc}]}{right=of x1d}
%         \xos{a1}{x1d}
%       \end{tikzinline}}
%   \end{gather*}
%   This is what we would hope: that the program $\PW{x}{1}\SEMI \PW{x}{2}$ 
%   should only be quiescent if there is a $(\DW{x}{2})$ event.
% \end{example}


% \begin{comment}
%   Read to write dependency, first separately:
%   \begin{align*}
%     \begin{gathered}
%       \PR{x}{r} 
%       \\
%       \hbox{\begin{tikzinline}[node distance=.5em and 1.5em]
%           \xform{xd}{(1{=}r)\limplies\bForm}{}
%           \xform{xi}{((x{=}r\lor1{=}r)\limplies\bForm)\land \lnot\Q{}}{below=of xd}
%           \event{a1}{\DR{x}{1}}{above=of xd}
%           \xo{a1}{xd}
%         \end{tikzinline}}    
%     \end{gathered}
%     &&
%     \begin{gathered}
%       \PW{y}{r}
%       \\
%       \hbox{\begin{tikzinline}[node distance=.5em and 1.5em]
%           \xform{xd}{\bForm\noSUB{[r/y]}\land (\Q{}\limplies r{=}1)}{}
%           \xform{xi}{\bForm\noSUB{[r/y]}\land \lnot\Q{}}{below=of xd}
%           \event{a2}{r{=}1\mid\DW{y}{1}}{above=of xd}      
%           \xo{a2}{xd}
%         \end{tikzinline}}    
%     \end{gathered}
%   \end{align*}
%   Putting these together without order:
%   \begin{gather*}
%     \PR{x}{r} \SEMI
%     \PW{y}{r}
%     \\
%     \hbox{\begin{tikzinline}[node distance=.5em and 1.5em]
%         \xform{xdi}{((1{=}r)\limplies\bForm)[r/y]\land \lnot\Q{}}{}
%         \xform{xid}{(((x{=}r\lor1{=}r)\limplies\bForm)\land \lnot\Q{})[r/y]\land (\Q{}\limplies r{=}1)}{below right=.5em and -10em of xdi}
%         \event{a1}{\DR{x}{1}}{above=of xdi}
%         \event{a2}{((x{=}r\lor1{=}r)\limplies r{=}1) \land\lnot\Q{}\mid\DW{y}{1}}{above right=2.7em and -16em of xid}
%         \xform{xdd}{((1{=}r)\limplies\bForm)[r/y]\land (\Q{}\limplies r{=}1)}{above right=.5em and -1em of a1}
%         \xform{xii}{(((x{=}r\lor1{=}r)\limplies\bForm)\land \lnot\Q{})[r/y]\land \lnot\Q{}}{below=of xid}
%         \xo{a1}{xdi}
%         \xo{a2}{xid}
%         \xo{a1}{xdd}
%         \xo{a2}{xdd}
%       \end{tikzinline}}
%   \end{gather*}
%   Note that
%   \begin{math}
%     \lnot\Q{}\land (\Q{}\limplies r{=}1)
%   \end{math}
%   simplifies to 
%   \begin{math}
%     \lnot\Q{}.
%   \end{math}
%   \begin{gather*}
%     \PR{x}{r} \SEMI
%     \PW{y}{r}
%     \\
%     \hbox{\begin{tikzinline}[node distance=.5em and 1.5em]
%         \xform{xdi}{((1{=}r)\limplies\bForm\noSUB{[r/y]})\land \lnot\Q{}}{}
%         \xform{xid}{((x{=}r\lor1{=}r)\limplies\bForm\noSUB{[r/y]})\land \lnot\Q{}}{below right=.5em and -4em of xdi}
%         \event{a1}{\DR{x}{1}}{above=of xdi}
%         \event{a2}{((x{=}r\lor1{=}r)\limplies r{=}1) \land\lnot\Q{}\mid\DW{y}{1}}{above=2.7em of xid}
%         \xform{xdd}{((1{=}r)\limplies\bForm\noSUB{[r/y]})\land (\Q{}\limplies r{=}1)}{above right=.5em and -1em of a1}
%         \xform{xii}{((x{=}r\lor1{=}r)\limplies\bForm\noSUB{[r/y]})\land \lnot\Q{}}{below=of xid}
%         \xo{a1}{xdi}
%         \xo{a2}{xid}
%         \xo{a1}{xdd}
%         \xo{a2}{xdd}
%       \end{tikzinline}}
%   \end{gather*}
%   With order:
%   \begin{gather*}
%     % \PR{x}{r} \SEMI
%     % \PW{y}{r}
%     % \\
%     \hbox{\begin{tikzinline}[node distance=.5em and 1.5em]
%         \event{a1}{\DR{x}{1}}{}
%         \event{a2}{1{=}r\limplies r{=}1\mid\DW{y}{1}}{right=of a1}
%         \po{a1}{a2}
%       \end{tikzinline}}
%   \end{gather*}
% \end{comment}



\subsection{Synchronized Access (\xRASC)}
\label{sec:sync}
% Write $\Q{}$ as $\Q{\mSC}$ and introduce $\Q{\mRA}$.

% $\Q{\mSC}$ implies $\Q{\mRA}$.

\begin{example}
  \label{ex:pub2}
  The publication idiom requires that we disallow the execution below, which is
  allowed by \refdef{def:pomsets-co}.
  \begin{gather*}
    \begin{gathered}
      x\GETS0\SEMI %y\GETS0\SEMI
      x\GETS 1\SEMI y^\mRA\GETS1 \PAR r\GETS y^\mRA\SEMI s\GETS x
      \\[-.4ex]
      \nonumber
      \hbox{\begin{tikzinline}[node distance=1.5em]
          \event{wx0}{\DW{x}{0}}{}
          \event{wx1}{\DW{x}{1}}{right=of wx0}
          \raevent{wy1}{\DW[\mRA]{y}{1}}{right=of wx1}
          \raevent{ry1}{\DR[\mRA]{y}{1}}{right=2.5em of wy1}
          \event{rx0}{\DR{x}{0}}{right=of ry1}
          % \sync{wx1}{wy1}
          % \sync{ry1}{rx0}
          \rf{wy1}{ry1}
          \wk[in=-15,out=-165]{rx0}{wx1}
          \wk{wx0}{wx1}
          \rf[out=15,in=165]{wx0}{rx0}
        \end{tikzinline}}
    \end{gathered}
  \end{gather*}
  We disallow this by introducing order
  $\DWP{x}{1}\xsync \DWP[\mRA]{y}{1}$
  and
  $\DRP[\mRA]{y}{1}\xsync \DRP{x}{0}$.
  \begin{gather*}
    \hbox{\begin{tikzinline}[node distance=1.5em]
        \event{wx0}{\DW{x}{0}}{}
        \event{wx1}{\DW{x}{1}}{right=of wx0}
        \raevent{wy1}{\DW[\mRA]{y}{1}}{right=of wx1}
        \raevent{ry1}{\DR[\mRA]{y}{1}}{right=2.5em of wy1}
        \event{rx0}{\DR{x}{0}}{right=of ry1}
        \sync{wx1}{wy1}
        \sync{ry1}{rx0}
        \rf{wy1}{ry1}
        \wk[in=-15,out=-165]{rx0}{wx1}
        \wk{wx0}{wx1}
        % \rf[out=15,in=165]{wx0}{rx0}
      \end{tikzinline}}  
  \end{gather*}
\end{example}

%% In order to describe $\mRA$/$\mSC$ access, we use the uninterpreted logical
%% symbols $\Q{\mRA}$ and $\Q{\mSC}$, with the interpretation that
%% \begin{math}
%%   \Q{\mSC} \textimplies \Q{\mRA} \textimplies \Qx{\aLoc} \textimplies \Qw{\bLoc} \textwhen \aLoc=\bLoc.
%% \end{math}
 
\begin{definition}
  \label{def:QS}
  Let $\Qr{*}=\textstyle\bigwedge_\bLoc \Qr{\bLoc}$, and similarly for $\Qw{*}$.\\
  Let formulae $\QS{\aLoc}{\amode}$ and $\QL{\aLoc}{\amode}$ be defined:
  \begin{scope}
    \small
    \begin{align*}
      \QS{\aLoc}{\mRLX}&=\Qr{\aLoc}\land\Qw{\aLoc}
      &\QL{\aLoc}{\mRLX}&=\Qw{\aLoc}
      \\
      \QS{\aLoc}{\mRA}&=
      \Qr{*}\land\Qw{*} %\textstyle\bigwedge_\bLoc \Qr{\bLoc}\land\Qw{\bLoc}
      &\QL{\aLoc}{\mRA}&=\Qw{\aLoc}
      \\
      \QS{\aLoc}{\mSC}&=
      \Qr{*}\land\Qw{*} %\textstyle\bigwedge_\bLoc \Qr{\bLoc}\land\Qw{\bLoc}
      \land \Qsc
      &\QL{\aLoc}{\mSC}&=\Qw{\aLoc}\land\Qsc
    \end{align*}
  \end{scope}
  % \end{definition}
% \begin{definition}
  Let $[\aForm/\Qr{*}]$ substitute $\aForm$ for every $\Qr{\bLoc}$, and similarly for $\Qw{*}$.
  Let substitutions $[\aForm/\QS{\aLoc}{\amode}]$ and  $[\aForm/\QL{\aLoc}{\amode}]$ be defined:
  \begin{scope}
    \small
    \begin{align*}
      [\aForm/\QS{\aLoc}{\mRLX}] &= [\aForm/\Qw{\aLoc}]
      &{} [\aForm/\QL{\aLoc}{\mRLX}] &= [\aForm/\Qr{\aLoc}]
      \\
      [\aForm/\QS{\aLoc}{\mRA}] &= [\aForm/\Qw{\aLoc}]
      &{} [\aForm/\QL{\aLoc}{\mRA}] &= [\aForm/\Qr{*},\aForm/\Qw{*}]
      \\
      [\aForm/\QS{\aLoc}{\mSC}] &= [\aForm/\Qw{\aLoc},\aForm/\Qsc]
      &{} [\aForm/\QL{\aLoc}{\mSC}] &= [\aForm/\Qr{*},\aForm/\Qw{*},\aForm/\Qsc]
    \end{align*}
  \end{scope}
\end{definition}

\begin{definition}[\xCO/\xRASC]
  \label{def:pomsets-ra}
  Update \refdef{def:pomsets-trans} to: % (\ref{S4}/\ref{L4} unchanged):
  \begin{enumerate}
  \item[\ref{S3})]
    $\labelingForm(\aEv)$ implies $\QS{\aLoc}{\amode}\land\aExp{=}\aVal$,
  \item[\ref{L3})]
    $\labelingForm(\aEv)$ implies $\QL{\aLoc}{\amode}$,
  % \item[\ref{T3})]
  %   $\labelingForm(\aEv)$ implies $\labelingForm_1(\aEv)[\TRUE/\Q{}]$,
  \end{enumerate}
  \begin{enumerate}
  % \item[\ref{S4})]
  %   $\aTr{\bEvs}{\bForm}$ implies $\aExp{=}\aVal\land\bForm$,
  %  %$\aTr{\bEvs}{\bForm}$ implies $\bForm[(\Qw{\aLoc}\land\aExp{=}\aVal)/\Qw{\aLoc}]$, 
  \item[\ref{S5})]
    $\aTr{\cEvs}{\bForm}$ implies $\bForm[\FALSE/\QS{\aLoc}{\amode}]$,
  % \item[\ref{L4})]
  %   $\aTr{\bEvs}{\bForm}$ implies $\aVal{=}\aReg\limplies\bForm$, 
  \item[\ref{L5})]
    $\aTr{\cEvs}{\bForm}$ implies $\bForm[\FALSE/\QL{\aLoc}{\amode}]$.
  \end{enumerate}
\end{definition}

The quiescence formulae indicate what must precede an event.
For example, all preceding accesses must be ordered before a releasing write,
whereas only writes on $x$ must be ordered before a releasing read on $x$.

The quiescence substitutions update quiescence symbols in subsequent code.
For subsequent independent code, $\ref{S5}$ and $\ref{L5}$ substitute false.
In top-level pomsets, $\ref{sTOP}$ %$\ref{T3}$
substitutes true (\refdef{def:pomsets-top}).
%
For example, we substitute $\FALSE$ for $\QS{\aLoc}{\mRA}$ in the independent
case for a releasing write; this ensures that subsequent writes to $\aLoc$
follow the releasing write in top-level pomsets.  Similarly, we substitute
$\FALSE$ for $\QL{\aLoc}{\mRA}$ in the independent case for an acquiring
write; this ensures that all subsequent accesses follow the acquiring read in
top-level pomsets.

\begin{example}
  \label{ex:q1}
  The following pomsets show the effect of quiescence for each access mode.
  \begin{align*}
  \begin{gathered}
    \begin{gathered}[t]
      \PW{x}{\aExp}
      \\
      \hbox{\begin{tikzinline}[node distance=.5em and 1.5em]
          \event{a}{\aExp{=}v\land\Qr{x}\land\Qw{x}\mid\DW{x}{v}}{}
          \xform{xi}{\bForm[\FALSE/\Qw{x}]}{above=of a}
          \xform{xd}{\bForm[(\Qw{x}\land\aExp{=}\aVal)/\Qw{x}]}{below=of a}
          \xo{a}{xd}
        \end{tikzinline}}
    \end{gathered}
    \\[1ex]
    \begin{gathered}[t]
      \PW[\mRA]{x}{\aExp}
      \\
      \hbox{\begin{tikzinline}[node distance=.5em and 1.5em]
          \raevent{a}{\aExp{=}v\land\Qr{*}\land\Qw{*}\mid\DW[\mRA]{x}{v}}{}
          \xform{xi}{\bForm[\FALSE/\Qw{x}]}{above=of a}
          \xform{xd}{\bForm[(\Qw{x}\land\aExp{=}\aVal)/\Qw{x}]}{below=of a}
          \xo{a}{xd}
        \end{tikzinline}}
    \end{gathered}
    \\[1ex]
    \begin{gathered}[t]
      \PW[\mSC]{x}{\aExp}
      \\
      \hbox{\begin{tikzinline}[node distance=.5em and 1.5em]
          \scevent{a}{\aExp{=}v\land\Qr{*}\land\Qw{*}\land\Qsc\mid\DW[\mSC]{x}{v}}{}
          \xform{xi}{\bForm[\FALSE/\Qw{x}][\FALSE/\Qsc]}{above=of a}
          \xform{xd}{\bForm[(\Qw{x}\land\aExp{=}\aVal)/\Qw{x}]}{below=of a}
          \xo{a}{xd}
        \end{tikzinline}}
    \end{gathered}
  \end{gathered}
  &&
  \begin{gathered}
    \begin{gathered}[t]
      \PR{x}{r}
      \\
      \hbox{\begin{tikzinline}[node distance=.5em and 1.5em]
          \event{a}{\Qw{x}\mid\DR{x}{v}}{}
          \xform{xi}{\bForm[\FALSE/\Qr{x}]}{above=of a}
          \xform{xd}{v{=}r\limplies\bForm}{below=of a}
          \xo{a}{xd}
        \end{tikzinline}}
    \end{gathered}
    \\[1ex]
    \begin{gathered}[t]
      \PR[\mRA]{x}{r}
      \\
      \hbox{\begin{tikzinline}[node distance=.5em and 1.5em]
          \raevent{a}{\Qw{x}\mid\DR[\mRA]{x}{v}}{}
          \xform{xi}{\bForm[\FALSE/\Qr{*}][\FALSE/\Qw{*}]}{above=of a}
          \xform{xd}{v{=}r\limplies\bForm}{below=of a}
          \xo{a}{xd}
        \end{tikzinline}}
    \end{gathered}
    \\[1ex]
    \begin{gathered}[t]
      \PR[\mSC]{x}{r}
      \\
      \hbox{\begin{tikzinline}[node distance=.5em and 1.5em]
          \scevent{a}{\Qw{x}\land\Qsc\mid\DR[\mSC]{x}{v}}{}
          \xform{xi}{\bForm[\FALSE/\Qr{*}][\FALSE/\Qw{*}][\FALSE/\Qsc]}{above=of a}
          \xform{xd}{v{=}r\limplies\bForm}{below=of a}
          \xo{a}{xd}
        \end{tikzinline}}
    \end{gathered}
  \end{gathered}
\end{align*}
\end{example}

\begin{example}
  \refdef{def:pomsets-co} enforces publication.  Consider:
  \begin{align*}
    \begin{gathered}
      \PW{x}{1}
      \\
      \hbox{\begin{tikzinline}[node distance=.5em and 1.5em]
          \event{a1}{1{=}1\land\QS{\aLoc}{\mRLX}\mid\DW{x}{1}}{}
          \xform{x1d}{1{=}1\land\bForm}{below=of a1}
          \xform{x1i}{\bForm[\FALSE/\QS{\aLoc}{\mRLX}]}{below=of x1d}
          \xo{a1}{x1d}
        \end{tikzinline}}
    \end{gathered}
    &&
    \begin{gathered}
      \PW[\mRA]{y}{1}
      \\
      \hbox{\begin{tikzinline}[node distance=.5em and 1.5em]
          \raevent{a2}{1{=}1\land\QS{\bLoc}{\mRA}\mid\DW{y}{2}}{}
          \xform{x2d}{1{=}1\land\bForm}{below=of a2}
          \xform{x2i}{\bForm[\FALSE/\QS{\bLoc}{\mRA}]}{below=of x2d}
          \xo{a2}{x2d}
        \end{tikzinline}}
    \end{gathered}
  \end{align*}
  Since $\QS{\bLoc}{\mRA}[\FALSE/\QS{\aLoc}{\mRLX}]$ is $\FALSE$,
  composing these without order simplifies to:
  \begin{gather*}
    \PW{x}{1}\SEMI \PW[\mRA]{y}{1}
    \\
    \hbox{\begin{tikzinline}[node distance=.5em and 1.5em]
          \event{a1}{\QS{\aLoc}{\mRLX}\mid\DW{x}{1}}{}
          \xform{x1d}{\bForm}{below right=of a1}
          \xform{x2i}{\bForm[\FALSE/\QS{\bLoc}{\mRA}]}{below=of a1}
          \xos{a1}{x1d}
          \raevent{a2}{\FALSE\mid\DW{\bLoc}{1}}{above right=of x1d}
          %\xform{x2d}{\bForm}{below left=of a2}
          \xform{x1i}{\bForm[\FALSE/\QS{\aLoc}{\mRLX}]}{below=of a2}
          \xform{xii}{\bForm[\FALSE/\QS{\bLoc}{\mRA}][\FALSE/\QS{\aLoc}{\mRLX}]}{below right=of a2}
          \xos{a2}{x1d}
          \xo[xleft]{a1}{x2i}
          \xo{a2}{x1i}
        \end{tikzinline}}
  \end{gather*}
  In order to get a satisfiable precondition for $\DWP{y}{1}$, we must
  introduce order:
  \begin{gather*}
    % \PW{x}{1}\SEMI \PW[\mRA]{y}{1}
    % \\
    \hbox{\begin{tikzinline}[node distance=.5em and 1.5em]
          \event{a1}{\QS{\aLoc}{\mRLX}\mid\DW{x}{1}}{}
          \xform{x1d}{\bForm}{below right=of a1}
          \xform{x2i}{\bForm[\FALSE/\QS{\bLoc}{\mRA}]}{below=of a1}
          \xos{a1}{x1d}
          \raevent{a2}{\QS{\bLoc}{\mRA}\mid\DW{\bLoc}{1}}{above right=of x1d}
          %\xform{x2d}{\bForm}{below left=of a2}
          \xform{x1i}{\bForm[\FALSE/\QS{\aLoc}{\mRLX}]}{below=of a2}
          \xform{xii}{\bForm[\FALSE/\QS{\bLoc}{\mRA}][\FALSE/\QS{\aLoc}{\mRLX}]}{below right=of a2}
          \xos{a2}{x1d}
          \xo[xleft]{a1}{x2i}
          \xo{a2}{x1i}
          \sync{a1}{a2}
        \end{tikzinline}}
  \end{gather*}
\end{example}

\begin{example}
  \label{ex:subscription}
  \refdef{def:pomsets-co} enforces subscription.  Consider:
  \begin{align*}
    \begin{gathered}
      \PR[\mRA]{y}{r}
      \\
      \hbox{\begin{tikzinline}[node distance=.5em and .5em]
          \raevent{a1}{\QL{\bLoc}{\mRA}\mid\DR{y}{1}}{}
          \xform{x1d}{r{=}1\limplies\bForm}{below=of a1}
          \xform{x1i}{\bForm[\FALSE/\QL{\bLoc}{\mRA}]}{right=of x1d}
          \xo[xleft]{a1}{x1d}
        \end{tikzinline}}
    \end{gathered}
    &&
    \begin{gathered}
      \PR{x}{s}
      \\
      \hbox{\begin{tikzinline}[node distance=.5em and .5em]
          \event{a2}{\QL{\aLoc}{\mRLX}\mid\DR{x}{1}}{}
          \xform{x2d}{s{=}1\limplies\bForm}{below=of a2}
          \xform{x2i}{\bForm[\FALSE/\QL{\aLoc}{\mRLX}]}{right=of x2d}
          \xo[xleft]{a2}{x2d}
        \end{tikzinline}}
    \end{gathered}
  \end{align*}
  Since $\QL{\aLoc}{\mRLX}[\FALSE/\QL{\bLoc}{\mRA}]$ is $\FALSE$,
  composing these without order simplifies to:
  \begin{gather*}
    \PR[\mRA]{y}{r}\SEMI \PR{x}{s}
    \\
    \hbox{\begin{tikzinline}[node distance=.5em and 1.5em]
          \raevent{a1}{\QL{\bLoc}{\mRA}\mid\DR{y}{1}}{}
          \xform{x1d}{r{=}1\limplies\bForm[\FALSE/\QL{\aLoc}{\mRLX}]}{below=of a1}
          \xform{xdd}{r{=}1\limplies s{=}1\limplies\bForm}{right=of x1d}
          \xform{xii}{\bForm[\FALSE/\QL{\bLoc}{\mRA}][\FALSE/\QL{\aLoc}{\mRLX}]}{above=of xdd}
          \xform{x2d}{s{=}1\limplies\bForm[\FALSE/\QL{\bLoc}{\mRA}]}{right=of xdd}
          \event{a2}{\FALSE\mid\DR{x}{1}}{above=of x2d}
          \xo[xleft]{a1}{x1d}
          \xo{a2}{x2d}
          \xos{a1}{xdd}
          \xos{a2}{xdd}
        \end{tikzinline}}
  \end{gather*}
  In order to get a satisfiable precondition for $\DRP{x}{1}$, we must
  introduce order:
  \begin{gather*}
    % \PR[\mRA]{y}{r}\SEMI \PR{x}{s}
    % \\
    \hbox{\begin{tikzinline}[node distance=.5em and 1.5em]
          \raevent{a1}{\QL{\bLoc}{\mRA}\mid\DR{y}{1}}{}
          \xform{x1d}{r{=}1\limplies\bForm[\FALSE/\QL{\aLoc}{\mRLX}]}{below=of a1}
          \xform{xdd}{r{=}1\limplies s{=}1\limplies\bForm}{right=of x1d}
          \xform{xii}{\bForm[\FALSE/\QL{\bLoc}{\mRA}][\FALSE/\QL{\aLoc}{\mRLX}]}{above=of xdd}
          \xform{x2d}{s{=}1\limplies\bForm[\FALSE/\QL{\bLoc}{\mRA}]}{right=of xdd}
          \event{a2}{\QL{\aLoc}{\mRLX}\mid\DR{x}{1}}{above=of x2d}
          \xo[xleft]{a1}{x1d}
          \xo{a2}{x2d}
          \xos{a1}{xdd}
          \xos{a2}{xdd}
          \sync[out=15,in=165]{a1}{a2}
        \end{tikzinline}}
  \end{gather*}
  
\end{example}


% $\QS{\aLoc}{\mRLX}=\Qx{\aLoc}$ and otherwise $\QS{\aLoc}{\amode}=\Q{\amode}$.

% $\QL{\aLoc}{\mSC}=\Q{\mSC}$ and otherwise $\QL{\aLoc}{\amode}=\Qw{\aLoc}$.

% $\DS{\aLoc}{\mRLX}{\bForm}=\bForm[\TRUE/\Dx{\aLoc}]$ and otherwise
% $\DS{\aLoc}{\amode}{\bForm}=\bForm[\FALSE/\D]$. 

% $\DL{\aLoc}{\mRLX}=\TRUE$ and otherwise $\DL{\aLoc}{\amode}=\Dx{\aLoc}$.

% \begin{definition}$\phantom{\;}$\par
%   \noindent
%   If $\aPS \in \sSTORE[\amode]{\aLoc}{\aExp}$ then
%   \begin{enumerate}
%   \item[\ref{S1}--\ref{S2})] as before,
%   \item[\ref{S3})]
%     $\labelingForm(\aEv)$ implies
%     \begin{math}
%       \aExp{=}\aVal
%       \land \RW
%       \land \QS{\aLoc}{\amode}
%     \end{math},
%   \item[\ref{S4})]
%     $\aTr{\bEvs}{\bForm}$ implies 
%     \begin{math}
%       (\Qw{\aLoc} \limplies \aExp{=}\aVal)
%       \land \DS{\aLoc}{\amode}{\bForm[\aExp/{\aLoc}]}
%     \end{math},
%   \item[\ref{S5})]
%     $\aTr{\emptyset}{\bForm}$ implies 
%     \begin{math}
%       \lnot\Qw{\aLoc}
%       \land \DS{\aLoc}{\amode}{\bForm[\aExp/{\aLoc}]}.
%     \end{math}
%   \end{enumerate}

%   \noindent
%   If $\aPS \in \sLOAD[\amode]{\aReg}{\aLoc}$ then
%   \begin{enumerate}
%   \item[\ref{L1}--\ref{L2})] as before,
%   \item[\ref{L3})]
%     $\labelingForm(\aEv)$ implies
%     \begin{math}
%       \RO
%       \land \QL{\aLoc}{\amode}
%     \end{math},
%   \item[\ref{L4})]
%     $\aTr{\bEvs}{\bForm}$ implies
%     \begin{math}
%       (\aVal{=}\aReg)
%       \limplies \bForm[\aReg/{\aLoc}]
%     \end{math}
%   \item[\ref{L5})] 
%     $\aTr{\emptyset}{\bForm}$ implies
%     \begin{math}
%       \DL{\aLoc}{\amode}
%       \land \lnot\Qx{\aLoc}
%       %     \end{math}
%       %     \\
%       %     \begin{math}
%       %       {}
%       \land 
%       (\RW
%       \limplies (\aVal{=}\aReg\lor\aLoc{=}\aReg) 
%       \limplies \bForm[\aReg/{\aLoc}]
%       ).
%     \end{math}
%   \end{enumerate}  
% \end{definition}



