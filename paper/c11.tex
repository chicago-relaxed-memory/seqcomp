\section{\PwTcTITLE: Pomsets with Predicate Transformers for C11}
\label{sec:c11}

\PwT{} can be used to generate semantic dependencies to prohibit thin-air
executions of \cXI, while preserving optimal lowering for relaxed access.  We
follow the approach of \citet{DBLP:conf/esop/PaviottiCPWOB20}, using our
semantics to generate \cXI{} candidate executions with a dependency relation,
then applying the rules of \rcXI{} \cite{DBLP:conf/pldi/LahavVKHD17}.  The
No-Thin-Air axiom of \rcXI{} is overly restrictive, requiring that
${\rrfx}\cup{\rpox}$ be acyclic.  Instead, we require that ${\rrfx}\cup{\lt}$
is acyclic.  This is a more precise categorisation of thin-air behavior, and 
it allows aggressive compiler optimizations that would be erroneously forbidden
by \rcXI's original No-Thin-Air axiom.

The chief difficulty is instrumenting our semantics to generate program
order, for use in the various axioms of \cXI{}.

\begin{definition}
  \label{def:po}
  A \PwTpo{} is a \PwT{} (\refdef{def:pomset}) equipped
  with relations $\rmrg$ and
  $\rpox$ such that 
  \begin{enumerate}[,label=(\textsc{m}\arabic*),ref=\textsc{m}\arabic*]
    \setcounter{enumi}{7}
    \makecounter{Bm}
  \item \label{pom-m} \makecounter{m}
    $\fmrg{} : (\aEvs\fun\aEvs)$
    is an idempotent function capturing \emph{merging}, such that 
    \\
    let $\reEvs=\{\aEv\mid\fmrg{\aEv}\EQ\aEv\}$ be \emph{real} events,
    let $\phEvs=(\aEvs\setminus\reEvs)$ be \emph{phantom} events,\\
    let $\umEvs=\{\aEv\mid\forall\bEv.\;\fmrg{\bEv}\EQ\aEv \limplies \bEv{=}\aEv\}$ be \emph{simple} events,
    let $\dmEvs=(\aEvs\setminus\umEvs)$ be \emph{compound} events,
    % % \\
    % let $\phEvs=\{\aEv\mid\fmrg{\aEv}{\neq}\aEv\}$ be \emph{phantom} events,
    % let $\reEvs=(\aEvs\setminus\phEvs)$ be \emph{real} events,\\
    % let $\dmEvs=\{\aEv\mid\exists\cEv{\neq}\bEv.\;\fmrg{\cEv}\EQ\fmrg{\bEv}\EQ\aEv\}$ be \emph{compound} events,
    % let $\umEvs=(\aEvs\setminus\dmEvs)$ be \emph{simple} events,
    \begin{multicols}{2}
    \begin{enumerate}%[leftmargin=0pt]
    % \item \label{pom-m-idempotent}
    %   $\fmrg{}$ is idempotent:
    %   \begin{math}
    %     \fmrg{\fmrg{\aEv}} = \fmrg{\aEv},
    %   \end{math}
      % $\fmrg{\aEv}\in\reEvs$,
      %where $\reEvs=\{\aEv\mid\fmrg{\aEv}\EQ\aEv\}$,
    % \item \label{pom-m-phantom-compount}
    %   if $\aEv\in\phEvs$ then $\fmrg{\aEv}\in\dmEvs$,
    \item \label{pom-m-lambda} 
      $\labeling(\aEv)=\labeling(\fmrg{\aEv})$,
    % \item \label{pom-m-tau} 
    %   \begin{math}
    %     \aTr{\bEvs}{\bForm}\riff
    %     \aTr{\fmrg{\bEvs}}{\bForm},
    %   \end{math}
      % if $\aEv\in\phEvs$ then
      % \begin{math}
      %   \aTr{\bEvs\cup\{\aEv\}}{\bForm}\riff \aTr{\bEvs\cup\{\fmrg{\aEv}\}}{\bForm},
      % \end{math}
    % \item \label{pom-m-le}
    %   if $\bEv\in\phEvs$ and $\bEv\le\aEv$ then $\aEv\in\phEvs$.
    %   %Real events don't depend on phantom events %${\le}\subseteq(\reEvs\times\reEvs)$,
    \item \label{pom-m-kappa} 
      if $\aEv\in\dmEvs$ then 
      \makebox[0pt][l]{\begin{math}
          \smash{\labelingForm(\aEv)\rimplies
            \bigvee{}\!\!\!_{\{\cEv\in\phEvs\mid\fmrg{\cEv}=\aEv\}}\labelingForm(\cEv)}
          .
      \end{math}}
    \end{enumerate}
    \end{multicols}

    \makecounter{Bpo}
  \item \label{pom-po} \makecounter{po}
    ${\rpox} \subseteq (\umEvs\times\umEvs)$ is a partial order capturing
    \emph{program order}.
    % ${\rpox} \subseteq (\aEvs\times\aEvs)$ is a pre-order capturing
    % \emph{program order}, such that
    % \begin{enumerate}
    % \item \label{pom-po-le}
    %   $\rpox$ is a partial order when restricted to simple events $(\umEvs\times\umEvs)$.
    % \end{enumerate}
  \end{enumerate}


  A \PwTpo{} is \emph{complete} if 
  \begin{multicols}{2}
    \begin{enumerate}[,label=(\textsc{c}\arabic*),ref=\textsc{c}\arabic*]

      \setcounter{enumi}{\value{Bkappa}}
    \item \label{top-kappa-c11}
      if $\aEv\in\reEvs$ then $\labelingForm(\aEv)$ is a tautology,

    %   \setcounter{enumi}{\value{kappa}}
    % \item[]
    %   \begin{enumerate}[leftmargin=0pt]
    %   \item \label{top-kappa-real}
    %     if $\aEv\in\reEvs$ then $\labelingForm(\aEv)$ is a tautology,
    %   \item \label{top-kappa-phantom}
    %     if $\aEv\in\dmEvs$ %(\aEvs\setminus\umEvs)$
    %     then
    %     $\labelingForm(\cEv)$
    %     \makebox[0cm][l]{is a tautology,
    %       for some $\cEv\in\phEvs$ %\aEvs\setminus\reEvs)$
    %       such that $\fmrg{\cEv}=\aEv$,
    %     }
    %   \end{enumerate}

      %\columnbreak
      \setcounter{enumi}{\value{Bterm}}
    \item \label{top-term-c11}
      $\aTerm$ is a tautology.
    \end{enumerate}
  \end{multicols}
\end{definition}
Since $\fmrg{}$ is idempotent, we have $\fmrg{\fmrg{\aEv}} = \fmrg{\aEv}$.
Equivalently, we could require $\fmrg{\aEv}\in\reEvs$.

We use $\fmrg{}$ to partition events $\aEvs$ in two ways: we distinguish
\emph{real} events $\reEvs$ from \emph{phantom} events $\phEvs$; we
distinguish \emph{simple} events $\umEvs$ from \emph{compound} events
$\dmEvs$.   From idempotency, it follows that all phantom events are simple
($\phEvs\subseteq\umEvs$) and all compound events are real
($\dmEvs\subseteq\reEvs$).  In addition, all phantom events map to compound
events (if $\aEv\in\phEvs$ then $\fmrg{\aEv}\in\dmEvs$).
% Note that the set of phantom events
% \begin{math}
%   \phEvs=\{\aEv\mid\fmrg{\aEv}{\neq}\aEv\}
%   %\reEvs=\{\aEv\mid\fmrg{\aEv}\EQ\aEv\}
% \end{math}
% and the set of compound events
% \begin{math}
%   \dmEvs=\{\aEv\mid\exists\cEv{\neq}\bEv.\;\fmrg{\cEv}\EQ\fmrg{\bEv}\EQ\aEv\}.
%   %\umEvs=\{\aEv\mid\forall\bEv.\;\fmrg{\bEv}\EQ\aEv \limplies \bEv{=}\aEv\}.
% \end{math}

% \ref{pom-m-tau} states that predicate transformers are insensitive to phantom
% events---the reverse implication holds from \refdef{def:family}.

\begin{lemma}
  \label{lem:po}
  If $\aPS$ is a \PwT{} then there is a \PwTpo{} $\aPS''$ that conservatively
  extends it.
  \begin{proof}
    The proof strategy is as follows: We extend the semantics of
    \reffig{fig:seq} with $\rpox$.  The obvious definition gives us a
    preorder rather than a partial order.  To get a partial order, we replay
    the semantics without merging to get an \emph{unmerged} pomset $\aPS'$;
    the construction also produces the map $\fmrg{}$.  We then construct
    $\aPS''$ as the union of $\aPS$ and $\aPS'$, using the dependency
    relation from $\aPS$.

    We extend the semantics with $\rpox$ as follows.  For pomsets with at
    most one event, ${\rpox}$ is the identity.  For sequential composition,
    ${\rpox}={\rpox_1}\cup {\rpox_2}\cup{\aEvs_1\times\aEvs_2}$.  For the
    conditional, ${\rpox}={\rpox_1}\cup {\rpox_2}$.  By construction, $\rpox$
    is a pre-order, which may include cycles due to coalescing.  For example:
    \begin{align*}
      % \PR{y}{r}
      % \SEMI
      \IF{r}\THEN
      \PW{x}{1}
      \SEMI
      \PW{y}{1}
      \ELSE
      \PW{y}{1}
      \SEMI
      \PW{x}{1}
      \FI
      &&
      \hbox{\begin{tikzinline}[node distance=1.5em]
          \event{d}{\DW{x}{1}}{}
          \event{e}{\DW{y}{1}}{right=of d}
          \pox[out=-15,in=-165]{d}[below]{e}
          \pox[out=165,in=15]{e}[above]{d}
        \end{tikzinline}}    
    \end{align*}

    To find an acyclic $\rpox'$, we replay the construction of $\aPS$ to get
    $\aPS'$.  When building $\aPS'$, we require disjoint union in \ref{seq-E}
    and \ref{if-E}: $\aEvs' = \aEvs'_1\uplus\aEvs'_2$.  If and event is
    unmerged in $\aPS$ (\emph{i.e.} $\aEv\in\aEvs_1\uplus\aEvs_2$) then we choose the
    same event name for $\aEvs'$ in $\aPS'$.  If an event is merged in
    $\aPS$ (\emph{i.e.} $\aEv\in\aEvs_1\cap\aEvs_2$) then we choose fresh event
    names---$\aEv'_1$ and $\aEv'_2$---and extend $\fmrg{}$
    accordingly: $\fmrg{\aEv'_1}=\fmrg{\aEv'_2}=\aEv$.  In $\aPS'$, we take
    ${\le'}={\rpox'}$.

    To arrive at $\aPS''$, we take
    \begin{enumerate*}
    \item
      $\aEvs''=\aEvs\cup\aEvs'$,
    \item
      $\labeling''=\labeling\cup\labeling'$,
    \item[(3a)]
      if $\aEv\in\aEvs$ then $\labelingForm''(\aEv)=\labelingForm(\aEv)$,
    \item[(3b)]
      if $\aEv\in\aEvs'\setminus\aEvs$ then $\labelingForm''(\aEv)=\labelingForm'(\aEv)$,
      \stepcounter{enumi}
    \item
      $\aTr{\prime\prime\bEvs}{}=\aTr{(\fmrginv{\bEvs})}{}$,
    \item
      $\aTerm''=\aTerm$,
    \item \label{le-item}
      ${\bEv}\lt'' {\aEv}$ exactly when $\fmrg{\bEv}\lt\fmrg{\aEv}$,
      %${\le''}=\fmrginv{\le}$,
    \item \label{pox-item}
      %${\rpox''}={\rpox}\cup{\rpox'}$, and
      ${\rpox''}={\rpox'}$, and
    \item 
      $\fmrg{}''$ is the constructed merge function. % $\fmrg{}$,
    \end{enumerate*}
  \end{proof}
\end{lemma}
\begin{definition}
  For a \PwTpo{}, let $\fextract{\aPS}$ be the projection
  of $\aPS$ onto the set
  \begin{math}
    \{\aEv\in\aEvs_1\mid\allowbreak \aEv \;\text{is simple and}\; \labelingForm_1(\aEv) \;\text{is a tautology} \}.
  \end{math}
  % % Let $\aPS$ be a \PwTpo{}.  We say that $\aPS$ is a \emph{candidate execution} if
  % % \begin{enumerate*}
  % % \item it is complete,
  % % \item all events in $\aEvs$ are simple (and therefore $\fmrg{}$ is the
  % %   identity function), and
  % % \item all preconditions in $\labelingForm{}$ are tautologies.
  % % \end{enumerate*}
  % % To extract a \cXI{} candidate execution, we use $\labeling$-consistency
  % % (\refdef{def:labeling:consistent}).
  % Let $\aPS_1$ be a \PwTpo{}. Then $\aPS\in\fextract{\aPS_1}$ if 
  % \begin{multicols}{2}
  %   \begin{enumerate}
  %   \item
  %     $\aPS_1$ is complete (\refdef{def:po}),
  %   \item 
  %     \begin{math}
  %       % \aEvs \subseteq \umEvs_1,
  %       \aEvs = \{\aEv\in\umEvs_1\mid\allowbreak \labelingForm_1(\aEv)
  %       \;\text{is a tautology} \}
  %       % \;\text{is}\; \labeling_1 \;\text{consistent}\},
  %     \end{math}
  %   \item
  %     if $\fmrg[1]{\bEv}\le_1 \fmrg[1]{\aEv}$
  %     then $\bEv\le\aEv$,
  %   \item $\rpox$, $\labeling$, $\labelingForm$, $\aTr{}{}$ and $\aTerm$ project $\aPS_1$ onto
  %     $\aEvs$.
  %   % \item
  %   %   $\aTerm_1$ is a tautology and $\aTerm=\aTerm_1$.
  %   \end{enumerate}
  % \end{multicols}
\end{definition}
By definition, $\fextract{\aPS}$ includes the simple events of $\aPS$ whose
preconditions are tautologies.  These are already in program order, as per
item \ref{pox-item} of the proof.  The dependency order is derived from the
real events using $\fmrg{}$, as per item \ref{le-item}.
% , which says that ${\bEv}\le'' {\aEv}$ exactly when $\fmrg{\bEv}\le\fmrg{\aEv}$.

The following lemma shows that if $\aPS$ is \emph{complete}, then
$\fextract{\aPS}$ includes at least one simple event for every compound event
in $\aPS$.

\begin{lemma}
  \label{lem:compound:phantom:exists}
  If $\aPS$ is a complete \PwTpo{} with compound event $\aEv$,
  %$\aEv\in\dmEvs$ 
  then
  %$(\exists \cEv\in\phEvs)$
  %$\fmrg{\cEv}=\aEv$
  there is a phantom event $\cEv\in\fmrginv{\aEv}$
  such that
  $\labelingForm(\cEv)$ is a tautology.
  \begin{proof}
    Immediate from \ref{pom-m-kappa}.
  \end{proof}
\end{lemma}

A pomset in the image of $\fextract{}$ is a \emph{candidate execution}.

As an example, consider \labeltext[\textsc{tc6}]{Java Causality Test Case 6}{JC-TC6}.  Taking $w=0$ and $v=1$,
the \PwTpo{} on the left below produces the candidate execution on the right.
In diagrams, we visualize $\rpox$ using a dotted arrow
\begin{math}
  \mkern-12mu
  \begin{tikzpicture}
    \node (a)  at (0, 0) {};
    \node (b)  at (.7, 0) {};
    \draw[pox](a)to(b);
  \end{tikzpicture}
  \mkern-12mu,
\end{math}
and
$\fmrg{}$ using a double arrow 
\begin{math}
  \mkern-12mu
  \begin{tikzpicture}
    \node (a)  at (0, 0) {};
    \node (b)  at (.7, 0) {};
    \draw[mrg](a)to(b);
  \end{tikzpicture}
  \mkern-12mu
\end{math}.
\begin{align*}
  \begin{gathered}[t]
    \makebox{\small$
      \PW{y}{w}
      \SEMI
      \PR{y}{r}
      \SEMI
      \IF{r{=}0}\THEN \PW{x}{1} \FI
      \SEMI
      \IF{r{=}1}\THEN \PW{x}{1} \FI
    $}
    \\
    \hbox{\begin{tikzinlinesmall}[node distance=1.5em]
        \event{e}{(v{=}r\lor w{=}r)\limplies (r{=}0\lor r{=}1)\mid\DW{x}{1}}{}
        \event{r}{\DR{y}{v}}{left=of e}
        \phevent{e1}{v{=}r\limplies r{=}0\mid\DW{x}{1}}{below left=1em and -7em of e}
        \phevent{e2}{v{=}r\limplies r{=}1\mid\DW{x}{1}}{below right=1em and -7em of e}
        \pox{r}[pos=0,below]{e1}
        \pox[out=-15,in=-167]{e1}[below]{e2}
        \mrg{e1}{e}
        \mrg{e2}{e}
        \event{i}{\DW{y}{w}}{left=of r}
        \pox[out=-15,in=-167]{i}[below]{r}
        %\pox{r}{e}
      \end{tikzinlinesmall}}    
  \end{gathered}
  &&
  \begin{gathered}[t]
    \makebox{\small$
      \PW{y}{0}
      \SEMI
      \PR{y}{r}
      \SEMI
      \IF{r{=}0}\THEN \PW{x}{1} \FI
      \SEMI
      \IF{r{=}1}\THEN \PW{x}{1} \FI
    $}
    \\
    \hbox{\begin{tikzinlinesmall}[node distance=1.5em]
        \event{e}{\DW{x}{1}}{}
        \event{r}{\DR{y}{1}}{left=of e}
        \event{i}{\DW{y}{0}}{left=of r}
        \pox[out=-15,in=-167]{i}[below]{r}
        \pox[out=-15,in=-167]{r}[below]{e}
      \end{tikzinlinesmall}}    
  \end{gathered}
  % &&
  % \begin{gathered}
  %   \PW{y}{0}
  %   \SEMI
  %   \PR{y}{r}
  %   \SEMI
  %   \IF{r{=}0}\THEN
  %   \PW{x}{1}
  %   \FI
  %   \SEMI
  %   \IF{r{=}1}\THEN
  %   \PW{x}{1}
  %   \FI
  %   \\
  %   \hbox{\begin{tikzinline}[node distance=1.5em]
  %       \event{e}{(1{=}r\lor 0{=}r)\limplies (r{=}0\lor r{=}1)\mid\DW{x}{2}}{}
  %       \event{r}{\DR{y}{1}}{left=of e}
  %       \phevent{e1}{1{=}r\limplies r{=}0\mid\DW{x}{2}}{below left=1em and -7em of e}
  %       \phevent{e2}{1{=}r\limplies r{=}1\mid\DW{x}{2}}{below right=1em and -7em of e}
  %       \pox{r}[pos=0,below]{e1}
  %       \pox[out=-15,in=-167]{e1}[below]{e2}
  %       \mrg{e1}{e}
  %       \mrg{e2}{e}
  %       \event{i}{\DW{y}{0}}{left=of r}
  %       \pox[out=-15,in=-167]{i}[below]{r}
  %     \end{tikzinline}}    
  % \end{gathered}
\end{align*}

We write $\sempo{}$ for the semantic function defined by applying the
construction of \reflem{lem:po} to the base semantics of \ref{fig:seq}.  

The dependency calculation of $\sempo{}$ is sufficient for \cXI; however, it
ignores synchronization and coherence completely.
\begin{align*}
  \tag{$\ddagger$}\label{eq5}
  \begin{gathered}
    \IF{r}\THEN\PW{x}{1}\FI
    \SEMI
    \IF{s}\THEN\PW{x}{2}\FI
    \SEMI
    \IF{\BANG r}\THEN\PW{x}{1}\FI
    \\
    \hbox{\begin{tikzinline}[node distance=1.5em]
        \phevent{d1}{r{\neq}0\mid\DW{x}{1}}{}
        \eventrr{e}{e}{s{\neq}0\mid\DW{x}{2}}{right=of d1}
        \phevent{d2}{r{=}0\mid\DW{x}{1}}{right=of e}
        \eventr{d}{d}{r{\neq}0\lor r{=}0\mid\DW{x}{1}}{above=1em of e}
        \pox[out=-15,in=-167]{d1}[below]{e}
        \pox[out=-15,in=-167]{e}[below]{d2}
        % \pox[bend right]{e}[right]{d}
        % \pox[bend right]{d}[left]{e}
        \mrg{d1}{d}
        \mrg{d2}{d}
      \end{tikzinline}}    
  \end{gathered}
\end{align*}
Adding a pair of reads to complete the pomset, we can extract the following candidate execution.
\begin{gather*}
  \PR{y}{r}
  \SEMI
  \PR{z}{s}
  \SEMI
  \IF{r}\THEN\PW{x}{1}\FI
  \SEMI
  \IF{s}\THEN\PW{x}{2}\FI
  \SEMI
  \IF{\BANG r}\THEN\PW{x}{1}\FI
  \\
  \hbox{\begin{tikzinline}[node distance=1.5em]
      \event{r}{\DR{y}{1}}{}
      \event{s}{\DR{z}{1}}{right=of r}
      \event{d}{\DW{x}{1}}{right=of s}
      \event{e}{\DW{x}{2}}{right=of d}
      \pox[out=-15,in=-167]{r}[below]{s}
      \pox[out=-15,in=-167]{s}[below]{d}
      \pox[out=-15,in=-167]{d}[below]{e}
    \end{tikzinline}}    
  \qquad\qquad\qquad
  \hbox{\begin{tikzinline}[node distance=1.5em]
      \event{r}{\DR{y}{0}}{}
      \event{s}{\DR{z}{1}}{right=of r}
      \event{d}{\DW{x}{2}}{right=of s}
      \event{e}{\DW{x}{1}}{right=of d}
      \pox[out=-15,in=-167]{r}[below]{s}
      \pox[out=-15,in=-167]{s}[below]{d}
      \pox[out=-15,in=-167]{d}[below]{e}
    \end{tikzinline}}    
\end{gather*}
It is somewhat surprising that the writes are independent of both reads!

In \PwTmca{}, delay stops the merge in \eqref{eq5}.
\begin{align*}
  \begin{gathered}
    \IF{r}\THEN\PW{x}{1}\FI
    \SEMI
    \IF{s}\THEN\PW{x}{2}\FI
    \SEMI
    \IF{\BANG r}\THEN\PW{x}{1}\FI
    \\
    \hbox{\begin{tikzinline}[node distance=1.5em]
        \event{d1}{r{\neq}0\mid\DW{x}{1}}{}
        \event{e}{s{\neq}0\mid\DW{x}{2}}{right=of d1}
        \event{d2}{r{=}0\mid\DW{x}{1}}{right=of e}
        \po{d1}{e}
        \po{e}{d2}
        \pox[out=-15,in=-167]{d1}[below]{e}
        \pox[out=-15,in=-167]{e}[below]{d2}
      \end{tikzinline}}    
  \end{gathered}
\end{align*}
It is possible to mimic this in \cXI{}, without introducing extra
dependencies: one can filter executions post-hoc using the relation
$\ledelay$, defined as follows:
\begin{displaymath}
  \fmrg{\bEv}\ledelay\fmrg{\aEv} \textif \bEv\xpox\aEv \textand \labeling(\bEv)\rdelays \labeling(\aEv).
\end{displaymath}
In \eqref{eq5}, we have both $d\ledelay e$ and $e \ledelay d$.  To rule out
this execution, it suffices to require that $\ledelay$ is a partial order.

Program \eqref{eq5} shows that the definition of semantic dependency is up for debate in \cXI, and
the International Standard Organisation's C++ concurrency subgroup acknowledges that semantic
dependency ($\rsdepx$) would address the Out-of-Thin-Air problem: \emph{Prohibiting executions that
have cycles in $\rrfx \cup \rsdepx$ can therefore be expected to prohibit Out-of-Thin-Air behaviors}
~\cite{mckenny:sdep}.
\PwTc{} resolves program structure into a dependency relation---not a complex state---that is
precise and easily adjusted. As refinements are made to \cXI, \PwTc{} can accommodate these and
test them automatically.





















\endinput


Old example:
\begin{gather*}
  %\tag{$\ddagger$}\label{eq5}
  % \PR{y}{r}
  % \SEMI
  \IF{r}\THEN
    \PW{x}{1}
    \SEMI
    \PW{x}{2}
  \ELSE
    \PW{x}{2}
    \SEMI
    \PW{x}{1}
  \FI
  % \IF{r}\THEN\PW{x}{2}\FI
  % \SEMI
  % \IF{r}\THEN\PW{x}{1}\FI
  % \SEMI
  % \IF{\BANG r}\THEN\PW{x}{2}\FI
  % \SEMI
  % \IF{\BANG r}\THEN\PW{x}{1}\FI
  \\
  \hbox{\begin{tikzinline}[node distance=1.5em]
      \phevent{d1}{r{\neq}0\mid\DW{x}{1}}{}
      \phevent{e1}{r{\neq}0\mid\DW{x}{2}}{right=of d1}
      \phevent{e2}{r{=}0\mid\DW{x}{2}}{right=of e1}
      \phevent{d2}{r{=}0\mid\DW{x}{1}}{right=of e2}
      \event{d}{r{\neq}0\lor r{=}0\mid\DW{x}{1}}{above right=1.5em and -3em of d1}
      \event{e}{r{\neq}0\lor r{=}0\mid\DW{x}{2}}{above right=1.5em and -3em of e2}
      %\event{r}{\DR{y}{1}}{left=of d}
      %\wki[out=5,in=175]{d}{e}
      \pox[out=-15,in=-167]{d1}[below]{e1}
      %\pox[out=-15,in=-167]{e1}[below]{e2}
      \pox[out=-15,in=-167]{e2}[below]{d2}
      \mrg{d1}[left]{d}
      \mrg{d2}[pos=.15,right]{d}
      \mrg{e1}[pos=.15,left]{e}
      \mrg{e2}[right]{e}
    \end{tikzinline}}    
\end{gather*}
Adding a read to complete the pomset, we can extract the following candidate executions.
\begin{gather*}
  \PR{y}{r}
  \SEMI
  \IF{r}\THEN
    \PW{x}{1}
    \SEMI
    \PW{x}{2}
  \ELSE
    \PW{x}{2}
    \SEMI
    \PW{x}{1}
  \FI
  \\
  \hbox{\begin{tikzinline}[node distance=1.5em]
      \event{r}{\DR{y}{1}}{}
      \event{d}{\DW{x}{1}}{right=of r}
      \event{e}{\DW{x}{2}}{right=of d}
      \pox[out=-15,in=-167]{r}[below]{d}
      \pox[out=-15,in=-167]{d}[below]{e}
    \end{tikzinline}}    
  \qquad\qquad\qquad\qquad
  \hbox{\begin{tikzinline}[node distance=1.5em]
      \event{r}{\DR{y}{0}}{}
      \event{e}{\DW{x}{2}}{right=of r}
      \event{d}{\DW{x}{1}}{right=of e}
      \pox[out=-15,in=-167]{r}[below]{e}
      \pox[out=-15,in=-167]{e}[below]{d}
    \end{tikzinline}}    
\end{gather*}

OLD def:
%Restrict the syntax to top-level parallel composition.

% \begin{comment}
%   \label{def:pomset}
%   A \emph{pomset with program order} is a tuple $(\aPS,\rmrg,\rpox)$,
%   where $\aPS$ is a pomset with predicate transformers
%   $(\aEvs, \labeling, \labelingForm, \aTr{}{}, {\aTerm}, {\le})$ %, {\ledep}, {\lesync}, {\leloc}, {\rrmw})$ 
%   and
%   \begin{enumerate}[,label=(\textsc{o}\arabic*),ref=\textsc{o}\arabic*]
%     %\setcounter{enumi}{9}

%   %   \makecounter{Bum}
%   % \item \label{pom-B} \makecounter{um}
%   %   $\umEvs \subseteq\aEvs$ is a set of \emph{unmerged events}, such that
%   %   \begin{enumerate}
%   %   \item ${\ledep}, {\lesync}, {\leloc}, {\rrmw}: ((\aEvs\setminus\umEvs)\times(\aEvs\setminus\umEvs))$,
%   %     % where $\pmEvs=(\aEvs\setminus\umEvs)$ is the set of \emph{potentially merged} events,
%   %   \end{enumerate}

%   %   \makecounter{Bdm}
%   % \item \label{pom-lambda} \makecounter{dm}
%   %   $\dmEvs\subseteq(\aEvs\setminus\umEvs)$ is a set of \emph{merged events}, 

%     \makecounter{Bmerge1}
%   \item \label{pom-lambda} \makecounter{merge1}
%     ${\rmrg} \subseteq (\phEvs\times\dmEvs)$
%     is a relation capturing \emph{merging},
%     where
%     \begin{enumerate}
%     \item $\phEvs\subseteq\aEvs$ represents \emph{phantom unmerged} events,
%       let $\reEvs$ be $\aEvs\setminus\phEvs$, %representing \emph{real} events,
%       % \item $\dmEvs$ and $\phEvs$ are disjoint, %$\phEvs\subseteq\umEvs$, and therefore 
%     \item $\dmEvs\subseteq\reEvs$ represents \emph{merged} events,
%       let $\umEvs$ be $\aEvs\setminus\dmEvs$, %, representing \emph{unmerged} events,
%     \item %${\ledep}$, ${\lesync}$, ${\leloc}$, ${\rrmw}
%       ${\le}\subseteq(\reEvs\times\reEvs)$,
%     \end{enumerate}
    
%     \makecounter{Bpo1}
%   \item \label{pom-lambda} \makecounter{po1}
%     ${\rpox} \subseteq (\umEvs\times\umEvs)$ is a partial order capturing
%     \emph{program order}.
%     % , such that 
%     % \begin{enumerate}
%     % \item ${\rpox}$ is total on events (in $\umEvs$) of the same thread:
%     %   \begin{itemize}
%     %   \item if $\labelingThrd(\bEv)=\labelingThrd(\aEv)$ then either
%     %     $\bEv\xpox\aEv$ or $\aEv\xpox\bEv$,
%     %   \item if $\bEv\xpox\aEv$ then $\labelingThrd(\bEv)=\labelingThrd(\aEv)$.
%     %   \end{itemize}
%     % \end{enumerate}
%   \end{enumerate}
% \end{comment}


OLD text:
% \begin{comment}
%   \begin{figure}
  \renewcommand{\phEvs}{F}
  \bookmark{\PwTcTITLE{} Semantics}
  \raggedright
  
  \noindent
  If $\aPS \in \sSEMI{\aPSS_1}{\aPSS_2}$ then
  $(\exists\aPS_1\in\aPSS_1)$ $(\exists\aPS_2\in\aPSS_2)$
  $(\exists\phEvs_1\subseteq\AllEvents)$ $(\exists\phEvs_2\subseteq\AllEvents)$
  \begin{multicols}{2}
    \begin{enumerate}[topsep=0pt,label=(\textsc{s}\arabic*),ref=\textsc{s}\arabic*]
      \setcounter{enumi}{\value{BE}}
    \item \label{seq-E-phantom}
      $\aEvs=(\aEvs_1\cup\aEvs_2)\uplus\phEvs_1 \uplus\phEvs_2$,

    \item[\eqref{seq-kappa}]
      \eqref{seq-tau}\;
      \eqref{seq-term}\;
      \eqref{seq-le}\;
      as in \reffig{fig:seq},

      \setcounter{enumi}{\value{lambda}}
    \item[] 
      \begin{enumerate}[leftmargin=0pt]
      \item \label{seq-lambda-include}
        ${\labeling}\supseteq\PBR{{\labeling_1}\cup {\labeling_2}}$, 
      \item \label{seq-lambda-phantom}
        if $\cEv\in \phEvs_i$ then $\labeling(\cEv)=\labeling_i(\fmrg{\cEv})$,
        %if $\cEv\in \phEvs_1$ then $\labeling(\cEv)=\labeling_1(\fmrg{\cEv})$,
        %if $\cEv\in \phEvs_2$ then $\labeling(\cEv)=\labeling_2(\fmrg{\cEv})$,
      \end{enumerate}

      \setcounter{enumi}{\value{kappa}}
    \item[] 
      \begin{enumerate}[leftmargin=0pt]
        \setcounter{enumii}{3}
      \item \label{seq-kappa-phantom1}
        if $\cEv\in \phEvs_1$ then $\labelingForm(\cEv)\riff\labelingForm_1(\fmrg{\cEv})$,
      \item \label{seq-kappa-phantom2}
        if $\cEv\in \phEvs_2$ then $\labelingForm(\cEv)\riff\aTr[1]{}{\labelingForm_2(\fmrg{\cEv})}$,
      \end{enumerate}

      \setcounter{enumi}{\value{m}}
    \item[] \labeltext[\textsc{s}8]{}{seq-m}
      \begin{enumerate}[leftmargin=0pt]
      \item \label{seq-m-include}
        $\fmrg{}\supseteq\fmrg[1]{}\cup\fmrg[2]{}$,
      \columnbreak
      \item \label{seq-m-phantom-forall}
        if $\cEv\in {\phEvs_1\cup\phEvs_2}$ then
        $\fmrg{\cEv}\in {\aEvs_1\cap\aEvs_2}$,
      \item \label{seq-m-exclude}
        if $\aEv\in {\aEvs_1\cap\aEvs_2}$
        then $\fmrg[1]{\aEv}=\fmrg[2]{\aEv}=\aEv$,
      % \item \label{seq-m-real-cup}
      %   if $\aEv\in(\aEvs_1\cup\aEvs_2)$ then $\fmrg{\aEv}=\aEv$,
      % \item \label{seq-m-real-cap}
      %   if $\fmrg{\aEv}=\aEv$ then $\aEv\not\in(\aEvs_1\cap\aEvs_2)$,
      \item \label{seq-m-phantom-exists}
        if $\aEv\in {\aEvs_1\cap\aEvs_2}$ then
        $(\existsone\cEv_i\in\phEvs_i)$
        %$(\existsone\cEv_2\in\phEvs_2)$
        $\fmrg{\cEv_1}=\fmrg{\cEv_2}=\aEv$,
      \end{enumerate}

      \setcounter{enumi}{\value{po}}
    \item[] \labeltext[\textsc{s}9]{}{seq-po}
      \begin{enumerate}[leftmargin=0pt]
      \item \label{seq-po-include}
        %${\rpox}\supseteq ({\rpox_1}\cup {\rpox_2})\setminus((\aEvs_1\times\aEvs_2)\cup(\aEvs_2\times\aEvs_1))$,
        ${\rpox}\supseteq \sigma_1({\rpox_1})\cup  \sigma_2({\rpox_2})$
      \item \label{seq-po-seq}
        if $\bEv\in\umEvs_1$ and $\aEv\in\umEvs_2$ then $\bEv\xpox\aEv$,
      \end{enumerate}
    \end{enumerate}
    where
    \begin{math}
      \umEvs_i= \phEvs_i \cup \{\aEv\in\aEvs_i\mid\forall\bEv.\;\fmrg[i]{\bEv}{=}\aEv \limplies \bEv{=}\aEv\}
    \end{math}
    and
    \begin{math}
      \sigma_i=\{[\cEv_i/\aEv]\mid\cEv_i\in\phEvs_i\land \fmrg{\cEv_i}=\aEv\}
    \end{math}
    % \makebox[0cm][l]{\begin{math}
    %   \umEvs_1= \phEvs_1 \cup (\aEvs_1\setminus\aEvs_2)
    % \end{math}
    % and
    % \begin{math}
    %   \umEvs_2= \phEvs_2 \cup (\aEvs_2\setminus\aEvs_1).
    % \end{math}}

  \end{multicols}
  \medskip

  \noindent
  If $\aPS \in \sIF{\aForm}\sTHEN\aPSS_1\sELSE\aPSS_2\sFI$ then
  $(\exists\aPS_1\in\aPSS_1)$ $(\exists\aPS_2\in\aPSS_2)$
  $(\exists\phEvs_1\subseteq\AllEvents)$ $(\exists\phEvs_2\subseteq\AllEvents)$
  \begin{multicols}{2}
    \begin{enumerate}[topsep=0pt,label=(\textsc{i}\arabic*),ref=\textsc{i}\arabic*]
    % \item[(\textsc{i}1)]
    %   (\textsc{i}2a)\;
    %   (\textsc{i}2b)\;
    %   (\textsc{i}8)\;
    %   (\textsc{i}9)\;
    %   as for $\sSEMI{}{}$,

    \item[\eqref{seq-E-phantom}]
      \eqref{seq-lambda}\;
      \eqref{seq-m}\;
      \eqref{seq-po-include}\;
      as for $\sSEMI{}{}$,

    \item[\eqref{if-kappa}]
      \eqref{if-tau}\;
      \eqref{if-term}\; 
      \eqref{if-le}\;
      as in \reffig{fig:seq},

      \columnbreak
      \setcounter{enumi}{\value{kappa}}
    \item[] 
      \begin{enumerate}[leftmargin=0pt]
        \setcounter{enumii}{3}
      \item \label{if-kappa-phantom1}
        if $\cEv\in \phEvs_1$ then $\labelingForm(\cEv) \riff \aForm\land\labelingForm_1(\aEv)$,
      \item \label{if-kappa-phantom2} 
        if $\cEv\in \phEvs_2$ then $\labelingForm(\cEv) \riff \lnot\aForm\land\labelingForm_2(\aEv)$.
      \end{enumerate}      

    %   \setcounter{enumi}{\value{Bpo}}
    % \item \label{if-po}
    %     ${\rpox}= {\rpox_1}\cup {\rpox_2}$.
      
    \end{enumerate}
  \end{multicols}
  \medskip

  \caption{\PwTcTITLE{} Semantics}
  \label{fig:c11}
\end{figure}


%   The semantic rules are given in \reffig{fig:c11}.  As usual, we write
%   $\existsone$ to mean ``there exists exactly one''.  For base cases, take
%   $\fmrg{}$ to be the identity and $\rpox$ to be empty.

%   \ref{seq-m-phantom-forall} states that new phantom events must merge into a
%   newly compound event.
%   \ref{seq-m-exclude} states that newly compound events must be real.
%   \ref{seq-m-phantom-exists} states that every newly compound event must have two
%   corresponding phantom events.

%   Here is a diagram elucidating \ref{seq-po-include}, where the top two are merged
%   sequentially into the bottom.
%   \begin{gather*}
%     \begin{tikzcenter}
%       \node (c1)  at (0, 0) {$\mathstrut c_1$};
%       \node (d1)  at (1, 0) {$\mathstrut d$};
%       \node (e1)  at (2, 0) {$\mathstrut e_1$};
%       \draw[->](c1)to(d1);
%       \draw[->](d1)to(e1);
%     \end{tikzcenter}
%     \qquad\qquad
%     \begin{tikzcenter}
%       \node (c2)  at (3, 0) {$\mathstrut c_2$};
%       \node (d2)  at (4, 0) {$\mathstrut d$};
%       \node (e2)  at (5, 0) {$\mathstrut e_2$};
%       \draw[->](c2)to(d2);
%       \draw[->](d2)to(e2);
%     \end{tikzcenter}
%     \\
%     \begin{tikzcenter}
%       \node (c1)  at (0, 0) {$\mathstrut c_1$};
%       \node (d1)  at (1, 0) {$\mathstrut d_1$};
%       \node (e1)  at (2, 0) {$\mathstrut e_1$};
%       \node (d)   at (2.5, .5) {$\mathstrut d$};
%       \node (c2)  at (3, 0) {$\mathstrut c_2$};
%       \node (d2)  at (4, 0) {$\mathstrut d_2$};
%       \node (e2)  at (5, 0) {$\mathstrut e_2$};
%       \draw[->](c1)to(d1);
%       \draw[->](d1)to(e1);
%       \draw[->](e1)to(c2);
%       \draw[->](c2)to(d2);
%       \draw[->](d2)to(e2);
%       \draw[->,dotted](d1)to(d);
%       \draw[->,dotted](d2)to(d);
%     \end{tikzcenter}
%   \end{gather*}

%   We have written \ref{seq-po-include} .
% \end{comment}



%Let $\phEvs=\aEvs\setminus\reEvs$ and $\dmEvs=\aEvs\setminus\umEvs$.

% Lots of fiddly details.  Intuitively, (1) compute the preconditions and order
% for $\reEvs$ as before, (2) create new events for the merged ones, compute
% preconditions for events outside $\reEvs$ by applying $\aTr[1]{}{}$.

$\rpox$ can have cycles when interpreted on merged events. For example:
% \begin{align*}
%   \begin{gathered}[t]
%     \PW{x}{1}
%     \\
%     \hbox{\begin{tikzinline}[node distance=.5em and 1em]
%         \eventl{d}{a}{\DW{x}{1}}{}
%       \end{tikzinline}}    
%   \end{gathered}
%   &&
%   \begin{gathered}[t]
%     \PW{y}{1}
%     \\
%     \hbox{\begin{tikzinline}[node distance=.5em and 1em]
%         \eventl{e}{a}{\DW{y}{1}}{}
%       \end{tikzinline}}    
%   \end{gathered}
%   &&
%   \begin{gathered}[t]
%     \PW{x}{1}
%     \\
%     \hbox{\begin{tikzinline}[node distance=.5em and 1em]
%         \eventl{d}{a}{\DW{x}{1}}{}
%       \end{tikzinline}}    
%   \end{gathered}
%   &&
%   \begin{gathered}[t]
%     \PW{y}{1}
%     \\
%     \hbox{\begin{tikzinline}[node distance=.5em and 1em]
%         \eventl{e}{a}{\DW{y}{1}}{}
%       \end{tikzinline}}    
%   \end{gathered}
% \end{align*}
% \begin{align*}
%   \begin{gathered}    
%     \PW{x}{1}
%     \SEMI 
%     \PW{y}{2}
%     \SEMI
%     \PW{x}{1}
%     \SEMI 
%     \PW{y}{2}
%     \\[-.4ex]
%     \nonumber
%     \hbox{\begin{tikzinline}[node distance=1.5em]
%         \phevent{d1}{\DW{x}{1}}{}               %x{d_1}
%         \phevent{e1}{\DW{y}{2}}{right=of d1}    %x{e_1}
%         \phevent{d2}{\DW{x}{1}}{right=of e1}    %x{d_2}
%         \phevent{e2}{\DW{y}{2}}{right=of d2}    %x{e_2}
%         \event{d}{\DW{x}{1}}{above=of d1}
%         \event{e}{\DW{y}{2}}{above=of e2}
%         \pox[out=-15,in=-170]{d1}[below]{e1}
%         \pox[out=-15,in=-170]{e1}[below]{d2}
%         \pox[out=-15,in=-170]{d2}[below]{e2}
%         \mrg{d1}[left]{d}
%         \mrg[out=140,in=0]{d2}[pos=.15,right]{d}
%         \mrg[out=40,in=-180]{e1}[pos=.15,left]{e}
%         \mrg{e2}[right]{e}
%       \end{tikzinline}}
%   \end{gathered}
% \end{align*}

\begin{align*}
  \begin{gathered}    
    \PR{x}{r_1}
    \SEMI 
    \PR{y}{r_2}
    \SEMI
    \PR{x}{s_1}
    \SEMI 
    \PR{y}{s_2}
    \\[-.4ex]
    \nonumber
    \hbox{\begin{tikzinline}[node distance=1.5em]
        \phevent{d1}{\DR{x}{1}}{}               %x{d_1}
        \phevent{e1}{\DR{y}{2}}{right=of d1}    %x{e_1}
        \phevent{d2}{\DR{x}{1}}{right=of e1}    %x{d_2}
        \phevent{e2}{\DR{y}{2}}{right=of d2}    %x{e_2}
        \event{d}{\DR{x}{1}}{above=of d1}
        \event{e}{\DR{y}{2}}{above=of e2}
        \pox[out=-15,in=-170]{d1}[below]{e1}
        \pox[out=-15,in=-170]{e1}[below]{d2}
        \pox[out=-15,in=-170]{d2}[below]{e2}
        \mrg{d1}[left]{d}
        \mrg[out=140,in=0]{d2}[pos=.15,right]{d}
        \mrg[out=40,in=-180]{e1}[pos=.15,left]{e}
        \mrg{e2}[right]{e}
      \end{tikzinline}}
  \end{gathered}
\end{align*}

For TC2, we have: 
\begin{gather*}
  \PR{x}{r}\SEMI
  \PR{x}{s}\SEMI
  \IF{r{=}s}\THEN \PW{y}{1}\FI
  \PAR
  \PW{x}{y}
  \\
  \hbox{\begin{tikzinline}[node distance=1.5em]
      \event{a1}{\DR{x}{1}}{}
      % \event{a2}{\DR{x}{1}}{right=of a1}
      \event{a3}{\DW{y}{1}}{right=of a1}
      % \po{a1}{a3}
      % \po[out=-20,in=-160]{a1}{a3}
      \event{b1}{\DR{y}{1}}{right=3em of a3}
      \event{b2}{\DW{x}{1}}{right=of b1}
      \rf{a3}{b1}
      \po{b1}{b2}
      % \rf[out=169,in=11]{b2}{a2}
      \rf[out=169,in=11]{b2}{a1}
      \phevent{p2}{\DR{x}{1}}{below=of a1}
      \phevent{p1}{\DR{x}{1}}{left=of p2}
      \mrg{p1}[pos=.3]{a1}
      \mrg{p2}[right]{a1}      
      \pox[out=-15,in=-170]{b1}[below]{b2}
      \pox[out=-15,in=-170]{p1}[below]{p2}
      \pox{p2}[below]{a3}
    \end{tikzinline}}
  \\
  \hbox{\begin{tikzinline}[node distance=1.5em]
      \event{a1}{\DR{x}{0}}{}
      % \event{a2}{\DR{x}{1}}{right=of a1}
      \event{a3}{\DW{y}{1}}{right=of a1}
      % \po{a1}{a3}
      % \po[out=-20,in=-160]{a1}{a3}
      \event{b1}{\DR{y}{1}}{right=3em of a3}
      \event{b2}{\DW{x}{1}}{right=of b1}
      \rf{a3}{b1}
      \po{b1}{b2}
      % \rf[out=169,in=11]{b2}{a2}
      %\rf[out=169,in=11]{b2}{a1}
      \phevent{p2}{\DR{x}{0}}{below=of a1}
      \phevent{p1}{\DR{x}{0}}{left=of p2}
      \mrg{p1}[pos=.3]{a1}
      \mrg{p2}[right]{a1}      
      \pox[out=-15,in=-170]{b1}[below]{b2}
      \pox[out=-15,in=-170]{p1}[below]{p2}
      \pox{p2}[below]{a3}
    \end{tikzinline}}
\end{gather*}
% Idea: merging reads can only make a difference is if there is a race.

% \begin{align*}
%   \begin{gathered}[t]
%     % \PR{y}{r}
%     % \SEMI
%     \IF{r}\THEN
%       \PW{x}{1}
%       \SEMI
%       \PW{x}{2}
%     \ELSE
%       \PW{x}{1}
%       \SEMI
%       \PW{x}{2}
%     \FI
%     % \IF{r}\THEN\PW{x}{2}\FI
%     % \SEMI
%     % \IF{r}\THEN\PW{x}{1}\FI
%     % \SEMI
%     % \IF{\BANG r}\THEN\PW{x}{2}\FI
%     % \SEMI
%     % \IF{\BANG r}\THEN\PW{x}{1}\FI
%     \\
%     \hbox{\begin{tikzinline}[node distance=1.5em]
%         %\event{a1}{\DR{y}{1}}
%         \phevent{d1}{r{\neq}0\mid\DW{x}{1}}{}               %x{d_1}
%         \phevent{e1}{r{\neq}0\mid\DW{x}{2}}{right=of d1}    %x{e_1}
%         \phevent{d2}{r{=}0\mid\DW{x}{1}}{right=of e1}    %x{d_2}
%         \phevent{e2}{r{=}0\mid\DW{x}{2}}{right=of d2}    %x{e_2}
%         \event{d}{\DW{x}{1}}{above=of d1}
%         \event{e}{\DW{x}{2}}{above=of e2}
%         \wki[out=5,in=175]{d}{e}
%         \pox[out=-15,in=-165]{d1}[below]{e1}
%         \pox[out=-15,in=-165]{e1}[below]{d2}
%         \pox[out=-15,in=-165]{d2}[below]{e2}
%         \mrg{d1}[left]{d}
%         \mrg[out=140,in=-10]{d2}[pos=.15,right]{d}
%         \mrg[out=40,in=-170]{e1}[pos=.15,left]{e}
%         \mrg{e2}[right]{e}
%       \end{tikzinline}}    
%   \end{gathered}
% \end{align*}

% \begin{align*}
%   \begin{gathered}[t]
%     \PR{y}{r}
%     \SEMI
%     \IF{r}\THEN
%       \PW{x}{1}
%       \SEMI
%       \PW{x}{2}
%     \ELSE
%       \PW{x}{1}
%       \SEMI
%       \PW{x}{2}
%     \FI
%     % \IF{r}\THEN\PW{x}{2}\FI
%     % \SEMI
%     % \IF{r}\THEN\PW{x}{1}\FI
%     % \SEMI
%     % \IF{\BANG r}\THEN\PW{x}{2}\FI
%     % \SEMI
%     % \IF{\BANG r}\THEN\PW{x}{1}\FI
%     \\
%     \hbox{\begin{tikzinline}[node distance=1.5em]
%         \phevent{d1}{1{\neq}0\mid\DW{x}{1}}{}               %x{d_1}
%         \phevent{e1}{1{\neq}0\mid\DW{x}{2}}{right=of d1}    %x{e_1}
%         \phevent{d2}{1{=}0\mid\DW{x}{1}}{right=of e1}    %x{d_2}
%         \phevent{e2}{1{=}0\mid\DW{x}{2}}{right=of d2}    %x{e_2}
%         \event{d}{\DW{x}{1}}{above=of d1}
%         \event{e}{\DW{x}{2}}{above=of e2}
%         \event{r}{\DR{y}{1}}{left=of d}
%         \wki[out=5,in=175]{d}{e}
%         \pox[out=-15,in=-167]{d1}[below]{e1}
%         \pox[out=-15,in=-167]{e1}[below]{d2}
%         \pox[out=-15,in=-167]{d2}[below]{e2}
%         \mrg{d1}[left]{d}
%         \mrg[out=140,in=-10]{d2}[pos=.15,right]{d}
%         \mrg[out=40,in=-170]{e1}[pos=.15,left]{e}
%         \mrg{e2}[right]{e}
%       \end{tikzinline}}    
%     \\
%     \hbox{\begin{tikzinline}[node distance=1.5em]
%         \phevent{d1}{\DW{x}{1}}{}               %x{d_1}
%         \phevent{e1}{\DW{x}{2}}{right=of d1}    %x{e_1}
%         \phnonevent{d2}{\DW{x}{1}}{right=of e1}    %x{d_2}
%         \phnonevent{e2}{\DW{x}{2}}{right=of d2}    %x{e_2}
%         \event{d}{\DW{x}{1}}{above=of d1}
%         \event{e}{\DW{x}{2}}{above=of e2}
%         \event{r}{\DR{y}{1}}{left=of d}
%         \wki[out=5,in=175]{d}{e}
%         \pox[out=-15,in=-170]{d1}[below]{e1}
%         \pox[out=-15,in=-170]{e1}[below]{d2}
%         \pox[out=-15,in=-170]{d2}[below]{e2}
%         \mrg{d1}[left]{d}
%         \mrg[out=140,in=-10]{d2}[pos=.15,right]{d}
%         \mrg[out=40,in=-170]{e1}[pos=.15,left]{e}
%         \mrg{e2}[right]{e}
%       \end{tikzinline}}    
%   \end{gathered}
% \end{align*}









