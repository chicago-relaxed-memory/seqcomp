\RequirePackage{amssymb}  %% bizarre error previous def of Bbbk if after documentclass
\documentclass[acmsmall,screen,anonymous,review]{acmart}\settopmatter{printfolios=true}
\hypersetup{bookmarksnumbered,bookmarksopen=true,bookmarksdepth=3}
\settopmatter{printfolios=true}
\AtEndPreamble{%
  \theoremstyle{acmdefinition}
  \newtheorem{remark}[theorem]{Remark}
  \newtheorem{candidate}[theorem]{Candidate}
  \renewcommand{\theequation}{\fnsymbol{equation}}
}
\bibliographystyle{ACM-Reference-Format}
\citestyle{acmauthoryear}   %% For author/year citations

\setcopyright{rightsretained}
\acmPrice{}
\acmDOI{}
\acmYear{2022}
\copyrightyear{2022}
\acmSubmissionID{}
\acmJournal{PACMPL}
\acmVolume{0}
\acmNumber{POPL}
\acmArticle{0}
\acmMonth{1}
\startPage{1}

\usepackage{macros}
% \showRAfalse
\showSCOPEfalse
\title{The Leaky Semicolon}


\usepackage{xr}
\externaldocument{paper}
\makecounter{Bkappa} \setcounter{Bkappa}{2}
\makecounter{Btau}   \setcounter{Btau}{3}
\makecounter{Bterm}  \setcounter{Bterm}{4}
\begin{document}
\appendix
\section{Discussion}
\label{sec:discussion}

\subsection{Further Comparison to ``Promising Semantics'' [POPL 2017]}
\label{sec:promising}

Recently, \citet{promising-ldrf} showed that certain combinations of compiler
optimizations are inconsistent with local \drf{} guarantees.  All of the
examples that prove inconsistency have the same shape: they combine
read-introduction and if-introduction (aka, case analysis).  Effectively,
this turns one read into two, where different conditional branches can be
taken for the two copies of the read.  This is reminiscent of the type of
\emph{bait and switch} behavior noted by
\citet{DBLP:journals/pacmpl/JagadeesanJR20}: the promising semantics (\PS{})
\cite{DBLP:conf/popl/KangHLVD17} and related models
\citep{DBLP:conf/esop/JagadeesanPR10,DBLP:journals/pacmpl/ChakrabortyV19,Manson:2005:JMM:1047659.1040336},
fail to validate compositional reasoning of temporal properties.  Consider
example \ref{oota4} from \cite{DBLP:journals/pacmpl/JagadeesanJR20}:
\begin{gather*}
  \taglabel{oota4}
  \begin{gathered}
    \PW{y}{x}
    \PAR
    \PR{y}{r} \SEMI\IF{b}\THEN  \PW{x}{r} \SEMI \PW{z}{r} \ELSE \PW{x}{1} \FI
    \PAR
    \PW{b}{1}
    % \\[-1ex]
    % \hbox{\begin{tikzinline}[node distance=1.5em]
    %     \event{rx}{\DR{x}{1}}{}
    %     \event{wy}{\DW{y}{1}}{right=of rx}
    %     \po{rx}{wy}
    %     \event{ry}{\DR{y}{1}}{right=3em of wy} 
    %     \event{wx}{\DW{x}{1}}{right=of ry}
    %     \event{wz}{\DW{z}{1}}{right=of wx}
    %     \event{rb}{\DR{b}{1}}{right=of wz}
    %     \event{wb1}{\DW{b}{1}}{right=3em of rb}
    %     \po{ry}{wx}
    %     \rf{wb1}{rb}
    %     \rf{wy}{ry}
    %     \rf[out=-170,in=-10]{wx}{rx}
    %     \po{rb}{wz}
    %     \po[out=15,in=165]{ry}{wz}
    %   \end{tikzinline}}
    \\[-1ex]
    \hbox{\begin{tikzinline}[node distance=1.5em]
        \event{rx}{\DR{x}{1}}{}
        \event{wy}{\DW{y}{1}}{right=of rx}
        \po{rx}{wy}
        \event{rb}{\DR{b}{1}}{right=3em of wy}
        \event{ry}{\DR{y}{1}}{right=of rb} 
        \event{wx}{\DW{x}{1}}{right=of ry}
        \event{wz}{\DW{z}{1}}{right=of wx}
        \event{wb1}{\DW{b}{1}}{right=3em of wz}
        \po{ry}{wx}
        \rf[out=-170,in=-10]{wb1}{rb}
        \rf[out=15,in=165]{wy}{ry}
        \rf[out=-170,in=-10]{wx}{rx}
        \po[out=15,in=165]{rb}{wz}
        \po[out=15,in=165]{ry}{wz}
      \end{tikzinline}}
  \end{gathered}
\end{gather*}
Under all variants of \PwT{}, this outcome is disallowed, due to the cycle
involving $x$ and $y$.\footnote{All of the reads in \ref{oota4} are
  cross-thread, so there is no difference between \PwTmca{1} and \PwTmca{2}.
  For \PwTc, there is a cycle in ${\rrfx}\cup{\lt}$.}  Under \PS{}, this
outcome is allowed by {baiting} with the \texttt{else} branch, then
{switching} to the \texttt{then} branch, based on a coin flip (${b}$).
% Using a combination of read-introduction and if-introduction,
% the middle thread can be rewritten to:
% \begin{displaymath}
%   \begin{array}{l}
%     \PR{y}{r} \SEMI \IF{b}\THEN\\
%     \quad \IF{b}\THEN  \PW{x}{r} \SEMI \PW{z}{r} \ELSE \PW{x}{1} \FI\\
%     \ELSE\\
%     \quad \IF{b}\THEN  \PW{x}{r} \SEMI \PW{z}{r} \ELSE \PW{x}{1} \FI\\
%     \FI
%   \end{array}
% \end{displaymath}
% \footnote{Call the threads \texttt{s}, \texttt{t}, and
%   \texttt{u}.  To get the result in the promising semantics, first execute
%   \texttt{u} to get message \texttt{<b:1@1>}.  Then \texttt{t} promises
%   \texttt{<x:1@1>}, which it can fulfill by reading \texttt{b}$=$\texttt{0}.
%   Then execute \texttt{s} to get message \texttt{<y:1@1>}.  Then execute
%   \texttt{t}, reading \texttt{b}$=$\texttt{1} and \texttt{y}$=$\texttt{1} and
%   fulfill the promise by writing \texttt{<x:1@1>}. The execution is exactly
%   the same in our speculative semantics \cite{DBLP:conf/esop/JagadeesanPR10},
%   removing timestamps and replacing the word \emph{promise} by
%   \emph{speculation}.}
% \citet[Fig.~8]{DBLP:journals/toplas/Lochbihler13} notes that such violations
% of temporal reasoning could result in violations of the Java security
% architecture.

\citet{promising-ldrf} introduce more complex examples to show that the
promising semantics fails \ldrfsc{}.\footnote{\citet{promising-ldrf} show
  that by restricting \RMW{}-store reorderings, one can establish \ldrfsc{}
  for \PS{}.  We speculate that no such restriction is required for \PwT{}.
  (We did not treat \RMW{}s in our proof of \ldrfsc{}.)}  Here is one, dubbed
\labeltext{\textsc{ldrf-fail-ps}}{LDRF-Fail-PS}.
\begin{gather*}  
  %\taglabel{LDRF-Fail-PS}
  \begin{gathered}
  \IF{x}\THEN
    \PFADD{w}{}{1}\SEMI
    \PW{y}{1}\SEMI
    \PW{z}{1}
  \FI
  \PAR
  \IF{z}\THEN
    \IF{\BANG \PFADD{w}{}{1}}\THEN
      \PW{x}{\PR{y}{}}
    \FI
  \ELSE
    \PW{x}{1}
  \FI
    \\
    \hbox{\begin{tikzinline}[node distance=1.5em]
        \event{a1}{\DR{x}{1}}{}
        \event{a2}{\DR{w}{1}}{right=of a1}
        \event{a3}{\DW{w}{2}}{right=of a2}
        \event{a4}{\DW{y}{1}}{right=of a3}
        \event{a5}{\DW{z}{1}}{right=of a4}
        \event{b1}{\DR{z}{1}}{right=4em of a5}
        \event{b2}{\DR{w}{0}}{right=of b1}
        \event{b3}{\DW{w}{1}}{right=of b2}
        \event{b4}{\DR{y}{1}}{right=of b3}
        \event{b5}{\DW{x}{1}}{right=of b4}
        \po[out=15,in=165]{a1}{a3}
        \po[out=15,in=165]{a1}{a4}
        \po[out=15,in=165]{a1}{a5}        
        \po{b4}{b5}
        \po[out=15,in=165]{b2}{b5}        
        \po[out=15,in=165]{b1}{b3}
        \rf{a5}{b1}
        \rf[out=10,in=170]{a4}{b4}
        \rf[out=-170,in=-10]{b3}{a2}
        \rf[out=-170,in=-10]{b5}{a1}
        \rmw{a2}{a3}
        \rmw{b2}{b3}
      \end{tikzinline}}
  \end{gathered}
\end{gather*}
Again, all variants of \PwT{} disallow the outcome due to the cycle involving
$x$ and $y$.  It is allowed by \PS{} by baiting the second thread with
$\PW{x}{1}$ in the \texttt{else} branch, then switching to the \texttt{then}
branch.  This shows some some structural resemblance to \ref{oota4}, with $z$
replacing $b$.


\citeauthor{promising-ldrf} argue that the outcome of \ref{LDRF-Fail-PS} is
inevitable due to compiler optimizations.  The examples crucially involve the
following sequence of operations:
\begin{itemize}
\item read-introduction,
\item if-introduction, branching on the read just introduced.
\end{itemize}
We believe this combination of optimizations is unsound.  This is obviously
the case in \cXI: read-introduction may cause undefined behavior (\ub), due to
the possible introduction of a data race.

The situation is more delicate in \llvm.  The short version of the story is
that load-hoisting followed by case analysis is unsound in \llvm, without freeze. 
This happens because:
\begin{itemize}
\item read-introduction may result in the undefined value $\UNDEFINED$, due
  to the possible introduction of a data race \cite{DBLP:conf/cgo/ChakrabortyV17}, and
\item branching on an undefined value in \llvm{} results in \ub. 
\end{itemize}
\llvm{} delays \ub{} using the undefined value.  This allows \llvm{} to
perform optimizations such as load hoisting, where
\begin{math}
  \IF{C}\THEN \PR{x}{r}\FI
\end{math}
is rewritten to 
\begin{math}
  \PR{x}{s}\SEMI
  \LET{r}{\TERNARY{C}{s}{r}}.
\end{math}
Despite this, other optimizations regularly performed by \llvm{} are
unsound \cite{DBLP:conf/pldi/LeeKSHDMRL17}.  An example is loop switching,
where
\begin{math}
  \WHILE{C_1}\DO \IF{C_2}\THEN \aCmd_1 \ELSE \aCmd_2 \FI \OD
\end{math}
is rewritten to 
\begin{math}
  \IF{C_2}\THEN \WHILE{C_1}\DO \aCmd_1 \OD \ELSE \WHILE{C_1}\DO \aCmd_2 \FI \OD.
\end{math}
Freeze was introduced in \llvm{} in order to make such optimizations sound by
allowing branch on frozen $\UNDEFINED$ to give nondeterministic choice rather
than \ub{}.  In the RFC for freeze, \citet{nuno} says: \emph{``Note that
  having branch on poison not trigger \ub{} has its own problems.  We believe
  this is a good tradeoff.''}  \ref{LDRF-Fail-PS} demonstrates a concrete
problem with this tradeoff.  Other compilers, such as Compcert, are more
conservative \cite[\textsection9]{DBLP:conf/pldi/LeeKSHDMRL17}.

Thus, the difference between \PS{} and \PwT{} can be understood in terms of
the valid program transformations.  \PS{} allows reads to be introduced, with
subsequent case analysis on the value read.  \PwT{} validates case analysis,
but invalidates read-introduction.

Allowing executions such as \ref{oota4} and \ref{LDRF-Fail-PS} also
invalidates compositional reasoning for temporal safety properties (see
\textsection\ref{sec:results}).

These differences highlight the subtle tensions between compiler
optimizations and program logics that are revealed by relaxed memory models.
It is not possible to have everything one wants. Thus, one is forced to
choose which optimizations and reasoning principles are most
important.\footnote{Another example is the tension between load
  hoisting---forbidden in \cXI{} but allowed by \llvm{}---and common
  subexpression elimination over an acquiring lock---allowed by \cXI{} but
  forbidden by \llvm{} \cite{DBLP:conf/cgo/ChakrabortyV17}.}

Finally, we note that it is possible that \PS{} is properly weaker than \PwT{}.  

% \todo{Write this.}

% Case analysis gives very weak results when combined with thread inlining.
% See \cite[\textsection B.1]{DBLP:journals/pacmpl/ChakrabortyV19appendix}.
% These happen by performing transformations that: 
% (1) introduce conditionals,
% (2) inline two threads on both sides of the introduced conditional,
% (3) choose different orders for the two threads for the two sides of the ( conditional.

% Case analysis gives very weak results when combined with read-introduction.
% See \cite{promising-ldrf}.
% These happen by performing transformations that: 
% (1) introduce reads,
% (2) introduce conditionals,
% (3) choose different values for the reads on the two sides of the conditional.


% The fact that the semantics is not verifiable a posteriori is something it
% shares with \weakestmo{}, where the justification relation must be built
% inductively.

% \weakestmo{} admits FADD, but \PS{} does not.
% \PS{} admits CohCYC, but \weakestmo{} does not.

% \subsubsection{Load hoisting in LLVM}
% Load-hoisting followed by case analysis is unsound in LLVM, without freeze.
% Introducing a read may cause a race, resulting in read value $\UNDEFINED$.
% Branch on $\UNDEFINED$ is UB.  Freeze was added to get around this...

% Examples from \cite{promising-ldrf} show that freeze is bullshit.  Compcert
% does not validate loop switching
% \cite[\textsection9]{DBLP:conf/pldi/LeeKSHDMRL17}.

% \cite[\textsection3.3]{DBLP:conf/pldi/LeeKSHDMRL17} Global Value Numbering
% (GVN) and Loop Unswitching require different semantics for branch on
% $\UNDEFINED$. \url{https://llvm.org/docs/LangRef.html#freeze-instruction}

% \href{https://lists.llvm.org/pipermail/llvm-dev/2016-October/106182.html}{Purpose of Freeze}
% \begin{quotation}
%   Poison is propagated aggressively throughout. However, there are cases
%   where this breaks certain optimizations, and therefore freeze is used to
%   selectively stop poison from being propagated.

%   A use of freeze is to enable speculative execution.  For example, loop
%   switching does the following transformation:
% \begin{verbatim}
% \end{verbatim}
% \begin{verbatim}
% while (C) {        if (C2) {     
%   if (C2) {           while (C)  
%    A                     A       
%   } else {   ==>   } else {      
%    B                   while (C) 
%   }                       B      
% }                  }             
% \end{verbatim}
%   Here we are evaluating C2 before C.  If the original loop never executed
%   then we had never evaluated C2, while now we do.  So we need to make sure
%   there's no UB for branching on C2.  Freeze ensures that so we would
%   actually have 'if (freeze(C2))' instead.  Note that
%   \emph{having branch on poison not trigger UB has its own problems.}
%   We believe this is a good tradeoff.
% \end{quotation}


% \begin{quotation}
%   LLVM frequently performs such load-introductions in the “simplify CFG”
%   pass; e.g., when hoisting loads outside of conditionals.
% \end{quotation}

% case analysis happens in ``function specialization'' also known as
% ``procedure cloning''

% \subsubsection{Hoisting and CSE}
% Example from Viktor's talk: ``Weak Memory Concurrency in C/C++11 and LLVM''

% Load Hoisting:
% \begin{displaymath}
%   \IF{c}\THEN \PR{x}{a}\FI
%   \rightsquigarrow
%   \PR{x}{t}\SEMI
%   \LET{a}{\TERNARY{c}{t}{a}}
% \end{displaymath}

% CSE over acquiring lock:
% \begin{displaymath}
%   \PR{x}{a}\SEMI
%   \LOCK\SEMI
%   \PR{x}{b}
%   \rightsquigarrow
%   \PR{x}{a}\SEMI
%   \LOCK\SEMI
%   \LET{b}{a}
% \end{displaymath}

% Having both is clearly wrong:
% \begin{displaymath}
%   \begin{array}{l}
%     \IF{c}\THEN\\
%     \quad\PR{x}{a}\FI\SEMI\\
%     \LOCK\SEMI\\
%     \PR{x}{b}    
%   \end{array}
%   \rightsquigarrow
%   \begin{array}{l}
%     \PR{x}{t}\SEMI\\
%     \LET{a}{\TERNARY{c}{t}{a}}\SEMI\\
%     \LOCK\SEMI\\
%     \PR{x}{b}    
%   \end{array}
%   \rightsquigarrow
%   \begin{array}{l}
%     \PR{x}{t}\SEMI\\
%     \LET{a}{\TERNARY{c}{t}{a}}\SEMI\\
%     \LOCK\SEMI\\
%     \LET{b}{a}    
%   \end{array}
% \end{displaymath}
% When c is false, x is moved out of the critical region!

% So we have to forbid one transformation.
% \begin{itemize}
% \item C11 forbids load hoisting, allows CSE over lock().
% \item LLVM allows load hoisting, forbids CSE over lock().
% \end{itemize}



\subsection{Further Comparison to ``Pomsets with Preconditions'' [OOPSLA 2020]}
\label{sec:diff}

\PwTmca{} is closely related to \PwP{} model of
\citep{DBLP:journals/pacmpl/JagadeesanJR20}.  The major difference is that
\PwTmca{} supports sequential composition.  In the remainder of this section,
we discuss other differences.  We also point out some errors in
\cite{DBLP:journals/pacmpl/JagadeesanJR20}, all of which have been confirmed
by the authors.

\myparagraph{Substitution}

\jjr{} uses substitution rather than Skolemizing.  Indeed our use of
Skolemization is motivated by disjunction closure for predicate transformers,
which do not appear in \jjr{}.  In \reffig{fig:seq}, 
we gave the semantics of read for nonempty pomsets as:
\begin{enumerate}
\item[{\labeltext[\textsc{r}4a]{(\textsc{r}4a)}{read-tau-dep-oopsla}}]
  if $(\aEvs\cap\bEvs)\neq\emptyset$ then
  \begin{math}
    \aTr{\bEvs}{\bForm} \riff
    \aVal{=}\aReg
    \limplies \bForm
  \end{math},    
\item[{\labeltext[\textsc{r}4b]{(\textsc{r}4b)}{read-tau-ind-oopsla}}]
  if $(\aEvs\cap\bEvs)=\emptyset$ then
  \begin{math}
   \aTr{\bEvs}{\bForm} \riff
    \PBR{\aVal{=}\aReg \lor \aLoc{=}\aReg} \limplies
    \bForm.
  \end{math}
\end{enumerate}
In \jjr{}, the definition is roughly as follows:
% (adding the case for $\ref{L6}$, which was missing):
\begin{enumerate}
\item[{\labeltext[\textsc{r}4a$'$]{(\textsc{r}4a$'$)}{read-tau-dep-oopsla-sub}}]
  if $(\aEvs\cap\bEvs)\neq\emptyset$ then
  \begin{math}
    \aTr{\bEvs}{\bForm} \riff
    \bForm[\aVal/\aReg][\aVal/\aLoc]
    % \aVal{=}\aReg
    % \limplies \bForm[\aReg/\aLoc]
  \end{math},    
\item[{\labeltext[\textsc{r}4b$'$]{(\textsc{r}4b$'$)}{read-tau-ind-oopsla-sub}}]
  if $(\aEvs\cap\bEvs)=\emptyset$ then
  \begin{math}
    \aTr{\bEvs}{\bForm} \riff
    \bForm[\aVal/\aReg][\aVal/\aLoc]\land\bForm[\aLoc/\aReg]
  \end{math}
\end{enumerate}
The use of conjunction in \ref{read-tau-ind-oopsla-sub} causes disjunction closure to fail
because the predicate transformer
% $\aTr{}{\bForm}=\bForm[\aVal/\aReg][\aVal/\aLoc]\land\bForm[\aLoc/\aReg]$ does not distribute through
% disjunction:
% \begin{math}
%   \aTr{}{\bForm_1\lor \bForm_2}=
%   (\bForm_1\lor \bForm_2)[\aVal/\aReg][\aVal/\aLoc]\land(\bForm_1\lor \bForm_2)[\aLoc/\aReg]
%   \neq
%   (\bForm_1[\aVal/\aReg][\aVal/\aLoc]\land\bForm_1[\aLoc/\aReg]) \lor
%   (\bForm_2[\aVal/\aReg][\aVal/\aLoc]\land\bForm_2[\aLoc/\aReg])
%   = \aTr{}{\bForm_1} \lor \aTr{}{\bForm_2}
% \end{math}
$\aTr{}{\bForm}=\bForm'\land\bForm''$ does not distribute through
disjunction, even assuming that the prime operations do:\footnote{%
  \begin{math}
    (\bForm_1\lor \bForm_2)'=(\bForm_1'\lor \bForm_2')
  \end{math}
  and
  \begin{math}
    (\bForm_1\lor \bForm_2)''=(\bForm_1''\lor \bForm_2'')
  \end{math}.
}
\begin{math}
  \aTr{}{\bForm_1\lor \bForm_2}=
  \href{https://www.wolframalpha.com/input/?i=\%28a+or+b\%29+and+\%28c+or+d\%29}{(\bForm_1'\lor \bForm_2')\land(\bForm_1''\lor \bForm_2'')}
  \neq
  \href{https://www.wolframalpha.com/input/?i=\%28a+and+c\%29+or+\%28b+and+d\%29}{(\bForm_1'\land\bForm_1'') \lor (\bForm_2'\land\bForm_2'')}
  = \aTr{}{\bForm_1} \lor \aTr{}{\bForm_2}
\end{math}.
% \begin{math}
%   (\bForm_{1}^{1}\lor \bForm_{1}^{2}) \land (\bForm_{2}^{1}\lor \bForm_{2}^{2})
%   \neq
%   (\bForm_{1}^{1}\land\bForm_{2}^{1}) \lor (\bForm_{1}^{2}\land\bForm_{1}^{2}).
% \end{math}
See also \textsection\ref{sec:ex:assoc}.

The substitutions collapse $\aLoc$ and $\aReg$, allowing local invariant
reasoning (\xLIR{}), as required by \jmm{} causality test case 1, discussed in
\textsection\ref{sec:q}.  Without Skolemizing it is necessary to
substitute $[\aLoc/\aReg]$, since the reverse substitution $[\aReg/\aLoc]$ is
useless when $\aReg$ is bound---compare with
\textsection\ref{sec:substitutions}.  As discussed below (\ref{p:downset}),
including this substitution affects the interaction of \xLIR{} and downset
closure.

Removing the substitution of $[x/r]$ in the independent case has a technical
advantage: we no longer require \emph{extended} expressions (which include
memory references), since substitutions no longer introduce memory
references.

\begin{scope}
  The substitution $[x/r]$ does not work with Skolemization, even for the
  dependent case, since we lose the unique marker for each read.  In effect,
  this forces all reads of a location to see the same values.
  % To be concrete, the candidate
  % definition would modify \ref{L4} to be:
  % \begin{enumerate}
  % \item[\ref{L4})]
  %   $\aTr{\bEvs}{\bForm} \riff \aVal{=}\aLoc\limplies\bForm[\aLoc/\aReg]$.
  %   % \item[\ref{L5})]
  %   %   $\aTr{\cEvs}{\bForm} \riff (\aVal{=}\aLoc\lor\TRUE)\limplies\bForm[\aLoc/\aReg]$. %, when $\aEvs\neq\emptyset$,
  %   % \item[\ref{L6})] 
  %   %   $\aTr{\dEvs}{\bForm}\; \riff \bForm$, when $\aEvs=\emptyset$.
  % \end{enumerate}
  Using this definition, consider the following:
  \begin{gather*}
    \PR{x}{r}\SEMI
    \PR{x}{s}\SEMI
    \IF{r{<}s}\THEN \PW{y}{1}\FI 
    \\[-1ex]
    \hbox{\begin{tikzinline}[node distance=0.5em and 1.5em]
        \event{a1}{\DR{x}{1}}{}
        \event{a2}{\DR{x}{2}}{right=of a1}
        \event{a3}{1{=}x\limplies 2{=}x\limplies x{<} x\bigmid\DW{y}{1}}{right=of a2}
        \po[out=20,in=160]{a1}{a3}
        \po{a2}{a3}
      \end{tikzinline}}
  \end{gather*}
  Although the execution seems reasonable, the precondition on the write is
  not a tautology.
\end{scope}


% There, item \ref{loadpre-kappa2}  of $\sLOADPRE{}{}{}$ is written 
% \begin{enumerate}
% \item[] %[\ref{loadpre-kappa2})]
%   if $\aEv\in\aEvs_2\setminus\aEvs_1$ then either \\
%   $\labelingForm(\aEv) \riff \labelingForm_2(\aEv)[\aLoc/\aReg][\aVal/\aLoc]$ and $(\exists\bEv\in\aEvs_1)\bEv{<}\aEv$, or \\
%   $\labelingForm(\aEv) \riff \labelingForm_2(\aEv)[\aLoc/\aReg][\aVal/\aLoc] \land \labelingForm_2(\aEv)[\aLoc/\aReg]$.
% \end{enumerate}


% [Skolemization ensures disjunction closure, which is necessary
% for associativity. Show example.]

\myparagraph[p:downset]{Downset closure}

\jjr{} enforces downset closure in the prefixing rule.  Even without this,
downset closure would be different for the two semantics, due to the use of
substitution in \jjr{}.  Consider the final pomset in the last example of
\textsection\ref{sec:downset} under the semantics of this paper, which elides
the middle read event:
\begin{align*}
  \begin{gathered}[t]
    \PW{x}{0} 
    \SEMI\PR{x}{r} 
    \SEMI\IF{r{\geq}0}\THEN \PW{y}{1} \FI
    \\
    \hbox{\begin{tikzinline}[node distance=.5em and 1.5em]
        \event{a0}{\DW{x}{0}}{}
        % \event{a1}{\DR{x}{1}}{right=of a0}
        \event{a2}{r{\geq}0\bigmid\DW{y}{1}}{right=3em of a1}      
        % \wk{a0}{a1}
      \end{tikzinline}}    
  \end{gathered}
\end{align*}
In \jjr{}, the substitution $[x/r]$ is performed by the middle read
regardless of whether it is included in the pomset, with the subsequent
substitution of $[0/x]$ by the preceding write, we have $[x/r][0/x]$, which
is $[0/r][0/x]$, resulting in:
\begin{align*}
  \begin{gathered}[t]
    \hbox{\begin{tikzinline}[node distance=.5em and 1.5em]
        \event{a0}{\DW{x}{0}}{}
        % \event{a1}{\DR{x}{1}}{right=of a0}
        \event{a2}{0{\geq}0\bigmid\DW{y}{1}}{right=3em of a1}      
        % \wk{a0}{a1}
      \end{tikzinline}}    
  \end{gathered}
\end{align*}



\myparagraph{Consistency}
\jjr{} imposes \emph{consistency}, which requires that for every pomset
$\aPS$, $\bigwedge_{\aEv}\labelingForm(\aEv)$ is satisfiable.  
\begin{scope}
  Associativity requires that we allow pomsets with inconsistent
  preconditions.  Consider a variant of the example from \textsection\ref{sec:semca}.
  \begin{scope}
    \footnotesize
    \begin{align*}
      \begin{gathered}
        \IF{\aExp}\THEN\PW{x}{1}\FI
        \\
        \hbox{\begin{tikzinline}[node distance=1em]
            \event{a}{\aExp\bigmid\DW{x}{1}}{}
          \end{tikzinline}}
      \end{gathered}
      &&
      \begin{gathered}
        \IF{\BANG\aExp}\THEN\PW{x}{1}\FI
        \\
        \hbox{\begin{tikzinline}[node distance=1em]
            \event{a}{\lnot\aExp\bigmid\DW{x}{1}}{}
          \end{tikzinline}}
      \end{gathered}
      &&
      \begin{gathered}
        \IF{\aExp}\THEN\PW{y}{1}\FI
        \\
        \hbox{\begin{tikzinline}[node distance=1em]
            \event{a}{\aExp\bigmid\DW{y}{1}}{}
          \end{tikzinline}}
      \end{gathered}
      &&
      \begin{gathered}
        \IF{\BANG\aExp}\THEN\PW{y}{1}\FI
        \\
        \hbox{\begin{tikzinline}[node distance=1em]
            \event{a}{\lnot\aExp\bigmid\DW{y}{1}}{}
          \end{tikzinline}}
      \end{gathered}
    \end{align*}
  \end{scope}
  Associating left and right, we have:
  \begin{scope}
    \footnotesize
    \begin{align*}
      \begin{gathered}
        \IF{\aExp}\THEN\PW{x}{1}\FI
        \SEMI
        \IF{\BANG\aExp}\THEN\PW{x}{1}\FI
        \\
        \hbox{\begin{tikzinline}[node distance=1em]
            \event{a}{\DW{x}{1}}{}
          \end{tikzinline}}
      \end{gathered}
      &&
      \begin{gathered}
        \IF{\aExp}\THEN\PW{y}{1}\FI
        \SEMI
        \IF{\BANG\aExp}\THEN\PW{y}{1}\FI
        \\
        \hbox{\begin{tikzinline}[node distance=1em]
            \event{a}{\DW{y}{1}}{}
          \end{tikzinline}}
      \end{gathered}
    \end{align*}
  \end{scope}  
  Associating into the middle, instead, we require:
  \begin{scope}
    \footnotesize
    \begin{align*}
      \begin{gathered}
        \IF{\aExp}\THEN\PW{x}{1}\FI
        \\
        \hbox{\begin{tikzinline}[node distance=1em]
            \event{a}{\aExp\bigmid\DW{x}{1}}{}
          \end{tikzinline}}
      \end{gathered}
      &&
      \begin{gathered}
        \IF{\BANG\aExp}\THEN\PW{x}{1}\FI
        \SEMI
        \IF{\aExp}\THEN\PW{y}{1}\FI
        \\
        \hbox{\begin{tikzinline}[node distance=1em]
            \event{a}{\lnot\aExp\bigmid\DW{x}{1}}{}
            \event{b}{\aExp\bigmid\DW{y}{1}}{right=of a}
          \end{tikzinline}}
      \end{gathered}
      &&
      \begin{gathered}
        \IF{\BANG\aExp}\THEN\PW{y}{1}\FI
        \\
        \hbox{\begin{tikzinline}[node distance=1em]
            \event{a}{\lnot\aExp\bigmid\DW{y}{1}}{}
          \end{tikzinline}}
      \end{gathered}
    \end{align*}
  \end{scope}
  Joining left and right, we have:
  \begin{scope}
    \footnotesize
    \begin{align*}
      \begin{gathered}
        \IF{\aExp}\THEN\PW{x}{1}\FI
        \SEMI
        \IF{\BANG\aExp}\THEN\PW{x}{1}\FI
        \SEMI
        \IF{\aExp}\THEN\PW{y}{1}\FI
        \SEMI
        \IF{\BANG\aExp}\THEN\PW{y}{1}\FI
        \\
        \hbox{\begin{tikzinline}[node distance=1em]
            \event{a}{\DW{x}{1}}{}
            \event{b}{\DW{y}{1}}{right=of a}
          \end{tikzinline}}
      \end{gathered}
    \end{align*}
  \end{scope}  
\end{scope}

\myparagraph{Causal Strengthening}
% \labeltext[]{Causal Strengthening}{xCausal}
\jjr{} imposes \emph{causal strengthening}, which requires for every pomset
$\aPS$, if $\bEv\lt\aEv$ then $\labelingForm(\aEv) \rimplies \labelingForm(\bEv)$. 
\begin{scope}
  Associativity requires that we allow pomsets without causal strengthening.
  Consider the following.
  \begin{align*}
    \begin{gathered}
      \IF{\aExp}\THEN\PR{x}{r}\FI
      \\
      \hbox{\begin{tikzinline}[node distance=1em]
          \event{a}{\aExp\bigmid\DR{x}{1}}{}
        \end{tikzinline}}
    \end{gathered}
    &&
    \begin{gathered}
      \PW{y}{r}
      \\
      \hbox{\begin{tikzinline}[node distance=1em]
          \event{a}{r{=}1\bigmid\DW{y}{1}}{}
        \end{tikzinline}}
    \end{gathered}
    &&
    \begin{gathered}
      \IF{\BANG\aExp}\THEN\PR{x}{s}\FI
      \\
      \hbox{\begin{tikzinline}[node distance=1em]
          \event{a}{\lnot\aExp\bigmid\DR{x}{1}}{}
        \end{tikzinline}}
    \end{gathered}
  \end{align*}
  Associating left, with causal strengthening:
  \begin{align*}
    \begin{gathered}
      \IF{\aExp}\THEN\PR{x}{r}\FI
      \SEMI
      \PW{y}{r}
      \\
      \hbox{\begin{tikzinline}[node distance=1em]
          \event{a}{\aExp\bigmid\DR{x}{1}}{}
          \event{b}{\aExp\bigmid\DW{y}{1}}{right=of a}
          \po{a}{b}
        \end{tikzinline}}
    \end{gathered}
    &&
    \begin{gathered}
      \IF{\BANG\aExp}\THEN\PR{x}{s}\FI
      \\
      \hbox{\begin{tikzinline}[node distance=1em]
          \event{a}{\lnot\aExp\bigmid\DR{x}{1}}{}
        \end{tikzinline}}
    \end{gathered}
  \end{align*}
  Finally, merging:
  \begin{align*}
    \begin{gathered}
      \IF{\aExp}\THEN\PR{x}{r}\FI
      \SEMI
      \PW{y}{r}
      \SEMI
      \IF{\BANG\aExp}\THEN\PR{x}{s}\FI
      \\
      \hbox{\begin{tikzinline}[node distance=1em]
          \event{a}{\DR{x}{1}}{}
          \event{b}{\aExp\bigmid\DW{y}{1}}{right=of a}
          \po{a}{b}
        \end{tikzinline}}
    \end{gathered}
  \end{align*}
  Instead, associating right:
  \begin{align*}
    \begin{gathered}
      \IF{\aExp}\THEN\PR{x}{r}\FI
      \\
      \hbox{\begin{tikzinline}[node distance=1em]
          \event{a}{\aExp\bigmid\DR{x}{1}}{}
        \end{tikzinline}}
    \end{gathered}
    &&
    \begin{gathered}
      \PW{y}{r}
      \SEMI
      \IF{\BANG\aExp}\THEN\PR{x}{s}\FI
      \\
      \hbox{\begin{tikzinline}[node distance=1em]
          \event{a}{\lnot\aExp\bigmid\DR{x}{1}}{}
          \event{b}{r{=}1\bigmid\DW{y}{1}}{left=of a}
        \end{tikzinline}}
    \end{gathered}
  \end{align*}
  Merging:
  \begin{align*}
    \begin{gathered}
      \IF{\aExp}\THEN\PR{x}{r}\FI
      \SEMI
      \PW{y}{r}
      \SEMI
      \IF{\BANG\aExp}\THEN\PR{x}{s}\FI
      \\
      \hbox{\begin{tikzinline}[node distance=1em]
          \event{a}{\DR{x}{1}}{}
          \event{b}{\DW{y}{1}}{right=of a}
          \po{a}{b}
        \end{tikzinline}}
    \end{gathered}
  \end{align*}
  With causal strengthening, the precondition of $\DW{y}{1}$ depends upon how
  we associate.  This is not an issue in \jjr{}, which always associates to
  the right.
\end{scope}

% \myparagraph{Causal Strengthening and Address Dependencies}
% \labeltext[]{Causal Strengthening and Address Dependencies}{xADDRxRRD}

\begin{scope}  
  One use of causal strengthening is to ensure that address dependencies do
  not introduce thin air reads.  Associating to the right, the intermediate
  state of \ref{addr2} (\textsection\ref{sec:addr}) is:
  \begin{align*}
    \begin{gathered}[t]
      \PR{\REF{r}}{s}
      \SEMI
      \PW{x}{s}
      \\
      \hbox{\begin{tikzinline}[node distance=.5em and 1.5em]
          \event{a2}{r\EQ2\bigmid\DR{\REF{2}}{1}}{}
          \event{a3}{(r\EQ2\limplies 1\EQ s) \limplies s\EQ1\bigmid\DW{x}{1}}{right=of a2}
          \po{a2}{a3}
        \end{tikzinline}}
    \end{gathered}
  \end{align*}
  In \jjr{}, we have, instead:
  \begin{gather*}
    % \begin{gathered}[t]
    %   \PW{x}{s}
    %   \\
    %   \hbox{\begin{tikzinline}[node distance=.5em and 1.5em]
    %     \event{b}{s\EQ1\bigmid\DW{x}{1}}{}
    %   \end{tikzinline}}
    % \end{gathered}
    % \\
    \begin{gathered}
      % \PR{y}{r}\SEMI
      \PR{\REF{r}}{s}\SEMI \PW{x}{s}
      \\
      \hbox{\begin{tikzinline}[node distance=.5em and 1.5em]
          % \event{a1}{\DR{y}{2}}{}
          \event{a2}{r\EQ2\bigmid\DR{\REF{2}}{1}}{}%right=of a1}
          \event{a3}{r\EQ2\land\REF{2}\EQ1\bigmid\DW{x}{1}}{right=of a2}
          \po{a2}{a3}
        \end{tikzinline}}
    \end{gathered}
  \end{gather*}
  Without causal strengthening, the precondition of $\DWP{x}{1}$ would be
  simply $\REF{2}\EQ1$.  The treatment in this paper, using implication
  rather than conjunction, is more precise.
\end{scope}

\myparagraph{Internal Acquiring Reads}

The proof of compilation to Arm in \jjr{} assumes that all internal reads can
be eliminated.
% Shortly after publication, \citet{anton}
% noticed a shortcoming of the implementation on \armeight{} in
% \jjr{\textsection 7}.  The proof given there assumes that all internal reads
% can be dropped.
However, this is not the case for acquiring reads.  For example, \jjr{}
disallows the following execution, where the final values of $x$ is $2$ and
the final value of $y$ is $2$.  This execution is allowed by \armeight{} and
\tso{}.
\begin{gather*}
  \PW{x}{2}\SEMI 
  \PR[\mACQ]{x}{r}\SEMI
  \PR{y}{s} \PAR
  \PW{y}{2}\SEMI
  \PW[\mREL]{x}{1}
  \\
  \hbox{\begin{tikzinline}[node distance=1.5em]
      \event{a}{\DW{x}{2}}{}
      \raevent{b}{\DR[\mACQ]{x}{2}}{right=of a}
      \event{c}{\DR{y}{0}}{right=of b}
      \event{d}{\DW{y}{2}}{right=2.5em of c}
      \raevent{e}{\DW[\mREL]{x}{1}}{right=of d}
      \rf{a}{b}
      \sync{b}{c}
      \wk{c}{d}
      \sync{d}{e}
      \wk[out=-165,in=-15]{e}{a}
      % \rfi{a}{b}
      % \bob{b}{c}
      % \fre{c}{d}
      % \bob{d}{e}
      % \coe[out=-165,in=-15]{e}{a}
    \end{tikzinline}}
\end{gather*}
We discuss two approaches to this problem in \textsection\ref{sec:arm}.
% The solution we have adopted is to allow an acquiring read to be downgraded
% to a relaxed read when it is preceded (sequentially) by a relaxed write that
% could fulfill it.  This solution allows executions that are not allowed under
% \armeight{} since we do not insist that the local relaxed write is actually
% read from.  This may seem counterintuitive, but we don't see a local way to
% be more precise.

% As a result, we use a different proof strategy for \armeight{}
% implementation, which does not rely on read elimination.  The proof idea uses
% a recent alternative characterization of \armeight{}
% \citep{alglave-git-alternate,arm-reference-manual}. %,armed-cats}.

\myparagraph{Redundant Read Elimination}

Contrary to the claim, redundant read elimination fails for \jjr{}.
We discuss redundant read elimination in \textsection\ref{sec:semreg}.
Consider JMM Causality Test Case 2, which we describe there.
\begin{gather*}
  \PR{x}{r}\SEMI
  \PR{x}{s}\SEMI
  \IF{r{=}s}\THEN \PW{y}{1}\FI
  \PAR
  \PW{x}{y}
  \\
  \hbox{\begin{tikzinline}[node distance=1.5em]
      \event{a1}{\DR{x}{1}}{}
      \event{a2}{\DR{x}{1}}{right=of a1}
      \event{a3}{\DW{y}{1}}{right=of a2}
      \event{b1}{\DR{y}{1}}{right=3em of a3}
      \event{b2}{\DW{x}{1}}{right=of b1}
      \rf{a3}{b1}
      \po{b1}{b2}
      \rf[out=-169,in=-11]{b2}{a2}
      \rf[out=-169,in=-11]{b2}{a1}
    \end{tikzinline}}
\end{gather*}
Under the semantics of \jjr{}, we have
\begin{gather*}
  \PR{x}{r}\SEMI
  \PR{x}{s}\SEMI
  \IF{r{=}s}\THEN \PW{y}{1}\FI
  \\
  \hbox{\begin{tikzinline}[node distance=1.5em]
      \event{a1}{\DR{x}{1}}{}
      \event{a2}{\DR{x}{1}}{right=of a1}
      \event{a3}{1\EQ1\land1\EQ x \land x\EQ1 \land x=x\bigmid\DW{y}{1}}{right=of a2}
    \end{tikzinline}}
\end{gather*}
The precondition of $\DWP{y}{1}$ is \emph{not} a tautology, and therefore
redundant read elimination fails.
(It is a tautology in
\begin{math}
  \PR{x}{r}\SEMI
  \LET{s}{r}\SEMI
  \IF{r{=}s}\THEN \PW{y}{1}\FI
\end{math}.)
\jjr{\textsection3.1} incorrectly stated that the precondition of
$\DWP{y}{1}$ was $1\EQ1\land x\EQ x$.  


\myparagraph{Termination Conditions and Parallel Composition}

In \jjr{\textsection2.4}, parallel composition is defined allowing coalescing
of events.  Here we have forbidden coalescing.  This difference appears to be
arbitrary.  In \jjr{}, however, there is a mistake in the handling of
termination actions.  The predicates should be joined using $\land$, not
$\lor$.  Here we have used termination conditions rather than termination
actions so that termination is handled separately.

\myparagraph{Read-Modify-Write Actions}

In \jjr{}, the atomicity axioms \ref{pom-rmw-atomic} erroneously applies only to
overlapping writes, not overlapping reads.  The difficulty can be seen in
\refex{ex:rmw-33}.

In addition, \jjr{} uses $\sLOAD{}{}$ instead of $\sLOADP{}{}$ when
calculating of dependency for \RMW{}s.  For a discussion, see the example at
the end of \textsection\ref{sec:rmw}.

\myparagraph{Data Race Freedom}

The definition of data race is wrong in \jjr{}.  It should require that that
at least one action is relaxed.

Note that the definition of \emph{$L$-stable} applies in the case that
conflicting writes are totally ordered.  This gives a result more in the
spirit of \cite{Dolan:2018:BDR:3192366.3192421}.  In particular, this special
case of the theorem clarifies the discussion of the \textsc{past} example
in \jjr{};


\myparagraph{Augmentation of Preconditions}
\jjr{} allows arbitrary augmentation of preconditions.  Here we are more
conservative, only allowing augmentation of preconditions in the semantics of
primitive actions, as in \textsection\ref{sec:semca}.
As discussed in \textsection\ref{sec:delay}, allowing arbitrary augmentation
causes associativity to fail when encoding $\rdelay$ logically. 
% Thus, we use \emph{weakest} preconditions, rather than general preconditions.
% As a result, we fail to validate the following
% refinement:
% \begin{math}
%   \aPSS_1
%   \not\supseteq
%   \xIFTHEN{\aForm}{\aPSS_1}{}.
% \end{math}









\endinput




Precondition of $\DWP{y}{1}$ is $(r{=}s)$ in
\begin{math}
  \sem{\IF{r{=}s}\THEN \PW{y}{1}\FI}.
\end{math}
Predicate transformers for $\emptyset$ in $\sem{\PR{x}{r}}$ and $\sem{\PR{x}{s}}$ are
\begin{align*}
  \PREDP{(r{=}1 \lor r{=}x)\limplies\bForm[r/x]},
  \\
  \PREDP{(s{=}1 \lor s{=}x)\limplies\bForm[s/x]}.
\end{align*}
Combining the transformers, we have
\begin{displaymath}
  \PREDP{(r{=}1 \lor r{=}x)\limplies(s{=}1 \lor s{=}r)\limplies\bForm[s/x]}.
\end{displaymath}
Applying this to $(r{=}s)$, we have
\begin{displaymath}
  \PREDP{(r{=}1 \lor r{=}x)\limplies (s{=}1 \lor s{=}r)\limplies (r{=}s)},
\end{displaymath}
which is not a tautology.

Same problem occurs \jjr{}, where we have:
\begin{align*}
  \PREDP{\bForm[v/x,r] \land \bForm[x/r]},
  \\
  \PREDP{\bForm[v/x,s] \land \bForm[x/s]}.
\end{align*}
Combining the transformers, we have
\begin{displaymath}
  \PREDP{\bForm[v/x,r,s] \land \bForm [v/x,r][x/s] \land \bForm[x/r][v/x,s] \land \bForm[x/r,s]}.
\end{displaymath}
Applying this to $(r{=}s)$, we have
\begin{displaymath}
  \PREDP{v{=}v \land v{=}x \land x{=}v \land x{=}x},
\end{displaymath}
which is not a tautology.

The semantics here allows this by coalescing:
\begin{gather*}
  \PR{x}{r}\SEMI
  \PR{x}{s}\SEMI
  \IF{r{=}s}\THEN \PW{y}{1}\FI
  \PAR
  \PW{x}{y}
  \\
  \hbox{\begin{tikzinline}[node distance=1.5em]
      \event{a1}{\DR{x}{1}}{}
      \event{a3}{\DW{y}{1}}{right=of a1}
      \event{b1}{\DR{y}{1}}{right=3em of a3}
      \event{b2}{\DW{x}{1}}{right=of b1}
      \rf{a3}{b1}
      \po{b1}{b2}
      \rf[out=169,in=11]{b2}{a1}
    \end{tikzinline}}
\end{gather*}

In \jjr{\textsection2.6} the semantics of read is defined as follows:
\begin{align*}
  \sem{\PR[\amode]{\aLoc}{\aReg}\SEMI \aCmd} & \eqdef \textstyle\bigcup_\aVal\;
  (\DRmode\aLoc\aVal) \prefix \sem{\aCmd} [\aLoc/\aReg]
\end{align*}
The definition of prefixing$((\aForm \mid \aAct) \prefix \aPSS)$ has several clauses.
The most relevant are as follows, where $\bEv$ is the new event labeled with
$(\aForm \mid \aAct)$ and $\aEv$ is an event from $\aPSS$:
\begin{description}
\item[{\labeltextsc[P4c]{(P4c)}{4c}}]
  If $\bEv$ reads $\aVal$ from $\aLoc$ then either $\aEv=\bEv$ or
  $\labelingForm'(\aEv) \rimplies \labelingForm(\aEv)[\aVal/\aLoc]$.
\item[{\labeltextsc[P5a]{(P5a)}{5a}}]\labeltextsc[P5]{}{5}%
  If $\bEv$ reads and $\aEv$ writes then either $\labelingForm'(\aEv) \rimplies \labelingForm(\aEv)$ or $\bEv\le'\aEv$.
  % \item[{\labeltextsc[P5b]{(P5b)}{5b}}]
  %   If $\bEv$ and $\aEv$ are in conflict then $\bEv\le'\aEv$.
\end{description}

We have discovered two issues with this definition.

The first issue concerns the substitution $[\aLoc/\aReg]$.  It should be
$[\aReg/\aLoc]$.  We noticed this error while developing the alternative
characterization presented here.  The error causes redundant read elimination
to fail in \jjr{}.  As a result, common subexpression elimination also fails.
The problem can be seen in \ref{TC2}.
\begin{gather*}
  \taglabel{TC2}
  \PR{x}{r}\SEMI
  \PR{x}{s}\SEMI
  \IF{r{=}s}\THEN \PW{y}{1}\FI
  \PAR
  \PW{x}{y}
\end{gather*}
% In \jjr{\textsection3.1},
We claimed that \ref{TC2} allowed the following
execution:
\begin{gather*}
  \hbox{\begin{tikzinline}[node distance=1.5em]
      \event{a1}{\DR{x}{1}}{}
      \event{a2}{\DR{x}{1}}{right=of a1}
      \event{a3}{\DW{y}{1}}{right=of a2}
      % \po{a2}{a3}
      % \po[out=15,in=165]{a1}{a3}
      \event{b1}{\DR{y}{1}}{right=3em of a3}
      \event{b2}{\DW{x}{1}}{right=of b1}
      \rf{a3}{b1}
      \po{b1}{b2}
      \rf[out=169,in=11]{b2}{a2}
      \rf[out=169,in=11]{b2}{a1}
    \end{tikzinline}}
\end{gather*}
But this execution is not possible using the semantics of \jjr{}:
$\DWP{y}{1}$ has precondition $r{=}s$ in
\begin{math}
  \sem{\IF{r{=}s}\THEN \PW{y}{1}\FI}.
\end{math}
Given the lack of order in the execution, the precondition of $\DWP{y}{1}$
must entail $r{=}1\land r{=}x$ in 
\begin{math}
  \sem{\PR{x}{s}\SEMI
    \IF{r{=}s}\THEN \PW{y}{1}\FI}.
\end{math}
\ref{4c} imposes $r{=}1$, and \ref{5a} imposes $r{=}x$.  Adding the second
read, the precondition of $\DWP{y}{1}$ must entail both $1{=}1\land 1{=}x$
and also $x{=}1\land x{=}x$.  This can be simplified to $x{=}1$.  This leaves
a requirement that must be satisfied by a preceding write.  Since the
preceding write is the initialization to $0$, the requirement cannot be
satisfied, and the execution is impossible.\footnote{In \jjr{} we ignore the
  middle terms, mistakenly simplifying this to $1{=}1\land x{=}x$.
  Correcting the error, the attempted execution is:
  \begin{gather*}
    \hbox{\begin{tikzinline}[node distance=1.5em]
        \event{a1}{\DR{x}{1}}{}
        \event{a2}{\DR{x}{1}}{right=of a1}
        \event{a3}{\DW{y}{1}}{right=of a2}
        \po{a2}{a3}
        \po[out=-20,in=-160]{a1}{a3}
        \event{b1}{\DR{y}{1}}{right=3em of a3}
        \event{b2}{\DW{x}{1}}{right=of b1}
        \rf{a3}{b1}
        \po{b1}{b2}
        \rf[out=169,in=11]{b2}{a2}
        \rf[out=169,in=11]{b2}{a1}
      \end{tikzinline}}
  \end{gather*}}

The substitution $[\aLoc/\aReg]$ leaves the obligation on $\aLoc$ to be
fulfilled by the preceding write.  Thus, the read does not update the
\emph{value} of $\aLoc$ in subsequent predicates.  The substitution
$[\aReg/\aLoc]$, instead, does update the value of $\aLoc$, thus removing any
obligation on $\aLoc$ for preceding code.

In order to write this, we must update the definition of prefixing reads to
include the register.  Then \ref{4c} becomes:
\begin{description}
\item[\textsc{(p4c)}] If $\bEv$ reads $\aVal$ from $\aLoc$ then either
  $\aEv=\bEv$ or $\labelingForm'(\aEv) \rimplies \labelingForm(\aEv)[\aVal/\aReg]$.
\end{description}

We can then reason with \ref{TC2} as follows: $\DWP{y}{1}$ has precondition
$r{=}s$ in
\begin{math}
  \sem{\IF{r{=}s}\THEN \PW{y}{1}\FI}.
\end{math}
To avoid introducing order in the execution, the precondition of $\DWP{y}{1}$
must entail $r{=}1\land r{=}s$ in 
\begin{math}
  \sem{\PR{x}{s}\SEMI
    \IF{r{=}s}\THEN \PW{y}{1}\FI}.
\end{math}
\ref{4c} imposes $r{=}1$, and \ref{5a} imposes $r{=}x$.  Adding the second
read, the precondition of $\DWP{y}{1}$ must entail both $1{=}1\land 1{=}x$
and also $x{=}1\land x{=}x$.  This can be simplified to $x{=}1$.  This leaves
a requirement that must be satisfied by a preceding write.


With read elimination, the rule for relaxed reads is as follows:
\begin{align*}
  \sem{\PR{\aLoc}{\aReg} \SEMI \aCmd} &\eqdef
  \sem{\aCmd}[\aLoc/\aReg]
  \cup
  \textstyle\bigcup_\aVal\;
  \DRP{\aLoc}{\aVal} \prefix_{\aReg} %\Rdis{\aLoc}{\aVal}
  \sem{\aCmd}[\aReg/\aLoc]
\end{align*}
It is interesting to note that the substitution is $[\aLoc/\aReg]$ on
eliminated reads, and $[\aReg/\aLoc]$ on non-eliminated reads.  Intuitively,
the subsequent value of $\aLoc$ is fixed by an explicit read, but not for an
eliminated read.  In the latter case, the value is fixed by some preceding
action.  The preceding action may itself be a read. This gives rise to some
fear that we might introduce thin-air reads, since we do not enforce
read-read coherence.  But this is not the case.  Consider the following example:
\begin{gather*}
  \PR{x}{r}\SEMI
  \PR{x}{s}\SEMI
  \PW{y}{s}
  \PAR
  \PW{x}{y}
  \\
  \hbox{\begin{tikzinline}[node distance=1.5em]
      \event{a1}{\DR{x}{1}}{}
      \event{a2}{\DR{x}{1}}{right=of a1}
      \event{a3}{\DW{y}{1}}{right=of a2}
      % \po{a2}{a3}
      \po[out=-20,in=-160]{a1}{a3}
      \event{b1}{\DR{y}{1}}{right=3em of a3}
      \event{b2}{\DW{x}{1}}{right=of b1}
      \rf{a3}{b1}
      \po{b1}{b2}
      \rf[out=169,in=11]{b2}{a2}
      \rf[out=169,in=11]{b2}{a1}
    \end{tikzinline}}
  \\
  \hbox{\begin{tikzinline}[node distance=1.5em]
      \event{a1}{\DR{x}{1}}{}
      \internal{a2}{\DR{x}{1}}{right=of a1}
      \event{a3}{\DW{y}{1}}{right=of a2}
      % \po{a2}{a3}
      \po[out=-20,in=-160]{a1}{a3}
      \event{b1}{\DR{y}{1}}{right=3em of a3}
      \event{b2}{\DW{x}{1}}{right=of b1}
      \rf{a3}{b1}
      \po{b1}{b2}
      % \rf[out=169,in=11]{b2}{a2}
      \rf[out=169,in=11]{b2}{a1}
    \end{tikzinline}}
\end{gather*}
But this is not a problem, since fulfillment requires that $\DWP{x}{1}$
precede both reads of $x$.




\subsection{Further Comparison with Sequential Predicate Transformers}

We compare traditional transformers to the dependent-case transformers of
\reffig{fig:seq}. %; thus we consider only totally ordered executions.
% Because
% we only consider the dependent case, we drop the superscript $\aEvs$ on
% $\aTr{\aEvs}{}$ throughout this section.  We also assume that each register
% appears at most once in a program, as we did throughout
% \textsection\ref{sec:model}--\ref{sec:arm}.

% Because of augment closure, we are not interested in isolating the
% \emph{weakest} precondition.  Thus we think of transformers as Hoare triples.
% In addition, 
All programs in our language are strongly normalizing, so we
need not distinguish strong and weak correctness.  In this setting, the Hoare
triple $\hoare{\aForm}{\aCmd}{\bForm}$ holds exactly when
$\aForm \limplies \fwp{\aCmd}{\bForm}$.

Hoare triples do not distinguish thread-local variables from shared
variables.  Thus, the assignment rule applies to all types of storage. The
rules can be written as on the left below:
\begin{align*}
\begin{aligned}
  \fwp{\PW{\aLoc}{\aExp}}{\bForm} &= \bForm[\aExp/\aLoc]
  \\
  \fwp{\LET{\aReg}{\aExp}}{\bForm} &= \bForm[\aExp/\aReg]
  \\
  \fwp{\PR{\aLoc}{\aReg}}{\bForm} &= \aLoc{=}\aReg\limplies\bForm
\end{aligned}
&&
\begin{aligned}
  \trd{\PW{\aLoc}{\aExp}}{\bForm} &= \bForm[\aExp/\aLoc]
  \\
  \trd{\LET{\aReg}{\aExp}}{\bForm} &= \bForm[\aExp/\aReg]
  \\
  \trd{\PR{\aLoc}{\aReg}}{\bForm} &= \aVal{=}\aReg\limplies\bForm &&
  \textwhere \labelingAct(\aEv)=\DR{\aLoc}{\aVal}
\end{aligned}
\end{align*}
Here we have chosen an alternative formulation for the read rule, which is
equivalent to the more traditional $\bForm[\aLoc/\aReg]$, as long as registers
are assigned at most once in a program.  Our predicate transformers for the
dependent case are shown on the right above.  Only the read rule differs from
the traditional one.

For programs where every register is bound and every read is fulfilled, our
dependent transformers are the same as the traditional ones.  Thus, when
comparing to weakest preconditions, let us only consider totally-ordered
executions of our semantics where every read could be fulfilled by prepending
some writes.  For example, we ignore pomsets of $\PW{x}{2}\SEMI\PR{x}{r}$
that read $1$ for $x$.

For example, let $\aCmd_i$ be defined:
% as follows.
\begin{align*}
  \aCmd_1&=\PR{x}{s}\SEMI\PW{x}{s{+}r}
  &  
  \aCmd_2&=\PW{x}{t}\SEMI\aCmd_1
  &  
  \aCmd_3&=\LET{t}{2}\SEMI\LET{r}{5}\SEMI\aCmd_2
\end{align*}
% \begin{itemize}
% \item
%   \begin{math}
%     \fwp{\LET{\aReg}{\aExp}}{\bForm} = \bForm[\aExp/\aReg]
%   \end{math}
% \item
%   \begin{math}
%     \fwp{\PR{\aLoc}{\aReg}\;\,}{\bForm} = %\bForm[x/r]
%     %     (\forall\bReg)
%     %     \aLoc{=}\bReg\limplies\bForm [\bReg/\aLoc]
%     \aLoc{=}\aReg\limplies\bForm
%   \end{math}
% \item
%   \begin{math}
%     \fwp{\PW{\aLoc}{\aExp}}{\bForm} = \bForm[\aExp/\aLoc]
%   \end{math}
% \end{itemize}
% General relation between Hoare triples and $\fwp{}{}$:
% \begin{itemize}
% \item $\hoare{\fwp{\aCmd}{\bForm}}{\aCmd}{\bForm}$,
% \item If $\hoare{\aForm}{\aCmd}{\bForm}$ and $\aCmd$ terminates when starting
%   in any state satisfying $\aForm$, then $\aForm \limplies \fwp{\aCmd}{\bForm}$.
% \end{itemize}
The following pomset appears in the semantics of $\aCmd_2$.  A pomset for
$\aCmd_3$ can be derived by substituting $[2/t,\allowbreak5/r]$.  A pomset
for $\aCmd_1$ can be derived by eliminating the initial write.
\begin{gather*}
  % \begin{gathered}[t]
  %   \PW{x}{3}
  %   \\
  %   \hbox{\begin{tikzinline}[node distance=.5em and 1.5em]
  %     \event{c}{\DW{x}{3}}{}
  %     \xform{xd}{\bForm}{below=of c}
  %     \xo[xright]{c}{xd}
  %   \end{tikzinline}}
  % \end{gathered}
  % \qquad\quad
  % \begin{gathered}[t]
  %   \PR{x}{s}\SEMI\PW{x}{s{+}r}
  %   \\
  %   \hbox{\begin{tikzinline}[node distance=.5em and 1.5em]
  %     \event{a}{\DR{x}{2}}{}
  %     \event{b}{2{=}s\limplies(s{+}r){=}7\bigmid\DW{x}{7}}{right=of a}%6.5em of a}
  %     \po{a}{b}
  %     \xform{xdd}{2{=}s \limplies \bForm[s{+}r/x]}{below right=.5em and -1em of a}
  %       %     \xform{xdd}{2{=}s \limplies \bForm[s{+}r/x]}{above=of a}
  %       %     \xform{xdi}{2{=}s \limplies \bForm[s{+}r/x]}{below=of a}
  %       %     \xform{xii}{(2{=}s\lor x{=}s)\limplies\bForm[s{+}r/x]}{above=of b}
  %       %     \xform{xid}{(2{=}s\lor x{=}s)\limplies\bForm[s{+}r/x]}{below=of b}
  %       %     \xo[xright]{a}{xdi}
  %       %     \xo[xright]{b}{xid}
  %     \xo[xright]{a}{xdd}
  %     \xo[xright]{b}{xdd}
  %   \end{tikzinline}}
  % \end{gathered}
  % \\[1ex]
  \begin{gathered}[t]
    % \LET{t}{2}\SEMI
    % \LET{r}{5}\SEMI
    \PW{x}{t}\SEMI
    \PR{x}{s}\SEMI\PW{x}{s{+}r}
    \\
    \hbox{\begin{tikzinline}[node distance=.5em and 1.5em]
        \event{a}{\DR{x}{2}}{}
        \event{b}{2{=}s\limplies(s{+}r){=}7\bigmid\DW{x}{7}}{right=of a}
        \event{c}{t{=}2\bigmid\DW{x}{2}}{left=of a}
        \xform{xdd}{2{=}s \limplies \bForm[s{+}r/x]}{right=of b}%below right=.5em and -1em of a}
        %\xo{a}{xdd}
        \xo{b}{xdd}
        % \xo{c}{xdd}
        \po{a}{b}
        \rf{c}{a}
      \end{tikzinline}}
  \end{gathered}
\end{gather*}
The predicate transformers are:
% \begin{align*}
%   \fwp{\aCmd_1}{\bForm} &= x{=}s\limplies\bForm[s{+}r/x] 
%   \\
%   \fwp{\aCmd_2}{\bForm} &= t\,{=}s\limplies\bForm[s{+}r/x] 
%   \\
%   \fwp{\aCmd_3}{\bForm} &= 2{=}s\limplies\bForm[s{+}5/x] 
%   \\
%   \trd{\aCmd_1}{\bForm} = \trd{\aCmd_2}{\bForm} &= 2{=}s\limplies\bForm[s{+}r/x] 
%   \\
%   \trd{\aCmd_3}{\bForm} &= 2{=}s\limplies\bForm[s{+}5/x] 
% \end{align*}
\begin{scope}
  \small
  \begin{align*}
    \fwp{\aCmd_1}{\bForm} &= x{=}s\limplies\bForm[s{+}r/x] 
    &
    \trd{\aCmd_1}{\bForm} &= 2{=}s\limplies\bForm[s{+}r/x] 
    \\
    \fwp{\aCmd_2}{\bForm} &= t\,{=}s\limplies\bForm[s{+}r/x] 
    &
    \trd{\aCmd_2}{\bForm} &= 2{=}s\limplies\bForm[s{+}r/x] 
    \\
    \fwp{\aCmd_3}{\bForm} &= 2{=}s\limplies\bForm[s{+}5/x] 
    &
    \trd{\aCmd_3}{\bForm} &= 2{=}s\limplies\bForm[s{+}5/x] 
  \end{align*}
\end{scope}

% % Let $\rho:\Reg\fun\Val$ and $\chi:\Loc\fun\Val$ be substitutions.
% Let $\aState$ and $\rho$ range over substitutions $(\Reg\cup\Loc)\fun\Val$.
% Treating substitutions as states, the big-step operational semantics of
% programs can be defined as a relation $\bigstep{\aState}{\aCmd}{\bState}$.
% % Ie, $\aForm\aState$ implies $\bForm\bState$.
% \begin{align}
%   \label{wp1}
%   \bigstep{[5/r,2/x]}{\aCmd_1&}{[5/r,2/s,7/x]}
%   \\
%   \label{wp2}
%   \bigstep{[\NEG5/r,2/x]}{\aCmd_1&}{[\NEG5/r,2/s,\NEG3/x]}
% \end{align}

% Then the semantics of Hoare triples guarantees that if
% $\aForm\limplies\fwp{\aCmd}{\bForm}$, $\bigstep{\aState}{\aCmd}{\rho}$ and
% $\aForm\aState$ is a tautology then $\bForm\bState$ is a tautology.
% \begin{align*}
%   \fwp{\aCmd_1}{x{>}0} &= (x{+}r{>}0) 
% \end{align*}
% In \eqref{wp1}, the pre- and post-conditions are satisfied.
% In \eqref{wp2}, they are not.


% \begin{itemize}  
% \item Suppose $\bigstep{\aState}{\aCmd}{\rho}$ and $\aForm\limplies\fwp{\aCmd}{\bForm}$.\\
%   If $\aForm\aState$ is a tautology then $\bForm\bState$ is a tautology.\\
%   Ie, $\aForm\aState$ implies $\bForm\bState$.
% \item Suppose $\bigstep{\aState}{\aCmd}{\rho}$ and $\hoare{\aForm}{\aCmd}{\bForm}$.\\
%   If $\aForm\aState$ is a tautology then $\bForm\bState$ is a tautology.\\
%   Ie, $\aForm\aState$ implies $\bForm\bState$.
% \item Suppose $\bigstep{\aState}{\aCmd}{\rho}$ and $\aForm=\fwp{\aCmd}{\bForm}$.\\
%   $\aForm\aState$ is a tautology if and only if $\bForm\bState$ is a tautology.\\
%   Ie, $\aForm\aState$ iff $\bForm\bState$.
% % \item Weakest: If $\aForm'\aState$ is a tautology, then $\aForm$ implies $\aForm'$.
% \end{itemize}
% Weakest preconditions are \emph{sound} in that if $\aForm$ holds in the
% initial state $\aState$, then $\bForm$ holds in the final state $\bState$.
% Formally, 


\begin{comment}
  If $\aPS\in\sem{\aCmd}$ is top-level and quiescent then 
  $\aTr{\aEvs}{\bForm}$ implies $\fwp{\aCmd}{\bForm}$.

  For any substitution $\aSub=[{v_1/r_1},\ldots, {v_n/r_n}]$ there is some
  $\aPS\in\sem{\aCmd}$ %that is top-level and quiescent
  such that all preconditions in $\aPS\aSub$ are tautologies then 
  $\fwp{\aCmd}{\bForm}\aSub$
\end{comment}


% For a language where all programs are
% terminating, we have for any statement $\aCmd$:
% \begin{align*}
%   \hoare{\aForm}{\aCmd}{\bForm} 
%   \;\;\Leftrightarrow\;\;
%   \aForm \textimplies \fwp{\aCmd}{\bForm}
% \end{align*}
% Interpretation is that if $\aState\models\fwp{\aCmd}{\bForm}$ and
% $\bigstep{\aState}{\aCmd}{\rho}$
% then $\bState\models\bForm$.

% Let $\aCmd_0$ be
% \begin{math}
%   \PW{\aLoc_1}{\aVal_1}\SEMI\cdots\SEMI \PW{\aLoc_n}{\aVal_n}, 
% \end{math}
% such that $\fwp{\aCmd_0}{\aForm}$ is a tautology, and $\aLoc_i=\aLoc_j$
% implies $i=j$.

% Let $\aSub_\aPS=[{\aVal_1/\aLoc_1},\ldots, {\aVal_n/\aLoc_n}]$ be the final
% state of $\aPS$.

% Let $\aState$ and $\rho$ range over substitutions $\Loc\fun\Exp$.
% If we leave the registers free, we have:
% \begin{align}
%   \label{wp1x}
%   \bigstep{[2/x]}{\aCmd&}{[6/x]}
% \end{align}

% Using \refdef{def:pomsets-trans}:
% \begin{align*}
%   \begin{gathered}[t]
%     %     \PR{x}{s}\SEMI\PW{x}{s{+}r}
%     \PR{x}{s}
%     \\
%     \hbox{\begin{tikzinline}[node distance=.5em and 1em]
%       \event{a}{\DR{x}{2}}{}
%       \xform{xi}{\bForm}{above=of a}
%       \xform{xd}{2{=}s \limplies \bForm}{below=of a}
%       \xo[xright]{a}{xd}
%     \end{tikzinline}}
%   \end{gathered}
%   &&
%   \begin{gathered}[t]
%     \PW{x}{s{+}r}
%     \\
%     \hbox{\begin{tikzinline}[node distance=.5em and 1em]
%       \event{a}{(s{+}r){=}7\bigmid\DW{x}{7}}{}
%       \xform{xi}{\bForm}{above=of a}
%       \xform{xd}{\bForm}{below=of a}
%       \xo[xright]{a}{xd}
%     \end{tikzinline}}
%   \end{gathered}
% \end{align*}
% Composing
% \begin{align*}
%   \begin{gathered}[t]
%     \PR{x}{s}\SEMI\PW{x}{s{+}r}
%     \\
%     \hbox{\begin{tikzinline}[node distance=.5em and 1em]
%       \event{a}{\DR{x}{2}}{}
%       \event{b}{(s{+}r){=}7\bigmid\DW{x}{7}}{right=of a}
%       \xform{xdd}{2{=}s \limplies \bForm}{above=of a}
%       \xform{xdi}{2{=}s \limplies \bForm}{below=of a}
%       \xform{xii}{\bForm}{above=of b}
%       \xform{xid}{\bForm}{below=of b}
%       \xo[xright]{a}{xdi}
%       \xo[xright]{b}{xid}
%       \xo[xright]{a}{xdd}
%       \xo[xright]{b}{xdd}
%     \end{tikzinline}}
%   \end{gathered}
% \end{align*}

% Using \refdef{def:pomsets-lir}:
% \begin{align*}
%   \begin{gathered}[t]
%     %     \PR{x}{s}\SEMI\PW{x}{s{+}r}
%     \PR{x}{s}
%     \\
%     \hbox{\begin{tikzinline}[node distance=.5em and 1em]
%       \event{a}{\DR{x}{2}}{}
%       \xform{xi}{(2{=}s\lor x{=}s)\limplies \bForm}{above=of a}
%       \xform{xd}{2{=}s \limplies \bForm}{below=of a}
%       \xo[xright]{a}{xd}
%     \end{tikzinline}}
%   \end{gathered}
%   &&
%   \begin{gathered}[t]
%     \PW{x}{s{+}r}
%     \\
%     \hbox{\begin{tikzinline}[node distance=.5em and 1em]
%       \event{a}{(s{+}r){=}7\bigmid\DW{x}{7}}{}
%       \xform{xi}{\bForm[s{+}r/x]}{above=of a}
%       \xform{xd}{\bForm[s{+}r/x]}{below=of a}
%       \xo[xright]{a}{xd}
%     \end{tikzinline}}
%   \end{gathered}
% \end{align*}
% Composing

% For example, let $\aCmd_1=\PR{x}{r}$ and $\aCmd_2=\PW{x}{r{+}1}$ and
% $\aCmd=\aCmd_1\SEMI \aCmd_2$.
% \begin{align*}
%   \fwp{\aCmd_2}{x{>}1}&=(r{+}1{>}1) = (r{>}0)
%   \\
%   \fwp{\aCmd_1}{r{>}0}=\fwp{\aCmd_0}{x{>}1}&=(x{>}0)
% \end{align*}
% Let $\aPS_i\in\sem{\aCmd_i}$.
% \begin{align*}
%   \aTr[2]{\aEvs_2}{x{>}1}&=(r{+}1{>}1) = (r{>}0)
%   \\
%   \aTr[0]{\aEvs_0}{x{>}1}&=(0{=}\aReg \limplies r{>}0)
%   \\
%   \aTr[0]{\aEvs_0}{x{>}1}&=(1{=}\aReg \limplies r{>}0)
%   \\
%   \aTr[0]{\aEvs_0}{x{>}1}&=(2{=}\aReg \limplies r{>}0)
% \end{align*}

% \begin{proposition}
%   If $\aPS\in\sem{\aCmd}$ is top-level and quiescent then 
%   $\aTr{\aEvs}{\aForm}$ implies $\fwp{\aCmd}{\aForm}$.

%   For any substitution $\aSub=[{\aVal_1/\aReg_1},\ldots, {\aVal_n/\aReg_n}]$ there is some
%   $\aPS\in\sem{\aCmd}$ %that is top-level and quiescent
%   such that all preconditions in $\aPS\aSub$ are tautologies then 
%   $\fwp{\aCmd}{\aForm}\aSub$
% \end{proposition}


\begin{comment}
Dead Store Elimination (Still) Considered Harmful

DSE "removes stores that have no effect on the program result, either
because the stored value is overwritten or because it is never read again."

The example they give is of the second type: "because it is never read
again".  To validate this, we would need a more complex model of memory
using provenance for pointers...

Section 3.3.3 (Volatile Function Pointer) has a lovely example of
if-introduction: Compilers are not allowed to get rid of calls through
virtual functions, but they can use if-introduction and then get rid of the
static call that they introduce on one side of the if/else...

Possibly interesting references:

[28] N. Benton. Simple relational correctness proofs for static analyses and program transformations. In ACM SIGPLAN Notices, volume 39, pages 14–25, 2004.
[29] C. Deng and K. S. Namjoshi. Securing a compiler transformation. In Proceedings of the 23rd Static Analysis Symposium, SAS ’16, pages 170– 188, 2016.
[30] V. D’Silva, M. Payer, and D. Song. The correctness-security gap in compiler optimization. In Security and Privacy Workshops, SPW ’15, pages 73–87, 2015.
[31] X. Leroy. Formal certification of a compiler backend or: programming a compiler with a proof assistant. In ACM SIGPLAN Notices, volume 41, pages 42–54, 2006.
\end{comment}

% \subsection{Post-Hoc Verification of Fulfillment for \PwTmcaTITLE{2}}
% \label{sec:post-hoc}

\subsection{Register Consistency}
\label{sec:false}


% If a precondition is false, you can be pretty sure it's useless.  In this
% subsection, we develop a criterion for eliminating such useless pomsets.

% To achieve this, we would like to bolt a requirement into the definition of
% pomsets in order to weed out the useless ones.  Something like this:
% \begin{enumerate}
% \item[{\labeltext[\textsc{m}3a$'$]{(\textsc{m}3a$'$)}{pom-kappa-sat'}}]
%   $\labelingForm(\aEv)$ is satisfiable.
% \end{enumerate}
% For associativity, \eqref{pom-kappa-sat'} would in turn require

In addition to the three criteria of \refdef{def:trans}
\citet{DBLP:journals/cacm/Dijkstra75} requires
\begin{enumerate}
\item[{\labeltext[\textsc{x}4$'$]{(\textsc{x}4$'$)}{tr-false'}}]
  $\aTr{}{\FALSE}\riff\FALSE$.
\end{enumerate}
% \citet{DBLP:journals/cacm/Dijkstra75} requires exactly \ref{tr-false'}.
% Problem solved!  
Unfortunately, our transformer for read actions
\eqref{read-tau-dep} does not obey \ref{tr-false'}, since $\FALSE$ is not
equivalent to $v{=}r\limplies\FALSE$.

In this subsection, we refine this requirement to one that does hold.  The
main insight is to pull values for registers from the actions of pomset itself.
%
% \eqref{tr-false'} is too strong.
% We say $\aForm$ \emph{conjunctively depends on $\uReg{\aEv}$}
% if $\aForm\rnotimplies\aForm[\aVal/\uReg{\aEv}]$
% %or $\aForm[\aVal/\uReg{\aEv}]\rnotimplies\aForm$,
% for some $\aVal$.
% We say that $\aForm$ \emph{cannot contribute to a top-level tautology} if
% $\aForm\riff\bigvee_{\aEv\in\aEvs}\aForm_\aEv$ where (for every $\aEv$) either
% $\aForm_\aEv\riff\FALSE$ or $\aForm_\aEv$ conjunctively depends on $\uReg{\aEv}$.
%
% We require the following:
% \begin{enumerate}
% \item[{\labeltext[\textsc{x}4]{(\textsc{x}4)}{tr-false}}]
%   $\aTr{}{\FALSE}$ cannot contribute to a top-level tautology.
% \end{enumerate}
%
% When put in \DNF, $\aTr{}{\FALSE}$ look like this (taking independent case
% for both reads):
% \begin{gather*}
%   \IF{r}\THEN \PR{x}{s_d} \ELSE \PR{y}{s_e} \FI
%   \\
%   (r{\neq}0 \land s_d{\neq}v_d \land s_d{\neq}x)
%   \lor  
%   (r{=}0 \land s_e{\neq}v_e \land s_e{\neq}y)  
% \end{gather*}
%
Thus, we define $\regForm{\labeling}$ to capture the \emph{register state} of a pomset.
\begin{definition}  
  \label{def:labeling:consistent}
  Let 
  \begin{math}
    \regForm{\labeling}=
    \textstyle\bigwedge_{\{(\aEv,\aVal)\in(\aEvs\times\Val)\mid\labeling(\aEv)=\DRP{}{\aVal}\}}(\uReg{\aEv}\EQ\aVal)
    \textwhere \aEvs=\fdom(\labeling)
  \end{math}.
  
  We say that $\aForm$ is \emph{$\labeling$-consistent} if $\aForm\land\regForm{\labeling}$ is satisfiable.
  We say that it is \emph{$\labeling$-inconsistent} otherwise.
\end{definition}


%tau maps (things incompatible with chi) to (things incompatible with chi)
Using this, we define the constraint on predicate transformers that we want.
We also need to update the definition of predicate transformer families to
carry the labeling.
\begin{definition}
  \label{def:trans'}
  A \emph{$\labeling$-predicate transformer} is a %monotone
  function
  $\aTr{}{}:\Formulae\fun\Formulae$ such that
  \begin{enumerate}[,label=(\textsc{x}\arabic*),ref=\textsc{x}\arabic*]
  % \item \label{tr-implies'}
  %   \labeltextXX{2}{x}{tr-and'}
  %   \labeltextXX{3}{x}{tr-or'} as in \refdef{def:trans},
  \item[\eqref{tr-and}]
    \eqref{tr-or}\;
    \eqref{tr-implies}\; as in \refdef{def:trans},
  \item[{\labeltext[\textsc{x}4]{(\textsc{x}4)}{tr-false}}] 
    % \item[\eqref{tr-false}]
    if $\bForm$ is $\labeling$-inconsistent then $\aTr{}{\bForm}$ is $\labeling$-inconsistent.
  \end{enumerate}

  \label{def:family'}
  A \emph{family of $\labeling$-predicate transformers} over consists of a
  $\labeling$-predicate transformer $\aTr{\bEvs}{}$ for each
  $\bEvs\subseteq\AllEvents$, such that if $\cEvs \cap \aEvs \subseteq \bEvs$
  then $\aTr{\cEvs}{\bForm} \rimplies \aTr{\bEvs}{\bForm}$.

  
  \begin{enumerate}[,label=(\textsc{m}\arabic*),ref=\textsc{m}\arabic*]
  \setcounter{enumi}{\value{Btau}}
  \item \label{pom-tau'}
    $\aTr{}{}:2^{\AllEvents}\fun\Formulae \fun\Formulae$ is a \emph{family of $\labeling$-predicate transformers}, 
  \end{enumerate}
\end{definition}

% Given these definitions, we can add the following requirement to the model,
% which enables us to prune pomsets that include $\labeling$-inconsistent
% preconditions and termination conditions.
% \begin{multicols}{2}
%   \begin{enumerate}
%     % \item[{\labeltext[\textsc{x}4]{(\textsc{x}4)}{tr-false}}] if $\bForm$ is
%     %   $\labeling$-inconsistent, then $\aTr{}{\bForm}$ is $\labeling$-inconsistent.
%   \item[{\labeltext[\textsc{m}3a]{(\textsc{m}3a)}{pom-kappa-sat}}]
%     $\labelingForm(\aEv)$ is $\labeling$-consistent,
%   \item[{\labeltext[\textsc{m}5b]{(\textsc{m}5b)}{pom-term-sat}}]
%     $\aTerm$ is $\labeling$-consistent.
%   \end{enumerate}
% \end{multicols}
% With this modification, dead-code elimination
% (\reflem{lem:if}\ref{lem:if:dead}) can be changed from an inclusion to an equation:
% \begin{displaymath}
%   \xIFTHEN{\aForm}{\aPSS_1}{\aPSS_2}
%   =
%   \aPSS_1
%   \;\text{if $\aForm$ is a tautology.}
% \end{displaymath}


It would seem reasonable to require that $\labelingForm(\aEv)$ be
$\labeling$-consistent.  However, this breaks associativity.  Compare the
following, where $\uReg{e}=r$:
\begin{align*}
  \begin{gathered}[t]
    \PR{y}{r}
    \SEMI
    \IF{r}\THEN\PW{x}{1}\FI
    \\
    \hbox{\begin{tikzinline}[node distance=1ex and 1.5em]
        \eventl{e}{a}{\DR{y}{1}}{}
        \event{b}{r{\neq}0\bigmid\DW{x}{1}}{right=of a}
        %\xform{xi}{\bForm[\uReg{e}/r]}{left=.5em of a}
      \end{tikzinline}}    
  \end{gathered}
  &&
  \begin{gathered}[t]
    \IF{\BANG r}\THEN\PW{x}{1}\FI
    \\
    \hbox{\begin{tikzinline}[node distance=.5em and 1em]
        \event{c}{r{=}0\bigmid\DW{x}{1}}{}
      \end{tikzinline}}    
  \end{gathered}
\end{align*}
and
\begin{align*}
  \begin{gathered}[t]
    \PR{y}{r}
    \\
    \hbox{\begin{tikzinline}[node distance=1ex and 1.5em]
        \eventl{e}{a}{\DR{y}{1}}{}
        %\xform{xi}{\bForm[\uReg{e}/r]}{left=.5em of a}
      \end{tikzinline}}    
  \end{gathered}
  &&
  \begin{gathered}[t]
    \IF{r}\THEN\PW{x}{1}\FI
    \SEMI
    \IF{\BANG r}\THEN\PW{x}{1}\FI
    \\
    \hbox{\begin{tikzinline}[node distance=.5em and 1em]
        \event{c}{\DW{x}{1}}{}
      \end{tikzinline}}    
  \end{gathered}
\end{align*}

It would also seem reasonable to require that $\aTerm$ be
$\labeling$-consistent in all pomsets.  However, doing so is incompatible
with our approach to untaken conditionals.  Consider that the empty pomset is
in the semantics of $\IF{\FALSE}\THEN\PW{x}{1}\FI$.  In order to construct
the final pomset with $\aTerm\riff\TRUE$, we must allow the
intermediate pomset with $\aTerm\riff\FALSE$.


% \todo{Drop \ref{pom-kappa-sat} and m5b---they are wrong.  Move this into a
%   discussion of the proof of compilation correctness.}


\subsection{The Need for Respect}
In \reffig{fig:seq}, we choose the weakest precondition.  Because of this,
associativity requires that \ref{seq-le} is $\PBR{{\lt}\rextends{\lt_1}{\lt_2}}$
rather than $\PBR{{\lt}\rsupset{\lt_1}{\lt_2}}$.  Consider
\begin{math}
  \PBR{
    \PR{x}{r}
    \SEMI
    \PW{y}{\aExp}
    \SEMI
    \SKIP
  }.
\end{math}
Associating to the left, we might have:
\begin{align*}
  \aPS_{12}=
  \hbox{\begin{tikzinline}[node distance=1em]
      \eventr{d}{d}{\DR{x}{}}{}
      \eventr{e}{e}{\aForm\bigmid \DW{y}{}}{right=of d}
    \end{tikzinline}}
  &&
  \aPS_{3}= \emptyset
  &&
  \aPS=
  \hbox{\begin{tikzinline}[node distance=1em]
      \eventr{d}{d}{\DR{x}{}}{}
      \eventr{e}{e}{\aForm\bigmid \DW{y}{}}{right=of d}
      \po{d}{e}
    \end{tikzinline}}
\end{align*}
When building $\aPS_{12}$, the dependent set of $e$ would be the empty set, and thus
$\aForm$ must have been constructed using the independent transformer
\ref{read-tau-ind}.  Attempting to repeat this, associating to the right:
\begin{align*}
  \aPS_{1}=
  \hbox{\begin{tikzinline}[node distance=1em]
      \eventr{d}{d}{\DR{x}{}}{}
    \end{tikzinline}}
  &&
  \aPS_{23}=
  \hbox{\begin{tikzinline}[node distance=1em]
      \eventr{e}{e}{\aForm'\bigmid \DW{y}{}}{}
    \end{tikzinline}}
  &&
  \aPS'=
  \hbox{\begin{tikzinline}[node distance=1em]
      \eventr{d}{d}{\DR{x}{}}{}
      \eventr{e}{e}{\aForm'\bigmid \DW{y}{}}{right=of d}
      \po{d}{e}
    \end{tikzinline}}
\end{align*}
In $\aPS'$, however, now the dependent set of $e$ is the singleton $\{d\}$; thus $\aForm'$ must be
constructed using the dependent transformer \ref{read-tau-dep}.
Since
\begin{math}
  \PBR{
    \PBR{\aVal{=}\aReg \lor \aLoc{=}\aReg}
    \limplies \bForm
  }
  \not\riff
  \PBR{
    \aVal{=}\aReg
    \limplies \bForm
  },
\end{math}
associativity fails.

If we allow stronger preconditions, as in
\cite{DBLP:journals/pacmpl/JagadeesanJR20}, then we could use inclusion
rather than $\rextendsdef{}{}$.  To arrive at this semantics, one would
replace every occurrence of $\riff$ in \reffig{fig:seq} with $\rimplies$.
Then $\PBR{{\lt}\rextends{\lt_1}{\lt_2}}$ can be replaced by
$\PBR{{\lt}\rsupset{\lt_1}{\lt_2}}$.

\subsection{Write Substitutions}

In the predicate transformer for $\PW{x}{\aExp}$, we substitute $\aExp$ for
$x$.  In an alternative semantics, one could substitute the value chosen for
the action.  This alternative semantics looses dependencies.  Consider:
\begin{gather*}
  \PR{x}{s}\SEMI \PW{z}{s}
  \\
  \hbox{\begin{tikzinline}[node distance=.8em and 1em]
      \event{a3}{\DR{x}{1}}{}
      \event{a4}{\PBR{1{=}s \lor x{=}s} \limplies s\EQ1\bigmid\DW{z}{1}}{right=of a3}
    \end{tikzinline}}
\end{gather*}
Prepending a write and then a read, our semantics gives the following:
\begin{align*}
  \begin{gathered}
    \PW{x}{r} \SEMI \PR{x}{s}\SEMI \PW{z}{s}
    \\
    \hbox{\begin{tikzinline}[node distance=.8em and 1em]
        \event{a2}{\DW{x}{1}}{}
        \event{a3}{\DR{x}{1}}{right=of a2}
        \event{a4}{\PBR{1{=}s \lor r{=}s} \limplies s\EQ1\bigmid\DW{z}{1}}{right=of a3}
      \end{tikzinline}}
  \end{gathered}
  &&
  \begin{gathered}
    \PR{y}{r}\SEMI \PW{x}{r}\SEMI \PR{x}{s} \SEMI \PW{z}{s}
    \\
    \hbox{\begin{tikzinline}[node distance=.8em and 1em]
        \event{a1}{\DR{y}{1}}{}
        \event{a2}{\DW{x}{1}}{right=of a1}
        \event{a3}{\DR{x}{1}}{right=of a2}
        \event{a4}{\DW{z}{1}}{right=of a3}
        \po[out=15,in=165]{a1}{a4}
        \po{a1}{a2}
      \end{tikzinline}}
  \end{gathered}
\end{align*}
With the alternative semantics, instead, we would have:
\begin{align*}
  \begin{gathered}
  \hbox{\begin{tikzinline}[node distance=.8em and 1em]
      \event{a2}{\DW{x}{1}}{}
      \event{a3}{\DR{x}{1}}{right=of a2}
      \event{a4}{\PBR{1{=}s \lor 1{=}s} \limplies s\EQ1\bigmid\DW{z}{1}}{right=of a3}
    \end{tikzinline}}
  \end{gathered}
  &&
  \begin{gathered}
    \PR{y}{r}\SEMI \PW{x}{r}\SEMI \PR{x}{s} \SEMI \PW{z}{s}
    \\
    \hbox{\begin{tikzinline}[node distance=.8em and 1em]
        \event{a1}{\DR{y}{1}}{}
        \event{a2}{\DW{x}{1}}{right=of a1}
        \event{a3}{\DR{x}{1}}{right=of a2}
        \event{a4}{\DW{z}{1}}{right=of a3}
        % \po[out=15,in=165]{a1}{a4}
        \po{a1}{a2}
      \end{tikzinline}}
  \end{gathered}
\end{align*}
The dependency from $\DR{y}{1}$ to $\DW{z}{1}$ has been lost.


\subsection{Read Substitutions}
\label{sec:substitutions}

In $\sLOAD{}{}$, it is also possible to collapse $\aLoc$ and $\aReg$ via substitution:
\begin{enumerate}
\item[{\labeltext[\textsc{r}4a$'$]{(\textsc{r}4a$'$)}{read-tau-dep-sub}}]
  if $(\aEvs\cap\bEvs)\neq\emptyset$ then
  \begin{math}
    \aTr{\bEvs}{\bForm} \riff
    \aVal{=}\aReg
    \limplies \bForm[\aReg/\aLoc]
  \end{math},    
\item[{\labeltext[\textsc{r}4b$'$]{(\textsc{r}4b$'$)}{read-tau-ind-sub}}]
  if $\aEvs\neq\emptyset$ and $(\aEvs\cap\bEvs)=\emptyset$ then
  \begin{math}
    \aTr{\bEvs}{\bForm} \riff
    \PBR{\aVal{=}\aReg \lor \aLoc{=}\aReg} \limplies
    \bForm[\aReg/\aLoc],
  \end{math}
\item[{\labeltext[\textsc{r}4c$'$]{(\textsc{r}4c$'$)}{read-tau-empty-sub}}]
  if $\aEvs=\emptyset$ then
  \begin{math}
    \aTr{\bEvs}{\bForm} \riff
    % \PBR{\aVal{=}\aReg \lor \aLoc{=}\aReg} \limplies
    \bForm[\aReg/\aLoc],
  \end{math}
\end{enumerate}
Perhaps surprisingly, this semantics is incomparable with that of
\reffig{fig:seq}.  Consider the following:
\begin{gather*}
  \IF{r\land s\;\mathsf{even}}\THEN \PW{y}{1}\FI\SEMI
  \IF{r\land s}\THEN \PW{z}{1}\FI
  \\
  \hbox{\begin{tikzinline}[node distance=0.5em and 1.5em]
      \event{a3}{r\land s\;\mathsf{even}\bigmid\DW{y}{1}}{}
      \event{a4}{r\land s\bigmid\DW{z}{1}}{right=of a3}
    \end{tikzinline}}
\end{gather*}
Prepending $\PRP{x}{s}$, we get the same result regardless of whether we
substitute $[s/x]$, since $x$ does not occur in either precondition.  Here
we show the independent case:
\begin{gather*}
  \PR{x}{s}\SEMI
  \IF{r\land s\;\mathsf{even}}\THEN \PW{y}{1}\FI\SEMI
  \IF{r\land s}\THEN \PW{z}{1}\FI
  \\
  \hbox{\begin{tikzinline}[node distance=0.5em and 1.5em]
      \event{a2}{\DR{x}{2}}{}
      \event{a3}{(2{=}s\lor x{=}s)\limplies (r\land s\;\mathsf{even})\bigmid\DW{y}{1}}{right=of a2}
      \event{a4}{(2{=}s\lor x{=}s)\limplies (r\land s)\bigmid\DW{z}{1}}{right=of a3}
    \end{tikzinline}}
\end{gather*}
Since the preconditions mention $x$, prepending $\PRP{x}{r}$, we now get
different results depending on whether we perform the substitution.  Without
any substitution, we have:
\begin{gather*}
  \PR{x}{r}\SEMI
  \PR{x}{s}\SEMI
  \IF{r\land s\;\mathsf{even}}\THEN \PW{y}{1}\FI\SEMI
  \IF{r\land s}\THEN \PW{z}{1}\FI
  \\[-2ex]
  \hbox{\begin{tikzinline}[node distance=0.5em and 1.5em]
      \event{a1}{\DR{x}{1}}{}
      \event{a2}{\DR{x}{2}}{right=of a1}
      \event{a3}{1{=}r\limplies  (2{=}s\lor x{=}s)\limplies (r\land s\;\mathsf{even})\bigmid\DW{y}{1}}{right=of a2}
      \event{a4}{1{=}r\limplies  (2{=}s\lor x{=}s)\limplies (r\land s)\bigmid\DW{z}{1}}{right=of a3}
      \po[out=12,in=168]{a1}{a3}
      \po[out=10,in=170]{a1}{a4}
    \end{tikzinline}}
\end{gather*}
Prepending $\PWP{x}{0}$, which substitutes $[0/x]$, the precondition of
$\DWP{y}{1}$ becomes
$(1{=}r\limplies (2{=}s\lor0{=}s)\limplies (r\land s\;\mathsf{even}))$, which
is a tautology, whereas the precondition of $\DW{z}{1}$ becomes
$(1{=}r\limplies(2{=}s\lor0{=}s)\limplies (r\land s))$, which is not.  In
order to be top-level, $\DWP{z}{1}$ must be dependency ordered after
$\DRP{x}{2}$; in this case the precondition becomes
$(1{=}r\limplies2{=}s\limplies (r\land s))$, which is a tautology.
\begin{gather*}
  % \PW{x}{0}\SEMI
  % \PR{x}{r}\SEMI
  % \PR{x}{s}\SEMI
  % \IF{r\land s\;\mathsf{even}}\THEN \PW{y}{1}\FI\SEMI
  % \IF{r\land s}\THEN \PW{z}{1}\FI
  % \\
  \hbox{\begin{tikzinline}[node distance=1.5em]
      \event{a0}{\DW{x}{0}}{}
      \event{a1}{\DR{x}{1}}{right=of a0}
      \event{a2}{\DR{x}{2}}{right=of a1}
      \event{a3}{\DW{y}{1}}{right=of a2}
      \event{a4}{\DW{z}{1}}{right=of a3}
      % \wk{a0}{a1}
      % \wk[out=-20,in=-160]{a0}{a2}
      \po[out=20,in=160]{a1}{a3}
      \po[out=20,in=160]{a1}{a4}
      \po[out=-20,in=-160]{a2}{a4}
    \end{tikzinline}}
\end{gather*}
The situation reverses with the substitution $[r/x]$:
\begin{gather*}
  \PR{x}{r}\SEMI
  \PR{x}{s}\SEMI
  \IF{r\land s\;\mathsf{even}}\THEN \PW{y}{1}\FI\SEMI
  \IF{r\land s}\THEN \PW{z}{1}\FI
  \\[-2ex]
  \hbox{\begin{tikzinline}[node distance=0.5em and 1.5em]
      \event{a1}{\DR{x}{1}}{}
      \event{a2}{\DR{x}{2}}{right=of a1}
      \event{a3}{1{=}r\limplies  (2{=}s\lor r{=}s)\limplies (r\land s\;\mathsf{even})\bigmid\DW{y}{1}}{right=of a2}
      \event{a4}{1{=}r\limplies  (2{=}s\lor r{=}s)\limplies (r\land s)\bigmid\DW{z}{1}}{right=of a3}
      \po[out=12,in=168]{a1}{a3}
      \po[out=10,in=170]{a1}{a4}
    \end{tikzinline}}
\end{gather*}
Prepending $\PWP{x}{0}$:
%\vspace{-.5\baselineskip}
\begin{gather*}
  % \PW{x}{0}\SEMI
  % \PR{x}{r}\SEMI
  % \PR{x}{s}\SEMI
  % \IF{r\land s\;\mathsf{even}}\THEN \PW{y}{1}\FI\SEMI
  % \IF{r\land s}\THEN \PW{z}{1}\FI
  % \\
  \hbox{\begin{tikzinline}[node distance=1.5em]
      \event{a0}{\DW{x}{0}}{}
      \event{a1}{\DR{x}{1}}{right=of a0}
      \event{a2}{\DR{x}{2}}{right=of a1}
      \event{a3}{\DW{y}{1}}{right=of a2}
      \event{a4}{\DW{z}{1}}{right=of a3}
      % \wk{a0}{a1}
      % \wk[out=-20,in=-160]{a0}{a2}
      \po[out=20,in=160]{a1}{a3}
      \po[out=20,in=160]{a1}{a4}
      \po{a2}{a3}
    \end{tikzinline}}
\end{gather*}
The dependency has changed from $\DRP{x}{2}\xpo\DWP{z}{1}$ to
$\DRP{x}{2}\xpo\DWP{y}{1}$.  The resulting sets of pomsets are
incomparable.


Thinking in terms of hardware, the difference is whether reads update the
cache, thus clobbering preceding writes.  With $[r/x]$, reads clobber the
cache, whereas without the substitution, they do not.  Since most caches work
this way, the model with $[r/x]$ is likely preferred for modeling hardware.
However, this substitution only makes sense in a model with read-read
coherence and read-read dependencies, which is not the case for \armeight{}.  





% \begin{figure*}[t]
%   \showRAtrue
%   \begin{center}
%     \begin{minipage}{.91\textwidth}
%       \renewcommand{\cEvs}{D}
\renewcommand{\dEvs}{D}
\noindent
If $\aPS \in \sSTORE[\amode]{\aLoc}{\aExp}$ then
$(\exists\aVal:\aEvs\fun\Val)$
$(\exists\cForm:\aEvs\fun\Formulae)$
\begin{enumerate}
\item[{\labeltext[S1]{S1)}{S1no-q-or-addr}}] 
  if $\cForm_\bEv\land\cForm_\aEv$ is satisfiable then $\bEv=\aEv$,
\item[{\labeltext[S2]{S2)}{S2no-q-or-addr}}] 
  $\labelingAct(\aEv) = \DW[\amode]{\aLoc}{\aVal_\aEv}$,
\item[{\labeltext[S3]{S3)}{S3no-q-or-addr}}] 
  $\labelingForm(\aEv)$ implies
  \begin{math}
    \cForm_\aEv
    \land \aExp{=}\aVal_\aEv
  \end{math},
  
  
\item[{\labeltext[S4]{S4)}{S4no-q-or-addr}}] 
  \begin{math}
    (\forall\aEv\in\aEvs\cap\bEvs)
  \end{math}
  $\aTr{\bEvs}{\bForm}$ implies 
  \begin{math}
    \cForm_\aEv
    \limplies {
      \bForm
      [\aExp/\aLoc]
      \DS{\aLoc}{\amode}
      [(\Q{}\land\aExp{=}\aVal_\aEv)/\Q{}]
    }
  \end{math},
\item[{\labeltext[S5]{S5)}{S5no-q-or-addr}}] 
  \begin{math}    
    (\forall\aEv\in\aEvs\setminus\cEvs)
  \end{math}
  $\aTr{\cEvs}{\bForm}$ implies
  \begin{math}
    \cForm_\aEv
    \limplies {
      \bForm
      [\aExp/\aLoc]
      \DS{\aLoc}{\amode}
      [\FALSE/\Q{}]
    },
  \end{math}
\item[{\labeltext[S6]{S6)}{S6no-q-or-addr}}] 
  $\aTr{\dEvs}{\bForm}$ implies
  \begin{math}
    (\!\not\exists\aEv\in\aEvs \suchthat \cForm_\aEv)
    \limplies {
      \bForm
      [\aExp/\aLoc]
      \DS{\aLoc}{\amode}
      [\FALSE/\Q{}]
    }.
  \end{math}
\end{enumerate}

\noindent
If $\aPS \in \sLOAD[\amode]{\aReg}{\aLoc}$ then
$(\exists\aVal:\aEvs\fun\Val)$
$(\exists\cForm:\aEvs\fun\Formulae)$
$(\exists\bmode\in\{\amode,\mRLX\})$

\begin{enumerate}
\item[{\labeltext[L1]{L1)}{L1no-q-or-addr}}] 
  if $\cForm_\bEv\land\cForm_\aEv$ is satisfiable then $\bEv=\aEv$,
\item[{\labeltext[L2]{L2)}{L2no-q-or-addr}}] 
  $\labelingAct(\aEv) = \DR[\bmode]{\aLoc}{\aVal_\aEv}$
\item[{\labeltext[L3]{L3)}{L3no-q-or-addr}}] 
  $\labelingForm(\aEv)$ implies
  \begin{math}
    \cForm_\aEv
  \end{math},
    
\item[{\labeltext[L4]{L4)}{L4no-q-or-addr}}] 
  \begin{math}
    (\forall\aEv\in\aEvs\cap\bEvs)
  \end{math}
  $\aTr{\bEvs}{\bForm}$ implies
  \begin{math}
    \cForm_\aEv
    \limplies \aVal_\aEv{=}\uReg{\aEv}
    \limplies \bForm[\uReg{\aEv}/\aReg]
  \end{math},
  
\item[{\labeltext[L5]{L5)}{L5no-q-or-addr}}] 
  \begin{math}
    (\forall\aEv\in\aEvs\setminus\cEvs)
  \end{math}
  $\aTr{\cEvs}{\bForm}$ implies
  \begin{math}
    \cForm_\aEv 
    \limplies
    \DLX{\aLoc}{\amode}{\bmode}
    \land
    \PBRbig{
      \ABRbig{
        \aVal_\aEv{=}\uReg{\aEv}
        \lor
        \PBR{
          \RW\land
          \aLoc{=}\uReg{\aEv}
        }
      }
      \limplies
      \bForm
      [\uReg{\aEv}/\aReg]
      [\FALSE/\Q{}]
    }    
  \end{math},
\item[{\labeltext[L6]{L6)}{L6no-q-or-addr}}] 
  \begin{math}
    (\forall\bReg)
  \end{math}
  $\aTr{\dEvs}{\bForm}$  implies 
  \begin{math}
    (\!\not\exists\aEv\in\aEvs \suchthat \cForm_\aEv)
    \limplies \PBR{        
      \DLX{\aLoc}{\amode}{\bmode} \land
      \bForm
      [\bReg/\aReg]
      [\FALSE/\Q{}]
    }.
  \end{math}  
\end{enumerate}  





















































%     \end{minipage}
%   \end{center}
%   \caption{Simplified Quiescence Semantics w/o Address Calculation
%     (See %\refdef{def:QSx} for $\QS{}{\amode}$, $\QL{}{\amode}$, and
%     \refdef{def:dlx} for $\DLX{\aLoc}{\amode}{\bmode}$, $\DS{\aLoc}{\amode}$)
%   } 
%   \label{fig:no-q-or-addr}
% \end{figure*}    
% \begin{figure*}
%   \begin{center}
%     \begin{minipage}{.91\textwidth}
%       \renewcommand{\cEvs}{D}
\renewcommand{\dEvs}{D}
\noindent
If $\aPS \in \sSTORE[\amode]{\cExp}{\aExp}$ then
$(\exists\cVal:\aEvs\fun\Val)$
$(\exists\aVal:\aEvs\fun\Val)$
$(\exists\cForm:\aEvs\fun\Formulae)$
\begin{enumerate}
\item[{\labeltext[S1]{S1)}{S1full}}] % [\ref{S1})]
  if $\cForm_\bEv\land\cForm_\aEv$ is satisfiable then $\bEv=\aEv$,
\item[{\labeltext[S2]{S2)}{S2full}}] %[\ref{S2})]
  $\labelingAct(\aEv) = \DWREF{\cVal_\aEv}{\aVal_\aEv}$,
\item[{\labeltext[S3]{S3)}{S3full}}] %[\ref{S3})] 
  $\labelingForm(\aEv)$ implies
  \begin{math}
    \cForm_\aEv
    \land \QS{\REF{\cVal_\aEv}}{\amode}
    % \land \RW
    \land \cExp{=}\cVal_\aEv
    \land \aExp{=}\aVal_\aEv
  \end{math},
  % where
  % $\QS{}{\mRLX}=\QxREF{\cVal_\aEv}$ and otherwise $\QS{}{\amode}=\Q{\amode}$, % for $\amode\neq\mRLX$,
\item[{\labeltext[S4]{S4)}{S4full}}] %[\ref{S4})]
  \begin{math}
    (\forall\aEv\in\aEvs\cap\bEvs)
  \end{math}
  $\aTr{\bEvs}{\bForm}$ implies 
  \begin{math}
    \cForm_\aEv
    %\limplies (\cExp{=}\cVal_\aEv)
    \limplies {
      \bForm
      [\aExp/\REF{\cVal_\aEv}]
      \DS{\REF{\cVal_\aEv}}{\amode}
      [(\QwREF{\cVal_\aEv}\land\aExp{=}\aVal_\aEv\land\cExp{=}\cVal_\aEv)/\QwREF{\cVal_\aEv}]
    }
  \end{math},
\item[{\labeltext[S5]{S5)}{S5full}}] %[\ref{S5})] 
  \begin{math}    
    (\forall\aEv\in\aEvs\setminus\cEvs)
  \end{math}
  $\aTr{\cEvs}{\bForm}$ implies
  \begin{math}
    \cForm_\aEv
    %\limplies (\cExp{=}\cVal_\aEv)
    \limplies {
      \bForm
      [\aExp/\REF{\cVal_\aEv}]
      \DS{\REF{\cVal_\aEv}}{\amode}
      [\FALSE/\QS{\REF{\cVal_\aEv}}{\amode}]
    },
  \end{math}
\item[{\labeltext[S6]{S6)}{S6full}}] %[S6)]%\ref{S6})] 
  \begin{math}
    (\forall\dVal)
  \end{math}
  $\aTr{\dEvs}{\bForm}$ implies
  \begin{math}
    (\!\not\exists\aEv\in\aEvs \suchthat \cForm_\aEv)
    %(\!\not\exists\aEv\in\aEvs\cap\cEvs \suchthat \cForm_\aEv)
    \limplies (\cExp{=}\dVal)
    \limplies {
      \bForm
      [\aExp/\REF{\dVal}]
      \DS{\REF{\dVal}}{\amode}
      [\FALSE/\QS{\REF{\dVal}}{\amode}]
    }.
  \end{math}
% \item[S5-6)]%\ref{S6})] 
%   \begin{math}
%     (\forall\dVal)
%   \end{math}
%   $\aTr{\cEvs}{\bForm}$ implies
%   \begin{math}
%     %(\!\not\exists\aEv\in\aEvs \suchthat \cForm_\aEv)
%     (\!\not\exists\aEv\in\aEvs\cap\cEvs \suchthat \cForm_\aEv)
%     \limplies (\cExp{=}\dVal)
%     \limplies \PBR{
%       \bForm
%       [\aExp/\REF{\dVal}]
%       \DS{\REF{\dVal}}{\amode}
%       [\FALSE/\QS{\REF{\dVal}}{\amode}]
%     }.
%   \end{math}
  % \\ where 
  % $\DS{}{\mRLX}{}=[\TRUE/\DxREF{\dVal}]$ and otherwise
  % $\DS{}{\amode}{}=[\FALSE/\D]$. % for $\amode\neq\mRLX$.
\end{enumerate}
% \item if $\amode=\mRLX$ then
%   $\labelingForm(\aEv)$ implies
%   \begin{math}
%     \cForm_\aEv
%     \land \cExp{=}\cVal_\aEv
%     \land \aExp{=}\aVal_\aEv
%     \land \RW
%     \land \QxREF{\cVal_\aEv},
%   \end{math}
% \item if $\amode\neq\mRLX$ then
%   $\labelingForm(\aEv)$ implies
%   \begin{math}
%     \cForm_\aEv
%     \land \cExp{=}\cVal_\aEv
%     \land \aExp{=}\aVal_\aEv
%     \land \RW
%     \land \Q{},
%   \end{math}
% \item if
%   $\aEv\in\bEvs$
%   and
%   $\amode=\mRLX$ then
%   \begin{math}
%     (\forall\dVal)
%   \end{math}
%   $\aTr{\bEvs}{\bForm}$ implies 
%   \begin{math}
%     \cForm_\aEv
%     \limplies (\cExp{=}\dVal)
%     \limplies \PBRbig{
%     (\QwREF{\dVal} \limplies \aExp{=}\aVal_\aEv)
%     \land \bForm[\aExp/\REF{\dVal}][\TRUE/\DxREF{\dVal}]
%   }
%   \end{math}
% \item if
%   $\aEv\in\bEvs$
%   and
%   $\amode\neq\mRLX$ then
%   \begin{math}
%     (\forall\dVal)
%   \end{math}
%   $\aTr{\bEvs}{\bForm}$ implies 
%   \begin{math}
%     \cForm_\aEv
%     \limplies (\cExp{=}\dVal)
%     \limplies \PBRbig{
%     (\QwREF{\dVal} \limplies \aExp{=}\aVal_\aEv)
%     \land \bForm[\aExp/\REF{\dVal}][\FALSE/\D]
%   }
%   \end{math}
% \item if 
%   \begin{math}
%     (\forall\aEv\in\bEvs)(\cForm \textimplies
%     \lnot\cForm_\aEv)
%   \end{math}
%   and $\amode=\mRLX$ 
%   then
%   \begin{math}
%     (\forall\dVal)
%   \end{math}
%   $\aTr{\bEvs}{\bForm}$ implies 
%   \begin{math}
%     \cForm
%     \limplies (\cExp{=}\dVal)
%     \limplies \PBRbig{
%     \lnot\QwREF{\dVal}
%     \land \bForm[\aExp/\REF{\dVal}][\TRUE/\DxREF{\dVal}]
%   }
%   \end{math}
% \item if 
%   \begin{math}
%     (\forall\aEv\in\bEvs)
%     (\cForm \textimplies \lnot\cForm_\aEv)
%   \end{math}
%   and $\amode\neq\mRLX$ 
%   then
%   \begin{math}
%     (\forall\dVal)
%   \end{math}
%   $\aTr{\bEvs}{\bForm}$ implies 
%   \begin{math}
%     \cForm
%     \limplies (\cExp{=}\dVal)
%     \limplies \PBRbig{
%     \lnot\QwREF{\dVal}
%     \land \bForm[\aExp/\REF{\dVal}][\FALSE/\D]
%   }
%   \end{math}

\noindent
If $\aPS \in \sLOAD[\amode]{\aReg}{\cExp}$ then
$(\exists\cVal:\aEvs\fun\Val)$
$(\exists\aVal:\aEvs\fun\Val)$
$(\exists\cForm:\aEvs\fun\Formulae)$
% $(\forall\uReg{\aEv}\in\uRegs{\aEvs})$
\begin{enumerate}
\item[{\labeltext[L1]{L1)}{L1full}}] %[\ref{L1})]
  if $\cForm_\bEv\land\cForm_\aEv$ is satisfiable then $\bEv=\aEv$,
\item[{\labeltext[L2]{L2)}{L2full}}] %[\ref{L2})]
  $\labelingAct(\aEv) = \DRREF{\cVal_\aEv}{\aVal_\aEv}$,
\item[{\labeltext[L3]{L3)}{L3full}}] %[\ref{L3})]
  $\labelingForm(\aEv)$ implies
  \begin{math}
    \cForm_\aEv
    \land \QL{\REF{\cVal_\aEv}}{\amode}
    % \land \RO
    \land \cExp{=}\cVal_\aEv
  \end{math},
  % where    
  % $\QL{}{\mSC}=\Q{\mSC}$ and otherwise $\QL{}{\amode}=\QwREF{\cVal_\aEv}$, % for $\amode\neq\mRLX$,
\item[{\labeltext[L4]{L4)}{L4full}}] %[\ref{L4})]
  \begin{math}
    (\forall\aEv\in\aEvs\cap\bEvs)
  \end{math}
  $\aTr{\bEvs}{\bForm}$ implies
  \begin{math}
    \cForm_\aEv
    \limplies (\cExp{=}\cVal_\aEv\limplies\aVal_\aEv{=}\uReg{\aEv})
    \limplies \bForm[\uReg{\aEv}/\aReg]
  \end{math},
  %\makebox[6.2cm]{}
\item[{\labeltext[L5]{L5)}{L5full}}] %[\ref{L5})] 
  \begin{math}
    (\forall\aEv\in\aEvs\setminus\cEvs)
  \end{math}
  $\aTr{\cEvs}{\bForm}$ implies
  \begin{math}
    \cForm_\aEv 
    \limplies
    \DL{\REF{\cVal_\aEv}}{\amode}
    \land
    \PBRbig{
      \ABRbig{
        \PBR{\cExp{=}\cVal_\aEv\limplies\aVal_\aEv{=}\uReg{\aEv}}
        \lor
        \PBR{
          \RW\lor
          \PBR{\cExp{=}\cVal_\aEv\limplies\REF{\cVal_\aEv}{=}\uReg{\aEv}}
        }
      }
      \limplies
      \bForm
      [\uReg{\aEv}/\aReg]
      [\FALSE/\QL{\REF{\cVal_\aEv}}{\amode}]
    }    
  \end{math},
\item[{\labeltext[L6]{L6)}{L6full}}] %[\ref{L6})] 
  \begin{math}
    (\forall\dVal)
    (\forall\bReg)
  \end{math}
  $\aTr{\dEvs}{\bForm}$  implies 
  \begin{math}
    (\!\not\exists\aEv\in\aEvs \suchthat \cForm_\aEv)
    \limplies (\cExp{=}\dVal)
    \limplies \PBR{        
      \DL{\REF{\dVal}}{\amode} \land
      \bForm
      [\bReg/\aReg]
      [\FALSE/\QL{\REF{\dVal}}{\amode}]
    }.
  \end{math}
  % \\ where $\DL{}{\mRLX}=\TRUE$ and otherwise $\DL{}{\amode}=\DxREF{\dVal}$.
  % Recall that $\uRegs{\bEvs}=\{\uReg{\aEv}\mid\aEv\in\bEvs\}$.
\end{enumerate}  
% \item if $\amode=\mRLX$ and $\bEv\notin\bEvs$ then
%   \begin{math}
%     (\forall\dVal)
%   \end{math}
%   $\aTr{\bEvs}{\bForm}$ implies
%   \begin{math}
%     \cForm_\bEv
%     \limplies (\cExp{=}\dVal)
%     \limplies \PBRbig{
%     (
%     \RW
%     \limplies (\aVal{=}\uReg{\bEv}\lor\aLoc{=}\uReg{\bEv}) 
%     \limplies \bForm[\uReg{\bEv}/\aReg][\uReg{\bEv}/\REF{\dVal}]
%     )
%     \land \lnot\QxREF{\dVal}
%   }
%     \phantom{\land\; \Dx{\dVal}}
%   \end{math}
% \item if $\amode\neq\mRLX$ and $\bEv\notin\bEvs$ then
%   \begin{math}
%     (\forall\dVal)
%   \end{math}
%   $\aTr{\bEvs}{\bForm}$ implies
%   \begin{math}
%     \cForm_\bEv
%     \limplies (\cExp{=}\dVal)
%     \limplies \PBRbig{
%     (
%     \RW
%     \limplies (\aVal{=}\uReg{\bEv}\lor\aLoc{=}\uReg{\bEv}) 
%     \limplies \bForm[\uReg{\bEv}/\aReg][\uReg{\bEv}/\REF{\dVal}]
%     )
%     \land \lnot\QxREF{\dVal}
%     \land \Dx{\dVal}
%   }
%   \end{math}

\noindent
If $\aPS \in \sTHREAD{\aPSS}$ then
$(\exists\aPS_1\in\aPSS)$
\begin{enumerate}
\item[{\labeltext[T1]{T1)}{T1full}}] %[\ref{T1})]
  $\aEvs=\aEvs_1$,
\item[{\labeltext[T2]{T2)}{T2full}}] %[\ref{T2})]
  $\labelingAct(\aEv) = \labelingAct_1(\aEv)$,
\item[{\labeltext[T3]{T3)}{T3full}}] %[\ref{T3})]
  $\labelingForm(\aEv)$ implies
  $\labelingForm_1(\aEv) [\TRUE/\Q{}][\TRUE/\RW]$ if $\labelingAct_1(\aEv)$ is a write,
  \\
  $\labelingForm(\aEv)$ implies
  $\labelingForm_1(\aEv) [\TRUE/\Q{}][\FALSE/\RW]$ otherwise.
\end{enumerate}  

%       % \noindent
If $\aPS \in \sSTORE[\amode]{\cExp}{\aExp}$ then
$(\exists\cVal:\aEvs\fun\Val)$
$(\exists\aVal:\aEvs\fun\Val)$
$(\exists\bForm:\aEvs\fun\Formulae)$
\begin{enumerate}
\item if $\bForm_\bEv\land\bForm_\aEv$ is satisfiable then $\bEv=\aEv$,
\item $\labelingAct(\aEv) = \DWREFP{\cVal_\aEv}{\aVal_\aEv}$,
\item 
  $\labelingForm(\aEv)$ implies
  \begin{math}
    \bForm_\aEv
    \land \cExp{=}\cVal_\aEv
    \land \aExp{=}\aVal_\aEv
    \land \RW
    \land \Qmode{\amode}
  \end{math},
  where
  $\Qmode{\mRLX}=\QxREF{\cVal_\aEv}$ and otherwise $\Qmode{\amode}=\Q{\amode}$, % for $\amode\neq\mRLX$,
\item
  \begin{math}
    (\forall\dVal)
  \end{math}
  if
  $\bEv\in\bEvs$
  then
  $\aTr[\bEvs](\aForm)$ implies 
  \begin{math}
    \bForm_\bEv
    \limplies (\cExp{=}\dVal)
    \limplies \PBRbig{
      (\QwREF{\dVal} \limplies \aExp{=}\aVal_\bEv)
      \land \aForm [\aExp/\REF{\dVal}]\Dmode{\amode}
    }
  \end{math},
\item %if 
  % \begin{math}
  %   (\forall\bEv\in\bEvs)(\cForm \textimplies
  %   \lnot\bForm_\bEv)
  % \end{math}
  % then
  \begin{math}
    (\forall\dVal)
  \end{math}
  $\aTr[\bEvs](\aForm)$ implies 
  \begin{math}
    (\not\exists\bEv\in\bEvs.\; \bForm_\bEv)
    \limplies (\cExp{=}\dVal)
    \limplies \PBR{
      \lnot\QwREF{\dVal}
      \land \aForm [\aExp/\REF{\dVal}]\Dmode{\amode}
    }
  \end{math},
  \\ where 
  $\Dmode{\mRLX}=[\TRUE/\DxREF{\dVal}]$ and otherwise
  $\Dmode{\amode}=[\FALSE/\D]$. % for $\amode\neq\mRLX$.
\end{enumerate}
% \item if $\amode=\mRLX$ then
%   $\labelingForm(\aEv)$ implies
%   \begin{math}
%     \bForm_\aEv
%     \land \cExp{=}\cVal_\aEv
%     \land \aExp{=}\aVal_\aEv
%     \land \RW
%     \land \QxREF{\cVal_\aEv},
%   \end{math}
% \item if $\amode\neq\mRLX$ then
%   $\labelingForm(\aEv)$ implies
%   \begin{math}
%     \bForm_\aEv
%     \land \cExp{=}\cVal_\aEv
%     \land \aExp{=}\aVal_\aEv
%     \land \RW
%     \land \Q{},
%   \end{math}
% \item if
%   $\bEv\in\bEvs$
%   and
%   $\amode=\mRLX$ then
%   \begin{math}
%     (\forall\dVal)
%   \end{math}
%   $\aTr[\bEvs](\aForm)$ implies 
%   \begin{math}
%     \bForm_\bEv
%     \limplies (\cExp{=}\dVal)
%     \limplies \PBRbig{
%     (\QwREF{\dVal} \limplies \aExp{=}\aVal_\bEv)
%     \land \aForm[\aExp/\REF{\dVal}][\TRUE/\DxREF{\dVal}]
%   }
%   \end{math}
% \item if
%   $\bEv\in\bEvs$
%   and
%   $\amode\neq\mRLX$ then
%   \begin{math}
%     (\forall\dVal)
%   \end{math}
%   $\aTr[\bEvs](\aForm)$ implies 
%   \begin{math}
%     \bForm_\bEv
%     \limplies (\cExp{=}\dVal)
%     \limplies \PBRbig{
%     (\QwREF{\dVal} \limplies \aExp{=}\aVal_\bEv)
%     \land \aForm[\aExp/\REF{\dVal}][\FALSE/\D]
%   }
%   \end{math}
% \item if 
%   \begin{math}
%     (\forall\bEv\in\bEvs)(\cForm \textimplies
%     \lnot\bForm_\bEv)
%   \end{math}
%   and $\amode=\mRLX$ 
%   then
%   \begin{math}
%     (\forall\dVal)
%   \end{math}
%   $\aTr[\bEvs](\aForm)$ implies 
%   \begin{math}
%     \cForm
%     \limplies (\cExp{=}\dVal)
%     \limplies \PBRbig{
%     \lnot\QwREF{\dVal}
%     \land \aForm[\aExp/\REF{\dVal}][\TRUE/\DxREF{\dVal}]
%   }
%   \end{math}
% \item if 
%   \begin{math}
%     (\forall\bEv\in\bEvs)
%     (\cForm \textimplies \lnot\bForm_\bEv)
%   \end{math}
%   and $\amode\neq\mRLX$ 
%   then
%   \begin{math}
%     (\forall\dVal)
%   \end{math}
%   $\aTr[\bEvs](\aForm)$ implies 
%   \begin{math}
%     \cForm
%     \limplies (\cExp{=}\dVal)
%     \limplies \PBRbig{
%     \lnot\QwREF{\dVal}
%     \land \aForm[\aExp/\REF{\dVal}][\FALSE/\D]
%   }
%   \end{math}

\noindent
If $\aPS \in \sLOAD[\amode]{\aReg}{\cExp}$ then
$(\exists\cVal:\aEvs\fun\Val)$
$(\exists\aVal:\aEvs\fun\Val)$
$(\exists\bForm:\aEvs\fun\Formulae)$
% $(\forall\uReg{\aEv}\in\uRegs{\aEvs})$
\begin{enumerate}
\item if $\bForm_\bEv\land\bForm_\aEv$ is satisfiable then $\bEv=\aEv$,
\item $\labelingAct(\aEv) = \DRREFP{\cVal_\aEv}{\aVal_\aEv}$,
\item $\labelingForm(\aEv)$ implies
  \begin{math}
    \bForm_\aEv
    \land \cExp{=}\cVal_\aEv
    \land \RO
    \land \Qmode{\amode}
  \end{math},
  where    
  $\Qmode{\mSC}=\Q{\mSC}$ and otherwise $\Qmode{\amode}=\QwREF{\cVal_\aEv}$, % for $\amode\neq\mRLX$,
\item
  \begin{math}
    (\forall\dVal)
  \end{math}
  if $\bEv\in\bEvs$ then
  $\aTr[\bEvs](\aForm)$ implies
  \begin{math}
    \bForm_\bEv
    \limplies (\cExp{=}\dVal)
    \limplies (\aVal{=}\uReg{\bEv})
    \limplies \aForm[\uReg{\bEv}/\aReg][\uReg{\bEv}/\REF{\dVal}]
  \end{math},
  \makebox[4.4cm]{}
\item 
  \begin{math}
    (\forall\dVal)
  \end{math}
  if $\bEv\notin\bEvs$ then
  $\aTr[\bEvs](\aForm)$ implies
  \begin{math}
    \bForm_\bEv
    \limplies (\cExp{=}\dVal)
    \limplies \PBRbig{        
      \Dmode{\amode}
      \land \lnot\QxREF{\dVal}
      \land
      (\RW
      \limplies (\aVal{=}\uReg{\bEv}\lor\aLoc{=}\uReg{\bEv}) 
      \limplies \aForm[\uReg{\bEv}/\aReg][\uReg{\bEv}/\REF{\dVal}]
      )
    }      
  \end{math},
\item % if 
  % \begin{math}
  %   (\forall\bEv\in\bEvs)(\cForm \textimplies
  %   \lnot\bForm_\bEv)
  % \end{math}
  % then
  \begin{math}
    (\forall\dVal)
    (\forall\bReg)
  \end{math}
  $\aTr[\bEvs](\aForm)$ implies 
  \begin{math}
    (\not\exists\bEv\in\bEvs.\; \bForm_\bEv)
    \limplies (\cExp{=}\dVal)
    \limplies \PBR{        
      \Dmode{\amode}
      \land \lnot\QxREF{\dVal}
      \land
      \limplies \aForm[\bReg/\aReg][\bReg/\REF{\dVal}]
    }      
  \end{math},
  \\ where $\Dmode{\mRLX}=\TRUE$ and otherwise $\Dmode{\amode}=\Dx{\dVal}$.
  Recall that $\uRegs{\bEvs}=\{\uReg{\bEv}\mid\bEv\in\bEvs\}$.
\end{enumerate}  
% \item if $\amode=\mRLX$ and $\bEv\notin\bEvs$ then
%   \begin{math}
%     (\forall\dVal)
%   \end{math}
%   $\aTr[\bEvs](\aForm)$ implies
%   \begin{math}
%     \bForm_\bEv
%     \limplies (\cExp{=}\dVal)
%     \limplies \PBRbig{
%     (
%     \RW
%     \limplies (\aVal{=}\uReg{\bEv}\lor\aLoc{=}\uReg{\bEv}) 
%     \limplies \aForm[\uReg{\bEv}/\aReg][\uReg{\bEv}/\REF{\dVal}]
%     )
%     \land \lnot\QxREF{\dVal}
%   }
%     \phantom{\land\; \Dx{\dVal}}
%   \end{math}
% \item if $\amode\neq\mRLX$ and $\bEv\notin\bEvs$ then
%   \begin{math}
%     (\forall\dVal)
%   \end{math}
%   $\aTr[\bEvs](\aForm)$ implies
%   \begin{math}
%     \bForm_\bEv
%     \limplies (\cExp{=}\dVal)
%     \limplies \PBRbig{
%     (
%     \RW
%     \limplies (\aVal{=}\uReg{\bEv}\lor\aLoc{=}\uReg{\bEv}) 
%     \limplies \aForm[\uReg{\bEv}/\aReg][\uReg{\bEv}/\REF{\dVal}]
%     )
%     \land \lnot\QxREF{\dVal}
%     \land \Dx{\dVal}
%   }
%   \end{math}

%       % \noindent
If $\aPS \in \sSTORE[\amode]{\cExp}{\aExp}$ then
$(\exists\cVal:\aEvs\fun\Val)$
$(\exists\aVal:\aEvs\fun\Val)$
$(\exists\bForm:\aEvs\fun\Formulae)$
\begin{enumerate}
\item if $\bForm_\bEv\land\bForm_\aEv$ is satisfiable then $\bEv=\aEv$,
\item $\labelingAct(\aEv) = \DWREFP{\cVal_\aEv}{\aVal_\aEv}$,
\item 
  $\labelingForm(\aEv)$ implies
  \begin{math}
    \bForm_\aEv
    \land \cExp{=}\cVal_\aEv
    \land \aExp{=}\aVal_\aEv
    \land \RW
    \land \QS{}{\amode}
  \end{math},
\item
  \begin{math}
    (\forall\dVal)
  \end{math}
  if
  $\bEv\in\bEvs$
  then
  $\aTr[\bEvs](\aForm)$ implies 
  \begin{math}
    \bForm_\bEv
    \limplies (\cExp{=}\dVal)
    \limplies \PBRbig{
      \aExp{=}\aVal_\bEv
      \land \DS{\REF{\dVal}}{\amode}{\aForm[\aExp/\REF{\dVal}]}
    }
  \end{math},
\item 
  \begin{math}
    (\forall\dVal)
  \end{math}
  $\aTr[\bEvs](\aForm)$ implies 
  \begin{math}
    (\not\exists\bEv\in\bEvs.\; \bForm_\bEv)
    \limplies (\cExp{=}\dVal)
    \limplies \PBR{
      \lnot\Q{\mRA}
      \land \DS{\REF{\dVal}}{\amode}{\aForm[\aExp/\REF{\dVal}]}
    }.
  \end{math}
\end{enumerate}

\noindent
If $\aPS \in \sLOAD[\amode]{\aReg}{\cExp}$ then
$(\exists\cVal:\aEvs\fun\Val)$
$(\exists\aVal:\aEvs\fun\Val)$
$(\exists\bForm:\aEvs\fun\Formulae)$
\begin{enumerate}
\item if $\bForm_\bEv\land\bForm_\aEv$ is satisfiable then $\bEv=\aEv$,
\item $\labelingAct(\aEv) = \DRREFP{\cVal_\aEv}{\aVal_\aEv}$,
\item $\labelingForm(\aEv)$ implies
  \begin{math}
    \bForm_\aEv
    \land \cExp{=}\cVal_\aEv
    \land \RO
    \land \QL{}{\amode}
  \end{math},
\item
  \begin{math}
    (\forall\dVal)
  \end{math}
  if $\bEv\in\bEvs$ then
  $\aTr[\bEvs](\aForm)$ implies
  \begin{math}
    \bForm_\bEv
    \limplies (\cExp{=}\dVal)
    \limplies (\aVal{=}\uReg{\bEv})
    \limplies \aForm[\uReg{\bEv}/\aReg][\uReg{\bEv}/\REF{\dVal}]
  \end{math},
  \makebox[4.8cm]{}
\item 
  \begin{math}
    (\forall\dVal)
  \end{math}
  if $\bEv\notin\bEvs$ then
  $\aTr[\bEvs](\aForm)$ implies
  \begin{math}
    \bForm_\bEv
    \limplies (\cExp{=}\dVal)
    \limplies \PBRbig{        
      \DL{\REF{\dVal}}{\amode}
      \land \lnot\Q{\mRA}
      \land
      (\RW
      \limplies (\aVal{=}\uReg{\bEv}\lor\aLoc{=}\uReg{\bEv}) 
      \limplies \aForm[\uReg{\bEv}/\aReg][\uReg{\bEv}/\REF{\dVal}]
      )
    }      
  \end{math},
\item 
  \begin{math}
    (\forall\dVal)
    (\forall\bReg)
  \end{math}
  $\aTr[\bEvs](\aForm)$ implies 
  \begin{math}
    (\not\exists\bEv\in\bEvs.\; \bForm_\bEv)
    \limplies (\cExp{=}\dVal)
    \limplies \PBR{        
      \DL{\REF{\dVal}}{\amode}
      \land \lnot\Q{\mRA}
      \land
      \limplies \aForm[\bReg/\aReg][\bReg/\REF{\dVal}]
    }.
  \end{math}
\end{enumerate}  

%     \end{minipage}
%   \end{center}
%   \caption{Full Semantics with Address Calculation
%     (See \refdef{def:QS} for $\QS{\aLoc}{\amode}$, $\QL{\aLoc}{\amode}$
%     and \refdef{def:DS} for $\DL{\aLoc}{\amode}$, $\DS{\aLoc}{\amode}$)
%   }
%   \label{fig:full}
% \end{figure*}    

%\section{Discussion}
\subsection{Downset Closure}
\label{sec:downset}

% We would like the semantics to be closed with respect to \emph{augments} and
% \emph{downsets}.

% Augments include more order and stronger formulae; in examples, we typically
% consider pomsets that are augment-minimal.  One intuitive reading of augment
% closure is that adding order can only cause preconditions to weaken.
% \begin{definition}
%   \label{def:augment}
%   $\aPS_2$ is an \emph{augment} of $\aPS_1$ if
%   \begin{enumerate}
%   \item $\aEvs_2=\aEvs_1$,
%   \item $\labelingAct_2(\aEv)=\labelingAct_1(\aEv)$,
%   \item $\labelingForm_2(\aEv) \rimplies \labelingForm_1(\aEv)$,
%   \item $\aTr[2]{\bEvs}{\aEv} \rimplies \aTr[1]{\bEvs}{\aEv}$,
%   \item if $\bEv\le_2\aEv$ then $\bEv\le_1\aEv$.
%   \end{enumerate}
% \end{definition}

% \begin{proposition}
%   %   Suppose $\aPS_1\in\sem{\aCmd}$.
%   If $\aPS_1\in\sem{\aCmd}$ and $\aPS_2$  augments $\aPS_1$ then $\aPS_2\in\sem{\aCmd}$.
%   % \item If $\aPS_2$ is a downset of $\aPS_1$ then $\aPS_2\in\sem{\aCmd}$.
%   % \end{enumerate}
% \end{proposition}

We would like the semantics to be closed with respect to \emph{downsets}.
Downsets include a subset of initial events, similar to \emph{prefixes} for
strings.
\begin{definition}
  \label{def:downset}
  $\aPS_2$ is an \emph{downset} of $\aPS_1$ if
  \begin{multicols}{2}
    \begin{enumerate}
    \item $\aEvs_2\subseteq\aEvs_1$,
    \item $(\forall \aEv\in\aEvs_2)$ $\labelingAct_2(\aEv)=\labelingAct_1(\aEv)$,
    \item $(\forall \aEv\in\aEvs_2)$ $\labelingForm_2(\aEv)\riff\labelingForm_1(\aEv)$,
    \item $(\forall \aEv\in\aEvs_2)$ $\aTr[2]{\bEvs}{\aEv}\riff\aTr[1]{\bEvs}{\aEv}$,
    \item $\aTerm[2] \rimplies \aTerm[1]$,
      \stepcounter{enumi}
    \item[] 
      \begin{enumerate}[leftmargin=0pt]
      \item $(\forall \bEv\in\aEvs_2)$ $(\forall \aEv\in\aEvs_2)$ $\bEv\lt_2\aEv$ iff $\bEv\lt_1\aEv$,
      \item $(\forall \bEv\in\aEvs_1)$ $(\forall \aEv\in\aEvs_2)$ if
        $\bEv\lt_1\aEv$ then $\bEv\in\aEvs_2$,
      \end{enumerate}
    \item $(\forall \bEv\in\aEvs_2)$ $(\forall \aEv\in\aEvs_2)$ $\bEv\rrfx_2\aEv$ iff $\bEv\rrfx_1\aEv$.
    \end{enumerate}
  \end{multicols}
\end{definition}

Downset closure fails due to for two reasons.  The key property is that the
empty set transformer should behave the same as the independent transformer.

First, downset closure fails for read-read independency \textsection\ref{sec:read-read}.
  % For \xRRD{}, \refdef{def:pomsets-rr} states:
  % \begin{enumerate}
  % \item[\ref{L4})]
  %   $\aTr{\bEvs}{\bForm} \rimplies \aVal{=}\aReg\limplies\bForm$, 
  % \item[\ref{L5})]
  %   $\aTr{\cEvs}{\bForm} \rimplies (\aVal{=}\aReg\lor\RW)\limplies\bForm$,
  % \item[\ref{L6})] 
  %   $\aTr{\dEvs}{\bForm} \rimplies \bForm$, when $\aEvs=\emptyset$.
  % \end{enumerate}
  % This semantics is not downset closed due to the lack of read-read dependencies.
  % In both cases, for subsequent writes, \ref{L5} is the same as \ref{L6}.  For
  % subsequent reads, \ref{L5} is the same as \ref{L4}.
Consider
\begin{gather*}
  \begin{gathered}[t]
    \PR{x}{r}\SEMI\IF{\BANG r}\THEN\PR{y}{s}\FI
    \\
    \hbox{\begin{tikzinline}[node distance=.5em and 1.5em]
        \event{a}{\DR{x}{0}}{}
        \event{b}{\DR{y}{0}}{right=of a}
      \end{tikzinline}}
  \end{gathered}    
\end{gather*}
The semantics of this program includes the singleton pomset $\DRP{x}{0}$,
but not the singleton pomset $\DRP{y}{0}$.
To get $\DRP{x}{0}$, we combine:
\begin{align*}
  \begin{gathered}[t]
    \PR{x}{r}
    \\
    \hbox{\begin{tikzinline}[node distance=.5em and 1.5em]
        \event{a}{\DR{x}{0}}{}
      \end{tikzinline}}
  \end{gathered}    
  &&
  \begin{gathered}[t]
    \IF{\BANG r}\THEN\PR{y}{s}\FI
    \\
    \emptyset
  \end{gathered}    
\end{align*}
Attempting to get $\DRP{y}{0}$, we instead get:
\begin{align*}
  \begin{gathered}[t]
    \PR{x}{r}
    \\
    \emptyset
  \end{gathered}    
  &&
  \begin{gathered}[t]
    \IF{\BANG r}\THEN\PR{y}{s}\FI
    \\
    \hbox{\begin{tikzinline}[node distance=.5em and 1.5em]
        \event{b}{r\EQ0\bigmid\DR{y}{0}}{}
      \end{tikzinline}}
  \end{gathered}    
\end{align*}
Since $r$ appears only once in the program, this pomset cannot contribute
to a top-level pomset.


Second, the semantics is not downset closed because the independency reasoning of
\ref{read-tau-ind} is only applicable for pomsets where the ignored read is present!
Revisiting \jmm{} causality test case 1 from the end of \textsection\ref{sec:ex:control}:
\begin{align*}
  \begin{gathered}[t]
    \PW{x}{0} 
    \\
    \hbox{\begin{tikzinline}[node distance=.5em and 1.5em]
        \event{a0}{\DW{x}{0}}{}
        \xform{xi}{\bForm[0/x]}{below=of a0}
      \end{tikzinline}}    
  \end{gathered}
  &&
  \begin{gathered}[t]
    \PR{x}{r} 
    \\
    \hbox{\begin{tikzinline}[node distance=.5em and 1.5em]
        \event{a1}{\DR{x}{1}}{}
        \xform{xi}{(1{=}r\lor x{=}r)\limplies\bForm}{below=of a1}
      \end{tikzinline}}    
  \end{gathered}
  &&
  \begin{gathered}[t]
    \IF{r{\geq}0}\THEN \PW{y}{1} \FI
    \SEMI
    \PW{z}{r}
    \\
    \hbox{\begin{tikzinline}[node distance=.5em and 1.5em]
        \event{a2}{r{\geq}0\bigmid\DW{y}{1}}{}      
        \event{a3}{r{=}1\bigmid\DW{z}{1}}{right=of a2}      
      \end{tikzinline}}    
  \end{gathered}
\end{align*}
% Composing:
\begin{align*}
  \begin{gathered}[t]
    \PW{x}{0} 
    \SEMI\PR{x}{r} 
    \SEMI\IF{r{\geq}0}\THEN \PW{y}{1} \FI
    \SEMI
    \PW{z}{r}
    \\
    \hbox{\begin{tikzinline}[node distance=.5em and 1.5em]
        \event{a0}{\DW{x}{0}}{}
        \event{a1}{\DR{x}{1}}{right=of a0}
        \event{a2}{(1{=}r\lor 0{=}r)\limplies r{\geq}0\bigmid\DW{y}{1}}{right=of a1}      
        \event{a3}{1{=}r\limplies r{=}1\bigmid\DW{z}{1}}{right=of a2}
        \po[out=15,in=165]{a1}{a3}
        \wki{a0}{a1}
      \end{tikzinline}}    
  \end{gathered}
\end{align*}
The precondition of $\DWP{y}{1}$ is a tautology.

Taking the empty set for the read, however,
the precondition of $\DWP{y}{1}$ is not a tautology:
\begin{align*}
  \begin{gathered}[t]
    \PW{x}{0} 
    \SEMI\PR{x}{r} 
    \SEMI\IF{r{\geq}0}\THEN \PW{y}{1} \FI
    \SEMI
    \PW{z}{r}
    \\
    \hbox{\begin{tikzinline}[node distance=.5em and 1.5em]
        \event{a0}{\DW{x}{0}}{}
        % \event{a1}{\DR{x}{1}}{right=of a0}
        \event{a2}{r{\geq}0\bigmid\DW{y}{1}}{right=6em of a0}      
        \event{a3}{r{=}1\bigmid\DW{z}{1}}{right=of a2}
        % \wk{a0}{a1}
      \end{tikzinline}}    
  \end{gathered}
\end{align*}
One way to deal with the second issue would be to allow general access
elimination to merge $\DWP{x}{0}$ and $\DRP{x}{0}$:
\begin{align*}
  \begin{gathered}[t]
    \PW{x}{0} 
    \SEMI\PR{x}{r} 
    \SEMI\IF{r{\geq}0}\THEN \PW{y}{1} \FI
    \SEMI
    \PW{z}{r}
    \\
    \hbox{\begin{tikzinline}[node distance=.5em and 1.5em]
        \event{a0}{\DW{x}{0}}{}
        %\event{a1}{\DR{x}{1}}{right=of a0}
        \event{a2}{(0{=}r\lor 0{=}r)\limplies r{\geq}0\bigmid\DW{y}{1}}{right=6em of a0}      
        \event{a3}{r{=}1\bigmid\DW{z}{1}}{right=of a2}
        %\po[out=-15,in=-165]{a1}{a3}
        %\wki{a0}{a1}
      \end{tikzinline}}    
  \end{gathered}
\end{align*}
We leave the elaboration of this idea to future work.

\begin{comment}
  if in L6 we said [x/r], that says we know read the local version...  ignoring
  the value read...  Perhaps there is some intervening stuff that stops you
  from seeing the local state, such as release-acquire

  We could potentially get rid of [x/r] If you do two reads, your not allowed
  to be independent of the second based on the value that was read in the first
  read.

  x=0; r=x; if (r=1) { s=x; if (s=?) {y=1}}
  read 1 then 2.


  In order for the write to be independent of second read what does its
  precondition have to be.
  [r/x] then s==1
  no sub then s==0

  (s=? | Wy1)

  if (phi) z=1
  phi = s is even
  phi = s < 2

  With substitution you are saying you know that the ``local copy'' of x is the
  same as r.  Sitting in the local cache.  Read might have gone to main
  memory, but if it did it has updated the cache line so that the local copy is
  what I just read.

  If second read is a cache hit, then I know that I am seeing the same value.

  If we take substitution out then 
\end{comment}

\subsection{Logical Encoding of Delay for \PwTmcaTITLE{}}
\label{sec:delay}

In this subsection, we develop a logical encoding of $\rdelay$, which can
replace \ref{seq-le-delays} in \PwTmca{1}.  It is not obvious how to repeat
this trick for \PwTmca{2}, due to thread-local reads-from
(\ref{seq-le-delays-rf} in \refdef{def:pwt:mca2}).

As motivation, recall that we stated 
\reflem{lem:if}\eqref{lem:ifelse:if:if} %--\eqref{lem:ifelse:if:if2}
using inclusions:
\begin{enumerate}
  \item[\eqref{lem:ifelse:if:if}]
    \begin{math}
      \sem{\xSEMI{
        \xIFTHEN{\lnot\aForm}{\aCmd_2}{}
      }{
        \xIFTHEN{\aForm}{\aCmd_1}{}
      }}
      \subseteq
      \sem{\xIFTHEN{\aForm}{\aCmd_1}{\aCmd_2}}
      \supseteq
      \sem{\xSEMI{
        \xIFTHEN{\aForm}{\aCmd_1}{}
      }{
        \xIFTHEN{\lnot\aForm}{\aCmd_2}{}
      }}.
    \end{math}
  
% \item[\eqref{lem:ifelse:if:if1}]
%   \begin{math} 
%     \xIFTHEN{\aForm}{\aPSS_1}{\aPSS_2}
%     \supseteq
%     \xSEMI{
%       \xIFTHEN{\aForm}{\aPSS_1}{}
%     }{
%       \xIFTHEN{\lnot\aForm}{\aPSS_2}{}
%     }.
%   \end{math}
  
% \item[\eqref{lem:ifelse:if:if2}]
%   \begin{math} 
%     \xIFTHEN{\aForm}{\aPSS_1}{\aPSS_2}
%     \supseteq
%     \xSEMI{
%       \xIFTHEN{\lnot\aForm}{\aPSS_2}{}
%     }{
%       \xIFTHEN{\aForm}{\aPSS_1}{}
%     }.
%   \end{math}
\end{enumerate}
\PwTmca{} does not satisfy the reverse inclusion.
The culprit is $\rdelay$, which introduces order regardless of whether
preconditions are disjoint.  As an example, 
\begin{math}
  \sem{\IF{r}
  \THEN \PW{x}{1}
  \ELSE \PW{x}{2}
  \FI}
\end{math}
has an execution with
\begin{math}
  (r{=}0\mid\DW{x}{2})
  \xwki
  (r{\neq}0\mid\DW{x}{1}),
\end{math}
(using augmentation), whereas
\begin{math}
  \sem{
    \IF{r} \THEN \PW{x}{1}\FI
    \SEMI
    \IF{\BANG r} \THEN \PW{x}{2}\FI
  \FI}
\end{math}
has no such execution.


In order to validate the reverse inclusions, we could require that
\ref{seq-le-delays} not impose order when
$\labelingForm_1(\bEv) \land \labelingForm_2(\aEv)$ is unsatisfiable.
Thus, following on \textsection\ref{sec:false}, we would also like this:
\begin{enumerate}
\item[{\labeltext[\textsc{s}6b$'$]{(\textsc{s}6b$'$)}{seq-le-delays'}}] if
  $\labeling_1(\bEv) \rdelays \labeling_2(\aEv)$ and
  $\labelingForm_1(\bEv) \land \labelingForm_2'(\aEv)$ is
  $\labeling$-consistent then $\bEv\le\aEv$.
\end{enumerate}

However, \eqref{seq-le-delays'} fails associativity.
Example where $\cForm_\labeling=(r{=}0)$
\begin{align*}
  \begin{gathered}    
    \PR{y}{r}
    \\
    \hbox{\begin{tikzinline}[node distance=1.5em]
        \event{a}{\DR{y}{0}}{}
      \end{tikzinline}}
  \end{gathered}  
  &&
  \begin{gathered}    
    \IF{r\OR s}\THEN\PW{x}{1}\FI
    \\
    \hbox{\begin{tikzinline}[node distance=1.5em]
        \event{b}{r{\neq}0\lor s{\neq}0\bigmid\DW{x}{1}}{}
      \end{tikzinline}}
  \end{gathered}    
  &&
  \begin{gathered}    
    \IF{\BANG s}\THEN\PW{x}{2}\FI
    \\
    \hbox{\begin{tikzinline}[node distance=1.5em]
        \event{c}{s{=}0\bigmid\DW{x}{2}}{}
      \end{tikzinline}}
  \end{gathered}    
\end{align*}
Associating right, order is required since
$((r{\neq}0 \lor s{\neq}0)\land s{=}0)$ is satisfiable (take $r{=}1$ and $s{=}0$):
\begin{align*}
  \begin{gathered}    
    \PR{y}{r}
    \\
    \hbox{\begin{tikzinline}[node distance=1.5em]
        \event{a}{\DR{y}{0}}{}
      \end{tikzinline}}
  \end{gathered}    
  &&
  \begin{gathered}    
    \IF{r\OR s}\THEN\PW{x}{1}\FI
    \SEMI
    \IF{\BANG s}\THEN\PW{x}{2}\FI
    \\
    \hbox{\begin{tikzinline}[node distance=1.5em]
        \event{b}{r{\neq}0\lor s{\neq}0\bigmid\DW{x}{1}}{}
        \event{c}{s{=}0\bigmid\DW{x}{2}}{right=of b}
        \wki{b}{c}
      \end{tikzinline}}
  \end{gathered}    
\end{align*}
\begin{align*}
  \begin{gathered}    
    \PR{y}{r}
    \SEMI
    \IF{r\OR s}\THEN\PW{x}{1}\FI
    \SEMI
    \IF{\BANG s}\THEN\PW{x}{2}\FI
    \\
    \hbox{\begin{tikzinline}[node distance=1.5em]
        \event{a}{\DR{y}{0}}{}
        \event{b}{r{=}0\limplies (r{\neq}0\lor s{\neq}0)\bigmid\DW{x}{1}}{right=of a}
        \event{c}{s{=}0\bigmid\DW{x}{2}}{right=of b}
        \po{a}{b}
        \wki{b}{c}
      \end{tikzinline}}
  \end{gathered}    
\end{align*}
Associating left, order is not required between the writes since
$(s{\neq}0\land s{=}0)$ is unsatisfiable:
\begin{align*}
  \begin{gathered}    
    \PR{y}{r}
    \SEMI
    \IF{r\OR s}\THEN\PW{x}{1}\FI
    \\
    \hbox{\begin{tikzinline}[node distance=1.5em]
        \event{a}{\DR{y}{0}}{}
        \event{b}{r{=}0\limplies (r{\neq}0\lor s{\neq}0)\bigmid\DW{x}{1}}{right=of a}
        \po{a}{b}
      \end{tikzinline}}
  \end{gathered}    
  &&
  \begin{gathered}    
    \IF{\BANG s}\THEN\PW{x}{2}\FI
    \\
    \hbox{\begin{tikzinline}[node distance=1.5em]
        \event{c}{s{=}0\bigmid\DW{x}{2}}{}
      \end{tikzinline}}
  \end{gathered}    
\end{align*}
\begin{align*}
  \begin{gathered}    
    \PR{y}{r}
    \SEMI
    \IF{r\OR s}\THEN\PW{x}{1}\FI
    \SEMI
    \IF{\BANG s}\THEN\PW{x}{2}\FI
    \\
    \hbox{\begin{tikzinline}[node distance=1.5em]
        \event{a}{\DR{y}{0}}{}
        \event{b}{r{=}0\limplies (r{\neq}0\lor s{\neq}0)\bigmid\DW{x}{1}}{right=of a}
        \event{c}{s{=}0\bigmid\DW{x}{2}}{right=of b}
        \po{a}{b}
      \end{tikzinline}}
  \end{gathered}    
\end{align*}

This motivates the logic-based presentation of $\rdelay$.  We make the following
changes to the data model:
\begin{itemize}
\item actions need not include access modes---for readability, we color
  synchronizing events in example diagrams throughout this section,
\item there exists a symbol $\RW$, indicating a write action---this is needed
  to handle read-read independency (\textsection\ref{sec:read-read}),
\item there exist symbols $\Qsc$, $\Qr{\aLoc}$, and $\Qw{\aLoc}$---we refer
  to these collectively as \emph{quiescence symbols}.  Roughly, the old
  $\Q{x}$ correspond to $\Qw{x}$.
\end{itemize}
We define some shorthand, using the symbols $\mathsf{S}$ for \emph{stores}
(aka writes) and $\mathsf{L}$ for \emph{loads} (aka reads).
\begin{definition}
  \label{def:QS}
  Let $\Qr{*}=\textstyle\bigwedge_\bLoc \Qr{\bLoc}$, and similarly for
  $\Qw{*}$.
  Let $\Qall=\Qr{*}\land\Qw{*} \land\Qsc$.

  Let $[\aForm/\Qr{*}]$ substitute $\aForm$ for every $\Qr{\bLoc}$, and
  similarly for $\Qw{*}$.
  %
  %Let $[\aForm/\Q{}]$ substitute $\aForm$ for every quiescence symbol.
  Let $\Qsuball{\aForm} = [\aForm/\Qr{*}][\aForm/\Qw{*}][\aForm/\Qsc]$.
  
  Let formulae $\QF{}{\amode}$, $\QS{\aLoc}{\amode}$, and $\QL{\aLoc}{\amode}$ be defined:
  \begin{scope}
    \small
    \begin{align*}
      \QF{}{\fREL}&=\Qr{*}\land\Qw{*} 
      &\QS{\aLoc}{\mRLX}&=\Qr{\aLoc}\land\Qw{\aLoc}
      &\QL{\aLoc}{\mRLX}&=\Qw{\aLoc}
      \\
      \QF{}{\fACQ} &=\Qr{*}
      &\QS{\aLoc}{\mREL}&= \Qr{*}\land\Qw{*} 
      &\QL{\aLoc}{\mACQ}&=\Qw{\aLoc}
      \\
      \QF{}{\fSC} &= \Qall
      &\QS{\aLoc}{\mSC}&= \Qall
      &\QL{\aLoc}{\mSC}&=\Qw{\aLoc}\land\Qsc      
    \end{align*}
  \end{scope}
  % \end{definition}
  % \begin{definition}
  
  Let substitutions $[\aForm/\QF{}{\amode}]$, $[\aForm/\QS{\aLoc}{\amode}]$, and $[\aForm/\QL{\aLoc}{\amode}]$ be defined:
  \begin{scope}
    \small
    \begin{align*}
      [\aForm/\QF{}{\fREL}] &= [\aForm/\Qw{*}]
      &{} [\aForm/\QS{\aLoc}{\mRLX}] &= [\aForm/\Qw{\aLoc}]
      &{} [\aForm/\QL{\aLoc}{\mRLX}] &= [\aForm/\Qr{\aLoc}]
      \\
      [\aForm/\QF{}{\fACQ}] &= [\aForm/\Qr{*},\aForm/\Qw{*}]
      &{} [\aForm/\QS{\aLoc}{\mREL}] &= [\aForm/\Qw{\aLoc}]
      &{} [\aForm/\QL{\aLoc}{\mACQ}] &= [\aForm/\Qr{*},\aForm/\Qw{*}]
      \\
      [\aForm/\QF{}{\fSC}] &= \Qsuball{\aForm}%[\aForm/\Qr{*},\aForm/\Qw{*},\aForm/\Qsc]
      &{} [\aForm/\QS{\aLoc}{\mSC}] &= [\aForm/\Qw{\aLoc},\aForm/\Qsc]
      &{} [\aForm/\QL{\aLoc}{\mSC}] &= \Qsuball{\aForm}%[\aForm/\Qr{*},\aForm/\Qw{*},\aForm/\Qsc]
    \end{align*}
  \end{scope}
\end{definition}

With these notations in hand, we can modify the semantics of
\textsection\ref{sec:model} as follows.  (We leave the generalization to the
semantics of \textsection\ref{sec:additional} as future work.)
\begin{definition}
  \label{def:q}
  Update the following rules from \reffig{fig:seq}.  
    \begin{enumerate}[topsep=0pt,label=(\textsc{f}\arabic*),ref=\textsc{f}\arabic*]
    \setcounter{enumi}{\value{Bkappa}}
  \item \label{fence-kappa-q}
    $\labelingForm(\aEv) \riff \QF{}{\amode}$,
    \stepcounter{enumi}
  \item[] \labeltext[\textsc{f}4]{}{fence-tau-q}
    \begin{enumerate}[leftmargin=0pt]
      \item \label{fence-tau-dep-q}
        if $\aEvs\cap\bEvs\neq\emptyset$ then
      \begin{math}
        \aTr{\bEvs}{\bForm} \riff 
        \bForm,
      \end{math}
      \item \label{fence-tau-ind-q}
        if $\aEvs\cap\bEvs=\emptyset$ then
      \begin{math}
        \aTr{\bEvs}{\bForm} \riff 
        \bForm
        [\FALSE/\QF{}{\amode}].
      \end{math}
    \end{enumerate}
  \end{enumerate}
  \smallskip

  \begin{enumerate}[topsep=0pt,label=(\textsc{w}\arabic*),ref=\textsc{w}\arabic*]
    \setcounter{enumi}{\value{Bkappa}}
  \item \label{write-kappa-q}
    \begin{math}
      \labelingForm(\aEv) \riff
      \QS{\aLoc}{\amode}
      \land
      \aExp{=}\aVal
    \end{math},
    \stepcounter{enumi}
  \item[] \labeltext[\textsc{w}4]{}{write-tau-q}
    \begin{enumerate}[leftmargin=0pt]
      \item \label{write-tau-dep-q}
        if $\aEvs\cap\bEvs\neq\emptyset$ then
      \begin{math}
        \aTr{\bEvs}{\bForm} \riff 
        \bForm
        [\aExp/\aLoc][(\Qw{\aLoc}\land\aExp{=}\aVal)/\Qw{\aLoc}],
      \end{math}
      \item \label{write-tau-ind-q}
        if $\aEvs\cap\bEvs=\emptyset$ then
      \begin{math}
        \aTr{\bEvs}{\bForm} \riff 
        \bForm
        [\aExp/\aLoc][\FALSE/\QS{\aLoc}{\amode}].
      \end{math}
    \end{enumerate}
  \end{enumerate}
  \smallskip

  \begin{enumerate}[topsep=0pt,label=(\textsc{r}\arabic*),ref=\textsc{r}\arabic*]
    \setcounter{enumi}{\value{Bkappa}}
  \item \label{read-kappa-q}
    \begin{math}
      \labelingForm(\aEv) \riff \QL{\aLoc}{\amode},
    \end{math}
    \stepcounter{enumi}
  \item[] \labeltext[\textsc{r}4]{}{read-tau-q}
    \begin{enumerate}[leftmargin=0pt]
      \item \label{read-tau-dep-q}
        if $\aEv\in\aEvs\cap\bEvs$ then
        \makebox[0pt][l]{%
          \begin{math}
            \aTr{\bEvs}{\bForm} \riff
            (\Qw{\aLoc}\limplies\aVal{=}\aReg)
            \limplies \bForm,
          \end{math}
        }
    \item \label{read-tau-ind-q}
      if $\aEv\in\aEvs\setminus\bEvs$ then
      \makebox[0pt][l]{\begin{math}
          \aTr{\bEvs}{\bForm} \riff
          \PBR{
            \Qw{\aLoc}
            \limplies
            \PBR{
              \aVal{=}\aReg \lor \aLoc{=}\aReg \lor\RW
            }
          }
          \limplies \bForm [\FALSE/\QL{\aLoc}{\amode}],
        \end{math}}
    \item \label{read-tau-empty-q}
      if $\aEvs=\emptyset$ then
      \begin{math}
        \aTr{\bEvs}{\bForm} \riff
        \bForm [\FALSE/\QL{\aLoc}{\amode}].
      \end{math}
    \end{enumerate}
  \end{enumerate}    
  \smallskip

  A \PwT{} is \emph{complete} if
  \begin{enumerate}[,label=(\textsc{c}\arabic*),ref=\textsc{c}\arabic*]
    \setcounter{enumi}{3}
  \item[] \labeltext[\textsc{c}3]{}{top-kappa-q}
    \begin{enumerate}[leftmargin=0pt]
    \item \label{top-kappa-write-q}
      if $\labeling(\aEv)$ is a write then $\labelingForm(\aEv)[\TRUE/\RW]\Qsuball{\TRUE}$ is a tautology,
    \item \label{top-kappa-read-q}
      if $\labeling(\aEv)$ is a read then $\labelingForm(\aEv)[\FALSE/\RW]\Qsuball{\TRUE}$ is a tautology,
    \end{enumerate}
    \setcounter{enumi}{\value{Bterm}}
  \item \label{top-term-q} $\aTerm{}\Qsuball{\TRUE}$ is a tautology.
  \end{enumerate}

\end{definition}
% \begin{definition}
%   \label{def:q:ca:addr}
%   Update the following rules from \refdef{def:semcaaddr}.  
%   \begin{enumerate}[topsep=0pt,label=(\textsc{w}\arabic*),ref=\textsc{w}\arabic*]
  \setcounter{enumi}{\value{Bkappa}}
\item \label{write-kappa-q-ca-addr}
  \begin{math}
    \labelingForm(\aEv) \riff
    \aForm_\aEv \land
    \cExp{=}\cVal_\aEv
    \land
    \QS{\REF{\cVal_\aEv}}{\amode}
    \land
    \aExp{=}\aVal_\aEv
  \end{math},
  \stepcounter{enumi}
\item[] \labeltext[\textsc{w}4]{}{write-tau-q-ca-addr}
  \begin{enumerate}[leftmargin=0pt]
  \item \label{write-tau-dep-q-ca-addr}
    if $\aEvs\cap\bEvs\neq\emptyset$ then
    \begin{math}
      \aTr{\bEvs}{\bForm} \riff 
      \bForm
      [\aExp/\aLoc][(\Qw{\aLoc}\land\aExp{=}\aVal)/\Qw{\aLoc}],
    \end{math}
  \item \label{write-tau-ind-q-ca-addr}
    if $\aEvs\cap\bEvs=\emptyset$ then
    \begin{math}
      \aTr{\bEvs}{\bForm} \riff 
      \bForm
      [\aExp/\aLoc][\FALSE/\QS{\aLoc}{\amode}],
    \end{math}
  \end{enumerate}
\item Hmmmm....
  \begin{math}
    \aTr{\bEvs}{\bForm} \riff
    \textstyle\bigwedge_{\dVal\in\Val}
    \cExp{=}\dVal
    \limplies
    \bForm
    [\aExp/{\dVal}]
    [\kappaE{\REF{\dVal}}/\Q{\REF{\dVal}}],
  \end{math}
\end{enumerate}
\smallskip

\begin{enumerate}[topsep=0pt,label=(\textsc{r}\arabic*),ref=\textsc{r}\arabic*]
  \setcounter{enumi}{\value{Bkappa}}
\item \label{read-kappa-q-ca-addr}
  \begin{math}
    \labelingForm(\aEv) \riff 
        \aForm_\aEv
        \land \cExp{=}\cVal_\aEv
        \land \QL{\REF{\cVal_\aEv}}{\amode}
  \end{math}
\item \label{read-tau-q-ca-addr}
  \begin{math}
    \begin{aligned}[t]
      \aTr{\bEvs}{\bForm} \riff
      &\textstyle\bigwedge_{\aEv\in\aEvs\cap\bEvs}
      {
        \aForm_\aEv
        \limplies
        \PBR{
          \labelingForm(\aEv)
          \limplies
          \aVal_\aEv{=}\uReg{\aEv}
        }
        \limplies \bForm[\uReg{\aEv}/\aReg]
      }
      \\[-.5ex]
      \land
      &\textstyle\bigwedge_{\aEv\in\aEvs\setminus\bEvs}
      {
        \aForm_\aEv
        \limplies
        \PBR{
          \labelingForm(\aEv)
          \limplies
          \PBR{
            \aVal_\aEv{=}\uReg{\aEv}
            \lor
            \REF{\cVal_\aEv}\EQ\uReg{\aEv}
          }
        }
        \limplies
        \bForm[\uReg{\aEv}/\aReg][\FALSE/\QL{\aLoc}{\amode}]
      }
      \\[-.5ex]
      \land
      &
      % \lnot\kappaE{} % \emptyForm
      \textstyle (\bigwedge_{\aEv\in\aEvs}\lnot\aForm_\aEv)
      \limplies 
      (\forall\bReg)\;
      \bForm[\bReg/\aReg][\FALSE/\QL{\aLoc}{\amode}],
    \end{aligned}
  \end{math}
\end{enumerate}    

% \end{definition}
The preconditions and the independent transformers have changed.  With the
exception of write, the dependent transformers are unchanged.  For writes,
the interpretation of $\Qw{x}$ of subtly different from that of the old
$\Q{x}$---the transformer strengthens $\Qw{\aLoc}$ to
$(\Qw{\aLoc}\land\aExp{=}\aVal)$ rather than replacing it by $\aExp{=}\aVal$.
In order to ensure coherence, we have given up on initialization.



The precondition indicates which sequentially preceding events must be
ordered before.  For example, all preceding accesses must be ordered before a
releasing write, whereas only writes to the same location must be ordered
before a acquiring read---the latter is due to coherence.

Symmetrically, the transformer indicates which sequentially following must be
ordered after.  For example, all following accesses must be ordered after an
acquiring read, whereas only writes to the same location must be ordered
after a releasing write read---again, the latter is due to coherence.

% The quiescence substitutions update quiescence symbols in subsequent code.
% For subsequent independent code, $\ref{write-kappa-q}$ and $\ref{read-kappa-q}$ substitute false.
% In complete pomsets, we substitute true for .
% %
% For example, we substitute $\FALSE$ for $\QS{\aLoc}{\mREL}$ in the independent
% case for a releasing write; this ensures that subsequent writes to $\aLoc$
% follow the releasing write in top-level pomsets.  Similarly, we substitute
% $\FALSE$ for $\QL{\aLoc}{\mACQ}$ in the independent case for an acquiring
% write; this ensures that all subsequent accesses follow the acquiring read in
% top-level pomsets.

\reffig{fig:q:example} shows  the effect of quiescence for each access mode.
\begin{figure}
  \input{fig-q-example.tex}
  \caption{The Effect of Quiescence for Each Access Mode}
  \label{fig:q:example}
\end{figure}

\begin{example}
  The definition enforces publication.  Consider:
  \begin{align*}
    \begin{gathered}[t]
      \PW{x}{1}
      \\
      \hbox{\begin{tikzinlinesmall}[node distance=.5em and 1.5em]
          \event{a}{1{=}v\land\Qr{x}\land\Qw{x}\bigmid\DW{x}{v}}{}
          \xform{xi}{\bForm[1/x][\FALSE/\Qw{x}]}{above=of a}
          \xform{xd}{\bForm[1/x][(1{=}v\land \Qw{x})/\Qw{x}]}{below=of a}
          \xos{a}{xd}
        \end{tikzinlinesmall}}
    \end{gathered}
    &&
    \begin{gathered}[t]
      \PW[\mREL]{y}{1}
      \\
      \hbox{\begin{tikzinlinesmall}[node distance=.5em and 1.5em]
          \raeventX{a}{1{=}v\land\Qr{*}\land\Qw{*}\bigmid\DW{y}{u}}{}
          \xform{xi}{\bForm[1/y][\FALSE/\Qw{y}]}{above=of a}
          \xform{xd}{\bForm[1/y][(1{=}u\land \Qw{y})/\Qw{y}]}{below=of a}
          \xos{a}{xd}
        \end{tikzinlinesmall}}
    \end{gathered}
  \end{align*}
  Since $\Qw{*}[\FALSE/\Qw{\aLoc}]$ is $\FALSE$, we must
  introduce order to get a satisfiable precondition for $\DWP{y}{u}$.
  % composing these without order simplifies to:
  % \begin{gather*}
  %   \PW{x}{1}\SEMI \PW[\mREL]{y}{1}
  %   \\
  %   \hbox{\begin{tikzinline}[node distance=.5em and 1.5em]
  %         \event{a1}{\QS{\aLoc}{\mRLX}\bigmid\DW{x}{1}}{}
  %         \xform{x1d}{\bForm}{below right=of a1}
  %         \xform{x2i}{\bForm[\FALSE/\QS{\bLoc}{\mREL}]}{below=of a1}
  %         \xo{a1}{x1d}
  %         \raevent{a2}{\FALSE\bigmid\DW{\bLoc}{1}}{above right=of x1d}
  %         %\xform{x2d}{\bForm}{below left=of a2}
  %         \xform{x1i}{\bForm[\FALSE/\QS{\aLoc}{\mRLX}]}{below=of a2}
  %         \xform{xii}{\bForm[\FALSE/\QS{\bLoc}{\mREL}][\FALSE/\QS{\aLoc}{\mRLX}]}{below right=of a2}
  %         \xo{a2}{x1d}
  %         \xos[xleft]{a1}{x2i}
  %         \xos{a2}{x1i}
  %       \end{tikzinline}}
  % \end{gather*}
  % In order to get a satisfiable precondition for $\DWP{y}{1}$, we must
  % introduce order:
  % \begin{gather*}
  %   % \PW{x}{1}\SEMI \PW[\mREL]{y}{1}
  %   % \\
  %   \hbox{\begin{tikzinline}[node distance=.5em and 1.5em]
  %         \event{a1}{\QS{\aLoc}{\mRLX}\bigmid\DW{x}{1}}{}
  %         \xform{x1d}{\bForm}{below right=of a1}
  %         \xform{x2i}{\bForm[\FALSE/\QS{\bLoc}{\mREL}]}{below=of a1}
  %         \xo{a1}{x1d}
  %         \raevent{a2}{\QS{\bLoc}{\mREL}\bigmid\DW{\bLoc}{1}}{above right=of x1d}
  %         %\xform{x2d}{\bForm}{below left=of a2}
  %         \xform{x1i}{\bForm[\FALSE/\QS{\aLoc}{\mRLX}]}{below=of a2}
  %         \xform{xii}{\bForm[\FALSE/\QS{\bLoc}{\mREL}][\FALSE/\QS{\aLoc}{\mRLX}]}{below right=of a2}
  %         \xo{a2}{x1d}
  %         \xos[xleft]{a1}{x2i}
  %         \xos{a2}{x1i}
  %         \sync{a1}{a2}
  %       \end{tikzinline}}
  % \end{gather*}
\end{example}

\begin{example}
  \label{ex:subscription}
  The definition enforces subscription.  Consider:
  \begin{align*}
    \begin{gathered}[t]
      \PR[\mACQ]{y}{r}
      \\
      \hbox{\begin{tikzinlinesmall}[node distance=.5em and 1.5em]
          \raeventX{a}{\Qw{y}\bigmid\DR{y}{v}}{}
          \xform{xi}{\PBR{\Qw{y}\limplies\PBR{\aVal{=}\aReg \lor \aLoc{=}\aReg} \lor \RW} \limplies \bForm[\FALSE/\Qr{*}][\FALSE/\Qw{*}]}{above=of a}
          \xform{xd}{\PBR{\Qw{y}\limplies v{=}r}\limplies\bForm}{below=of a}
          \xos{a}{xd}
        \end{tikzinlinesmall}}
    \end{gathered}
    &&
    \begin{gathered}[t]
      \PR{x}{r}
      \\
      \hbox{\begin{tikzinlinesmall}[node distance=.5em and 1.5em]
          \event{a}{\Qw{x}\bigmid\DR{x}{u}}{}
          \xform{xi}{\PBR{\Qw{x}\limplies\PBR{\aVal{=}\aReg \lor \aLoc{=}\aReg} \lor \RW} \limplies \bForm[\FALSE/\Qr{x}]}{above=of a}
          \xform{xd}{\PBR{\Qw{x}\limplies u{=}r}\limplies\bForm}{below=of a}
          \xos{a}{xd}
        \end{tikzinlinesmall}}
    \end{gathered}
  \end{align*}
  Since $\Qw{x}[\FALSE/\Qw{*}]$ is $\FALSE$, we must
  introduce order to get a satisfiable precondition for $\DRP{x}{u}$.
  % Since $\QL{\aLoc}{\mRLX}[\FALSE/\QL{\bLoc}{\mACQ}]$ is $\FALSE$,
  % composing these without order simplifies to:
  % \begin{gather*}
  %   \PR[\mACQ]{y}{r}\SEMI \PR{x}{s}
  %   \\
  %   \hbox{\begin{tikzinline}[node distance=.5em and 1.5em]
  %         \raevent{a1}{\QL{\bLoc}{\mACQ}\bigmid\DR{y}{1}}{}
  %         \xform{x1d}{r{=}1\limplies\bForm[\FALSE/\QL{\aLoc}{\mRLX}]}{below=of a1}
  %         \xform{xdd}{r{=}1\limplies s{=}1\limplies\bForm}{right=of x1d}
  %         \xform{xii}{\bForm[\FALSE/\QL{\bLoc}{\mACQ}][\FALSE/\QL{\aLoc}{\mRLX}]}{above=of xdd}
  %         \xform{x2d}{s{=}1\limplies\bForm[\FALSE/\QL{\bLoc}{\mACQ}]}{right=of xdd}
  %         \event{a2}{\FALSE\bigmid\DR{x}{1}}{above=of x2d}
  %         \xos[xleft]{a1}{x1d}
  %         \xos{a2}{x2d}
  %         \xo{a1}{xdd}
  %         \xo{a2}{xdd}
  %       \end{tikzinline}}
  % \end{gather*}
  % In order to get a satisfiable precondition for $\DRP{x}{1}$, we must
  % introduce order:
  % \begin{gather*}
  %   % \PR[\mACQ]{y}{r}\SEMI \PR{x}{s}
  %   % \\
  %   \hbox{\begin{tikzinline}[node distance=.5em and 1.5em]
  %         \raevent{a1}{\QL{\bLoc}{\mACQ}\bigmid\DR{y}{1}}{}
  %         \xform{x1d}{r{=}1\limplies\bForm[\FALSE/\QL{\aLoc}{\mRLX}]}{below=of a1}
  %         \xform{xdd}{r{=}1\limplies s{=}1\limplies\bForm}{right=of x1d}
  %         \xform{xii}{\bForm[\FALSE/\QL{\bLoc}{\mACQ}][\FALSE/\QL{\aLoc}{\mRLX}]}{above=of xdd}
  %         \xform{x2d}{s{=}1\limplies\bForm[\FALSE/\QL{\bLoc}{\mACQ}]}{right=of xdd}
  %         \event{a2}{\QL{\aLoc}{\mRLX}\bigmid\DR{x}{1}}{above=of x2d}
  %         \xos[xleft]{a1}{x1d}
  %         \xos{a2}{x2d}
  %         \xo{a1}{xdd}
  %         \xo{a2}{xdd}
  %         \sync[out=15,in=165]{a1}{a2}
  %       \end{tikzinline}}
  % \end{gather*}
\end{example}

\begin{example}
Even in its logical form, \ref{seq-le-delays'} is incompatible with the
ability to strengthen preconditions using augment closure, which is allowed
in \cite{DBLP:journals/pacmpl/JagadeesanJR20}.  Consider the following.
\begin{align*}
  \begin{gathered}[t]
    \IF{r}\THEN\PW{x}{2}\FI
    \\
    \hbox{\begin{tikzinline}[node distance=.5em and 1.5em]
        \event{a1}{r{\neq}0\bigmid\DW{x}{2}}{}
      \end{tikzinline}}    
  \end{gathered}
  &&
  \begin{gathered}[t]
    \PW{x}{1}
    \\
    \hbox{\begin{tikzinline}[node distance=.5em and 1.5em]
        \event{a2}{            \DW{x}{1}}{}
      \end{tikzinline}}    
  \end{gathered}
  &&
  \begin{gathered}[t]
    \PW{x}{2}
    \\
    \hbox{\begin{tikzinline}[node distance=.5em and 1.5em]
        \event{a3}{            \DW{x}{2}}{}
      \end{tikzinline}}    
  \end{gathered}
  &&
  \begin{gathered}[t]
    \IF{\BANG r}\THEN\PW{x}{1}\FI
    \\
    \hbox{\begin{tikzinline}[node distance=.5em and 1.5em]
        \event{a4}{r{=}0   \bigmid\DW{x}{1}}{}
      \end{tikzinline}}    
  \end{gathered}
\end{align*}
% \begin{align*}
%   \begin{gathered}[t]
%     \IF{r}\THEN\PW{x}{2}\FI
%     \SEMI
%     \PW{x}{1}
%     \SEMI
%     \PW{x}{2}
%     \SEMI
%     \IF{\BANG r}\THEN\PW{x}{1}\FI
%     \\
%     \hbox{\begin{tikzinline}[node distance=.5em and 1.5em]
%         \event{a1}{r{\neq}0\bigmid\DW{x}{2}}{}
%         \event{a2}{            \DW{x}{1}}{right=of a1}
%         \event{a3}{            \DW{x}{2}}{right=of a2}
%         \event{a4}{r{=}0   \bigmid\DW{x}{1}}{right=of a3}
%       \end{tikzinline}}    
%   \end{gathered}
% \end{align*}
If $r{=}0$ then $x$ is $1,2,1$.  If $r{\neq}0$ then $x$ is $2,1,2$.
Augmenting the middle preconditions and then using sequential composition, we have:
\begin{align*}
  \begin{gathered}[t]
    \IF{r}\THEN\PW{x}{2}\FI
    \\
    \hbox{\begin{tikzinline}[node distance=.5em and 1.5em]
        \event{a1}{r{\neq}0\bigmid\DW{x}{2}}{}
      \end{tikzinline}}    
  \end{gathered}
  &&
  \begin{gathered}[t]
    \PW{x}{1}
    \SEMI
    \PW{x}{2}
    \\
    \hbox{\begin{tikzinline}[node distance=.5em and 1.5em]
        \event{a2}{r{\neq}0\bigmid\DW{x}{1}}{}
        \event{a3}{r{=}0   \bigmid\DW{x}{2}}{right=of a1}
      \end{tikzinline}}    
  \end{gathered}
  &&
  \begin{gathered}[t]
    \IF{\BANG r}\THEN\PW{x}{1}\FI
    \\
    \hbox{\begin{tikzinline}[node distance=.5em and 1.5em]
        \event{a4}{r{=}0   \bigmid\DW{x}{1}}{}
      \end{tikzinline}}    
  \end{gathered}
\end{align*}
Note that \ref{seq-le-delays'} does not require any order between the two
writes of the middle pomset.
% \begin{align*}
%   \begin{gathered}[t]
%     \hbox{\begin{tikzinline}[node distance=.5em and 1.5em]
%         \event{a1}{r{\neq}0\bigmid\DW{x}{2}}{}
%         \event{a2}{r{=}0   \bigmid\DW{x}{1}}{right=of a1}
%         \event{a3}{r{\neq}0\bigmid\DW{x}{2}}{right=of a2}
%         \event{a4}{r{=}0   \bigmid\DW{x}{1}}{right=of a3}
%       \end{tikzinline}}    
%   \end{gathered}
% \end{align*}
Merging left and right, we have:
\begin{align*}
  \begin{gathered}[t]
    \IF{r}\THEN\PW{x}{2}\FI
    \SEMI
    \PW{x}{1}
    \SEMI
    \PW{x}{2}
    \SEMI
    \IF{\BANG r}\THEN\PW{x}{1}\FI
    \\
    \hbox{\begin{tikzinline}[node distance=.5em and 1.5em]
        \event{a1}{\DW{x}{2}}{}
        \event{a4}{\DW{x}{1}}{right=of a1}
        \wki{a1}{a4}
      \end{tikzinline}}    
  \end{gathered}
\end{align*}
As shown by the following single-threaded code, allowing this outcome would violate \drfsc{}.
\begin{align*}
  \begin{gathered}[t]
    \PW{y}{1}
    \SEMI
    \PR{y}{r}
    \SEMI
    \IF{r}\THEN\PW{x}{2}\FI
    \SEMI
    \PW{x}{1}
    \SEMI
    \PW{x}{2}
    \SEMI
    \IF{\BANG r}\THEN\PW{x}{1}\FI
    \\
    \hbox{\begin{tikzinline}[node distance=.5em and 1.5em]
        \event{a1}{\DW{x}{2}}{}
        \event{a4}{\DW{x}{1}}{right=of a1}
        \wki{a1}{a4}
        \event{b2}{\DR{y}{1}}{left=of a1}
        \event{b1}{\DW{y}{1}}{left=of b2}
        \rf{b1}{b2}
      \end{tikzinline}}    
  \end{gathered}
\end{align*}
This is one reason that we use \emph{weakest} preconditions, rather than
preconditions.

The same problem does not occur due to if-introduction, since complete
pomsets require that the termination condition is a tautology; therefore we
cannot arbitrarily strengthen preconditions without introducing a second
event to cover.
% , so you can't
% arbitrarily choose to partition $\emptyForm\neq\TRUE$:
\begin{align*}
  \begin{gathered}[t]
    \IF{r}\THEN\PW{x}{2}\FI
    \\
    \hbox{\begin{tikzinline}[node distance=.5em and 1.5em]
        \event{a1}{r{\neq}0\bigmid\DW{x}{2}}{}
      \end{tikzinline}}    
  \end{gathered}
  &&
  \begin{gathered}[t]
    \PW{x}{1}
    \SEMI
    \PW{x}{2}
    \\
    \hbox{\begin{tikzinline}[node distance=.5em and 1.5em]
        \event{a2}{r{\neq}0\bigmid\DW{x}{1}}{}
        \event{a2p}{r{=}0\bigmid\DW{x}{1}}{below=of a2}
        \event{a3}{r{=}0\bigmid\DW{x}{2}}{right=of a1}
        \event{a3p}{r{\neq}0\bigmid\DW{x}{2}}{below=of a3}
        \wki{a2}{a3p}
        \wki{a2p}{a3}
      \end{tikzinline}}    
  \end{gathered}
  &&
  \begin{gathered}[t]
    \IF{\BANG r}\THEN\PW{x}{1}\FI
    \\
    \hbox{\begin{tikzinline}[node distance=.5em and 1.5em]
        \event{a4}{r{=}0\bigmid\DW{x}{1}}{}
      \end{tikzinline}}    
  \end{gathered}
\end{align*}
Merging left and right, we have
\begin{align*}
  \begin{gathered}[t]
    \IF{r}\THEN\PW{x}{2}\FI
    \SEMI
    \PW{x}{1}
    \SEMI
    \PW{x}{2}
    \SEMI
    \IF{\BANG r}\THEN\PW{x}{1}\FI
    \\
    \hbox{\begin{tikzinline}[node distance=.5em and 1.5em]
        \event{a1a3}{\DW{x}{2}}{}
        \event{a2p}{r{=}0\bigmid\DW{x}{1}}{right=of a1a3}
        \event{a3p}{r{\neq}0\bigmid\DW{x}{2}}{right=of a2p}
        \event{a2a4}{\DW{x}{1}}{right=of a3p}
        \wki[out=-165,in=-10]{a2a4}{a3p}
        \wki[out=-170,in=-15]{a2p}{a1a3}
        \wki{a3p}{a2a4}
        \wki{a1a3}{a2p} 
      \end{tikzinline}}    
  \end{gathered}
\end{align*}
\end{example}


\subsection{Optimizations Not Considered}

We have not considered the following optimizations advocated by
\citet{Manson:2005:JMM:1047659.1040336}:
\begin{itemize}
\item synchronization on thread local objects can be ignored or removed
  altogether (the caveat to this is the fact that invocations of methods like
  wait and notify have to obey the correct semantics – for example, even if
  the lock is thread local, it must be acquired when perform- ing a wait),
\item volatile fields of thread local objects can be treated as normal
  fields,
\item redundant synchronization (e.g., when a synchronized method is called
  from another synchronized method on the same object) can be ignored or
  removed.
\end{itemize}
Nor have we attempted to capture the following:
\begin{itemize}
\item read introduction,
\item \emph{monotonicity}, which allows the access mode to strength, for
  example from $\mRLX$ to $\mACQ$ to $\mSC$,
\item access elimination, such as store forwarding, dead-write-removal,
  redundant write after read elimination \cite[\textsection4.1]{DBLP:conf/ecoop/SevcikA08}.
\end{itemize}
One approach to elimination would be to allow {merging} of actions with
different labels.  A list of safe merges can be found in \cite[\textsection
E]{DBLP:conf/cgo/ChakrabortyV17} and \cite[\textsection7.1]{Kang19}.  For
examples of unsafe merges and reorderings, see \cite[\textsection
D]{DBLP:conf/cgo/ChakrabortyV17}.  See also
\cite[\textsection6.2]{DBLP:journals/pacmpl/ChakrabortyV19}

% It would be nice if we could get at these with a strength reducing result:
% synchronization actions can be replaced by relaxed actions in some cases.
% Then the rules for relaxed read elimination and relaxed write elimination can
% be used to get rid of them.

Certain combinations of optimizations are quite delicate.  For example,
consider if-introduction and dead-write-removal.  With if-introduction, the
following equation should hold:
\begin{align*}
  \sem{
    \IF{r}\THEN\PW{x}{2}\FI
    \SEMI
    \PW{x}{1}
    \SEMI
    \PW{x}{2}
    \SEMI
    \IF{\BANG r}\THEN\PW{x}{1}\FI
    \SEMI
    \PW{x}{3}
  }&
  \\ =
  \sem{
    \IF{\BANG r}\THEN\PW{x}{1}\FI
    \SEMI
    \PW{x}{2}
    \SEMI
    \PW{x}{1}
    \SEMI
    \IF{r}\THEN\PW{x}{2}\FI
    \SEMI
    \PW{x}{3}
  }&
\end{align*}
Using dead write removal naively, these could be refined, respectively, to:
\begin{align*}
  \sem{
    \PW{x}{1}
    \SEMI
    \PW{x}{2}
    \SEMI
    \PW{x}{3}
  }&
  \\ \mathrel{\smash{\overset{?}{=}}}
  \sem{
    \PW{x}{2}
    \SEMI
    \PW{x}{1}
    \SEMI
    \PW{x}{3}
  }&
\end{align*}
Depending upon the details of the model, these may be observably different.

What has become of coherence?

% Counterexample for first two (at least, for our MCA1 semantics):
% \begin{verbatim}
%  y=1; x^AR=1; r=x^AR; z=1
% \end{verbatim}
% If you see $z=1$ you must see $y=1$

\subsection{The State of the Art Circa 2021}

\citet{DBLP:conf/java/Pugh99} noticed that the semantics of Java 1.0 disabled
common subexpression elimination.  This lead to a repaired model five years
later \cite{Manson:2005:JMM:1047659.1040336}.  Shortly thereafter,
\citet[\textsection7]{DBLP:conf/esop/CenciarelliKS07} noticed that the
repaired model disabled the reordering of independent statements.  Here is
the example:
\begin{gather*}
  \IF{x\land y}\THEN \PW{z}{1}\FI
  \PAR
  \IF{z}\THEN \PW{x}{1}\SEMI \PW{y}{1} \ELSE \PW{y}{1}\SEMI \PW{x}{1} \FI
  \\
  \hbox{\begin{tikzinline}[node distance=.5em and 1em]
      \event{a1}{\DR{x}{1}}{}
      \event{a2}{\DR{y}{1}}{right=of a1}
      \event{a3}{\DW{z}{1}}{right=of a2}
      \po{a2}{a3}
      \po[out=15,in=165]{a1}{a3}      
      \event{b1}{\DR{z}{1}}{right=3em of a3}
      \event{b2}{\DW{y}{1}}{right=of b1}
      \event{b3}{\DW{x}{1}}{right=of b2}
      % \po{b1}{b2}
      % \po[out=15,in=165]{b1}{b3}
      \rf{a3}{b1}
      \rf[out=-165,in=-15]{b2}{a2}
      \rf[out=-165,in=-15]{b3}{a1}
    \end{tikzinline}}
\end{gather*}
Quoting \citet[\textsection7]{DBLP:conf/esop/CenciarelliKS07}:
\begin{quote}
  After reordering the independent statements in the else branch, a compiler
  may execute assignments $\PW{x}{1}$ and $\PW{y}{1}$ early, so that [the
  execution is allowed].  However, such a behaviour is not legal according to
  the current JMM, as it violates the condition that the happens-before
  orders during validation be consistent with the final happens-before on the
  committed actions. In fact, the latter will have the write to $x$ before the
  write to $y$, but during validation the write to $y$ happens before the write
  to $x$.
\end{quote}

Since then, several models have been proposed.  Many have been revised
repeatedly to repair bugs.  (For example, this paper fixes several errors of
\citet{DBLP:journals/pacmpl/JagadeesanJR20}.)

In this subsection, we provide series of quotations from a discussion on the
OpenJDK mailing lists, which provides an excellent summary of the state of
the art in 2021, when this paper was written.  (The quotes are ordered for
readability, with hyperlinks to the original discussion.)

\begin{quotation}
\href{https://mail.openjdk.java.net/pipermail/jdk-dev/2021-August/005904.html}{Raffaello Giulietti}:
``JEP 188: Java Memory Model Update'' \href{https://openjdk.java.net/jeps/188}{[1]}, the JMM wiki \href{https://wiki.openjdk.java.net/display/jmm/Main}{[2]} and the 
jmm-dev mailing list \href{https://mail.openjdk.java.net/pipermail/jmm-dev/}{[3]} seem quite inactive. (The latter point explains 
why I'm posting to this list instead.)

The introduction of \texttt{j.l.i.VarHandle} \href{https://docs.oracle.com/en/java/javase/16/docs/api/java.base/java/lang/invoke/VarHandle.html}{[4]} brought more access modes to 
Java, but in a narrative and informal way. A paper by Bender \& Palsberg 
\href{https://dl.acm.org/doi/10.1145/3360568}{[5]}, addressing the formalization of the concurrent access modes, has 
been published in 2019 but I'm not sure if it caught the attention of 
the OpenJDK community.

So what is the current thinking for progressing the JMM spec?

\smallskip

\href{https://mail.openjdk.java.net/pipermail/jdk-dev/2021-August/005909.html}{Hans Boehm}:
I think it's safe to say that it has been slow going, not just for Java,
but for other languages as well.

In my view, the core problem, shared by pretty much all of them, is that we
don't have an established way to give well-defined semantics to potentially
racing unordered accesses, like ordinary variable accesses in Java, or
\texttt{memory\_order\_relaxed} accesses in C and C++. That's particularly essential
with the traditional Java language-based-security model, since we can't
just give up on racing accesses to ordinary variables.

I'm aware of a number of proposed solutions. But I don't think we currently
have enough confidence that they
\begin{enumerate}[label=(\alph*)]
\item\label{HBa} Are correct, and don't have issues similar to the older models,
\item Don't have unintended consequences, particularly for compilation, and
\item Are sufficiently comprehensible by programmers to actually be useful.
\end{enumerate}
\ref{HBa} is hard because the models have gotten complex enough that reviewers
are scarce. (A problem that I gather you're familiar with.) The authors are
commonly experts at formally analyzing the models, but it's hard to analyze
whether the model conflicts with some well-known, but perhaps not
well-written-down compilation technique.

Probably even more controversially, I think we've realized that existing
compiler technology can compile such racing code in ways that some of us are
not 100\% sure should really be allowed. Demonstrably unexecuted code can
affect the semantics in ways that strike me as scary. (See
\url{https://wg21.link/p1217} for a down-to-assembly C++ version; [if I
understand correctly], Lochbihler and others earlier came up with some
closely related observations for Java.)

It might be possible to do what we've involuntarily done for C++: Punt the
hard cases for now, and define what the model is for programs without
racing ordinary accesses.

%[p1217 is \cite{BoehmOOTA}.]

\smallskip

\href{https://mail.openjdk.java.net/pipermail/jmm-dev/2021-August/000447.html}{Andrew Haley}:
\begin{quote}
  (Quoting
  \href{https://mail.openjdk.java.net/pipermail/jdk-dev/2021-August/005909.html}{Hans
    Boehm}) Probably even more controversially, I think we've realized that
  existing compiler technology can compile such racing code in ways that some
  of us are not 100\% sure should really be allowed.
\end{quote}
This implies, does it not, that the problem is not formalization as
such, but that we don't really understand what the language is
supposed to mean? That's always been my problem with OOTA: I'm unsure
whether the problem is due to the inadequacy of formal models, in
which case the formalists can fix their own problem, or something we
all have to pay attention to.

\smallskip

\href{https://mail.openjdk.java.net/pipermail/jmm-dev/2021-August/000450.html}{Hans
  Boehm}: In some sense, I'm not sure either. The p1217 examples [formalized
below as \ref{RFUB} and \ref{RFUB-NC}] bother me in that they seem to violate
some global programming rules (``if \texttt{x} is only ever null or refers to
an object properly constructed by the same thread, then \texttt{x} should
never appear to refer to an incompletely constructed object'').  And there
seems to be disagreement about whether the currently allowed behavior is
``correct.''

On the other hand, in practice the weirdness doesn't seem to break things.
If you ask people advocating the current behavior, the answer will be
that it doesn't matter because nobody writes code that way. If you ask
people trying to analyzer or verify code, they'll probably be unhappy.
And I haven't been able to convince myself that you cannot get yourself
into these situations just by linking components together, each of which
does something perfectly reasonable.

And there are very common code
patterns (like the standard implementation of reentrant locks used
by all Java implementations) that break if you allow general OOTA
behavior. Which at least means that you can't currently formally verify such
code. The theorem you'd be trying to prove is false with respect to the
part of the language spec we know how to formalize.

It's a mess.


\smallskip

\href{https://mail.openjdk.java.net/pipermail/jmm-dev/2021-August/000447.html}{Andrew Haley}:
\begin{quote}
  (Quoting
  \href{https://mail.openjdk.java.net/pipermail/jdk-dev/2021-August/005909.html}{Hans
    Boehm}) Demonstrably unexecuted code can affect the semantics in ways
  that strike me as scary. (See wg21.link/p1217 for a down-to-assembly C++
  version; [if I understand correctly], Lochbihler and others earlier came up
  with some closely related observations for Java.)
\end{quote}
Looking again at p1217, it seems to me that enforcing load-store
ordering would have severe effects on compilers, at least without new
optimization techniques. We hoist loads before loops and sink stores
after them. When it all works out, there are no memory accesses in the
loop. A load-store barrier in a loop would have the effect of forcing
succeeding stores out to memory, and forcing preceding loads to reload
from memory. It's not hard to imagine that this would cause an
order-of-margnitude performance reduction in common cases.

I suppose one could argue that such optimizations would continue to be
valid, so only those stores which would have been emitted anyway would
be affected. But that's not how compilers work, as far as I know. In
our IR for C2, memory accesses are not pinned in any way, so the only
way to make unrelated accesses execute in any particular order is to
add a dependency between all loads and stores.

\smallskip

\href{https://mail.openjdk.java.net/pipermail/jmm-dev/2021-August/000450.html}{Hans Boehm}:
I think it would be a fairly pervasive change to optimizers. It has also
become clear in WG21, the C++ committee, that there is not enough
support for requiring this. In that case, Ou and Demsky have a paper
saying that the overhead is likely to be on the order of 1\% or less.
For Java if it were applied everywhere, it would probably be
appreciably higher.

On the other hand, it's a bit harder than that to come up with examples
where
the generated x86 code has to be worse. Moving loads earlier in the
code, or delaying stores, as you suggest, would still be fine. The only
issue is with delaying loads past stores, which seems less common,
though it can certainly be beneficial for reducing live ranges, probably
some
vectorization etc.

But it seems unlikely that such a restriction will be applied even to
C++ \texttt{memory\_order\_relaxed}, much less Java ordinary variables.

\smallskip

\href{https://mail.openjdk.java.net/pipermail/jmm-dev/2021-August/000449.html}{Doug
  Lea}:
My stance in the less formal account 
(\url{http://gee.cs.oswego.edu/dl/html/j9mm.html}) as well as Shuyang Liu et 
al's ongoing formalization (see links from 
\url{http://compilers.cs.ucla.edu/people/}) is that the most you want to say 
about racy Java programs is that they are typesafe. As in: you can't see 
a String when expecting an int. Even this looser constraint is 
challenging to specify, prove, and extend. But it is a path for Java 
that might not apply to languages like C that are not guaranteed 
typesafe anyway, and so enter Undefined Behavior territory (as opposed 
to possibly-unexpected but still typesafe behavior).

\smallskip

\href{https://mail.openjdk.java.net/pipermail/jmm-dev/2021-August/000451.html}{Han Boehm}:
But this now breaks some common idioms, right? In particular, I think a
bunch of
code assumes that racing assignments of equivalent primitive values or
immutable
objects to the same field are OK.

If, in 2004, our view of language-based security had been the same as it is
now,
then I completely agree that this would have been the right approach. But I
think
doing it now would require significant user code changes. Which might still
be the best way forward ...
\end{quotation}


\section{Efficient Implementation on ARMv8}
\label{sec:arm}

Alternate characterization of ARM:
\cite[\textsection B2.3.6]{arm-reference-manual}
\cite{alglave-git-alternate}
\cite{armed-cats}

Internal visibility:
If $\bEv\xpoloc\aEv$ then either
\begin{itemize}
\item %$\bEv\xca\aEv$:
  $\bEv:\DWP{x}{1}\xco\aEv:\DWP{x}{2}$ 
\item %$\bEv\xrfx\aEv$:
  $\bEv:\DWP{x}{1}\xrfx\aEv:\DRP{x}{1}$
\item %$\bEv\mathrel{\xco\xrfx}\aEv$:
  $\bEv:\DWP{x}{1}\xco\DWP{x}{2}\xrfx\aEv:\DRP{x}{2}$
\item
  $\bEv:\DRP{x}{0}\xfr\aEv:\DWP{x}{1}$
\item %$\bEv\mathrel{\xfr\xrfx}\aEv$:
  $\bEv:\DRP{x}{0}\xfr\DWP{x}{1}\xrfx\aEv:\DRP{x}{1}$
\item %$\bEv\mathrel{\xrfxinv\xrfx}\aEv$:
  $\bEv:\DRP{x}{1}\xrfxinv\DWP{x}{1}\xrfx\aEv:\DRP{x}{1}$
  % \begin{tikzinline}
  %   \node(a){$\DWP{x}{1}$};
  %   \node(b)[right=1.5em of a]{$\DRP{x}{1}$};
  %   \node(c)[right=1.5em of b]{$\DRP{x}{1}$};
  %   \rfx{a}{b}
  %   \rfx[out=15,in=165]{a}{c}
  % \end{tikzinline}
\end{itemize}
If $\bEv\xpoloc\aEv$ then the following are impossible:
\begin{itemize}
\item $\bEv:\DWP{x}{2}\xcoinv\aEv:\DWP{x}{1}$ 
\item $\bEv:\DWP{x}{1}\xfrinv\aEv:\DRP{x}{0}$ 
\item $\bEv:\DRP{x}{1}\xrfxinv\aEv:\DWP{x}{1}$ 
\item $\bEv:\DRP{x}{2}\xrfxinv\DWP{x}{2}\xcoinv\aEv:\DWP{x}{1}$
\item $\bEv:\DRP{x}{1}\xrfxinv\DWP{x}{1}\xfrinv\aEv:\DRP{x}{0}$
\end{itemize}

\begin{figure*}
\begin{verbatim}
let IM0 = loc & ((IW * (M\IW)) | ((W\FW) * FW))

(* Local read successor *)
let lrs = [W]; po-loc \ intervening-write(po-loc); [R]

(* Local write successor *)
let lws = po-loc; [W]

(* Dependency-ordered-before *)
let dob =  addr | data               
        | (addr | data); lrs
        | addr; po; [W]         |  addr; po; [ISB]; po; [R]
        | ctrl;     [W]         |  ctrl;     [ISB]; po; [R]
our dob = addr; [W]
        | data; [W]
        | ctrl; [W]

(* Barrier-ordered-before *)
let bob = [A | Q]; po            | po; [dmb.full]; po          | po; ([A];amo;[L]); po
        | po; [L]                | [R]; po; [dmb.ld]; po
        | [L]; po; [A]           | [W]; po; [dmb.st]; po; [W]

(* Locally-ordered-before *)                let aob = rmw 
let rec lob = lws | dob | bob | lob; lob            | [W & range(rmw)]; lrs; [A | Q] 

(* Internal visibility requirement *)
acyclic po-loc | fr | co | rf as internal

(* External global completion requirement *)
let gc-req = (W * _) | (R * _) & ((range(rfe) * _) | (rfi^-1; lob))
let preorder-gcb = IM0 | lob & gc-req

with gcb from linearisations(M, preorder-gcb)
~empty gcb

let gcbl = gcb & loc
let rf-gcb = (W * R) & gcbl \ intervening-write(gcbl)
let co-gcb = (W * W) & gcbl

call equal(rf, rf-gcb)
call equal(co, co-gcb)

James:
let po-conflict = po-loc \ (R * R)
Claim we can add po-conflict to preorder-gcb without changing things
let preorder-gcb = IM0 | lob & gc-req | po-conflict
\end{verbatim}
  \caption{ARM}
  \label{fig:arm}
\end{figure*}

\textcolor{red}{[This is out of date, from an old submission]}


In this section, we consider the fragment of our language without
restriction.  For simplicity, we allow release and acquire synchronization
but ban fences.  We assume that all memory locations are initialized to $0$
and parallel-composition occurs only at top level.  We take the set of memory
locations to be finite.  In other words, we assume that programs have the
form
\begin{displaymath}
  {\aLoc_1}\GETS{0}\SEMI
  \cdots\SEMI
  {\aLoc_m}\GETS{0}\SEMI
  (\aCmd^1 \PAR \cdots \PAR \aCmd^n)
\end{displaymath}
where $\aCmd^1$, \ldots, $\aCmd^n$ do not include composition, restriction or
fence operations.

Our language can be translated to ARM using \texttt{ldr} for relaxed read,
\texttt{ldar} for acquiring read, \texttt{str} for relaxed write, and
\texttt{stlr} for releasing write.  Relative to the ARM specification, we
have removed loops and read-modify-write (RMW) operations, in addition to
fences\footnote{We leave out fences for simplicity.  Following
  \citet{DBLP:journals/pacmpl/PodkopaevLV19}, our $\FENCE$ instruction can be
  translated to \texttt{dmb.sy}, since it has release-acquire semantics.
  Acquire fences map to \texttt{dmb.ld}, and release fences to
  \texttt{dmb.sy} --- \texttt{dmb.st} does not provide order to prior
  reads.}.

We show that any ARM-consistent execution graph for this sublanguage can be
considered an execution of our semantics.  
%
% Syntactically, we drop the superscript \textsf{rlx} on relaxed reads and
% writes; in addition, we use structured conditionals rather than the more
% general \textsf{goto}.  We refer to this sublanguage as $\muIMM$.
% (Because the source language lacks RMW operations, the ``is
% exclusive'' flag on every read will be \textsf{not-ex} and the RMW mode on
% every write will be \textsf{normal}.)
%
Due to space limitations, we do not include a full description ARM
consistency in the main text .
Here we provide a birds eye view of the details, drawing on the intuitions gleaned from~\citep{DBLP:journals/pacmpl/PulteFDFSS18}.  
Interested readers should see \textsection\ref{sec:arm:proof}
for further details.

An ARM execution graph $G$ defines many relations, including program order
($\rpox$), reads-from ($\rrfx$), coherence ($\rco$) and several dependency
orders.  From these are derived:
\begin{itemize}
\item ${\rpoloc}$, which is the subrelation of $\rpox$ that only relates
  actions on the same location,
\item ${\rob}$, which is required to be acyclic (\ref{external}), and
\item $\reco$, with the requirement that ${\rpoloc}\cup{\reco}$ be acyclic (\ref{sc-per-loc}).
\end{itemize}
% Let $G$ be an execution graph satisfying the ARM consistency 
%requirements.
Given an execution graph $G$, we say that $\aEv$ is an \emph{internal read} if $\aEv\in\fcodom({\rpox}\cap {\rrfx})$.

The ${\rob}$ order is an acyclic global order on events, agreed upon by all threads, reflecting the progress of time in an \armeight\ execution.  The cross thread component of the ordering is induced by the ordering on conflicting actions on the same location from different threads.    The intra thread component of the ordering is induced by barrier ordering and data ordering.  Notably, these dependencies  are determined syntactically.  In particular, $\rob$
may not necessarily include the intra thread component of $\rpoloc$ ordering.  

This motivates the translation of an \armeight\ execution into our setting.  In our setting, the progress of time is given by $\lt$.   We accommodate intra-thread reordering by internal read actions, thus excusing us from the obligation of placing them on the global $\lt$-timeline.  

Formally, from $G$ we construct a candidate pomset $\aPS$ as follows:
\begin{itemize}
\item $\Event= \textsf{E}$,
\item $\labelingAct(\aEv)=\tau \mathsf{lab}(e)$, if $\aEv$ is a relaxed
  internal read, 
\item $\labelingAct(\aEv)=\mathsf{lab}(e)$, if $\aEv$ is not a relaxed
  internal read,
\item $\labelingForm(\aEv)=\TRUE$,
% \item ${\le} = {\rob}^?$, where $?$ denotes reflexive closure, and
\item ${\gtN} = ({\rob} \cup {\reco})^*$, where $*$ denotes reflexive and transitive closure.
\end{itemize}

\begin{theorem}
  If $G$ is ARM consistent, the constructed candidate satisfies the
  requirements for a top-level memory model pomset.
\end{theorem}
Any $\lt$-ordering imposed in our model
is enforced by \armeight, since our notion of semantic dependency is more
permissive than \armeight 's syntactic dependency.  So, the heart of the proof is showing the acyclicity of $({\rob} \cup {\reco})^*$ for the events under consideration.  Since the cross thread portion of $\reco$ ($\rcoe,\rfre,\rrfe$) is included in $\rob$, this result is really about the influence of $\reco \cap \rpox$.  Our translation of \armeight 's \rrfi\  as silent internal actions removes them from order considerations.   Consequently, we only have to consider the suborder of ${\rob}$ derived without ever using  $\rrfi$ for the following key property demonstrated in \textsection\ref{sec:arm:proof}.
\begin{lemma}\label{extendob}
Let $\aEv, \bEv$ be distinct events and $\bEv'\ (\xob\cap \xpox) \ \bEv\ ((\xeco\cap \xpox) \setminus \xrfi) \  \aEv\ (\xob \cap \xpox)  \ \aEv'$.  Then $\bEv' \xob \aEv'$.
\end{lemma}



\begin{remark}[Proof for TSO]
  The proof for compilation into \tso\ is very similar.  In particular the facts listed above hold for \tso, where $\rob$ is replaced by (the
  transitive closure of) the propagation relation defined for \tso\ 
  \citep{alglave}.
\end{remark}

\section{Proof of compilation for ARMv8}
\label{sec:arm:proof}
\textcolor{red}{[This is out of date, from an old submission]}

In this section, we develop the proof of correctness of compilation to \armeight.  In order to ease readability, we reproduce the definitions from the main text. 



Given a relation $R$, $R^?$ denotes reflexive closure, $R^+$ denotes
transitive closure and $R^*$ denotes reflexive and transitive closure.  Given relations $R$ and $S$, $R;S$ denotes composition.


The ARMv8 model is described using the following relations.
\begin{itemize}
\item $\IDR$, $\IDW$, $\IDAcq$, $\IDRel$: identity on reads, writes, acquires
  and releases.
% \item $\IDR$ identity on reads
% \item $\IDW$: identity on writes
% \item $\IDAcq$: identity on acquires
% \item $\IDRel$: identity on releases
\item $\IDLoc$: relates any two events that touch the same location.
\item $\rpox$: program order.
\item $\rdata$, $\rctrl$, $\raddr$: data, control and address dependencies.
\item $\rrfx$: reads-from. $\rrfx^{-1}$ relates each read to a matching write
  on the same location.
\item $\rco$: coherence, which is a total order on the writes to a single
  location.
\item ${\rfr}\eqdef{\rco};\rrfx^{-1}$: from-read, which relates reads to
  subsequent writes.
\end{itemize}
For any relation, the cross-thread subrelation is denoted by appending $e$;
the intra-thread subrelation is denoted by appending $i$.  For example,
${\rrfe}\eqdef{\rrfx}\setminus{\rpox}$ and ${\rrfi}\eqdef{\rrfx}\cap{\rpox}$.
The subrelation restriction attention to actions on the same location is
given by appending $\mathsf{loc}$.  For example, ${\rpoloc}\eqdef{\rpox}\cap{\IDLoc}$.

The ARMv8 model defines the following relations.
In our presentation, we have elided rules concerning fences and RMW operations.
\begin{align*}
  \tag{Extended coherence}
  {\reco} &\eqdef {\rrf} \cup {\rfr} \cup {\rco}
  \\
  \tag{Observed externally}
  {\robs} &\eqdef \smash{
    {\rrfe} \cup {\rfre} \cup {\rcoe}
  }
  \\
  \tag{Dependency order}
  {\rdob} &\eqdef\smash{
    ({\raddr}\cup{\rdata}); {\rrfi}^?
    \cup ({\rctrl}\cup{\rdata}); {\IDW}; {\rcoi}^?
    \cup {\raddr}; {\rpox}; {\IDW}
  }
  \\
  \tag{Barrier order}
  {\rbob} &\eqdef\smash{
    {\IDAcq}; {\rpox}
    \cup {\rpox};{\IDRel}; {\rcoi}^?
  }
  \\
  \tag{Acyclic order}
  {\rob} &\eqdef\smash{
    ({\robs} \cup {\rdob} \cup {\rbob})^+
  }
\end{align*}
\begin{definition}
  An RMW-free and fence-free execution is \emph{ARM-consistent} if
  \begin{align*}&
    \tag{\textsc{$\rrfx$-completeness}}\label{rf-comp}
    \fcodom(\rrfx)=\fdom(\rreads)
    \\[-1ex]&
    \tag{\textsc{$\rco$-totality}}\label{co-tot}
    \text{For every location $\aLoc$, $\rco$ totally orders the writes of $\aLoc$}  
    \\[-1ex]&
    \tag{\textsc{sc-per-loc}}\label{sc-per-loc}
    {\rpoloc} \cup {\rrfx} \cup {\rfr} \cup {\rco}\;\text{is acyclic}
    \\[-1ex]&
    \tag{\textsc{external}}\label{external}
    {\rob}\;\text{is acyclic}
  \end{align*}
\end{definition}

% Use these to refer to the rules in text:
%\ref{rf-comp} 
%\ref{co-tot}
%\ref{sc-per-loc}
%\ref{external}


Given an execution graph $G$, we say that $\aEv$ is an \emph{internal read} if
$\aEv\in\fcodom(\mathsf{po}\cap \mathsf{rf})$.    We are going to translate internal reads of execution graphs into internal reads of the semantics.  

From $G$ we construct a candidate pomset $\aPS$ as follows:
\begin{itemize}
\item $\Event= \textsf{E}$,
\item $\labelingAct(\aEv)=\tau \mathsf{lab}(e)$, if $\aEv$ is a relaxed
  internal read, 
\item $\labelingAct(\aEv)=\mathsf{lab}(e)$, if $\aEv$ is not a relaxed
  internal read,
\item $\labelingForm(\aEv)=\TRUE$,
%\item ${\le} = {\rob}$, and
\item ${\gtN} = ({\rob} \cup {\reco})^*$
\end{itemize}
To reempphasize, in this candidate pomset, $\rob$ is calculated by considering the definition of $\rob$ without $\rrfi$, ie.:
\begin{align*}
  \tag{Dependency order}
  {\rdob} &\eqdef\smash{
    ({\raddr}\cup{\rdata});
    \cup ({\rctrl}\cup{\rdata}); {\IDW}; {\rcoi}^?
    \cup {\raddr}; {\rpox}; {\IDW}
  }
  \\
  \tag{Barrier order}
  {\rbob} &\eqdef\smash{
    {\IDAcq}; {\rpox}
    \cup {\rpox};{\IDRel}; {\rcoi}^?
  }
  \\
  \tag{Acyclic order}
  {\rob} &\eqdef\smash{
    ({\robs} \cup {\rdob} \cup {\rbob})^+
  }
\end{align*}


We show that $\aPS$ is a top-level pomset, reasoning as follows.
% We establish the criteria for a top-level memory-model pomset:
\begin{itemize}
\item ${\le}$ is a partial order.  This holds since $G.{\rar}$ is acyclic.
% \item If $\bEv \le \aEv$ then $\bEv \gtN \aEv$.  By construction.
% \item If $\bEv \le \aEv$ and $\aEv \gtN \bEv$ then $\bEv = \aEv$.  Proved below.
% \item If $\cEv \le \bEv \gtN \aEv$ or $\cEv \gtN \bEv \le \aEv$ then
%   $\cEv \gtN \aEv$. By construction.
% \end{itemize}

% Next, we establish the criteria for a 3-valued pomset with preconditions (Definition~\ref{def:3pre}).
% \begin{itemize}
\item $\labelingForm(\aEv)$ implies $\labelingForm(\bEv)$ whenever
  $\bEv\le\aEv$.   Trivial, since every formula is $\TRUE$.
% \item $\aPS$ is $\aLoc$-coherent; that is, when restricted to events that
%   read or write $\aLoc$, $\gtN$ forms a partial order.
% \end{itemize}

% Finally, we establish the criteria for a top-level pomset
% (Definition~\ref{def:x-closed}).
% \begin{itemize}
\item $\aEv$ is location independent. Trivial, since every formula is $\TRUE$.
\item If $\aEv$ reads $\aVal$ from $\aLoc$, then there is some $\bEv$ such that
  \begin{itemize}
  \item $\bEv \lt \aEv$,  
  \item $\bEv$ writes $\aVal$ to $\aLoc$, and
  \item if $\cEv$ writes to $\aLoc$
    then either $\cEv \gtN \bEv$ or $\aEv \gtN \cEv$.
  \end{itemize}    
\end{itemize}

\subsection{Proof that  $({\rob} \cup {\reco})^*$ is irreflexive. }

\paragraph*{Proof of lemma~\ref{extendob}. } 


Let $\aEv, \bEv$ be distinct events and $\bEv'\ (\xob\cap \xpox) \ \bEv\ ((\xeco\cap \xpox) \setminus \xrfi) \  \aEv\ (\xob \cap \xpox)  \ \aEv'$.  Then $\bEv' \xob \aEv'$.

\begin{proof}
If $\bEv'$ is an acquire,  or $\aEv$ is an release, or $\aEv'$ is a release, result is immediate.

We next consider the case where $\aEv$ is a read.  In this case,  $\bEv$ is a write.  Since $\bEv\ ((\xeco\cap \xpox) \setminus \xrfi) \  \aEv$, there is a write $\bEv_1$ such that $ \bEv \xcoe\ \bEv_1 \ \xrfe\ \aEv' $.  So, $\bEv \xob \aEv$ and result follows in this case. 


So, it suffices to prove the following assuming that $\bEv'$ is not an acquire and $\aEv'$ is not a release and $\aEv$ is not a release or a read and $\aEv, \bEv$ are distinct.
\begin{itemize}
\item If $\bEv'\ (\xob\cap \xpox)  \ \bEv(\xeco\cap\xpox)\aEv$ then $\bEv'\xob\aEv$.
\item If $\bEv\ (\xeco\cap\xpox) \ \aEv(\xob\cap\xpox)\aEv'$ then $\bEv\xob\aEv'$.
\end{itemize}


We first prove that if $\bEv'\ (\xob \cap \xpox) \ \bEv\ (\xeco \cap \xpox) \ \aEv$ then $\bEv'\xob\aEv$.   Proof proceeds by cases on the witness for $\bEv'\ (\xob\cap \xpox) \ \bEv$. 
\begin{itemize}
\item  If $\bEv' \xbob  \bEv$, then: 
\[ \bEv'\ (\smash{
    {\IDAcq}; {\rpox}
    \cup {\rpox};{\IDRel}; {\rcoi}^?) \ 
  }
\bEv
\]
Since $\bEv'$ is not an acquire, $\bEv' ({\rpox};{\IDRel}; {\rcoi}^?) \bEv$, so $\bEv$ is a write.  Since $\aEv$ is not a read,  $\bEv \xcoi\ \aEv$. Thus, result follows.

\item If $\bEv' \xdob  \bEv$, then: 
\[ \bEv'\ 
\smash{
    ( ({\rctrl}\cup{\rdata}); {\IDW}; {\rcoi}^?
    \cup {\raddr}; {\rpox}; {\IDW}
  } \
\bEv
\]
So, $\bEv$ is a write.  Since $\aEv$ is also a write, we deduce that 
\[ \bEv'\ 
\smash{
    ( ({\rctrl}\cup{\rdata}); {\IDW}; {\rcoi}^?
    \cup {\raddr}; {\rpox}; {\IDW}
  } \
\aEv
\]
\end{itemize}


We next prove  that if $\bEv\ (\xeco \cap \xpox) \ \aEv\ (\xob\cap \xpox) \ \aEv'$ then $\bEv\xob\aEv'$, under the assumptions that  $\aEv'$ is not a release and $\aEv$ is not a release or a read and $\aEv, \bEv$ are distinct.


 Proof proceeds by cases on the witness for $\aEv (\xob\cap \xpox) \aEv'$.  

\begin{itemize}
\item  If $\aEv \xbob  \aEv'$, then: 
\[ \aEv\ (\smash{
    {\IDAcq}; {\rpox}
    \cup {\rpox};{\IDRel}; {\rcoi}^?) \ 
  }
\aEv'
\]
Since $\aEv$ is not a read, $\aEv ({\rpox};{\IDRel}; {\rcoi}^?) \aEv'$.  Result follows since  $\bEv \xpox\ \aEv$.


\item If $\aEv \xdob  \aEv'$, then $\aEv$ is a read.  \qedhere

\end{itemize}
\end{proof}


\begin{lemma}\label{obeco1}
If $\bEv\xob\aEv$ then $\lnot(\aEv\xeco\bEv)$.

\begin{proof}
Proof by contradiction.  Let 
\[ \aEv \xob \aEv' \xeco \bEv' \xob \bEv \xob \cEv \xob \cEv' \xob \aEv \]
where $\aEv' \xpox \bEv'$.

By lemma~\ref{extendob}, if $\aEv \not=\aEv'$, we deduce $\aEv \xob \bEv'$, and thus $\aEv \xob \bEv$.  If $\bEv \not=\bEv'$, we deduce $\aEv' \xob \bEv$ and thus $\aEv \xob \bEv$.

Thus, if $\aEv \not=\aEv'$ or $\bEv \not=\bEv'$, then there is a cycle $\aEv \xob \bEv \xob \cEv \xob \cEv' \xob \aEv$.  

So we can assume that  $\aEv' = \aEv$, $\bEv' = \bEv$ and 
\[ \aEv  \xeco \bEv \xob \cEv \xob \cEv' \xeco \aEv \]
where all of $\aEv, \bEv, \cEv, \cEv'$ access the same location and at least one of $\aEv,\bEv$ is a write, at least one of $\aEv,\cEv'$ is a write, and at least one of $\bEv,\cEv$ is a write.

We reason by cases.
\begin{itemize}
\item If $\cEv'$ is a write or both $(\aEv, \bEv)$ are writes.

We deduce that $\bEv \xeco \cEv' \xeco \aEv$ and thus $\bEv \xeco \aEv$.

\item $\cEv'$ is a read.  $\aEv$ is a write.  $\bEv$ is a read.

In this case $\cEv$ is a write.  From $\cEv \xob \aEv$, we deduce $\cEv \xeco \aEv$. Combining with $\bEv \xeco \cEv$, we deduce that $\bEv \xeco \aEv$.  


\end{itemize}
In either case, there is a contradiction $\aEv \xeco \bEv \xeco \aEv$.
\end{proof}
\end{lemma}


\begin{lemma}\label{obeco2}
$({\rob} \cup {\reco})^*$ is irrreflexive.
\end{lemma}
\begin{proof}
The simple case that $\rob; \reco$ is irreflexive is proved above.  The full proof by contradiction.  

Let $n \geq 1$ be the minimum such that:
\begin{align*} 
&\aEv^0_0 \xob \aEv^0_1 \xeco \bEv^0_0 \xob \bEv^0_1  \\
(\xeco \cap \xob) &  \   \aEv^1_0 \xob \aEv^1_1 \xeco \bEv^1_0 \xob \bEv^1_1 \\
(\xeco \cap \xob) & \ \ldots \\
& \ldots \bEv^n_1 \\
 (\xeco \cap \xob) & \  \aEv^0_0
\end{align*}
where  for all $i$, we have:
\[ \aEv^i_0 \xpox \aEv^i_1 (\xeco \cap \xpox) \bEv^i_0 \xpox \bEv^i_1\] and 
\[ \neg (\bEv^i_1 \xpox (\aEv^{(i+1) \mod n}_0 \]

For any $i$, if $\aEv^i_0 \not= \aEv^i_1$ or $\bEv^i_0 \xpox \bEv^i_1$, via lemma~\ref{extendob}, we deduce that $\aEv^i_0  \xob \bEv^i_1$, contradicting minimality of $n$.  

So, we can assume that $n \geq 1$ is such that:
\begin{align*} 
&\ \aEv^0 \xeco  \bEv^0 \\
(\xeco \cap \xob) &  \   \aEv^1  \xeco \bEv^1 \\
(\xeco \cap \xob) & \ \ldots \\
& \ldots \bEv^n \\
 (\xeco \cap \xob) & \ \aEv^0
\end{align*}
which is a contradiction since it is a cycle in $\xeco$ and since at least one of $\aEv^i ,\bEv^i$ is a write for all $i$. 
\end{proof}


% \section{Local Data Race Freedom and Sequential Consistency}
\section{LDRF-SC for \PwTmcaTITLE{}}
\label{sec:sc}

\todo{Remove this section?}

\begin{changed}
  In this appendix, we establish a \drfsc{} for \PwTmca{2}.  We prove an
  \emph{external} result, where the notion of \emph{data-race} is independent
  of the semantics itself.  Since every \PwTmca{2} is also a \PwTmca{1}, the
  result also applies there.  Our result is also \emph{local}.  Using
  \citeauthor{Dolan:2018:BDR:3192366.3192421}'s
  [\citeyear{Dolan:2018:BDR:3192366.3192421}] notion of \emph{Local Data Race
    Freedom (LDRF)}.

  We do not address \PwTc{}.  The internal \drfsc{} result for \cXI{}
  \cite{DBLP:phd/ethos/Batty15} does not rely on dependencies and thus
  applies to \PwTc{}.  In internal \drfsc{}, data-races are defined using the
  semantics of the language itself.  Using the notion of dependency defined
  here, it should be possible to prove an stronger external result for
  \cXI{}, similar to that of \cite{DBLP:conf/pldi/LahavVKHD17}---we leave
  this as future work.

  \citet{DBLP:journals/pacmpl/JagadeesanJR20} prove \ldrfsc{} for Pomsets
  with Preconditions (\PwP{}).  \PwTmca{} generalizes \PwP{} to account for
  sequential composition.  Most of the machinery of \ldrfsc{}, however, has
  little to do with sequential semantics.  Thus, we have borrowed heavily
  from the text of \cite{DBLP:journals/pacmpl/JagadeesanJR20}; indeed, we
  have copied directly from the \LaTeX{} source, which is publicly available.
  We indicate substantial changes or additions using a change-bar on the
  right.

  There are several changes:
  \begin{itemize}
  \item \PwP{} imposes several conditions that we have dropped:
    \emph{consistency}, \emph{causal strengthening}, \emph{downset closure}
    (see \textsection\ref{sec:diff}).
  \item \PwP{} allows preconditions that are stronger than the weakest precondition.
  \item \PwP{} imposes \ref{pom-rf-le} ($\rrfx$ implies $\lt$) and thus is
    similar to \PwTmca{1}.  \PwTmca{2} is a weaker model that is new to this
    paper.  %Hence we cannot use the \PwP{} result directly.
  \item \PwP{} did not provide an accurate account of program order for
    merged actions.  We use \reflem{lem:po} to correct this deficiency.
  \end{itemize}
  The first two items require us to define $\semmin{}$ differently, below.
\end{changed}

The result requires that locations are properly initialized.  We assume a
sufficient condition: that programs have the form
``$\aLoc_1\GETS\aVal_1\SEMI \cdots \aLoc_n\GETS\aVal_n\SEMI\aCmd$'' where
every location mentioned in $\aCmd$ is some $\aLoc_i$.  To simplify the
definition of \emph{happens-before}, we ban fences and \RMWs.

To state the theorem, we require several technical definitions.  The reader
unfamiliar with \citep{Dolan:2018:BDR:3192366.3192421} may prefer to skip to
the examples in the proof sketch, referring back as needed.

\bookmark{Definitions}
\begin{changed}
  \paragraph{Program Order}
  Let $\sempomca{2}{}$ be defined by applying the construction of
  \reflem{lem:po} to $\semmca{2}{}$.  We consider only \emph{complete}
  pomsets.  For these, we derive program order on compound events as follows.
  By \reflem{lem:compound:phantom:exists}, if there is a compound event
  $\aEv$, then there is a phantom event $\cEv\in\fmrginv{\aEv}$ such that
  $\labelingForm(\cEv)$ is a tautology.  If there is exactly one tautology,
  we identify $\aEv$ with $\cEv$ in program order.  If there is more than one
  tautology, \reflem{lem:sequential:simple}, below, shows that it suffices to
  pick an arbitrary one---we identify $\aEv$ with the $\cEv\in\fmrginv{\aEv}$
  that is minimal in program order.
\end{changed}
For example, consider \jmm{} causality test case 2, with an added write to $z$:
\begin{gather*}
  \tag{$\ddagger\ddagger$}\label{eq6}
  \begin{gathered}
    \PR{x}{r}\SEMI
    \PW{z}{1}\SEMI
    \PR{x}{s}\SEMI
    \IF{r{=}s}\THEN \PW{y}{1}\FI
    \PAR
    \PW{x}{y}
    \\
    \hbox{\begin{tikzinline}[node distance=1em and 1.5em]
        \phevent{p1}{\DR{x}{1}}{}
        \event{z}{\DW{z}{1}}{right=of p1}
        \phevent{p2}{\DR{x}{0}}{right=of z}
        \event{a3}{\DW{y}{1}}{right=of p2}
        \event{b1}{\DR{y}{1}}{right=3em of a3}
        \event{b2}{\DW{x}{1}}{right=of b1}
        \rf{a3}{b1}
        \po{b1}{b2}
        \pox[out=-15,in=-170]{p1}{z}
        \pox[out=-15,in=-170]{z}{p2}
        \pox[out=-15,in=-170]{p2}{a3}
        \pox[out=-15,in=-170]{b1}{b2}
        \event{a1}{\DR{x}{1}}{above=of z}
        \mrg{p1}[pos=.3]{a1}
        \mrg{p2}[right]{a1}      
        \rf{b2}{a1}
        \pox{a1}{z}
      \end{tikzinline}}
  \end{gathered}
\end{gather*}
% For TC2, we have: 
% \begin{gather*}
%   \PR{x}{r}\SEMI
%   \PR{x}{s}\SEMI
%   \IF{r{=}s}\THEN \PW{y}{1}\FI
%   \PAR
%   \PW{x}{y}
%   \\
%   \hbox{\begin{tikzinline}[node distance=1.5em]
%       \event{a1}{\DR{x}{1}}{}
%       % \event{a2}{\DR{x}{1}}{right=of a1}
%       \event{a3}{\DW{y}{1}}{right=of a1}
%       % \po{a1}{a3}
%       % \po[out=-20,in=-160]{a1}{a3}
%       \event{b1}{\DR{y}{1}}{right=3em of a3}
%       \event{b2}{\DW{x}{1}}{right=of b1}
%       \rf{a3}{b1}
%       \po{b1}{b2}
%       % \rf[out=169,in=11]{b2}{a2}
%       \rf[out=169,in=11]{b2}{a1}
%       \phevent{p2}{\DR{x}{1}}{below=of a1}
%       \phevent{p1}{\DR{x}{1}}{left=of p2}
%       \mrg{p1}[pos=.3]{a1}
%       \mrg{p2}[right]{a1}      
%       \pox[out=-15,in=-170]{b1}[below]{b2}
%       \pox[out=-15,in=-170]{p1}[below]{p2}
%       \pox{p2}[below]{a3}
%     \end{tikzinline}}
%   \\
%   \hbox{\begin{tikzinline}[node distance=1.5em]
%       \event{a1}{\DR{x}{0}}{}
%       % \event{a2}{\DR{x}{1}}{right=of a1}
%       \event{a3}{\DW{y}{1}}{right=of a1}
%       % \po{a1}{a3}
%       % \po[out=-20,in=-160]{a1}{a3}
%       \event{b1}{\DR{y}{1}}{right=3em of a3}
%       \event{b2}{\DW{x}{1}}{right=of b1}
%       \rf{a3}{b1}
%       \po{b1}{b2}
%       % \rf[out=169,in=11]{b2}{a2}
%       %\rf[out=169,in=11]{b2}{a1}
%       \phevent{p2}{\DR{x}{0}}{below=of a1}
%       \phevent{p1}{\DR{x}{0}}{left=of p2}
%       \mrg{p1}[pos=.3]{a1}
%       \mrg{p2}[right]{a1}      
%       \pox[out=-15,in=-170]{b1}[below]{b2}
%       \pox[out=-15,in=-170]{p1}[below]{p2}
%       \pox{p2}[below]{a3}
%     \end{tikzinline}}
% \end{gather*}




\paragraph{Data Race}
Data races are defined using \emph{program} order $(\rpox)$, not
\emph{pomset} order $(\lt)$. %, and thus is stable with respect to augmentation.


Because we ban fences and \RMWs, we can adopt the simplest definition of
\emph{synchronizes\hyp{}with}~($\rsw$): Let $\bEv\xsw\aEv$ exactly when
$\bEv$ fulfills $\aEv$, $\bEv$ is a release, $\aEv$ is an acquire, and
$\lnot(\bEv\xpox\aEv)$.

Let ${\rhb}=({\rpox}\cup{\rsw})^+$ be the \emph{happens-before} relation.

% \begin{changed}
%   Note that ${\rhb}$ is a preorder in general. It is a partial order on
%   \emph{simple} events (\refdef{def:po}).
% \end{changed}

Let $L\subseteq\Loc$ be a set of locations.  We say that $\bEv$ \emph{has an
  $L$-race with} $\aEv$ (notation $\bEv\lrace{L}\aEv$) when
\begin{enumerate*}
\item at least one is relaxed, 
\item at least one is a write,
\item they access the same location in $L$, and
\item they are unordered by $\rhb$: neither $\bEv\xhb\aEv$ nor
  $\aEv\xhb\bEv$.
\end{enumerate*}

\begin{changed}
  \paragraph{Generators}
  We say that $\aPS'\in\downclose{\aPSS}$ if there is some $\aPS\in\aPSS$ such that
  $\aPS$ is \emph{complete} (\refdef{def:pwt:mca:complete}) and $\aPS'$ is a
  \emph{downset} of $\aPS$ (\refdef{def:downset}).
  
  % We say that $\aPS'$ \emph{generates} $\aPS$ if
  % $\aPS$ augments $\aPS'$ (\refdef{def:augment}).

  Let $\aPS$ be \emph{augmentation-minimal} in $\aPSS$ if $\aPS\in\aPSS$ and
  there is no $\aPS{\neq}\aPS'{\in}\aPSS$ such that $\aPS$ augments $\aPS'$.

  Let $\semmin{\aCmd}=\{\aPS\in\downclose{}\sempomca{2}{\aCmd} \mid \aPS$ is
  augmentation-minimal in $\downclose{}\sempomca{2}{\aCmd}\}$.
\end{changed}

\paragraph{Extensions}

We say that $\aPS'$ \emph{$\aCmd$-extends} $\aPS$ if %$\aPS\in\semmin{\aCmd}$,
$\aPS\neq\aPS'\in\semmin{\aCmd}$ and $\aPS$ is a downset of $\aPS'$.

\paragraph{Similarity}
We say that \emph{$\aPS'$ is $\aEv$-similar to $\aPS$} if they differ at most
in
\begin{enumerate*}
\item pomset order adjacent to $\aEv$, 
\item the value associated with event $\aEv$, if it is a read, and
\item the addition and removal of read events $\rpox$-after $\aEv$.
\end{enumerate*}
% We say they are \emph{similar} if they are $\aEv$-similar for some $\aEv$.
% Formally: $\Event'=\Event$, $\labelingForm'=\labelingForm$,
% $\restrict {\le'}{\Event\setminus\{\aEv\}}=\restrict {\le}{\Event\setminus\{\aEv\}}$,
% if $\aEv$ is not a read then $\labelingAct'=\labelingAct$, and if $\aEv$ is a
% read then
% $\restrict{\labelingAct'}{\Event\setminus\{\aEv\}}=\restrict{\labelingAct}{\Event\setminus\{\aEv\}}$
% and $\labelingAct'(\aEv) = \labelingAct(\aEv)[\aVal'/\aVal]$, for some
% $\aVal'$, $\aVal$.

\paragraph{Stability}
We say that $\aPS$ is \emph{$L$-stable in $\aCmd$} if
\begin{enumerate*}  
\item $\aPS\in\semmin{\aCmd}$, 
\item $\aPS$ is $\rpox$-convex (nothing missing in program order),
\item there is no $\aCmd$-extension of $\aPS$ with a \emph{crossing}
  $L$-race: that is, there is no $\bEv\in\Event$, no $\aPS'$
  $\aCmd$-extending $\aPS$, and no $\aEv\in\Event'\setminus\Event$ such that
  $\bEv\lrace{L}\aEv$.
\end{enumerate*}
The empty pomset is $L$-stable.

\paragraph{Sequentiality}
Let ${\pole{L}}={\lt_L}\cup{\rpox}$, where $\lt_L$ is the restriction of $\lt$ to events that access locations in $L$.
We say that $\aPS'$ is \emph{$L$-sequential after $\aPS$} if 
\begin{enumerate*}
\item $\aPS'$ is $\rpox$-convex,
\item $\pole{L}$ is acyclic in $\Event'\setminus\Event$.
\end{enumerate*}

\begin{changed}
  \paragraph{Simplicity}
  We say that $\aPS'$ is \emph{$L$-simple after $\aPS$} if all of the events
  in $\Event'\setminus\Event$ that access locations in $L$ are \emph{simple}
  (\refdef{def:po}).

  \begin{lemma}
    \label{lem:sequential:simple}
    Suppose $\aPS'\in\semmin{\aCmd}$ and $\aPS$ is {$L$-sequential after
      $\aPS$}.
    Let $\aPS''$ be the restriction of $\aPS'$ that is {$L$-simple after
      $\aPS$} (throwing out compound $L$-events after $\aPS$).
    Then $\aPS''\in\semmin{\aCmd}$.
    % projection of $\aPS$ is a $\aPS'\in\semmin{\aCmd}$ such that
    % \begin{enumerate*}
    % \item $\aPS'$ is $L$-sequential and $L$-simple after $\aPS$, and
    % \item  $\aPS'$ is  derived from  $\aPS$  by removing  every compound  event
    %   $\aEv$ that  accesses a location in  $L$, thus changing the remaining
    %   events in $\fmrginvp{\aEv}$ from phantom events to real ones.
    % \end{enumerate*}
  \end{lemma}
  As a negative example, note that \eqref{eq6} is not $L$-sequential---in
  fact there is no execution of the program that results in the simple events
  of \eqref{eq6}: without merging the reads, there would be a dependency
  $\DRP{x}{1} \xpo \DWP{y}{1}$.  $L$-sequential executions of this code must read
  $0$ for $x$:
  \begin{gather*}
    \PR{x}{r}\SEMI
    \PW{z}{1}\SEMI
    \PR{x}{s}\SEMI
    \IF{r{=}s}\THEN \PW{y}{1}\FI
    \PAR
    \PW{x}{y}
    \\
    \hbox{\begin{tikzinline}[node distance=1em and 1.5em]
        \phevent{p1}{\DR{x}{0}}{}
        \event{z}{\DW{z}{1}}{right=of p1}
        \phevent{p2}{\DR{x}{0}}{right=of z}
        \event{a3}{\DW{y}{1}}{right=of p2}
        \event{b1}{\DR{y}{1}}{right=3em of a3}
        \event{b2}{\DW{x}{1}}{right=of b1}
        \rf{a3}{b1}
        \po{b1}{b2}
        \pox[out=-15,in=-170]{p1}{z}
        \pox[out=-15,in=-170]{z}{p2}
        \pox[out=-15,in=-170]{p2}{a3}
        \pox[out=-15,in=-170]{b1}{b2}
        \event{a1}{\DR{x}{0}}{above=of z}
        \mrg{p1}[pos=.3]{a1}
        \mrg{p2}[right]{a1}      
        \pox{a1}{z}
      \end{tikzinline}}
  \end{gather*}
\end{changed}

\bookmark{Theorem and Proof Sketch}
\begin{theorem}
  Let $\aPS$ be $L$-stable in $\aCmd$.  Let $\aPS'$ be a $\aCmd$-extension of
  $\aPS$ that is $L$-sequential after $\aPS$.  Let $\aPS''$ be a
  $\aCmd$-extension of $\aPS'$ that is $\rpox$-convex, such that no subset of
  $\Event''$ satisfies these criteria.
  Then either (1) $\aPS''$ is $L$-sequential
  \begin{changed}
    and $L$-simple
  \end{changed}
  after $\aPS$ or (2) there is some $\aCmd$-extension $\aPS'''$ of $\aPS'$
  and some $\aEv\in(\Event''\setminus\Event')$ such that (a) $\aPS'''$ is
  $\aEv$-similar to $\aPS''$, (b) $\aPS'''$ is $L$-sequential
  \begin{changed}
    and $L$-simple
  \end{changed}
  after $\aPS$, and (c) $\bEv\lrace{L}\aEv$, for some
  $\bEv\in(\Event''\setminus\Event)$.
\end{theorem}
The theorem provides an inductive characterization of \emph{Sequential
  Consistency for Local
  Data-Race Freedom (SC-LDRF)}: Any extension of a $L$-stable pomset is either
$L$-sequential, or is $\aEv$-similar to a $L$-sequential extension that
includes a race involving $\aEv$.
\begin{proof}[Proof Sketch]
  \begin{changed}
    We show $L$-sequentiality.  $L$-simplicity then follows from
    \reflem{lem:sequential:simple}.
  \end{changed}

  In order to develop a technique to find $\aPS'''$ from $\aPS''$, we analyze
  pomset order in generation-minimal top-level pomsets.  First, we note that
  $\lt_*$ (the transitive reduction $\lt$) can be decomposed into three
  disjoint relations.  Let ${\rppo}=({\lt_*}\cap{\rpox})$ denote
  \emph{preserved} program order, as required by sequential composition and
  conditional.  The other two relations are cross-thread subsets of
  $({\lt_*}\setminus{\rpox})$: $\rrfe$ (reads-from-external) orders writes
  before reads, satisfying \ref{par-rf-le}; $\rcae$
  (coherence-after-external) orders read and write accesses before writes,
  satisfying \ref{pom-rf-block}. (Within a thread, \ref{seq-le-mca2} induces
  order that is included in ${\rppo}$.)

  Using this decomposition, we can show the following.
  \begin{lemma}
    Suppose $\aPS''\in\semmin{\aCmd}$ has an external read $\bEv\xrfxpp\aEv$ that is
    maximal in $({\rppo}\cup{\rrfe})$.  Further suppose that there another
    write $\bEv'$ that could fulfill $\aEv$.
    Then there exists an $\aEv$-similar $\aPS'''$ with $\bEv'\xrfxppp\aEv$
    such that $\aPS'''\in\semmin{\aCmd}$.
    % and such that every $\rpox$-following read is also $\le$-following
    % ($\aEv\xpox\bEv$ implies $\aEv\le\bEv$, for every read $\bEv$).
    % Further, suppose there is an $\aEv$-similar $\aPS'''$
    % that satisfies the requirements of fulfillment.  Then
    % $\aPS'''\in\semmin{\aCmd}$.
  \end{lemma}
  The proof of the lemma follows an inductive construction of
  $\semmin{\aCmd}$, starting from a large set with little order, and
  pruning the set as order is added: We begin with all pomsets generated by
  the semantics without imposing the requirements of fulfillment (including
  only $\rppo$).  We then prune reads which cannot be fulfilled, starting
  with those that are minimally ordered.
  % This proof is simplified by
  % precluding local declarations.

  We can prove a similar result for $({\rpox}\cup{\rrfe})$-maximal read
  and write accesses.

  Turning to the proof of the theorem, if $\aPS''$ is $L$-sequential after
  $\aPS$, then the result follows from (1).  Otherwise, there must be a
  $\pole{L}$ cycle in $\aPS''$ involving all of the actions in
  $(\Event''\setminus\Event')$: If there were no such cycle, then $\aPS''$
  would be $L$-sequential; if there were elements outside the cycle, then
  there would be a subset of $\Event''$ that satisfies these criteria.

  If there is a $({\rpox}\cup{\rrfe})$-maximal access, we select one of
  these as $\aEv$.  If $\aEv$ is a write, we reverse the outgoing order in
  $\rcae$; the ability to reverse this order witnesses the race.  If $\aEv$
  is a read, we switch its fulfilling write to a ``newer'' one, updating
  $\rcae$; the ability to switch witnesses the race.  For
  example, for $\aPS''$ on the left below, we choose the $\aPS'''$ on the
  right;  $\aEv$ is the read of $x$, which races with $(\DW{x}{1})$.  % Program order
  \begin{gather*}
    x\GETS 0 \SEMI y\GETS 0 \SEMI  (x \GETS 1  \SEMI y \GETS 1
    \PAR
    \IF{y}\THEN \aReg \GETS x \FI)
    \\[-.5ex]
    \hbox{\begin{tikzinline}[node distance=1.5em and 2em]
        \event{wy0}{\DW{y}{0}}{}
        \event{wx0}{\DW{x}{0}}{below=of wy0}
        \event{wx1}{\DW{x}{1}}{right=3em of wy0}
        \event{wy1}{\DW{y}{1}}{right=of wx1}
        \event{ry1}{\DR{y}{1}}{below=of wx1}
        \event{rx}{\DR{x}{0}}{below=of wy1}
        \rf[bend right]{wx0}{rx}
        \rf{wy1}{ry1}
        \wk[bend left]{wy0}{wy1}
        \pox{wx1}{wy1}
        \pox{ry1}[below]{rx}
        \wk{rx}{wx1}
        \node(ix)[left=of wx0]{};
        \node(iy)[left=of wy0]{};
        \bgoval[yellow!50]{(ix)(iy)}{P}
        \bgoval[pink!50]{(wx0)(wy0)}{P'\setminus P}
        \bgoval[green!10]{(ry1)(wx1)(rx)(wy1)}{P''\setminus P'}
        \pox{wx0}{wy0}
        \pox{wy0}{wx1}
        \pox{wy0}[below]{ry1}
      \end{tikzinline}}
    \qquad
    \hbox{\begin{tikzinline}[node distance=1.5em and 2em]
        \event{wy0}{\DW{y}{0}}{}
        \event{wx0}{\DW{x}{0}}{below=of wy0}
        \event{wx1}{\DW{x}{1}}{right=3em of wy0}
        \event{wy1}{\DW{y}{1}}{right=of wx1}
        \event{ry1}{\DR{y}{1}}{below=of wx1}
        \event{rx}{\DR{x}{1}}{below=of wy1}
        \rf{wx1}{rx}
        \rf{wy1}{ry1}
        \wk[bend left]{wy0}{wy1}
        \pox{wx1}{wy1}
        \pox{ry1}[below]{rx}
        \wk{wx0}{wx1}
        \node(ix)[left=of wx0]{};
        \node(iy)[left=of wy0]{};
        \bgoval[yellow!50]{(ix)(iy)}{P}
        \bgoval[pink!50]{(wx0)(wy0)}{P'\setminus P}
        \bgoval[green!10]{(ry1)(wx1)(rx)(wy1)}{P'''\setminus P'}
        \pox{wx0}{wy0}
        \pox{wy0}{wx1}
        \pox{wy0}[below]{ry1}
      \end{tikzinline}}
  \end{gather*}
  It is important that $\aEv$ be $({\rpox}\cup{\rrfe})$-maximal, not just
  $({\rppo}\cup{\rrfe})$-maximal.  The latter criterion would allow us to
  choose $\aEv$ to be the read of $y$, but then there would be no
  $\aEv$-similar pomset: if an execution reads $0$ for $y$ then there is no
  read of $x$, due to the conditional.

  \begin{changed}
    In the above argument, it is unimportant whether $\aEv$ reads-from an
    internal or an external write; thus the argument applies to \PwTmca{2}
    and \PwTmca{1} as it does for \PwTmca{1}.
  \end{changed}
  
  If there is no $({\rpox}\cup{\rrfe})$-maximal access, then all cross-thread
  order must be from $\rrfe$.  In this case, we select a
  $({\rppo}\cup{\rrfe})$-maximal read, switching its fulfilling write to an
  ``older'' one.  If there are several of these, we choose one that is
  $\rpox$-minimal.  As an example, consider the following; once again, $\aEv$
  is the read of $x$, which races with $(\DW{x}{1})$.
  \begin{gather*}
    x\GETS 0 \SEMI y\GETS 0 \SEMI (\aReg \GETS x  \SEMI y \GETS 1
    \PAR
    \bReg \GETS y \SEMI x \GETS \bReg)
    \\[-.5ex]
    \hbox{\begin{tikzinline}[node distance=1.5em and 2em]
        \event{wx0}{\DW{y}{0}}{}
        \event{ry}{\DR{x}{1}}{right=3em of wx0}
        \event{wx1}{\DW{y}{1}}{right=of ry}
        \event{wy0}{\DW{x}{0}}{below=of wx0}
        \event{rx1}{\DR{y}{1}}{right=3em of wy0}
        \event{wy1}{\DW{x}{1}}{right=of rx1}
        \rf{wx1}{rx1}
        \rf{wy1}{ry}
        \po{rx1}{wy1}
        \pox{ry}{wx1}
        \wk[bend left]{wx0}{wx1}
        \wk[bend right]{wy0}{wy1}
        \node(ix)[left=of wx0]{};
        \node(iy)[left=of wy0]{};
        \bgoval[yellow!50]{(ix)(iy)}{P}
        \bgoval[pink!50]{(wx0)(wy0)}{P'\setminus P}
        \bgoval[green!10]{(ry)(wx1)(rx1)(wy1)}{P''\setminus P'}
        \pox{wy0}{wx0}
        \pox{wx0}{ry}
        \pox{wx0}[below]{rx1}
      \end{tikzinline}}
    \qquad
    \hbox{\begin{tikzinline}[node distance=1.5em and 2em]
        \event{wx0}{\DW{y}{0}}{}
        \event{ry}{\DR{x}{0}}{right=3em of wx0}
        \event{wx1}{\DW{y}{1}}{right=of ry}
        \event{wy0}{\DW{x}{0}}{below=of wx0}
        \event{rx1}{\DR{y}{1}}{right=3em of wy0}
        \event{wy1}{\DW{x}{1}}{right=of rx1}
        \pox{ry}{wx1}
        \wk[bend left]{wx0}{wx1}
        \rf{wx1}{rx1}
        \rf{wy0}{ry}
        \po{rx1}{wy1}
        \wk{ry}{wy1}
        \node(ix)[left=of wx0]{};
        \node(iy)[left=of wy0]{};
        \bgoval[yellow!50]{(ix)(iy)}{P}
        \bgoval[pink!50]{(wx0)(wy0)}{P'\setminus P}
        \bgoval[green!10]{(ry)(wx1)(rx1)(wy1)}{P'''\setminus P'}
        \pox{wy0}{wx0}
        \pox{wx0}{ry}
        \pox{wx0}[below]{rx1}
      \end{tikzinline}}
  \end{gather*}
  This example requires $(\DW{x}{0})$.  Proper initialization ensures the
  existence of such ``older'' writes.
\end{proof}



\section{\PwTmcaTITLE{}: Additional Examples}
\label{sec:extras}

This appendix includes additional examples.  They all apply equally to
\PwTmca{1} and \PwTmca{2}.  Several of these are taken directly from \cite{DBLP:journals/pacmpl/JagadeesanJR20}.
% \subsection{Arm}
% The following execution is allowed by Arm.
% \begin{gather*}
%   {
%     \PW{x}{1}
%     \SEMI
%     \PW[\mREL]{y}{1}
%   }\PAR{
%     \PR{y}{r}
%     \SEMI
%     \PW{y}{2}
%     \SEMI
%     \PR[\mACQ]{y}{s}
%      \SEMI
%     \PR{x}{t}
%   }
%   \\
%   \hbox{\begin{tikzinline}[node distance=1.5em]
%       \event{a}{\DW{x}{1}}{}
%       \raevent{b}{\DW[\mREL]{y}{1}}{right=of a}
%       \event{c}{\DR{y}{1}}{right=3em of b}
%       \event{d}{\DW{y}{2}}{right=of c}
%       \raevent{e}{\DR[\mACQ]{y}{2}}{right=of d}
%       \event{f}{\DR{x}{0}}{right=of e}
%       \lob{a}{b}
%       \rfx{b}{c}
%       %\sync[out=15,in=165]{c}{e}
%       \lob{c}{d}
%       \rfx{d}{e}
%       \lob{e}{f}
%       \fr[out=-165,in=-15]{f}[above,pos=.45]{a}
%       %\close[out=-15,in=-165]{b}{e}
%     \end{tikzinline}}
%   \\
%   \tag{$\rgcb$}
%   \hbox{\begin{tikzinline}[node distance=1.5em]
%       \event{a}{\DW{x}{1}}{}
%       \raevent{b}{\DW[\mREL]{y}{1}}{right=of a}
%       \event{c}{\DR{y}{1}}{right=3em of b}
%       \event{d}{\DW{y}{2}}{right=of c}
%       \raevent{e}{\DR[\mACQ]{y}{2}}{right=of d}
%       \event{f}{\DR{x}{0}}{right=of e}
%       \gcbz{a}{b}
%       \gcbz{b}{c}
%       \gcbz{c}{d}
%       \gcbz{d}{e}
%       %\gcbz{e}{f}
%       \gcbz[out=-165,in=-15]{f}{a}
%     \end{tikzinline}}
%   \\
%   \tag{$\rcb$}
%   \hbox{\begin{tikzinline}[node distance=1.5em]
%       \event{a}{\DW{x}{1}}{}
%       \raevent{b}{\DW[\mREL]{y}{1}}{right=of a}
%       \event{c}{\DR{y}{1}}{right=3em of b}
%       \event{d}{\DW{y}{2}}{right=of c}
%       \raevent{e}{\DR[\mACQ]{y}{2}}{right=of d}
%       \event{f}{\DR{x}{0}}{right=of e}
%       \cbz{a}{b}
%       \cbz{b}{c}
%       \cbz{c}{d}
%       %\cbz{d}{e}
%       \cbz{e}{f}
%       \cbz[out=-165,in=-15]{f}{a}
%     \end{tikzinline}}
% \end{gather*}

\subsection{Coherence}

The following execution is disallowed by fulfillment (\ref{pom-rf-match} and
\ref{pom-rf-block}).
\begin{gather*}
  \tag{\textsc{coh}}
  \begin{gathered}
    \PW{x}{1}\SEMI
    \PR{x}{r}
    \PAR
    \PW{x}{2}\SEMI
    \PR{x}{s}
    \\\nonumber
    \hbox{\begin{tikzinline}[node distance=1.5em]
        \event{a1}{\DW{x}{1}}{}
        \event{a2}{\DR{x}{2}}{right=of a1}
        \event{b1}{\DW{x}{2}}{right=3em of a2}
        \event{b2}{\DR{x}{1}}{right=of b1}
        \wki{a1}{a2}
        \wki{b1}{b2}
        \rf{b1}{a2}
        \rf[out=20,in=160]{a1}{b2}
        \wk[out=15,in=155]{a1}{b1}
        %\wk[out=-155,in=-15]{b1}{a1}
      \end{tikzinline}}
  \end{gathered}
\end{gather*}
\ref{pom-rf-block} requires that we order one write with respect to the
other, either before the write or after the read (and therefore after the
write).  Suppose we pick $1$ before $2$, as shown.  This satisfies
\ref{pom-rf-block} for $\DRP{x}{2}$.  But to satisfy the requirement for
$\DRP{x}{1}$ we must have either $\DWP{x}{2}\le\DWP{x}{1}$ or
$\DRP{x}{1}\le\DWP{x}{2}$.   Either way, we have a cycle.

Our model is more coherent than Java, which permits the following:
\begin{gather*}
  \taglabel{TC16}
  \begin{gathered}
    \PR{x}{r}\SEMI \PW{x}{1}
    \PAR
    \PR{x}{s}\SEMI \PW{x}{2}
    \\[-1ex]
    \hbox{\begin{tikzinline}[node distance=1.5em]
        \event{a1}{\DR{x}{2}}{}
        \event{a2}{\DW{x}{1}}{right=of a1}
        \wki{a1}{a2}
        \event{b1}{\DR{x}{1}}{right=3em of a2}
        \event{b2}{\DW{x}{2}}{right=of b1}
        \wki{b1}{b2}
        \rf{a2}{b1}
        \rf[out=-165,in=-15]{b2}{a1}
      \end{tikzinline}}
  \end{gathered}
\end{gather*}
We also forbid the following, which Java allows:
\begin{gather*}
  \taglabel{Co3}
  \begin{gathered}
    \PW{x}{1}\SEMI \PW[\mRA]{y}{1}
    \PAR
    \PW{x}{2}\SEMI \PW[\mRA]{z}{1}
    \PAR
    \PR[\mRA]{z}{r} \SEMI 
    \PR[\mRA]{y}{r} \SEMI 
    \PR{x}{r} \SEMI 
    \PR{x}{r}
    \\[-1ex]
    \hbox{\begin{tikzinline}[node distance=1.5em]
        \event{a1}{\DW{x}{1}}{}
        \event{a2}{\DW[\mRA]{y}{1}}{right=of a1}
        \sync{a1}{a2}
        \event{b1}{\DW{x}{2}}{right=3em of a2}
        \event{b2}{\DW[\mRA]{\,z}{1}}{right=of b1}
        \sync{b1}{b2}
        \event{c1}{\DR[\mRA]{\,z}{1}}{right=3em of b2}
        \event{c2}{\DR[\mRA]{y}{1}}{right=of c1}
        \event{c3}{\DR{x}{2}}{right=of c2}
        \event{c4}{\DR{x}{1}}{right=of c3}
        \sync{c1}{c2}
        \sync{c2}{c3}
        \sync[out=20,in=160]{c2}{c4}
        \rf[out=8,in=172]{a2}{c2}
        \rf{b2}{c1}
        \wk[out=19,in=161]{a1}{b1}
        \wk[out=-172,in=-8]{c4}{b1}
      \end{tikzinline}}
  \end{gathered}
\end{gather*}


The following outcome is allowed by the promising semantics
\cite{DBLP:conf/popl/KangHLVD17}, but not in \weakestmo{}
\cite[Fig.~3]{DBLP:journals/pacmpl/ChakrabortyV19} nor in our semantics, due
to the cycle:
\begin{gather*}
  \tag{\textsc{coh-cyc}}
  \begin{gathered}
    x\GETS 2\SEMI
    \IF{x\NOTEQ2}\THEN y\GETS 1 \FI
    \PAR
    x\GETS 1\SEMI
    r\GETS x\SEMI
    \IF{y}\THEN x\GETS 3 \FI
    \\\nonumber
    \hbox{\begin{tikzinline}[node distance=1.5em]
        \event{wx2}{\DW{x}{2}}{}
        \event{rx3}{\DR{x}{3}}{right=of wx2}
        \wki{wx2}{rx3}
        \event{wy1}{\DW{y}{1}}{right=of rx3}
        \po{rx3}{wy1}
        \event{wx1}{\DW{x}{1}}{right=2em of wy1}
        \event{rx2}{\DR{x}{2}}{right=of wx1}
        \wki{wx1}{rx2}
        \event{ry1}{\DR{y}{1}}{right=of rx2}
        \event{wx3}{\DW{x}{3}}{right=of ry1}
        \po{ry1}{wx3}
        \wki[in=165,out=15]{rx2}{wx3}
        \rf[in=-170,out=-10]{wy1}{ry1}
        \rf[in=170,out=10]{wx2}{rx2}
        \rf[out=-170,in=-10]{wx3}{rx3}
        \wk[out=-170,in=-10]{wx1}{wx2}
      \end{tikzinline}}
  \end{gathered}
\end{gather*}

Since reads are not ordered by intra-thread coherence,
we {allow} the following unintuitive behavior. C11 includes read-read
coherence between relaxed atomics in order to forbid this:
\begin{gather*}
  \taglabel{Co2}
  \begin{gathered}
    \PW{x}{1}\SEMI \PW{x}{2}
    \PAR
    \PW{y}{x} \SEMI \PW{z}{x}
    \\[-1ex]
    \hbox{\begin{tikzinline}[node distance=1.5em]
        \event{a}{\DW{x}{1}}{}
        \event{b}{\DW{x}{2}}{right=of a}
        \wki{a}{b}
        \event{c}{\DR{x}{2}}{right=3em of b}
        \event{d}{\DW{y}{2}}{right=of c}
        \po{c}{d}
        \event{e}{\DR{x}{1}}{right=of d}
        \event{f}{\DW{z}{1}}{right=of e}
        \po{e}{f}
        \rf{b}{c}
        \rf[out=10,in=170]{a}{e}
        \wk[out=-165,in=-15]{e}{b}
      \end{tikzinline}}
  \end{gathered}
\end{gather*}
Here, the reader sees $2$ then $1$, although they are written in the reverse
order.
This behavior is allowed by Java in order to validate CSE without requiring
aliasing analysis.

\subsection{RMWs}
It is not possible for two \RMW{}s to see the same write.
\begin{gather*}
  \begin{gathered}
    \PW{x}{0} \SEMI \bigl(\PFADD[\mRLX][\mRLX]{x}{}{1} \PAR \PFADD[\mRLX][\mRLX]{x}{}{1}\bigr)
    \\
    \hbox{\begin{tikzinline}[node distance=2em]
        \event{a0}{\DW{x}{0}}{}
        \event{a1}{\DR{x}{0}}{right=3em of a0}
        \event{a2}{\DW{x}{1}}{right=of a1}
        \event{b1}{\DR{x}{0}}{right=3em of a2}
        \event{b2}{\DW{x}{1}}{right=of b1}
        \rmw{a1}{a2}
        \rf{a0}{a1}
        \rf[out=-15,in=-165]{a0}{b1}
        \wk[out=-15,in=-165]{a1}{b2}
        \wk{b1}{a2}
        \graywk[bend left]{a2}{b1}
        \rmw{b1}{b2}
      \end{tikzinline}}
  \end{gathered}
  \taglabel{rmw0}
\end{gather*}
The gray arrow is required the \RMW{} atomicity axioms.

\citet{DBLP:conf/pldi/LeeCPCHLV20} introduce \PS{2.0} to refine the treatment of
\RMW{}s in the promising semantics (\PS{}).  Their examples have the expected
results here, with far less work.  First they recall that \PS{} requires
quantification over multiple futures in order to disallow executions such as
\ref{CDRF}:
\begin{gather*}
  \taglabel{CDRF}
    \begin{gathered}
      \PFADD[\mACQ][\mREL]{x}{r}{1}\SEMI \IF{r{=}0}\THEN \PW{y}{1} \FI
      \PAR
      \PFADD[\mACQ][\mREL]{x}{r}{1}\SEMI \IF{r{=}0}\THEN \IF{y}\THEN \PW{x}{0} \FI \FI
      \\
      \hbox{\begin{tikzinline}[node distance=2em]
          \event{a1}{\DR[\mACQ]{x}{0}}{}
          \event{a1b}{\DW[\mREL]{x}{1}}{below=1em of a1}
          \event{a2}{\DW{y}{1}}{right=of a1}
          \event{b0}{\DR[\mACQ]{x}{0}}{right=3em of a2}
          \event{b0b}{\DW[\mREL]{x}{1}}{below=1em of b0}
          \event{b1}{\DR{y}{1}}{right=of b0}
          \event{b2}{\DW{x}{0}}{right=of b1}
          \rmw{a1}{a1b}
          \rmw{b0}{b0b}
          \rf[out=-13,in=-163]{a2}{b1}
          \po{a1}{a2}
          \sync{b0}{b1}
          \po{b1}{b2}
          \rf[out=-165,in=-12]{b2}{a1}
        \end{tikzinline}}
    \end{gathered}
  \end{gather*}
This execution is clearly impossible, due to the cycle above.  In this
diagram, we have not drawn order adjacent to the writes of the \RMW{}s, since
this is not necessary to produce the cycle.
If \ref{CDRF} is allowed then \drfra{} fails.


  
\PS{} does not support global value range analysis, as modeled by \ref{GA+E} below.  Our
semantics permits \ref{GA+E}:
\begin{gather*}
  \taglabel{GA+E}
    \begin{gathered}
      \PW{x}{0} \SEMI
      \bigl(
        \PCAS[\mRLX][\mRLX]{x}{r}{0}{1}\SEMI \IF{r{<}10}\THEN \PW{y}{1} \FI
        \PAR
        \PW{x}{42}\SEMI \PW{x}{y}
      \bigr)
      \\
      \hbox{\begin{tikzinline}[node distance=2em]
          \event{a0}{\DW{x}{0}}{}
          \event{a1}{\DR{x}{1}}{right=3em of a0}
          \event{a2}{0{<}10\mid\DW{y}{1}}{right=of a1}
          \event{b0}{\DW{x}{42}}{right=3em of a2}
          \event{b1}{\DR{y}{1}}{right=of b0}
          \event{b2}{\DW{x}{1}}{right=of b1}
          %\rmw{a1}{a2}
          \rf[out=-15,in=-160]{a2}{b1}
          \po{b1}{b2}
          \rf[out=-165,in=-15]{b2}{a1}
          \wk[out=10,in=170]{a0}{b0}
          \wk[out=15,in=165]{b0}{b2}
        \end{tikzinline}}
    \end{gathered}
\end{gather*}
\PS{} also does not support register promotion, as modeled by \ref{RP} below.    Our
semantics permits \ref{RP}:
\begin{gather*}
  \taglabel{RP}
    \begin{gathered}
      \PR{x}{r}\SEMI
      \PFADD[\mRLX][\mRLX]{z}{s}{r}\SEMI \PW{y}{s{+}1}
      \PAR
      \PW{x}{y}
      \\
      \hbox{\begin{tikzinline}[node distance=2em]
          \event{a0}{\DR{x}{1}}{}
          \event{a1}{\DR{z}{0}}{right=of a0}
          \event{a1b}{\DW{z}{1}}{right=of a1}
          \event{a2}{\DW{y}{1}}{right=of a1b}
          \event{b0}{\DR{y}{1}}{right=3em of a2}
          \event{b1}{\DW{x}{1}}{right=of b0}
          \rmw{a1}{a1b}
          \po[out=20,in=160]{a0}{a1b}
          \po[out=20,in=160]{a1}{a2}
          \po{b0}{b1}
          \rf{a2}{b0}
          \rf[out=-165,in=-15]{b1}{a0}
        \end{tikzinline}}
    \end{gathered}
\end{gather*}



\begin{example}
  This definition ensures atomicity, disallowing executions such as
  \cite[Ex.~3.2]{DBLP:journals/pacmpl/PodkopaevLV19}:
  \begin{gather*}
    % \taglabel{RMW1}
    \begin{gathered}
      \PW{x}{0}\SEMI \PINC[\mRLX][\mRLX]{x}{}
      \PAR
      \PW{x}{2}\SEMI \PR{x}{r}
      % \\
      % \hbox{\begin{tikzinline}[node distance=1.5em]
      %     \event{a2}{\DR{x}{0}}{}
      %     \event{a1}{\DW{x}{0}}{left=of a2}
      %     \rf{a1}{a2}
      %     \event{a3}{\DW{x}{2}}{right=of a2}
      %     \wk{a2}{a3}
      %     \event{b2}{\DW{x}{1}}{right=of a3}
      %     \event{b3}{\DR{x}{1}}{right=of b2}
      %     \rmw[out=-15,in=-165]{a2}[below]{b2}
      %     \wk{a3}{b2}
      %     \rf{b2}{b3}
      %     \liftrmw[out=165,in=15]{a3}{a2}
      %   \end{tikzinline}}
      \\
      \hbox{\begin{tikzinline}[node distance=1.5em]
          \event{a2}{\DR{x}{0}}{}
          \event{a1}{\DW{x}{0}}{left=of a2}
          \rf{a1}{a2}
          \event{a3}{\DW{x}{1}}{right=of a2}
          \event{b2}{\DW{x}{2}}{right=3em of a3}
          \event{b3}{\DR{x}{1}}{right=of b2}
          \rmw{a2}[below]{a3}
          \wk{b2}{a3}
          \wk[out=-15,in=-165]{a2}{b2}
          \rf[out=-15,in=-165]{a3}{b3}
          \liftrmw[out=165,in=15]{b2}{a2}
        \end{tikzinline}}
    \end{gathered}
  \end{gather*}
  By \ref{pom-rmw-atomic1}, since $\DWP{x}{2}\xwk\DWP{x}{1}$, it must be that
  $\DWP{x}{2}\xwk\DRP{x}{0}$, creating a cycle.
\end{example}

\begin{example}
  \label{ex:rmw-33}
  Two successful \RMW{}s cannot see the same write:
  \begin{gather*}
    \begin{gathered}
      \PW{x}{0}\SEMI (\PINC[\mRLX][\mRLX]{x}{} \PAR \PINC[\mRLX][\mRLX]{x}{})
      \\
      \hbox{\begin{tikzinline}[node distance=1.5em]
          \event{i}{\DW{x}{0}}{}
          \event{a1}{a{:}\DR{x}{0}}{right=3em of i}
          \event{a2}{b{:}\DW{x}{1}}{right=of a1}
          \event{b1}{c{:}\DR{x}{0}}{right=3em of a2}
          \event{b2}{d{:}\DW{x}{1}}{right=of b1}
          \rmw{a1}{a2}
          \rmw{b1}{b2}
          \rf{i}{a1}
          \rf[out=15,in=165]{i}{b1}
          \wk[out=-15,in=-165]{a1}{b2}
          \liftrmw[out=-15,in=-165]{a2}{b1}
          % \wk{a1}{b2}
          \wk{b1}{a2}
        \end{tikzinline}}
    \end{gathered}
  \end{gather*}
  The order from read-to-write is required by fulfillment.  
  Apply \ref{pom-rmw-atomic1} of the second \RMW{} to $a\xwk d$, we have that $a\xwk c$.  Subsequently
  applying \ref{pom-rmw-atomic2} of the first \RMW{}, we have $b \xwk c$, creating a cycle.
\end{example}

\begin{example}
  By using two actions rather than one, the definition allows examples such as the
  following, which is allowed by \armeight{} 
  \cite[Ex.~3.10]{DBLP:journals/pacmpl/PodkopaevLV19}:
  \begin{gather*}
    % \taglabel{RMW2}
    \begin{gathered}
      \PR{z}{r}\SEMI
      % \PW{x}{0}\SEMI
      \PINC[\mRLX][\mREL]{x}{s} \SEMI
      \PW{y}{s}{+}1
      \PAR
      \PR{y}{r}\SEMI
      \PW{z}{r}
      \\[-1ex]
      \hbox{\begin{tikzinline}[node distance=1.5em]
          \event{b1}{\DR{z}{1}}{}
          % \event{b2}{\DW{x}{0}}{right=of b1}
          \event{b3}{\DR{x}{0}}{right=of b1}
          %\rf{b2}{b3}
          \event{b4}{\DWRel{x}{1}}{right=2em of b3}
          \rmw{b3}{b4}
          \event{b5}{\DW{y}{1}}{right=of b4}
          \sync[out=-20,in=-160]{b1}{b4}
          \po[out=-20,in=-160]{b3}{b5}
          \event{a1}{\DR{y}{1}}{right=3em of b5}
          \event{a2}{\DW{z}{1}}{right=of a1}
          \po{a1}{a2}
          \rf{b5}{a1}
          \rf[out=170,in=10]{a2}{b1}
        \end{tikzinline}}
    \end{gathered}
  \end{gather*}
  A similar example, also allowed by \armeight{}
  \cite[Fig.~6]{DBLP:journals/pacmpl/ChakrabortyV19}:
  \begin{gather*}
    % \taglabel{RMW2}
    \begin{gathered}
      \PR{z}{r}\SEMI
      % \PW{x}{0}\SEMI
      \PFADD[\mRLX][\mRLX]{x}{s}{r} \SEMI
      \PW{y}{s}{+}1
      \PAR
      \PR{y}{r}\SEMI
      \PW{z}{r}
      \\[-1ex]
      \hbox{\begin{tikzinline}[node distance=1.5em]
          \event{b1}{\DR{z}{1}}{}
          %\event{b2}{\DW{x}{0}}{right=of b1}
          \event{b3}{\DR{x}{0}}{right=of b1}
          %\rf{b2}{b3}
          \event{b4}{\DW{x}{1}}{right=2em of b3}
          \rmw{b3}{b4}
          \event{b5}{\DW{y}{1}}{right=of b4}
          \po[out=-20,in=-160]{b1}{b4}
          \po[out=-20,in=-160]{b3}{b5}
          \event{a1}{\DR{y}{1}}{right=3em of b5}
          \event{a2}{\DW{z}{1}}{right=of a1}
          \po{a1}{a2}
          \rf{b5}{a1}
          \rf[out=170,in=10]{a2}{b1}
        \end{tikzinline}}
    \end{gathered}
  \end{gather*}
\end{example}
This is allowed by \weakestmo{}, but not \PS{}.

\begin{example}
  Consider the \textsc{cdrf} example from \cite{DBLP:conf/pldi/LeeCPCHLV20}:
  \begin{gather*}
    \begin{gathered}
      \begin{aligned}
        &\PINC[\mACQ][\mREL]{x}{r}\SEMI \IF{r{=}0}\THEN \PW{y}{1} \FI
        \\\PAR\;\;&
        \PINC[\mACQ][\mREL]{x}{r}\SEMI \IF{r{=}0}\THEN \IF{y}\THEN \PW{x}{0} \FI \FI
      \end{aligned}
      \\
      \hbox{\footnotesize\begin{tikzinline}[node distance=1.5em]
          \raevent{a1}{\DR[\mACQ]{x}{0}}{}
          \raevent{a1b}{\DW[\mREL]{x}{1}}{right=of a1}
          \event{a2}{\DW{y}{1}}{right=of a1b}
          \raevent{b0}{\DR[\mACQ]{x}{0}}{right=3em of a2}
          \raevent{b0b}{\DW[\mREL]{x}{1}}{right=of b0}
          \event{b1}{\DR{y}{1}}{right=of b0b}
          \event{b2}{\DW{x}{0}}{right=of b1}
          \rmw{a1}{a1b}
          \rmw{b0}{b0b}
          \rf[out=-13,in=-163]{a2}{b1}
          \sync[out=20,in=160]{a1}{a2}
          \sync[out=20,in=160]{b0}{b1}
          \po{b1}{b2}
          \rf[out=-165,in=-12]{b2}{a1}
        \end{tikzinline}}
    \end{gathered}
  \end{gather*}
\end{example}

\begin{example}
  Consider this example from \cite[\textsection C]{DBLP:conf/pldi/LeeCPCHLV20}:
  \begin{gather*}
    \begin{gathered}
      \begin{aligned}
        &\PCAS[\mRLX][\mRLX]{x}{r}{0}{1}\SEMI \IF{r{\leq}1}\THEN \PW{y}{1} \FI
        \\\PAR\;\;&
        \PCAS[\mRLX][\mRLX]{x}{r}{0}{2}\SEMI \IF{r{=}0}\THEN \IF{y}\THEN \PW{x}{0} \FI \FI
      \end{aligned}
      \\
      \hbox{\footnotesize\begin{tikzinline}[node distance=1.5em]
          \event{a1}{\DR{x}{0}}{}
          \event{a1b}{\DW{x}{1}}{right=of a1}
          \event{a2}{\DW{y}{1}}{right=of a1b}
          \event{b0}{\DR{x}{0}}{right=3em of a2}
          \event{b0b}{\DW{x}{2}}{right=of b0}
          \event{b1}{\DR{y}{1}}{right=of b0b}
          \event{b2}{\DW{x}{0}}{right=of b1}
          \rmw{a1}{a1b}
          \rmw{b0}{b0b}
          \rf[out=-13,in=-163]{a2}{b1}
          \po[out=20,in=160]{a1}{a2}
          \po[out=20,in=160]{b0}{b1}
          \po{b1}{b2}
          \rf[out=-165,in=-12]{b2}{a1}
        \end{tikzinline}}
    \end{gathered}
  \end{gather*}
\end{example}

\subsection{MCA}

\begin{gather*}
  \taglabel{MCA1}
  \begin{gathered}
    \IF{z}\THEN \PW{x}{0} \FI \SEMI \PW{x}{1}
    {\PAR}
    \IF{x}\THEN \PW{y}{0} \FI \SEMI \PW{y}{1}
    {\PAR}
    \IF{y}\THEN \PW{z}{0} \FI \SEMI \PW{z}{1}
    \\[-1ex]
    \hbox{\begin{tikzinline}[node distance=1.5em]
        \event{a1}{\DR{z}{1}}{}
        \event{a2}{\DW{x}{0}}{right=of a1}
        \po{a1}{a2}
        \event{a3}{\DW{x}{1}}{right=of a2}
        \wki{a2}{a3}
        \event{b1}{\DR{x}{1}}{right=3em of a3}
        \event{b2}{\DW{y}{0}}{right=of b1}
        \po{b1}{b2}
        \event{b3}{\DW{y}{1}}{right=of b2}
        \wki{b2}{b3}
        \event{c1}{\DR{y}{1}}{right=3em of b3}
        \event{c2}{\DW{z}{0}}{right=of c1}
        \po{c1}{c2}
        \event{c3}{\DW{z}{1}}{right=of c2}
        \wki{c2}{c3}
        \rf{a3}{b1}
        \rf{b3}{c1}
        \rf[out=173,in=7]{c3}{a1}  
      \end{tikzinline}}
  \end{gathered}
  \\[1ex]
  \taglabel{MCA2}
  \begin{gathered}
    \PW{x}{0}\SEMI \PW{x}{1}
    \PAR
    \PW{y}{x}
    \PAR
    \PR[\mRA]{y}{r} \SEMI \PR{x}{s}
    \\[-1ex]
    \hbox{\begin{tikzinline}[node distance=1.5em]
        \event{wx0}{\DW{x}{0}}{}
        \event{wx1}{\DW{x}{1}}{right=of wx0}
        \wki{wx0}{wx1}
        \event{rx1}{\DR{x}{1}}{right=3em of wx1}
        \event{wy1}{\DW{y}{1}}{right=of rx1}
        \po{rx1}{wy1}
        \event{ry1}{\DRAcq{y}{1}}{right=3em of wy1}
        \event{rx0}{\DR{x}{0}}{right=of ry1}
        \rf{wx1}{rx1}
        \rf{wy1}{ry1}
        \sync{ry1}{rx0}
        \wk[out=170,in=10]{rx0}{wx1}
      \end{tikzinline}}
  \end{gathered}
\end{gather*}

These candidate executions are invalid, due to cycles.

% \endinput
\section{Not for publication}


\subsection{TC18}

\begin{verbatim}
Initially,  x = y = 0
Thread 1:         Thread 2: 
r3 = x            r2 = y    
if (r3 == 0)      x = r2    
  x = 1
r1 = x
y = r1
Behavior in question: r1 == r2 == r3 == 1
\end{verbatim}
Decision: Allowed. A compiler could determine that the only legal values for
$x$ are $0$ and $1$. From that, the compiler could deduce that $r3 \neq 0$
implies $r3 = 1$.  A compiler could then determine that at $r1 = x$ in thread
1, is must be legal for to read $x$ and see the value $1$. Changing $r1 = x$
to $r1 = 1$ would allow $y = r1$ to be transformed to $y = 1$ and performed
earlier, resulting in the behavior in question.
\begin{displaymath}
  \begin{gathered}
    \PR{x}{r}
    \SEMI
    \IF{r{=}0}\THEN \PW{x}{1}\FI
    \SEMI
    \PR{x}{s}
    \SEMI
    \PW{y}{s}
    \PAR
    \PW{x}{\PR{y}{}}
    \\[-1ex]
    \hbox{\begin{tikzinline}[node distance=1.5em]
        \event{a}{\DR{x}{1}}{}
        %\event{b}{\DW{x}{1}}{right=of a}
        \event{c}{\DR{x}{1}}{right=of a}
        \event{d}{\aForm\mid\DW{y}{1}}{right=of c}
        \event{e}{\DR{y}{1}}{right=3em of d}
        \event{f}{\DW{x}{1}}{right=of e}
        \po{e}{f}
        \rf{d}{e}
        \rf[out=-165,in=-15]{f}{a}
        \rf[out=-165,in=-15]{f}{c}
      \end{tikzinline}}
    \\[-1ex]
    \hbox{\begin{tikzinline}[node distance=1.5em]
        \event{a}{\DR{x}{1}}{}
        %\event{b}{\DW{x}{1}}{right=of a}
        %\event{c}{\DR{x}{1}}{right=of a}
        \event{d}{\aForm\mid\DW{y}{1}}{right=of a}
        \event{e}{\DR{y}{1}}{right=3em of d}
        \event{f}{\DW{x}{1}}{right=of e}
        \po{e}{f}
        \rf{d}{e}
        \rf[out=-165,in=-15]{f}{a}
        %\rf[out=-165,in=-15]{f}{c}
      \end{tikzinline}}
  \end{gathered}
\end{displaymath}
\begin{displaymath}
  \aForm=
  (1{=}r \lor x{=}r)
  \limplies
  \PBR{
    \SBR{r{=}0\land\PBR{(1{=}s \lor 1{=}s) \limplies s{=}1}}
    \lor
    \SBR{r{\neq}0\land\PBR{(1{=}s \lor x{=}s) \limplies s{=}1}}
  }
\end{displaymath}
If we coalesce $s$ and $r$ and prefix $\PW{x}{0}$
\begin{displaymath}
  \aForm=
  (1{=}r \lor 0{=}r)
  \limplies
  \PBR{
    r{=}0
    \lor
    \SBR{r{\neq}0\land\PBR{(1{=}r \lor 0{=}r) \limplies r{=}1}}
  }
\end{displaymath}
which is
\begin{displaymath}
  \aForm=
  1{=}r
  \limplies
  \PBR{(1{=}r \lor 0{=}r) \limplies r{=}1}
\end{displaymath}
which is a tautology.

\subsection{Load hoisting in LLVM}
Load-hoisting followed by case analysis is unsound in LLVM, without freeze.
Introducing a read may cause a race, resulting in read value $\UNDEFINED$.
Branch on $\UNDEFINED$ is UB.  Freeze was added to get around this...

Examples from \cite{promising-ldrf} show that freeze is bullshit.  Compcert
does not validate loop switching
\cite[\textsection9]{DBLP:conf/pldi/LeeKSHDMRL17}.

\cite[\textsection3.3]{DBLP:conf/pldi/LeeKSHDMRL17} Global Value Numbering
(GVN) and Loop Unswitching require different semantics for branch on
$\UNDEFINED$. \url{https://llvm.org/docs/LangRef.html#freeze-instruction}

\href{https://lists.llvm.org/pipermail/llvm-dev/2016-October/106182.html}{Purpose of Freeze}
\begin{quotation}
  Poison is propagated aggressively throughout. However, there are cases
  where this breaks certain optimizations, and therefore freeze is used to
  selectively stop poison from being propagated.

  A use of freeze is to enable speculative execution.  For example, loop
  switching does the following transformation:
\begin{verbatim}
while (C) {        if (C2) {     
  if (C2) {           while (C)  
   A                     A       
  } else {   ==>   } else {      
   B                   while (C) 
  }                       B      
}                  }             
\end{verbatim}
  Here we are evaluating C2 before C.  If the original loop never executed
  then we had never evaluated C2, while now we do.  So we need to make sure
  there's no UB for branching on C2.  Freeze ensures that so we would
  actually have 'if (freeze(C2))' instead.  Note that
  \emph{having branch on poison not trigger UB has its own problems.}
  We believe this is a good tradeoff.
\end{quotation}


\cite[\textsection6.3]{DBLP:conf/cgo/ChakrabortyV17}:
\begin{quotation}
  LLVM frequently performs such load introductions in the “simplify CFG”
  pass; e.g., when hoisting loads outside of conditionals.
\end{quotation}

case analysis happens in ``function specialization'' also known as
``procedure cloning''

\subsection{Hoisting and CSE}
Example from Viktor's talk: ``Weak Memory Concurrency in C/C++11 and LLVM''

Load Hoisting:
\begin{displaymath}
  \IF{c}\THEN \PR{x}{a}\FI
  \rightsquigarrow
  \PR{x}{t}\SEMI
  \LET{a}{\TERNARY{c}{t}{a}}
\end{displaymath}

CSE over acquiring lock:
\begin{displaymath}
  \PR{x}{a}\SEMI
  \LOCK\SEMI
  \PR{x}{b}
  \rightsquigarrow
  \PR{x}{a}\SEMI
  \LOCK\SEMI
  \LET{b}{a}
\end{displaymath}

Having both is clearly wrong:
\begin{displaymath}
  \begin{array}{l}
    \IF{c}\THEN\\
    \quad\PR{x}{a}\FI\SEMI\\
    \LOCK\SEMI\\
    \PR{x}{b}    
  \end{array}
  \rightsquigarrow
  \begin{array}{l}
    \PR{x}{t}\SEMI\\
    \LET{a}{\TERNARY{c}{t}{a}}\SEMI\\
    \LOCK\SEMI\\
    \PR{x}{b}    
  \end{array}
  \rightsquigarrow
  \begin{array}{l}
    \PR{x}{t}\SEMI\\
    \LET{a}{\TERNARY{c}{t}{a}}\SEMI\\
    \LOCK\SEMI\\
    \LET{b}{a}    
  \end{array}
\end{displaymath}
When c is false, x is moved out of the critical region!

So we have to forbid one transformation.
\begin{itemize}
\item C11 forbids load hoisting, allows CSE over lock().
\item LLVM allows load hoisting, forbids CSE over lock().
\end{itemize}
\subsection{More RMW}
These following examples are from \cite{promising-ldrf}.

\ref{CDRF} shows that \PwT{} semantics is not too permissive for $\mRA$-\RMW{}s.
But what about $\mRLX$-\RMW{}s.  The following execution is allowed by \armeight,
and \PS{2.0}, but disallowed by \PS{2.1}.
\begin{gather*}
  \taglabel{RMW-W}
  \begin{gathered}
    \PFADD[\mRLX][\mRLX]{x}{r}{1}\SEMI \PW{y}{1}
    \PAR
    \PR{y}{r}\SEMI \PFADD[\mRLX][\mRLX]{x}{s}{r}
    \\
    \hbox{\begin{tikzinline}[node distance=2em]
        \event{a1}{\DR{x}{1}}{}
        \event{a1b}{\DW{x}{2}}{below=1em of a1}
        \event{a2}{\DW{y}{1}}{right=of a1}
        \event{b1}{\DR{y}{1}}{right=3em of a2}
        \event{b2}{\DR{x}{0}}{right=of b1}
        \event{b2b}{\DW{x}{1}}{below=1em of b2}
        \rmw{a1}{a1b}
        \rmw{b2}{b2b}
        \rf{a2}{b1}
        \po{b1}{b2b}
        \rf[out=-175,in=-20]{b2b}{a1}
      \end{tikzinline}}
  \end{gathered}
\end{gather*}

If this $\ldrfra{z}$?
\begin{gather*}
  \taglabel{Naive-LDRF-RA-Fail}
  \begin{gathered}
    \IF{y}\THEN \PW{x}{z} \ELSE \PW{x}{1} \FI
    \PAR
    \PR{x}{r}\SEMI \PW{z}{1}\SEMI \PW{y}{r}
    \\
    \hbox{\begin{tikzinline}[node distance=2em]
        \event{a1}{\DR{y}{1}}{}
        \event{a2}{\DR{z}{1}}{right=of a1}
        \event{a3}{\DW{x}{1}}{right=of a2}
        \event{b1}{\DR{x}{1}}{right=3em of a3}
        \event{b2}{\DW{z}{1}}{right=of b1}
        \event{b3}{\DW{y}{1}}{right=of b2}
        \po{a2}{a3}
        \po[in=165,out=15]{b1}{b3}
        \rf[out=-165,in=-15]{b2}{a2}
        \rf[out=-165,in=-15]{b3}{a1}
        \rf{a3}{b1}
      \end{tikzinline}}
  \end{gathered}
\intertext{Interpreting $\{z\}$ as $\mRA$:}
    \\
  \begin{gathered}
    \hbox{\begin{tikzinline}[node distance=2em]
        \event{a1}{\DR{y}{1}}{}
        \event{a2}{\DR[\mACQ]{z}{1}}{right=of a1}
        \event{a3}{\DW{x}{1}}{right=of a2}
        \event{b1}{\DR{x}{1}}{right=3em of a3}
        \event{b2}{\DW[\mREL]{z}{1}}{right=of b1}
        \event{b3}{\DW{y}{1}}{right=of b2}
        \po{a2}{a3}
        \po[in=165,out=15]{b1}{b3}
        \rf[out=-165,in=-15]{b2}{a2}
        \rf[out=-165,in=-15]{b3}{a1}
        \rf{a3}{b1}
        \sync{a1}{a2}
        \sync{b2}{b3}
      \end{tikzinline}}
  \end{gathered}
\end{gather*}

\PwT{} disallows \ref{LDRF-Fail-PS}, which is similar to \ref{OOTA4}.
\begin{gather*}  
  \taglabel{LDRF-Fail-PS}
  \begin{gathered}
  \IF{x}\THEN
    \PFADD{w}{}{1}\SEMI
    \PW{y}{1}\SEMI
    \PW{z}{1}
  \FI
  \PAR
  \IF{\BANG z}\THEN
    \PW{x}{1}
  \ELSE
    \IF{\BANG \PFADD{w}{}{1}}\THEN
      \PW{x}{\PR{y}{}}
    \FI
  \FI
    \\
    \hbox{\begin{tikzinline}[node distance=2em]
        \event{a1}{\DR{x}{1}}{}
        \event{a2}{\DR{w}{1}}{right=of a1}
        \event{a3}{\DW{w}{2}}{right=of a2}
        \event{a4}{\DW{y}{1}}{right=of a3}
        \event{a5}{\DW{z}{1}}{right=of a4}
        \event{b1}{\DR{z}{1}}{right=5em of a5}
        \event{b2}{\DR{w}{0}}{right=of b1}
        \event{b3}{\DW{w}{1}}{right=of b2}
        \event{b4}{\DR{y}{1}}{right=of b3}
        \event{b5}{\DW{x}{1}}{right=of b4}
        \rmw{a2}{a3}
        \po[out=15,in=165]{a1}{a3}
        \po[out=15,in=165]{a1}{a4}
        \po[out=15,in=165]{a1}{a5}        
        \rmw{b2}{b3}
        \po{b4}{b5}
        \po[out=15,in=165]{b2}{b5}        
        \po[out=15,in=165]{b1}{b3}
        \rf{a5}{b1}
        \rf[out=10,in=170]{a4}{b4}
        \rf[out=-170,in=-10]{b3}{a2}
        \rf[out=-170,in=-10]{b5}{a1}
      \end{tikzinline}}
  \end{gathered}
\end{gather*}
\begin{gather}
  \taglabel{OOTA4}
  \begin{gathered}
    \PW{y}{x}
    \PAR
    \PR{y}{r} \SEMI \IF{b}\THEN  \PW{x}{r} \SEMI \PW{z}{r} \ELSE \PW{x}{1} \FI
    \PAR
    \PW{b}{1}
    % \\[-1ex]
    % \hbox{\begin{tikzinline}[node distance=1.5em]
    %     \event{rx}{\DR{x}{1}}{}
    %     \event{wy}{\DW{y}{1}}{right=of rx}
    %     \po{rx}{wy}
    %     \event{ry}{\DR{y}{1}}{right=3em of wy} 
    %     \event{wx}{\DW{x}{1}}{right=of ry}
    %     \event{wz}{\DW{z}{1}}{right=of wx}
    %     \event{rb}{\DR{b}{1}}{right=of wz}
    %     \event{wb1}{\DW{b}{1}}{right=3em of rb}
    %     \po{ry}{wx}
    %     \rf{wb1}{rb}
    %     \rf{wy}{ry}
    %     \rf[out=-170,in=-10]{wx}{rx}
    %     \po{rb}{wz}
    %     \po[out=15,in=165]{ry}{wz}
    %   \end{tikzinline}}
    \\[-1ex]
    \hbox{\begin{tikzinline}[node distance=1.5em]
        \event{rx}{\DR{x}{1}}{}
        \event{wy}{\DW{y}{1}}{right=of rx}
        \po{rx}{wy}
        \event{rb}{\DR{b}{1}}{right=3em of wy}
        \event{ry}{\DR{y}{1}}{right=of rb} 
        \event{wx}{\DW{x}{1}}{right=of ry}
        \event{wz}{\DW{z}{1}}{right=of wx}
        \event{wb1}{\DW{b}{1}}{right=3em of wz}
        \po{ry}{wx}
        \rf[out=-170,in=-10]{wb1}{rb}
        \rf[out=15,in=165]{wy}{ry}
        \rf[out=-170,in=-10]{wx}{rx}
        \po[out=15,in=165]{rb}{wz}
        \po[out=15,in=165]{ry}{wz}
      \end{tikzinline}}
  \end{gathered}  
\end{gather}

% \begin{comment}
%   \centering  
% \begin{verbatim}
% a := X                  b := Z                 
% if a = 1 then           if b = 0 then          
%   _ := FADD(W , 1)        X := 1               
%   Y := 1                else                   
%   Z := 1                  c := FADD(W, 1) /0   
%                           if c = 0 then        
%                             d := Y             
%                             X := d             
% \end{verbatim}
% \includegraphics[width=\textwidth]{LDRF-Fail-PS}
% \caption{LDRF-Fail-PS}
% \end{comment}


If \RMW{}s simply use the same semantics as read and write, then we allow
\ref{LDRF-PF-Fail}, which is used to show failure of $\ldrfsc{}$.
\begin{gather*}  
  \taglabel{LDRF-PF-Fail}
  \begin{gathered}
    \PW{y}{0}\SEMI
    \IF{y}\THEN
      \IF{\BANG\PCAS{x}{}{0}{1}}\THEN
        \IF{z}\THEN
          \PW{x}{2}
    \FI\FI\FI
    \PAR
    \PW{y}{1}\SEMI
    \IF{1{\neq}\PCAS{x}{}{0}{3}}\THEN
      \PW{z}{1}
    \FI
    \\
    \hbox{\begin{tikzinline}[node distance=2em]
        \event{a1}{\DW{y}{0}}{}
        \event{a2}{\DR{y}{1}}{right=of a1}
        \event{a3}{\DR{x}{0}}{right=of a2}
        \event{a4}{\DW{x}{1}}{right=of a3}
        \event{a5}{\DR{z}{1}}{right=of a4}
        \event{a6}{\DW{x}{2}}{right=of a5}
        \event{b1}{\DW{y}{1}}{right=5em of a6}
        \event{b2}{\DR{x}{2}}{right=of b1}
        \event{b3}{\DW{z}{1}}{right=of b2}
        \wk{a1}{a2}
        \rmw{a3}{a4}
        \po[out=10,in=170]{a2}{a6}
        %\po[out=15,in=165]{a3}{a6}
        \po{a5}{a6}
        \wk[out=-20,in=-160]{a4}{a6}
        %\po{b2}{b3}
        \rf[out=15,in=165]{a6}{b2}
        \rf[out=-170,in=-10]{b3}{a5}
        \rf[out=-170,in=-10]{b1}{a2}
      \end{tikzinline}}
  \end{gathered}
\end{gather*}
To disallow this, we need to retain the dependency
\begin{math}
  \DRP{x}{2}\xpo \DWP{z}{1}.
\end{math}
For this, we need to avoid the substitution for $x$.  This is why we use
$\sLOADP{}{}$ instead of $\sLOAD{}{}$ in the independent case for \RMW{}s.

\begin{comment}
  \centering  
\begin{verbatim}
Y := 0                   Y := 1                 
a := Y                   d := CAS(X,0,1) /37?   
if a != 0 then           if d != 42 then        
  b := CAS(X,0,42)         L := 1               
  if b = 0 then
    c := L
    if c = 1 then
      Xsrlx := 37
\end{verbatim}
\includegraphics[width=.8\textwidth]{LDRF-PF-Fail.png}
\caption{LDRF-PF-Fail}
\end{comment}

\subsection{A Note on Mixed-Mode Data Races}

In preparing this paper, we came across the following example, which appears
to invalidate Theorem 4.1 of \cite{DBLP:conf/ppopp/DongolJR19}.
\begin{gather}
  \nonumber
  \PW{x}{1}\SEMI
  \PW[\mREL]{y}{1}\SEMI
  \PR[\mACQ]{x}{r}
  \PAR
  \IF{\PR[\mACQ]{y}{}}\THEN \PW[\mREL]{x}{2}\FI
  \\
  \tag{\P}
  \label{mix1}
  \hbox{\begin{tikzinline}[node distance=1.5em]
      \event{a1}{\DW{x}{1}}{}
      \raevent{a2}{\DW[\mREL]{y}{1}}{right=of a1}
      \raevent{a3}{\DR[\mACQ]{x}{1}}{right=of a2}
      \raevent{b1}{\DR[\mACQ]{y}{1}}{right=3em of a3}
      % \raevent{b1}{\DR[\mACQ]{y}{1}}{below=of a1}
      \raevent{b2}{\DW[\mREL]{x}{2}}{right=of b1}
      \sync{a1}{a2}
      \rf[out=20,in=160]{a1}{a3}
      \rf[out=20,in=160]{a2}{b1}
      \wk[out=-20,in=-160]{a3}{b2}
      \sync{b1}{b2}
      % \node(ai)[left=3em of a1]{};
      % \bgoval[yellow!50]{(ai)}{P}
      % \bgoval[pink!50]{(a1)(a2)(b1)(b2)}{P'\setminus P}
      % \bgoval[green!10]{(a3)}{P'''\setminus P'}
    \end{tikzinline}}
  \\
  \nonumber
  %\label{mix2}
  \hbox{\begin{tikzinline}[node distance=1.5em]
      \event{a1}{\DW{x}{1}}{}
      \raevent{a2}{\DW[\mREL]{y}{1}}{right=of a1}
      \raevent{a3}{\DR[\mACQ]{x}{2}}{right=of a2}
      \raevent{b1}{\DR[\mACQ]{y}{1}}{right=3em of a3}
      \raevent{b2}{\DW[\mREL]{x}{2}}{right=of b1}
      \sync{a1}{a2}
      \rf[out=20,in=160]{a2}{b1}
      \rf[out=160,in=20]{b2}{a3}
      \sync{b1}{b2}
    \end{tikzinline}}
\end{gather}
The program is data-race free.  The two executions shown are the only
top-level executions that include $\DWP[\mREL]{x}{2}$.

Theorem 4.1 of \cite{DBLP:conf/ppopp/DongolJR19} is stated by extending
execution sequences.  In the terminology of
\cite{DBLP:conf/ppopp/DongolJR19}, a read is \emph{$L$-weak} if it is
sequentially stale.  Let
\begin{math}
  \rho=\DWP{x}{1}\allowbreak
  \DWP[\mREL]{y}{1}\allowbreak
  \DRP[\mACQ]{y}{1}\allowbreak
  \DWP[\mREL]{x}{2}
\end{math}
be a sequence and
\begin{math}
  \alpha=\DRP[\mACQ]{x}{1}.
\end{math}
$\rho$ is $L$-sequential and $\alpha$ is $L$-weak in $\rho\alpha$.  But there
is no execution of this program that includes a data race, contradicting the
theorem.  The error seems to be in Lemma A.4 of
\cite{DBLP:conf/ppopp/DongolJR19}, which states that if $\alpha$ is $L$-weak
after an $L$-sequential $\rho$, then $\alpha$ must be in a data race.  That
is clearly false here, since $\DRP[\mACQ]{x}{1}$ is stale, but the program is
data race free.

In proving the SC-LDRF result in \cite[\textsection8]{DBLP:journals/pacmpl/JagadeesanJR20}, we noted that our proof
technique is more robust than that of \cite{DBLP:conf/ppopp/DongolJR19},
because it limits the prefixes that must be considered.  In \eqref{mix1}, the
induction hypothesis requires that we add $\DRP[\mACQ]{x}{1}$ before
$\DWP[\mREL]{x}{2}$ since $\DRP[\mACQ]{x}{1}\xwk\DWP[\mREL]{x}{2}$.  In
particular,
\begin{gather*}
  \hbox{\begin{tikzinline}[node distance=1.5em]
      \event{a1}{\DW{x}{1}}{}
      \raevent{a2}{\DW[\mREL]{y}{1}}{right=of a1}
      % \raevent{a3}{\DR[\mACQ]{x}{1}}{right=of a2}
      \raevent{b1}{\DR[\mACQ]{y}{1}}{right=3em of a3}
      % \raevent{b1}{\DR[\mACQ]{y}{1}}{below=of a1}
      \raevent{b2}{\DW[\mREL]{x}{2}}{right=of b1}
      \sync{a1}{a2}
      % \rf[out=20,in=160]{a1}{a3}
      \rf[out=20,in=160]{a2}{b1}
      % \wk[out=-20,in=-160]{a3}{b2}
      \sync{b1}{b2}
      % \node(ai)[left=3em of a1]{};
      % \bgoval[yellow!50]{(ai)}{P}
      % \bgoval[pink!50]{(a1)(a2)(b1)(b2)}{P'\setminus P}
      % \bgoval[green!10]{(a3)}{P'''\setminus P'}
    \end{tikzinline}}
\end{gather*}
is not a downset of \eqref{mix1}, because
$\DRP[\mACQ]{x}{1}\xwk\DWP[\mREL]{x}{2}$.  As noted in \cite[\textsection8]{DBLP:journals/pacmpl/JagadeesanJR20},
this affects the inductive order in which we move across pomsets, but does
not affect the set of pomsets that are considered.  In particular,
\begin{gather*}
  \hbox{\begin{tikzinline}[node distance=1.5em]
      \event{a1}{\DW{x}{1}}{}
      \raevent{a2}{\DW[\mREL]{y}{1}}{right=of a1}
      % \raevent{a3}{\DR[\mACQ]{x}{1}}{right=of a2}
      \raevent{b1}{\DR[\mACQ]{y}{1}}{right=3em of a3}
      % \raevent{b1}{\DR[\mACQ]{y}{1}}{below=of a1}
      % \raevent{b2}{\DW[\mREL]{x}{2}}{right=of b1}
      \sync{a1}{a2}
      % \rf[out=20,in=160]{a1}{a3}
      \rf[out=20,in=160]{a2}{b1}
      % \wk[out=-20,in=-160]{a3}{b2}
      % \sync{b1}{b2}
      % \node(ai)[left=3em of a1]{};
      % \bgoval[yellow!50]{(ai)}{P}
      % \bgoval[pink!50]{(a1)(a2)(b1)(b2)}{P'\setminus P}
      % \bgoval[green!10]{(a3)}{P'''\setminus P'}
    \end{tikzinline}}
\end{gather*}
is a downset of \eqref{mix1}.


\section{Old Notes}
\subsection{JMM examples}

TC17: Should be possible to read 1 everywhere
\begin{gather*}
  x\GETS y
  \PAR
  r\GETS x\SEMI\IF{r{\neq}1}\THEN x\GETS 1 \SEMI y\GETS x\ELSE y\GETS x\FI
  \\
  \begin{gathered}
    \IF{r{\neq}1}\THEN x\GETS 1 \SEMI y\GETS x\FI
    \\
    \hbox{\begin{tikzinline}[node distance=.5em and 1em]
        \event{a1}{r{\neq}1 \mid \DW{x}{1}}{}
        \event{a2}{r{\neq}1 \mid \DR{x}{1}}{right=of a1}
        \event{a3}{r{\neq}1 \land 1{=}1\mid \DW{y}{1}}{right=of a2}
        \wk{a1}{a2}
      \end{tikzinline}}
  \end{gathered}
  \\
  \begin{gathered}
    \IF{r{=}1}\THEN y\GETS x\FI
    \\
    \hbox{\begin{tikzinline}[node distance=.5em and 1em]
        \event{a4}{r{=}1 \mid \DR{x}{1}}{}
        \event{a5}{r{=}1 \land x{=}1\mid \DW{y}{1}}{right=of a4}
      \end{tikzinline}}
  \end{gathered}  
  \\
  \begin{gathered}
    \IF{r{\neq}1}\THEN x\GETS 1 \SEMI y\GETS x\ELSE y\GETS x\FI
    \\
    \hbox{\begin{tikzinline}[node distance=.5em and 1em]
        \event{a1}{r{\neq}1 \mid \DW{x}{1}}{}
        \event{a2}{\DR{x}{1}}{right=of a1}
        \event{a3}{r{\neq}1\lor x{=}1\mid \DW{y}{1}}{right=of a2}
        \wk{a1}{a2}
      \end{tikzinline}}
  \end{gathered}    
  \\
  \begin{gathered}
    r\GETS x\SEMI\IF{r{\neq}1}\THEN x\GETS 1 \SEMI y\GETS x\ELSE y\GETS x\FI
    \\
    \hbox{\begin{tikzinline}[node distance=.5em and 1em]
        \event{a1}{x{\neq}1 \mid \DW{x}{1}}{}
        \event{a2}{\DR{x}{1}}{right=of a1}
        \event{a3}{x{\neq}1\lor x{=}1\mid \DW{y}{1}}{right=of a2}
        \event{a0}{\DR{x}{1}}{left=of a1}
        \wk{a1}{a2}
        \wk{a0}{a1}
      \end{tikzinline}}
  \end{gathered}    
  \\
  \begin{gathered}
    x\GETS0\SEMI r\GETS x\SEMI\IF{r{\neq}1}\THEN x\GETS 1 \SEMI y\GETS x\ELSE y\GETS x\FI
    \\
    \hbox{\begin{tikzinline}[node distance=.5em and 1em]
        \event{a1}{0{\neq}1 \mid \DW{x}{1}}{}
        \event{a2}{\DR{x}{1}}{right=of a1}
        \event{a3}{\DW{y}{1}}{right=of a2}
        \event{a0}{\DR{x}{1}}{left=of a1}
        \event{a-1}{\DW{x}{0}}{left=of a0}
        \wk{a1}{a2}
        \wk{a0}{a1}
        \wk{a-1}{a0}
      \end{tikzinline}}
  \end{gathered}    
\end{gather*}

TC18
\begin{gather*}
  x\GETS y
  \PAR
  r\GETS x\SEMI\IF{r{=}0}\THEN x\GETS 1 \SEMI y\GETS x\ELSE y\GETS x\FI
  \\
  \begin{gathered}
    \IF{r{=}0}\THEN x\GETS 1 \SEMI y\GETS x\FI
    \\
    \hbox{\begin{tikzinline}[node distance=.5em and 1em]
        \event{a1}{r{=}0 \mid \DW{x}{1}}{}
        \event{a2}{r{=}0 \mid \DR{x}{1}}{right=of a1}
        \event{a3}{r{=}0 \land 1{\neq}0\mid \DW{y}{1}}{right=of a2}
        \wk{a1}{a2}
      \end{tikzinline}}
  \end{gathered}
  \\
  \begin{gathered}
    \IF{r{\neq}0}\THEN y\GETS x\FI
    \\
    \hbox{\begin{tikzinline}[node distance=.5em and 1em]
        \event{a4}{r{\neq}0 \mid \DR{x}{1}}{}
        \event{a5}{r{\neq}0 \land x{\neq}0\mid \DW{y}{1}}{right=of a4}
      \end{tikzinline}}
  \end{gathered}  
  \\
  \begin{gathered}
    \IF{r{=}0}\THEN x\GETS 1 \SEMI y\GETS x\ELSE y\GETS x\FI
    \\
    \hbox{\begin{tikzinline}[node distance=.5em and 1em]
        \event{a1}{r{=}0 \mid \DW{x}{1}}{}
        \event{a2}{\DR{x}{1}}{right=of a1}
        \event{a3}{r{=}0\lor x{\neq}0\mid \DW{y}{1}}{right=of a2}
        \wk{a1}{a2}
      \end{tikzinline}}
  \end{gathered}    
  \\
  \begin{gathered}
    r\GETS x\SEMI\IF{r{=}0}\THEN x\GETS 1 \SEMI y\GETS x\ELSE y\GETS x\FI
    \\
    \hbox{\begin{tikzinline}[node distance=.5em and 1em]
        \event{a1}{x{=}0 \mid \DW{x}{1}}{}
        \event{a2}{\DR{x}{1}}{right=of a1}
        \event{a3}{x{=}0\lor x{\neq}0\mid \DW{y}{1}}{right=of a2}
        \event{a0}{\DR{x}{1}}{left=of a1}
        \wk{a1}{a2}
        \wk{a0}{a1}
      \end{tikzinline}}
  \end{gathered}    
  \\
  \begin{gathered}
    x\GETS0\SEMI r\GETS x\SEMI\IF{r{=}0}\THEN x\GETS 1 \SEMI y\GETS x\ELSE y\GETS x\FI
    \\
    \hbox{\begin{tikzinline}[node distance=.5em and 1em]
        \event{a1}{0{=}0 \mid \DW{x}{1}}{}
        \event{a2}{\DR{x}{1}}{right=of a1}
        \event{a3}{\DW{y}{1}}{right=of a2}
        \event{a0}{\DR{x}{1}}{left=of a1}
        \event{a-1}{\DW{x}{0}}{left=of a0}        
        \wk{a1}{a2}
        \wk{a0}{a1}
        \wk{a-1}{a0}
      \end{tikzinline}}
  \end{gathered}    
\end{gather*}

TC19 --- where join keeps the right hand side.
\begin{gather*}
  (x\GETS y
  \RPAR
  r\GETS x\SEMI\IF{r{\neq}1}\THEN x\GETS 1 \FI )
  \SEMI y\GETS x
\end{gather*}

TC20
\begin{gather*}
  (x\GETS y
  \RPAR
  r\GETS x\SEMI\IF{r{=}0}\THEN x\GETS 1 \FI )
  \SEMI y\GETS x
\end{gather*}

TC16  Not allowed by us or by \cite{Dolan:2018:BDR:3192366.3192421}.
\begin{gather*}
  r\GETS x\SEMI x\GETS 1
  \PAR
  s\GETS x\SEMI x\GETS 2
  \\
  \hbox{\begin{tikzinline}[node distance=1em]
      \event{a1}{\DR{x}{2}}{}
      \event{a2}{\DW{x}{1}}{right=of a1}
      \wk{a1}{a2}
      \event{b1}{\DR{x}{1}}{below=of a1}
      \event{b2}{\DW{x}{2}}{right=of b1}
      \wk{b1}{b2}
      \rf{b2}{a1}
      \rf{a2}{b1}
   \end{tikzinline}}
\end{gather*}
But we allow the following, which \cite{Dolan:2018:BDR:3192366.3192421} disallows.
\begin{gather*}
  r\GETS y\SEMI x\GETS 1
  \PAR
  s\GETS x\SEMI y\GETS 2
  \\
  \hbox{\begin{tikzinline}[node distance=1em]
      \event{a1}{\DR{y}{2}}{}
      \event{a2}{\DW{x}{1}}{right=of a1}
      %\wk{a1}{a2}
      \event{b1}{\DR{x}{1}}{below=of a1}
      \event{b2}{\DW{y}{2}}{right=of b1}
      %\wk{b1}{b2}
      \rf{b2}{a1}
      \rf{a2}{b1}
   \end{tikzinline}}
\end{gather*}

\subsection{Sevcik examples}

\citet[\textsection7]{DBLP:conf/esop/CenciarelliKS07} example. (I
incorrectly credit \citet{DBLP:conf/ecoop/SevcikA08}.)

\begin{gather*}
  \IF{x\land y}\THEN z\GETS 1\FI
  \PAR
  \IF{z}\THEN x\GETS1\SEMI y\GETS1 \ELSE y\GETS1\SEMI x\GETS1 \FI
  \\
  \hbox{\begin{tikzinline}[node distance=.5em and 1em]
      \event{a1}{\DR{x}{1}}{}
      \event{a2}{\DR{y}{1}}{right=of a1}
      \event{a3}{\DW{z}{1}}{right=of a2}
      \po{a2}{a3}
      \po[out=15,in=165]{a1}{a3}      
      \event{b1}{\DR{z}{1}}{right=3em of a3}
      \event{b2}{\DW{y}{1}}{right=of b1}
      \event{b3}{\DW{x}{1}}{right=of b2}
      % \po{b1}{b2}
      % \po[out=15,in=165]{b1}{b3}
      \rf{a3}{b1}
      \rf[out=-165,in=-15]{b2}{a2}
      \rf[out=-165,in=-15]{b3}{a1}
   \end{tikzinline}}
\end{gather*}


Examples from \cite[\textsection4.1]{DBLP:conf/ecoop/SevcikA08} are interesting:
Redundant write after read elimination:
\begin{verbatim}
|| lock m2; x=1; unlock m2
|| lock m1; x=2; unlock m1
|| lock m1; lock m2; r1=x; [x=r1;] r2=x; unlock m2; unlock m1 // [bracketed line removed]
\end{verbatim}
Even without the write, r1 and r2 must see the same values, whereas JMM
allows different values for the reads when the write is missing.

Redundant read after read elimination:
\begin{verbatim}
|| y=x
|| r2=y; if (r2==1){[r3=y]; x=r3}else{x=1} // [r3=r2]
\end{verbatim}
Interesting case is left $\DW{x}{1}$.  Initially has predicate
$r_3=1$. With read rule, we have $y=1$.  In read prefixing, we don't weaken.
Instead we weaken with the read into r2.
\begin{gather*}
  \begin{gathered}
    \IF{r_2{=}1}\THEN r_3\GETS y\SEMI x\GETS r_3\FI
    \\
    \hbox{\begin{tikzinline}[node distance=.5em and 1em]
        \event{a1}{r_2{=}1 \mid \DR{y}{1}}{}
        \event{a2}{r_2{=}1 \land y{=}1 \mid \DW{x}{1}}{right=of a1}
      \end{tikzinline}}
  \end{gathered}
  \qquad
  \begin{gathered}
    \IF{r_2{\neq}1}\THEN x\GETS 1\FI
    \\
    \hbox{\begin{tikzinline}[node distance=.5em and 1em]
        \event{a2}{r_2{\neq}1 \mid \DW{x}{1}}{}
      \end{tikzinline}}
  \end{gathered}
  \\
  \IF{r_2{=}1}\THEN r_3\GETS y\SEMI x\GETS r_3 \ELSE x\GETS 1\FI
  \\
  \hbox{\begin{tikzinline}[node distance=.5em and 1em]
      \event{a1}{r_2{=}1 \mid \DR{y}{1}}{}
      \event{a2}{(r_2{=}1 \land y{=}1) \lor (r_2{\neq}1) \mid \DW{x}{1}}{right=of a1}
   \end{tikzinline}}
  \\
  r_2\GETS y \SEMI\IF{r_2{=}1}\THEN r_3\GETS y\SEMI x\GETS r_3 \ELSE x\GETS 1\FI
  \\
  \hbox{\begin{tikzinline}[node distance=.5em and 1em]
      \event{a1}{\DR{y}{1}}{}
      \event{a2}{(y{=}1 \land y{=}1) \lor (y{\neq}1) \mid\DW{x}{1}}{right=of a1}
      \event{a0}{\DR{y}{1}}{left=of a1}
      %\po[out=-15,in=-165]{a0}{a2}
   \end{tikzinline}}
\end{gather*}
To ignore the second read, we use the ``delay'' trick that we used for JMM
TC1, but this is fulfilled by a read rather than a write.
In any case, the execution with $x=y=1$ is allowed.


Roach Motel---all reads 1 impossible, but passible after swapping \verb:r1=x:
and \verb:lock m:
\begin{verbatim}
|| lock m; x=1; unlock m
|| lock m; x=2; unlock m
|| r1=x; lock m; r2=z; if(r1==2){y=1}else{y=r2}; unlock m
|| z=y
\end{verbatim}
So Question is whether you can read all 1 in
\begin{verbatim}
|| lock m; x=1; unlock m
|| lock m; x=2; unlock m
|| lock m; r1=x; r2=z; if(r1==2){y=1}else{y=r2}; unlock m
|| z=y
\end{verbatim}
In any execution, we must have 1 before 2, or 2 before 1.
\begin{itemize}
\item If thread sees 2, then read x is 2.
\item If thread sees 1, then read x is 1.
  \begin{gather*}
    \begin{gathered}
      \IF{r_1{=}2}\THEN y\GETS 1 \ELSE y\GETS r_2\FI
      \\
      \hbox{\begin{tikzinline}[node distance=.5em and 1em]
          \event{a2}{r_1{=}2\lor (r_1{\neq}2 \land r_2{=}1)\mid \DW{y}{1}}{}
        \end{tikzinline}}
    \end{gathered}
    \\
    \begin{gathered}
      r_1\GETS x\SEMI
      r_2\GETS z\SEMI
      \IF{r_1{=}2}\THEN y\GETS 1 \ELSE y\GETS r_2\FI
      \\
      \hbox{\begin{tikzinline}[node distance=.5em and 1em]
          \event{a2}{\DW{y}{1}}{}
          %\event{a2}{1{=}2\lor 1{=}1\mid \DW{y}{1}}{}
          \event{a1}{\DR{z}{1}}{left=of a2}
          \event{a0}{\DR{x}{1}}{left=of a1}
          \po{a1}{a2}
        \end{tikzinline}}
    \end{gathered}    
  \end{gather*}
  So impossible for y and z to be 1.
\end{itemize}

Irrelevant Read Introduction (can I read 1 for both y and z?)
\begin{verbatim}
|| r=z; if(!r){if(x){y=1}}else{[s=x;]y=r}
|| x=1; z=y
\end{verbatim}

\begin{gather*}
    \begin{gathered}
      \IF{\BANG r}\THEN \IF{x}\THEN y\GETS 1\FI\FI
      \\
      \hbox{\begin{tikzinline}[node distance=.5em and 1em]
          \event{a2}{r{=}0\mid\DW{y}{1}}{}
          \event{a1}{r{=}0\mid\DR{x}{1}}{left=of a2}
          \po{a1}{a2}
        \end{tikzinline}}
    \end{gathered}      
    \qquad
    \begin{gathered}
      \IF{r}\THEN s\GETS x\SEMI y\GETS r\FI
      \\
      \hbox{\begin{tikzinline}[node distance=.5em and 1em]
          \event{a2}{r{=}1\mid\DW{y}{1}}{}
          \event{a1}{r{\neq}0\mid\DR{x}{1}}{left=of a2}
        \end{tikzinline}}
    \end{gathered}      
    \\
    \begin{gathered}
      \IF{\BANG r}\THEN \IF{x}\THEN y\GETS 1\FI\ELSE y\GETS r\FI
      \\
      \hbox{\begin{tikzinline}[node distance=.5em and 1em]
          \event{a2}{r{=}0\lor r{=}1\mid\DW{y}{1}}{}
          \event{a1}{\DR{x}{1}}{left=of a2}
          \po{a1}{a2}
        \end{tikzinline}}
    \end{gathered}          
    \\
    \begin{gathered}
      z\GETS 0\SEMI r\GETS z\SEMI \IF{\BANG r}\THEN \IF{x}\THEN y\GETS 1\FI\ELSE y\GETS r\FI
      \\
      \hbox{\begin{tikzinline}[node distance=.5em and 1em]
          \event{a2}{0{=}0\lor 0{=}1\mid\DW{y}{1}}{}
          \event{a1}{\DR{x}{1}}{left=of a2}
          \po{a1}{a2}
          \event{a0}{\DR{z}{1}}{left=of a1}
          \event{a00}{\DW{z}{0}}{left=of a0}
          %\po[out=-15,in=-165]{a0}{a2}
        \end{tikzinline}}
    \end{gathered}          
  \end{gather*}
\begin{gather*}
    \begin{gathered}
      \IF{\BANG r}\THEN \IF{x}\THEN y\GETS 1\FI\FI
      \\
      \hbox{\begin{tikzinline}[node distance=.5em and 1em]
          \event{a2}{r{=}0\mid\DW{y}{1}}{}
          \event{a1}{r{=}0\mid\DR{x}{1}}{left=of a2}
          \po{a1}{a2}
        \end{tikzinline}}
    \end{gathered}      
    \qquad
    \begin{gathered}
      \IF{r}\THEN y\GETS r\FI
      \\
      \hbox{\begin{tikzinline}[node distance=.5em and 1em]
          \event{a2}{r{=}1\mid\DW{y}{1}}{}
        \end{tikzinline}}
    \end{gathered}      
    \\
    \begin{gathered}
      \IF{\BANG r}\THEN \IF{x}\THEN y\GETS 1\FI\ELSE y\GETS r\FI
      \\
      \hbox{\begin{tikzinline}[node distance=.5em and 1em]
          \event{a2}{r{=}0\lor r{=}1\mid\DW{y}{1}}{}
          \event{a1}{r{=}0\mid\DR{x}{1}}{left=of a2}
          \po{a1}{a2}
        \end{tikzinline}}
    \end{gathered}          
    \\
    \begin{gathered}
      z\GETS 0\SEMI r\GETS z\SEMI \IF{\BANG r}\THEN \IF{x}\THEN y\GETS 1\FI\ELSE y\GETS r\FI
      \\
      \hbox{\begin{tikzinline}[node distance=.5em and 1em]
          \event{a2}{0{=}0\lor 0{=}1\mid\DW{y}{1}}{}
          \event{a1}{\DR{x}{1}}{left=of a2}
          \po{a1}{a2}
          \event{a0}{\DR{z}{1}}{left=of a1}
          \event{a00}{\DW{z}{0}}{left=of a0}
          %\po[out=-15,in=-165]{a0}{a2}
        \end{tikzinline}}
    \end{gathered}          
  \end{gather*}
  If z is initialized to 2, rather than 0, then the dependencies remain and
  both are disallowed.  This relies crucially on the fact that par takes
  order from both sides.


\subsection{More optimizations}

\begin{itemize}
\item Sound to strengthen the annotation on an action from $\mRLX$ to
  $\mRA$, and from $\mRA$ to $\mSC$.
\end{itemize}

From \cite{Manson:2005:JMM:1047659.1040336}:
\begin{itemize}
\item synchronization on thread local objects can be ignored or removed
  altogether (the caveat to this is the fact that invocations of methods like
  wait and notify have to obey the correct semantics – for example, even if
  the lock is thread local, it must be acquired when perform- ing a wait),
   
\item volatile fields of thread local objects can be treated as normal
  fields.

\item redundant synchronization (e.g., when a synchronized method is called
  from another synchronized method on the same object) can be ignored or
  removed,
  
\end{itemize}

Counterexample for first two:
\begin{verbatim}
 y=1; x^AR=1; r=X^AR; z=1
\end{verbatim}
If you see $z=1$ you must see $y=1$

It would be nice if we could get at these with a strength reducing result:
synchronization actions can be replaced by relaxed actions in some cases.
Then the rules for relaxed read elimination and relaxed write elimination can
be used to get rid of them.

\subsection{Examples for semicolon semantics}

\begin{itemize}
\item Parallel asymmetric: state result for \emph{joint free} programs. 
\item Subsumption can be allowed on registers only
\item We build substitutions
\item Ignore substitutions when considering semantic equality.
\end{itemize}


Value for $r$ in $(r\EQ1\mid\DW{z}{1})$ from $(\DW{x}{1})$:
\begin{gather*}
  x\GETS 1 \PAR x\GETS 1\SEMI r\GETS x \SEMI y\GETS r\SEMI z\GETS r
  \\
  \hbox{\begin{tikzinline}[node distance=.5em and 1em]
      \event{a1}{\DW{x}{1}}{}
      \event{a2}{\DR{x}{1}}{right=of a1}
      \event{a3}{\DW{y}{1}}{right=of a2}
      \event{a4}{\DW{z}{1}}{right=of a3}
      \po{a2}{a3}
      %\po[out=-15,in=-165]{a1}{a4}
      \event{a0}{\DW{x}{1}}{left=2em of a1}
      \rf[out=15,in=165]{a0}{a2}
    \end{tikzinline}}
\end{gather*}          
Value for $r$ in $(r\EQ1\mid\DW{z}{1})$ from $(\DW{x}{1})$:
\begin{gather*}
  x\GETS 2 \PAR x\GETS 1\SEMI r\GETS x \SEMI \IF{r{>}0}\THEN y\GETS 1\FI \SEMI \IF{r{>}0}\THEN z\GETS 1\FI
  \\
  \hbox{\begin{tikzinline}[node distance=.5em and 1em]
      \event{a1}{\DW{x}{1}}{}
      \event{a2}{\DR{x}{2}}{right=of a1}
      \event{a3}{\DW{y}{1}}{right=of a2}
      \event{a4}{\DW{z}{1}}{right=of a3}
      \po{a2}{a3}
      %\po[out=-15,in=-165]{a1}{a4}
      \event{a0}{\DW{x}{2}}{left=2em of a1}
      \rf[out=15,in=165]{a0}{a2}
    \end{tikzinline}}
\end{gather*}
Note that this also contains pomset where value for $r$ in
$(r\EQ1\mid\DW{y}{1})$ also comes from $(\DW{x}{1})$:
\begin{gather*}
  x\GETS 2 \PAR x\GETS 1\SEMI r\GETS x \SEMI \IF{r{>}0}\THEN y\GETS 1\FI \SEMI \IF{r{>}0}\THEN z\GETS 1\FI
  \\
  \hbox{\begin{tikzinline}[node distance=.5em and 1em]
      \event{a1}{\DW{x}{1}}{}
      \event{a2}{\DR{x}{2}}{right=of a1}
      \event{a3}{\DW{y}{1}}{right=of a2}
      \event{a4}{\DW{z}{1}}{right=of a3}
      %\po{a2}{a3}
      %\po[out=-15,in=-165]{a1}{a4}
      \event{a0}{\DW{x}{2}}{left=2em of a1}
      \rf[out=15,in=165]{a0}{a2}
    \end{tikzinline}}
\end{gather*}
So our semantics will calculate the least ordered version.  Then rely on
augmentation to get the others.
\begin{gather*}
  \begin{gathered}
    x\GETS 1
    \\[-1ex]
    \hbox{\begin{tikzinline}[node distance=.2em]
      \event{a}{\DW{x}{1}}{}
      \final{f}{\SUB{1/x}}{below=of a}
      \end{tikzinline}}
  \end{gathered}
  \qquad
  \begin{gathered}
    r\GETS x
    \\[-1ex]
    \hbox{\begin{tikzinline}[node distance=.2em]
      \event{b}{\DRreg{r}{x}{2}}{}
      \final{f}{\SUB{x/r}}{below=of b}
      \end{tikzinline}}
  \end{gathered}
  \qquad
  \begin{gathered}
    \IF{r{>}0}\THEN y\GETS 1\FI
    \\[-1ex]
    \hbox{\begin{tikzinline}[node distance=.2em]
      \event{c}{r{>}0 \mid \DW{y}{1}}{}
      \final{f}{r{>}0 \mid \SUB{1/y}}{below=of c}
      \end{tikzinline}}
  \end{gathered}
  \qquad
  \begin{gathered}
    \IF{r{>}0}\THEN z\GETS 1\FI
    \\[-1ex]
    \hbox{\begin{tikzinline}[node distance=.2em]
      \event{d}{r{>}0 \mid \DW{z}{1}}{}
      \final{f}{r{>}0 \mid \SUB{1/z}}{below=of d}
      \end{tikzinline}}
  \end{gathered}
  \\
  \begin{gathered}
    x\GETS 1
    \SEMI
    r\GETS x    
    \\[-1ex]
    \hbox{\begin{tikzinline}[node distance=.2em]
      \event{a}{\DW{x}{1}}{}
      \event{b}{\DRreg{r}{x}{2} \mid \SUB{1/x,1/r}}{right=of a}
      \final{f}{\SUB{1/x}}{below=of a}
      \end{tikzinline}}
  \end{gathered}
  \qquad
  \begin{gathered}
    \IF{r{>}0}\THEN y\GETS 1\FI
    \SEMI
    \IF{r{>}0}\THEN z\GETS 1\FI
    \\[-1ex]
    \hbox{\begin{tikzinline}[node distance=.2em]
      \event{c}{r{>}0 \mid \DW{y}{1}}{}
      \event{d}{r{>}0 \mid \DW{z}{1}}{right=of c}
      \final{f}{r{>}0 \mid \SUB{1/y,1/z}}{below=of c}
      \end{tikzinline}}
  \end{gathered}
\end{gather*}
It is also possible that the read is necessary to give a value for $r$:
\begin{gather*}
  x\GETS 2 \PAR x\GETS 0\SEMI r\GETS x \SEMI \IF{r{>}0}\THEN y\GETS 1\FI \SEMI \IF{r{>}0}\THEN z\GETS 1\FI
  \\
  \hbox{\begin{tikzinline}[node distance=.5em and 1em]
      \event{a1}{\DW{x}{0}}{}
      \event{a2}{\DR{x}{2}}{right=of a1}
      \event{a3}{\DW{y}{1}}{right=of a2}
      \event{a4}{\DW{z}{1}}{right=of a3}
      \po{a2}{a3}
      \po[out=15,in=165]{a2}{a4}
      \event{a0}{\DW{x}{2}}{left=2em of a1}
      \rf[out=15,in=165]{a0}{a2}
    \end{tikzinline}}
\end{gather*}
\begin{gather*}
  \begin{gathered}
    x\GETS 0
    \SEMI
    r\GETS x    
    \\[-1ex]
    \hbox{\begin{tikzinline}[node distance=.2em]
      \event{a}{\DW{x}{0}}{}
      \event{b}{\DRreg{r}{x}{2} \mid \SUB{0/x,0/r}}{right=of a}
      \final{f}{\SUB{0/x}}{below=of a}
      \end{tikzinline}}
  \end{gathered}
  \qquad
  \begin{gathered}
    \IF{r{>}0}\THEN y\GETS 1\FI
    \SEMI
    \IF{r{>}0}\THEN z\GETS 1\FI
    \\[-1ex]
    \hbox{\begin{tikzinline}[node distance=.2em]
      \event{c}{r{>}0 \mid \DW{y}{1}}{}
      \event{d}{r{>}0 \mid \DW{z}{1}}{right=of c}
      \final{f}{r{>}0 \mid \SUB{1/y,1/z}}{below=of c}
      \end{tikzinline}}
  \end{gathered}  
\end{gather*}
Dependency on two reads:
\begin{gather*}
  r\GETS x \SEMI s\GETS y\SEMI \IF{r{<}s}\THEN z\GETS 1\FI
  \\
  \hbox{\begin{tikzinline}[node distance=.5em and 1em]
      \event{a1}{\DRreg{r}{x}{1}}{}
      \event{a2}{\DRreg{s}{y}{2}}{right=of a1}
      \event{a3}{\DW{z}{1}}{right=of a2}
      \po{a2}{a3}
      \po[out=15,in=165]{a1}{a3}
    \end{tikzinline}}
\end{gather*}          
\begin{gather*}
  \begin{gathered}
    r\GETS x\SEMI s\GETS y
    \\[-1ex]
    \hbox{\begin{tikzinline}[node distance=.2em]
      \event{a}{\DRreg{r}{x}{1}}{}
      \event{b}{\DRreg{s}{y}{2}}{right=of a}
      \final{f}{\SUB{x/r,y/s}}{below=of a}
      \end{tikzinline}}
  \end{gathered}
  \qquad
  \begin{gathered}
    \IF{r{<}s}\THEN z\GETS 1\FI
    \\[-1ex]
    \hbox{\begin{tikzinline}[node distance=.2em]
      \event{c}{r{<}s \mid \DW{z}{1}}{}
      \final{f}{r{<}s \mid \SUB{1/z}}{below=of c}
      \end{tikzinline}}
  \end{gathered}
  \\
  \begin{gathered}
    r\GETS x\SEMI s\GETS y \SEMI \IF{r{<}s}\THEN z\GETS 1\FI
    \\[-1ex]
    \hbox{\begin{tikzinline}[node distance=.2em]
      \event{a}{\DRreg{r}{x}{1}}{}
      \event{b}{\DRreg{s}{y}{2}}{right=of a}
      \event{c}{x{<}2 \mid \DW{z}{1}}{right=1em of b}
      \po{b}{c}
      \final{f}{r{<}s \mid \SUB{x/r,y/s,1/z}}{below=of a}
      \end{tikzinline}}
  \end{gathered}
\end{gather*}
Don't need to worry about confusing reads:
\begin{gather*}
  r\GETS x \SEMI s\GETS x\SEMI \IF{s{<}0}\THEN z\GETS 1\FI
  \\
  \hbox{\begin{tikzinline}[node distance=.5em and 1em]
      \event{a1}{\DRreg{r}{x}{1}}{}
      \event{a2}{\DRreg{s}{x}{2}}{right=of a1}
      \event{a3}{\DW{z}{1}}{right=of a2}
      \po{a2}{a3}
    \end{tikzinline}}
\end{gather*}          
\begin{gather*}
  \begin{gathered}
    r\GETS x\SEMI s\GETS x
    \\[-1ex]
    \hbox{\begin{tikzinline}[node distance=.2em]
      \event{a}{\DRreg{r}{x}{1}}{}
      \event{b}{\DRreg{s}{x}{2}}{right=of a}
      \final{f}{\SUB{x/r,x/s}}{below=of a}
      \end{tikzinline}}
  \end{gathered}
  \qquad
  \begin{gathered}
    z\GETS s
    \\[-1ex]
    \hbox{\begin{tikzinline}[node distance=.2em]
      \event{c}{s{<}0 \mid \DW{z}{1}}{}
      \final{f}{s{<}0 \mid \SUB{1/z}}{below=of c}
      \end{tikzinline}}
  \end{gathered}
\end{gather*}
But we also have
\begin{gather*}
  r\GETS x \SEMI s\GETS x\SEMI \IF{s{<}0}\THEN z\GETS 1\FI
  \\
  \hbox{\begin{tikzinline}[node distance=.5em and 1em]
      \event{a1}{\DRreg{r}{x}{1}}{}
      \event{a2}{\DRreg{s}{x}{2}}{right=of a1}
      \event{a3}{\DW{z}{1}}{right=of a2}
      \po[out=15,in=165]{a1}{a3}
    \end{tikzinline}}
\end{gather*}          
\begin{gather*}
  \begin{gathered}
    r\GETS x
    \\[-1ex]
    \hbox{\begin{tikzinline}[node distance=.2em]
      \event{a}{\DRreg{r}{x}{1}}{}
      \final{f}{\SUB{x/r}}{below=of a}
      \end{tikzinline}}
  \end{gathered}
  \qquad
  \begin{gathered}
    s\GETS x \SEMI \IF{s{<}0}\THEN z\GETS 1\FI
    \\[-1ex]
    \hbox{\begin{tikzinline}[node distance=.2em]
      \event{b}{\DRreg{s}{x}{2}}{}
      \event{c}{x{<}0 \mid \DW{z}{1}}{right=of b}
      \final{f}{x{<}0 \mid \SUB{x/s,1/z}}{below=of c}
      \end{tikzinline}}
  \end{gathered}
\end{gather*}
Dependency on two reads (No dependency here):
\begin{gather*}
  r\GETS x \SEMI s\GETS x\SEMI \IF{r{=}s}\THEN z\GETS 1\FI
  \\
  \hbox{\begin{tikzinline}[node distance=.5em and 1em]
      \event{a1}{\DRreg{r}{x}{1}}{}
      \event{a2}{\DRreg{s}{x}{2}}{right=of a1}
      \event{a3}{\DW{z}{1}}{right=of a2}
      %\po{a2}{a3}
      %\po[out=15,in=165]{a1}{a3}
    \end{tikzinline}}
\end{gather*}          
\begin{gather*}
  \begin{gathered}
    r\GETS x
    \\[-1ex]
    \hbox{\begin{tikzinline}[node distance=.2em]
      \event{a}{\DRreg{r}{x}{1}}{}
      \final{f}{\SUB{x/r}}{below=of a}
      \end{tikzinline}}
  \end{gathered}
  \qquad
  \begin{gathered}
    s\GETS x \SEMI \IF{r{=}s}\THEN z\GETS 1\FI
    \\[-1ex]
    \hbox{\begin{tikzinline}[node distance=.2em]
      \event{b}{\DRreg{s}{x}{2}}{}
      \event{c}{r{=}x \mid \DW{z}{1}}{right=of b}
      \final{f}{r{=}x \mid \SUB{x/s,1/z}}{below=of c}
      \end{tikzinline}}
  \end{gathered}
\end{gather*}
Another example:
\begin{gather*}
  r\GETS x \SEMI s\GETS x\SEMI  z\GETS s
  \\
  \hbox{\begin{tikzinline}[node distance=.5em and 1em]
      \event{a1}{\DRreg{r}{x}{1}}{}
      \event{a2}{\DRreg{s}{x}{1}}{right=of a1}
      \event{a3}{\DW{z}{1}}{right=of a2}
      %\po{a2}{a3}
      \po[out=15,in=165]{a1}{a3}
    \end{tikzinline}}
\end{gather*}          
\begin{gather*}
  \begin{gathered}
    r\GETS x
    \\[-1ex]
    \hbox{\begin{tikzinline}[node distance=.2em]
      \event{a}{\DRreg{r}{x}{1}}{}
      \final{f}{\SUB{x/r}}{below=of a}
      \end{tikzinline}}
  \end{gathered}
  \qquad
  \begin{gathered}
    s\GETS x \SEMI z \GETS s
    \\[-1ex]
    \hbox{\begin{tikzinline}[node distance=.2em]
      \event{b}{\DRreg{s}{x}{1}}{}
      \event{c}{x{=}1 \mid \DW{z}{1}}{right=of b}
      \final{f}{x{=}1 \mid \SUB{x/s,1/z}}{below=of c}
      \end{tikzinline}}
  \end{gathered}
\end{gather*}

Value for $r$ in $(r{<}s\mid\DW{z}{1})$ from $(\DW{x}{0})$:
\begin{gather*}
  x\GETS 0\SEMI r\GETS x \SEMI s\GETS y\SEMI \IF{r{<}s}\THEN z\GETS 1\FI
  \\
  \hbox{\begin{tikzinline}[node distance=.5em and 1em]
      \event{a1}{\DRreg{r}{x}{1}}{}
      \event{a2}{\DRreg{s}{y}{2}}{right=of a1}
      \event{a3}{\DW{z}{1}}{right=of a2}
      \event{a0}{\DW{x}{0}}{left=of a1}
      \po{a2}{a3}
      %\po[out=15,in=165]{a0}{a3}
    \end{tikzinline}}
\end{gather*}          


Contrary to submission, reverse subsumption not okay.
\begin{gather*}
  \begin{gathered}
    x\GETS 1
    \\[-1ex]
    \hbox{\begin{tikzinline}[node distance=.2em]
      \event{a}{\DRreg{r}{x}{1}}{}
      \final{f}{\SUB{1/x}}{below=of a}
      \end{tikzinline}}
  \end{gathered}
  \qquad
  \begin{gathered}
    x\GETS 2
    \\[-1ex]
    \hbox{\begin{tikzinline}[node distance=.2em]
      \event{b}{\DRreg{s}{x}{2}}{}
      \final{f}{\SUB{}}{below=of b}
      \end{tikzinline}}
  \end{gathered}
\end{gather*}

\subsection{Commuting release and acquire}

RA example.  This is impossible, since $\DR{x}{1}$ unfulfilled.
\begin{gather*}
  x\GETS1 \SEMI
  a^\mRA\GETS1 \SEMI
  %y \GETS b^\mRA + x
  r \GETS b^\mRA\SEMI
  s \GETS x\SEMI
  y \GETS r+s
  \PAR
  r\GETS a^\mRA\SEMI
  x\GETS 2\SEMI
  b^\mRA\GETS10
  \\
  \hbox{\begin{tikzinline}[node distance=.8em and 1em]
  \event{a1}{\DW{x}{1}}{}
  \event{a2}{\DWRel{a}{1}}{right=of a1}
  \sync{a1}{a2}
  \event{b3}{\DRAcq{a}{1}}{below=of a2}
  \rf{a2}{b3}
  \event{b4}{\DW{x}{2}}{right=of b3}
  \sync{b3}{b4}
  \event{b5}{\DWRel{b}{10}}{right=of b4}
  \sync{b4}{b5}
  \event{a6}{\DRAcq{b}{10}}{above=of b5}
  \rf{b5}{a6}
  \event{a7}{\DR{x}{1}}{right=of a6}
  \sync{a6}{a7}
  \event{a8}{\DW{y}{11}}{right=of a7}
  \po{a7}{a8}
  %\sync[out=10,in=170]{a6}{a8}
    \end{tikzinline}}
\end{gather*}
If you swap the release and acquire, then it is impossible for the second
thread to get in the middle.
\begin{gather*}
  x\GETS1 \SEMI
  %y \GETS b^\mRA + x
  r \GETS b^\mRA\SEMI
  a^\mRA\GETS1 \SEMI
  % s \GETS x\SEMI
  % y \GETS r+s
  \PAR
  r\GETS a^\mRA\SEMI
  x\GETS 2\SEMI
  b^\mRA\GETS10
  \\
  \hbox{\begin{tikzinline}[node distance=.8em and 1em]
  \event{a1}{\DW{x}{1}}{}
  \event{a2}{\DWRel{a}{1}}{right=of a1}
  \sync{a1}{a2}
  \event{b3}{\DRAcq{a}{1}}{below=of a2}
  \rf{a2}{b3}
  \event{b4}{\DW{x}{2}}{right=of b3}
  \sync{b3}{b4}
  \event{b5}{\DWRel{b}{10}}{right=of b4}
  \sync{b4}{b5}
  \event{a6}{\DRAcq{b}{10}}{above=of b5}
  \rf{b5}{a6}
  % \event{a7}{\DR{x}{1}}{right=of a6}
  % \sync{a6}{a7}
  % \event{a8}{\DW{y}{11}}{right=of a7}
  % \po{a7}{a8}
  \sync{a6}{a2}
  %\sync[out=10,in=170]{a6}{a8}
    \end{tikzinline}}
\end{gather*}
In this case, the following execution is possible:
\begin{gather*}
  x\GETS1 \SEMI
  r \GETS b^\mRA\SEMI
  a^\mRA\GETS1 \SEMI
  %y \GETS b^\mRA + x
  s \GETS x\SEMI
  y \GETS r+s
  \PAR
  r\GETS a^\mRA\SEMI
  x\GETS 2\SEMI
  b^\mRA\GETS10
  \\
  \hbox{\begin{tikzinline}[node distance=.8em and 1em]
  \event{a1}{\DW{x}{1}}{}
  \event{a2}{\DRAcq{b}{10}}{right=of a1}
  \event{b5}{\DWRel{b}{10}}{below=of a2}
  \event{b4}{\DW{x}{2}}{left=of b5}
  \event{b3}{\DRAcq{a}{0}}{left=of b4}
  \sync{b4}{b5}
  \sync{b3}{b4}
  \event{a6}{\DWRel{a}{1}}{right=of a2}
  \rf{b5}{a2}
  \event{a7}{\DR{x}{1}}{right=of a6}
  \sync{a6}{a7}
  \event{a8}{\DW{y}{11}}{right=of a7}
  \po{a7}{a8}
  \sync[out=15,in=165]{a1}{a6}
  \sync{a2}{a6}
  \wk{b4}{a1}
  %\sync[out=10,in=170]{a6}{a8}
    \end{tikzinline}}
\end{gather*}
But not:
\begin{gather*}
  x\GETS1 \SEMI
  r \GETS b^\mRA\SEMI
  a^\mRA\GETS1 \SEMI
  %y \GETS b^\mRA + x
  s \GETS x\SEMI
  y \GETS r+s
  \PAR
  r\GETS a^\mRA\SEMI
  x\GETS 2\SEMI
  b^\mRA\GETS10
  \\
  \hbox{\begin{tikzinline}[node distance=.8em and 1em]
  \event{a1}{\DW{x}{1}}{}
  \event{a2}{\DRAcq{b}{10}}{right=of a1}
  \event{b5}{\DWRel{b}{10}}{below=of a2}
  \event{b4}{\DW{x}{2}}{left=of b5}
  \event{b3}{\DRAcq{a}{0}}{left=of b4}
  \sync{b4}{b5}
  \sync{b3}{b4}
  \event{a6}{\DWRel{a}{1}}{right=of a2}
  \rf{b5}{a2}
  \event{a7}{\DR{x}{1}}{right=of a6}
  \sync{a6}{a7}
  \event{a8}{\DW{y}{11}}{right=of a7}
  \po{a7}{a8}
  \sync[out=15,in=165]{a1}{a6}
  \sync{a2}{a6}
  \wk{a1}{b4}
  \wk[out=-155,in=-30]{a7}{b4}
  %\sync[out=10,in=170]{a6}{a8}
    \end{tikzinline}}
\end{gather*}

\subsection{Coherence}
 Our model of coherence is as weak as that of
\cite{Dolan:2018:BDR:3192366.3192421}:
\begin{gather*}
  x\GETS1\SEMI x\GETS 2
  \PAR
  r_1\GETS x \SEMI
  r_2\GETS x \SEMI
  r_3\GETS x \SEMI
  \\
  \hbox{\begin{tikzinline}[node distance=.8em]
    \event{a1}{\DW{x}{1}}{}
    \event{a2}{\DW{x}{2}}{right=1em of a1}
    \wk{a1}{a2}
    \event{b1}{\DRreg{r_1}{x}{2}}{below=of a1}
    \event{b2}{\DRreg{r_2}{x}{1}}{below=of a2}
    \event{b3}{\DRreg{r_3}{x}{2}}{right=1em of b2}
    % \po{b1}{b2}
    % \po{b2}{b3}
    \rf{a2}{b1}
    \rf{a1}{b2}
    \rf{a2}{b3}
    \wk{b2}{a2}
    \end{tikzinline}}
\end{gather*}

\subsection{Write rule}
Alan example of why substitute M/x rather than v/x in the write rule:
\begin{gather*}
  r\GETS y\SEMI x\GETS r\SEMI s\GETS  x\SEMI z\GETS s
  \\
  \hbox{\begin{tikzinline}[node distance=.8em and 1em]
      \event{a1}{\DR{y}{1}}{}
      \event{a2}{\DW{x}{1}}{right=of a1}
      \event{a3}{\DR{x}{1}}{right=of a2}
      \event{a4}{\DW{z}{1}}{right=of a3}
      % \po[out=15,in=165]{a1}{a4}
      \po{a1}{a2}
    \end{tikzinline}}
\end{gather*}
We lost the order from $\DR{y}{1}$ to $\DW{z}{1}$.
\begin{gather*}
   s\GETS  x\SEMI z\GETS s
  \\
  \hbox{\begin{tikzinline}[node distance=.8em and 1em]
      \event{a3}{\DR{x}{1}}{}
      \event{a4}{x\EQ1\mid\DW{z}{1}}{right=of a3}
    \end{tikzinline}}
\end{gather*}
\begin{gather*}
  x\GETS r \SEMI s\GETS  x\SEMI z\GETS s
  \\
  \hbox{\begin{tikzinline}[node distance=.8em and 1em]
      \event{a2}{\DW{x}{1}}{}
      \event{a3}{\DR{x}{1}}{right=of a2}
      \event{a4}{1\EQ1\mid\DW{z}{1}}{right=of a3}
    \end{tikzinline}}
  \\
  \hbox{\begin{tikzinline}[node distance=.8em and 1em]
      \event{a2}{\DW{x}{1}}{}
      \event{a3}{\DR{x}{1}}{right=of a2}
      \event{a4}{r\EQ1\mid\DW{z}{1}}{right=of a3}
    \end{tikzinline}}
\end{gather*}


\subsection{CSE example}
Pugh example for alias analysis and CSE:
\begin{gather*}
  r_1\GETS x \SEMI
  r_2\GETS x \SEMI  
  r_3\GETS x \SEMI
  \IF{r_3{\leq}1}\THEN y=r_2\FI
  \\
  \hbox{\begin{tikzinline}[node distance=.8em and 1em]
      \event{a1}{\DR{x}{1}}{}
      \event{a2}{\DR{x}{2}}{right=of a1}
      \event{a3}{\DR{x}{1}}{right=of a2}
      \event{a4}{\DW{y}{2}}{right=of a3}
      \po{a3}{a4}
      \po[out=15,in=165]{a2}{a4}
    \end{tikzinline}}
\end{gather*}
\begin{gather*}
  r_1\GETS x \SEMI
  r_2\GETS x \SEMI  
  r_3\GETS r_1 \SEMI
  \IF{r_3{\leq}1}\THEN y=r_2\FI
  \\
  \hbox{\begin{tikzinline}[node distance=.8em and 1em]
      \event{a1}{\DR{x}{1}}{}
      \event{a2}{\DR{x}{2}}{right=of a1}
      %\event{a3}{\DR{x}{1}}{right=of a2}
      \event{a4}{\DW{y}{2}}{right=4em of a2}
      \po{a2}{a4}
      \po[out=15,in=165]{a1}{a4}
    \end{tikzinline}}
\end{gather*}
I don't see the problem with this for now.  Is this sound????


\subsection{Playing around with 5a and 4b}
If we do this, then swap 4b and 4c, In definition 2.10, take 1-4b of def 2.8,
rather than all of it.

Another
\begin{gather*}
  r\GETS x
  \SEMI s\GETS x
  \SEMI \IF{r{>}0}\THEN y\GETS 1\FI
  \SEMI \IF{s{>}0}\THEN z\GETS 1\FI
  \\
  r\GETS x
  \SEMI \IF{r{>}0}\THEN y\GETS 1\FI
  \SEMI s\GETS x
  \SEMI \IF{s{>}0}\THEN z\GETS 1\FI
  \\
  \hbox{\begin{tikzinline}[node distance=1em]
      \event{a}{\DRreg{r}{x}{1}}{}
      \event{b}{\DRreg{s}{x}{2}}{right=of a}
      \event{c}{\DW{y}{1}}{right=of b}
      \event{d}{\DW{z}{1}}{right=of c}
      \po[out=-15,in=-165]{a}{c}
      \po[out=15,in=165]{b}{d}
    \end{tikzinline}}
  \\
  \hbox{\begin{tikzinline}[node distance=1em]
      \event{a}{\DRreg{r}{x}{1}}{}
      \event{b}{\DRreg{s}{x}{2}}{right=of a}
      \event{c}{\DW{y}{1}}{right=of b}
      \event{d}{\DW{z}{1}}{right=of c}
      \po[out=-15,in=-165]{a}{c}
      \po[out=15,in=165]{a}{d}
    \end{tikzinline}}
\end{gather*}          
\begin{gather*}
  s\GETS x
  \SEMI r\GETS x
  \SEMI \IF{r{>}0}\THEN y\GETS 1\FI
  \SEMI \IF{s{>}0}\THEN z\GETS 1\FI
  \\
  s\GETS x
  \SEMI \IF{s{>}0}\THEN z\GETS 1\FI
  \SEMI r\GETS x
  \SEMI \IF{r{>}0}\THEN y\GETS 1\FI
  \\
  \hbox{\begin{tikzinline}[node distance=1em]
      \event{a}{\DRreg{r}{x}{1}}{}
      \event{b}{\DRreg{s}{x}{2}}{right=of a}
      \event{c}{\DW{y}{1}}{right=of b}
      \event{d}{\DW{z}{1}}{right=of c}
      \po[out=-15,in=-165]{a}{c}
      \po[out=15,in=165]{b}{d}
    \end{tikzinline}}
  \\
  \hbox{\begin{tikzinline}[node distance=1em]
      \event{a}{\DRreg{r}{x}{1}}{}
      \event{b}{\DRreg{s}{x}{2}}{right=of a}
      \event{c}{\DW{y}{1}}{right=of b}
      \event{d}{\DW{z}{1}}{right=of c}
      \po{b}{c}
      \po[out=15,in=165]{b}{d}
    \end{tikzinline}}
\end{gather*}          
\begin{gather*}
  s\GETS x
  \SEMI \IF{r{>}0}\THEN y\GETS 1\FI
  \SEMI \IF{s{>}0}\THEN z\GETS 1\FI
  \\
  \hbox{\begin{tikzinline}[node distance=1em]
      \event{b}{\DRreg{s}{x}{2}}{}
      \event{c}{r{>}0\mid\DW{y}{1}}{right=of b}
      \event{d}{\DW{z}{1}}{right=of c}
      %\po{b}{c}
      \po[out=15,in=165]{b}{d}
    \end{tikzinline}}
\end{gather*}          
% For the desired result, it is sufficient if
% \begin{gather*}
%   s\GETS x
%   \SEMI \IF{r{>}0}\THEN y\GETS 1\FI
%   \SEMI \IF{s{>}0}\THEN z\GETS 1\FI
%   \\
%   \hbox{\begin{tikzinline}[node distance=1em]
%       \event{b}{\DRreg{s}{x}{2}}{}
%       \event{c}{r{=}x\mid\DW{y}{1}}{right=of b}
%       \event{d}{\DW{z}{1}}{right=of c}
%       \po{b}{c}
%       \po[out=15,in=165]{b}{d}
%     \end{tikzinline}}
% \end{gather*}          

\begin{gather*}
  \begin{gathered}
  r\GETS x
  \SEMI s\GETS x
    \\[-1ex]
    \hbox{\begin{tikzinline}[node distance=.2em]
      \event{a}{\DRreg{r}{x}{1}}{}
      \event{b}{\DRreg{s}{x}{2}}{right=of a}
      \final{f}{\SUB{x/r,x/s}}{below=of a}
      \end{tikzinline}}
  \end{gathered}
  \qquad
  \begin{gathered}
    \IF{r{>}0}\THEN y\GETS 1\FI
    \SEMI \IF{s{>}0}\THEN z\GETS 1\FI
\\[-1ex]
    \hbox{\begin{tikzinline}[node distance=.2em]
      \event{c}{r{>}0 \mid \DW{y}{1}}{}
      \event{d}{s{>}0 \mid \DW{z}{1}}{right=of c}
      \final{f}{r{>}0 \land s{>}0 \mid \SUB{1/y,1/z}}{below=of c}
      \end{tikzinline}}
  \end{gathered}
\end{gather*}

Idea to get rid of 4b and change 5a to the following:
\begin{itemize}
\item[5a.]  if $\aEv$ writes then either $\labelingForm'(\aEv)$ implies
  $\labelingForm(\aEv)$, or some $\cEv\lt'\aEv$ reads $\aVal$
  from $\aLoc$ and $\labelingForm'(\aEv)$ implies $\labelingForm(\aEv)[\aVal/\aLoc]$,
\end{itemize}
Need to get rid of 4b because it is sensitive to order of reads.

This change seems sound, because of consistency.  But it also fails to
validate read reordering on same variable, due to consistency.

Without 4b, we still do not allow:
\begin{gather*}
  r\GETS x\SEMI
  s\GETS x\SEMI
  y\GETS r\SEMI
  z\GETS r
  \\[-1ex]
  \nonumber
  \hbox{\begin{tikzinline}[node distance=.5em and 1em]
      \event{a1}{\DR{x}{1}}{}
      \event{a2}{\DR{x}{2}}{right=of a1}
      \event{a3}{\DW{y}{1}}{right=of a2}
      \event{a4}{\DW{z}{2}}{right=of a3}
      \po[out=15,in=165]{a1}{a3}
      \po[out=15,in=165]{a2}{a4}
    \end{tikzinline}}
\end{gather*}
The following is not a pomset (consistency):
\begin{gather*}
  y\GETS r\SEMI
  z\GETS r
  \\[-1ex]
  \nonumber
  \hbox{\begin{tikzinline}[node distance=.5em and 1em]
      \event{a3}{r\EQ1\mid\DW{y}{1}}{right=of a2}
      \event{a4}{r\EQ2\mid\DW{z}{2}}{right=of a3}
    \end{tikzinline}}
\end{gather*}

Without 4b, we still do not allow:
\begin{gather*}
  r\GETS x\SEMI
  s\GETS x\SEMI
  y\GETS r\SEMI
  z\GETS s\SEMI
  \IF{r{=}s}\THEN a\GETS 1\FI\SEMI
  \\[-1ex]
  \nonumber
  \hbox{\begin{tikzinline}[node distance=.5em and 1em]
      \event{a1}{\DR{x}{1}}{}
      \event{a2}{\DR{x}{2}}{right=of a1}
      \event{a3}{\DW{y}{1}}{right=of a2}
      \event{a4}{\DW{z}{2}}{right=of a3}
      \event{a5}{x{=}x\mid\DW{a}{1}}{right=of a4}
      \po[out=15,in=165]{a1}{a3}
      \po[out=15,in=165]{a2}{a4}
    \end{tikzinline}}
\end{gather*}
The following is not a pomset (consistency):
\begin{gather*}
  y\GETS r\SEMI
  z\GETS s\SEMI
  \IF{r{=}s}\THEN a\GETS 1\FI\SEMI
  \\[-1ex]
  \nonumber
  \hbox{\begin{tikzinline}[node distance=.5em and 1em]
      \event{a3}{r{=}1\mid\DW{y}{1}}{}
      \event{a4}{s{=}2\mid\DW{z}{2}}{right=of a3}
      \event{a5}{r{=}s\mid\DW{a}{1}}{right=of a4}
    \end{tikzinline}}
\end{gather*}

We do allow:
\begin{gather*}
  r\GETS x\SEMI
  s\GETS x\SEMI
  \IF{r{=}s}\THEN a\GETS 1\FI\SEMI
  \\[-1ex]
  \nonumber
  \hbox{\begin{tikzinline}[node distance=.5em and 1em]
      \event{a1}{\DR{x}{1}}{}
      \event{a2}{\DR{x}{2}}{right=of a1}
      \event{a3}{x{=}x\mid\DW{a}{1}}{right=of a2}
    \end{tikzinline}}
\end{gather*}
% \begin{gather*}
%   s\GETS x\SEMI
%   \IF{r{=}s}\THEN a\GETS 1\FI\SEMI
%   \\[-1ex]
%   \nonumber
%   \hbox{\begin{tikzinline}[node distance=.5em and 1em]
%       \event{a2}{\DR{x}{2}}{}
%       \event{a3}{r{=}x\mid\DW{a}{1}}{right=of a2}
%     \end{tikzinline}}
% \end{gather*}
And also
\begin{gather*}
  r_1\GETS x\SEMI
  r_2\GETS x\SEMI
  r_3\GETS x\SEMI
  \IF{r_3{\leq}1}\THEN y\GETS 1\FI\SEMI
  \\[-1ex]
  \nonumber
  \hbox{\begin{tikzinline}[node distance=.5em and 1em]
      \event{a0}{\DR{x}{0}}{}
      \event{a1}{\DR{x}{2}}{right=of a0}
      \event{a2}{\DR{x}{1}}{right=of a1}
      \event{a3}{1{\leq}1\mid\DW{y}{1}}{right=of a2}
      \po[out=15,in=165]{a0}{a3}
    \end{tikzinline}}
\end{gather*}

But we cannot wait forever to satisfy a precondition.
This is not a pomset:
\begin{gather*}
  r\GETS x\SEMI
  s\GETS x\SEMI
  y\GETS r\SEMI
  z\GETS s
  \\[-1ex]
  \nonumber
  \hbox{\begin{tikzinline}[node distance=.5em and 1em]
      \event{a3}{\DR{x}{3}}{}
      \event{a4}{\DR{x}{4}}{right=of a3}
      \event{a5}{x{=}1\mid\DW{y}{1}}{right=of a4}
      \event{a6}{x{=}2\mid\DW{z}{2}}{right=of a5}
    \end{tikzinline}}
\end{gather*}
Note that reads that we delay must all be consistent.

Also note that we cannot have:
\begin{gather*}
  r\GETS x\SEMI a\GETS r\SEMI
  s\GETS x\SEMI b\GETS s\SEMI
  y\GETS r\SEMI
  z\GETS s
  \\[-1ex]
  \nonumber
  \hbox{\begin{tikzinline}[node distance=.5em and 1em]
      \event{a3}{\DR{x}{3}}{}
      \event{a4}{\DR{x}{4}}{right=of a3}
      \event{a5}{x{=}1\mid\DW{y}{1}}{right=of a4}
      \event{a6}{x{=}1\mid\DW{z}{2}}{right=of a5}
      \event{b3}{\DW{a}{3}}{below=of a3}
      \event{b4}{\DW{b}{4}}{below=of a4}
      \po{a3}{b3}
      \po{a4}{b4}
    \end{tikzinline}}
\end{gather*}
Because the following is not a pomset:
\begin{gather*}
  b\GETS s\SEMI
  y\GETS r\SEMI
  z\GETS s
  \\[-1ex]
  \nonumber
  \hbox{\begin{tikzinline}[node distance=.5em and 1em]
      \event{a5}{r{=}1\mid\DW{y}{1}}{right=of a4}
      \event{a6}{s{=}1\mid\DW{z}{2}}{right=of a5}
      \event{b4}{s{=}4\mid\DW{b}{4}}{below left=of a5}
    \end{tikzinline}}
\end{gather*}
But we can have the following, since there is no order the reads:
\begin{gather*}
  r_1\GETS x\SEMI
  s_1\GETS x\SEMI  
  r_2\GETS x\SEMI
  s_2\GETS x\SEMI
  y\GETS r_2\SEMI
  z\GETS s_2
  \\[-1ex]
  \nonumber
  \hbox{\begin{tikzinline}[node distance=.5em and 1em]
      \event{a1}{\DR{x}{1}}{}
      \event{a2}{\DR{x}{2}}{right=of a1}
      \event{a3}{\DR{x}{3}}{right=of a2}
      \event{a4}{\DR{x}{4}}{right=of a3}
      \event{a5}{\DW{y}{1}}{right=of a4}
      \event{a6}{\DW{z}{2}}{right=of a5}
      \po[out=15,in=165]{a1}{a5}
      \po[out=15,in=165]{a2}{a6}
    \end{tikzinline}}
\end{gather*}
Because this is indistinguishable from:
\begin{gather*}
  r_1\GETS x\SEMI
  s_1\GETS x\SEMI  
  r_2\GETS x\SEMI
  s_2\GETS x\SEMI
  y\GETS r_2\SEMI
  z\GETS s_2
  \\[-1ex]
  \nonumber
  \hbox{\begin{tikzinline}[node distance=.5em and 1em]
      \event{a1}{\DR{x}{3}}{}
      \event{a2}{\DR{x}{4}}{right=of a1}
      \event{a3}{\DR{x}{1}}{right=of a2}
      \event{a4}{\DR{x}{2}}{right=of a3}
      \event{a5}{\DW{y}{1}}{right=of a4}
      \event{a6}{\DW{z}{2}}{right=of a5}
      \po[out=15,in=165]{a3}{a5}
      \po[out=15,in=165]{a4}{a6}
    \end{tikzinline}}
\end{gather*}
which is the same as:
\begin{gather*}
  r_1\GETS x\SEMI
  r_2\GETS x\SEMI
  y\GETS r_2\SEMI
  s_1\GETS x\SEMI  
  s_2\GETS x\SEMI
  z\GETS s_2
  \\[-1ex]
  \nonumber
  \hbox{\begin{tikzinline}[node distance=.5em and 1em]
      \event{a1}{\DR{x}{1}}{}
      \event{a2}{\DR{x}{3}}{right=of a1}
      \event{a3}{\DW{y}{1}}{right=of a2}
      \event{a4}{\DR{x}{2}}{right=of a3}
      \event{a5}{\DR{x}{4}}{right=of a4}
      \event{a6}{\DW{z}{2}}{right=of a5}
      \po[out=15,in=165]{a1}{a3}
      \po[out=15,in=165]{a4}{a6}
    \end{tikzinline}}
\end{gather*}

But we can have:
\begin{gather*}
  p\GETS x\SEMI
  r\GETS x\SEMI
  s\GETS x\SEMI
  y\GETS r\SEMI
  z\GETS s
  \\[-1ex]
  \nonumber
  \hbox{\begin{tikzinline}[node distance=.5em and 1em]
      \event{a1}{\DR{x}{3}}{}
      \event{a2}{\DR{x}{4}}{right=of a1}
      \event{a3}{x{=}1\mid\DW{y}{1}}{right=of a2}
      \event{a4}{x{=}1\mid\DW{z}{1}}{right=of a3}
      \event{b2}{\DR{x}{1}}{left=of a1}
      \po[out=15,in=165]{b2}{a3}
      \po[out=-15,in=-165]{b2}{a4}
      %\po[out=15,in=165]{a1}{a3}
      %\po[out=15,in=165]{a2}{a4}
    \end{tikzinline}}
\end{gather*}

Reads can only swap when their values are interchangeable in the following
program.

\subsection{Commments for revision}

Can move prefix closure out of model and just put it in the logic section.

In semicolon semantics, join is asymmetric.

State theorem for the join-free fragment, since prefixing has no joins.

Our address calculation NO-TAR example should be replaced by JMM TC12.


\subsection{Alan comments}

\begin{verbatim}
  x=s; y=r; z=3s+2r

  x=s; y=r; z1=s; if(r odd){ z2=1} // using 1 and 3 as the reads
\end{verbatim}

\bibliography{bib}
\end{document}






































Alex Aiken PLDI keynote:
\begin{itemize}
\item PAD Machines: Parallel, Accelerated (GPUs..), Distributed (memory)
\item ``Registered'' memory: shared with network
\item ``Zero-copy'' memory: shared with GPUs
  \begin{itemize}
  \item
    \url{http://docs.nvidia.com/cuda/cuda-c-programming-guide/index.html#um-unified-memory-programming-hd}
    Unified Memory offers a “single-pointer-to-data” model that is
    conceptually similar to CUDA’s zero-copy memory. One key difference
    between the two is that with zero-copy allocations the physical location
    of memory is pinned in CPU system memory such that a program may have
    fast or slow access to it depending on where it is being accessed
    from. Unified Memory, on the other hand, decouples memory and execution
    spaces so that all data accesses are fast.
  \end{itemize}
\item Software manages ``NUMA domains''
\item In PAD machines, most computation happening on the accelerators.
\item Bad Programming models:
  \begin{itemize}
  \item Data Analytics, Machine Learning: Single program in Spark Hadoop,
    Etc.  Accessible to many people, Low performance
  \item HPC: MPI, OpenMP, Vector Intrinsics, Cuda, etc...  Experts only, High performance
  \end{itemize}
\item Task based models are good.
  \begin{itemize}
  \item Asynchronous functions that get mapped to real resources.  May have
    sub-tasks or not.  
  \item Tasks capture locality:
    \begin{itemize}
    \item Task is co-located with its arguments.  
    \item Tasks are a ``local address space'' model.
    \end{itemize}
  \item Tasks capture asynchrony
  \end{itemize}
\item Semantics
  \begin{itemize}
  \item \xmark Explicit Parallelism: X10, Chapel
  \item \cmark Implicit Parallelism (dataflow) --- much easier to reason about
    (sequential semantics) --- automated data movement
    \begin{itemize}
    \item Layout Explicitly programmed: StarPU 
    \item Layout computed statically: PaRSEC/PTG, DPJ, Sequoia
    \item Layout computed dynamically: Legion, PaRSEC/DTD, TensorFlow
    \end{itemize}
  \end{itemize}
\item Partitioning:
  \begin{itemize}
  \item Flat disjoint partitions: TensorFlow, PyTorch.
    \begin{itemize}
    \item Not compositional: Problem is that different parts of the
      application may want different views of the data.
    \end{itemize}
  \end{itemize}
\item Hierarchical: StarPU
  \begin{itemize}
  \item Partitions created/destroyed
  \item Partitions are disjoint
  \item Only one partition of collection at a time
  \item Only leaves can be used by application
  \item Eager Repartitioning move data unnecessarily
  \end{itemize}
\item Legion allows multiple hierarchical partitions
  \begin{itemize}
  \item Compositional: data movement is lazy
  \item Task dependence analysis requires alias analysis
  \end{itemize}
\end{itemize}


% Local Variables:
% mode: latex
% TeX-master: t
% End:
