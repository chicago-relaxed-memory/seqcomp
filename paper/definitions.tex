\section{The Basic Model}
\label{sec:model}

After some preliminaries (\textsection\ref{sec:prelim}--\ref{sec:actions}),
we define the basic model and establish some basic properties
(\textsection\ref{sec:pomsets} and \reffig{fig:sem}).  We then explain the
model using examples (\textsection\ref{sec:ex:pomset}--\ref{sec:ex:tau}) and
discuss program transformations (\textsection\ref{sec:valid}).  We encourage
readers to skip to the examples, coming back as needed.

%% Batty suggest example where dependencies are added and also go away, perhaps
%% by store forwarding. Something like:
%% \texttt{(r=x; y=1); (s=y; z=s+r)}

% In this section, we present the mathematical preliminaries for the
% model (which can be skipped on first reading). We then present the
% model incrementally, starting with a model built using
% \emph{partially ordered multisets}
% (\emph{pomsets})~\cite{GISCHER1988199,Plotkin:1997:TSP:266557.266600},
% and then adding preconditions and finally predicate transformers.

% In later sections, we will discuss extensions to the logic, and to the
% semantics of load, store and thread initialization, in order to model
% relaxed memory more faithfully. We stress that these features do
% \emph{not} change any of the structures of the language: conditionals,
% parallel composition, and sequential composition are as defined in this section.

\subsection{Preliminaries}
\label{sec:prelim}
The syntax is built from
\begin{itemize}
\item a set of \emph{values} $\Val$, ranged over by
  $\aVal$, $\bVal$, $\cVal$, $\dVal$,
\item a set of \emph{registers} $\Reg$, ranged over by
  $\aReg$, $\bReg$,
\item a set of \emph{expressions} $\Exp$, ranged over by
  $\aExp$, $\bExp$,  $\cExp$.
\end{itemize}

\emph{Memory references} are tagged values, written $\REF{\cVal}$.  Let $\Loc$
be the set of memory references, ranged over by $\aLoc$, $\bLoc$, $\cLoc$.
%
We require that
\begin{itemize}
\item values and registers are disjoint, 
\item values include at least the constants $0$ and $1$,  
\item expressions include at least registers and values, 
\item expressions do \emph{not} include references: $\aExp[\bExp/\aLoc]=\aExp$.
\end{itemize}

We model the following language.
\begin{gather*}
  \begin{aligned}
    \amode \BNFDEF& \mRLX
    \BNFSEP \mRA 
    \BNFSEP \mSC
    &\mkern100mu
    \fmode \BNFDEF& \fACQ 
    \BNFSEP \fREL
    \BNFSEP \fSC
  \end{aligned}
  \\
  \aCmd
  \BNFDEF \LET{\aReg}{\aExp}
  \BNFSEP \PR[\amode]{\REF{\cExp}}[\ascope]{\aReg}
  \BNFSEP \PW[\amode]{\REF{\cExp}}[\ascope]{\aExp}
  \BNFSEP \PF[\ascope]{\fmode}
  \BNFSEP \SKIP
  \BNFSEP \aCmd_1 \SEMI \aCmd_2
  \BNFSEP \IF{\aExp} \THEN \aCmd_1 \ELSE \aCmd_2 \FI
  \BNFSEP \aCmd_1 \LPAR[\bThrd] \aCmd_2
  % \\[-.5ex]
  % \BNFSEP& \PCAS[\amode_1][\amode_2]{\REF{\cExp}}[\ascope]{\aReg}{\aExp}{\bExp}
  % \BNFSEP \PFADD[\amode_1][\amode_2]{\REF{\cExp}}[\ascope]{\aReg}{\aExp}
  % \BNFSEP \PEXCHG[\amode_1][\amode_2]{\REF{\cExp}}[\ascope]{\aReg}{\aExp}
\end{gather*}

\emph{Memory modes}, $\amode$, are {relaxed} ($\mRLX$), {release-acquire}
($\mRA$), and {sequentially consistent} ($\mSC$).  Relaxed mode is the
default; we regularly elide it from examples.  $\mRA$/$\mSC$ accesses are
collectively known as \emph{synchronized accesses}.  

\emph{Fence modes}, $\bmode$, are {acquire} ($\fACQ$), {release} ($\fREL$), 
and {acquire-release} ($\fSC$).  

\emph{Commands}, aka \emph{statements}, $\aCmd$, include memory accesses at a
given mode, as well as the usual structural constructs. Following
\cite{DBLP:conf/icfp/FerreiraHJ96}, $\LPAR$ denotes parallel composition,
preserving thread state on the left after a join.  In examples and
sublanguages without join, we use the symmetric $\PAR$ operator.

Throughout \textsection\ref{sec:intro}--\ref{sec:arm} we 
require that
\begin{itemize}
\item each register is assigned at most once in a program.
  % \end{itemize}
  % In \textsection\ref{sec:complications} and following, we
  % require instead that
  % \begin{itemize}
  % \end{itemize}
\end{itemize}
In \textsection\ref{sec:additional}, we drop this restriction, requiring
instead that
\begin{itemize}
\item there are registers
  $\uRegs{\AllEvents}=\{\uReg{\aEv}\mid\aEv\in\AllEvents\}$, that do not
  appear in programs: $\aCmd[\bExp/\uReg{\aEv}]=\aCmd$.
\end{itemize}
% In contexts that make no use of $\uRegs{\AllEvents}$, we make the first
% assumption.

The semantics is built from the following.
\begin{itemize}
\item a set of \emph{events} $\AllEvents$, ranged over by $\aEv$, $\bEv$,
  $\cEv$, %$\dEv$,
  and subsets ranged over by $\aEvs$, $\bEvs$, $\cEvs$,  
\item a set of \emph{logical formulae} $\Formulae$, ranged over by $\aForm$,
  $\bForm$, $\cForm$,
\item a set of \emph{actions} $\Act$, ranged over by $\aAct$, $\bAct$.
\end{itemize}
%$\dEvs$.

We require that
\begin{itemize}
\item formulae include $\TRUE$, $\FALSE$ and the equalities $(\aExp{=}\bExp)$ and $(\aLoc{=}\aExp)$,
\item formulae are closed under $\lnot$, $\land$, $\lor$, $\limplies$, and
  substitutions $[\aExp/\aReg]$, $[\aExp/\aLoc]$,
\item there is a relation $\rimpliesdef$ between
  formulae, capturing entailment, %\subseteq(\Formulae\times\Formulae)$ %
\item $\rimpliesdef$ has the expected semantics for $=$, $\lnot$,
  $\land$, $\lor$, $\limplies$ and substitutions $[\aExp/\aReg]$, $[\aExp/\aLoc]$,
\item there are three binary relations over $\Act\times\Act$:
  $\rmatchesdef$, $\rblocksdef$, and $\rdelaysdef$,
\item there are two subsets of $\Act$, distinguishing
  $\sreaddef$ and $\sreleasedef$ actions.
\end{itemize}

Logical formulae include equations over registers and memory references, such as
$(\aReg{=}\bReg{+}1)$ and $(\aLoc{=}1)$.
%For use in \textsection\ref{sec:tc1}, we also include equations over memory references, such as $(\aLoc{=}1)$.
% I would like to drop this, an be careful about program vs logical syntax
We use expressions as formulae, coercing $\aExp$ to $\aExp{\neq}0$.
% Equations have precedence over logical operators; thus
% $\aReg{=}\aVal\limplies\bReg{>}\bVal$ is read
% $(\aReg{=}\aVal)\limplies(\bReg{>}\bVal)$.  As usual, implication associates to the
% right; thus $\aForm\limplies\bForm\limplies\cForm$ is read
% $\aForm\limplies(\bForm\limplies\cForm)$.
As usual, implication associates to the right; thus
$\aReg{=}\aVal\limplies\bReg{>}\bVal\limplies\bForm$ is read
$(\aReg{=}\aVal)\limplies((\bReg{>}\bVal)\limplies\bForm)$.

% Formulae are subject to substitutions; % of the form $[\aExp/\aReg]$ and
% % $[\aExp/\aLoc]$;
% actions are not.

% \begin{definition}
%   \label{def:independent}
%   We say $\aForm$ is \emph{independent of $\aLoc$} if, for every
%   $\aVal$, $\aForm \vDash \aForm[\aVal/\aLoc] \vDash \aForm$; it is
%   \emph{dependent} otherwise.
%   We say $\aForm$ is \emph{location independent} if it is independent of
%   every location.
%   We say
%   $\aForm$ \emph{implies} $\bForm$ if $\aForm\vDash\bForm$.
%   We say that
We say
$\aForm$ is a \emph{tautology} if $\TRUE \rimplies \aForm$.
% We say that
We say
$\aForm$ is \emph{unsatisfiable} if $\aForm \rimplies \FALSE$.
% \end{definition}




\subsection{Actions in This Paper}
\label{sec:actions}
In this paper, we let actions be reads and writes and fences:
\begin{displaymath}
  \aAct,\bAct \BNFDEF \DW[\amode]{\aLoc}[\ascope]{\aVal}[\aThrd]
  \BNFSEP \DR[\amode]{\aLoc}[\ascope]{\aVal}[\aThrd]
  \BNFSEP \DF[\ascope]{\fmode}[\aThrd]
\end{displaymath}
% % writes $\DWP[\amode]{\aLoc}[\ascope]{\aVal}[\aThrd]$,
% reads $\DRP[\amode]{\aLoc}[\ascope]{\aVal}[\aThrd]$,
% and fences $\DFP[\ascope]{\fmode}[\aThrd]$.

We use shorthand when referring to actions.  In definitions, we drop elements
of actions that are existentially quantified.  In examples, we drop elements
of actions, using defaults.
% We write $\DXP[\amode]{}[\ascope]{}[\aThrd]$ to
% stand for $\DWP[\amode]{}[\ascope]{}[\aThrd]$ or
% $\DRP[\amode]{}[\ascope]{}[\aThrd]$.
%
Let $\lemode$ be the least order over access and fence modes such that
$\mRLX\lemode\mRA\lemode\mSC$ and $\fREL\lemode\fSC$ and $\fACQ\lemode\fSC$.
We write $\DWP[\gemode\mRA]{}{}$ to stand for either $\DWP[\mRA]{}{}$ or
$\DWP[\mSC]{}{}$, and similarly for the other actions and modes.

% We also define shorthand for sets of
% actions using an order on access and fence modes:  
% \begin{align*}
%   \begin{tikzcenter}
%     \node (rlx) at (0, 0) {$\mathstrut\mRLX$};
%     \node (ra)  at (1, 0) {$\mathstrut\mRA$};
%     \node (sc)  at (2, 0) {$\mathstrut\mSC$};
%     \draw[->](rlx)to(ra);
%     \draw[->](ra)to(sc);
%   \end{tikzcenter}
%   &&
%   \begin{tikzcenter}
%     \node (fsc) at (3, 0) {$\mathstrut\fSC$};
%     \node (rel) at (2, -0.2) {$\mathstrut\fREL$};
%     \node (acq) at (2,  0.2) {$\mathstrut\fACQ$};
%     \draw[->](rel)to(fsc);
%     \draw[->](acq)to(fsc);
%   \end{tikzcenter}
% \end{align*}
% We write $\amode\lemode\bmode$ for this order.
%Let $\amode\lubmode\bmode$ denote the least upper bound of $\amode$ and $\bmode$.

% or $\DFP[\ascope]{\amode}[\aThrd]$.
% We write $\DWP[\gemode\mREL]{}{}$ to stand for either
% $\DWP[\mREL]{}{}$ or $\DWP[\mSC]{}{}$, and similarly for other actions and
% modes.

\begin{definition}
  \label{def:actions}
  Actions $\DRP{}{}$ are $\sreaddef$ actions.
  Actions $\DWP[\gemode\mREL]{}{}$ and $\DFP{\gemode\fREL}$ are
  $\sreleasedef$ actions.

  We say $\aAct \rmatchesdef \bAct$ if $\aAct=\DWP{\aLoc}{\aVal}$ and $\bAct=\DRP{\aLoc}{\aVal}$.
  % We say $\DWP{\aLoc}{\aVal} \rmatchesdef \DRP{\aLoc}{\bVal}$ when $\aVal=\bVal$.
  % Action $\DWP{\aLoc}{\aVal} \rmatchesdef \DRP{\aLoc}{\bVal}$ when $\aVal=\bVal$.

  We say $\aAct \rblocksdef \bAct$ if $\aAct=\DWP{\aLoc}{}$ and $\bAct=\DRP{\aLoc}{}$, regardless of value.
  % We say $\DWP{\aLoc}{\aVal} \rblocksdef \DRP{\aLoc}{\bVal}$, for any $\aVal$, $\bVal$.
  % Action $\DWP{\aLoc}{\aVal} \rblocksdef \DRP{\aLoc}{\bVal}$, for any $\aVal$, $\bVal$.

  % Let two actions \emph{overlap} if they access the same location.
  % We say $\aAct \roverlapsdef \bAct$ if they access the same location.

  We say $\aAct \rdelaysdef \bAct$ if $\aAct\eqreorderco\bAct$ or $\aAct\reorderra\bAct$ or $\aAct\eqreordersc\bAct$.

  Let ${\eqreorderco}$ capture write-write, read-write coherence:
  \begin{math}
    {\eqreorderco}
    =
    \{(\DW{\aLoc}{}\Cb \DW{\aLoc}{})\Cc(\DR{\aLoc}{}\Cb \DW{\aLoc}{})\Cc(\DW{\aLoc}{}\Cb \DR{\aLoc}{})\}
  \end{math}.

  Let ${\reorderra}$ capture order due to synchronization:
  \begin{math}
    {\reorderra}
    =
    \{(\aAct\Cb             \DW[\gemode\mREL]{}{}     )\Cc
    (\aAct\Cb               \DF{\gemode\fREL}        )\Cc
    (\DR{}{}\Cb             \DF{\gemode\fACQ}        )\Cc
    (\DR{\aLoc}{}\Cb        \DR[\gemode\mACQ]{\aLoc}{})\Cc
    (\DR[\gemode\mACQ]{}{}\Cb\aAct                    )\Cc
    (\DF{\gemode\fACQ}\Cb   \aAct                    )\Cc
    (\DF{\gemode\fREL}\Cb   \DW{}{}                  )\Cc
    (\DW[\gemode\mREL]{\aLoc}{}\Cb\DW{\aLoc}{})\}
  \end{math}.

  Let ${\eqreordersc}$ capture order due to $\mSC$ access:
  \begin{math}
    {\eqreordersc}
    =
    %\{(\DX[\mSC]{}{}\Cb\DX[\mSC]{}{})\}
    \{(\DW[\mSC]{}{}\Cb \DW[\mSC]{}{})\Cc(\DR[\mSC]{}{}\Cb \DW[\mSC]{}{}) \Cc(\DW[\mSC]{}{}\Cb \DR[\mSC]{}{})\Cc(\DR[\mSC]{}{}\Cb \DR[\mSC]{}{})\}
  \end{math}.
\end{definition}

\subsection{Definition: Pomsets with Predicate Transformers}
\label{sec:pomsets}

\emph{Predicate transformers} are functions on formulae which preserve
logical structure, providing a natural model of sequential composition.


\begin{definition}
  \label{def:trans}
  A \emph{predicate transformer} is a %monotone
  function
  $\aTr{}{}:\Formulae\fun\Formulae$ such that
  \begin{multicols}{2}
    \begin{enumerate}[,label=(\textsc{x}\arabic*),ref=\textsc{x}\arabic*]
    \item \label{tr-false}
      $\aTr{}{\FALSE}$ is $\FALSE$,    
    \item \label{tr-and}
      $\aTr{}{\bForm_1\land\bForm_2}$ is $\aTr{}{\bForm_1}\land\aTr{}{\bForm_2}$,    
    \item \label{tr-or}
      $\aTr{}{\bForm_1\lor\bForm_2}$ is $\aTr{}{\bForm_1}\lor\aTr{}{\bForm_2}$, 
    \item \label{tr-implies}
      if $\aForm \rimplies \bForm$, then $\aTr{}{\aForm} \rimplies
      \aTr{}{\bForm}$.
    \end{enumerate}
  \end{multicols}
\end{definition}
\noindent
The definition follows \citet{DBLP:journals/cacm/Dijkstra75}.  Note that
substitutions ($\aTr{}{\psi}=\psi[\aExp/\aReg]$ and
$\aTr{}{\psi}=\psi[\aExp/\aLoc]$) and implications on the right
($\aTr{}{\psi}=\phi\limplies\psi$) are predicate transformers.

As discussed in \S\ref{sec:intro}, predicate transformers suffice for sequentially consistent
models, but not relaxed models, where dependency calculation is crucial.
For dependency calculation, we use a \emph{family} of predicate transformers,
indexed by sets of events. We use $\aTr{\bEvs}{}$ as the predicate transformer
applied to any event $\aEv$ where if $\bEv\in\bEvs$ then $\bEv<\aEv$.

\begin{definition}
  \label{def:family}
  A \emph{family of predicate transformers} for $\aEvs$ consists of a
  predicate transformer $\aTr{\bEvs}{}$ for each $\bEvs\subseteq\AllEvents$,
  such that if $\cEvs \cap \aEvs \subseteq \bEvs$ then $\aTr{\cEvs}{\bForm}
  \rimplies \aTr{\bEvs}{\bForm}$.

  We write $\aTr{}{}$ as an abbreviation of $\aTr{\aEvs}{}$.
\end{definition}

\begin{definition}
  \label{def:pomset}
  A \emph{pomset with predicate transformers} over $\Act$
  is a tuple $(\aEvs, \labeling, \labelingForm, \aTr{}{}, \aTerm, {\rrfx}, {\le})$ where
  \begin{enumerate}[,label=(\textsc{m}\arabic*),ref=\textsc{m}\arabic*]
  \item \label{pom-E}
    $\Event\subseteq\AllEvents$ is a set of \emph{events},
  \item \label{pom-lambda}
    $\labeling: \Event \fun \Act$ defines a \emph{label} for each event,
  \item \label{pom-kappa}
    $\labelingForm:\aEvs\fun\Formulae$ defines a \emph{precondition} for each event,
  \item \label{pom-tau}
    $\aTr{}{}:2^{\AllEvents}\fun\Formulae \fun\Formulae$ is a \emph{family of
      predicate transformers} over $\aEvs$, %defines a \emph{predicate transformer} for each set of events,
  \item \label{pom-term}
    $\aTerm:\Formulae$ defines a \emph{termination condition},
  \item \label{pom-rf}
    ${\rrfx} : \Event\fun\Event$ is an injective relation capturing
    \emph{reads-from} such that 
    % \end{enumerate}
    % A pomset is a \emph{candidate} if there is an injective relation
    % ${\rrfx} : \Event\times\Event$, capturing \emph{reads-from}, such that
    \begin{enumerate}
      % \begin{enumerate}[,label=(\textsc{c}\arabic*),ref=\textsc{c}\arabic*]
      % \item \label{rf-injective}
      %   if $\bEv\xrfx\aEv$ and $\cEv\xrfx\aEv$ then $\bEv=\cEv$, that is, ${\rrfx}$ is injective,
    \item \label{rf-match}
      if $\bEv\xrfx\aEv$ then $\labeling(\bEv) \rmatches \labeling(\aEv)$,
      % \item \label{rf-block}
      %   if $\bEv\xrfx\aEv$ and $\labeling(\cEv) \rblocks \labeling(\aEv)$ then either $\cEv\le\bEv$ or $\aEv\le\cEv$.
      % \item \label{rf-le}
      %   if $\bEv\xrfx\aEv$ then $\bEv\le\aEv$.
    \end{enumerate}
  \item \label{pom-le}
    ${\le} : \Event\times\Event$, is a partial order capturing
    \emph{causality}, such that
    \begin{enumerate}
      \item \label{rf-block}
        if $\bEv\xrfx\aEv$ and $\labeling(\cEv) \rblocks \labeling(\aEv)$ then either $\cEv\le\bEv$ or $\aEv\le\cEv$.
    \end{enumerate}
  \end{enumerate}
  A pomset is \emph{top-level} if \labeltext[\textsc{t}1]{(\textsc{t}1)}{top-term} $\aTerm$ is a tautology and \labeltext[\textsc{t}2]{(\textsc{t}2)}{top-ev} for every $\aEv\in\aEvs$,
  % $\labelingForm(\aEv)$ is a tautology and
  % if $\labeling(\aEv)$ is a $\sread$ then there is some $\bEv\xrfx\aEv$.
  %\begin{multicols}{2}
    \begin{enumerate}[label=(\textsc{t}\arabic*),ref=\textsc{t}\arabic*]
      \setcounter{enumi}{2}
    \item[]
      \begin{enumerate}[leftmargin=0pt]
      \item \label{top-kappa}
        $\labelingForm(\aEv)$ is a tautology,    
      \item \label{top-rf}
        if $\labeling(\aEv)$ is a $\sread$ then there is some $\bEv\xrfx\aEv$.
      \end{enumerate}
    \end{enumerate}
  %\end{multicols}
\end{definition}
\begin{figure}
  \bookmark{Semantics of programs}
  \raggedright
  \noindent
  Suppose $\aRel_1:\aEvs_1\times\aEvs_1$ and $\aRel_2:\aEvs_2\times\aEvs_2$.

  We say $\aRel \rextendsdef{\aRel_1}{\aRel_2}$ if
  $\aRel\supseteq (\aRel_1\cup \aRel_2)$ and
  $\aRel\cap(\aEvs_1\times \aEvs_1) = \aRel_1$ and
  $\aRel\cap(\aEvs_2\times \aEvs_2) = \aRel_2$.
  \medskip
  
  \noindent
  If $\aPS\in\sSKIP$ then $\aEvs = \emptyset$ and
  $\aTr{\bEvs}{\bForm} \rimplies \bForm$.
  \medskip

  \noindent
  If $\aPS \in \sLPAR{\aPSS_1}{\aPSS_2}$ then  
  $(\exists\aPS_1\in\aPSS_1)$ $(\exists\aPS_2\in\aPSS_2)$
  \begin{multicols}{2}
    \begin{enumerate}[topsep=0pt,label=(\textsc{p}\arabic*),ref=\textsc{p}\arabic*]
    \item \label{par-E}
      $\aEvs = (\aEvs_1\uplus\aEvs_2)$,
    \item \label{par-lambda}
      ${\labeling}=\PBR{{\labeling_1}\cup {\labeling_2}}$, 
      \stepcounter{enumi}
    \item[] \labeltext[\textsc{p}3]{}{par-kappa}
      \begin{enumerate}[leftmargin=0pt]
      \item \label{par-kappa1}
        if $\aEv\in\aEvs_1$ then $\labelingForm(\aEv) \rimplies \labelingForm_1(\aEv)$,
      \item \label{par-kappa2}
        if $\aEv\in\aEvs_2$ then $\labelingForm(\aEv) \rimplies \labelingForm_2(\aEv)$,
      \end{enumerate}
    \item \label{par-tau}
      $\aTr{\bEvs}{\bForm} \rimplies \aTr[1]{\bEvs}{\bForm}$,
    \item \label{par-term}
      $\aTerm \rimplies \aTerm[1]\land\aTerm[2]$,
    \item \label{par-rf}
      ${\rrfx}\rextends{\rrfx_1}{\rrfx_2}$,
      \stepcounter{enumi}
    \item[] \labeltext[\textsc{p}7]{}{par-le}
      \begin{enumerate}[leftmargin=0pt]
      \item \label{par-le-extends}
        ${\le}\rextends{\le_1}{\le_2}$,
      \item \label{par-le-rf}
        % ${\rrfx}\subseteq\PBR{{\rrfx_1}\cup{\rrfx_2}\cup{\le}}$.
        % if $\bEv\xrfx\aEv$ then either $\bEv\xrfx_1\aEv$, $\bEv\xrfx_2\aEv$, or $\bEv\le\aEv$.
        if $\bEv\in\aEvs_1$, $\aEv\in\aEvs_2$ and $\bEv\xrfx\aEv$ then $\bEv\le\aEv$.
      \end{enumerate}
    \end{enumerate}
  \end{multicols}
  \medskip

  \noindent
  If $\aPS \in \sSEMI{\aPSS_1}{\aPSS_2}$ then $(\exists\aPS_1\in\aPSS_1)$
  $(\exists\aPS_2\in\aPSS_2)$\\
  let $\labelingForm'_2(\aEv)=\aTr[1]{\Cdown{\aEv}}{\labelingForm_2(\aEv})$,
  where
  $\Cdown{\aEv}=\{ \cEv \mid \cEv \lt \aEv \}$%, if $\labeling(\aEv)$ is a write, and $\Cdown{\aEv}=\aEvs_1$, otherwise
  % \begin{math}
  %   \Cdown{\aEv}=
  %   \begin{cases}
  %     \{ \cEv \mid \cEv \lt \aEv \} & \textif \labeling(\aEv) \;\text{is a write}
  %     \\
  %     \aEvs_1 & \textotherwise
  %   \end{cases}
  % \end{math}
  % let
  % \begin{math}
  %   \smash{
  %   \downarrow\aEv=\begin{cases}
  %     \{ \cEv \mid \cEv \lt \aEv \} &\textif \labeling(\aEv) \;\text{is a write}\\
  %     \aEvs_1 &\textotherwise
  %   \end{cases}}
  % \end{math}
  \begin{multicols}{2}
    \begin{enumerate}[topsep=0pt,label=(\textsc{s}\arabic*),ref=\textsc{s}\arabic*]
    \item \label{seq-E}
      $\aEvs = (\aEvs_1\cup\aEvs_2)$,
    \item \label{seq-lambda}
      ${\labeling}=\PBR{{\labeling_1}\cup {\labeling_2}}$, 
      \stepcounter{enumi}
    \item[] \labeltext[\textsc{s}3]{}{seq-kappa}
      \begin{enumerate}[leftmargin=0pt]
      \item \label{seq-kappa1}
        if $\aEv\in\aEvs_1\setminus\aEvs_2$ then $\labelingForm(\aEv) \rimplies \labelingForm_1(\aEv)$,
      \item \label{seq-kappa2}
        if $\aEv\in\aEvs_2\setminus\aEvs_1$ then $\labelingForm(\aEv) \rimplies \labelingForm'_2(\aEv)$,
      \item \label{seq-kappa12}
        % if $\aEv{\in}\aEvs_1{\cap}\aEvs_2$ then $\labelingForm(\aEv) \rimplies \labelingForm_1(\aEv){\lor}\labelingForm'_2(\aEv)$,\\
        if $\aEv\in\aEvs_1\cap\aEvs_2$ then $\labelingForm(\aEv) \rimplies \labelingForm_1(\aEv)\lor\labelingForm'_2(\aEv)$,
        % where
        % $\labelingForm'_2(\aEv)=\aTr[1]{\Cdown{\aEv}}{\labelingForm_2(\aEv})$,
        % where $\Cdown{\aEv}=\{ \cEv \mid \cEv \lt \aEv \}$ if $\labeling(\aEv)$
        % is a write, and $\Cdown{\aEv}=\aEvs_1$, otherwise,
      \item \label{seq-kappa-release}
        if $\labeling_2(\aEv)$ is a release then $\labelingForm(\aEv) \rimplies \aTerm[1]$,
      \end{enumerate}
    \item \label{seq-tau}
      $\aTr{\bEvs}{\bForm} \rimplies \aTr[1]{\bEvs}{\aTr[2]{\bEvs}{\bForm}}$,
    \item \label{seq-term}
      $\aTerm \rimplies \aTerm[1]\land\aTr[1]{}{\aTerm[2]}$,
    \item \label{seq-rf}
      ${\rrfx}\rextends{\rrfx_1}{\rrfx_2}$,
      \stepcounter{enumi}
    \item[] \labeltext[\textsc{s}7]{}{seq-le}
      \begin{enumerate}[leftmargin=0pt]
      \item \label{seq-le-extends}
        ${\le}\rextends{\le_1}{\le_2}$, 
      \item \label{seq-le-rf}
        if $\bEv\in\aEvs_1$, $\aEv\in\aEvs_2$ and $\bEv\xrfx\aEv$ then $\bEv\le\aEv$,
      \item \label{seq-le-delays}
        if $\labeling_1(\bEv) \rdelays \labeling_2(\aEv)$ then $\bEv\le\aEv$.
      \end{enumerate}
      %   % if $\cEv\in\aEvs_1$ and $(\bEv,\aEv)\in\PBR{{\rrfx}\cap(\aEvs_1\times \aEvs_2)}$ then 
      %   % either $\labeling(\cEv) \reorderca \labeling(\bEv)$
      %   % or $\cEv\le\bEv$,
    \end{enumerate}
  \end{multicols}
  \medskip

  \noindent
  If $\aPS \in \sIF{\aForm}\sTHEN\aPSS_1\sELSE\aPSS_2\sFI$ then
  $(\exists\aPS_1\in\aPSS_1)$ $(\exists\aPS_2\in\aPSS_2)$
  \begin{multicols}{2}
    \begin{enumerate}[topsep=0pt,label=(\textsc{c}\arabic*),ref=\textsc{c}\arabic*]
    \item \label{if-E}
      $\aEvs = (\aEvs_1\cup\aEvs_2)$,
    \item \label{if-lambda}
      ${\labeling}=\PBR{{\labeling_1}\cup {\labeling_2}}$, 
      % \item[(\ref{par-E}--\ref{par-le})] as for $\sLPAR{}{}$,
      \stepcounter{enumi}
    \item[] \labeltext[\textsc{c}3]{}{if-kappa}
      \begin{enumerate}[leftmargin=0pt]
      \item \label{if-kappa1}
        if $\aEv\in\aEvs_1\setminus\aEvs_2$ then $\labelingForm(\aEv) \rimplies \aForm\land\labelingForm_1(\aEv)$,
      \item \label{if-kappa2}
        if $\aEv\in\aEvs_2\setminus\aEvs_1$ then $\labelingForm(\aEv) \rimplies \neg\aForm\land\labelingForm_2(\aEv)$, 
      \item \label{if-kappa12}
        if $\aEv\in\aEvs_1\cap\aEvs_2$\\ then
        % $\labelingForm(\aEv) \rimplies (\aForm\limplies\labelingForm_1(\aEv))\land(\neg\aForm\limplies\labelingForm_2(\aEv))$,
        $\labelingForm(\aEv) \rimplies (\aForm\land\labelingForm_1(\aEv))\lor(\neg\aForm\land\labelingForm_2(\aEv))$,
      \end{enumerate}
    \item \label{if-tau}
      % $\aTr{\bEvs}{\bForm} \rimplies (\aForm\limplies\aTr[1]{\bEvs}{\bForm})\land(\neg\aForm\limplies\aTr[2]{\bEvs}{\bForm})$,
      $\aTr{\bEvs}{\bForm} \rimplies (\aForm\land\aTr[1]{\bEvs}{\bForm})\lor(\neg\aForm\land\aTr[2]{\bEvs}{\bForm})$,
    \item \label{if-term}
      % $\aTerm \rimplies (\aForm\limplies\aTerm[1])\land(\neg\aForm\limplies\aTerm[2])$.
      $\aTerm \rimplies (\aForm\land\aTerm[1])\lor(\neg\aForm\land\aTerm[2])$.
      \stepcounter{enumi}
    \item[] \labeltext[\textsc{c}6]{}{if-rf}
      \begin{enumerate}[leftmargin=0pt]
      \item \label{if-rf-extends}
        ${\rrfx}\rextends{\rrfx_1}{\rrfx_2}$,
      \item \label{if-le-subset}
        ${\rrfx}\rsubset{\rrfx_1}{\rrfx_2}$,
      \end{enumerate}
      \stepcounter{enumi}
    \item[] \labeltext[\textsc{c}7]{}{if-le}
      \begin{enumerate}[leftmargin=0pt]
        % \item \label{if-le-rf}
        %   if $\bEv\in\aEvs_1$ and $\aEv\in\aEvs_2$ and $\bEv\xrfx\aEv$ then $\bEv\le\aEv$,
      \item \label{if-le-extends}
        ${\le}\rextends{\le_1}{\le_2}$,
      \item \label{if-le-subset}
        ${\le}\rsubset{\le_1}{\le_2}$.
      \end{enumerate}
    \end{enumerate}
  \end{multicols}
  \medskip

  \noindent
  If $\aPS\in\sLET{\aReg}{\aExp}$ then $\aEvs = \emptyset$ and
  $\aTr{\bEvs}{\bForm} \rimplies \bForm[\aExp/\aReg]$.
  \medskip

  \noindent
  If $\aPS \in \sLOAD[\amode]{\aReg}[\ascope]{\aLoc}[\aThrd]$ then
  $(\exists\aVal\in\Val)$
  \begin{multicols}{2}
    \begin{enumerate}[topsep=0pt,label=(\textsc{r}\arabic*),ref=\textsc{r}\arabic*]
    \item \label{read-E}
      if $\bEv,\aEv\in\aEvs$ then $\bEv=\aEv$,
    \item \label{read-lambda}
      $\labelingAct(\aEv) = \DR[\amode]{\aLoc}[\ascope]{\aVal}[\aThrd]$,
      \stepcounter{enumi}
      \stepcounter{enumi}
    \item[] \labeltext[\textsc{r}4]{}{read-tau}
      \begin{enumerate}[leftmargin=0pt]
      \item \label{read-tau-dep}
        if $(\aEvs\cap\bEvs)\neq\emptyset$ then
        \begin{math}
          \aTr{\bEvs}{\bForm} \rimplies
          \aVal{=}\aReg
          \limplies \bForm
        \end{math},    
      \item \label{read-tau-ind}
        if $\aEvs\neq\emptyset$ and $(\aEvs\cap\bEvs)=\emptyset$ then
        \begin{math}
          \aTr{\bEvs}{\bForm} \rimplies
          \PBR{\aVal{=}\aReg \lor \aLoc{=}\aReg} \limplies
          \bForm.
        \end{math}
      \item \label{read-tau-empty}
        if $\aEvs=\emptyset$ then
        \begin{math}
          \aTr{\bEvs}{\bForm} \rimplies
          % \PBR{\aVal{=}\aReg \lor \aLoc{=}\aReg} \limplies
          \bForm.
        \end{math}
      \end{enumerate}
    \end{enumerate}
  \end{multicols}
  \medskip

  \noindent
  If $\aPS \in \sSTORE[\amode]{\aLoc}[\ascope]{\aExp}[\aThrd]$ then
  $(\exists\aVal\in\Val)$
  \begin{multicols}{2}
    \begin{enumerate}[topsep=0pt,label=(\textsc{w}\arabic*),ref=\textsc{w}\arabic*]
    \item \label{write-E}
      if $\bEv,\aEv\in\aEvs$ then $\bEv=\aEv$,
    \item \label{write-lambda}
      $\labelingAct(\aEv) = \DW[\amode]{\aLoc}[\ascope]{\aVal}[\aThrd]$,
    \item \label{write-kappa}
      \begin{math}
        \labelingForm(\aEv) \rimplies
        \aExp{=}\aVal
      \end{math},    
    \item \label{write-tau}
      % if $(\aEvs\cap\bEvs)\neq\emptyset$ then
      \begin{math}
        \aTr{\bEvs}{\bForm} \rimplies 
        \bForm
        [\aExp/\aLoc]
        % \land\aExp{=}\aVal
      \end{math},
      \stepcounter{enumi}      
    \item[] \labeltext[\textsc{w}5]{}{write-term}
      \begin{enumerate}[leftmargin=0pt]
      \item \label{write-term-nonempty}
        if $\aEvs\neq\emptyset$ then $\aTerm \rimplies \aExp{=}\aVal$,
      \item \label{write-term-empty}
        if $\aEvs=\emptyset$ then $\aTerm \rimplies \FALSE$.
      \end{enumerate}
    \end{enumerate}
  \end{multicols}
  \medskip

  \noindent
  If $\aPS \in \sFENCE[\ascope]{\amode}[\aThrd]$ then
  % $(\exists\aLocs\subseteq\Loc)$
  % $(\exists\bmode\in\{\amode,\mRLX\})$
  \begin{multicols}{2}
    \begin{enumerate}[topsep=0pt,label=(\textsc{f}\arabic*),ref=\textsc{f}\arabic*]
    \item \label{fence-E}
      % $\aEvs\neq\emptyset$ and
      if $\bEv,\aEv\in\aEvs$ then $\bEv=\aEv$,
    \item \label{fence-lambda}
      $\labelingAct(\aEv) = \DF[\ascope]{\amode}[\aThrd]$,
      % \item%[{\labeltext[F3]{(F3)}{F3}}]
      %   $\labelingForm(\aEv) \rimplies \bigwedge_{\aLoc\in\aLocs}\Dx{\aLoc}$,
      \stepcounter{enumi}
    \item \label{fence-tau}
      $\aTr{\bEvs}{\bForm} \rimplies \bForm$,
    \item \label{fence-term}
      if $\aEvs=\emptyset$ then $\aTerm \rimplies \FALSE$.
    \end{enumerate}
  \end{multicols}

  \begin{align*}
    \begin{aligned}
      \sembase{\LET{\aReg}{\aExp}} &= \sLET{\aReg}{\aExp}
      \\
      \sembase{\PR[\amode]{\aLoc}{\aReg}} &= \sLOAD[\amode]{\aReg}[\ascope]{\aLoc}[\aThrd]
      \\
      \sembase{\PW[\amode]{\aLoc}{\aExp}} &= \sSTORE[\amode]{\aLoc}[\ascope]{\aExp}[\aThrd]
      \\
      \sembase{\PF[\ascope]{\fmode}} &= \sFENCE[\ascope]{\fmode}[\aThrd]
    \end{aligned}
    &&
    \begin{aligned}
      \sembase{\SKIP} &= \sSKIP 
      \\
      \sembase{\aCmd_1 \LPAR[\bThrd] \aCmd_2} &= \sLPAR{\sembase{\aCmd_1}}{\sembase{\aCmd_2}}
      \\
      \sembase{\aCmd_1 \SEMI \aCmd_2} &= \sSEMI{\sembase{\aCmd_1}}{\sembase{\aCmd_2}}
      \\
      \sembase{\IF{\aExp}\THEN\aCmd_1\ELSE\aCmd_2\FI} &= \sIFTHEN{\aExp{\neq}0}{\sembase{\aCmd_1}}{\sembase{\aCmd_2}}
    \end{aligned}
  \end{align*}
  \caption{Semantics of programs}
  \label{fig:sem}
\end{figure}


We give the semantics of programs $\sembase{}$ in \reffig{fig:sem}.

Let $\aPS$ range over pomsets, and $\aPSS$ over sets of pomsets.
%Let $\Pom$ be the set of all pomsets.

The model has seven components, which can be daunting at first glance.  To
aid the reader, we use consistent numbering throughout. For example, item $7$
always refers to the order relation.

The core of the model is a pomset, which includes a set of events
\eqref{pom-E}, a labeling \eqref{pom-lambda}, and an order \eqref{pom-le}.
We also include the \emph{reads-from} relation explicitly in the model
\eqref{pom-rf}.

On top of this basic structure, \ref{pom-kappa}--\ref{pom-term} add a layer
of logic.  For each pomset, \ref{pom-term} provides a termination condition.
For each event in a pomset, \ref{pom-kappa} provides a precondition.  For
each set of events in a pomset, \ref{pom-tau} provides a predicate
transformer.  Sequential dependency is calculated by $\labelingForm'_2$ in
the semantics of sequential composition.

Before discussing the details of the model, we note that the semantics
satisfies the expected monoid laws and is closed with respect to
\emph{augmentation}.  Augments include more order and stronger formulae; in
examples, we typically consider pomsets that are augment-minimal.  One
intuitive reading of augment closure is that adding order can only cause
preconditions to weaken.
\begin{lemma}
  $(\aPSS_1\SEMI\aPSS_2)\SEMI\aPSS_3=\aPSS_1\SEMI(\aPSS_2\SEMI\aPSS_3)$
  and    
  $\aPSS\SEMI\SKIP=\aPSS=\SKIP\SEMI\aPSS$.

  $(\aPSS_1\LPAR\aPSS_2)\LPAR\aPSS_3=\aPSS_1\LPAR(\aPSS_2\LPAR\aPSS_3)$
  and    
  $\aPSS\LPAR\SKIP=\aPSS$.  

  \vspace{-.5\baselineskip}
  \begin{proof}
    Straightforward calculation.  Associativity of $\SEMI\!$ requires
    disjunction closure \eqref{tr-or}.
  \end{proof}
\end{lemma}
\begin{definition}
  \label{def:augment}
  $\aPS_2$ is an \emph{augment} of $\aPS_1$ if
  \begin{multicols}{4}
    \begin{enumerate}
    \item $\aEvs_2=\aEvs_1$,
    \item $\labelingAct_2(\aEv)=\labelingAct_1(\aEv)$,
    \item $\labelingForm_2(\aEv) \rimplies \labelingForm_1(\aEv)$,
    \item $\aTr[2]{\bEvs}{\aEv} \rimplies \aTr[1]{\bEvs}{\aEv}$,
    \item $\aTerm[2] \rimplies \aTerm[1]$,
    \item ${\rrfx_2}={\rrfx_1}$,
    \item ${\le_2}\supseteq{\le_1}$.
    \end{enumerate}
  \end{multicols}
\end{definition}
\begin{lemma}
  % Suppose $\aPS_1\in\sembase{\aCmd}$.
  If $\aPS_1\in\sembase{\aCmd}$ and $\aPS_2$  augments $\aPS_1$ then $\aPS_2\in\sembase{\aCmd}$.
  % \item If $\aPS_2$ is a downset of $\aPS_1$ then $\aPS_2\in\sembase{\aCmd}$.
  % \end{lemma}

  \vspace{-.5\baselineskip}
  \begin{proof}
    Induction on the definition of $\sembase{}$.
  \end{proof}
\end{lemma}





% \begin{definition}
%   Suppose $\aRel_1:\aEvs_1\times\aEvs_1$ and $\aRel_2:\aEvs_2\times\aEvs_2$.
%   We say $\aRel \rextendsdef{\aRel_1}{\aRel_2}$ if
%   $\aRel\supseteq (\aRel_1\cup \aRel_2)$ and
%   $\aRel\cap(\aEvs_1\times \aEvs_1) = \aRel_1$ and
%   $\aRel\cap(\aEvs_2\times \aEvs_2) = \aRel_2$.
% \end{definition}

%We include empty sets as prep for adding while loops.


% In diagrams, we use different colors for arrows.  We distinguish
% $\rrfx$ edges that are included in order from those that are not.
% \begin{itemize}
% \item \makebox{$\aEv\xrf\bEv$} arises from $\rrfx$, where $\aEv\le\bEv$,
% \item \makebox{$\aEv\xrfint\bEv$} arises from $\rrfx$, where $\lnot(\aEv\le\bEv)$.
% \end{itemize}
% To help the reader understand why order is included, we also different colors
% for arrows induced by order.  We adopt the following conventions:
% \begin{itemize}  
%   % \item relaxed accesses are blue, with a single border,
%   % \item synchronized accesses are red, with a double border,
%   % \item \makebox{$\aEv\xrf\bEv$} arises from fulfillment, where $\aEv$ \emph{matches} $\bEv$,
%   % \item \makebox{$\aEv\xwk\bEv$} arises either from fulfillment, where $\aEv$
%   %   \emph{blocks} $\bEv$, or from prefixing, where $\aEv$ was prefixed before
%   %   $\bEv$ and their actions \emph{conflict},
% \item \makebox{$\aEv\xwk\bEv$} arises from \emph{fulfillment},
% \item \makebox{$\aEv\xpo\bEv$} arises from control/data/address \emph{dependency},
% \item \makebox{$\aEv\xsync\bEv$} arises from \emph{synchronized access}.
% \end{itemize}



\subsection{Discussion: Pomsets}
\label{sec:ex:pomset}

We first explain the core of model, ignoring the logic (rules 3--5).  We
defer discussion of $\sIF{}$ to \textsection\ref{sec:ex:kappa}.  We defer the
discussion of $\sLET{}{}$ to \textsection\ref{sec:ex:tau}.

Reads, writes, and fences map to pomsets with at most one event.  $\SKIP$
maps to the empty pomset.  Ignoring the logic, the definitions are
straightforward.  Note only that $\sembase{\PW{x}{1}}$ can write any value
$\aVal$; the fact that $\aVal$ must be $1$ is captured in the logic (see
\textsection\ref{sec:ex:term}).

The structural rules combine pomsets:  Parallel composition is disjoint
union, inheriting labeling, order and $\rrfx$ from the two sides.  Any
$\rrfx$ edges added between the two sides must also be added to the order
\eqref{par-le-rf}.
%
Sequential composition is similar, with two changes: \ref{seq-E} does not
require disjointness (see \textsection\ref{sec:ex:term}), and
\ref{seq-le-delays} may require order (see example \ref{pub}, below).

Note that reads-from implies order.
\begin{lemma}
  \label{lem:rf:implies:le}
  For any $\aPS$ in the range of $\sembase{}$, $\bEv\xrfx\aEv$ implies
  $\bEv\le\aEv$.

  \vspace{-.5\baselineskip}
  \begin{proof}
    Induction on the definition of $\sembase{}$, using \ref{par-le-rf} and \ref{seq-le-rf}.
  \end{proof}
\end{lemma}
In top-level pomsets, every read must have a matching write in $\rrfx$
\eqref{top-rf}.  Together with \ref{rf-match} and \ref{rf-block}, the lemma
guarantees that reads are \emph{fulfilled} at top-level, as in
\cite[\textsection 2.7]{DBLP:journals/pacmpl/JagadeesanJR20}.\footnote{The
  basic model would be the same if we move $\rrfx$ from the model itself to
  be existentially quantified in the definition of top-level pomset, along
  with \ref{rf-match} and \ref{rf-block}.  This was the approach of
  \citeauthor{DBLP:journals/pacmpl/JagadeesanJR20} We include $\rrfx$
  explicitly for use in \textsection\ref{sec:arm2}, where we introduce a
  variant semantics $\frf{\semrr{}}$ for which \reflem{lem:rf:implies:le}
  fails.}

From \refdef{def:actions}, recall that $\aAct \rdelaysdef \bAct$ if
$\aAct\eqreorderco\bAct$ or $\aAct\reorderra\bAct$ or
$\aAct\eqreordersc\bAct$.  \ref{seq-le-delays} guarantees that sequential
order is enforced between conflicting accesses of the same location
($\eqreorderco$), into a release and out of an acquire ($\reorderra$), and
between SC accesses ($\eqreordersc$).  Combined with the fulfillment
requirements (\ref{rf-match}, \ref{rf-block} and \reflem{lem:rf:implies:le}),
these ensure coherence, publication, subscription and other idioms.  For
example, consider the following:
\begin{gather*}
  \taglabel{pub}
  \begin{gathered}    
    \PW{x}{0}\SEMI 
    \PW{x}{1}\SEMI \PW[\mRA]{y}{1} \PAR \PR[\mRA]{y}{r}\SEMI \PR{x}{s}
    \\[-.4ex]
    \nonumber
    \hbox{\begin{tikzinline}[node distance=1.5em]
        \event{wx0}{\DW{x}{0}}{}
        \event{wx1}{\DW{x}{1}}{right=of wx0}
        \event{wy1}{\DWRel{y}{1}}{right=of wx1}
        \event{ry1}{\DRAcq{y}{1}}{right=2.5em of wy1}
        \event{rx0}{\DR{x}{0}}{right=of ry1}
        \sync{wx1}{wy1}
        \sync{ry1}{rx0}
        \rf{wy1}{ry1}
        \wk[out=170,in=10]{rx0}{wx1}
        \wk{wx0}{wx1}
      \end{tikzinline}}
  \end{gathered}
\end{gather*}
The execution is disallowed due to the evident cycle.  All of the order shown is
required at top-level: The intra-thread order comes from \ref{seq-le-delays}:
$\DWP{x}{0}\xwk \DWP{x}{1}$ is required by $\eqreorderco$.
$\DWP{x}{1}\xwk \DWP[\mREL]{y}{1}$ and $\DRP[\mACQ]{y}{1}\xwk \DRP{x}{0}$ are
required by $\reorderra$.  The cross-thread order is required by fulfillment:
\ref{top-rf} requires that all top-level reads are in the image of $\xrfx$.
\ref{rf-match} ensures that $\DWP[\mREL]{y}{1}\xrfx \DRP[\mACQ]{y}{1}$, and
\ref{seq-le-rf} subsequently ensures that
$\DWP[\mREL]{y}{1}\xwk \DRP[\mACQ]{y}{1}$.  The \emph{antidependency}
$\DRP{x}{0}\xwk \DWP{x}{1}$ is required by \ref{rf-block}.  (Alternatively,
we could have $\DWP{x}{1}\xwk \DWP{x}{0}$, again resulting in a cycle.)

The semantics gives the expected results for store buffer and load buffering,
as well as litmus tests involving fences and SC access.  The model of
coherence is weaker than C11, in order to support common subexpression
elimination, and stronger than Java, in order to support local reasoning
about data races.  See
\cite[\textsection3.1]{DBLP:journals/pacmpl/JagadeesanJR20} for a discussion.



\subsection{Discussion: Termination}
\label{sec:ex:term}

In top-level pomsets, \ref{top-term} requires that $\aTerm$ is a tautology,
capturing termination.  Ignoring predicate transformers, the structural
rules, \ref{par-term} and \ref{seq-term}, take $\aTerm$ to be
$\aTerm_1\land\aTerm_2$.  This is as expected: the program terminates if both
subprograms terminate.  The interesting rules are $\sLOAD{}{}$, $\sFENCE{}$,
and $\sSTORE{}{}$.


In $\sLOAD{}{}$, there is no restriction on $\aTerm$.  From this, it is easy
to see that
\begin{math}
  \sembase{\PR{x}{r}}
  \supseteq
  \sembase{\SKIP}
\end{math}
is a valid refinement.

In $\sFENCE{}$, instead, \ref{fence-term} ensures that all fences are included at
top-level.

In $\sSTORE{}{}$, \ref{write-term-empty} is similar.  In addition,
\ref{write-term-nonempty} ensures that top-level pomsets do not include bogus
writes.  Suppose $\aPS\in\sembase{\PW{x}{1}}$.  As we noted above, $\aPS$ can
include $\DWP{x}{\aVal}$, for any value $\aVal$.  At top-level, however,
\ref{write-term-nonempty} requires that $\aTerm$ implies $1{=}\aVal$.

In the structural rules, if $\bEv\in\aEvs_1$ and $\aEv\in\aEvs_2$, we say
that $\bEv$ and $\aEv$ \emph{coalesce} if $\bEv=\aEv$.

\ref{seq-E} allows \emph{mumbling} \cite{DBLP:journals/iandc/Brookes96} by
coalescing events.  For example
\begin{math}
  \sembase{\PW{x}{1} \SEMI \PW{x}{1}}
\end{math}
includes the singleton pomset
\begin{tikzinlinesmall}[node distance=.5em and 1.5em]
  \event{a1}{\DW{x}{1}}{}
\end{tikzinlinesmall}.  From this it is easy to see that
\begin{math}
  \sembase{\PW{x}{1}\SEMI\PW{x}{1}}
  \supseteq
  \sembase{\PW{x}{1}}
\end{math}
is a valid refinement.  It is equally obvious that
\begin{math}
  \sem{\PW{\aLoc}{1}} 
  \not\supseteq
  \sem{\PW{\aLoc}{1} \SEMI \PW{\aLoc}{1}}
\end{math}
is not a valid refinement, since the latter includes a two-element pomset,
but the former does not.\footnote{These are distinguished by the context:
\begin{math}
  \hole{} \PAR
  \PR{x}{r} \SEMI
  \PW{x}{2} \SEMI
  \PR{x}{s}\SEMI
  \IF{\aReg {=} \bReg} \THEN \PW{\cLoc}{1} \FI.
\end{math}}

\subsection{Discussion: Preconditions}
\label{sec:ex:kappa}

% Predicate transformers affect preconditions in the definition of
% $\labelingForm'_2$, which is used in \ref{seq-kappa2} and \ref{seq-kappa12}
% to calculate dependencies.  In this subsection we ignore predicate
% transformers, taking $\labelingForm'_2=\labelingForm_2$.

In top-level pomsets, \ref{top-kappa} requires that every precondition
$\labelingForm(\aEv)$ is a tautology.

Preconditions are introduced in three places: \ref{write-kappa} introduces
data dependencies for write actions, \ref{if-kappa} introduces control
dependencies, and \ref{seq-kappa-release} ensures that writes are up-to-date
before a release action.

Preconditions are discharged by sequential composition in \ref{seq-kappa2}
and \ref{seq-kappa12}, discussed \textsection\ref{sec:ex:tau}.

The definition of $\rextendsdef{}{}$ ensures that no new order
is introduced between events in $\aEvs_1\cap\aEvs_2$.\footnote{This ensures
  that
  \begin{math}
    \frf{\semrr{\IF{b}\THEN\PR{x}{r}\PAR\PW{x}{1}\ELSE\PR{x}{r}\SEMI\PW{x}{1}\FI}}
  \end{math}
  does not include 
  \begin{math}
    \smash{\hbox{\begin{tikzinlinesmaller}[node distance=1.5em]
        \event{a}{\DR{x}{1}}{}
        \event{b}{\DW{x}{1}}{right=of a}
        \rf[out=165,in=15]{b}{a}
        \wk{a}{b}
      \end{tikzinlinesmaller}}}
  \end{math},
  taking $\xrfx$ from the left and $\le$ from the right.}

We have given the semantics of $\sIF{}{}{}$ using 
disjunctive normal form. \citet{DBLP:journals/cacm/Dijkstra75} used
conjunctive normal form.  Note that 
\href{https://www.wolframalpha.com/input/?i=\%28a+and+b\%29+or+\%28not+a+and+c\%29}{$(\aForm\land\cForm_1)\lor(\neg\aForm\land\cForm_2)$}
is logically equivalent to
\href{https://www.wolframalpha.com/input/?i=\%28a+implies+b\%29+and+\%28\%28not+a\%29+implies+c\%29}{$(\aForm\limplies\cForm_1)\land(\neg\aForm\limplies\cForm_2)$}.

\begin{example}
  Consider $\aPS\in\sem{\IF {r{=}1} \THEN \PW{y}{r} \ELSE \PW{y}{1}\FI}$, so
  there must be $\aPS_1\in\sem{\PW{y}{r}}$,
  and $\aPS_2\in\sem{\PW{y}{1}}$, such as:
  \begin{align*}
    \begin{gathered}
      \PW{y}{r}
      \\
      \hbox{\begin{tikzinline}[node distance=1ex and 1.5em]
          \event{wy1}{r{=}1 \mid \DW{y}{1}}{}
        \end{tikzinline}}
    \end{gathered}
    &&
    \begin{gathered}
      \PW{y}{1}
      \\
      \hbox{\begin{tikzinline}[node distance=1ex and 1.5em]
          \event{wy1}{1{=}1 \mid \DW{y}{1}}{}
        \end{tikzinline}}
    \end{gathered}
  \end{align*}
  Since there is no requirement for disjointness in the semantics of conditionals,
  we can consider the case where the event \emph{coalesces} from the two pomsets,
  in which case:
  \begin{gather*}
    \IF{r{=}1} \THEN \PW{y}{r} \ELSE \PW{y}{1}\FI
    \\
    \nonumber
    \hbox{\begin{tikzinline}[node distance=1ex and 1.5em]
        \event{wy1}{(r{=}1 \limplies r{=}1) \land (r{\ne}1 \limplies 1{=}1) \mid \DW{y}{1}}{}
      \end{tikzinline}}
  \end{gather*}
  Here, the precondition is a tautology, 
  independent of $r$.
  % We refer to event sharing as \emph{coalescing} or \emph{merging}.
\end{example}

\subsection{Discussion: Predicate Transformers}
\label{sec:ex:tau}

the treatment of sequential composition
is new.  This uses the usual rule for composition of predicate
transformers (but preserving the indexing set). For the pomset, we
take the union of their events, preserving actions, but crucially in
cases~\ref{seq-kappa2} and~\ref{seq-kappa12} we apply a predicate
transformer $\aTr[1]{\cEvs}{}$ from the left-hand side to a precondition
$\labelingForm_2(\aEv)$ from the right-hand side to build the precondition
$\labelingForm'_2(\aEv)$.  The indexing set $\cEvs$ for the predicate
transformer is $\{ \cEv\mid \cEv<\aEv \}$, so can depend on the causal
order.
\begin{example}
  \label{ex:data}
  A simple example of a data dependency
  is a pomset $\aPS\in\sem{\PR{x}{r}\SEMI \PW{y}{r}}$,
  for which there must be an $\aVal\in\Val$ and $\aPS'\in \sem{\PW{y}{r}}$
  such as the following, where $\aVal=1$:
  % \vspace{-.5\baselineskip}
  \begin{gather*}
    \PW{y}{r}
    \\
    \nonumber
    \hbox{\begin{tikzinline}[node distance=1ex and 1.5em]
        \event{wy1}{r{=}1\mid\DW{y}{1}}{}
      \end{tikzinline}}
  \end{gather*}
  The value chosen for the read may be different from that chosen for the write:
  % \vspace{-.5\baselineskip}
  \begin{gather*}
    \PR{x}{r}\SEMI \PW{y}{r}
    \\
    \nonumber
    \hbox{\begin{tikzinline}[node distance=1ex and 1.5em]
        \event{rx0}{\DR{x}{0}}{}
        \event{wy1}{r{=}0\limplies r{=}1 \mid \DW{y}{1}}{right=of rx0}
        \po{rx0}{wy1}
      \end{tikzinline}}
  \end{gather*}
  In this case, the pomset's preconditions depend on a bound register, so
  cannot contribute to a top-level pomset.

  If the values chosen for read and write are compatible, then we have two
  cases: the independent case, which again cannot be part of a top-level
  pomset,
  \begin{gather*}
    \PR{x}{r}\SEMI \PW{y}{r}
    \\
    \nonumber
    \hbox{\begin{tikzinline}[node distance=1ex and 1.5em]
        \event{rx1}{\DR{x}{1}}{}
        \event{wy1}{r{=}1 \mid \DW{y}{1}}{right=of rx1}
      \end{tikzinline}}
  \end{gather*}
  and the dependent case:
  \begin{gather*}
    % \PR{x}{r}\SEMI \PW{y}{r}
    % \\
    % \nonumber
    \hbox{\begin{tikzinline}[node distance=1ex and 1.5em]
        \event{rx1}{\DR{x}{1}}{}
        \event{wy1}{r{=}1\limplies r{=}1 \mid \DW{y}{1}}{right=of rx1}
        \po{rx1}{wy1}
      \end{tikzinline}}
  \end{gather*}
  Since $r{=}1\limplies r{=}1$ is a tautology, this can be part of
  a top-level pomset.
\end{example}

\begin{example}
  Control dependencies are similar, for example
  for any $\aPS\in\sem{\PR{x}{r}\SEMI \IF r \THEN \PW{y}{1} \FI}$,
  there must be an $\aVal\in\Val$ and $\aPS'\in \sem{\IF r \THEN \PW{y}{1} \FI}$
  such as:
  \begin{gather*}
    \IF r \THEN \PW{y}{1} \FI
    \\
    \nonumber
    \hbox{\begin{tikzinline}[node distance=1ex and 1.5em]
        \event{wy1}{r{\ne}0\mid\DW{y}{1}}{}
      \end{tikzinline}}
  \end{gather*}
  The rest of the reasoning is the same as \refex{ex:data}.
\end{example}

\begin{example}
  A simple example of an independency
  is a pomset $\aPS\in\sem{\PR{x}{r}\SEMI \PW{y}{1}}$,
  for which there must be:
  % an %$\aVal\in\Val$ and
  % $\aPS'\in \sem{\PW{y}{r}}$
  % such as:
  % \vspace{-1\baselineskip}
  \begin{gather*}
    \PW{y}{1}
    \\
    \nonumber
    \hbox{\begin{tikzinline}[node distance=1ex and 1.5em]
        \event{wy1}{1{=}1\mid\DW{y}{1}}{}
      \end{tikzinline}}
  \end{gather*}
  In this case it doesn't matter what value the read chooses:
  \begin{gather*}
    \PR{x}{r}\SEMI \PW{y}{1}
    \\
    \nonumber
    \hbox{\begin{tikzinline}[node distance=1ex and 1.5em]
        \event{rx0}{\DR{x}{0}}{}
        \event{wy1}{1{=}1 \mid \DW{y}{1}}{right=of rx0}
      \end{tikzinline}}
  \end{gather*}
  % Since $1{=}1$ is a tautology, this can be part of
  % a top-level pomset.
\end{example}

\begin{example}
  For read to write dependency, consider:
  \begin{align*}
    \begin{gathered}
      \PR{x}{r} 
      \\
      \hbox{\begin{tikzinline}[node distance=.5em and 1.5em]
          \xform{xd}{1{=}r\limplies\bForm}{}
          \xform{xi}{\noOR{(x{=}r\lor1{=}r)\limplies}\bForm}{right=.5em of xd}
          \eventl{\bEv}{a1}{\DR{x}{1}}{left=of xd}
          \xo{a1}{xd}
        \end{tikzinline}}    
    \end{gathered}
    &&
    \begin{gathered}
      \PW{y}{r}
      \\
      \hbox{\begin{tikzinline}[node distance=.5em and 1.5em]
          \xform{xd}{r{=}1\land\bForm\noSUB{[r/y]}}{}
          \xform{xi}{\bForm\noSUB{[r/y]}}{right=.5em of xd}
          \eventl{\aEv}{a2}{r{=}1\mid\DW{y}{1}}{left=of xd}      
          \xo{a2}{xd}
        \end{tikzinline}}    
    \end{gathered}
  \end{align*}
  Putting these together without order,
  we calculate the precondition $\labelingForm(\aEv)$
  as $\aTr[1]{\cEvs}{\labelingForm_2(\aEv)}$, where $\cEvs$ is $\{\cEv\mid\cEv<\aEv\}$, which is $\emptyset$.
  Since $\aTr[1]{\emptyset}{\bForm}$ is $\bForm$, this gives that 
  $\labelingForm(\aEv)$ is $\labelingForm_2(\aEv)$, which is $\aReg{=}1$.
  This gives the pomset with predicate transformers:
  \begin{gather*}
    \PR{x}{r} \SEMI
    \PW{y}{r}
    \\
    \hbox{\begin{tikzinline}[node distance=.5em and 1.5em]
        \xform{xdd}{1{=}r\limplies(r{=}1\land \bForm\noSUB{[r/y]})}{}
        \xform{xdi}{1{=}r\limplies\bForm\noSUB{[r/y]}}{left=of xdd}
        \xform{xid}{r{=}1\land\noOR{(x{=}r\lor1{=}r)\limplies}\bForm\noSUB{[r/y]}}{right=of xdd}
        \xform{xii}{\noOR{(x{=}r\lor1{=}r)\limplies}\bForm\noSUB{[r/y]}}{right=of xid}
        \eventl{\bEv}{a1}{\DR{x}{1}}{above left=.5em and -2em of xdd}
        \eventl{\aEv}{a2}{\noOR{(x{=}r\lor1{=}r)\limplies} r{=}1\mid\DW{y}{1}}{above right=.5em and -4em of xdd}
        \xo{a1}{xdi}
        \xo{a2}{xid}
        \xo{a1}{xdd}
        \xo{a2}{xdd}
      \end{tikzinline}}
  \end{gather*}
  This pomset's preconditions depend on a bound register, so cannot contribute
  to a top-level pomset.
  
  Putting them together with order,
  we calculate the precondition $\labelingForm(\aEv)$
  as $\aTr[1]{\cEvs}{\labelingForm_2(\aEv)}$, where $\cEvs$ is $\{\cEv\mid\cEv<\aEv\}$, which is $\{\bEv\}$.
  Since $\aTr[1]{\{\bEv\}}{\bForm}$ is $(1{=}\aReg \limplies \bForm)$, this gives that 
  $\labelingForm(\aEv)$ is $(1{=}\aReg \limplies \labelingForm_2(\aEv))$, which is $(1{=}\aReg \limplies \aReg{=}1)$.
  This gives the pomset with predicate transformers:
  \begin{gather*}
    \PR{x}{r} \SEMI
    \PW{y}{r}
    \\
    \hbox{\begin{tikzinline}[node distance=.5em and 1.5em]
        \xform{xdd}{1{=}r\limplies(r{=}1\land \bForm\noSUB{[r/y]})}{}
        \xform{xdi}{1{=}r\limplies\bForm\noSUB{[r/y]}}{left=of xdd}
        \xform{xid}{r{=}1\land\noOR{(x{=}r\lor1{=}r)\limplies}\bForm\noSUB{[r/y]}}{right=of xdd}
        \xform{xii}{\noOR{(x{=}r\lor1{=}r)\limplies}\bForm\noSUB{[r/y]}}{right=of xid}
        \eventl{\bEv}{a1}{\DR{x}{1}}{above left=.5em and -2em of xdd}
        \eventl{\aEv}{a2}{1{=}r\limplies r{=}1\mid\DW{y}{1}}{above right=.5em and -4em of xdd}
        \xo{a1}{xdi}
        \xo{a2}{xid}
        \xo{a1}{xdd}
        \xo{a2}{xdd}
        \po{a1}{a2}
      \end{tikzinline}}
  \end{gather*}
  This pomset's preconditions do not depend on a bound register, so can contribute
  to a top-level pomset.
\end{example}

\begin{example}
  If the read and write choose different values:
  \begin{align*}
    \begin{gathered}
      \PR{x}{r} 
      \\
      \hbox{\begin{tikzinline}[node distance=.5em and 1.5em]
          \xform{xd}{1{=}r\limplies\bForm}{}
          \xform{xi}{\noOR{(x{=}r\lor1{=}r)\limplies}\bForm}{right=.5em of xd}
          \event{a1}{\DR{x}{1}}{left=of xd}
          \xo{a1}{xd}
        \end{tikzinline}}    
    \end{gathered}
    &&
    \begin{gathered}
      \PW{y}{r}
      \\
      \hbox{\begin{tikzinline}[node distance=.5em and 1.5em]
          \xform{xd}{\bForm\noSUB{[r/y]}}{}
          \xform{xi}{r{=}2\land\bForm\noSUB{[r/y]}}{right=.5em of xd}
          \event{a2}{r{=}2\mid\DW{y}{2}}{left=of xd}      
          \xo{a2}{xd}
        \end{tikzinline}}    
    \end{gathered}
  \end{align*}
  Putting these together with order, we have the following, which cannot be
  part of a top-level pomset:
  \begin{gather*}
    \PR{x}{r} \SEMI
    \PW{y}{r}
    \\
    \hbox{\begin{tikzinline}[node distance=.5em and 1.5em]
        \xform{xdd}{1{=}r\limplies(r{=}2\land \bForm\noSUB{[r/y]})}{}
        \xform{xdi}{1{=}r\limplies\bForm\noSUB{[r/y]}}{left=of xdd}
        \xform{xid}{r{=}2\land\noOR{(x{=}r\lor1{=}r)\limplies}\bForm\noSUB{[r/y]}}{right=of xdd}
        \xform{xii}{\noOR{(x{=}r\lor1{=}r)\limplies}\bForm\noSUB{[r/y]}}{right=of xid}
        \eventl{\bEv}{a1}{\DR{x}{1}}{above left=.5em and -2em of xdd}
        \eventl{\aEv}{a2}{1{=}r\limplies r{=}2\mid\DW{y}{2}}{above right=.5em and -4em of xdd}
        \xo{a1}{xdi}
        \xo{a2}{xid}
        \xo{a1}{xdd}
        \xo{a2}{xdd}
        \po{a1}{a2}
      \end{tikzinline}}
  \end{gather*}

  % The precondition of $\DWP{y}{2}$ is unsatisfiable if no further assumptions
  % can be made on $r$; that is the case here, since we assume each register
  % occurs at most once in source programs. If we choose a pomset that without
  % the order $\DRP{x}{1}\xpo\DWP{y}{2}$, the precondition of $\DWP{y}{1}$ is
  % strengthened to $(r{=}2)$, which is also unsatisfiable without further
  % assumptions on $r$; the predicate transformers are unchanged.
\end{example}

\begin{example}
  \label{ex:left-merge}
  \ref{S4} includes $\aExp{=}\aVal$ to ensure that
  spurious merges do not go undetected.  Consider the following.
  \begin{align*}
    \begin{gathered}
      \PW{x}{1}
      \\
      \hbox{\begin{tikzinline}[node distance=.5em and 1.5em]
          \event{a1}{1{=}1\mid\DW{x}{1}}{}
          \xform{x1d}{1{=}1\land\bForm}{right=of a1}
          \xform{x1i}{\bForm}{right=.5em of x1d}
          \xo{a1}{x1d}
        \end{tikzinline}}
    \end{gathered}
    &&
    \begin{gathered}
      \PW{x}{2}
      \\
      \hbox{\begin{tikzinline}[node distance=.5em and 1.5em]
          \event{a2}{2{=}1\mid\DW{x}{1}}{}
          \xform{x2d}{2{=}1\land\bForm}{right=of a2}
          \xform{x2i}{\bForm}{right=.5em of x2d}
          \xo{a2}{x2d}
        \end{tikzinline}}
    \end{gathered}
  \end{align*}
  % Simplifying
  %% \begin{align*}
  %%   \begin{gathered}
  %%     \PW{x}{1}
  %%     \\
  %%     \hbox{\begin{tikzinline}[node distance=.5em and 1.5em]
  %%       \event{a1}{\Qr{\aLoc}\land\Qw{\aLoc}\mid\DW{x}{1}}{}
  %%       \xform{x1d}{\bForm}{below=of a1}
  %%       \xform{x1i}{\bForm[\FALSE/\Qw{\aLoc}]}{below=of x1d}
  %%       \xo[xright]{a1}{x1d}
  %%     \end{tikzinline}}
  %%   \end{gathered}
  %%   &&
  %%   \begin{gathered}
  %%     \PW{x}{2}
  %%     \\
  %%     \hbox{\begin{tikzinline}[node distance=.5em and 1.5em]
  %%       \event{a2}{\FALSE\mid\DW{x}{1}}{}
  %%       \xform{x2d}{\bForm[\FALSE/\Qw{\aLoc}]}{below=of a2}
  %%       \xform{x2i}{\bForm[\FALSE/\Qw{\aLoc}]}{below=of x2d}
  %%       \xo[xright]{a2}{x2d}
  %%     \end{tikzinline}}
  %%   \end{gathered}
  %% \end{align*}
  Merging the actions, since $2{=}1$ is unsatisfiable, we have:
  \begin{gather*}
    \PW{x}{1}\SEMI \PW{x}{2}
    \\
    \hbox{\begin{tikzinline}[node distance=.5em and 1.5em]
        \event{a1}{\DW{x}{1}}{}
        \xform{x1d}{\FALSE}{right=of a1}
        \xform{x1i}{\bForm}{right=.5em of x1d}
        \xo{a1}{x1d}
      \end{tikzinline}}
  \end{gather*}
  This pomset cannot be part of a top-level pomset, since
  $\aTr{\aEvs}{\aSym}=\FALSE$ for every quiescence symbol $\aSym$.  This is
  what we would hope: that the program $\PW{x}{1}\SEMI \PW{x}{2}$ should only
  be top-level if there is a $(\DW{x}{2})$ event.
\end{example}

\begin{example}
  \label{ex:skolem}
  The predicate transformer we have chosen for \ref{L4} is different from the
  one used traditionally, which is written using substitution.  Substitution
  is also used in \jjr{}.  Attempting to
  write the predicate transformers in this style we have:
  \begin{enumerate}[topsep=0pt]
  \item[\ref{L4})]
    $\aTr{\bEvs}{\bForm} \rimplies \bForm[\aVal/\aReg]$, 
  \item[\ref{L5})]
    $\aTr{\cEvs}{\bForm} \rimplies (\forall\aReg)\bForm$.
  \end{enumerate}
  This phrasing of \ref{L5} says that $\bForm$ must be independent of $\aReg$
  in order to appear in a top-level pomset.  This choice for \ref{L5} is forced
  by \refdef{def:family}, which states that the predicate transformer for a
  small subset of $\aEvs$ must imply the transformer for a larger subset.

  Sadly, this definition fails associativity.

  Consider the following, eliding transformers:
  \begin{align*}
    \begin{gathered}[t]
      \PR{y}{r}
      \\
      \hbox{\begin{tikzinline}[node distance=.5em and 1.5em]
          \event{a}{\DR{y}{1}}{}
        \end{tikzinline}}    
    \end{gathered}
    &&
    \begin{gathered}[t]
      \PW{x}{\BANG r} 
      \\
      \hbox{\begin{tikzinline}[node distance=.5em and 1.5em]
          \event{b}{r{=}0\mid\DW{x}{1}}{}
        \end{tikzinline}}    
    \end{gathered}
    &&
    \begin{gathered}[t]
      \PW{x}{\BANG\BANG r}
      \\
      \hbox{\begin{tikzinline}[node distance=.5em and 1.5em]
          \event{b}{r{\neq}0\mid\DW{x}{1}}{}
        \end{tikzinline}}    
    \end{gathered}
    &&
    \begin{gathered}[t]
      \PW{x}{0} 
      \\
      \hbox{\begin{tikzinline}[node distance=.5em and 1.5em]
          \event{b}{\DW{x}{0}}{}
        \end{tikzinline}}    
    \end{gathered}
  \end{align*}  
  Associating to the right and merging:
  \begin{align*}
    \begin{gathered}[t]
      \PR{y}{r}
      \\
      \hbox{\begin{tikzinline}[node distance=.5em and 1.5em]
          \event{a}{\DR{y}{1}}{}
        \end{tikzinline}}    
    \end{gathered}
    &&
    \begin{gathered}[t]
      \PW{x}{\BANG r} 
      \SEMI \PW{x}{\BANG\BANG r} 
      \SEMI \PW{x}{0} 
      \\
      \hbox{\begin{tikzinline}[node distance=.5em and 1.5em]
          \event{b}{r{=}0 \lor r{\neq}0\mid\DW{x}{1}}{}
          \event{c}{\DW{x}{0}}{right=of b}
          \wk{b}{c}
        \end{tikzinline}}    
    \end{gathered}
  \end{align*}  
  The precondition of $\DWP{x}{1}$ is a tautology, thus we have:
  \begin{align*}
    \begin{gathered}[t]
      \PR{y}{r}
      \SEMI \PW{x}{\BANG r} 
      \SEMI \PW{x}{\BANG\BANG r} 
      \SEMI \PW{x}{0} 
      \\
      \hbox{\begin{tikzinline}[node distance=.5em and 1.5em]
          \event{a}{\DR{y}{1}}{}
          \event{b}{\DW{x}{1}}{right=of a}
          \event{c}{\DW{x}{0}}{right=of b}
          \wk{b}{c}
        \end{tikzinline}}    
    \end{gathered}
  \end{align*}
  If, instead, we associate to the left:
  \begin{align*}
    \begin{gathered}[t]
      \PR{y}{r}
      \SEMI \PW{x}{\BANG r} 
      \\
      \hbox{\begin{tikzinline}[node distance=.5em and 1.5em]
          \event{a}{\DR{y}{1}}{}
          \event{b}{1{=}0\mid\DW{x}{1}}{right=of a}
        \end{tikzinline}}    
    \end{gathered}
    &&
    \begin{gathered}[t]
      \PW{x}{\BANG\BANG r} 
      \SEMI \PW{x}{0} 
      \\
      \hbox{\begin{tikzinline}[node distance=.5em and 1.5em]
          \event{b}{r{\neq}0\mid\DW{x}{1}}{}
          \event{c}{\DW{x}{0}}{right=of b}
          \wk{b}{c}
        \end{tikzinline}}    
    \end{gathered}
  \end{align*}  
  Sequencing and merging:
  \begin{align*}
    \begin{gathered}[t]
      \PR{y}{r}
      \SEMI \PW{x}{\BANG r} 
      \SEMI \PW{x}{\BANG\BANG r} 
      \SEMI \PW{x}{0} 
      \\
      \hbox{\begin{tikzinline}[node distance=.5em and 1.5em]
          \event{a}{\DR{y}{1}}{}
          \event{b}{1{=}0 \lor r{\neq}0\mid\DW{x}{1}}{right=of a}
          \event{c}{\DW{x}{0}}{right=of b}
          \wk{b}{c}
        \end{tikzinline}}    
    \end{gathered}
  \end{align*}
  In this case, the precondition of $\DWP{x}{1}$ is not a tautology, forcing
  a dependency $\DRP{y}{1}\xpo \DWP{x}{1}$.

  Our solution is to Skolemize, replacing 
  uses of $\bForm[\aVal/\aReg]$
  by $(\aReg{=}\aVal)\limplies\bForm$,
  for uniquely chosen $\aReg$.
%  We have proven associativity of
%  \refdef{def:pomsets-trans} in Agda.
  The proof of associativity requires that predicate
  transformers distribute through disjunction (\refdef{def:trans}).  The
  attempt to define predicate transformers using substitution fails for
  \ref{L5} because the predicate transformer
  $\aTr{}{\bForm}=(\forall\aReg)\bForm$ does not distribute through
  disjunction:
  \begin{math}
    \aTr{}{\bForm_1\lor \bForm_2}=
    (\forall r)(\bForm_1\lor \bForm_2)
    \neq
    ((\forall r)(\bForm_1)) \lor ((\forall r)(\bForm_2))
    = \aTr{}{\bForm_1} \lor \aTr{}{\bForm_2}
  \end{math}.
  Since $\aTr{}\bForm=(\forall\aReg)\bForm$ does not distribute through disjunction,
  we use $\aTr{}\bForm=\bForm$ instead (which trivially distributes through disjunction).
  Unforutunately, this change means we cannot use substitution, since
  $\bForm$ does not imply $\bForm[\aVal/\aReg]$. Fortunately, Skolemizing
  solves this problem, since $\bForm$ implies $(\aReg{=}\aVal)\limplies\bForm$.
\end{example}


\subsection{Discussion: Valid Transformations}
\label{sec:valid}
When $\sem{\aCmd} \supseteq \sem{\aCmd'}$, we say that $\aCmd'$ is a
\emph{valid transformation} of $\aCmd$.  In this subsection, we show the
validity of specific optimizations.  
Let $\free(\aCmd)$ be the set of locations and registers that occur in $\aCmd$.

The semantics validates many peephole optimizations.  Most apply only to
relaxed access.
\begin{align*}
  \taglabel{RR}
  \sem{\PR{\aLoc}{\aReg} \SEMI \PR{\bLoc}{\bReg}\SEMI\aCmd} &=
  \sem{\PR{\bLoc}{\bReg}\SEMI \PR{\aLoc}{\aReg}\SEMI\aCmd} &&\text{if } \aReg\neq\bReg
  \\
  \taglabel{WW}
  \sem{\aLoc \GETS \aExp \SEMI \bLoc  \GETS \bExp\SEMI\aCmd} &=
  \sem{\bLoc  \GETS \bExp\SEMI \aLoc \GETS \aExp\SEMI\aCmd} &&\text{if } \aLoc\neq\bLoc
  \\
  \taglabel{RW}
  \sem{\aLoc \GETS \aExp  \SEMI \PR{\bLoc}{\bReg} \SEMI\aCmd} &=
  \sem{\PR{\bLoc}{\bReg} \SEMI\aLoc \GETS \aExp\SEMI\aCmd} &&\text{if }
  \aLoc\neq\bLoc \textand \bReg\not\in\free(\aExp)%\disjoint{{\free(\aLoc \GETS \aExp)}}{{\free(\PR{\bLoc}{\bReg})}}
\end{align*}
% \ref{WW}, \ref{RW} and \ref{RR} require that two sides of the semicolon
% have disjoint ids; for example, \ref{RW} requires $\disjoint{{\free(\aReg
%     \GETS \aLoc)}}{{\free(\bLoc \GETS \bExp)}}$. 
% \ref{RR} requires either $\aReg\neq\bReg$ or
%   $\aLoc=\bLoc$.  \ref{WW} and \ref{RW} require that two sides of the
% semicolon have disjoint ids; for example, \ref{RW} requires
% $\disjoint{{\free(\PR{\aLoc}{\aReg})}}{{\free(\bLoc \GETS \bExp)}}$.
\ref{5} imposes no order between events in \ref{RR}--\ref{RW}.  %Note that \ref{RR} allows aliasing.
Using augmentation closure, \ref{5} also validates roach-motel reorderings \cite{SevcikThesis}.  For
example, on read/write pairs:
  \begin{align*}
    \tag{\textsc{roach1}}\label{AcqW}
    \sem{x^\amode \GETS \aExp \SEMI\PR{y}{\bReg} \SEMI\aCmd} &\supseteq
    \sem{\PR{y}{\bReg}  \SEMI x^\amode\GETS \aExp \SEMI \aCmd} 
    &&\text{if }
    \aLoc\neq\bLoc \textand \bReg\not\in\free(\aExp)%\disjoint{{\free(\aLoc \GETS \aExp)}}{{\free(\PR{\bLoc}{\bReg})}}
    \\
    \tag{\textsc{roach2}}\label{RelW}
    \sem{x \GETS \aExp \SEMI\PR[\amode]{y}{\bReg} \SEMI\aCmd} &\supseteq
    \sem{\PR[\amode]{y}{\bReg}  \SEMI x\GETS \aExp \SEMI \aCmd} 
    &&\text{if }
    \aLoc\neq\bLoc \textand \bReg\not\in\free(\aExp)%\disjoint{{\free(\aLoc \GETS \aExp)}}{{\free(\PR{\bLoc}{\bReg})}}
  \end{align*}

  Redundant load elimination \eqref{RL} follows
  from \ref{1}, taking $\bEv\in\Event$, regardless of the access mode:
\begin{align*}
  \taglabel{RL}
  \sem{\PR[\amode]{\aLoc}{\aReg} \SEMI \PR[\amode]{\aLoc}{\bReg}\SEMI\aCmd} &\supseteq 
  \sem{\PR[\amode]{\aLoc}{\aReg} \SEMI \bReg  \GETS \aReg\SEMI\aCmd}
\end{align*}

Many laws hold for the conditional, such as dead code elimination \eqref{DC}
and code lifting \eqref{CL}:
\begin{align*}
  \taglabel{DC}
  \sem{\IF{\aExp}\THEN\aCmd_1\ELSE\aCmd_2\FI} &=
  \sem{\aCmd_1}
  &&\textif \aExp \text{ is a tautology}
  \\
  \taglabel{CL}
  \sem{\IF{\aExp}\THEN\aCmd\ELSE\aCmd\FI} &\supseteq
  \sem{\aCmd}
\end{align*}
Code lifting also applies to program fragments inside a conditional.  For example:
\begin{align*}
  \sem{\IF{\aExp}\THEN x\GETS \bExp \SEMI\aCmd_1\ELSE x\GETS \bExp \SEMI\aCmd_2\FI} &\supseteq
  \sem{x\GETS \bExp \SEMI \IF{\aExp}\THEN\aCmd_1\ELSE\aCmd_2\FI}
  % \sem{\IF{r{=}1}\THEN \PW{x}{r}\SEMI \PW{x}{1} \ELSE \PW{x}{1}\FI}
  % \supseteq
  % \sem{\PW{x}{1} \SEMI \IF{r{=}1}\THEN \PW{x}{r}\FI}
\end{align*}
We discuss the inverse of \ref{CL} in \textsection\ref{sec:refine}.

% As expected, %sequential and
% parallel composition commutes with conditionals and declarations, and
% conditionals and declarations commute with each other.  For example,
% we have \emph{scope extrusion}~\cite{Milner:1999:CMS:329902}:
% \begin{align*}
%   \taglabel{SE}
%   \sem{\aCmd\PAR \VAR\aLoc\SEMI\bCmd} &=
%   \sem{\VAR\aLoc\SEMI(\aCmd\PAR\bCmd)}
%   &&\text{if } \aLoc\not\in\free(\aCmd)
% \end{align*}

