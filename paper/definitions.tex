\section{Model}

Batty suggest example where dependencies are added and also go away, perhaps
by store forwarding. Something like:
\texttt{(r=x; y=1); (s=y; z=s+r)}

\subsection{Preliminaries}
The syntax is built from
\begin{itemize}
\item a set of \emph{values} $\Val$, ranged over by
  $\aVal$, $\bVal$, $\cVal$,
\item a set of \emph{registers} $\Reg$, ranged over by
  $\aReg$, $\bReg$,
\item a set of \emph{expressions} $\Exp$, ranged over by
  $\aExp$, $\bExp$,  $\cExp$.
\end{itemize}
\emph{Memory locations} are tagged values, written $\REF{\cVal}$.  Let $\Loc$
be the set of memory locations, ranged over by $\aLoc$, $\bLoc$, $\cLoc$.

We require that
\begin{itemize}
\item values and registers are disjoint,
\item values include at least the constants $0$ and $1$,
\item expressions include at least registers and values, 
\item expressions do \emph{not} include memory locations.
\end{itemize}
We model the following language.
\begin{align*}
  \amode \BNFDEF& \mRLX
  \BNFSEP \mRA 
  \BNFSEP \mSC
  \\
  \aCmd,\,\bCmd
  \BNFDEF& \SKIP
  \BNFSEP \LET{\aReg}{\aExp}
  % \BNFSEP \PR[\amode]{\aLoc}{\aReg}
  % \BNFSEP \PW[\amode]{\aLoc}{\aExp}
  \BNFSEP \PRREF[\amode]{\cExp}{\aReg}
  \BNFSEP \PWREF[\amode]{\cExp}{\aExp}
  %\BNFSEP \PA{\aLoc}{\aExp} 
  \\[-.5ex]
  \BNFSEP& \FORK{\aGrp}
  \BNFSEP \aCmd \SEMI \bCmd
  \BNFSEP \IF{\aExp} \THEN \aCmd \ELSE \bCmd \FI
  \\
  \aGrp ,\,\bGrp
  \BNFDEF& \NIL
  \BNFSEP\THREAD{\aCmd}
  \BNFSEP\aGrp \PAR \bGrp
\end{align*}
\emph{Memory modes}, $\amode$, are {relaxed} ($\mRLX$), {release-acquire}
($\mRA$), and {sequentially consistent} ($\mSC$).  Relaxed is the default.
%
\emph{Commands}, $\aCmd$, include reads from and writes to memory at a given
mode, as well as the usual structural constructs.
%
\emph{Thread groups}, $\aGrp$, include commands and $\NIL$, which denotes
inaction. %, or deadlock.
%
The $\FORK{}$ command spawns a thread group.  We often drop the words
$\FORK{}$ and $\THREAD{}$.

The semantics is built from the following.
\begin{itemize}
\item a set of \emph{actions} $\Act$, ranged over by $\aAct$, 
\item a set of \emph{logical formulae} $\Formulae$, ranged over by $\aForm$,
  $\bForm$.
\end{itemize}
We require that
\begin{itemize}
\item actions include writes $\DWP{\aLoc}{\aVal}$ and reads $\DRP{\aLoc}{\aVal}$,
\item formulae include equalities $(\aExp{=}\bExp)$ and $(\aExp{=}\aLoc)$,
\item formulae are closed under negation, conjunction, disjunction, and
  substitutions $[\aExp/\aReg]$ and $[\aExp/\aLoc]$, 
\item there is an entailment relation $\vDash$ between formulae, with the
  expected semantics.
\end{itemize}

% Action $\DWP{\aLoc}{\aVal}$ models a write of $\aVal$ to $\aLoc$, and
% $\DRP{\aLoc}{\aVal}$ models a read of $\aVal$ from $\aLoc$.
% We use $\DAP{\aLoc}{\aVal}$ to indicate a generic access, which could be
% either a read or a write.

Logical formulae include equations over locations and registers, such
$(\aLoc{=}1)$ and $(\aReg{=}\bReg{+}1)$.
We use expressions as formulae, coercing $\aExp$ to $\aExp\neq 0$.
Formulae are subject to substitutions of the form $[\aExp/\aLoc]$; actions are not.

%\begin{definition}
  %\label{def:independent}
  % We say $\aForm$ is \emph{independent of $\aLoc$} if, for every
  % $\aVal$, $\aForm \vDash \aForm[\aVal/\aLoc] \vDash \aForm$; it is
  % \emph{dependent} otherwise.
  % We say $\aForm$ is \emph{location independent} if it is independent of
  % every location.
  We say
  $\aForm$ \emph{implies} $\bForm$ if $\aForm\vDash\bForm$.
  % We say that
  We say
  $\aForm$ is a \emph{tautology} if $\TRUE\vDash\aForm$.
  % We say that
  We say
  $\aForm$ is \emph{unsatisfiable} if $\aForm\vDash\FALSE$.
%\end{definition}

\subsection{Pomsets}

We first consider a fragment of our language that can be modeled using simple
pomsets.
\begin{definition}
  A \emph{pomset} over $\Act$ is a tuple
  $(\Event, {\le}, \labeling)$ where
  \begin{itemize}
  \item $\Event$ is a set of \emph{events},
  \item
    ${\le} \subseteq (\Event\times\Event)$ is the \emph{causality} partial order, 
  \item
    $\labeling: \Event \fun \Act$ is a \emph{labeling}.
  \end{itemize}
\end{definition}
Let $\aPS$ range over pomsets, and $\aPSS$ over sets of pomsets.

We lift terminology from actions to events.  For example, we say that $\aEv$
writes $\aLoc$ if $\labeling(\aEv)$ writes $\aLoc$.  We also drop quantifiers
when clear from context, such as
$(\forall\aEv\in\Event)(\forall\aLoc\in\Loc)$.

\begin{definition}
  Action $\DWP{\aLoc}{\aVal}$ \emph{matches} $\DRP{\aLoc}{\bVal}$ when $\aVal=\bVal$.
  Action $\DWP{\aLoc}{\aVal}$ \emph{blocks} $\DRP{\aLoc}{\bVal}$.
  
  % A pomset $\aPS$ is \emph{closed} if for every $\aEv$ which can be
  % fulfilled, there is a $\bEv\le\aEv$ which fulfills it, and for any $\cEv$
  % which can block $\aEv$, either $\cEv\le\bEv$ or $\aEv\le\cEv$.

  We say that $\aEv$ is \emph{fulfilled} if there is a $\bEv\le\aEv$ which
  matches it and, for any $\cEv$ which can block $\aEv$, either $\cEv\le\bEv$
  or $\aEv\le\cEv$.

  We say that $\aPS$ is \emph{fulfilled} if every read in $\aPS$ is fulfilled.

  We define \emph{independency} (${\reorder}\subseteq\Act\times\Act$) as follows.
  \begin{align*}
    % {\reorderSC} &= \emptyset
    % \\
    % {\reorderRLX} &=
    \reorder &=
    \{(\DR{\aLoc}{\aVal}, \DR{\bLoc}{\bVal}) \}
    \\[-.5ex]&\mkern2mu\cup\mkern2mu
    \{(\DW{\aLoc}{\aVal}, \DW{\bLoc}{\bVal}) \mid \aLoc\neq\bLoc \lor \aVal=\bVal\}
    \\[-.5ex]&\mkern2mu\cup\mkern2mu
    \{(\DR{\aLoc}{\aVal}, \DW{\bLoc}{\bVal}), (\DW{\aLoc}{\aVal}, \DR{\bLoc}{\bVal}) \mid \aLoc\neq\bLoc\}
  \end{align*}
\end{definition}

In order to give the semantics, we define several operators over sets of
pomsets.

\begin{definition} $\phantom{\;}$\par
\label{tab:pomsets}
  %Let $\aPSS$ range over sets of pomsets.

  \noindent
  If $\aPS\SB0\in\sSTOP$ then $\aEvs\SB0 = \emptyset$.

  \noindent
  If $\aPS\SB0 \in (\aPSS_1\sPAR\aPSS_2)$ then  
  $(\exists\aPS_1\in\aPSS_1)$ $(\exists\aPS_2\in\aPSS_2)$
  %there are $\aPS_1\in\aPSS_1$ and $\aPS_2\in\aPSS_2$ such that
  \begin{enumerate}
  \item $\aEvs\SB0 = (\aEvs_1\cup\aEvs_2)$,
  \item if $\aEv\in\aEvs_1$ then $\labeling\SB0(\aEv) = \labeling_1(\aEv)$, 
  \item if $\aEv\in\aEvs_2$ then $\labeling\SB0(\aEv) = \labeling_2(\aEv)$,
  \item if $\bEv\le_1\aEv$ then $\bEv\le\SB0\aEv$, 
  \item if $\bEv\le_2\aEv$ then $\bEv\le\SB0\aEv$, 
    \newcounter{pomsetParCount}
    \setcounter{pomsetParCount}{\value{enumi}}
  \item $\aEvs_1$ and $\aEvs_2$ are disjoint.
    \newcounter{pomsetParDisjointCount}
    \setcounter{pomsetParDisjointCount}{\value{enumi}}
  \end{enumerate}

  \noindent
  If $\aPS\SB0 \in (\aAct\sPREFIX\aPSS)$ then
  $(\exists\aPS_2\in\aPSS)$
  %there is $\aPS_2\in\aPSS$ such that
  \begin{enumerate}
  \item $\aEvs\SB0=(\aEvs_1 \cup \aEvs_2)$,
  \item if $\aEv\in\aEvs_1$ then $\labelingAct\SB0(\aEv) = \aAct$,
  \item if $\aEv\in\aEvs_2$ then $\labelingAct\SB0(\aEv) = \labelingAct_2(\aEv)$,
  \item if $\bEv,\aEv\in\aEvs_1$ then $\bEv=\aEv$,
  \item if $\bEv\le_2\aEv$ then $\bEv\le\SB0\aEv$, 
    % \newcounter{pomsetPrefixCount}
    % \setcounter{pomsetPrefixCount}{\value{enumi}}
  \item if $\bEv\in\aEvs_1$ and $\aEv\in\aEvs_2$, either $\bEv<\SB0\aEv$ or $a\reorder\labeling_2(\aEv)$.
    \newcounter{pomsetPrefixOrderCount}
    \setcounter{pomsetPrefixOrderCount}{\value{enumi}}
  \end{enumerate}
\end{definition}
Using these operators, we can give the semantics for a simple fragment of our
language.
% To allow interesting examples, we add the operator
% ``$\PA{\aLoc}{\aVal}$.''
\begin{align*}
  \sem{\NIL} &= \sSTOP
  \\
  % \sem{\THREAD{\aCmd} \PAR \THREAD{\bCmd}} &= \sem{\aCmd}\sPAR\sem{\bCmd}
  % \\
  \sem{\aGrp \PAR \bGrp} &= \sem{\aGrp}\sPAR\sem{\bGrp}
  \\
  % \sem{{\aCmd} \PAR {\bCmd}} &= \sem{\aCmd}\sPAR\sem{\bCmd}
  % \\
  \sem{\PW{\aLoc}{\aVal}\SEMI \aCmd} &= \textstyle\DWP{\aLoc}{\aVal}\sPREFIX\sem{\aCmd}
  \\
  \sem{\PR{\aLoc}{\aReg}\SEMI \aCmd} &= \textstyle\bigcup_{\aVal}\DRP{\aLoc}{\aVal}\sPREFIX\sem{\aCmd}
  % \\
  % \sem{\PA{\aLoc}{\aVal}\SEMI \aCmd} &= \textstyle\DRP{\aLoc}{\aVal}\sPREFIX\sem{\aCmd}
\end{align*}

If we take ${\reorder} = \emptyset$, then we have sequentially consistent
execution.

[Do Examples.]

[Do examples with coherence.]

[Note that this allows mumbling for reads and writes.]

[Use refinement (that is subset order) as notion of compiler optimization.]

[Talk about Mazurkiewicz traces.]

\subsection{Pomsets with Preconditions}

[Problem with previous section is that  notion of dependency is impoverished]

\begin{definition}
  A \emph{pomset with preconditions} is
  a pomset together with $\labelingForm:\aEvs\fun\Formulae$.
\end{definition}

\begin{definition}
  A pomset with preconditions is \emph{top level} if it is fulfilled and
  every precondition is a tautology.
\end{definition}

\begin{definition}$\phantom{\;}$\par
  \label{tab:pomsets-pre}
  
  \noindent
  If $\aPS\SB0\in\sSTOP$ then $\aEvs\SB0 = \emptyset$.


  \noindent
  If $\aPS\SB0 \in (\aPSS_1\sPAR\aPSS_2)$ then
  $(\exists\aPS_1\in\aPSS_1)$ $(\exists\aPS_2\in\aPSS_2)$
  %there are $\aPS_1\in\aPSS_1$ and $\aPS_2\in\aPSS_2$ such that
\begin{enumerate}
     \setcounter{enumi}{\value{pomsetParDisjointCount}}
  \item[1--\thepomsetParDisjointCount)] as for $\sPAR$ in Definition~\ref{tab:pomsets},
  \item if $\aEv\in\aEvs_1$ then $\labelingForm\SB0(\aEv)$ implies $\labelingForm_1(\aEv)$,
  \item if $\aEv\in\aEvs_2$ then $\labelingForm\SB0(\aEv)$ implies $\labelingForm_2(\aEv)$.
    \newcounter{pomsetPreParCount}
    \setcounter{pomsetPreParCount}{\value{enumi}}
  \end{enumerate}

  \noindent
  If $\aPS\SB0 \in \sIF{\bForm}\sTHEN\aPSS_1\sELSE\aPSS_2\sFI$ then
  $(\exists\aPS_1\in\aPSS_1)$ $(\exists\aPS_2\in\aPSS_2)$
  %there are $\aPS_1\in\aPSS_1$ and $\aPS_2\in\aPSS_2$ such that
  \begin{enumerate}
  \setcounter{enumi}{\value{pomsetParCount}}
  \item[1--\thepomsetParCount)] as for $\sPAR$  in
    Definition~\ref{tab:pomsets} (ignoring disjointness),
  \item if $\aEv\in\aEvs_1\setminus\aEvs_2$ then $\labelingForm\SB0(\aEv)$ implies $\bForm\land\labelingForm_1(\aEv)$,
  \item if $\aEv\in\aEvs_2\setminus\aEvs_1$ then $\labelingForm\SB0(\aEv)$ implies $\neg\bForm\land\labelingForm_2(\aEv)$, 
  \item if $\aEv\in\aEvs_1\cap\aEvs_2$ then\\ $\labelingForm\SB0(\aEv)$ implies $(\bForm\land\labelingForm_1(\aEv))\lor(\neg\bForm\land\labelingForm_2(\aEv))$.
    \newcounter{pomsetPreIfCount}
    \setcounter{pomsetPreIfCount}{\value{enumi}}
  \end{enumerate}

  \noindent
  If $\aPS\SB0 \in \sSTOREPRE{\aLoc}{\aExp}{\aPSS_2}$ then
  $(\exists\aPS_2\in\aPSS_2)$
  $(\exists\aVal\in\Val)$
  %there is $\aPS_2\in\aPSS$ such that
  \begin{enumerate}
     \setcounter{enumi}{\value{pomsetPrefixOrderCount}}
  \item[1--\thepomsetPrefixOrderCount)] as for $\DWP{\aLoc}{\aVal}\sPREFIX\aPS_2$ in
    Definition~\ref{tab:pomsets}, % (ignoring \thepomsetPrefixOrderCount),
  \item if $\bEv\in\aEvs_1\setminus\aEvs_2$ then $\labelingForm\SB0(\bEv)$ implies $(\aExp{=}\aVal)$,
  % \item if $\aEv\in\aEvs_2\setminus\aEvs_1$ then $\labelingForm\SB0(\aEv)$ implies $\labelingForm_2(\aEv)$,
  % \item if $\bEv\in\aEvs_1\cap\aEvs_2$ then $\labelingForm\SB0(\bEv)$ implies $(\aExp{=}\aVal)\lor \labelingForm_2(\bEv)$.
  \item if $\aEv\in\aEvs_2$ then either
    $\labelingForm\SB0(\aEv)$ implies $\labelingForm_2(\aEv)$ or\\
    $\aEv\in\aEvs_1$ and $\labelingForm\SB0(\aEv)$ implies $(\aExp{=}\aVal)\lor \labelingForm_2(\aEv)$.
  \end{enumerate}

  \noindent
  If $\aPS\SB0 \in \sLOADPRE{\aLoc}{\aReg}{\aPSS_2}$ then
  $(\exists\aPS_2\in\aPSS_2)$
  $(\exists\aVal\in\Val)$
  %there is $\aPS_2\in\aPSS$ such that
  \begin{enumerate}
     \setcounter{enumi}{\value{pomsetPrefixOrderCount}}
  \item[1--\thepomsetPrefixOrderCount)] as for $\DRP{\aLoc}{\aVal}\sPREFIX\aPS_2$ in
    Definition~\ref{tab:pomsets}, % (ignoring \thepomsetPrefixOrderCount),
  \item if $\aEv\in\aEvs_2$ then either $\aEv\in\aEvs_1$ or\\
    $\labelingForm\SB0(\aEv)$ implies $(\aReg{=}\aVal\lor\aReg{=}\aLoc)\limplies\labelingForm_2(\aEv)[\aReg/\aLoc]$ or\\
    $\labelingForm\SB0(\aEv)$ implies $(\aReg{=}\aVal)\limplies\labelingForm_2(\aEv)[\aReg/\aLoc]$ and $\bEv<\SB0\aEv$ \\
    \mbox{$\quad$} for some $\bEv\in\aEvs_1$.
  \end{enumerate}  
\end{definition}
Note that when $\bEv\in\aEvs_1\setminus\aEvs_2$, $\sLOADPRE{}{}{}$ does not
constrain $\labelingForm\SB0(\bEv)$.

The semantics of $\NIL$ and $\PAR$ are as before.
\begin{align*}
  % \sem{\NIL} &= \sSTOP
  % \\
  % \sem{\aCmd \PAR \bCmd} &= \sem{\aCmd}\sPAR\sem{\bCmd}
  % \\
  \sem{\IF{\bForm}\THEN\aCmd\ELSE\bCmd\FI} &= \sIF{\bForm}\sTHEN\sem{\aCmd}\sELSE\sem{\bCmd}\sFI
  \\
  \sem{\LET{\aReg}{\aExp}\SEMI \aCmd} &= \sem{\aCmd}[\aExp/\aReg]
  \\
  \sem{\PW{\aLoc}{\aExp}\SEMI \aCmd} &= \sSTOREPRE{\aLoc}{\aExp}{\sem{\aCmd}}
  \\
  \sem{\PR{\aLoc}{\aReg}\SEMI \aCmd} &= \sLOADPRE{\aLoc}{\aReg}{\sem{\aReg}}
\end{align*}

This is essentially the model of \cite{DBLP:journals/pacmpl/JagadeesanJR20}, restricted to
relaxed access.  There are only two substantial differences.  First, for
simplicity, we enforce read-read dependencies.  Second, in the rule for read
prefixing we have substituted $[r/x]$, rather than $[x/r]$.  This means that
reads clobber local state.  We assume registers are only used
once---otherwise, one needs to generate a fresh register for the
substitution.


With read-read dependencies, the second difference can be seen.  For example,
the following execution is allowed with $[x/r]$, but not $[r/x]$.
\begin{gather*}
  \PW{x}{0}\SEMI
  \PR{x}{r}\SEMI
  \IF{r}\THEN \PR{x}{s}\FI\SEMI
  \PW{y}{s{+}1}
  \PAR
  \PW{x}{\PR{y}{}}
  \\[-1ex]
  \hbox{\begin{tikzinline}[node distance=1.5em]
      \event{a1}{\DW{x}{0}}{}
      \event{a2}{\DW{x}{1}}{right=of a1}
      \event{a3}{\DR{x}{0}}{right=of a2}
      \event{a4}{\DW{y}{1}}{right=of a3}
      \event{b1}{\DR{y}{1}}{right=3em of a4}
      \event{b2}{\DW{x}{1}}{right=of b1}
      \wk{a1}{a2}
      \po{a2}{a3}
      \po{b1}{b2}
      \rf{a4}{b1}
      \rf[out=165,in=15]{b2}{a2}
    \end{tikzinline}}
\end{gather*}
[Is there a difference w/o read-read dependencies?]



[Stuff about conditionals and merging events.]



\subsection{Pomsets with Predicate Transformers}

[The problem with the previous section is that there's no story for sequential composition.]

\begin{definition}
  A \emph{predicate transformer} is a monotone function
  $\aTr:\Formulae\fun\Formulae$ such that
  $\aTr(\FALSE)$ is $\FALSE$,
  $\aTr(\aForm\land\bForm)$ is $\aTr(\aForm)\land\aTr(\bForm)$, and
  $\aTr(\aForm\lor\bForm)$ is $\aTr(\aForm)\lor\aTr(\bForm)$.
\end{definition}

\begin{definition}
  A \emph{family of predicate transformers}
  indexed by subsets of $\aEvs$
  consists of predicate transformers
  $\aTr[\bEvs]$ for each set of events $\bEvs$,
  such that if $\cEvs \subseteq \bEvs$
  then $\aTr[\cEvs](\aForm)$ implies $\aTr[\bEvs](\aForm)$.
\end{definition}

\begin{definition}
  A pomset with predicate tansformers is a pomset with preconditions,
  together with a family of predicate transformers $\aTr$ indexed by subsets
  of $\aEvs$.
\end{definition}

% \begin{definition}
%   For pomset $\aPS\SB0$ and $\aEv\in\aEvs\SB0$, let
%   $\downclose[0]{\aEv} = \{ \bEv\in\aEvs\SB0 \mid \bEv <\SB0 \aEv \}$.
% \end{definition}

Define $\sTHREAD{}$ to embed pomsets with predicate transformers into pomsets with preconditions
simply by dropping the predicate transformer.
For the reverse embedding, $\sFORK{}$ adopts the identity transformer.
\begin{definition}
  If $\aPS\SB0 \in \sFORK{\aPSS}$ then
  $(\exists\aPS_1\in\aPSS)$
  %there is $\aPS_1\in\aPSS$ such that
  \begin{enumerate}
  \item $\aEvs\SB0=\aEvs_1$,
  \item $\labelingAct\SB0(\aEv) = \labelingAct_1(\aEv)$,
  \item $\labelingForm\SB0(\aEv)$ implies $\labelingForm_1(\aEv)$, 
  \item $\aTr[\bEvs]\SB0(\aForm)$ implies $\aForm$.
  \end{enumerate}  
\end{definition}

\begin{definition}
  If $\aPS\SB0\in\sSTOP$ then $\aEvs\SB0 = \emptyset$ and
  \begin{enumerate}
  \item $\aTr[\bEvs]\SB0(\aForm)$ implies $\FALSE$.
  \end{enumerate}

  \noindent
  If $\aPS\SB0\in\sSKIP$ then $\aEvs\SB0 = \emptyset$ and
  \begin{enumerate}
  \item $\aTr[\bEvs]\SB0(\aForm)$ implies $\aForm$.
  \end{enumerate}

  \noindent
  If $\aPS\SB0\in\sLET{\aReg}{\aExp}$ then $\aEvs\SB0 = \emptyset$ and
  \begin{enumerate}
  \item $\aTr[\bEvs]\SB0(\aForm)$ implies $\aForm[\aExp/\aReg]$.
  \end{enumerate}

  \noindent
  If $\aPS\SB0 \in \sIF{\bForm}\sTHEN\aPSS_1\sELSE\aPSS_2\sFI$ then
  $(\exists\aPS_1\in\aPSS_1)$ $(\exists\aPS_2\in\aPSS_2)$
  %there are $\aPS_1\in\aPSS_1$ and $\aPS_2\in\aPSS_2$ such that
  \begin{enumerate}
  \setcounter{enumi}{\value{pomsetPreIfCount}}
  \item[1--\thepomsetPreIfCount)] as for $\sIF{}$ in Definition~\ref{tab:pomsets-pre},
  \item $\aTr[\bEvs]\SB0(\aForm)$ implies $(\bForm\land\aTr[\bEvs]_1(\aEv))\lor(\neg\bForm\land\aTr[\bEvs]_2(\aForm))$.
  \end{enumerate}

  \noindent
  If $\aPS\SB0 \in (\aPSS_1\sSEMI\aPSS_2)$ then
  $(\exists\aPS_1\in\aPSS_1)$ $(\exists\aPS_2\in\aPSS_2)$,
  %there are $\aPS_1\in\aPSS_1$ and $\aPS_2\in\aPSS_2$ such that
  %let $\labelingForm'_2(\aEv)=\aTr[{\downclose[0]{\aEv}}]_1(\labelingForm_2(\aEv))$  
  %let $\labelingForm'_2(\aEv)=\aTr[\{ \bEv \mid \bEv <\SB0 \aEv \}]_1(\labelingForm_2(\aEv))$  
  \begin{enumerate}
  \setcounter{enumi}{\value{pomsetParCount}}
  \item[1--\thepomsetParCount)] as for $\sPAR$  in
    Definition~\ref{tab:pomsets} (ignoring disjointness),
  \item if $\aEv\in\aEvs_1\setminus\aEvs_2$ then $\labelingForm\SB0(\aEv)$ implies $\labelingForm_1(\aEv)$,
  \item if $\aEv\in\aEvs_2\setminus\aEvs_1$ then $\labelingForm\SB0(\aEv)$ implies $\labelingForm'_2(\aEv)$,
  \item if $\aEv\in\aEvs_1\cap\aEvs_2$ then $\labelingForm\SB0(\aEv)$ implies $\labelingForm_1(\aEv)\lor\labelingForm'_2(\aEv)$,
  \\ where $\labelingForm'_2(\aEv)=\aTr[\cEvs]_1(\labelingForm_2(\aEv))$, where $\cEvs=\{ \cEv \mid \cEv <\SB0 \aEv \}$,
  \item $\aTr[\bEvs]\SB0(\aForm)$ implies $\aTr[\bEvs]_2(\aTr[\bEvs]_1(\aForm))$.
  %\item if $\bEv\in\aEvs_1$ and $\aEv\in\aEvs_2$ either $\bEv<\SB0\aEv$ or $a\reorder\labeling_2(\aEv)$.
  \end{enumerate}
  % where we define $\labelingForm'_2(\aEv)$ to be $\aTr[\{ \bEv \mid \bEv <\SB0 \aEv \}]_1(\labelingForm_2(\aEv))$.

  \noindent
  If $\aPS\SB0 \in \sSTORE{\aLoc}{\aExp}$ then
  $(\exists\aVal\in\Val)$
  \begin{enumerate}
  \item $\labelingAct\SB0(\aEv) = \DWP{\aLoc}{\aVal}$,
  \item $\labelingForm\SB0(\aEv)$ implies $(\aExp{=}\aVal)$,
  \item %if $\bEvs=\emptyset$ then
    $\aTr[\emptyset]\SB0(\aForm)$ implies $\aForm[\aExp/\aLoc]$,
  \item %if $\bEvs\neq\emptyset$ then
    $\aTr[\bEvs]\SB0(\aForm)$ implies $(\aExp{=}\aVal) \land \aForm[\aExp/\aLoc]$, if $\bEvs\neq\emptyset$,
  \item if $\bEv,\aEv\in\aEvs\SB0$ then $\bEv=\aEv$.
  \end{enumerate}

  \noindent
  If $\aPS\SB0 \in \sLOAD{\aLoc}{\aReg}$ then
  $(\exists\aVal\in\Val)$
  \begin{enumerate}
  \item $\labelingAct\SB0(\aEv) = \DRP{\aLoc}{\aVal}$,
  \item %if $\bEvs=\emptyset$ then
    $\aTr[\emptyset]\SB0(\aForm)$ implies $(\aReg{=}\aVal\lor\aReg{=}\aLoc)\limplies\aForm[\aReg/\aLoc]$,
  \item %if $\bEvs\neq\emptyset$ then
    $\aTr[\bEvs]\SB0(\aForm)$ implies $(\aReg{=}\aVal)\limplies\aForm[\aReg/\aLoc]$, if $\bEvs\neq\emptyset$,
  \item if $\bEv,\aEv\in\aEvs\SB0$ then $\bEv=\aEv$.
  \end{enumerate}
\end{definition}
[$(\aExp{=}\aVal)$ appears in $\aTr[\bEvs]$ for $\sSTORE{}{}$ in order to
ensure that anything ordered after this is in a compatible world. ?]

The complete semantics is as follows.
\begin{align*}
  \sem{\SKIP} &= \sSKIP
  \\
  \sem{\PR{\aLoc}{\aReg}} &= \sLOAD{\aLoc}{\aReg}
  \\
  \sem{\PW{\aLoc}{\aExp}} &= \sSTORE{\aLoc}{\aExp}
  \\
  \sem{\LET{\aReg}{\aExp}} &= \sLET{\aReg}{\aExp}
  \\
  \sem{\FORK{\aGrp}} &= \sFORK{}\sem{\aGrp}
  \\
  \sem{\aCmd \SEMI \bCmd} &= \sem{\aCmd}\sSEMI\sem{\bCmd}
  \\
  \sem{\IF{\bForm}\THEN\aCmd\ELSE\bCmd\FI} &= \sIF{\bForm}\sTHEN\sem{\aCmd}\sELSE\sem{\bCmd}\sFI
  \\
  \sem{\NIL} &= \sSTOP
  \\
  \sem{\THREAD{\aCmd}} &= \sTHREAD{}\sem{\aCmd}
  \\
  \sem{\aGrp \PAR \bGrp} &= \sem{\aGrp}\sPAR\sem{\bGrp}
\end{align*}
[Examples.]


[Skolemization ensures disjunction closure, which is necessary
for associativity. Show example.]

