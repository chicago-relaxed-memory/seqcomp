\section{\PwTmcaTITLE{} Results}
\label{sec:results}

Prop.~6.1 of \citet{DBLP:journals/pacmpl/JagadeesanJR20} establishes a
compositional principle for proving that programs validate formula in
past-time temporal logic.  The principal is based entirely on the pomset
order relation.  Its proof, and all of the no-thin-air examples in
\cite[\textsection6]{DBLP:journals/pacmpl/JagadeesanJR20} hold equally for
the models described here.

% We \PwP{} %\cite{DBLP:journals/pacmpl/JagadeesanJR20}
% is a novel memory model, intended to serve as a semantic basis for a safe
% languages such as Java and Javascript.  \PwTmca{} generalizes \PwP{}, making
% several small but significant changes.  As a result, we have had to re-prove
% most of the theorems from \PwP{}.  These can be found in proofs can be found
% in \refAppendices{}.

In \refAppendix{sec:arm}, we show that \PwTmca{1} supports the optimal
lowering of relaxed accesses to \armeight{} and that \PwTmca{2} supports the
optimal lowering of \emph{all} accesses to \armeight{}.  The proofs are based
on two recent characterizations of \armeight{} \cite{armed}.  For \PwTmca{1},
we use \emph{External Global Consistency}.  For \PwTmca{2}, we use
\emph{External Consistency}.

In \refAppendix{sec:sc}, we also sketch a proof of sequential consistency for
local-data-race-free programs.  The proof uses \emph{program order}, which we
construct for \cXI{} in \textsection\ref{sec:c11}.  The same construction
works for \PwTmca{}.  (This proof sketch assumes there are no \RMW{} operations.)

The semantics validates many peephole optimizations, such as reorderings on relaxed access:
\begin{align*}
  %\taglabel{RR}
  \sembase{\PR{\aLoc}{\aReg} \SEMI \PR{\bLoc}{\bReg}} &=
  \sembase{\PR{\bLoc}{\bReg}\SEMI \PR{\aLoc}{\aReg}} &&\text{if } \aReg\neq\bReg
  \\
  %\taglabel{WW}
  \sembase{\aLoc \GETS \aExp \SEMI \bLoc  \GETS \bExp} &=
  \sembase{\bLoc  \GETS \bExp\SEMI \aLoc \GETS \aExp} &&\text{if } \aLoc\neq\bLoc
  \\
  %\taglabel{RW}
  \sembase{\aLoc \GETS \aExp  \SEMI \PR{\bLoc}{\bReg}} &=
  \sembase{\PR{\bLoc}{\bReg} \SEMI\aLoc \GETS \aExp} &&\text{if }
  \aLoc\neq\bLoc \textand \bReg\not\in\free(\aExp)%\disjoint{{\free(\aLoc \GETS \aExp)}}{{\free(\PR{\bLoc}{\bReg})}}
\end{align*}
% \ref{WW}, \ref{RW} and \ref{RR} require that two sides of the semicolon
% have disjoint ids; for example, \ref{RW} requires $\disjoint{{\free(\aReg
% \GETS \aLoc)}}{{\free(\bLoc \GETS \bExp)}}$. 
% \ref{RR} requires either $\aReg\neq\bReg$ or
% $\aLoc=\bLoc$.  \ref{WW} and \ref{RW} require that two sides of the
% semicolon have disjoint ids; for example, \ref{RW} requires
% $\disjoint{{\free(\PR{\aLoc}{\aReg})}}{{\free(\bLoc \GETS \bExp)}}$.
%\ref{5} imposes no order between events in \ref{RR}--\ref{RW}.  %Note that \ref{RR} allows aliasing.
Here $\free(\aExp)$ is the set of locations and registers that occur in $\aExp$.
Using augmentation closure, the semantics also validates roach-motel reorderings \cite{SevcikThesis}.  For
example, on read/write pairs:
\begin{align*}
  % \tag{\textsc{roach1}}\label{AcqW}
  \sembase{x^\amode \GETS \aExp \SEMI\PR{y}{\bReg}} &\supseteq
  \sembase{\PR{y}{\bReg}  \SEMI x^\amode\GETS \aExp} 
  &&\text{if }
  \aLoc\neq\bLoc \textand \bReg\not\in\free(\aExp)%\disjoint{{\free(\aLoc \GETS \aExp)}}{{\free(\PR{\bLoc}{\bReg})}}
  \\
  % \tag{\textsc{roach2}}\label{RelW}
  \sembase{x \GETS \aExp \SEMI\PR[\amode]{y}{\bReg}} &\supseteq
  \sembase{\PR[\amode]{y}{\bReg}  \SEMI x\GETS \aExp} 
  &&\text{if }
  \aLoc\neq\bLoc \textand \bReg\not\in\free(\aExp)%\disjoint{{\free(\aLoc \GETS \aExp)}}{{\free(\PR{\bLoc}{\bReg})}}
\end{align*}
Notably, the semantics does \emph{not} validate read-introduction.  When
combined with if-introduction (\textsection\ref{sec:semca}), read-introduction
can break temporal reasoning.  This combination is allowed by speculative
operational models.  See \textsection\ref{sec:related} for a discussion.

% As expected, %sequential and
% parallel composition commutes with conditionals and declarations, and
% conditionals and declarations commute with each other.  For example,
% we have \emph{scope extrusion}~\cite{Milner:1999:CMS:329902}:
% \begin{align*}
%   \taglabel{SE}
%   \sem{\aCmd\PAR \VAR\aLoc\SEMI\bCmd} &=
%   \sem{\VAR\aLoc\SEMI(\aCmd\PAR\bCmd)}
%   &&\text{if } \aLoc\not\in\free(\aCmd)
% \end{align*}

