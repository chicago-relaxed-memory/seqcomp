\section{Additional Examples}

\subsection{Arm}
The following execution is allowed by Arm.
\begin{gather*}
  {
    \PW{x}{1}
    \SEMI
    \PW[\mREL]{y}{1}
  }\PAR{
    \PR{y}{r}
    \SEMI
    \PW{y}{2}
    \SEMI
    \PR[\mACQ]{y}{s}
     \SEMI
    \PR{x}{t}
  }
  \\
  \hbox{\begin{tikzinline}[node distance=1.5em]
      \event{a}{\DW{x}{1}}{}
      \raevent{b}{\DW[\mREL]{y}{1}}{right=of a}
      \event{c}{\DR{y}{1}}{right=3em of b}
      \event{d}{\DW{y}{2}}{right=of c}
      \raevent{e}{\DR[\mACQ]{y}{2}}{right=of d}
      \event{f}{\DR{x}{0}}{right=of e}
      \lob{a}{b}
      \rfx{b}{c}
      %\sync[out=15,in=165]{c}{e}
      \lob{c}{d}
      \rfx{d}{e}
      \lob{e}{f}
      \fr[out=-165,in=-15]{f}[above,pos=.45]{a}
      %\close[out=-15,in=-165]{b}{e}
    \end{tikzinline}}
  \\
  \tag{$\rgcb$}
  \hbox{\begin{tikzinline}[node distance=1.5em]
      \event{a}{\DW{x}{1}}{}
      \raevent{b}{\DW[\mREL]{y}{1}}{right=of a}
      \event{c}{\DR{y}{1}}{right=3em of b}
      \event{d}{\DW{y}{2}}{right=of c}
      \raevent{e}{\DR[\mACQ]{y}{2}}{right=of d}
      \event{f}{\DR{x}{0}}{right=of e}
      \gcbz{a}{b}
      \gcbz{b}{c}
      \gcbz{c}{d}
      \gcbz{d}{e}
      %\gcbz{e}{f}
      \gcbz[out=-165,in=-15]{f}{a}
    \end{tikzinline}}
  \\
  \tag{$\rcb$}
  \hbox{\begin{tikzinline}[node distance=1.5em]
      \event{a}{\DW{x}{1}}{}
      \raevent{b}{\DW[\mREL]{y}{1}}{right=of a}
      \event{c}{\DR{y}{1}}{right=3em of b}
      \event{d}{\DW{y}{2}}{right=of c}
      \raevent{e}{\DR[\mACQ]{y}{2}}{right=of d}
      \event{f}{\DR{x}{0}}{right=of e}
      \cbz{a}{b}
      \cbz{b}{c}
      \cbz{c}{d}
      %\cbz{d}{e}
      \cbz{e}{f}
      \cbz[out=-165,in=-15]{f}{a}
    \end{tikzinline}}
\end{gather*}

\subsection{RMWs}
It is not possible for two \RMW{}s to see the same write.
\begin{gather*}
  \begin{gathered}
    \PW{x}{0} \SEMI \bigl(\PFADD[\mRLX][\mRLX]{x}{}{1} \PAR \PFADD[\mRLX][\mRLX]{x}{}{1}\bigr)
    \\
    \hbox{\begin{tikzinline}[node distance=2em]
        \event{a0}{\DW{x}{0}}{}
        \event{a1}{\DR{x}{0}}{right=3em of a0}
        \event{a2}{\DW{x}{1}}{right=of a1}
        \event{b1}{\DR{x}{0}}{right=3em of a2}
        \event{b2}{\DW{x}{1}}{right=of b1}
        \rmw{a1}{a2}
        \rf{a0}{a1}
        \rf[out=-15,in=-165]{a0}{b1}
        \wk[out=-15,in=-165]{a1}{b2}
        \wk{b1}{a2}
        \graywk[bend left]{a2}{b1}
        \rmw{b1}{b2}
      \end{tikzinline}}
  \end{gathered}
  \taglabel{rmw0}
\end{gather*}
The gray arrow is required the \RMW{} atomicity axioms.

\citet{DBLP:conf/pldi/LeeCPCHLV20} introduce \PS{2.0} to refine the treatment of
\RMW{}s in the promising semantics (\PS{}).  Their examples have the expected
results here, with far less work.  First they recall that \PS{} requires
quantification over multiple futures in order to disallow executions such as
\ref{CDRF}:
\begin{gather*}
  \taglabel{CDRF}
    \begin{gathered}
      \PFADD[\mACQ][\mREL]{x}{r}{1}\SEMI \IF{r{=}0}\THEN \PW{y}{1} \FI
      \PAR
      \PFADD[\mACQ][\mREL]{x}{r}{1}\SEMI \IF{r{=}0}\THEN \IF{y}\THEN \PW{x}{0} \FI \FI
      \\
      \hbox{\begin{tikzinline}[node distance=2em]
          \event{a1}{\DR[\mACQ]{x}{0}}{}
          \event{a1b}{\DW[\mREL]{x}{1}}{below=1em of a1}
          \event{a2}{\DW{y}{1}}{right=of a1}
          \event{b0}{\DR[\mACQ]{x}{0}}{right=3em of a2}
          \event{b0b}{\DW[\mREL]{x}{1}}{below=1em of b0}
          \event{b1}{\DR{y}{1}}{right=of b0}
          \event{b2}{\DW{x}{0}}{right=of b1}
          \rmw{a1}{a1b}
          \rmw{b0}{b0b}
          \rf[out=-13,in=-163]{a2}{b1}
          \po{a1}{a2}
          \sync{b0}{b1}
          \po{b1}{b2}
          \rf[out=-165,in=-12]{b2}{a1}
        \end{tikzinline}}
    \end{gathered}
  \end{gather*}
This execution is clearly impossible, due to the cycle above.  In this
diagram, we have not drawn order adjacent to the writes of the \RMW{}s, since
this is not necessary to produce the cycle.
If \ref{CDRF} is allowed then \drfra{} fails.


  
\PS{} does not support global value range analysis, as modeled by \ref{GA+E} below.  Our
semantics permits \ref{GA+E}:
\begin{gather*}
  \taglabel{GA+E}
    \begin{gathered}
      \PW{x}{0} \SEMI
      \bigl(
        \PCAS[\mRLX][\mRLX]{x}{r}{0}{1}\SEMI \IF{r{<}10}\THEN \PW{y}{1} \FI
        \PAR
        \PW{x}{42}\SEMI \PW{x}{y}
      \bigr)
      \\
      \hbox{\begin{tikzinline}[node distance=2em]
          \event{a0}{\DW{x}{0}}{}
          \event{a1}{\DR{x}{1}}{right=3em of a0}
          \event{a2}{0{<}10\mid\DW{y}{1}}{right=of a1}
          \event{b0}{\DW{x}{42}}{right=3em of a2}
          \event{b1}{\DR{y}{1}}{right=of b0}
          \event{b2}{\DW{x}{1}}{right=of b1}
          %\rmw{a1}{a2}
          \rf[out=-15,in=-160]{a2}{b1}
          \po{b1}{b2}
          \rf[out=-165,in=-15]{b2}{a1}
          \wk[out=10,in=170]{a0}{b0}
          \wk[out=15,in=165]{b0}{b2}
        \end{tikzinline}}
    \end{gathered}
\end{gather*}
\PS{} also does not support register promotion, as modeled by \ref{RP} below.    Our
semantics permits \ref{RP}:
\begin{gather*}
  \taglabel{RP}
    \begin{gathered}
      \PR{x}{r}\SEMI
      \PFADD[\mRLX][\mRLX]{z}{s}{r}\SEMI \PW{y}{s{+}1}
      \PAR
      \PW{x}{y}
      \\
      \hbox{\begin{tikzinline}[node distance=2em]
          \event{a0}{\DR{x}{1}}{}
          \event{a1}{\DR{z}{0}}{right=of a0}
          \event{a1b}{\DW{z}{1}}{right=of a1}
          \event{a2}{\DW{y}{1}}{right=of a1b}
          \event{b0}{\DR{y}{1}}{right=3em of a2}
          \event{b1}{\DW{x}{1}}{right=of b0}
          \rmw{a1}{a1b}
          \po[out=20,in=160]{a0}{a1b}
          \po[out=20,in=160]{a1}{a2}
          \po{b0}{b1}
          \rf{a2}{b0}
          \rf[out=-165,in=-15]{b1}{a0}
        \end{tikzinline}}
    \end{gathered}
\end{gather*}



These following examples are from ``Modular Data-Race-Freedom Guarantees in
the Promising Semantics'' to appear in PLDI21.

\ref{CDRF} shows that our semantics is not too permissive for $\mRA$-\RMW{}s.
But what about $\mRLX$-\RMW{}s.  The following execution is allowed by \armeight,
and \PS{2.0}, but disallowed by \PS{2.1}.
\begin{gather*}
  \taglabel{RMW-W}
  \begin{gathered}
    \PFADD[\mRLX][\mRLX]{x}{r}{1}\SEMI \PW{y}{1}
    \PAR
    \PR{y}{r}\SEMI \PFADD[\mRLX][\mRLX]{x}{s}{r}
    \\
    \hbox{\begin{tikzinline}[node distance=2em]
        \event{a1}{\DR{x}{1}}{}
        \event{a1b}{\DW{x}{2}}{below=1em of a1}
        \event{a2}{\DW{y}{1}}{right=of a1}
        \event{b1}{\DR{y}{1}}{right=3em of a2}
        \event{b2}{\DR{x}{0}}{right=of b1}
        \event{b2b}{\DW{x}{1}}{below=1em of b2}
        \rmw{a1}{a1b}
        \rmw{b2}{b2b}
        \rf{a2}{b1}
        \po{b1}{b2b}
        \rf[out=-175,in=-20]{b2b}{a1}
      \end{tikzinline}}
  \end{gathered}
\end{gather*}

If this $\ldrfra{z}$?
\begin{gather*}
  \taglabel{Naive-LDRF-RA-Fail}
  \begin{gathered}
    \IF{y}\THEN \PW{x}{z} \ELSE \PW{x}{1} \FI
    \PAR
    \PR{x}{r}\SEMI \PW{z}{1}\SEMI \PW{y}{r}
    \\
    \hbox{\begin{tikzinline}[node distance=2em]
        \event{a1}{\DR{y}{1}}{}
        \event{a2}{\DR{z}{1}}{right=of a1}
        \event{a3}{\DW{x}{1}}{right=of a2}
        \event{b1}{\DR{x}{1}}{right=3em of a3}
        \event{b2}{\DW{z}{1}}{right=of b1}
        \event{b3}{\DW{y}{1}}{right=of b2}
        \po{a2}{a3}
        \po[in=165,out=15]{b1}{b3}
        \rf[out=-165,in=-15]{b2}{a2}
        \rf[out=-165,in=-15]{b3}{a1}
        \rf{a3}{b1}
      \end{tikzinline}}
  \end{gathered}
\intertext{Interpreting $\{z\}$ as $\mRA$:}
    \\
  \begin{gathered}
    \hbox{\begin{tikzinline}[node distance=2em]
        \event{a1}{\DR{y}{1}}{}
        \event{a2}{\DR[\mACQ]{z}{1}}{right=of a1}
        \event{a3}{\DW{x}{1}}{right=of a2}
        \event{b1}{\DR{x}{1}}{right=3em of a3}
        \event{b2}{\DW[\mREL]{z}{1}}{right=of b1}
        \event{b3}{\DW{y}{1}}{right=of b2}
        \po{a2}{a3}
        \po[in=165,out=15]{b1}{b3}
        \rf[out=-165,in=-15]{b2}{a2}
        \rf[out=-165,in=-15]{b3}{a1}
        \rf{a3}{b1}
        \sync{a1}{a2}
        \sync{b2}{b3}
      \end{tikzinline}}
  \end{gathered}
\end{gather*}

Our semantics already disallows \ref{LDRF-Fail-PS}, which is similar to \ref{OOTA4}.
\begin{gather*}  
  \taglabel{LDRF-Fail-PS}
  \begin{gathered}
  \IF{x}\THEN
    \PFADD{w}{}{1}\SEMI
    \PW{y}{1}\SEMI
    \PW{z}{1}
  \FI
  \PAR
  \IF{\BANG z}\THEN
    \PW{x}{1}
  \ELSE
    \IF{\BANG \PFADD{w}{}{1}}\THEN
      \PW{x}{\PR{y}{}}
    \FI
  \FI
    \\
    \hbox{\begin{tikzinline}[node distance=2em]
        \event{a1}{\DR{x}{1}}{}
        \event{a2}{\DR{w}{1}}{right=of a1}
        \event{a3}{\DW{w}{2}}{right=of a2}
        \event{a4}{\DW{y}{1}}{right=of a3}
        \event{a5}{\DW{z}{1}}{right=of a4}
        \event{b1}{\DR{z}{1}}{right=5em of a5}
        \event{b2}{\DR{w}{0}}{right=of b1}
        \event{b3}{\DW{w}{1}}{right=of b2}
        \event{b4}{\DR{y}{1}}{right=of b3}
        \event{b5}{\DW{x}{1}}{right=of b4}
        \rmw{a2}{a3}
        \po[out=15,in=165]{a1}{a3}
        \po[out=15,in=165]{a1}{a4}
        \po[out=15,in=165]{a1}{a5}        
        \rmw{b2}{b3}
        \po{b4}{b5}
        \po[out=15,in=165]{b2}{b5}        
        \po[out=15,in=165]{b1}{b3}
        \rf{a5}{b1}
        \rf[out=15,in=165]{a4}{b4}
        \rf[out=-165,in=-15]{b3}{a2}
        \rf[out=-165,in=-15]{b5}{a1}
      \end{tikzinline}}
  \end{gathered}
\end{gather*}
\begin{gather}
  \taglabel{OOTA4}
  \begin{gathered}
    \PW{y}{x}
    \PAR
    \PR{y}{r} \SEMI \IF{b}\THEN  \PW{x}{r} \SEMI \PW{z}{r} \ELSE \PW{x}{1} \FI
    \PAR
    \PW{b}{1}
    \\[-1ex]
    \hbox{\begin{tikzinline}[node distance=1.5em]
        \event{rx}{\DR{x}{1}}{}
        \event{wy}{\DW{y}{1}}{right=of rx}
        \po{rx}{wy}
        \event{ry}{\DR{y}{1}}{right=3em of wy} 
        \event{wx}{\DW{x}{1}}{right=of ry}
        \event{wz}{\DW{z}{1}}{right=of wx}
        \event{rb}{\DR{b}{1}}{right=of wz}
        \event{wb1}{\DW{b}{1}}{right=3em of rb}
        \po{ry}{wx}
        \rf{wb1}{rb}
        \rf{wy}{ry}
        \rf[out=-170,in=-10]{wx}{rx}
        \po{rb}{wz}
        \po[out=15,in=165]{ry}{wz}
      \end{tikzinline}}
  \end{gathered}  
\end{gather}
\begin{example}
  This definition ensures atomicity, disallowing executions such as
  \cite[Ex.~3.2]{DBLP:journals/pacmpl/PodkopaevLV19}:
  \begin{gather*}
    % \taglabel{RMW1}
    \begin{gathered}
      \PW{x}{0}\SEMI \PINC[\mRLX][\mRLX]{x}{}
      \PAR
      \PW{x}{2}\SEMI \PR{x}{r}
      \\
      \hbox{\begin{tikzinline}[node distance=1.5em]
          \event{a2}{\DR{x}{0}}{}
          \event{a1}{\DW{x}{0}}{left=of a2}
          \rf{a1}{a2}
          \event{a3}{\DW{x}{2}}{right=of a2}
          \wk{a2}{a3}
          \event{b2}{\DW{x}{1}}{right=of a3}
          \event{b3}{\DR{x}{1}}{right=of b2}
          \rmw[out=-15,in=-165]{a2}[below]{b2}
          \wk{a3}{b2}
          \rf{b2}{b3}
          \liftrmw[out=165,in=15]{a3}{a2}
        \end{tikzinline}}
    \end{gathered}
  \end{gather*}
  By \ref{pom-rmw-atomic1}, since $\DWP{x}{2}\xwk\DWP{x}{1}$, it must be that
  $\DWP{x}{2}\xwk\DRP{x}{0}$, creating a cycle.
\end{example}

\begin{example}
  \label{ex:rmw-33}
  Two successful \RMW{}s cannot see the same write:
  \begin{gather*}
    \begin{gathered}
      \PW{x}{0}\SEMI (\PINC[\mRLX][\mRLX]{x}{} \PAR \PINC[\mRLX][\mRLX]{x}{})
      \\
      \hbox{\begin{tikzinline}[node distance=1.5em]
          \event{i}{\DW{x}{0}}{}
          \event{a1}{a{:}\DR{x}{0}}{right=3em of i}
          \event{a2}{b{:}\DW{x}{1}}{right=of a1}
          \event{b1}{c{:}\DR{x}{0}}{right=3em of a2}
          \event{b2}{d{:}\DW{x}{1}}{right=of b1}
          \rmw{a1}{a2}
          \rmw{b1}{b2}
          \rf{i}{a1}
          \rf[out=15,in=165]{i}{b1}
          \wk[out=-15,in=-165]{a1}{b2}
          \liftrmw[out=-15,in=-165]{a2}{b1}
          % \wk{a1}{b2}
          \wk{b1}{a2}
        \end{tikzinline}}
    \end{gathered}
  \end{gather*}
  The order from read-to-write is required by fulfillment.  
  Apply \ref{pom-rmw-atomic1} of the second \RMW{} to $a\xwk d$, we have that $a\xwk c$.  Subsequently
  applying \ref{pom-rmw-atomic2} of the first \RMW{}, we have $b \xwk c$, creating a cycle.
\end{example}

\begin{example}
  By using two actions rather than one, the definition allows examples such as the
  following, which is allowed by \armeight{} 
  \cite[Ex.~3.10]{DBLP:journals/pacmpl/PodkopaevLV19}:
  \begin{gather*}
    % \taglabel{RMW2}
    \begin{gathered}
      \PR{z}{r}\SEMI
      % \PW{x}{0}\SEMI
      \PINC[\mRLX][\mREL]{x}{s} \SEMI
      \PW{y}{s}{+}1
      \PAR
      \PR{y}{r}\SEMI
      \PW{z}{r}
      \\[-1ex]
      \hbox{\begin{tikzinline}[node distance=1.5em]
          \event{b1}{\DR{z}{1}}{}
          % \event{b2}{\DW{x}{0}}{right=of b1}
          \event{b3}{\DR{x}{0}}{right=of b1}
          %\rf{b2}{b3}
          \event{b4}{\DWRel{x}{1}}{right=2em of b3}
          \rmw{b3}{b4}
          \event{b5}{\DW{y}{1}}{right=of b4}
          \sync[out=-20,in=-160]{b1}{b4}
          \po[out=-20,in=-160]{b3}{b5}
          \event{a1}{\DR{y}{1}}{right=3em of b5}
          \event{a2}{\DW{z}{1}}{right=of a1}
          \po{a1}{a2}
          \rf{b5}{a1}
          \rf[out=170,in=10]{a2}{b1}
        \end{tikzinline}}
    \end{gathered}
  \end{gather*}
  A similar example, also allowed by \armeight{}
  \cite[Fig.~6]{DBLP:journals/pacmpl/ChakrabortyV19}:
  \begin{gather*}
    % \taglabel{RMW2}
    \begin{gathered}
      \PR{z}{r}\SEMI
      % \PW{x}{0}\SEMI
      \PFADD[\mRLX][\mRLX]{x}{s}{r} \SEMI
      \PW{y}{s}{+}1
      \PAR
      \PR{y}{r}\SEMI
      \PW{z}{r}
      \\[-1ex]
      \hbox{\begin{tikzinline}[node distance=1.5em]
          \event{b1}{\DR{z}{1}}{}
          %\event{b2}{\DW{x}{0}}{right=of b1}
          \event{b3}{\DR{x}{0}}{right=of b1}
          %\rf{b2}{b3}
          \event{b4}{\DW{x}{1}}{right=2em of b3}
          \rmw{b3}{b4}
          \event{b5}{\DW{y}{1}}{right=of b4}
          \po[out=-20,in=-160]{b1}{b4}
          \po[out=-20,in=-160]{b3}{b5}
          \event{a1}{\DR{y}{1}}{right=3em of b5}
          \event{a2}{\DW{z}{1}}{right=of a1}
          \po{a1}{a2}
          \rf{b5}{a1}
          \rf[out=170,in=10]{a2}{b1}
        \end{tikzinline}}
    \end{gathered}
  \end{gather*}
\end{example}
This is allowed by \weakestmo{}, but not \PS{}.

\begin{example}
  Consider the \textsc{cdrf} example from \cite{DBLP:conf/pldi/LeeCPCHLV20}:
  \begin{gather*}
    \begin{gathered}
      \begin{aligned}
        &\PINC[\mACQ][\mREL]{x}{r}\SEMI \IF{r{=}0}\THEN \PW{y}{1} \FI
        \\\PAR\;\;&
        \PINC[\mACQ][\mREL]{x}{r}\SEMI \IF{r{=}0}\THEN \IF{y}\THEN \PW{x}{0} \FI \FI
      \end{aligned}
      \\
      \hbox{\footnotesize\begin{tikzinline}[node distance=1.5em]
          \raevent{a1}{\DR[\mACQ]{x}{0}}{}
          \raevent{a1b}{\DW[\mREL]{x}{1}}{right=of a1}
          \event{a2}{\DW{y}{1}}{right=of a1b}
          \raevent{b0}{\DR[\mACQ]{x}{0}}{right=3em of a2}
          \raevent{b0b}{\DW[\mREL]{x}{1}}{right=of b0}
          \event{b1}{\DR{y}{1}}{right=of b0b}
          \event{b2}{\DW{x}{0}}{right=of b1}
          \rmw{a1}{a1b}
          \rmw{b0}{b0b}
          \rf[out=-13,in=-163]{a2}{b1}
          \sync[out=20,in=160]{a1}{a2}
          \sync[out=20,in=160]{b0}{b1}
          \po{b1}{b2}
          \rf[out=-165,in=-12]{b2}{a1}
        \end{tikzinline}}
    \end{gathered}
  \end{gather*}
\end{example}

\begin{example}
  Consider this example from \cite[\textsection C]{DBLP:conf/pldi/LeeCPCHLV20}:
  \begin{gather*}
    \begin{gathered}
      \begin{aligned}
        &\PCAS[\mRLX][\mRLX]{x}{r}{0}{1}\SEMI \IF{r{\leq}1}\THEN \PW{y}{1} \FI
        \\\PAR\;\;&
        \PCAS[\mRLX][\mRLX]{x}{r}{0}{2}\SEMI \IF{r{=}0}\THEN \IF{y}\THEN \PW{x}{0} \FI \FI
      \end{aligned}
      \\
      \hbox{\footnotesize\begin{tikzinline}[node distance=1.5em]
          \event{a1}{\DR{x}{0}}{}
          \event{a1b}{\DW{x}{1}}{right=of a1}
          \event{a2}{\DW{y}{1}}{right=of a1b}
          \event{b0}{\DR{x}{0}}{right=3em of a2}
          \event{b0b}{\DW{x}{2}}{right=of b0}
          \event{b1}{\DR{y}{1}}{right=of b0b}
          \event{b2}{\DW{x}{0}}{right=of b1}
          \rmw{a1}{a1b}
          \rmw{b0}{b0b}
          \rf[out=-13,in=-163]{a2}{b1}
          \po[out=20,in=160]{a1}{a2}
          \po[out=20,in=160]{b0}{b1}
          \po{b1}{b2}
          \rf[out=-165,in=-12]{b2}{a1}
        \end{tikzinline}}
    \end{gathered}
  \end{gather*}
\end{example}

\subsection{Coherence}

The following execution is disallowed by fulfillment.
\begin{gather*}
  \tag{\textsc{coh}}
  \begin{gathered}
    \PW{x}{1}\SEMI
    \PR{x}{r}
    \PAR
    \PW{x}{2}\SEMI
    \PR{x}{s}
    \\\nonumber
    \hbox{\begin{tikzinline}[node distance=1.5em]
        \event{a1}{\DW{x}{1}}{}
        \event{a2}{\DR{x}{2}}{right=of a1}
        \event{b1}{\DW{x}{2}}{right=3em of a2}
        \event{b2}{\DR{x}{1}}{right=of b1}
        \wki{a1}{a2}
        \wki{b1}{b2}
        \rf{b1}{a2}
        \rf[out=20,in=160]{a1}{b2}
        \wk[out=15,in=155]{a1}{b1}
        \wk[out=-155,in=-15]{b1}{a1}
      \end{tikzinline}}
  \end{gathered}
\end{gather*}

Our model is more coherent than Java, which permits the following:
\begin{gather*}
  \taglabel{TC16}
  \begin{gathered}
    \PR{x}{r}\SEMI \PW{x}{1}
    \PAR
    \PR{x}{s}\SEMI \PW{x}{2}
    \\[-1ex]
    \hbox{\begin{tikzinline}[node distance=1.5em]
        \event{a1}{\DR{x}{2}}{}
        \event{a2}{\DW{x}{1}}{right=of a1}
        \wki{a1}{a2}
        \event{b1}{\DR{x}{1}}{right=3em of a2}
        \event{b2}{\DW{x}{2}}{right=of b1}
        \wki{b1}{b2}
        \rf{a2}{b1}
        \rf[out=-165,in=-15]{b2}{a1}
      \end{tikzinline}}
  \end{gathered}
\end{gather*}
We also forbid the following, which Java allows:
\begin{gather*}
  \taglabel{Co3}
  \begin{gathered}
    \PW{x}{1}\SEMI \PW[\mRA]{y}{1}
    \PAR
    \PW{x}{2}\SEMI \PW[\mRA]{z}{1}
    \PAR
    \PR[\mRA]{z}{r} \SEMI 
    \PR[\mRA]{y}{r} \SEMI 
    \PR{x}{r} \SEMI 
    \PR{x}{r}
    \\[-1ex]
    \hbox{\begin{tikzinline}[node distance=1.5em]
        \event{a1}{\DW{x}{1}}{}
        \event{a2}{\DW[\mRA]{y}{1}}{right=of a1}
        \sync{a1}{a2}
        \event{b1}{\DW{x}{2}}{right=3em of a2}
        \event{b2}{\DW[\mRA]{\,z}{1}}{right=of b1}
        \sync{b1}{b2}
        \event{c1}{\DR[\mRA]{\,z}{1}}{right=3em of b2}
        \event{c2}{\DR[\mRA]{y}{1}}{right=of c1}
        \event{c3}{\DR{x}{2}}{right=of c2}
        \event{c4}{\DR{x}{1}}{right=of c3}
        \sync{c1}{c2}
        \sync{c2}{c3}
        \sync[out=20,in=160]{c2}{c4}
        \rf[out=8,in=172]{a2}{c2}
        \rf{b2}{c1}
        \wk[out=19,in=161]{a1}{b1}
        \wk[out=-172,in=-8]{c4}{b1}
      \end{tikzinline}}
  \end{gathered}
\end{gather*}


The following outcome is allowed by the promising semantics
\cite{DBLP:conf/popl/KangHLVD17}, but not in \weakestmo{}
\cite[Fig.~3]{DBLP:journals/pacmpl/ChakrabortyV19} nor in our semantics, due
to the cycle:
\begin{gather*}
  \tag{\textsc{coh-cyc}}
  \begin{gathered}
    x\GETS 2\SEMI
    \IF{x\NOTEQ2}\THEN y\GETS 1 \FI
    \PAR
    x\GETS 1\SEMI
    r\GETS x\SEMI
    \IF{y}\THEN x\GETS 3 \FI
    \\\nonumber
    \hbox{\begin{tikzinline}[node distance=1.5em]
        \event{wx2}{\DW{x}{2}}{}
        \event{rx3}{\DR{x}{3}}{right=of wx2}
        \wki{wx2}{rx3}
        \event{wy1}{\DW{y}{1}}{right=of rx3}
        \po{rx3}{wy1}
        \event{wx1}{\DW{x}{1}}{right=2em of wy1}
        \event{rx2}{\DR{x}{2}}{right=of wx1}
        \wki{wx1}{rx2}
        \event{ry1}{\DR{y}{1}}{right=of rx2}
        \event{wx3}{\DW{x}{3}}{right=of ry1}
        \po{ry1}{wx3}
        \wki[in=165,out=15]{rx2}{wx3}
        \rf[in=-170,out=-10]{wy1}{ry1}
        \rf[in=170,out=10]{wx2}{rx2}
        \rf[out=-170,in=-10]{wx3}{rx3}
        \wk[out=-170,in=-10]{wx1}{wx2}
      \end{tikzinline}}
  \end{gathered}
\end{gather*}

Since reads are not ordered by intra-thread coherence,
we {allow} the following unintuitive behavior. C11 includes read-read
coherence between relaxed atomics in order to forbid this:
\begin{gather*}
  \taglabel{Co2}
  \begin{gathered}
    \PW{x}{1}\SEMI \PW{x}{2}
    \PAR
    \PW{y}{x} \SEMI \PW{z}{x}
    \\[-1ex]
    \hbox{\begin{tikzinline}[node distance=1.5em]
        \event{a}{\DW{x}{1}}{}
        \event{b}{\DW{x}{2}}{right=of a}
        \wki{a}{b}
        \event{c}{\DR{x}{2}}{right=3em of b}
        \event{d}{\DW{y}{2}}{right=of c}
        \po{c}{d}
        \event{e}{\DR{x}{1}}{right=of d}
        \event{f}{\DW{z}{1}}{right=of e}
        \po{e}{f}
        \rf{b}{c}
        \rf[out=10,in=170]{a}{e}
        \wk[out=-165,in=-15]{e}{b}
      \end{tikzinline}}
  \end{gathered}
\end{gather*}
Here, the reader sees $2$ then $1$, although they are written in the reverse
order.
This behavior is allowed by Java in order to validate CSE without requiring
aliasing analysis.

\subsection{MCA}

\begin{gather*}
  \taglabel{MCA1}
  \begin{gathered}
    \IF{z}\THEN \PW{x}{0} \FI \SEMI \PW{x}{1}
    {\PAR}
    \IF{x}\THEN \PW{y}{0} \FI \SEMI \PW{y}{1}
    {\PAR}
    \IF{y}\THEN \PW{z}{0} \FI \SEMI \PW{z}{1}
    \\[-1ex]
    \hbox{\begin{tikzinline}[node distance=1.5em]
        \event{a1}{\DR{z}{1}}{}
        \event{a2}{\DW{x}{0}}{right=of a1}
        \po{a1}{a2}
        \event{a3}{\DW{x}{1}}{right=of a2}
        \wki{a2}{a3}
        \event{b1}{\DR{x}{1}}{right=3em of a3}
        \event{b2}{\DW{y}{0}}{right=of b1}
        \po{b1}{b2}
        \event{b3}{\DW{y}{1}}{right=of b2}
        \wki{b2}{b3}
        \event{c1}{\DR{y}{1}}{right=3em of b3}
        \event{c2}{\DW{z}{0}}{right=of c1}
        \po{c1}{c2}
        \event{c3}{\DW{z}{1}}{right=of c2}
        \wki{c2}{c3}
        \rf{a3}{b1}
        \rf{b3}{c1}
        \rf[out=173,in=7]{c3}{a1}  
      \end{tikzinline}}
  \end{gathered}
  \\[1ex]
  \taglabel{MCA2}
  \begin{gathered}
    \PW{x}{0}\SEMI \PW{x}{1}
    \PAR
    \PW{y}{x}
    \PAR
    \PR[\mRA]{y}{r} \SEMI \PR{x}{s}
    \\[-1ex]
    \hbox{\begin{tikzinline}[node distance=1.5em]
        \event{wx0}{\DW{x}{0}}{}
        \event{wx1}{\DW{x}{1}}{right=of wx0}
        \wki{wx0}{wx1}
        \event{rx1}{\DR{x}{1}}{right=3em of wx1}
        \event{wy1}{\DW{y}{1}}{right=of rx1}
        \po{rx1}{wy1}
        \event{ry1}{\DRAcq{y}{1}}{right=3em of wy1}
        \event{rx0}{\DR{x}{0}}{right=of ry1}
        \rf{wx1}{rx1}
        \rf{wy1}{ry1}
        \sync{ry1}{rx0}
        \wk[out=170,in=10]{rx0}{wx1}
      \end{tikzinline}}
  \end{gathered}
\end{gather*}

These candidate executions are invalid, due to cycles.

\subsection{IRIW}
Status of IRIW is unclear in our model, since we allow everything allowed by
power...
\begin{gather*}
  \begin{gathered}
    % \PW{x}{0}\SEMI
    \PW{x}{1}
    \PAR
    \PR[\mRA]{x}{r}\SEMI \PR{y}{s}
    \PAR
    % \PW{y}{0}\SEMI
    \PW{y}{1}
    \PAR
    \PR[\mRA]{y}{s} \SEMI \PR{x}{r}
    \\
    %\smash[b]{
      \hbox{\begin{tikzinline}[node distance=1.5em]
          % \event{wx0}{\DW{x}{0}}{}
          % \event{wx1}{\DW{x}{1}}{right=of wx0}
          % \event{wy0}{\DW{y}{0}}{below=4ex of wx0}
          % \event{wy1}{\DW{y}{1}}{right=of wy0}
          \event{wx1}{\DW{x}{1}}{}
          \event{rx1}{\DR[\mRA]{x}{1}}{right=3em of wx1}
          \event{ry0}{\DR{y}{0}}{right=of rx1}
          \event{wy1}{\DW{y}{1}}{right=3em of ry0}
          \event{ry1}{\DR[\mRA]{y}{1}}{right=3em of wy1}
          \event{rx0}{\DR{x}{0}}{right=of ry1}
          % \wk{wx0}{wx1}
          % \wk{wy0}{wy1}
          % \rf[bend left]{wy0}{ry0}
          % \rf[bend right]{wx0}{rx0}
          \sync{rx1}{ry0}
          \sync{ry1}{rx0}
          \rf{wx1}{rx1}
          \rf{wy1}{ry1}
          \wk[out=170,in=10]{rx0}{wx1}
          \wk{ry0}{wy1}
        \end{tikzinline}}
    %}
  \end{gathered}
\end{gather*}


\section{A Note on Mixed-Mode Data Races}

In preparing this paper, we came across the following example, which appears
to invalidate Theorem 4.1 of \cite{DBLP:conf/ppopp/DongolJR19}.
\begin{gather}
  \nonumber
  \PW{x}{1}\SEMI
  \PW[\mREL]{y}{1}\SEMI
  \PR[\mACQ]{x}{r}
  \PAR
  \IF{\PR[\mACQ]{y}{}}\THEN \PW[\mREL]{x}{2}\FI
  \\
  \tag{\P}
  \label{mix1}
  \hbox{\begin{tikzinline}[node distance=1.5em]
      \event{a1}{\DW{x}{1}}{}
      \raevent{a2}{\DW[\mREL]{y}{1}}{right=of a1}
      \raevent{a3}{\DR[\mACQ]{x}{1}}{right=of a2}
      \raevent{b1}{\DR[\mACQ]{y}{1}}{right=3em of a3}
      % \raevent{b1}{\DR[\mACQ]{y}{1}}{below=of a1}
      \raevent{b2}{\DW[\mREL]{x}{2}}{right=of b1}
      \sync{a1}{a2}
      \rf[out=20,in=160]{a1}{a3}
      \rf[out=20,in=160]{a2}{b1}
      \wk[out=-20,in=-160]{a3}{b2}
      \sync{b1}{b2}
      % \node(ai)[left=3em of a1]{};
      % \bgoval[yellow!50]{(ai)}{P}
      % \bgoval[pink!50]{(a1)(a2)(b1)(b2)}{P'\setminus P}
      % \bgoval[green!10]{(a3)}{P'''\setminus P'}
    \end{tikzinline}}
  \\
  \nonumber
  %\label{mix2}
  \hbox{\begin{tikzinline}[node distance=1.5em]
      \event{a1}{\DW{x}{1}}{}
      \raevent{a2}{\DW[\mREL]{y}{1}}{right=of a1}
      \raevent{a3}{\DR[\mACQ]{x}{2}}{right=of a2}
      \raevent{b1}{\DR[\mACQ]{y}{1}}{right=3em of a3}
      \raevent{b2}{\DW[\mREL]{x}{2}}{right=of b1}
      \sync{a1}{a2}
      \rf[out=20,in=160]{a2}{b1}
      \rf[out=160,in=20]{b2}{a3}
      \sync{b1}{b2}
    \end{tikzinline}}
\end{gather}
The program is data-race free.  The two executions shown are the only
top-level executions that include $\DWP[\mREL]{x}{2}$.

Theorem 4.1 of \cite{DBLP:conf/ppopp/DongolJR19} is stated by extending
execution sequences.  In the terminology of
\cite{DBLP:conf/ppopp/DongolJR19}, a read is \emph{$L$-weak} if it is
sequentially stale.  Let
\begin{math}
  \rho=\DWP{x}{1}\allowbreak
  \DWP[\mREL]{y}{1}\allowbreak
  \DRP[\mACQ]{y}{1}\allowbreak
  \DWP[\mREL]{x}{2}
\end{math}
be a sequence and
\begin{math}
  \alpha=\DRP[\mACQ]{x}{1}.
\end{math}
$\rho$ is $L$-sequential and $\alpha$ is $L$-weak in $\rho\alpha$.  But there
is no execution of this program that includes a data race, contradicting the
theorem.  The error seems to be in Lemma A.4 of
\cite{DBLP:conf/ppopp/DongolJR19}, which states that if $\alpha$ is $L$-weak
after an $L$-sequential $\rho$, then $\alpha$ must be in a data race.  That
is clearly false here, since $\DRP[\mACQ]{x}{1}$ is stale, but the program is
data race free.

In proving the SC-LDRF result in \jjr{\textsection8}, we noted that our proof
technique is more robust than that of \cite{DBLP:conf/ppopp/DongolJR19},
because it limits the prefixes that must be considered.  In \eqref{mix1}, the
induction hypothesis requires that we add $\DRP[\mACQ]{x}{1}$ before
$\DWP[\mREL]{x}{2}$ since $\DRP[\mACQ]{x}{1}\xwk\DWP[\mREL]{x}{2}$.  In
particular,
\begin{gather*}
  \hbox{\begin{tikzinline}[node distance=1.5em]
      \event{a1}{\DW{x}{1}}{}
      \raevent{a2}{\DW[\mREL]{y}{1}}{right=of a1}
      % \raevent{a3}{\DR[\mACQ]{x}{1}}{right=of a2}
      \raevent{b1}{\DR[\mACQ]{y}{1}}{right=3em of a3}
      % \raevent{b1}{\DR[\mACQ]{y}{1}}{below=of a1}
      \raevent{b2}{\DW[\mREL]{x}{2}}{right=of b1}
      \sync{a1}{a2}
      % \rf[out=20,in=160]{a1}{a3}
      \rf[out=20,in=160]{a2}{b1}
      % \wk[out=-20,in=-160]{a3}{b2}
      \sync{b1}{b2}
      % \node(ai)[left=3em of a1]{};
      % \bgoval[yellow!50]{(ai)}{P}
      % \bgoval[pink!50]{(a1)(a2)(b1)(b2)}{P'\setminus P}
      % \bgoval[green!10]{(a3)}{P'''\setminus P'}
    \end{tikzinline}}
\end{gather*}
is not a downset of \eqref{mix1}, because
$\DRP[\mACQ]{x}{1}\xwk\DWP[\mREL]{x}{2}$.  As we noted in \jjr{\textsection8},
this affects the inductive order in which we move across pomsets, but does
not affect the set of pomsets that are considered.  In particular,
\begin{gather*}
  \hbox{\begin{tikzinline}[node distance=1.5em]
      \event{a1}{\DW{x}{1}}{}
      \raevent{a2}{\DW[\mREL]{y}{1}}{right=of a1}
      % \raevent{a3}{\DR[\mACQ]{x}{1}}{right=of a2}
      \raevent{b1}{\DR[\mACQ]{y}{1}}{right=3em of a3}
      % \raevent{b1}{\DR[\mACQ]{y}{1}}{below=of a1}
      % \raevent{b2}{\DW[\mREL]{x}{2}}{right=of b1}
      \sync{a1}{a2}
      % \rf[out=20,in=160]{a1}{a3}
      \rf[out=20,in=160]{a2}{b1}
      % \wk[out=-20,in=-160]{a3}{b2}
      % \sync{b1}{b2}
      % \node(ai)[left=3em of a1]{};
      % \bgoval[yellow!50]{(ai)}{P}
      % \bgoval[pink!50]{(a1)(a2)(b1)(b2)}{P'\setminus P}
      % \bgoval[green!10]{(a3)}{P'''\setminus P'}
    \end{tikzinline}}
\end{gather*}
is a downset of \eqref{mix1}.
