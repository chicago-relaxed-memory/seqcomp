\section{\PwTmcaTITLE{}: Additional Examples}
\label{sec:extras}

This appendix includes additional examples.  They all apply equally to
\PwTmca{1} and \PwTmca{2}.  Several of these are taken directly from \cite{DBLP:journals/pacmpl/JagadeesanJR20}.
% \subsection{Arm}
% The following execution is allowed by Arm.
% \begin{gather*}
%   {
%     \PW{x}{1}
%     \SEMI
%     \PW[\mREL]{y}{1}
%   }\PAR{
%     \PR{y}{r}
%     \SEMI
%     \PW{y}{2}
%     \SEMI
%     \PR[\mACQ]{y}{s}
%      \SEMI
%     \PR{x}{t}
%   }
%   \\
%   \hbox{\begin{tikzinline}[node distance=1.5em]
%       \event{a}{\DW{x}{1}}{}
%       \raevent{b}{\DW[\mREL]{y}{1}}{right=of a}
%       \event{c}{\DR{y}{1}}{right=3em of b}
%       \event{d}{\DW{y}{2}}{right=of c}
%       \raevent{e}{\DR[\mACQ]{y}{2}}{right=of d}
%       \event{f}{\DR{x}{0}}{right=of e}
%       \lob{a}{b}
%       \rfx{b}{c}
%       %\sync[out=15,in=165]{c}{e}
%       \lob{c}{d}
%       \rfx{d}{e}
%       \lob{e}{f}
%       \fr[out=-165,in=-15]{f}[above,pos=.45]{a}
%       %\close[out=-15,in=-165]{b}{e}
%     \end{tikzinline}}
%   \\
%   \tag{$\rgcb$}
%   \hbox{\begin{tikzinline}[node distance=1.5em]
%       \event{a}{\DW{x}{1}}{}
%       \raevent{b}{\DW[\mREL]{y}{1}}{right=of a}
%       \event{c}{\DR{y}{1}}{right=3em of b}
%       \event{d}{\DW{y}{2}}{right=of c}
%       \raevent{e}{\DR[\mACQ]{y}{2}}{right=of d}
%       \event{f}{\DR{x}{0}}{right=of e}
%       \gcbz{a}{b}
%       \gcbz{b}{c}
%       \gcbz{c}{d}
%       \gcbz{d}{e}
%       %\gcbz{e}{f}
%       \gcbz[out=-165,in=-15]{f}{a}
%     \end{tikzinline}}
%   \\
%   \tag{$\rcb$}
%   \hbox{\begin{tikzinline}[node distance=1.5em]
%       \event{a}{\DW{x}{1}}{}
%       \raevent{b}{\DW[\mREL]{y}{1}}{right=of a}
%       \event{c}{\DR{y}{1}}{right=3em of b}
%       \event{d}{\DW{y}{2}}{right=of c}
%       \raevent{e}{\DR[\mACQ]{y}{2}}{right=of d}
%       \event{f}{\DR{x}{0}}{right=of e}
%       \cbz{a}{b}
%       \cbz{b}{c}
%       \cbz{c}{d}
%       %\cbz{d}{e}
%       \cbz{e}{f}
%       \cbz[out=-165,in=-15]{f}{a}
%     \end{tikzinline}}
% \end{gather*}

\subsection{Coherence}

The following execution is disallowed by fulfillment (\ref{pom-rf-match} and
\ref{pom-rf-block}).
\begin{gather*}
  \tag{\textsc{coh}}
  \begin{gathered}
    \PW{x}{1}\SEMI
    \PR{x}{r}
    \PAR
    \PW{x}{2}\SEMI
    \PR{x}{s}
    \\\nonumber
    \hbox{\begin{tikzinline}[node distance=1.5em]
        \event{a1}{\DW{x}{1}}{}
        \event{a2}{\DR{x}{2}}{right=of a1}
        \event{b1}{\DW{x}{2}}{right=3em of a2}
        \event{b2}{\DR{x}{1}}{right=of b1}
        \wki{a1}{a2}
        \wki{b1}{b2}
        \rf{b1}{a2}
        \rf[out=20,in=160]{a1}{b2}
        \wk[out=15,in=155]{a1}{b1}
        %\wk[out=-155,in=-15]{b1}{a1}
      \end{tikzinline}}
  \end{gathered}
\end{gather*}
\ref{pom-rf-block} requires that we order one write with respect to the
other, either before the write or after the read (and therefore after the
write).  Suppose we pick $1$ before $2$, as shown.  This satisfies
\ref{pom-rf-block} for $\DRP{x}{2}$.  But to satisfy the requirement for
$\DRP{x}{1}$ we must have either $\DWP{x}{2}\le\DWP{x}{1}$ or
$\DRP{x}{1}\le\DWP{x}{2}$.   Either way, we have a cycle.

Our model is more coherent than Java, which permits the following:
\begin{gather*}
  \taglabel{TC16}
  \begin{gathered}
    \PR{x}{r}\SEMI \PW{x}{1}
    \PAR
    \PR{x}{s}\SEMI \PW{x}{2}
    \\[-1ex]
    \hbox{\begin{tikzinline}[node distance=1.5em]
        \event{a1}{\DR{x}{2}}{}
        \event{a2}{\DW{x}{1}}{right=of a1}
        \wki{a1}{a2}
        \event{b1}{\DR{x}{1}}{right=3em of a2}
        \event{b2}{\DW{x}{2}}{right=of b1}
        \wki{b1}{b2}
        \rf{a2}{b1}
        \rf[out=-165,in=-15]{b2}{a1}
      \end{tikzinline}}
  \end{gathered}
\end{gather*}
We also forbid the following, which Java allows:
\begin{gather*}
  \taglabel{Co3}
  \begin{gathered}
    \PW{x}{1}\SEMI \PW[\mRA]{y}{1}
    \PAR
    \PW{x}{2}\SEMI \PW[\mRA]{z}{1}
    \PAR
    \PR[\mRA]{z}{r} \SEMI 
    \PR[\mRA]{y}{r} \SEMI 
    \PR{x}{r} \SEMI 
    \PR{x}{r}
    \\[-1ex]
    \hbox{\begin{tikzinline}[node distance=1.5em]
        \event{a1}{\DW{x}{1}}{}
        \event{a2}{\DW[\mRA]{y}{1}}{right=of a1}
        \sync{a1}{a2}
        \event{b1}{\DW{x}{2}}{right=3em of a2}
        \event{b2}{\DW[\mRA]{\,z}{1}}{right=of b1}
        \sync{b1}{b2}
        \event{c1}{\DR[\mRA]{\,z}{1}}{right=3em of b2}
        \event{c2}{\DR[\mRA]{y}{1}}{right=of c1}
        \event{c3}{\DR{x}{2}}{right=of c2}
        \event{c4}{\DR{x}{1}}{right=of c3}
        \sync{c1}{c2}
        \sync{c2}{c3}
        \sync[out=20,in=160]{c2}{c4}
        \rf[out=8,in=172]{a2}{c2}
        \rf{b2}{c1}
        \wk[out=19,in=161]{a1}{b1}
        \wk[out=-172,in=-8]{c4}{b1}
      \end{tikzinline}}
  \end{gathered}
\end{gather*}


The following outcome is allowed by the promising semantics
\cite{DBLP:conf/popl/KangHLVD17}, but not in \weakestmo{}
\cite[Fig.~3]{DBLP:journals/pacmpl/ChakrabortyV19} nor in our semantics, due
to the cycle:
\begin{gather*}
  \tag{\textsc{coh-cyc}}
  \begin{gathered}
    x\GETS 2\SEMI
    \IF{x\NOTEQ2}\THEN y\GETS 1 \FI
    \PAR
    x\GETS 1\SEMI
    r\GETS x\SEMI
    \IF{y}\THEN x\GETS 3 \FI
    \\\nonumber
    \hbox{\begin{tikzinline}[node distance=1.5em]
        \event{wx2}{\DW{x}{2}}{}
        \event{rx3}{\DR{x}{3}}{right=of wx2}
        \wki{wx2}{rx3}
        \event{wy1}{\DW{y}{1}}{right=of rx3}
        \po{rx3}{wy1}
        \event{wx1}{\DW{x}{1}}{right=2em of wy1}
        \event{rx2}{\DR{x}{2}}{right=of wx1}
        \wki{wx1}{rx2}
        \event{ry1}{\DR{y}{1}}{right=of rx2}
        \event{wx3}{\DW{x}{3}}{right=of ry1}
        \po{ry1}{wx3}
        \wki[in=165,out=15]{rx2}{wx3}
        \rf[in=-170,out=-10]{wy1}{ry1}
        \rf[in=170,out=10]{wx2}{rx2}
        \rf[out=-170,in=-10]{wx3}{rx3}
        \wk[out=-170,in=-10]{wx1}{wx2}
      \end{tikzinline}}
  \end{gathered}
\end{gather*}

Since reads are not ordered by intra-thread coherence,
we {allow} the following unintuitive behavior. C11 includes read-read
coherence between relaxed atomics in order to forbid this:
\begin{gather*}
  \taglabel{Co2}
  \begin{gathered}
    \PW{x}{1}\SEMI \PW{x}{2}
    \PAR
    \PW{y}{x} \SEMI \PW{z}{x}
    \\[-1ex]
    \hbox{\begin{tikzinline}[node distance=1.5em]
        \event{a}{\DW{x}{1}}{}
        \event{b}{\DW{x}{2}}{right=of a}
        \wki{a}{b}
        \event{c}{\DR{x}{2}}{right=3em of b}
        \event{d}{\DW{y}{2}}{right=of c}
        \po{c}{d}
        \event{e}{\DR{x}{1}}{right=of d}
        \event{f}{\DW{z}{1}}{right=of e}
        \po{e}{f}
        \rf{b}{c}
        \rf[out=10,in=170]{a}{e}
        \wk[out=-165,in=-15]{e}{b}
      \end{tikzinline}}
  \end{gathered}
\end{gather*}
Here, the reader sees $2$ then $1$, although they are written in the reverse
order.
This behavior is allowed by Java in order to validate CSE without requiring
aliasing analysis.

\subsection{RMWs}
It is not possible for two \RMW{}s to see the same write.
\begin{gather*}
  \begin{gathered}
    \PW{x}{0} \SEMI \bigl(\PFADD[\mRLX][\mRLX]{x}{}{1} \PAR \PFADD[\mRLX][\mRLX]{x}{}{1}\bigr)
    \\
    \hbox{\begin{tikzinline}[node distance=2em]
        \event{a0}{\DW{x}{0}}{}
        \event{a1}{\DR{x}{0}}{right=3em of a0}
        \event{a2}{\DW{x}{1}}{right=of a1}
        \event{b1}{\DR{x}{0}}{right=3em of a2}
        \event{b2}{\DW{x}{1}}{right=of b1}
        \rmw{a1}{a2}
        \rf{a0}{a1}
        \rf[out=-15,in=-165]{a0}{b1}
        \wk[out=-15,in=-165]{a1}{b2}
        \wk{b1}{a2}
        \graywk[bend left]{a2}{b1}
        \rmw{b1}{b2}
      \end{tikzinline}}
  \end{gathered}
  \taglabel{rmw0}
\end{gather*}
The gray arrow is required the \RMW{} atomicity axioms.

\citet{DBLP:conf/pldi/LeeCPCHLV20} introduce \PS{2.0} to refine the treatment of
\RMW{}s in the promising semantics (\PS{}).  Their examples have the expected
results here, with far less work.  First they recall that \PS{} requires
quantification over multiple futures in order to disallow executions such as
\ref{CDRF}:
\begin{gather*}
  \taglabel{CDRF}
    \begin{gathered}
      \PFADD[\mACQ][\mREL]{x}{r}{1}\SEMI \IF{r{=}0}\THEN \PW{y}{1} \FI
      \PAR
      \PFADD[\mACQ][\mREL]{x}{r}{1}\SEMI \IF{r{=}0}\THEN \IF{y}\THEN \PW{x}{0} \FI \FI
      \\
      \hbox{\begin{tikzinline}[node distance=2em]
          \event{a1}{\DR[\mACQ]{x}{0}}{}
          \event{a1b}{\DW[\mREL]{x}{1}}{below=1em of a1}
          \event{a2}{\DW{y}{1}}{right=of a1}
          \event{b0}{\DR[\mACQ]{x}{0}}{right=3em of a2}
          \event{b0b}{\DW[\mREL]{x}{1}}{below=1em of b0}
          \event{b1}{\DR{y}{1}}{right=of b0}
          \event{b2}{\DW{x}{0}}{right=of b1}
          \rmw{a1}{a1b}
          \rmw{b0}{b0b}
          \rf[out=-13,in=-163]{a2}{b1}
          \po{a1}{a2}
          \sync{b0}{b1}
          \po{b1}{b2}
          \rf[out=-165,in=-12]{b2}{a1}
        \end{tikzinline}}
    \end{gathered}
  \end{gather*}
This execution is clearly impossible, due to the cycle above.  In this
diagram, we have not drawn order adjacent to the writes of the \RMW{}s, since
this is not necessary to produce the cycle.
If \ref{CDRF} is allowed then \drfra{} fails.


  
\PS{} does not support global value range analysis, as modeled by \ref{GA+E} below.  Our
semantics permits \ref{GA+E}:
\begin{gather*}
  \taglabel{GA+E}
    \begin{gathered}
      \PW{x}{0} \SEMI
      \bigl(
        \PCAS[\mRLX][\mRLX]{x}{r}{0}{1}\SEMI \IF{r{<}10}\THEN \PW{y}{1} \FI
        \PAR
        \PW{x}{42}\SEMI \PW{x}{y}
      \bigr)
      \\
      \hbox{\begin{tikzinline}[node distance=2em]
          \event{a0}{\DW{x}{0}}{}
          \event{a1}{\DR{x}{1}}{right=3em of a0}
          \event{a2}{0{<}10\mid\DW{y}{1}}{right=of a1}
          \event{b0}{\DW{x}{42}}{right=3em of a2}
          \event{b1}{\DR{y}{1}}{right=of b0}
          \event{b2}{\DW{x}{1}}{right=of b1}
          %\rmw{a1}{a2}
          \rf[out=-15,in=-160]{a2}{b1}
          \po{b1}{b2}
          \rf[out=-165,in=-15]{b2}{a1}
          \wk[out=10,in=170]{a0}{b0}
          \wk[out=15,in=165]{b0}{b2}
        \end{tikzinline}}
    \end{gathered}
\end{gather*}
\PS{} also does not support register promotion, as modeled by \ref{RP} below.    Our
semantics permits \ref{RP}:
\begin{gather*}
  \taglabel{RP}
    \begin{gathered}
      \PR{x}{r}\SEMI
      \PFADD[\mRLX][\mRLX]{z}{s}{r}\SEMI \PW{y}{s{+}1}
      \PAR
      \PW{x}{y}
      \\
      \hbox{\begin{tikzinline}[node distance=2em]
          \event{a0}{\DR{x}{1}}{}
          \event{a1}{\DR{z}{0}}{right=of a0}
          \event{a1b}{\DW{z}{1}}{right=of a1}
          \event{a2}{\DW{y}{1}}{right=of a1b}
          \event{b0}{\DR{y}{1}}{right=3em of a2}
          \event{b1}{\DW{x}{1}}{right=of b0}
          \rmw{a1}{a1b}
          \po[out=20,in=160]{a0}{a1b}
          \po[out=20,in=160]{a1}{a2}
          \po{b0}{b1}
          \rf{a2}{b0}
          \rf[out=-165,in=-15]{b1}{a0}
        \end{tikzinline}}
    \end{gathered}
\end{gather*}



\begin{example}
  This definition ensures atomicity, disallowing executions such as
  \cite[Ex.~3.2]{DBLP:journals/pacmpl/PodkopaevLV19}:
  \begin{gather*}
    % \taglabel{RMW1}
    \begin{gathered}
      \PW{x}{0}\SEMI \PINC[\mRLX][\mRLX]{x}{}
      \PAR
      \PW{x}{2}\SEMI \PR{x}{r}
      % \\
      % \hbox{\begin{tikzinline}[node distance=1.5em]
      %     \event{a2}{\DR{x}{0}}{}
      %     \event{a1}{\DW{x}{0}}{left=of a2}
      %     \rf{a1}{a2}
      %     \event{a3}{\DW{x}{2}}{right=of a2}
      %     \wk{a2}{a3}
      %     \event{b2}{\DW{x}{1}}{right=of a3}
      %     \event{b3}{\DR{x}{1}}{right=of b2}
      %     \rmw[out=-15,in=-165]{a2}[below]{b2}
      %     \wk{a3}{b2}
      %     \rf{b2}{b3}
      %     \liftrmw[out=165,in=15]{a3}{a2}
      %   \end{tikzinline}}
      \\
      \hbox{\begin{tikzinline}[node distance=1.5em]
          \event{a2}{\DR{x}{0}}{}
          \event{a1}{\DW{x}{0}}{left=of a2}
          \rf{a1}{a2}
          \event{a3}{\DW{x}{1}}{right=of a2}
          \event{b2}{\DW{x}{2}}{right=3em of a3}
          \event{b3}{\DR{x}{1}}{right=of b2}
          \rmw{a2}[below]{a3}
          \wk{b2}{a3}
          \wk[out=-15,in=-165]{a2}{b2}
          \rf[out=-15,in=-165]{a3}{b3}
          \liftrmw[out=165,in=15]{b2}{a2}
        \end{tikzinline}}
    \end{gathered}
  \end{gather*}
  By \ref{pom-rmw-atomic1}, since $\DWP{x}{2}\xwk\DWP{x}{1}$, it must be that
  $\DWP{x}{2}\xwk\DRP{x}{0}$, creating a cycle.
\end{example}

\begin{example}
  \label{ex:rmw-33}
  Two successful \RMW{}s cannot see the same write:
  \begin{gather*}
    \begin{gathered}
      \PW{x}{0}\SEMI (\PINC[\mRLX][\mRLX]{x}{} \PAR \PINC[\mRLX][\mRLX]{x}{})
      \\
      \hbox{\begin{tikzinline}[node distance=1.5em]
          \event{i}{\DW{x}{0}}{}
          \event{a1}{a{:}\DR{x}{0}}{right=3em of i}
          \event{a2}{b{:}\DW{x}{1}}{right=of a1}
          \event{b1}{c{:}\DR{x}{0}}{right=3em of a2}
          \event{b2}{d{:}\DW{x}{1}}{right=of b1}
          \rmw{a1}{a2}
          \rmw{b1}{b2}
          \rf{i}{a1}
          \rf[out=15,in=165]{i}{b1}
          \wk[out=-15,in=-165]{a1}{b2}
          \liftrmw[out=-15,in=-165]{a2}{b1}
          % \wk{a1}{b2}
          \wk{b1}{a2}
        \end{tikzinline}}
    \end{gathered}
  \end{gather*}
  The order from read-to-write is required by fulfillment.  
  Apply \ref{pom-rmw-atomic1} of the second \RMW{} to $a\xwk d$, we have that $a\xwk c$.  Subsequently
  applying \ref{pom-rmw-atomic2} of the first \RMW{}, we have $b \xwk c$, creating a cycle.
\end{example}

\begin{example}
  By using two actions rather than one, the definition allows examples such as the
  following, which is allowed by \armeight{} 
  \cite[Ex.~3.10]{DBLP:journals/pacmpl/PodkopaevLV19}:
  \begin{gather*}
    % \taglabel{RMW2}
    \begin{gathered}
      \PR{z}{r}\SEMI
      % \PW{x}{0}\SEMI
      \PINC[\mRLX][\mREL]{x}{s} \SEMI
      \PW{y}{s}{+}1
      \PAR
      \PR{y}{r}\SEMI
      \PW{z}{r}
      \\[-1ex]
      \hbox{\begin{tikzinline}[node distance=1.5em]
          \event{b1}{\DR{z}{1}}{}
          % \event{b2}{\DW{x}{0}}{right=of b1}
          \event{b3}{\DR{x}{0}}{right=of b1}
          %\rf{b2}{b3}
          \event{b4}{\DWRel{x}{1}}{right=2em of b3}
          \rmw{b3}{b4}
          \event{b5}{\DW{y}{1}}{right=of b4}
          \sync[out=-20,in=-160]{b1}{b4}
          \po[out=-20,in=-160]{b3}{b5}
          \event{a1}{\DR{y}{1}}{right=3em of b5}
          \event{a2}{\DW{z}{1}}{right=of a1}
          \po{a1}{a2}
          \rf{b5}{a1}
          \rf[out=170,in=10]{a2}{b1}
        \end{tikzinline}}
    \end{gathered}
  \end{gather*}
  A similar example, also allowed by \armeight{}
  \cite[Fig.~6]{DBLP:journals/pacmpl/ChakrabortyV19}:
  \begin{gather*}
    % \taglabel{RMW2}
    \begin{gathered}
      \PR{z}{r}\SEMI
      % \PW{x}{0}\SEMI
      \PFADD[\mRLX][\mRLX]{x}{s}{r} \SEMI
      \PW{y}{s}{+}1
      \PAR
      \PR{y}{r}\SEMI
      \PW{z}{r}
      \\[-1ex]
      \hbox{\begin{tikzinline}[node distance=1.5em]
          \event{b1}{\DR{z}{1}}{}
          %\event{b2}{\DW{x}{0}}{right=of b1}
          \event{b3}{\DR{x}{0}}{right=of b1}
          %\rf{b2}{b3}
          \event{b4}{\DW{x}{1}}{right=2em of b3}
          \rmw{b3}{b4}
          \event{b5}{\DW{y}{1}}{right=of b4}
          \po[out=-20,in=-160]{b1}{b4}
          \po[out=-20,in=-160]{b3}{b5}
          \event{a1}{\DR{y}{1}}{right=3em of b5}
          \event{a2}{\DW{z}{1}}{right=of a1}
          \po{a1}{a2}
          \rf{b5}{a1}
          \rf[out=170,in=10]{a2}{b1}
        \end{tikzinline}}
    \end{gathered}
  \end{gather*}
\end{example}
This is allowed by \weakestmo{}, but not \PS{}.

\begin{example}
  Consider the \textsc{cdrf} example from \cite{DBLP:conf/pldi/LeeCPCHLV20}:
  \begin{gather*}
    \begin{gathered}
      \begin{aligned}
        &\PINC[\mACQ][\mREL]{x}{r}\SEMI \IF{r{=}0}\THEN \PW{y}{1} \FI
        \\\PAR\;\;&
        \PINC[\mACQ][\mREL]{x}{r}\SEMI \IF{r{=}0}\THEN \IF{y}\THEN \PW{x}{0} \FI \FI
      \end{aligned}
      \\
      \hbox{\footnotesize\begin{tikzinline}[node distance=1.5em]
          \raevent{a1}{\DR[\mACQ]{x}{0}}{}
          \raevent{a1b}{\DW[\mREL]{x}{1}}{right=of a1}
          \event{a2}{\DW{y}{1}}{right=of a1b}
          \raevent{b0}{\DR[\mACQ]{x}{0}}{right=3em of a2}
          \raevent{b0b}{\DW[\mREL]{x}{1}}{right=of b0}
          \event{b1}{\DR{y}{1}}{right=of b0b}
          \event{b2}{\DW{x}{0}}{right=of b1}
          \rmw{a1}{a1b}
          \rmw{b0}{b0b}
          \rf[out=-13,in=-163]{a2}{b1}
          \sync[out=20,in=160]{a1}{a2}
          \sync[out=20,in=160]{b0}{b1}
          \po{b1}{b2}
          \rf[out=-165,in=-12]{b2}{a1}
        \end{tikzinline}}
    \end{gathered}
  \end{gather*}
\end{example}

\begin{example}
  Consider this example from \cite[\textsection C]{DBLP:conf/pldi/LeeCPCHLV20}:
  \begin{gather*}
    \begin{gathered}
      \begin{aligned}
        &\PCAS[\mRLX][\mRLX]{x}{r}{0}{1}\SEMI \IF{r{\leq}1}\THEN \PW{y}{1} \FI
        \\\PAR\;\;&
        \PCAS[\mRLX][\mRLX]{x}{r}{0}{2}\SEMI \IF{r{=}0}\THEN \IF{y}\THEN \PW{x}{0} \FI \FI
      \end{aligned}
      \\
      \hbox{\footnotesize\begin{tikzinline}[node distance=1.5em]
          \event{a1}{\DR{x}{0}}{}
          \event{a1b}{\DW{x}{1}}{right=of a1}
          \event{a2}{\DW{y}{1}}{right=of a1b}
          \event{b0}{\DR{x}{0}}{right=3em of a2}
          \event{b0b}{\DW{x}{2}}{right=of b0}
          \event{b1}{\DR{y}{1}}{right=of b0b}
          \event{b2}{\DW{x}{0}}{right=of b1}
          \rmw{a1}{a1b}
          \rmw{b0}{b0b}
          \rf[out=-13,in=-163]{a2}{b1}
          \po[out=20,in=160]{a1}{a2}
          \po[out=20,in=160]{b0}{b1}
          \po{b1}{b2}
          \rf[out=-165,in=-12]{b2}{a1}
        \end{tikzinline}}
    \end{gathered}
  \end{gather*}
\end{example}

\subsection{MCA}

\begin{gather*}
  \taglabel{MCA1}
  \begin{gathered}
    \IF{z}\THEN \PW{x}{0} \FI \SEMI \PW{x}{1}
    {\PAR}
    \IF{x}\THEN \PW{y}{0} \FI \SEMI \PW{y}{1}
    {\PAR}
    \IF{y}\THEN \PW{z}{0} \FI \SEMI \PW{z}{1}
    \\[-1ex]
    \hbox{\begin{tikzinline}[node distance=1.5em]
        \event{a1}{\DR{z}{1}}{}
        \event{a2}{\DW{x}{0}}{right=of a1}
        \po{a1}{a2}
        \event{a3}{\DW{x}{1}}{right=of a2}
        \wki{a2}{a3}
        \event{b1}{\DR{x}{1}}{right=3em of a3}
        \event{b2}{\DW{y}{0}}{right=of b1}
        \po{b1}{b2}
        \event{b3}{\DW{y}{1}}{right=of b2}
        \wki{b2}{b3}
        \event{c1}{\DR{y}{1}}{right=3em of b3}
        \event{c2}{\DW{z}{0}}{right=of c1}
        \po{c1}{c2}
        \event{c3}{\DW{z}{1}}{right=of c2}
        \wki{c2}{c3}
        \rf{a3}{b1}
        \rf{b3}{c1}
        \rf[out=173,in=7]{c3}{a1}  
      \end{tikzinline}}
  \end{gathered}
  \\[1ex]
  \taglabel{MCA2}
  \begin{gathered}
    \PW{x}{0}\SEMI \PW{x}{1}
    \PAR
    \PW{y}{x}
    \PAR
    \PR[\mRA]{y}{r} \SEMI \PR{x}{s}
    \\[-1ex]
    \hbox{\begin{tikzinline}[node distance=1.5em]
        \event{wx0}{\DW{x}{0}}{}
        \event{wx1}{\DW{x}{1}}{right=of wx0}
        \wki{wx0}{wx1}
        \event{rx1}{\DR{x}{1}}{right=3em of wx1}
        \event{wy1}{\DW{y}{1}}{right=of rx1}
        \po{rx1}{wy1}
        \event{ry1}{\DRAcq{y}{1}}{right=3em of wy1}
        \event{rx0}{\DR{x}{0}}{right=of ry1}
        \rf{wx1}{rx1}
        \rf{wy1}{ry1}
        \sync{ry1}{rx0}
        \wk[out=170,in=10]{rx0}{wx1}
      \end{tikzinline}}
  \end{gathered}
\end{gather*}

These candidate executions are invalid, due to cycles.

% \endinput
\section{Not for publication}


\subsection{TC18}

\begin{verbatim}
Initially,  x = y = 0
Thread 1:         Thread 2: 
r3 = x            r2 = y    
if (r3 == 0)      x = r2    
  x = 1
r1 = x
y = r1
Behavior in question: r1 == r2 == r3 == 1
\end{verbatim}
Decision: Allowed. A compiler could determine that the only legal values for
$x$ are $0$ and $1$. From that, the compiler could deduce that $r3 \neq 0$
implies $r3 = 1$.  A compiler could then determine that at $r1 = x$ in thread
1, is must be legal for to read $x$ and see the value $1$. Changing $r1 = x$
to $r1 = 1$ would allow $y = r1$ to be transformed to $y = 1$ and performed
earlier, resulting in the behavior in question.
\begin{displaymath}
  \begin{gathered}
    \PR{x}{r}
    \SEMI
    \IF{r{=}0}\THEN \PW{x}{1}\FI
    \SEMI
    \PR{x}{s}
    \SEMI
    \PW{y}{s}
    \PAR
    \PW{x}{\PR{y}{}}
    \\[-1ex]
    \hbox{\begin{tikzinline}[node distance=1.5em]
        \event{a}{\DR{x}{1}}{}
        %\event{b}{\DW{x}{1}}{right=of a}
        \event{c}{\DR{x}{1}}{right=of a}
        \event{d}{\aForm\mid\DW{y}{1}}{right=of c}
        \event{e}{\DR{y}{1}}{right=3em of d}
        \event{f}{\DW{x}{1}}{right=of e}
        \po{e}{f}
        \rf{d}{e}
        \rf[out=-165,in=-15]{f}{a}
        \rf[out=-165,in=-15]{f}{c}
      \end{tikzinline}}
    \\[-1ex]
    \hbox{\begin{tikzinline}[node distance=1.5em]
        \event{a}{\DR{x}{1}}{}
        %\event{b}{\DW{x}{1}}{right=of a}
        %\event{c}{\DR{x}{1}}{right=of a}
        \event{d}{\aForm\mid\DW{y}{1}}{right=of a}
        \event{e}{\DR{y}{1}}{right=3em of d}
        \event{f}{\DW{x}{1}}{right=of e}
        \po{e}{f}
        \rf{d}{e}
        \rf[out=-165,in=-15]{f}{a}
        %\rf[out=-165,in=-15]{f}{c}
      \end{tikzinline}}
  \end{gathered}
\end{displaymath}
\begin{displaymath}
  \aForm=
  (1{=}r \lor x{=}r)
  \limplies
  \PBR{
    \SBR{r{=}0\land\PBR{(1{=}s \lor 1{=}s) \limplies s{=}1}}
    \lor
    \SBR{r{\neq}0\land\PBR{(1{=}s \lor x{=}s) \limplies s{=}1}}
  }
\end{displaymath}
If we coalesce $s$ and $r$ and prefix $\PW{x}{0}$
\begin{displaymath}
  \aForm=
  (1{=}r \lor 0{=}r)
  \limplies
  \PBR{
    r{=}0
    \lor
    \SBR{r{\neq}0\land\PBR{(1{=}r \lor 0{=}r) \limplies r{=}1}}
  }
\end{displaymath}
which is
\begin{displaymath}
  \aForm=
  1{=}r
  \limplies
  \PBR{(1{=}r \lor 0{=}r) \limplies r{=}1}
\end{displaymath}
which is a tautology.

\subsection{Load hoisting in LLVM}
Load-hoisting followed by case analysis is unsound in LLVM, without freeze.
Introducing a read may cause a race, resulting in read value $\UNDEFINED$.
Branch on $\UNDEFINED$ is UB.  Freeze was added to get around this...

Examples from \cite{promising-ldrf} show that freeze is bullshit.  Compcert
does not validate loop switching
\cite[\textsection9]{DBLP:conf/pldi/LeeKSHDMRL17}.

\cite[\textsection3.3]{DBLP:conf/pldi/LeeKSHDMRL17} Global Value Numbering
(GVN) and Loop Unswitching require different semantics for branch on
$\UNDEFINED$. \url{https://llvm.org/docs/LangRef.html#freeze-instruction}

\href{https://lists.llvm.org/pipermail/llvm-dev/2016-October/106182.html}{Purpose of Freeze}
\begin{quotation}
  Poison is propagated aggressively throughout. However, there are cases
  where this breaks certain optimizations, and therefore freeze is used to
  selectively stop poison from being propagated.

  A use of freeze is to enable speculative execution.  For example, loop
  switching does the following transformation:
\begin{verbatim}
while (C) {        if (C2) {     
  if (C2) {           while (C)  
   A                     A       
  } else {   ==>   } else {      
   B                   while (C) 
  }                       B      
}                  }             
\end{verbatim}
  Here we are evaluating C2 before C.  If the original loop never executed
  then we had never evaluated C2, while now we do.  So we need to make sure
  there's no UB for branching on C2.  Freeze ensures that so we would
  actually have 'if (freeze(C2))' instead.  Note that
  \emph{having branch on poison not trigger UB has its own problems.}
  We believe this is a good tradeoff.
\end{quotation}


\cite[\textsection6.3]{DBLP:conf/cgo/ChakrabortyV17}:
\begin{quotation}
  LLVM frequently performs such load introductions in the “simplify CFG”
  pass; e.g., when hoisting loads outside of conditionals.
\end{quotation}

case analysis happens in ``function specialization'' also known as
``procedure cloning''

\subsection{Hoisting and CSE}
Example from Viktor's talk: ``Weak Memory Concurrency in C/C++11 and LLVM''

Load Hoisting:
\begin{displaymath}
  \IF{c}\THEN \PR{x}{a}\FI
  \rightsquigarrow
  \PR{x}{t}\SEMI
  \LET{a}{\TERNARY{c}{t}{a}}
\end{displaymath}

CSE over acquiring lock:
\begin{displaymath}
  \PR{x}{a}\SEMI
  \LOCK\SEMI
  \PR{x}{b}
  \rightsquigarrow
  \PR{x}{a}\SEMI
  \LOCK\SEMI
  \LET{b}{a}
\end{displaymath}

Having both is clearly wrong:
\begin{displaymath}
  \begin{array}{l}
    \IF{c}\THEN\\
    \quad\PR{x}{a}\FI\SEMI\\
    \LOCK\SEMI\\
    \PR{x}{b}    
  \end{array}
  \rightsquigarrow
  \begin{array}{l}
    \PR{x}{t}\SEMI\\
    \LET{a}{\TERNARY{c}{t}{a}}\SEMI\\
    \LOCK\SEMI\\
    \PR{x}{b}    
  \end{array}
  \rightsquigarrow
  \begin{array}{l}
    \PR{x}{t}\SEMI\\
    \LET{a}{\TERNARY{c}{t}{a}}\SEMI\\
    \LOCK\SEMI\\
    \LET{b}{a}    
  \end{array}
\end{displaymath}
When c is false, x is moved out of the critical region!

So we have to forbid one transformation.
\begin{itemize}
\item C11 forbids load hoisting, allows CSE over lock().
\item LLVM allows load hoisting, forbids CSE over lock().
\end{itemize}
\subsection{More RMW}
These following examples are from \cite{promising-ldrf}.

\ref{CDRF} shows that \PwT{} semantics is not too permissive for $\mRA$-\RMW{}s.
But what about $\mRLX$-\RMW{}s.  The following execution is allowed by \armeight,
and \PS{2.0}, but disallowed by \PS{2.1}.
\begin{gather*}
  \taglabel{RMW-W}
  \begin{gathered}
    \PFADD[\mRLX][\mRLX]{x}{r}{1}\SEMI \PW{y}{1}
    \PAR
    \PR{y}{r}\SEMI \PFADD[\mRLX][\mRLX]{x}{s}{r}
    \\
    \hbox{\begin{tikzinline}[node distance=2em]
        \event{a1}{\DR{x}{1}}{}
        \event{a1b}{\DW{x}{2}}{below=1em of a1}
        \event{a2}{\DW{y}{1}}{right=of a1}
        \event{b1}{\DR{y}{1}}{right=3em of a2}
        \event{b2}{\DR{x}{0}}{right=of b1}
        \event{b2b}{\DW{x}{1}}{below=1em of b2}
        \rmw{a1}{a1b}
        \rmw{b2}{b2b}
        \rf{a2}{b1}
        \po{b1}{b2b}
        \rf[out=-175,in=-20]{b2b}{a1}
      \end{tikzinline}}
  \end{gathered}
\end{gather*}

If this $\ldrfra{z}$?
\begin{gather*}
  \taglabel{Naive-LDRF-RA-Fail}
  \begin{gathered}
    \IF{y}\THEN \PW{x}{z} \ELSE \PW{x}{1} \FI
    \PAR
    \PR{x}{r}\SEMI \PW{z}{1}\SEMI \PW{y}{r}
    \\
    \hbox{\begin{tikzinline}[node distance=2em]
        \event{a1}{\DR{y}{1}}{}
        \event{a2}{\DR{z}{1}}{right=of a1}
        \event{a3}{\DW{x}{1}}{right=of a2}
        \event{b1}{\DR{x}{1}}{right=3em of a3}
        \event{b2}{\DW{z}{1}}{right=of b1}
        \event{b3}{\DW{y}{1}}{right=of b2}
        \po{a2}{a3}
        \po[in=165,out=15]{b1}{b3}
        \rf[out=-165,in=-15]{b2}{a2}
        \rf[out=-165,in=-15]{b3}{a1}
        \rf{a3}{b1}
      \end{tikzinline}}
  \end{gathered}
\intertext{Interpreting $\{z\}$ as $\mRA$:}
    \\
  \begin{gathered}
    \hbox{\begin{tikzinline}[node distance=2em]
        \event{a1}{\DR{y}{1}}{}
        \event{a2}{\DR[\mACQ]{z}{1}}{right=of a1}
        \event{a3}{\DW{x}{1}}{right=of a2}
        \event{b1}{\DR{x}{1}}{right=3em of a3}
        \event{b2}{\DW[\mREL]{z}{1}}{right=of b1}
        \event{b3}{\DW{y}{1}}{right=of b2}
        \po{a2}{a3}
        \po[in=165,out=15]{b1}{b3}
        \rf[out=-165,in=-15]{b2}{a2}
        \rf[out=-165,in=-15]{b3}{a1}
        \rf{a3}{b1}
        \sync{a1}{a2}
        \sync{b2}{b3}
      \end{tikzinline}}
  \end{gathered}
\end{gather*}

\PwT{} disallows \ref{LDRF-Fail-PS}, which is similar to \ref{OOTA4}.
\begin{gather*}  
  \taglabel{LDRF-Fail-PS}
  \begin{gathered}
  \IF{x}\THEN
    \PFADD{w}{}{1}\SEMI
    \PW{y}{1}\SEMI
    \PW{z}{1}
  \FI
  \PAR
  \IF{\BANG z}\THEN
    \PW{x}{1}
  \ELSE
    \IF{\BANG \PFADD{w}{}{1}}\THEN
      \PW{x}{\PR{y}{}}
    \FI
  \FI
    \\
    \hbox{\begin{tikzinline}[node distance=2em]
        \event{a1}{\DR{x}{1}}{}
        \event{a2}{\DR{w}{1}}{right=of a1}
        \event{a3}{\DW{w}{2}}{right=of a2}
        \event{a4}{\DW{y}{1}}{right=of a3}
        \event{a5}{\DW{z}{1}}{right=of a4}
        \event{b1}{\DR{z}{1}}{right=5em of a5}
        \event{b2}{\DR{w}{0}}{right=of b1}
        \event{b3}{\DW{w}{1}}{right=of b2}
        \event{b4}{\DR{y}{1}}{right=of b3}
        \event{b5}{\DW{x}{1}}{right=of b4}
        \rmw{a2}{a3}
        \po[out=15,in=165]{a1}{a3}
        \po[out=15,in=165]{a1}{a4}
        \po[out=15,in=165]{a1}{a5}        
        \rmw{b2}{b3}
        \po{b4}{b5}
        \po[out=15,in=165]{b2}{b5}        
        \po[out=15,in=165]{b1}{b3}
        \rf{a5}{b1}
        \rf[out=10,in=170]{a4}{b4}
        \rf[out=-170,in=-10]{b3}{a2}
        \rf[out=-170,in=-10]{b5}{a1}
      \end{tikzinline}}
  \end{gathered}
\end{gather*}
\begin{gather}
  \taglabel{OOTA4}
  \begin{gathered}
    \PW{y}{x}
    \PAR
    \PR{y}{r} \SEMI \IF{b}\THEN  \PW{x}{r} \SEMI \PW{z}{r} \ELSE \PW{x}{1} \FI
    \PAR
    \PW{b}{1}
    % \\[-1ex]
    % \hbox{\begin{tikzinline}[node distance=1.5em]
    %     \event{rx}{\DR{x}{1}}{}
    %     \event{wy}{\DW{y}{1}}{right=of rx}
    %     \po{rx}{wy}
    %     \event{ry}{\DR{y}{1}}{right=3em of wy} 
    %     \event{wx}{\DW{x}{1}}{right=of ry}
    %     \event{wz}{\DW{z}{1}}{right=of wx}
    %     \event{rb}{\DR{b}{1}}{right=of wz}
    %     \event{wb1}{\DW{b}{1}}{right=3em of rb}
    %     \po{ry}{wx}
    %     \rf{wb1}{rb}
    %     \rf{wy}{ry}
    %     \rf[out=-170,in=-10]{wx}{rx}
    %     \po{rb}{wz}
    %     \po[out=15,in=165]{ry}{wz}
    %   \end{tikzinline}}
    \\[-1ex]
    \hbox{\begin{tikzinline}[node distance=1.5em]
        \event{rx}{\DR{x}{1}}{}
        \event{wy}{\DW{y}{1}}{right=of rx}
        \po{rx}{wy}
        \event{rb}{\DR{b}{1}}{right=3em of wy}
        \event{ry}{\DR{y}{1}}{right=of rb} 
        \event{wx}{\DW{x}{1}}{right=of ry}
        \event{wz}{\DW{z}{1}}{right=of wx}
        \event{wb1}{\DW{b}{1}}{right=3em of wz}
        \po{ry}{wx}
        \rf[out=-170,in=-10]{wb1}{rb}
        \rf[out=15,in=165]{wy}{ry}
        \rf[out=-170,in=-10]{wx}{rx}
        \po[out=15,in=165]{rb}{wz}
        \po[out=15,in=165]{ry}{wz}
      \end{tikzinline}}
  \end{gathered}  
\end{gather}

% \begin{comment}
%   \centering  
% \begin{verbatim}
% a := X                  b := Z                 
% if a = 1 then           if b = 0 then          
%   _ := FADD(W , 1)        X := 1               
%   Y := 1                else                   
%   Z := 1                  c := FADD(W, 1) /0   
%                           if c = 0 then        
%                             d := Y             
%                             X := d             
% \end{verbatim}
% \includegraphics[width=\textwidth]{LDRF-Fail-PS}
% \caption{LDRF-Fail-PS}
% \end{comment}


If \RMW{}s simply use the same semantics as read and write, then we allow
\ref{LDRF-PF-Fail}, which is used to show failure of $\ldrfsc{}$.
\begin{gather*}  
  \taglabel{LDRF-PF-Fail}
  \begin{gathered}
    \PW{y}{0}\SEMI
    \IF{y}\THEN
      \IF{\BANG\PCAS{x}{}{0}{1}}\THEN
        \IF{z}\THEN
          \PW{x}{2}
    \FI\FI\FI
    \PAR
    \PW{y}{1}\SEMI
    \IF{1{\neq}\PCAS{x}{}{0}{3}}\THEN
      \PW{z}{1}
    \FI
    \\
    \hbox{\begin{tikzinline}[node distance=2em]
        \event{a1}{\DW{y}{0}}{}
        \event{a2}{\DR{y}{1}}{right=of a1}
        \event{a3}{\DR{x}{0}}{right=of a2}
        \event{a4}{\DW{x}{1}}{right=of a3}
        \event{a5}{\DR{z}{1}}{right=of a4}
        \event{a6}{\DW{x}{2}}{right=of a5}
        \event{b1}{\DW{y}{1}}{right=5em of a6}
        \event{b2}{\DR{x}{2}}{right=of b1}
        \event{b3}{\DW{z}{1}}{right=of b2}
        \wk{a1}{a2}
        \rmw{a3}{a4}
        \po[out=10,in=170]{a2}{a6}
        %\po[out=15,in=165]{a3}{a6}
        \po{a5}{a6}
        \wk[out=-20,in=-160]{a4}{a6}
        %\po{b2}{b3}
        \rf[out=15,in=165]{a6}{b2}
        \rf[out=-170,in=-10]{b3}{a5}
        \rf[out=-170,in=-10]{b1}{a2}
      \end{tikzinline}}
  \end{gathered}
\end{gather*}
To disallow this, we need to retain the dependency
\begin{math}
  \DRP{x}{2}\xpo \DWP{z}{1}.
\end{math}
For this, we need to avoid the substitution for $x$.  This is why we use
$\sLOADP{}{}$ instead of $\sLOAD{}{}$ in the independent case for \RMW{}s.

\begin{comment}
  \centering  
\begin{verbatim}
Y := 0                   Y := 1                 
a := Y                   d := CAS(X,0,1) /37?   
if a != 0 then           if d != 42 then        
  b := CAS(X,0,42)         L := 1               
  if b = 0 then
    c := L
    if c = 1 then
      Xsrlx := 37
\end{verbatim}
\includegraphics[width=.8\textwidth]{LDRF-PF-Fail.png}
\caption{LDRF-PF-Fail}
\end{comment}

\subsection{A Note on Mixed-Mode Data Races}

In preparing this paper, we came across the following example, which appears
to invalidate Theorem 4.1 of \cite{DBLP:conf/ppopp/DongolJR19}.
\begin{gather}
  \nonumber
  \PW{x}{1}\SEMI
  \PW[\mREL]{y}{1}\SEMI
  \PR[\mACQ]{x}{r}
  \PAR
  \IF{\PR[\mACQ]{y}{}}\THEN \PW[\mREL]{x}{2}\FI
  \\
  \tag{\P}
  \label{mix1}
  \hbox{\begin{tikzinline}[node distance=1.5em]
      \event{a1}{\DW{x}{1}}{}
      \raevent{a2}{\DW[\mREL]{y}{1}}{right=of a1}
      \raevent{a3}{\DR[\mACQ]{x}{1}}{right=of a2}
      \raevent{b1}{\DR[\mACQ]{y}{1}}{right=3em of a3}
      % \raevent{b1}{\DR[\mACQ]{y}{1}}{below=of a1}
      \raevent{b2}{\DW[\mREL]{x}{2}}{right=of b1}
      \sync{a1}{a2}
      \rf[out=20,in=160]{a1}{a3}
      \rf[out=20,in=160]{a2}{b1}
      \wk[out=-20,in=-160]{a3}{b2}
      \sync{b1}{b2}
      % \node(ai)[left=3em of a1]{};
      % \bgoval[yellow!50]{(ai)}{P}
      % \bgoval[pink!50]{(a1)(a2)(b1)(b2)}{P'\setminus P}
      % \bgoval[green!10]{(a3)}{P'''\setminus P'}
    \end{tikzinline}}
  \\
  \nonumber
  %\label{mix2}
  \hbox{\begin{tikzinline}[node distance=1.5em]
      \event{a1}{\DW{x}{1}}{}
      \raevent{a2}{\DW[\mREL]{y}{1}}{right=of a1}
      \raevent{a3}{\DR[\mACQ]{x}{2}}{right=of a2}
      \raevent{b1}{\DR[\mACQ]{y}{1}}{right=3em of a3}
      \raevent{b2}{\DW[\mREL]{x}{2}}{right=of b1}
      \sync{a1}{a2}
      \rf[out=20,in=160]{a2}{b1}
      \rf[out=160,in=20]{b2}{a3}
      \sync{b1}{b2}
    \end{tikzinline}}
\end{gather}
The program is data-race free.  The two executions shown are the only
top-level executions that include $\DWP[\mREL]{x}{2}$.

Theorem 4.1 of \cite{DBLP:conf/ppopp/DongolJR19} is stated by extending
execution sequences.  In the terminology of
\cite{DBLP:conf/ppopp/DongolJR19}, a read is \emph{$L$-weak} if it is
sequentially stale.  Let
\begin{math}
  \rho=\DWP{x}{1}\allowbreak
  \DWP[\mREL]{y}{1}\allowbreak
  \DRP[\mACQ]{y}{1}\allowbreak
  \DWP[\mREL]{x}{2}
\end{math}
be a sequence and
\begin{math}
  \alpha=\DRP[\mACQ]{x}{1}.
\end{math}
$\rho$ is $L$-sequential and $\alpha$ is $L$-weak in $\rho\alpha$.  But there
is no execution of this program that includes a data race, contradicting the
theorem.  The error seems to be in Lemma A.4 of
\cite{DBLP:conf/ppopp/DongolJR19}, which states that if $\alpha$ is $L$-weak
after an $L$-sequential $\rho$, then $\alpha$ must be in a data race.  That
is clearly false here, since $\DRP[\mACQ]{x}{1}$ is stale, but the program is
data race free.

In proving the SC-LDRF result in \cite[\textsection8]{DBLP:journals/pacmpl/JagadeesanJR20}, we noted that our proof
technique is more robust than that of \cite{DBLP:conf/ppopp/DongolJR19},
because it limits the prefixes that must be considered.  In \eqref{mix1}, the
induction hypothesis requires that we add $\DRP[\mACQ]{x}{1}$ before
$\DWP[\mREL]{x}{2}$ since $\DRP[\mACQ]{x}{1}\xwk\DWP[\mREL]{x}{2}$.  In
particular,
\begin{gather*}
  \hbox{\begin{tikzinline}[node distance=1.5em]
      \event{a1}{\DW{x}{1}}{}
      \raevent{a2}{\DW[\mREL]{y}{1}}{right=of a1}
      % \raevent{a3}{\DR[\mACQ]{x}{1}}{right=of a2}
      \raevent{b1}{\DR[\mACQ]{y}{1}}{right=3em of a3}
      % \raevent{b1}{\DR[\mACQ]{y}{1}}{below=of a1}
      \raevent{b2}{\DW[\mREL]{x}{2}}{right=of b1}
      \sync{a1}{a2}
      % \rf[out=20,in=160]{a1}{a3}
      \rf[out=20,in=160]{a2}{b1}
      % \wk[out=-20,in=-160]{a3}{b2}
      \sync{b1}{b2}
      % \node(ai)[left=3em of a1]{};
      % \bgoval[yellow!50]{(ai)}{P}
      % \bgoval[pink!50]{(a1)(a2)(b1)(b2)}{P'\setminus P}
      % \bgoval[green!10]{(a3)}{P'''\setminus P'}
    \end{tikzinline}}
\end{gather*}
is not a downset of \eqref{mix1}, because
$\DRP[\mACQ]{x}{1}\xwk\DWP[\mREL]{x}{2}$.  As noted in \cite[\textsection8]{DBLP:journals/pacmpl/JagadeesanJR20},
this affects the inductive order in which we move across pomsets, but does
not affect the set of pomsets that are considered.  In particular,
\begin{gather*}
  \hbox{\begin{tikzinline}[node distance=1.5em]
      \event{a1}{\DW{x}{1}}{}
      \raevent{a2}{\DW[\mREL]{y}{1}}{right=of a1}
      % \raevent{a3}{\DR[\mACQ]{x}{1}}{right=of a2}
      \raevent{b1}{\DR[\mACQ]{y}{1}}{right=3em of a3}
      % \raevent{b1}{\DR[\mACQ]{y}{1}}{below=of a1}
      % \raevent{b2}{\DW[\mREL]{x}{2}}{right=of b1}
      \sync{a1}{a2}
      % \rf[out=20,in=160]{a1}{a3}
      \rf[out=20,in=160]{a2}{b1}
      % \wk[out=-20,in=-160]{a3}{b2}
      % \sync{b1}{b2}
      % \node(ai)[left=3em of a1]{};
      % \bgoval[yellow!50]{(ai)}{P}
      % \bgoval[pink!50]{(a1)(a2)(b1)(b2)}{P'\setminus P}
      % \bgoval[green!10]{(a3)}{P'''\setminus P'}
    \end{tikzinline}}
\end{gather*}
is a downset of \eqref{mix1}.


\section{Old Notes}
\subsection{JMM examples}

TC17: Should be possible to read 1 everywhere
\begin{gather*}
  x\GETS y
  \PAR
  r\GETS x\SEMI\IF{r{\neq}1}\THEN x\GETS 1 \SEMI y\GETS x\ELSE y\GETS x\FI
  \\
  \begin{gathered}
    \IF{r{\neq}1}\THEN x\GETS 1 \SEMI y\GETS x\FI
    \\
    \hbox{\begin{tikzinline}[node distance=.5em and 1em]
        \event{a1}{r{\neq}1 \mid \DW{x}{1}}{}
        \event{a2}{r{\neq}1 \mid \DR{x}{1}}{right=of a1}
        \event{a3}{r{\neq}1 \land 1{=}1\mid \DW{y}{1}}{right=of a2}
        \wk{a1}{a2}
      \end{tikzinline}}
  \end{gathered}
  \\
  \begin{gathered}
    \IF{r{=}1}\THEN y\GETS x\FI
    \\
    \hbox{\begin{tikzinline}[node distance=.5em and 1em]
        \event{a4}{r{=}1 \mid \DR{x}{1}}{}
        \event{a5}{r{=}1 \land x{=}1\mid \DW{y}{1}}{right=of a4}
      \end{tikzinline}}
  \end{gathered}  
  \\
  \begin{gathered}
    \IF{r{\neq}1}\THEN x\GETS 1 \SEMI y\GETS x\ELSE y\GETS x\FI
    \\
    \hbox{\begin{tikzinline}[node distance=.5em and 1em]
        \event{a1}{r{\neq}1 \mid \DW{x}{1}}{}
        \event{a2}{\DR{x}{1}}{right=of a1}
        \event{a3}{r{\neq}1\lor x{=}1\mid \DW{y}{1}}{right=of a2}
        \wk{a1}{a2}
      \end{tikzinline}}
  \end{gathered}    
  \\
  \begin{gathered}
    r\GETS x\SEMI\IF{r{\neq}1}\THEN x\GETS 1 \SEMI y\GETS x\ELSE y\GETS x\FI
    \\
    \hbox{\begin{tikzinline}[node distance=.5em and 1em]
        \event{a1}{x{\neq}1 \mid \DW{x}{1}}{}
        \event{a2}{\DR{x}{1}}{right=of a1}
        \event{a3}{x{\neq}1\lor x{=}1\mid \DW{y}{1}}{right=of a2}
        \event{a0}{\DR{x}{1}}{left=of a1}
        \wk{a1}{a2}
        \wk{a0}{a1}
      \end{tikzinline}}
  \end{gathered}    
  \\
  \begin{gathered}
    x\GETS0\SEMI r\GETS x\SEMI\IF{r{\neq}1}\THEN x\GETS 1 \SEMI y\GETS x\ELSE y\GETS x\FI
    \\
    \hbox{\begin{tikzinline}[node distance=.5em and 1em]
        \event{a1}{0{\neq}1 \mid \DW{x}{1}}{}
        \event{a2}{\DR{x}{1}}{right=of a1}
        \event{a3}{\DW{y}{1}}{right=of a2}
        \event{a0}{\DR{x}{1}}{left=of a1}
        \event{a-1}{\DW{x}{0}}{left=of a0}
        \wk{a1}{a2}
        \wk{a0}{a1}
        \wk{a-1}{a0}
      \end{tikzinline}}
  \end{gathered}    
\end{gather*}

TC18
\begin{gather*}
  x\GETS y
  \PAR
  r\GETS x\SEMI\IF{r{=}0}\THEN x\GETS 1 \SEMI y\GETS x\ELSE y\GETS x\FI
  \\
  \begin{gathered}
    \IF{r{=}0}\THEN x\GETS 1 \SEMI y\GETS x\FI
    \\
    \hbox{\begin{tikzinline}[node distance=.5em and 1em]
        \event{a1}{r{=}0 \mid \DW{x}{1}}{}
        \event{a2}{r{=}0 \mid \DR{x}{1}}{right=of a1}
        \event{a3}{r{=}0 \land 1{\neq}0\mid \DW{y}{1}}{right=of a2}
        \wk{a1}{a2}
      \end{tikzinline}}
  \end{gathered}
  \\
  \begin{gathered}
    \IF{r{\neq}0}\THEN y\GETS x\FI
    \\
    \hbox{\begin{tikzinline}[node distance=.5em and 1em]
        \event{a4}{r{\neq}0 \mid \DR{x}{1}}{}
        \event{a5}{r{\neq}0 \land x{\neq}0\mid \DW{y}{1}}{right=of a4}
      \end{tikzinline}}
  \end{gathered}  
  \\
  \begin{gathered}
    \IF{r{=}0}\THEN x\GETS 1 \SEMI y\GETS x\ELSE y\GETS x\FI
    \\
    \hbox{\begin{tikzinline}[node distance=.5em and 1em]
        \event{a1}{r{=}0 \mid \DW{x}{1}}{}
        \event{a2}{\DR{x}{1}}{right=of a1}
        \event{a3}{r{=}0\lor x{\neq}0\mid \DW{y}{1}}{right=of a2}
        \wk{a1}{a2}
      \end{tikzinline}}
  \end{gathered}    
  \\
  \begin{gathered}
    r\GETS x\SEMI\IF{r{=}0}\THEN x\GETS 1 \SEMI y\GETS x\ELSE y\GETS x\FI
    \\
    \hbox{\begin{tikzinline}[node distance=.5em and 1em]
        \event{a1}{x{=}0 \mid \DW{x}{1}}{}
        \event{a2}{\DR{x}{1}}{right=of a1}
        \event{a3}{x{=}0\lor x{\neq}0\mid \DW{y}{1}}{right=of a2}
        \event{a0}{\DR{x}{1}}{left=of a1}
        \wk{a1}{a2}
        \wk{a0}{a1}
      \end{tikzinline}}
  \end{gathered}    
  \\
  \begin{gathered}
    x\GETS0\SEMI r\GETS x\SEMI\IF{r{=}0}\THEN x\GETS 1 \SEMI y\GETS x\ELSE y\GETS x\FI
    \\
    \hbox{\begin{tikzinline}[node distance=.5em and 1em]
        \event{a1}{0{=}0 \mid \DW{x}{1}}{}
        \event{a2}{\DR{x}{1}}{right=of a1}
        \event{a3}{\DW{y}{1}}{right=of a2}
        \event{a0}{\DR{x}{1}}{left=of a1}
        \event{a-1}{\DW{x}{0}}{left=of a0}        
        \wk{a1}{a2}
        \wk{a0}{a1}
        \wk{a-1}{a0}
      \end{tikzinline}}
  \end{gathered}    
\end{gather*}

TC19 --- where join keeps the right hand side.
\begin{gather*}
  (x\GETS y
  \RPAR
  r\GETS x\SEMI\IF{r{\neq}1}\THEN x\GETS 1 \FI )
  \SEMI y\GETS x
\end{gather*}

TC20
\begin{gather*}
  (x\GETS y
  \RPAR
  r\GETS x\SEMI\IF{r{=}0}\THEN x\GETS 1 \FI )
  \SEMI y\GETS x
\end{gather*}

TC16  Not allowed by us or by \cite{Dolan:2018:BDR:3192366.3192421}.
\begin{gather*}
  r\GETS x\SEMI x\GETS 1
  \PAR
  s\GETS x\SEMI x\GETS 2
  \\
  \hbox{\begin{tikzinline}[node distance=1em]
      \event{a1}{\DR{x}{2}}{}
      \event{a2}{\DW{x}{1}}{right=of a1}
      \wk{a1}{a2}
      \event{b1}{\DR{x}{1}}{below=of a1}
      \event{b2}{\DW{x}{2}}{right=of b1}
      \wk{b1}{b2}
      \rf{b2}{a1}
      \rf{a2}{b1}
   \end{tikzinline}}
\end{gather*}
But we allow the following, which \cite{Dolan:2018:BDR:3192366.3192421} disallows.
\begin{gather*}
  r\GETS y\SEMI x\GETS 1
  \PAR
  s\GETS x\SEMI y\GETS 2
  \\
  \hbox{\begin{tikzinline}[node distance=1em]
      \event{a1}{\DR{y}{2}}{}
      \event{a2}{\DW{x}{1}}{right=of a1}
      %\wk{a1}{a2}
      \event{b1}{\DR{x}{1}}{below=of a1}
      \event{b2}{\DW{y}{2}}{right=of b1}
      %\wk{b1}{b2}
      \rf{b2}{a1}
      \rf{a2}{b1}
   \end{tikzinline}}
\end{gather*}

\subsection{Sevcik examples}

\citet[\textsection7]{DBLP:conf/esop/CenciarelliKS07} example. (I
incorrectly credit \citet{DBLP:conf/ecoop/SevcikA08}.)

\begin{gather*}
  \IF{x\land y}\THEN z\GETS 1\FI
  \PAR
  \IF{z}\THEN x\GETS1\SEMI y\GETS1 \ELSE y\GETS1\SEMI x\GETS1 \FI
  \\
  \hbox{\begin{tikzinline}[node distance=.5em and 1em]
      \event{a1}{\DR{x}{1}}{}
      \event{a2}{\DR{y}{1}}{right=of a1}
      \event{a3}{\DW{z}{1}}{right=of a2}
      \po{a2}{a3}
      \po[out=15,in=165]{a1}{a3}      
      \event{b1}{\DR{z}{1}}{right=3em of a3}
      \event{b2}{\DW{y}{1}}{right=of b1}
      \event{b3}{\DW{x}{1}}{right=of b2}
      % \po{b1}{b2}
      % \po[out=15,in=165]{b1}{b3}
      \rf{a3}{b1}
      \rf[out=-165,in=-15]{b2}{a2}
      \rf[out=-165,in=-15]{b3}{a1}
   \end{tikzinline}}
\end{gather*}


Examples from \cite[\textsection4.1]{DBLP:conf/ecoop/SevcikA08} are interesting:
Redundant write after read elimination:
\begin{verbatim}
|| lock m2; x=1; unlock m2
|| lock m1; x=2; unlock m1
|| lock m1; lock m2; r1=x; [x=r1;] r2=x; unlock m2; unlock m1 // [bracketed line removed]
\end{verbatim}
Even without the write, r1 and r2 must see the same values, whereas JMM
allows different values for the reads when the write is missing.

Redundant read after read elimination:
\begin{verbatim}
|| y=x
|| r2=y; if (r2==1){[r3=y]; x=r3}else{x=1} // [r3=r2]
\end{verbatim}
Interesting case is left $\DW{x}{1}$.  Initially has predicate
$r_3=1$. With read rule, we have $y=1$.  In read prefixing, we don't weaken.
Instead we weaken with the read into r2.
\begin{gather*}
  \begin{gathered}
    \IF{r_2{=}1}\THEN r_3\GETS y\SEMI x\GETS r_3\FI
    \\
    \hbox{\begin{tikzinline}[node distance=.5em and 1em]
        \event{a1}{r_2{=}1 \mid \DR{y}{1}}{}
        \event{a2}{r_2{=}1 \land y{=}1 \mid \DW{x}{1}}{right=of a1}
      \end{tikzinline}}
  \end{gathered}
  \qquad
  \begin{gathered}
    \IF{r_2{\neq}1}\THEN x\GETS 1\FI
    \\
    \hbox{\begin{tikzinline}[node distance=.5em and 1em]
        \event{a2}{r_2{\neq}1 \mid \DW{x}{1}}{}
      \end{tikzinline}}
  \end{gathered}
  \\
  \IF{r_2{=}1}\THEN r_3\GETS y\SEMI x\GETS r_3 \ELSE x\GETS 1\FI
  \\
  \hbox{\begin{tikzinline}[node distance=.5em and 1em]
      \event{a1}{r_2{=}1 \mid \DR{y}{1}}{}
      \event{a2}{(r_2{=}1 \land y{=}1) \lor (r_2{\neq}1) \mid \DW{x}{1}}{right=of a1}
   \end{tikzinline}}
  \\
  r_2\GETS y \SEMI\IF{r_2{=}1}\THEN r_3\GETS y\SEMI x\GETS r_3 \ELSE x\GETS 1\FI
  \\
  \hbox{\begin{tikzinline}[node distance=.5em and 1em]
      \event{a1}{\DR{y}{1}}{}
      \event{a2}{(y{=}1 \land y{=}1) \lor (y{\neq}1) \mid\DW{x}{1}}{right=of a1}
      \event{a0}{\DR{y}{1}}{left=of a1}
      %\po[out=-15,in=-165]{a0}{a2}
   \end{tikzinline}}
\end{gather*}
To ignore the second read, we use the ``delay'' trick that we used for JMM
TC1, but this is fulfilled by a read rather than a write.
In any case, the execution with $x=y=1$ is allowed.


Roach Motel---all reads 1 impossible, but passible after swapping \verb:r1=x:
and \verb:lock m:
\begin{verbatim}
|| lock m; x=1; unlock m
|| lock m; x=2; unlock m
|| r1=x; lock m; r2=z; if(r1==2){y=1}else{y=r2}; unlock m
|| z=y
\end{verbatim}
So Question is whether you can read all 1 in
\begin{verbatim}
|| lock m; x=1; unlock m
|| lock m; x=2; unlock m
|| lock m; r1=x; r2=z; if(r1==2){y=1}else{y=r2}; unlock m
|| z=y
\end{verbatim}
In any execution, we must have 1 before 2, or 2 before 1.
\begin{itemize}
\item If thread sees 2, then read x is 2.
\item If thread sees 1, then read x is 1.
  \begin{gather*}
    \begin{gathered}
      \IF{r_1{=}2}\THEN y\GETS 1 \ELSE y\GETS r_2\FI
      \\
      \hbox{\begin{tikzinline}[node distance=.5em and 1em]
          \event{a2}{r_1{=}2\lor (r_1{\neq}2 \land r_2{=}1)\mid \DW{y}{1}}{}
        \end{tikzinline}}
    \end{gathered}
    \\
    \begin{gathered}
      r_1\GETS x\SEMI
      r_2\GETS z\SEMI
      \IF{r_1{=}2}\THEN y\GETS 1 \ELSE y\GETS r_2\FI
      \\
      \hbox{\begin{tikzinline}[node distance=.5em and 1em]
          \event{a2}{\DW{y}{1}}{}
          %\event{a2}{1{=}2\lor 1{=}1\mid \DW{y}{1}}{}
          \event{a1}{\DR{z}{1}}{left=of a2}
          \event{a0}{\DR{x}{1}}{left=of a1}
          \po{a1}{a2}
        \end{tikzinline}}
    \end{gathered}    
  \end{gather*}
  So impossible for y and z to be 1.
\end{itemize}

Irrelevant Read Introduction (can I read 1 for both y and z?)
\begin{verbatim}
|| r=z; if(!r){if(x){y=1}}else{[s=x;]y=r}
|| x=1; z=y
\end{verbatim}

\begin{gather*}
    \begin{gathered}
      \IF{\BANG r}\THEN \IF{x}\THEN y\GETS 1\FI\FI
      \\
      \hbox{\begin{tikzinline}[node distance=.5em and 1em]
          \event{a2}{r{=}0\mid\DW{y}{1}}{}
          \event{a1}{r{=}0\mid\DR{x}{1}}{left=of a2}
          \po{a1}{a2}
        \end{tikzinline}}
    \end{gathered}      
    \qquad
    \begin{gathered}
      \IF{r}\THEN s\GETS x\SEMI y\GETS r\FI
      \\
      \hbox{\begin{tikzinline}[node distance=.5em and 1em]
          \event{a2}{r{=}1\mid\DW{y}{1}}{}
          \event{a1}{r{\neq}0\mid\DR{x}{1}}{left=of a2}
        \end{tikzinline}}
    \end{gathered}      
    \\
    \begin{gathered}
      \IF{\BANG r}\THEN \IF{x}\THEN y\GETS 1\FI\ELSE y\GETS r\FI
      \\
      \hbox{\begin{tikzinline}[node distance=.5em and 1em]
          \event{a2}{r{=}0\lor r{=}1\mid\DW{y}{1}}{}
          \event{a1}{\DR{x}{1}}{left=of a2}
          \po{a1}{a2}
        \end{tikzinline}}
    \end{gathered}          
    \\
    \begin{gathered}
      z\GETS 0\SEMI r\GETS z\SEMI \IF{\BANG r}\THEN \IF{x}\THEN y\GETS 1\FI\ELSE y\GETS r\FI
      \\
      \hbox{\begin{tikzinline}[node distance=.5em and 1em]
          \event{a2}{0{=}0\lor 0{=}1\mid\DW{y}{1}}{}
          \event{a1}{\DR{x}{1}}{left=of a2}
          \po{a1}{a2}
          \event{a0}{\DR{z}{1}}{left=of a1}
          \event{a00}{\DW{z}{0}}{left=of a0}
          %\po[out=-15,in=-165]{a0}{a2}
        \end{tikzinline}}
    \end{gathered}          
  \end{gather*}
\begin{gather*}
    \begin{gathered}
      \IF{\BANG r}\THEN \IF{x}\THEN y\GETS 1\FI\FI
      \\
      \hbox{\begin{tikzinline}[node distance=.5em and 1em]
          \event{a2}{r{=}0\mid\DW{y}{1}}{}
          \event{a1}{r{=}0\mid\DR{x}{1}}{left=of a2}
          \po{a1}{a2}
        \end{tikzinline}}
    \end{gathered}      
    \qquad
    \begin{gathered}
      \IF{r}\THEN y\GETS r\FI
      \\
      \hbox{\begin{tikzinline}[node distance=.5em and 1em]
          \event{a2}{r{=}1\mid\DW{y}{1}}{}
        \end{tikzinline}}
    \end{gathered}      
    \\
    \begin{gathered}
      \IF{\BANG r}\THEN \IF{x}\THEN y\GETS 1\FI\ELSE y\GETS r\FI
      \\
      \hbox{\begin{tikzinline}[node distance=.5em and 1em]
          \event{a2}{r{=}0\lor r{=}1\mid\DW{y}{1}}{}
          \event{a1}{r{=}0\mid\DR{x}{1}}{left=of a2}
          \po{a1}{a2}
        \end{tikzinline}}
    \end{gathered}          
    \\
    \begin{gathered}
      z\GETS 0\SEMI r\GETS z\SEMI \IF{\BANG r}\THEN \IF{x}\THEN y\GETS 1\FI\ELSE y\GETS r\FI
      \\
      \hbox{\begin{tikzinline}[node distance=.5em and 1em]
          \event{a2}{0{=}0\lor 0{=}1\mid\DW{y}{1}}{}
          \event{a1}{\DR{x}{1}}{left=of a2}
          \po{a1}{a2}
          \event{a0}{\DR{z}{1}}{left=of a1}
          \event{a00}{\DW{z}{0}}{left=of a0}
          %\po[out=-15,in=-165]{a0}{a2}
        \end{tikzinline}}
    \end{gathered}          
  \end{gather*}
  If z is initialized to 2, rather than 0, then the dependencies remain and
  both are disallowed.  This relies crucially on the fact that par takes
  order from both sides.


\subsection{More optimizations}

\begin{itemize}
\item Sound to strengthen the annotation on an action from $\mRLX$ to
  $\mRA$, and from $\mRA$ to $\mSC$.
\end{itemize}

From \cite{Manson:2005:JMM:1047659.1040336}:
\begin{itemize}
\item synchronization on thread local objects can be ignored or removed
  altogether (the caveat to this is the fact that invocations of methods like
  wait and notify have to obey the correct semantics – for example, even if
  the lock is thread local, it must be acquired when perform- ing a wait),
   
\item volatile fields of thread local objects can be treated as normal
  fields.

\item redundant synchronization (e.g., when a synchronized method is called
  from another synchronized method on the same object) can be ignored or
  removed,
  
\end{itemize}

Counterexample for first two:
\begin{verbatim}
 y=1; x^AR=1; r=X^AR; z=1
\end{verbatim}
If you see $z=1$ you must see $y=1$

It would be nice if we could get at these with a strength reducing result:
synchronization actions can be replaced by relaxed actions in some cases.
Then the rules for relaxed read elimination and relaxed write elimination can
be used to get rid of them.

\subsection{Examples for semicolon semantics}

\begin{itemize}
\item Parallel asymmetric: state result for \emph{joint free} programs. 
\item Subsumption can be allowed on registers only
\item We build substitutions
\item Ignore substitutions when considering semantic equality.
\end{itemize}


Value for $r$ in $(r\EQ1\mid\DW{z}{1})$ from $(\DW{x}{1})$:
\begin{gather*}
  x\GETS 1 \PAR x\GETS 1\SEMI r\GETS x \SEMI y\GETS r\SEMI z\GETS r
  \\
  \hbox{\begin{tikzinline}[node distance=.5em and 1em]
      \event{a1}{\DW{x}{1}}{}
      \event{a2}{\DR{x}{1}}{right=of a1}
      \event{a3}{\DW{y}{1}}{right=of a2}
      \event{a4}{\DW{z}{1}}{right=of a3}
      \po{a2}{a3}
      %\po[out=-15,in=-165]{a1}{a4}
      \event{a0}{\DW{x}{1}}{left=2em of a1}
      \rf[out=15,in=165]{a0}{a2}
    \end{tikzinline}}
\end{gather*}          
Value for $r$ in $(r\EQ1\mid\DW{z}{1})$ from $(\DW{x}{1})$:
\begin{gather*}
  x\GETS 2 \PAR x\GETS 1\SEMI r\GETS x \SEMI \IF{r{>}0}\THEN y\GETS 1\FI \SEMI \IF{r{>}0}\THEN z\GETS 1\FI
  \\
  \hbox{\begin{tikzinline}[node distance=.5em and 1em]
      \event{a1}{\DW{x}{1}}{}
      \event{a2}{\DR{x}{2}}{right=of a1}
      \event{a3}{\DW{y}{1}}{right=of a2}
      \event{a4}{\DW{z}{1}}{right=of a3}
      \po{a2}{a3}
      %\po[out=-15,in=-165]{a1}{a4}
      \event{a0}{\DW{x}{2}}{left=2em of a1}
      \rf[out=15,in=165]{a0}{a2}
    \end{tikzinline}}
\end{gather*}
Note that this also contains pomset where value for $r$ in
$(r\EQ1\mid\DW{y}{1})$ also comes from $(\DW{x}{1})$:
\begin{gather*}
  x\GETS 2 \PAR x\GETS 1\SEMI r\GETS x \SEMI \IF{r{>}0}\THEN y\GETS 1\FI \SEMI \IF{r{>}0}\THEN z\GETS 1\FI
  \\
  \hbox{\begin{tikzinline}[node distance=.5em and 1em]
      \event{a1}{\DW{x}{1}}{}
      \event{a2}{\DR{x}{2}}{right=of a1}
      \event{a3}{\DW{y}{1}}{right=of a2}
      \event{a4}{\DW{z}{1}}{right=of a3}
      %\po{a2}{a3}
      %\po[out=-15,in=-165]{a1}{a4}
      \event{a0}{\DW{x}{2}}{left=2em of a1}
      \rf[out=15,in=165]{a0}{a2}
    \end{tikzinline}}
\end{gather*}
So our semantics will calculate the least ordered version.  Then rely on
augmentation to get the others.
\begin{gather*}
  \begin{gathered}
    x\GETS 1
    \\[-1ex]
    \hbox{\begin{tikzinline}[node distance=.2em]
      \event{a}{\DW{x}{1}}{}
      \final{f}{\SUB{1/x}}{below=of a}
      \end{tikzinline}}
  \end{gathered}
  \qquad
  \begin{gathered}
    r\GETS x
    \\[-1ex]
    \hbox{\begin{tikzinline}[node distance=.2em]
      \event{b}{\DRreg{r}{x}{2}}{}
      \final{f}{\SUB{x/r}}{below=of b}
      \end{tikzinline}}
  \end{gathered}
  \qquad
  \begin{gathered}
    \IF{r{>}0}\THEN y\GETS 1\FI
    \\[-1ex]
    \hbox{\begin{tikzinline}[node distance=.2em]
      \event{c}{r{>}0 \mid \DW{y}{1}}{}
      \final{f}{r{>}0 \mid \SUB{1/y}}{below=of c}
      \end{tikzinline}}
  \end{gathered}
  \qquad
  \begin{gathered}
    \IF{r{>}0}\THEN z\GETS 1\FI
    \\[-1ex]
    \hbox{\begin{tikzinline}[node distance=.2em]
      \event{d}{r{>}0 \mid \DW{z}{1}}{}
      \final{f}{r{>}0 \mid \SUB{1/z}}{below=of d}
      \end{tikzinline}}
  \end{gathered}
  \\
  \begin{gathered}
    x\GETS 1
    \SEMI
    r\GETS x    
    \\[-1ex]
    \hbox{\begin{tikzinline}[node distance=.2em]
      \event{a}{\DW{x}{1}}{}
      \event{b}{\DRreg{r}{x}{2} \mid \SUB{1/x,1/r}}{right=of a}
      \final{f}{\SUB{1/x}}{below=of a}
      \end{tikzinline}}
  \end{gathered}
  \qquad
  \begin{gathered}
    \IF{r{>}0}\THEN y\GETS 1\FI
    \SEMI
    \IF{r{>}0}\THEN z\GETS 1\FI
    \\[-1ex]
    \hbox{\begin{tikzinline}[node distance=.2em]
      \event{c}{r{>}0 \mid \DW{y}{1}}{}
      \event{d}{r{>}0 \mid \DW{z}{1}}{right=of c}
      \final{f}{r{>}0 \mid \SUB{1/y,1/z}}{below=of c}
      \end{tikzinline}}
  \end{gathered}
\end{gather*}
It is also possible that the read is necessary to give a value for $r$:
\begin{gather*}
  x\GETS 2 \PAR x\GETS 0\SEMI r\GETS x \SEMI \IF{r{>}0}\THEN y\GETS 1\FI \SEMI \IF{r{>}0}\THEN z\GETS 1\FI
  \\
  \hbox{\begin{tikzinline}[node distance=.5em and 1em]
      \event{a1}{\DW{x}{0}}{}
      \event{a2}{\DR{x}{2}}{right=of a1}
      \event{a3}{\DW{y}{1}}{right=of a2}
      \event{a4}{\DW{z}{1}}{right=of a3}
      \po{a2}{a3}
      \po[out=15,in=165]{a2}{a4}
      \event{a0}{\DW{x}{2}}{left=2em of a1}
      \rf[out=15,in=165]{a0}{a2}
    \end{tikzinline}}
\end{gather*}
\begin{gather*}
  \begin{gathered}
    x\GETS 0
    \SEMI
    r\GETS x    
    \\[-1ex]
    \hbox{\begin{tikzinline}[node distance=.2em]
      \event{a}{\DW{x}{0}}{}
      \event{b}{\DRreg{r}{x}{2} \mid \SUB{0/x,0/r}}{right=of a}
      \final{f}{\SUB{0/x}}{below=of a}
      \end{tikzinline}}
  \end{gathered}
  \qquad
  \begin{gathered}
    \IF{r{>}0}\THEN y\GETS 1\FI
    \SEMI
    \IF{r{>}0}\THEN z\GETS 1\FI
    \\[-1ex]
    \hbox{\begin{tikzinline}[node distance=.2em]
      \event{c}{r{>}0 \mid \DW{y}{1}}{}
      \event{d}{r{>}0 \mid \DW{z}{1}}{right=of c}
      \final{f}{r{>}0 \mid \SUB{1/y,1/z}}{below=of c}
      \end{tikzinline}}
  \end{gathered}  
\end{gather*}
Dependency on two reads:
\begin{gather*}
  r\GETS x \SEMI s\GETS y\SEMI \IF{r{<}s}\THEN z\GETS 1\FI
  \\
  \hbox{\begin{tikzinline}[node distance=.5em and 1em]
      \event{a1}{\DRreg{r}{x}{1}}{}
      \event{a2}{\DRreg{s}{y}{2}}{right=of a1}
      \event{a3}{\DW{z}{1}}{right=of a2}
      \po{a2}{a3}
      \po[out=15,in=165]{a1}{a3}
    \end{tikzinline}}
\end{gather*}          
\begin{gather*}
  \begin{gathered}
    r\GETS x\SEMI s\GETS y
    \\[-1ex]
    \hbox{\begin{tikzinline}[node distance=.2em]
      \event{a}{\DRreg{r}{x}{1}}{}
      \event{b}{\DRreg{s}{y}{2}}{right=of a}
      \final{f}{\SUB{x/r,y/s}}{below=of a}
      \end{tikzinline}}
  \end{gathered}
  \qquad
  \begin{gathered}
    \IF{r{<}s}\THEN z\GETS 1\FI
    \\[-1ex]
    \hbox{\begin{tikzinline}[node distance=.2em]
      \event{c}{r{<}s \mid \DW{z}{1}}{}
      \final{f}{r{<}s \mid \SUB{1/z}}{below=of c}
      \end{tikzinline}}
  \end{gathered}
  \\
  \begin{gathered}
    r\GETS x\SEMI s\GETS y \SEMI \IF{r{<}s}\THEN z\GETS 1\FI
    \\[-1ex]
    \hbox{\begin{tikzinline}[node distance=.2em]
      \event{a}{\DRreg{r}{x}{1}}{}
      \event{b}{\DRreg{s}{y}{2}}{right=of a}
      \event{c}{x{<}2 \mid \DW{z}{1}}{right=1em of b}
      \po{b}{c}
      \final{f}{r{<}s \mid \SUB{x/r,y/s,1/z}}{below=of a}
      \end{tikzinline}}
  \end{gathered}
\end{gather*}
Don't need to worry about confusing reads:
\begin{gather*}
  r\GETS x \SEMI s\GETS x\SEMI \IF{s{<}0}\THEN z\GETS 1\FI
  \\
  \hbox{\begin{tikzinline}[node distance=.5em and 1em]
      \event{a1}{\DRreg{r}{x}{1}}{}
      \event{a2}{\DRreg{s}{x}{2}}{right=of a1}
      \event{a3}{\DW{z}{1}}{right=of a2}
      \po{a2}{a3}
    \end{tikzinline}}
\end{gather*}          
\begin{gather*}
  \begin{gathered}
    r\GETS x\SEMI s\GETS x
    \\[-1ex]
    \hbox{\begin{tikzinline}[node distance=.2em]
      \event{a}{\DRreg{r}{x}{1}}{}
      \event{b}{\DRreg{s}{x}{2}}{right=of a}
      \final{f}{\SUB{x/r,x/s}}{below=of a}
      \end{tikzinline}}
  \end{gathered}
  \qquad
  \begin{gathered}
    z\GETS s
    \\[-1ex]
    \hbox{\begin{tikzinline}[node distance=.2em]
      \event{c}{s{<}0 \mid \DW{z}{1}}{}
      \final{f}{s{<}0 \mid \SUB{1/z}}{below=of c}
      \end{tikzinline}}
  \end{gathered}
\end{gather*}
But we also have
\begin{gather*}
  r\GETS x \SEMI s\GETS x\SEMI \IF{s{<}0}\THEN z\GETS 1\FI
  \\
  \hbox{\begin{tikzinline}[node distance=.5em and 1em]
      \event{a1}{\DRreg{r}{x}{1}}{}
      \event{a2}{\DRreg{s}{x}{2}}{right=of a1}
      \event{a3}{\DW{z}{1}}{right=of a2}
      \po[out=15,in=165]{a1}{a3}
    \end{tikzinline}}
\end{gather*}          
\begin{gather*}
  \begin{gathered}
    r\GETS x
    \\[-1ex]
    \hbox{\begin{tikzinline}[node distance=.2em]
      \event{a}{\DRreg{r}{x}{1}}{}
      \final{f}{\SUB{x/r}}{below=of a}
      \end{tikzinline}}
  \end{gathered}
  \qquad
  \begin{gathered}
    s\GETS x \SEMI \IF{s{<}0}\THEN z\GETS 1\FI
    \\[-1ex]
    \hbox{\begin{tikzinline}[node distance=.2em]
      \event{b}{\DRreg{s}{x}{2}}{}
      \event{c}{x{<}0 \mid \DW{z}{1}}{right=of b}
      \final{f}{x{<}0 \mid \SUB{x/s,1/z}}{below=of c}
      \end{tikzinline}}
  \end{gathered}
\end{gather*}
Dependency on two reads (No dependency here):
\begin{gather*}
  r\GETS x \SEMI s\GETS x\SEMI \IF{r{=}s}\THEN z\GETS 1\FI
  \\
  \hbox{\begin{tikzinline}[node distance=.5em and 1em]
      \event{a1}{\DRreg{r}{x}{1}}{}
      \event{a2}{\DRreg{s}{x}{2}}{right=of a1}
      \event{a3}{\DW{z}{1}}{right=of a2}
      %\po{a2}{a3}
      %\po[out=15,in=165]{a1}{a3}
    \end{tikzinline}}
\end{gather*}          
\begin{gather*}
  \begin{gathered}
    r\GETS x
    \\[-1ex]
    \hbox{\begin{tikzinline}[node distance=.2em]
      \event{a}{\DRreg{r}{x}{1}}{}
      \final{f}{\SUB{x/r}}{below=of a}
      \end{tikzinline}}
  \end{gathered}
  \qquad
  \begin{gathered}
    s\GETS x \SEMI \IF{r{=}s}\THEN z\GETS 1\FI
    \\[-1ex]
    \hbox{\begin{tikzinline}[node distance=.2em]
      \event{b}{\DRreg{s}{x}{2}}{}
      \event{c}{r{=}x \mid \DW{z}{1}}{right=of b}
      \final{f}{r{=}x \mid \SUB{x/s,1/z}}{below=of c}
      \end{tikzinline}}
  \end{gathered}
\end{gather*}
Another example:
\begin{gather*}
  r\GETS x \SEMI s\GETS x\SEMI  z\GETS s
  \\
  \hbox{\begin{tikzinline}[node distance=.5em and 1em]
      \event{a1}{\DRreg{r}{x}{1}}{}
      \event{a2}{\DRreg{s}{x}{1}}{right=of a1}
      \event{a3}{\DW{z}{1}}{right=of a2}
      %\po{a2}{a3}
      \po[out=15,in=165]{a1}{a3}
    \end{tikzinline}}
\end{gather*}          
\begin{gather*}
  \begin{gathered}
    r\GETS x
    \\[-1ex]
    \hbox{\begin{tikzinline}[node distance=.2em]
      \event{a}{\DRreg{r}{x}{1}}{}
      \final{f}{\SUB{x/r}}{below=of a}
      \end{tikzinline}}
  \end{gathered}
  \qquad
  \begin{gathered}
    s\GETS x \SEMI z \GETS s
    \\[-1ex]
    \hbox{\begin{tikzinline}[node distance=.2em]
      \event{b}{\DRreg{s}{x}{1}}{}
      \event{c}{x{=}1 \mid \DW{z}{1}}{right=of b}
      \final{f}{x{=}1 \mid \SUB{x/s,1/z}}{below=of c}
      \end{tikzinline}}
  \end{gathered}
\end{gather*}

Value for $r$ in $(r{<}s\mid\DW{z}{1})$ from $(\DW{x}{0})$:
\begin{gather*}
  x\GETS 0\SEMI r\GETS x \SEMI s\GETS y\SEMI \IF{r{<}s}\THEN z\GETS 1\FI
  \\
  \hbox{\begin{tikzinline}[node distance=.5em and 1em]
      \event{a1}{\DRreg{r}{x}{1}}{}
      \event{a2}{\DRreg{s}{y}{2}}{right=of a1}
      \event{a3}{\DW{z}{1}}{right=of a2}
      \event{a0}{\DW{x}{0}}{left=of a1}
      \po{a2}{a3}
      %\po[out=15,in=165]{a0}{a3}
    \end{tikzinline}}
\end{gather*}          


Contrary to submission, reverse subsumption not okay.
\begin{gather*}
  \begin{gathered}
    x\GETS 1
    \\[-1ex]
    \hbox{\begin{tikzinline}[node distance=.2em]
      \event{a}{\DRreg{r}{x}{1}}{}
      \final{f}{\SUB{1/x}}{below=of a}
      \end{tikzinline}}
  \end{gathered}
  \qquad
  \begin{gathered}
    x\GETS 2
    \\[-1ex]
    \hbox{\begin{tikzinline}[node distance=.2em]
      \event{b}{\DRreg{s}{x}{2}}{}
      \final{f}{\SUB{}}{below=of b}
      \end{tikzinline}}
  \end{gathered}
\end{gather*}

\subsection{Commuting release and acquire}

RA example.  This is impossible, since $\DR{x}{1}$ unfulfilled.
\begin{gather*}
  x\GETS1 \SEMI
  a^\mRA\GETS1 \SEMI
  %y \GETS b^\mRA + x
  r \GETS b^\mRA\SEMI
  s \GETS x\SEMI
  y \GETS r+s
  \PAR
  r\GETS a^\mRA\SEMI
  x\GETS 2\SEMI
  b^\mRA\GETS10
  \\
  \hbox{\begin{tikzinline}[node distance=.8em and 1em]
  \event{a1}{\DW{x}{1}}{}
  \event{a2}{\DWRel{a}{1}}{right=of a1}
  \sync{a1}{a2}
  \event{b3}{\DRAcq{a}{1}}{below=of a2}
  \rf{a2}{b3}
  \event{b4}{\DW{x}{2}}{right=of b3}
  \sync{b3}{b4}
  \event{b5}{\DWRel{b}{10}}{right=of b4}
  \sync{b4}{b5}
  \event{a6}{\DRAcq{b}{10}}{above=of b5}
  \rf{b5}{a6}
  \event{a7}{\DR{x}{1}}{right=of a6}
  \sync{a6}{a7}
  \event{a8}{\DW{y}{11}}{right=of a7}
  \po{a7}{a8}
  %\sync[out=10,in=170]{a6}{a8}
    \end{tikzinline}}
\end{gather*}
If you swap the release and acquire, then it is impossible for the second
thread to get in the middle.
\begin{gather*}
  x\GETS1 \SEMI
  %y \GETS b^\mRA + x
  r \GETS b^\mRA\SEMI
  a^\mRA\GETS1 \SEMI
  % s \GETS x\SEMI
  % y \GETS r+s
  \PAR
  r\GETS a^\mRA\SEMI
  x\GETS 2\SEMI
  b^\mRA\GETS10
  \\
  \hbox{\begin{tikzinline}[node distance=.8em and 1em]
  \event{a1}{\DW{x}{1}}{}
  \event{a2}{\DWRel{a}{1}}{right=of a1}
  \sync{a1}{a2}
  \event{b3}{\DRAcq{a}{1}}{below=of a2}
  \rf{a2}{b3}
  \event{b4}{\DW{x}{2}}{right=of b3}
  \sync{b3}{b4}
  \event{b5}{\DWRel{b}{10}}{right=of b4}
  \sync{b4}{b5}
  \event{a6}{\DRAcq{b}{10}}{above=of b5}
  \rf{b5}{a6}
  % \event{a7}{\DR{x}{1}}{right=of a6}
  % \sync{a6}{a7}
  % \event{a8}{\DW{y}{11}}{right=of a7}
  % \po{a7}{a8}
  \sync{a6}{a2}
  %\sync[out=10,in=170]{a6}{a8}
    \end{tikzinline}}
\end{gather*}
In this case, the following execution is possible:
\begin{gather*}
  x\GETS1 \SEMI
  r \GETS b^\mRA\SEMI
  a^\mRA\GETS1 \SEMI
  %y \GETS b^\mRA + x
  s \GETS x\SEMI
  y \GETS r+s
  \PAR
  r\GETS a^\mRA\SEMI
  x\GETS 2\SEMI
  b^\mRA\GETS10
  \\
  \hbox{\begin{tikzinline}[node distance=.8em and 1em]
  \event{a1}{\DW{x}{1}}{}
  \event{a2}{\DRAcq{b}{10}}{right=of a1}
  \event{b5}{\DWRel{b}{10}}{below=of a2}
  \event{b4}{\DW{x}{2}}{left=of b5}
  \event{b3}{\DRAcq{a}{0}}{left=of b4}
  \sync{b4}{b5}
  \sync{b3}{b4}
  \event{a6}{\DWRel{a}{1}}{right=of a2}
  \rf{b5}{a2}
  \event{a7}{\DR{x}{1}}{right=of a6}
  \sync{a6}{a7}
  \event{a8}{\DW{y}{11}}{right=of a7}
  \po{a7}{a8}
  \sync[out=15,in=165]{a1}{a6}
  \sync{a2}{a6}
  \wk{b4}{a1}
  %\sync[out=10,in=170]{a6}{a8}
    \end{tikzinline}}
\end{gather*}
But not:
\begin{gather*}
  x\GETS1 \SEMI
  r \GETS b^\mRA\SEMI
  a^\mRA\GETS1 \SEMI
  %y \GETS b^\mRA + x
  s \GETS x\SEMI
  y \GETS r+s
  \PAR
  r\GETS a^\mRA\SEMI
  x\GETS 2\SEMI
  b^\mRA\GETS10
  \\
  \hbox{\begin{tikzinline}[node distance=.8em and 1em]
  \event{a1}{\DW{x}{1}}{}
  \event{a2}{\DRAcq{b}{10}}{right=of a1}
  \event{b5}{\DWRel{b}{10}}{below=of a2}
  \event{b4}{\DW{x}{2}}{left=of b5}
  \event{b3}{\DRAcq{a}{0}}{left=of b4}
  \sync{b4}{b5}
  \sync{b3}{b4}
  \event{a6}{\DWRel{a}{1}}{right=of a2}
  \rf{b5}{a2}
  \event{a7}{\DR{x}{1}}{right=of a6}
  \sync{a6}{a7}
  \event{a8}{\DW{y}{11}}{right=of a7}
  \po{a7}{a8}
  \sync[out=15,in=165]{a1}{a6}
  \sync{a2}{a6}
  \wk{a1}{b4}
  \wk[out=-155,in=-30]{a7}{b4}
  %\sync[out=10,in=170]{a6}{a8}
    \end{tikzinline}}
\end{gather*}

\subsection{Coherence}
 Our model of coherence is as weak as that of
\cite{Dolan:2018:BDR:3192366.3192421}:
\begin{gather*}
  x\GETS1\SEMI x\GETS 2
  \PAR
  r_1\GETS x \SEMI
  r_2\GETS x \SEMI
  r_3\GETS x \SEMI
  \\
  \hbox{\begin{tikzinline}[node distance=.8em]
    \event{a1}{\DW{x}{1}}{}
    \event{a2}{\DW{x}{2}}{right=1em of a1}
    \wk{a1}{a2}
    \event{b1}{\DRreg{r_1}{x}{2}}{below=of a1}
    \event{b2}{\DRreg{r_2}{x}{1}}{below=of a2}
    \event{b3}{\DRreg{r_3}{x}{2}}{right=1em of b2}
    % \po{b1}{b2}
    % \po{b2}{b3}
    \rf{a2}{b1}
    \rf{a1}{b2}
    \rf{a2}{b3}
    \wk{b2}{a2}
    \end{tikzinline}}
\end{gather*}

\subsection{Write rule}
Alan example of why substitute M/x rather than v/x in the write rule:
\begin{gather*}
  r\GETS y\SEMI x\GETS r\SEMI s\GETS  x\SEMI z\GETS s
  \\
  \hbox{\begin{tikzinline}[node distance=.8em and 1em]
      \event{a1}{\DR{y}{1}}{}
      \event{a2}{\DW{x}{1}}{right=of a1}
      \event{a3}{\DR{x}{1}}{right=of a2}
      \event{a4}{\DW{z}{1}}{right=of a3}
      % \po[out=15,in=165]{a1}{a4}
      \po{a1}{a2}
    \end{tikzinline}}
\end{gather*}
We lost the order from $\DR{y}{1}$ to $\DW{z}{1}$.
\begin{gather*}
   s\GETS  x\SEMI z\GETS s
  \\
  \hbox{\begin{tikzinline}[node distance=.8em and 1em]
      \event{a3}{\DR{x}{1}}{}
      \event{a4}{x\EQ1\mid\DW{z}{1}}{right=of a3}
    \end{tikzinline}}
\end{gather*}
\begin{gather*}
  x\GETS r \SEMI s\GETS  x\SEMI z\GETS s
  \\
  \hbox{\begin{tikzinline}[node distance=.8em and 1em]
      \event{a2}{\DW{x}{1}}{}
      \event{a3}{\DR{x}{1}}{right=of a2}
      \event{a4}{1\EQ1\mid\DW{z}{1}}{right=of a3}
    \end{tikzinline}}
  \\
  \hbox{\begin{tikzinline}[node distance=.8em and 1em]
      \event{a2}{\DW{x}{1}}{}
      \event{a3}{\DR{x}{1}}{right=of a2}
      \event{a4}{r\EQ1\mid\DW{z}{1}}{right=of a3}
    \end{tikzinline}}
\end{gather*}


\subsection{CSE example}
Pugh example for alias analysis and CSE:
\begin{gather*}
  r_1\GETS x \SEMI
  r_2\GETS x \SEMI  
  r_3\GETS x \SEMI
  \IF{r_3{\leq}1}\THEN y=r_2\FI
  \\
  \hbox{\begin{tikzinline}[node distance=.8em and 1em]
      \event{a1}{\DR{x}{1}}{}
      \event{a2}{\DR{x}{2}}{right=of a1}
      \event{a3}{\DR{x}{1}}{right=of a2}
      \event{a4}{\DW{y}{2}}{right=of a3}
      \po{a3}{a4}
      \po[out=15,in=165]{a2}{a4}
    \end{tikzinline}}
\end{gather*}
\begin{gather*}
  r_1\GETS x \SEMI
  r_2\GETS x \SEMI  
  r_3\GETS r_1 \SEMI
  \IF{r_3{\leq}1}\THEN y=r_2\FI
  \\
  \hbox{\begin{tikzinline}[node distance=.8em and 1em]
      \event{a1}{\DR{x}{1}}{}
      \event{a2}{\DR{x}{2}}{right=of a1}
      %\event{a3}{\DR{x}{1}}{right=of a2}
      \event{a4}{\DW{y}{2}}{right=4em of a2}
      \po{a2}{a4}
      \po[out=15,in=165]{a1}{a4}
    \end{tikzinline}}
\end{gather*}
I don't see the problem with this for now.  Is this sound????


\subsection{Playing around with 5a and 4b}
If we do this, then swap 4b and 4c, In definition 2.10, take 1-4b of def 2.8,
rather than all of it.

Another
\begin{gather*}
  r\GETS x
  \SEMI s\GETS x
  \SEMI \IF{r{>}0}\THEN y\GETS 1\FI
  \SEMI \IF{s{>}0}\THEN z\GETS 1\FI
  \\
  r\GETS x
  \SEMI \IF{r{>}0}\THEN y\GETS 1\FI
  \SEMI s\GETS x
  \SEMI \IF{s{>}0}\THEN z\GETS 1\FI
  \\
  \hbox{\begin{tikzinline}[node distance=1em]
      \event{a}{\DRreg{r}{x}{1}}{}
      \event{b}{\DRreg{s}{x}{2}}{right=of a}
      \event{c}{\DW{y}{1}}{right=of b}
      \event{d}{\DW{z}{1}}{right=of c}
      \po[out=-15,in=-165]{a}{c}
      \po[out=15,in=165]{b}{d}
    \end{tikzinline}}
  \\
  \hbox{\begin{tikzinline}[node distance=1em]
      \event{a}{\DRreg{r}{x}{1}}{}
      \event{b}{\DRreg{s}{x}{2}}{right=of a}
      \event{c}{\DW{y}{1}}{right=of b}
      \event{d}{\DW{z}{1}}{right=of c}
      \po[out=-15,in=-165]{a}{c}
      \po[out=15,in=165]{a}{d}
    \end{tikzinline}}
\end{gather*}          
\begin{gather*}
  s\GETS x
  \SEMI r\GETS x
  \SEMI \IF{r{>}0}\THEN y\GETS 1\FI
  \SEMI \IF{s{>}0}\THEN z\GETS 1\FI
  \\
  s\GETS x
  \SEMI \IF{s{>}0}\THEN z\GETS 1\FI
  \SEMI r\GETS x
  \SEMI \IF{r{>}0}\THEN y\GETS 1\FI
  \\
  \hbox{\begin{tikzinline}[node distance=1em]
      \event{a}{\DRreg{r}{x}{1}}{}
      \event{b}{\DRreg{s}{x}{2}}{right=of a}
      \event{c}{\DW{y}{1}}{right=of b}
      \event{d}{\DW{z}{1}}{right=of c}
      \po[out=-15,in=-165]{a}{c}
      \po[out=15,in=165]{b}{d}
    \end{tikzinline}}
  \\
  \hbox{\begin{tikzinline}[node distance=1em]
      \event{a}{\DRreg{r}{x}{1}}{}
      \event{b}{\DRreg{s}{x}{2}}{right=of a}
      \event{c}{\DW{y}{1}}{right=of b}
      \event{d}{\DW{z}{1}}{right=of c}
      \po{b}{c}
      \po[out=15,in=165]{b}{d}
    \end{tikzinline}}
\end{gather*}          
\begin{gather*}
  s\GETS x
  \SEMI \IF{r{>}0}\THEN y\GETS 1\FI
  \SEMI \IF{s{>}0}\THEN z\GETS 1\FI
  \\
  \hbox{\begin{tikzinline}[node distance=1em]
      \event{b}{\DRreg{s}{x}{2}}{}
      \event{c}{r{>}0\mid\DW{y}{1}}{right=of b}
      \event{d}{\DW{z}{1}}{right=of c}
      %\po{b}{c}
      \po[out=15,in=165]{b}{d}
    \end{tikzinline}}
\end{gather*}          
% For the desired result, it is sufficient if
% \begin{gather*}
%   s\GETS x
%   \SEMI \IF{r{>}0}\THEN y\GETS 1\FI
%   \SEMI \IF{s{>}0}\THEN z\GETS 1\FI
%   \\
%   \hbox{\begin{tikzinline}[node distance=1em]
%       \event{b}{\DRreg{s}{x}{2}}{}
%       \event{c}{r{=}x\mid\DW{y}{1}}{right=of b}
%       \event{d}{\DW{z}{1}}{right=of c}
%       \po{b}{c}
%       \po[out=15,in=165]{b}{d}
%     \end{tikzinline}}
% \end{gather*}          

\begin{gather*}
  \begin{gathered}
  r\GETS x
  \SEMI s\GETS x
    \\[-1ex]
    \hbox{\begin{tikzinline}[node distance=.2em]
      \event{a}{\DRreg{r}{x}{1}}{}
      \event{b}{\DRreg{s}{x}{2}}{right=of a}
      \final{f}{\SUB{x/r,x/s}}{below=of a}
      \end{tikzinline}}
  \end{gathered}
  \qquad
  \begin{gathered}
    \IF{r{>}0}\THEN y\GETS 1\FI
    \SEMI \IF{s{>}0}\THEN z\GETS 1\FI
\\[-1ex]
    \hbox{\begin{tikzinline}[node distance=.2em]
      \event{c}{r{>}0 \mid \DW{y}{1}}{}
      \event{d}{s{>}0 \mid \DW{z}{1}}{right=of c}
      \final{f}{r{>}0 \land s{>}0 \mid \SUB{1/y,1/z}}{below=of c}
      \end{tikzinline}}
  \end{gathered}
\end{gather*}

Idea to get rid of 4b and change 5a to the following:
\begin{itemize}
\item[5a.]  if $\aEv$ writes then either $\labelingForm'(\aEv)$ implies
  $\labelingForm(\aEv)$, or some $\cEv\lt'\aEv$ reads $\aVal$
  from $\aLoc$ and $\labelingForm'(\aEv)$ implies $\labelingForm(\aEv)[\aVal/\aLoc]$,
\end{itemize}
Need to get rid of 4b because it is sensitive to order of reads.

This change seems sound, because of consistency.  But it also fails to
validate read reordering on same variable, due to consistency.

Without 4b, we still do not allow:
\begin{gather*}
  r\GETS x\SEMI
  s\GETS x\SEMI
  y\GETS r\SEMI
  z\GETS r
  \\[-1ex]
  \nonumber
  \hbox{\begin{tikzinline}[node distance=.5em and 1em]
      \event{a1}{\DR{x}{1}}{}
      \event{a2}{\DR{x}{2}}{right=of a1}
      \event{a3}{\DW{y}{1}}{right=of a2}
      \event{a4}{\DW{z}{2}}{right=of a3}
      \po[out=15,in=165]{a1}{a3}
      \po[out=15,in=165]{a2}{a4}
    \end{tikzinline}}
\end{gather*}
The following is not a pomset (consistency):
\begin{gather*}
  y\GETS r\SEMI
  z\GETS r
  \\[-1ex]
  \nonumber
  \hbox{\begin{tikzinline}[node distance=.5em and 1em]
      \event{a3}{r\EQ1\mid\DW{y}{1}}{right=of a2}
      \event{a4}{r\EQ2\mid\DW{z}{2}}{right=of a3}
    \end{tikzinline}}
\end{gather*}

Without 4b, we still do not allow:
\begin{gather*}
  r\GETS x\SEMI
  s\GETS x\SEMI
  y\GETS r\SEMI
  z\GETS s\SEMI
  \IF{r{=}s}\THEN a\GETS 1\FI\SEMI
  \\[-1ex]
  \nonumber
  \hbox{\begin{tikzinline}[node distance=.5em and 1em]
      \event{a1}{\DR{x}{1}}{}
      \event{a2}{\DR{x}{2}}{right=of a1}
      \event{a3}{\DW{y}{1}}{right=of a2}
      \event{a4}{\DW{z}{2}}{right=of a3}
      \event{a5}{x{=}x\mid\DW{a}{1}}{right=of a4}
      \po[out=15,in=165]{a1}{a3}
      \po[out=15,in=165]{a2}{a4}
    \end{tikzinline}}
\end{gather*}
The following is not a pomset (consistency):
\begin{gather*}
  y\GETS r\SEMI
  z\GETS s\SEMI
  \IF{r{=}s}\THEN a\GETS 1\FI\SEMI
  \\[-1ex]
  \nonumber
  \hbox{\begin{tikzinline}[node distance=.5em and 1em]
      \event{a3}{r{=}1\mid\DW{y}{1}}{}
      \event{a4}{s{=}2\mid\DW{z}{2}}{right=of a3}
      \event{a5}{r{=}s\mid\DW{a}{1}}{right=of a4}
    \end{tikzinline}}
\end{gather*}

We do allow:
\begin{gather*}
  r\GETS x\SEMI
  s\GETS x\SEMI
  \IF{r{=}s}\THEN a\GETS 1\FI\SEMI
  \\[-1ex]
  \nonumber
  \hbox{\begin{tikzinline}[node distance=.5em and 1em]
      \event{a1}{\DR{x}{1}}{}
      \event{a2}{\DR{x}{2}}{right=of a1}
      \event{a3}{x{=}x\mid\DW{a}{1}}{right=of a2}
    \end{tikzinline}}
\end{gather*}
% \begin{gather*}
%   s\GETS x\SEMI
%   \IF{r{=}s}\THEN a\GETS 1\FI\SEMI
%   \\[-1ex]
%   \nonumber
%   \hbox{\begin{tikzinline}[node distance=.5em and 1em]
%       \event{a2}{\DR{x}{2}}{}
%       \event{a3}{r{=}x\mid\DW{a}{1}}{right=of a2}
%     \end{tikzinline}}
% \end{gather*}
And also
\begin{gather*}
  r_1\GETS x\SEMI
  r_2\GETS x\SEMI
  r_3\GETS x\SEMI
  \IF{r_3{\leq}1}\THEN y\GETS 1\FI\SEMI
  \\[-1ex]
  \nonumber
  \hbox{\begin{tikzinline}[node distance=.5em and 1em]
      \event{a0}{\DR{x}{0}}{}
      \event{a1}{\DR{x}{2}}{right=of a0}
      \event{a2}{\DR{x}{1}}{right=of a1}
      \event{a3}{1{\leq}1\mid\DW{y}{1}}{right=of a2}
      \po[out=15,in=165]{a0}{a3}
    \end{tikzinline}}
\end{gather*}

But we cannot wait forever to satisfy a precondition.
This is not a pomset:
\begin{gather*}
  r\GETS x\SEMI
  s\GETS x\SEMI
  y\GETS r\SEMI
  z\GETS s
  \\[-1ex]
  \nonumber
  \hbox{\begin{tikzinline}[node distance=.5em and 1em]
      \event{a3}{\DR{x}{3}}{}
      \event{a4}{\DR{x}{4}}{right=of a3}
      \event{a5}{x{=}1\mid\DW{y}{1}}{right=of a4}
      \event{a6}{x{=}2\mid\DW{z}{2}}{right=of a5}
    \end{tikzinline}}
\end{gather*}
Note that reads that we delay must all be consistent.

Also note that we cannot have:
\begin{gather*}
  r\GETS x\SEMI a\GETS r\SEMI
  s\GETS x\SEMI b\GETS s\SEMI
  y\GETS r\SEMI
  z\GETS s
  \\[-1ex]
  \nonumber
  \hbox{\begin{tikzinline}[node distance=.5em and 1em]
      \event{a3}{\DR{x}{3}}{}
      \event{a4}{\DR{x}{4}}{right=of a3}
      \event{a5}{x{=}1\mid\DW{y}{1}}{right=of a4}
      \event{a6}{x{=}1\mid\DW{z}{2}}{right=of a5}
      \event{b3}{\DW{a}{3}}{below=of a3}
      \event{b4}{\DW{b}{4}}{below=of a4}
      \po{a3}{b3}
      \po{a4}{b4}
    \end{tikzinline}}
\end{gather*}
Because the following is not a pomset:
\begin{gather*}
  b\GETS s\SEMI
  y\GETS r\SEMI
  z\GETS s
  \\[-1ex]
  \nonumber
  \hbox{\begin{tikzinline}[node distance=.5em and 1em]
      \event{a5}{r{=}1\mid\DW{y}{1}}{right=of a4}
      \event{a6}{s{=}1\mid\DW{z}{2}}{right=of a5}
      \event{b4}{s{=}4\mid\DW{b}{4}}{below left=of a5}
    \end{tikzinline}}
\end{gather*}
But we can have the following, since there is no order the reads:
\begin{gather*}
  r_1\GETS x\SEMI
  s_1\GETS x\SEMI  
  r_2\GETS x\SEMI
  s_2\GETS x\SEMI
  y\GETS r_2\SEMI
  z\GETS s_2
  \\[-1ex]
  \nonumber
  \hbox{\begin{tikzinline}[node distance=.5em and 1em]
      \event{a1}{\DR{x}{1}}{}
      \event{a2}{\DR{x}{2}}{right=of a1}
      \event{a3}{\DR{x}{3}}{right=of a2}
      \event{a4}{\DR{x}{4}}{right=of a3}
      \event{a5}{\DW{y}{1}}{right=of a4}
      \event{a6}{\DW{z}{2}}{right=of a5}
      \po[out=15,in=165]{a1}{a5}
      \po[out=15,in=165]{a2}{a6}
    \end{tikzinline}}
\end{gather*}
Because this is indistinguishable from:
\begin{gather*}
  r_1\GETS x\SEMI
  s_1\GETS x\SEMI  
  r_2\GETS x\SEMI
  s_2\GETS x\SEMI
  y\GETS r_2\SEMI
  z\GETS s_2
  \\[-1ex]
  \nonumber
  \hbox{\begin{tikzinline}[node distance=.5em and 1em]
      \event{a1}{\DR{x}{3}}{}
      \event{a2}{\DR{x}{4}}{right=of a1}
      \event{a3}{\DR{x}{1}}{right=of a2}
      \event{a4}{\DR{x}{2}}{right=of a3}
      \event{a5}{\DW{y}{1}}{right=of a4}
      \event{a6}{\DW{z}{2}}{right=of a5}
      \po[out=15,in=165]{a3}{a5}
      \po[out=15,in=165]{a4}{a6}
    \end{tikzinline}}
\end{gather*}
which is the same as:
\begin{gather*}
  r_1\GETS x\SEMI
  r_2\GETS x\SEMI
  y\GETS r_2\SEMI
  s_1\GETS x\SEMI  
  s_2\GETS x\SEMI
  z\GETS s_2
  \\[-1ex]
  \nonumber
  \hbox{\begin{tikzinline}[node distance=.5em and 1em]
      \event{a1}{\DR{x}{1}}{}
      \event{a2}{\DR{x}{3}}{right=of a1}
      \event{a3}{\DW{y}{1}}{right=of a2}
      \event{a4}{\DR{x}{2}}{right=of a3}
      \event{a5}{\DR{x}{4}}{right=of a4}
      \event{a6}{\DW{z}{2}}{right=of a5}
      \po[out=15,in=165]{a1}{a3}
      \po[out=15,in=165]{a4}{a6}
    \end{tikzinline}}
\end{gather*}

But we can have:
\begin{gather*}
  p\GETS x\SEMI
  r\GETS x\SEMI
  s\GETS x\SEMI
  y\GETS r\SEMI
  z\GETS s
  \\[-1ex]
  \nonumber
  \hbox{\begin{tikzinline}[node distance=.5em and 1em]
      \event{a1}{\DR{x}{3}}{}
      \event{a2}{\DR{x}{4}}{right=of a1}
      \event{a3}{x{=}1\mid\DW{y}{1}}{right=of a2}
      \event{a4}{x{=}1\mid\DW{z}{1}}{right=of a3}
      \event{b2}{\DR{x}{1}}{left=of a1}
      \po[out=15,in=165]{b2}{a3}
      \po[out=-15,in=-165]{b2}{a4}
      %\po[out=15,in=165]{a1}{a3}
      %\po[out=15,in=165]{a2}{a4}
    \end{tikzinline}}
\end{gather*}

Reads can only swap when their values are interchangeable in the following
program.

\subsection{Commments for revision}

Can move prefix closure out of model and just put it in the logic section.

In semicolon semantics, join is asymmetric.

State theorem for the join-free fragment, since prefixing has no joins.

Our address calculation NO-TAR example should be replaced by JMM TC12.


\subsection{Alan comments}

\begin{verbatim}
  x=s; y=r; z=3s+2r

  x=s; y=r; z1=s; if(r odd){ z2=1} // using 1 and 3 as the reads
\end{verbatim}
