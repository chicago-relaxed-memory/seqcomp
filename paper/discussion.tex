\begin{figure*}
  \begin{center}
    \begin{minipage}{.91\textwidth}
      \noindent
If $\aPS \in \sSTORE[\amode]{\cExp}{\aExp}$ then
$(\exists\cVal:\aEvs\fun\Val)$
$(\exists\aVal:\aEvs\fun\Val)$
$(\exists\cForm:\aEvs\fun\Formulae)$
\begin{enumerate}
\item[\ref{S1})] if $\cForm_\bEv\land\cForm_\aEv$ is satisfiable then $\bEv=\aEv$,
\item[\ref{S2})] $\labelingAct(\aEv) = \DWREF{\cVal_\aEv}{\aVal_\aEv}$,
\item[\ref{S3})] 
  $\labelingForm(\aEv)$ implies
  \begin{math}
    \cForm_\aEv
    \land \QS{\REF{\cVal_\aEv}}{\amode}
    % \land \RW
    \land \cExp{=}\cVal_\aEv
    \land \aExp{=}\aVal_\aEv
  \end{math},
  % where
  % $\QS{}{\mRLX}=\QxREF{\cVal_\aEv}$ and otherwise $\QS{}{\amode}=\Q{\amode}$, % for $\amode\neq\mRLX$,
\item[\ref{S4})]
  \begin{math}
    (\forall\dVal)
    (\forall\aEv\in\aEvs\cap\bEvs)
  \end{math}
  $\aTr{\bEvs}{\bForm}$ \;implies \,
  \begin{math}
    \cForm_\aEv
    \limplies (\cExp{=}\dVal)
    \limplies \PBR{
      %(\QwREF{\dVal} \limplies \aExp{=}\aVal_\aEv) \land
      \bForm
      [\aExp/\REF{\dVal}]
      \DS{\REF{\dVal}}{\amode}
      [(\QwREF{\dVal}\land\aExp{=}\aVal)/\QwREF{\dVal}]
    }
  \end{math},
\item[\ref{S5})] %if 
  % \begin{math}
  %   (\forall\aEv\in\bEvs)(\cForm \textimplies
  %   \lnot\cForm_\aEv)
  % \end{math}
  % then
  \begin{math}
    (\forall\dVal)
  \end{math}
  $\aTr{\cEvs}{\bForm}$ implies
  \begin{math}
    (\!\not\exists\aEv\in\aEvs\cap\cEvs \suchthat \cForm_\aEv)
    \limplies (\cExp{=}\dVal)
    \limplies \PBR{
      % \lnot\QwREF{\dVal} \land
      \bForm
      [\aExp/\REF{\dVal}]
      \DS{\REF{\dVal}}{\amode}
      [\FALSE/\QS{\REF{\dVal}}{\amode}]
    }.
  \end{math}
  % \\ where 
  % $\DS{}{\mRLX}{}=[\TRUE/\DxREF{\dVal}]$ and otherwise
  % $\DS{}{\amode}{}=[\FALSE/\D]$. % for $\amode\neq\mRLX$.
\end{enumerate}
% \item if $\amode=\mRLX$ then
%   $\labelingForm(\aEv)$ implies
%   \begin{math}
%     \cForm_\aEv
%     \land \cExp{=}\cVal_\aEv
%     \land \aExp{=}\aVal_\aEv
%     \land \RW
%     \land \QxREF{\cVal_\aEv},
%   \end{math}
% \item if $\amode\neq\mRLX$ then
%   $\labelingForm(\aEv)$ implies
%   \begin{math}
%     \cForm_\aEv
%     \land \cExp{=}\cVal_\aEv
%     \land \aExp{=}\aVal_\aEv
%     \land \RW
%     \land \Q{},
%   \end{math}
% \item if
%   $\aEv\in\bEvs$
%   and
%   $\amode=\mRLX$ then
%   \begin{math}
%     (\forall\dVal)
%   \end{math}
%   $\aTr{\bEvs}{\bForm}$ implies 
%   \begin{math}
%     \cForm_\aEv
%     \limplies (\cExp{=}\dVal)
%     \limplies \PBRbig{
%     (\QwREF{\dVal} \limplies \aExp{=}\aVal_\aEv)
%     \land \bForm[\aExp/\REF{\dVal}][\TRUE/\DxREF{\dVal}]
%   }
%   \end{math}
% \item if
%   $\aEv\in\bEvs$
%   and
%   $\amode\neq\mRLX$ then
%   \begin{math}
%     (\forall\dVal)
%   \end{math}
%   $\aTr{\bEvs}{\bForm}$ implies 
%   \begin{math}
%     \cForm_\aEv
%     \limplies (\cExp{=}\dVal)
%     \limplies \PBRbig{
%     (\QwREF{\dVal} \limplies \aExp{=}\aVal_\aEv)
%     \land \bForm[\aExp/\REF{\dVal}][\FALSE/\D]
%   }
%   \end{math}
% \item if 
%   \begin{math}
%     (\forall\aEv\in\bEvs)(\cForm \textimplies
%     \lnot\cForm_\aEv)
%   \end{math}
%   and $\amode=\mRLX$ 
%   then
%   \begin{math}
%     (\forall\dVal)
%   \end{math}
%   $\aTr{\bEvs}{\bForm}$ implies 
%   \begin{math}
%     \cForm
%     \limplies (\cExp{=}\dVal)
%     \limplies \PBRbig{
%     \lnot\QwREF{\dVal}
%     \land \bForm[\aExp/\REF{\dVal}][\TRUE/\DxREF{\dVal}]
%   }
%   \end{math}
% \item if 
%   \begin{math}
%     (\forall\aEv\in\bEvs)
%     (\cForm \textimplies \lnot\cForm_\aEv)
%   \end{math}
%   and $\amode\neq\mRLX$ 
%   then
%   \begin{math}
%     (\forall\dVal)
%   \end{math}
%   $\aTr{\bEvs}{\bForm}$ implies 
%   \begin{math}
%     \cForm
%     \limplies (\cExp{=}\dVal)
%     \limplies \PBRbig{
%     \lnot\QwREF{\dVal}
%     \land \bForm[\aExp/\REF{\dVal}][\FALSE/\D]
%   }
%   \end{math}

\noindent
If $\aPS \in \sLOAD[\amode]{\aReg}{\cExp}$ then
$(\exists\cVal:\aEvs\fun\Val)$
$(\exists\aVal:\aEvs\fun\Val)$
$(\exists\cForm:\aEvs\fun\Formulae)$
% $(\forall\uReg{\aEv}\in\uRegs{\aEvs})$
\begin{enumerate}
\item[\ref{L1})] if $\cForm_\bEv\land\cForm_\aEv$ is satisfiable then $\bEv=\aEv$,
\item[\ref{L2})] $\labelingAct(\aEv) = \DRREF{\cVal_\aEv}{\aVal_\aEv}$,
\item[\ref{L3})] $\labelingForm(\aEv)$ implies
  \begin{math}
    \cForm_\aEv
    \land \QL{\REF{\cVal_\aEv}}{\amode}
    % \land \RO
    \land \cExp{=}\cVal_\aEv
  \end{math},
  % where    
  % $\QL{}{\mSC}=\Q{\mSC}$ and otherwise $\QL{}{\amode}=\QwREF{\cVal_\aEv}$, % for $\amode\neq\mRLX$,
\item[\ref{L4})]
  \begin{math}
    (\forall\dVal)
    (\forall\aEv\in\aEvs\cap\bEvs)
  \end{math}
  $\aTr{\bEvs}{\bForm}$ implies
  \begin{math}
    \cForm_\aEv
    \limplies (\cExp{=}\dVal)
    \limplies (\aVal_\aEv{=}\uReg{\aEv})
    \limplies \bForm[\uReg{\aEv}/\aReg]%[\uReg{\aEv}/\REF{\dVal}]
  \end{math},
  \makebox[5.75cm]{}
\item[\ref{L5})] 
  \begin{math}
    (\forall\dVal)
    (\forall\aEv\in\aEvs\setminus\cEvs)
  \end{math}
  $\aTr{\cEvs}{\bForm}$ implies
  \begin{math}
    \cForm_\aEv
    \limplies (\cExp{=}\dVal)
    \limplies \PBR{        
      %\lnot\QxREF{\dVal}\land
      \DL{\REF{\dVal}}{\amode} \land
      (\RW
      \limplies (\aVal_\aEv{=}\uReg{\aEv}\lor\REF{\dVal}{=}\uReg{\aEv}) 
      \limplies
      \bForm
      [\uReg{\aEv}/\aReg]%[\uReg{\aEv}/\REF{\dVal}]
      [\FALSE/\QL{\REF{\dVal}}{\amode}]
      )
    }      
  \end{math},
\item[\ref{L6})] % if 
  % \begin{math}
  %   (\forall\aEv\in\bEvs)(\cForm \textimplies
  %   \lnot\cForm_\aEv)
  % \end{math}
  % then
  \begin{math}
    (\forall\dVal)
    (\forall\bReg)
  \end{math}
  $\aTr{\dEvs}{\bForm}$  implies 
  \begin{math}
    (\!\not\exists\aEv\in\aEvs \suchthat \cForm_\aEv)
    \limplies (\cExp{=}\dVal)
    \limplies \PBR{        
      %\lnot\QxREF{\dVal} \land
      \DL{\REF{\dVal}}{\amode} \land
      \bForm
      [\bReg/\aReg]%[\bReg/\REF{\dVal}]
      [\FALSE/\QL{\REF{\dVal}}{\amode}]
    }.
  \end{math}
  % \\ where $\DL{}{\mRLX}=\TRUE$ and otherwise $\DL{}{\amode}=\DxREF{\dVal}$.
  % Recall that $\uRegs{\bEvs}=\{\uReg{\aEv}\mid\aEv\in\bEvs\}$.
\end{enumerate}  
% \item if $\amode=\mRLX$ and $\bEv\notin\bEvs$ then
%   \begin{math}
%     (\forall\dVal)
%   \end{math}
%   $\aTr{\bEvs}{\bForm}$ implies
%   \begin{math}
%     \cForm_\bEv
%     \limplies (\cExp{=}\dVal)
%     \limplies \PBRbig{
%     (
%     \RW
%     \limplies (\aVal{=}\uReg{\bEv}\lor\aLoc{=}\uReg{\bEv}) 
%     \limplies \bForm[\uReg{\bEv}/\aReg][\uReg{\bEv}/\REF{\dVal}]
%     )
%     \land \lnot\QxREF{\dVal}
%   }
%     \phantom{\land\; \Dx{\dVal}}
%   \end{math}
% \item if $\amode\neq\mRLX$ and $\bEv\notin\bEvs$ then
%   \begin{math}
%     (\forall\dVal)
%   \end{math}
%   $\aTr{\bEvs}{\bForm}$ implies
%   \begin{math}
%     \cForm_\bEv
%     \limplies (\cExp{=}\dVal)
%     \limplies \PBRbig{
%     (
%     \RW
%     \limplies (\aVal{=}\uReg{\bEv}\lor\aLoc{=}\uReg{\bEv}) 
%     \limplies \bForm[\uReg{\bEv}/\aReg][\uReg{\bEv}/\REF{\dVal}]
%     )
%     \land \lnot\QxREF{\dVal}
%     \land \Dx{\dVal}
%   }
%   \end{math}

\noindent
If $\aPS \in \sTHREAD{\aPSS}$ then
$(\exists\aPS_1\in\aPSS)$
\begin{enumerate}
\item[\ref{T1})]
  $\aEvs=\aEvs_1$,
\item[\ref{T2})]
  $\labelingAct(\aEv) = \labelingAct_1(\aEv)$,
\item[\ref{T3})]
  $\labelingForm(\aEv)$ implies
  $\labelingForm_1(\aEv) [\TRUE/\Qr{*}][\TRUE/\Qw{*}][\TRUE/\Qsc][\TRUE/\RW]$ if $\labelingAct_1(\aEv)$ is a write,
  \\
  $\labelingForm(\aEv)$ implies
  $\labelingForm_1(\aEv) [\TRUE/\Qr{*}][\TRUE/\Qw{*}][\TRUE/\Qsc][\FALSE/\RW]$ otherwise.
\end{enumerate}  

      % \noindent
If $\aPS \in \sSTORE[\amode]{\cExp}{\aExp}$ then
$(\exists\cVal:\aEvs\fun\Val)$
$(\exists\aVal:\aEvs\fun\Val)$
$(\exists\cForm:\aEvs\fun\Formulae)$
\begin{enumerate}
\item if $\cForm_\bEv\land\cForm_\aEv$ is satisfiable then $\bEv=\aEv$,
\item $\labelingAct(\aEv) = \DWREFP{\cVal_\aEv}{\aVal_\aEv}$,
\item 
  $\labelingForm(\aEv)$ implies
  \begin{math}
    \cForm_\aEv
    \land \cExp{=}\cVal_\aEv
    \land \aExp{=}\aVal_\aEv
    \land \RW
    \land \Qmode{\amode}
  \end{math},
  where
  $\Qmode{\mRLX}=\QxREF{\cVal_\aEv}$ and otherwise $\Qmode{\amode}=\Q{\amode}$, % for $\amode\neq\mRLX$,
\item
  \begin{math}
    (\forall\dVal)
  \end{math}
  if
  $\bEv\in\bEvs$
  then
  $\aTr{\bEvs}{\aForm}$ implies 
  \begin{math}
    \cForm_\bEv
    \limplies (\cExp{=}\dVal)
    \limplies \PBRbig{
      (\QwREF{\dVal} \limplies \aExp{=}\aVal_\bEv)
      \land \aForm [\aExp/\REF{\dVal}]\Dmode{\amode}
    }
  \end{math},
\item %if 
  % \begin{math}
  %   (\forall\bEv\in\bEvs)(\cForm \textimplies
  %   \lnot\cForm_\bEv)
  % \end{math}
  % then
  \begin{math}
    (\forall\dVal)
  \end{math}
  $\aTr{\bEvs}{\aForm}$ implies 
  \begin{math}
    (\not\exists\bEv\in\bEvs.\; \cForm_\bEv)
    \limplies (\cExp{=}\dVal)
    \limplies \PBR{
      \lnot\QwREF{\dVal}
      \land \aForm [\aExp/\REF{\dVal}]\Dmode{\amode}
    }
  \end{math},
  \\ where 
  $\Dmode{\mRLX}=[\TRUE/\DxREF{\dVal}]$ and otherwise
  $\Dmode{\amode}=[\FALSE/\D]$. % for $\amode\neq\mRLX$.
\end{enumerate}
% \item if $\amode=\mRLX$ then
%   $\labelingForm(\aEv)$ implies
%   \begin{math}
%     \cForm_\aEv
%     \land \cExp{=}\cVal_\aEv
%     \land \aExp{=}\aVal_\aEv
%     \land \RW
%     \land \QxREF{\cVal_\aEv},
%   \end{math}
% \item if $\amode\neq\mRLX$ then
%   $\labelingForm(\aEv)$ implies
%   \begin{math}
%     \cForm_\aEv
%     \land \cExp{=}\cVal_\aEv
%     \land \aExp{=}\aVal_\aEv
%     \land \RW
%     \land \Q{},
%   \end{math}
% \item if
%   $\bEv\in\bEvs$
%   and
%   $\amode=\mRLX$ then
%   \begin{math}
%     (\forall\dVal)
%   \end{math}
%   $\aTr{\bEvs}{\aForm}$ implies 
%   \begin{math}
%     \cForm_\bEv
%     \limplies (\cExp{=}\dVal)
%     \limplies \PBRbig{
%     (\QwREF{\dVal} \limplies \aExp{=}\aVal_\bEv)
%     \land \aForm[\aExp/\REF{\dVal}][\TRUE/\DxREF{\dVal}]
%   }
%   \end{math}
% \item if
%   $\bEv\in\bEvs$
%   and
%   $\amode\neq\mRLX$ then
%   \begin{math}
%     (\forall\dVal)
%   \end{math}
%   $\aTr{\bEvs}{\aForm}$ implies 
%   \begin{math}
%     \cForm_\bEv
%     \limplies (\cExp{=}\dVal)
%     \limplies \PBRbig{
%     (\QwREF{\dVal} \limplies \aExp{=}\aVal_\bEv)
%     \land \aForm[\aExp/\REF{\dVal}][\FALSE/\D]
%   }
%   \end{math}
% \item if 
%   \begin{math}
%     (\forall\bEv\in\bEvs)(\cForm \textimplies
%     \lnot\cForm_\bEv)
%   \end{math}
%   and $\amode=\mRLX$ 
%   then
%   \begin{math}
%     (\forall\dVal)
%   \end{math}
%   $\aTr{\bEvs}{\aForm}$ implies 
%   \begin{math}
%     \cForm
%     \limplies (\cExp{=}\dVal)
%     \limplies \PBRbig{
%     \lnot\QwREF{\dVal}
%     \land \aForm[\aExp/\REF{\dVal}][\TRUE/\DxREF{\dVal}]
%   }
%   \end{math}
% \item if 
%   \begin{math}
%     (\forall\bEv\in\bEvs)
%     (\cForm \textimplies \lnot\cForm_\bEv)
%   \end{math}
%   and $\amode\neq\mRLX$ 
%   then
%   \begin{math}
%     (\forall\dVal)
%   \end{math}
%   $\aTr{\bEvs}{\aForm}$ implies 
%   \begin{math}
%     \cForm
%     \limplies (\cExp{=}\dVal)
%     \limplies \PBRbig{
%     \lnot\QwREF{\dVal}
%     \land \aForm[\aExp/\REF{\dVal}][\FALSE/\D]
%   }
%   \end{math}

\noindent
If $\aPS \in \sLOAD[\amode]{\aReg}{\cExp}$ then
$(\exists\cVal:\aEvs\fun\Val)$
$(\exists\aVal:\aEvs\fun\Val)$
$(\exists\cForm:\aEvs\fun\Formulae)$
% $(\forall\uReg{\aEv}\in\uRegs{\aEvs})$
\begin{enumerate}
\item if $\cForm_\bEv\land\cForm_\aEv$ is satisfiable then $\bEv=\aEv$,
\item $\labelingAct(\aEv) = \DRREFP{\cVal_\aEv}{\aVal_\aEv}$,
\item $\labelingForm(\aEv)$ implies
  \begin{math}
    \cForm_\aEv
    \land \cExp{=}\cVal_\aEv
    \land \RO
    \land \Qmode{\amode}
  \end{math},
  where    
  $\Qmode{\mSC}=\Q{\mSC}$ and otherwise $\Qmode{\amode}=\QwREF{\cVal_\aEv}$, % for $\amode\neq\mRLX$,
\item
  \begin{math}
    (\forall\dVal)
  \end{math}
  if $\bEv\in\bEvs$ then
  $\aTr{\bEvs}{\aForm}$ implies
  \begin{math}
    \cForm_\bEv
    \limplies (\cExp{=}\dVal)
    \limplies (\aVal{=}\uReg{\bEv})
    \limplies \aForm[\uReg{\bEv}/\aReg][\uReg{\bEv}/\REF{\dVal}]
  \end{math},
  \makebox[4.4cm]{}
\item 
  \begin{math}
    (\forall\dVal)
  \end{math}
  if $\bEv\notin\bEvs$ then
  $\aTr{\bEvs}{\aForm}$ implies
  \begin{math}
    \cForm_\bEv
    \limplies (\cExp{=}\dVal)
    \limplies \PBRbig{        
      \Dmode{\amode}
      \land \lnot\QxREF{\dVal}
      \land
      (\RW
      \limplies (\aVal{=}\uReg{\bEv}\lor\aLoc{=}\uReg{\bEv}) 
      \limplies \aForm[\uReg{\bEv}/\aReg][\uReg{\bEv}/\REF{\dVal}]
      )
    }      
  \end{math},
\item % if 
  % \begin{math}
  %   (\forall\bEv\in\bEvs)(\cForm \textimplies
  %   \lnot\cForm_\bEv)
  % \end{math}
  % then
  \begin{math}
    (\forall\dVal)
    (\forall\bReg)
  \end{math}
  $\aTr{\bEvs}{\aForm}$ implies 
  \begin{math}
    (\not\exists\bEv\in\bEvs.\; \cForm_\bEv)
    \limplies (\cExp{=}\dVal)
    \limplies \PBR{        
      \Dmode{\amode}
      \land \lnot\QxREF{\dVal}
      \land
      \limplies \aForm[\bReg/\aReg][\bReg/\REF{\dVal}]
    }      
  \end{math},
  \\ where $\Dmode{\mRLX}=\TRUE$ and otherwise $\Dmode{\amode}=\Dx{\dVal}$.
  Recall that $\uRegs{\bEvs}=\{\uReg{\bEv}\mid\bEv\in\bEvs\}$.
\end{enumerate}  
% \item if $\amode=\mRLX$ and $\bEv\notin\bEvs$ then
%   \begin{math}
%     (\forall\dVal)
%   \end{math}
%   $\aTr{\bEvs}{\aForm}$ implies
%   \begin{math}
%     \cForm_\bEv
%     \limplies (\cExp{=}\dVal)
%     \limplies \PBRbig{
%     (
%     \RW
%     \limplies (\aVal{=}\uReg{\bEv}\lor\aLoc{=}\uReg{\bEv}) 
%     \limplies \aForm[\uReg{\bEv}/\aReg][\uReg{\bEv}/\REF{\dVal}]
%     )
%     \land \lnot\QxREF{\dVal}
%   }
%     \phantom{\land\; \Dx{\dVal}}
%   \end{math}
% \item if $\amode\neq\mRLX$ and $\bEv\notin\bEvs$ then
%   \begin{math}
%     (\forall\dVal)
%   \end{math}
%   $\aTr{\bEvs}{\aForm}$ implies
%   \begin{math}
%     \cForm_\bEv
%     \limplies (\cExp{=}\dVal)
%     \limplies \PBRbig{
%     (
%     \RW
%     \limplies (\aVal{=}\uReg{\bEv}\lor\aLoc{=}\uReg{\bEv}) 
%     \limplies \aForm[\uReg{\bEv}/\aReg][\uReg{\bEv}/\REF{\dVal}]
%     )
%     \land \lnot\QxREF{\dVal}
%     \land \Dx{\dVal}
%   }
%   \end{math}

      % \noindent
If $\aPS \in \sSTORE[\amode]{\cExp}{\aExp}$ then
$(\exists\cVal:\aEvs\fun\Val)$
$(\exists\aVal:\aEvs\fun\Val)$
$(\exists\bForm:\aEvs\fun\Formulae)$
\begin{enumerate}
\item if $\bForm_\bEv\land\bForm_\aEv$ is satisfiable then $\bEv=\aEv$,
\item $\labelingAct(\aEv) = \DWREFP{\cVal_\aEv}{\aVal_\aEv}$,
\item 
  $\labelingForm(\aEv)$ implies
  \begin{math}
    \bForm_\aEv
    \land \cExp{=}\cVal_\aEv
    \land \aExp{=}\aVal_\aEv
    \land \RW
    \land \QS{}{\amode}
  \end{math},
\item
  \begin{math}
    (\forall\dVal)
  \end{math}
  if
  $\bEv\in\bEvs$
  then
  $\aTr[\bEvs](\aForm)$ implies 
  \begin{math}
    \bForm_\bEv
    \limplies (\cExp{=}\dVal)
    \limplies \PBRbig{
      \aExp{=}\aVal_\bEv
      \land \DS{\REF{\dVal}}{\amode}{\aForm[\aExp/\REF{\dVal}]}
    }
  \end{math},
\item 
  \begin{math}
    (\forall\dVal)
  \end{math}
  $\aTr[\bEvs](\aForm)$ implies 
  \begin{math}
    (\not\exists\bEv\in\bEvs.\; \bForm_\bEv)
    \limplies (\cExp{=}\dVal)
    \limplies \PBR{
      \lnot\Q{\mRA}
      \land \DS{\REF{\dVal}}{\amode}{\aForm[\aExp/\REF{\dVal}]}
    }.
  \end{math}
\end{enumerate}

\noindent
If $\aPS \in \sLOAD[\amode]{\aReg}{\cExp}$ then
$(\exists\cVal:\aEvs\fun\Val)$
$(\exists\aVal:\aEvs\fun\Val)$
$(\exists\bForm:\aEvs\fun\Formulae)$
\begin{enumerate}
\item if $\bForm_\bEv\land\bForm_\aEv$ is satisfiable then $\bEv=\aEv$,
\item $\labelingAct(\aEv) = \DRREFP{\cVal_\aEv}{\aVal_\aEv}$,
\item $\labelingForm(\aEv)$ implies
  \begin{math}
    \bForm_\aEv
    \land \cExp{=}\cVal_\aEv
    \land \RO
    \land \QL{}{\amode}
  \end{math},
\item
  \begin{math}
    (\forall\dVal)
  \end{math}
  if $\bEv\in\bEvs$ then
  $\aTr[\bEvs](\aForm)$ implies
  \begin{math}
    \bForm_\bEv
    \limplies (\cExp{=}\dVal)
    \limplies (\aVal{=}\uReg{\bEv})
    \limplies \aForm[\uReg{\bEv}/\aReg][\uReg{\bEv}/\REF{\dVal}]
  \end{math},
  \makebox[4.8cm]{}
\item 
  \begin{math}
    (\forall\dVal)
  \end{math}
  if $\bEv\notin\bEvs$ then
  $\aTr[\bEvs](\aForm)$ implies
  \begin{math}
    \bForm_\bEv
    \limplies (\cExp{=}\dVal)
    \limplies \PBRbig{        
      \DL{\REF{\dVal}}{\amode}
      \land \lnot\Q{\mRA}
      \land
      (\RW
      \limplies (\aVal{=}\uReg{\bEv}\lor\aLoc{=}\uReg{\bEv}) 
      \limplies \aForm[\uReg{\bEv}/\aReg][\uReg{\bEv}/\REF{\dVal}]
      )
    }      
  \end{math},
\item 
  \begin{math}
    (\forall\dVal)
    (\forall\bReg)
  \end{math}
  $\aTr[\bEvs](\aForm)$ implies 
  \begin{math}
    (\not\exists\bEv\in\bEvs.\; \bForm_\bEv)
    \limplies (\cExp{=}\dVal)
    \limplies \PBR{        
      \DL{\REF{\dVal}}{\amode}
      \land \lnot\Q{\mRA}
      \land
      \limplies \aForm[\bReg/\aReg][\bReg/\REF{\dVal}]
    }.
  \end{math}
\end{enumerate}  

    \end{minipage}
  \end{center}
  \caption{Full Semantics with Address Calculation
    (See \refdef{def:QS} for $\QS{\aLoc}{\amode}$, $\QL{\aLoc}{\amode}$
    and \refdef{def:DS} for $\DL{\aLoc}{\amode}$, $\DS{\aLoc}{\amode}$)
  }
  \label{fig:full}
\end{figure*}    

\section{Additional Features}

\subsection{Address Calculation (\xADDR)}
\reffig{fig:full} describes the full semantics with address calculation.
% \ref{S6full} is necessary because \ref{S5full} mentions $\cVal_\aEv$, and
% there is no such value when the set of events is empty.
\begin{definition}[\xADDR]
  \label{def:pomsets-addr}
  Update \refdef{def:pomsets-trans} to existentially quantify over $\cVal$
  in $\sSTORE{}{}$ and $\sLOAD{}{}$:
  \begin{enumerate}
  \item[\ref{S2})] $\labelingAct(\aEv) = \DW{\REF\cVal}{\aVal}$,
  \item[\ref{L2})] $\labelingAct(\aEv) = \DR{\REF{\cVal}}{\aVal}$.
  \end{enumerate}

  \begin{enumerate}
  \item[\ref{S3})] $\labelingForm(\aEv) \rimplies \cExp{=}\cVal \land \aExp{=}\aVal$,
  \item[\ref{L3})] $\labelingForm(\aEv) \rimplies \cExp{=}\cVal$.
  \end{enumerate}

  \begin{enumerate}
    % \item[\ref{S4})] $(\forall\dVal)$ $\aTr{\bEvs}{\bForm} \rimplies \cExp{=}\dVal \limplies \bForm\noSUB{[\aExp/\REF{\dVal}]}$,
    % \item[\ref{S5})] $(\forall\dVal)$ $\aTr{\cEvs}{\bForm} \rimplies \cExp{=}\dVal \limplies \bForm\noSUB{[\aExp/\REF{\dVal}]}$,
    % \item[\ref{L4})] $\phantom{(\forall\dVal)}$ $\aTr{\bEvs}{\bForm} \rimplies (\cExp{=}\cVal\limplies\aVal{=}\aReg)\limplies\bForm$, 
    % \item[\ref{L4})] $(\forall\dVal)$ $\aTr{\bEvs}{\bForm} \rimplies \cVal{=}\dVal \limplies (\cExp{=}\dVal\limplies\aVal{=}\aReg)\limplies\bForm$, 
  \item[\ref{L4})] $\aTr{\bEvs}{\bForm} \rimplies (\cExp{=}\cVal\limplies\aVal{=}\aReg)\limplies\bForm$, 
    % \item[\ref{L5})] $(\forall\dVal)$ $\aTr{\cEvs}{\bForm} \rimplies \cExp{=}\dVal \limplies (\aVal{=}\aReg\lor \REF{\dVal}{=}\aReg)\limplies\bForm$,
    % \item[\ref{L6})] $(\forall\dVal)$ $\aTr{\dEvs}{\bForm} \rimplies \cExp{=}\dVal \limplies \bForm$.
    % \item[\ref{L5})] $(\forall\dVal)$ $\aTr{\cEvs}{\bForm} \rimplies \cExp{=}\dVal \limplies \bForm$.
  \item[\ref{L5})] $\aTr{\cEvs}{\bForm} \rimplies \bForm$.
  \end{enumerate}  
\end{definition}

\begin{example}
  \refdef{def:pomsets-addr} is naive with respect to merging events.
  Consider the following example from \jjr{}:
  \begin{align*}
    \begin{gathered}
      \PW{\REF{r}}{0}\SEMI \PW{\REF{0}}{\BANG r}
      \\
      \hbox{\begin{tikzinline}[node distance=.8em]
          \eventl{c}{a}{r\EQ1\mathbin{\mid}\DW{\REF{1}}{0}}{}
          \eventl{d}{b}{r\EQ1\mathbin{\mid}\DW{\REF{0}}{0}}{right=of a}
        \end{tikzinline}}
    \end{gathered}
    &&
    \begin{gathered}
      \PW{\REF{r}}{0}\SEMI \PW{\REF{0}}{\BANG r}
      \\
      \hbox{\begin{tikzinline}[node distance=.8em]
          \eventl{d}{a}{r\EQ0\mathbin{\mid}\DW{\REF{0}}{0}}{}
          \eventl{e}{b}{r\EQ0\mathbin{\mid}\DW{\REF{0}}{1}}{right=of a}
          \wk{a}{b}
        \end{tikzinline}}
    \end{gathered}
  \end{align*}
  Merging, we have:
  % Thus, the disjunction closure also includes both of the following: % By using \!$\PAR$\!, it also includes:
  \begin{align*}
    \begin{gathered}
      \IF{\aExp}\THEN
      \PW{\REF{r}}{0}\SEMI \PW{\REF{0}}{\BANG r}
      \ELSE
      \PW{\REF{r}}{0}\SEMI \PW{\REF{0}}{\BANG r}
      \FI
      \\
      \hbox{\begin{tikzinline}[node distance=1em]
          \eventl{c}{a}{r\EQ1\mathbin{\mid}\DW{\REF{1}}{0}}{}
          \eventl{d}{b}{r\EQ0\lor r\EQ1\mathbin{\mid}\DW{\REF{0}}{0}}{right=of a}
          \eventl{e}{c}{r\EQ0\mathbin{\mid}\DW{\REF{0}}{1}}{right=of b}
          \wk{b}{c}
        \end{tikzinline}}
    \end{gathered}
  \end{align*}
  % \begin{align*}
  %   \begin{gathered}
  %     \IF{\aExp}\THEN
  %     \PW{\REF{r}}{0}\SEMI \PW{\REF{0}}{\BANG r}
  %     \ELSE
  %     \PW{\REF{r}}{0}\SEMI \PW{\REF{0}}{\BANG r}
  %     \FI
  %     \\
  %     \hbox{\begin{tikzinline}[node distance=1em]
  %       \eventl{c}{a}{r\EQ1\mathbin{\mid}\DW{\REF{1}}{0}}{}
  %       \eventl{d}{b}{r\EQ0\lor r\EQ1\mathbin{\mid}\DW{\REF{0}}{0}}{right=of a}
  %     \end{tikzinline}}
  %   \end{gathered}
  % \end{align*}
  The precondition of $\DWREF{0}{0}$ is a tautology; however, this is not
  possible for $(\PW{\REF{r}}{0}\SEMI \PW{\REF{0}}{\BANG r})$ alone, using \refdef{def:pomsets-addr}.
  % 
  The full semantics, given in \reffig{fig:full}, enables this execution using if-closure.  The
  individual commands have the pomsets:
  \begin{align*}
    \begin{gathered}
      \PW{\REF{r}}{0}
      \\
      \hbox{\begin{tikzinline}[node distance=.5em]
          \eventl{c}{a}{r\EQ1\mathbin{\mid}\DW{\REF{1}}{0}}{}
          \eventl{d}{b}{r\EQ0\mathbin{\mid}\DW{\REF{0}}{0}}{below=of a}
        \end{tikzinline}}
    \end{gathered}
    &&
    \begin{gathered}
      \PW{\REF{0}}{\BANG r}
      \\
      \hbox{\begin{tikzinline}[node distance=.5em]
          \eventl{d}{b}{r\EQ1\mathbin{\mid}\DW{\REF{0}}{0}}{}
          \eventl{e}{c}{r\EQ0\mathbin{\mid}\DW{\REF{0}}{1}}{below=of b}
        \end{tikzinline}}
    \end{gathered}
  \end{align*}
  % These pomsets contain inconsistent preconditions.  This is disallowed in
  % \jjr{}, but allowed here.
  Sequencing and merging, we have: 
  \begin{align*}
    \begin{gathered}
      \PW{\REF{r}}{0}
      \SEMI
      \PW{\REF{0}}{\BANG r}
      \\
      \hbox{\begin{tikzinline}[node distance=1em]
          \eventl{c}{a}{r\EQ1\mathbin{\mid}\DW{\REF{1}}{0}}{}
          \eventl{d}{b}{r\EQ0\lor r\EQ1\mathbin{\mid}\DW{\REF{0}}{0}}{right=of a}
          \eventl{e}{c}{r\EQ0\mathbin{\mid}\DW{\REF{0}}{1}}{right=of b}
          \wk{b}{c}
        \end{tikzinline}}
    \end{gathered}
  \end{align*}
  The precondition of $\DWP{\REF{0}}{0}$ is a tautology, as required.
\end{example}

\begin{example}
  \label{ex:xADDRxRRD}
  The combination of read-read independency and address calculation
  (\xADDR/\xRRD) is somewhat delicate.  Combing \refdef{def:pomsets-rr} and
  \refdef{def:pomsets-addr} we have:
  \begin{enumerate}
    % \item[\ref{L4})]
    %   $\aTr{\bEvs}{\bForm} \rimplies \aVal{=}\aReg\limplies\bForm$, 
    % \item[\ref{L5})]
    %   $\aTr{\cEvs}{\bForm} \rimplies (\aVal{=}\aReg\lor\RW)\limplies\bForm$,
    
  \item[\ref{L4})]
    \begin{math}
      \aTr{\bEvs}{\bForm} \rimplies
      (\cExp{=}\cVal\limplies\aVal{=}\aReg)\limplies\bForm,
    \end{math}
  \item[\ref{L5})]
    \begin{math}
      \aTr{\cEvs}{\bForm} \rimplies
      ((\cExp{=}\cVal\limplies\aVal{=}\aReg)\lor\RW)\limplies\bForm.
    \end{math}
  \end{enumerate}
  If we replace the use of $(\cExp{=}\cVal\limplies\aVal{=}\aReg)$ by
  $(\aVal{=}\aReg)$, thin air reads are possible.  The subsection of
  \textsection\ref{sec:diff} on \ref{xCausal} discusses this example using
  the semantics of \jjr{}.
  
  Consider the following program, from \jjr{\textsection5}, where initially $x=0$, $y=0$, $\REF{0}=0$,
  $\REF{1}=2$, and $\REF{2}=1$.  It should only be possible to read $0$,
  disallowing the attempted execution below:
  \begin{gather*}
    \begin{gathered}
      \PR{y}{r}\SEMI \PR{\REF{r}}{s}\SEMI \PW{x}{s}
      \PAR
      \PR{x}{r}\SEMI \PR{\REF{r}}{s}\SEMI \PW{y}{s}
      \\
      \hbox{\begin{tikzinline}[node distance=1.5em]
          \event{a1}{\DR{y}{2}}{}
          \event{a2}{\DR{\REF{2}}{1}}{right=of a1}
          \event{a3}{\DW{x}{1}}{right=of a2}
          \po{a2}{a3}
          \po[out=10,in=170]{a1}{a3}
          \event{b1}{\DR{x}{1}}{right=3em of a3}
          \event{b2}{\DR{\REF{1}}{2}}{right=of b1}
          \event{b3}{\DW{y}{2}}{right=of b2}
          \po{b2}{b3}
          \po[out=10,in=170]{b1}{b3}
          \rf[out=-170,in=-10]{b3}{a1}
          \rf{a3}{b1}
        \end{tikzinline}}
    \end{gathered}
  \end{gather*}
  Looking at the left thread:
  \begin{align*}
    \begin{gathered}[t]
      \PR{y}{r}
      \\
      \hbox{\begin{tikzinline}[node distance=.5em and 1.5em]
          \event{a}{\DR{y}{2}}{}
          \xform{xi}{(2{=}r\lor\RW)\limplies\bForm}{above=of a}
          \xform{xd}{2{=}r\limplies\bForm}{below=of a}
          \xo{a}{xd}
        \end{tikzinline}}
    \end{gathered}
    &&
    \begin{gathered}[t]
      \PR{\REF{r}}{s}
      \\
      \hbox{\begin{tikzinline}[node distance=.5em and 1.5em]
          \event{b}{r\EQ2\mid\DR{\REF{2}}{1}}{}
          \xform{xd}{(r\EQ2\limplies 1\EQ s) \limplies\bForm}{below=of b}
          \xform{xi}{((r\EQ2\limplies 1\EQ s)\lor\RW) \limplies\bForm}{above=of b}
          \xo{b}{xd}
        \end{tikzinline}}
    \end{gathered}
    &&
    \begin{gathered}[t]
      \PW{x}{s}
      \\
      \hbox{\begin{tikzinline}[node distance=.5em and 1.5em]
          \event{b}{s\EQ1\mid\DW{x}{1}}{}
          \xform{xd}{\bForm}{below=of b}
          \xform{xi}{\bForm}{above=of b}
          \xo[out=-30]{b}{xd}
        \end{tikzinline}}
    \end{gathered}
  \end{align*}
  Composing, we have:
  \begin{gather*}
    \begin{gathered}
      \PR{y}{r}\SEMI \PR{\REF{r}}{s}\SEMI \PW{x}{s}
      \\
      \hbox{\begin{tikzinline}[node distance=.5em and 1.5em]
          \event{a1}{\DR{y}{2}}{}
          \event{a2}{(2{=}r\lor\RW)\limplies r\EQ2\mid\DR{\REF{2}}{1}}{right=of a1}
          \event{a3}{(2{=}r\lor\RW)\limplies (r\EQ2\limplies 1\EQ s)
            \limplies s\EQ1\mid\DW{x}{1}}{below right=.5em and -8em of a2}
          \po{a2}{a3}
        \end{tikzinline}}
    \end{gathered}
  \end{gather*}  
  Substituting for $\RW$:
  \begin{gather*}
    \begin{gathered}
      % \PR{y}{r}\SEMI \PR{\REF{r}}{s}\SEMI \PW{x}{s}
      % \\
      \hbox{\begin{tikzinline}[node distance=.5em and 1.5em]
          \event{a1}{\DR{y}{2}}{}
          \event{a2}{(2{=}r\lor\FALSE)\limplies r\EQ2\mid\DR{\REF{2}}{1}}{right=of a1}
          \event{a3}{(2{=}r\lor\TRUE)\limplies (r\EQ2\limplies 1\EQ s)
            \limplies s\EQ1\mid\DW{x}{1}}{below right=.5em and -8em of a2}
          \po{a2}{a3}
        \end{tikzinline}}
    \end{gathered}
  \end{gather*}
  Which is:
  \begin{gather*}
    \begin{gathered}
      % \PR{y}{r}\SEMI \PR{\REF{r}}{s}\SEMI \PW{x}{s}
      % \\
      \hbox{\begin{tikzinline}[node distance=.5em and 1.5em]
          \event{a1}{\DR{y}{2}}{}
          \event{a2}{2{=}r\limplies r\EQ2\mid\DR{\REF{2}}{1}}{right=of a1}
          \event{a3}{(r\EQ2\limplies 1\EQ s)
            \limplies s\EQ1\mid\DW{x}{1}}{below right=.5em and -8em of a2}
          \po{a2}{a3}
        \end{tikzinline}}
    \end{gathered}
  \end{gather*}
  The precondition of $\DRP{\REF{2}}{1}$ is a tautology, but the precondition
  of $\DWP{x}{1}$ is not.  This forces a dependency:
  \begin{gather*}
    \begin{gathered}
      \PR{y}{r}\SEMI \PR{\REF{r}}{s}\SEMI \PW{x}{s}
      \\
      \hbox{\begin{tikzinline}[node distance=.5em and 1.5em]
          \event{a1}{\DR{y}{2}}{}
          \event{a2}{2{=}r\limplies r\EQ2\mid\DR{\REF{2}}{1}}{right=of a1}
          \event{a3}{2{=}r\limplies (r\EQ2\limplies 1\EQ s)
            \limplies s\EQ1\mid\DW{x}{1}}{below right=.5em and -5em of a2}
          \po[out=-20,in=180]{a1}{a3}
          \po{a2}{a3}
        \end{tikzinline}}
    \end{gathered}
  \end{gather*}
  All the preconditions are now tautologies.
\end{example}

\subsection{Access Elimination}
For reads, get rid of $\FALSE/\Q{}$ in \ref{L6full}.

For writes, change the label rules of sequential composition to:
\begin{enumerate}
\item %\label{par-lambda1}
  if $\aEv\in\aEvs_1\setminus\aEvs_2$ then $\labeling(\aEv) = \labeling_1(\aEv)$, 
\item %\label{par-lambda2}
  if $\aEv\in\aEvs_2\setminus\aEvs_1$ then $\labeling(\aEv) = \labeling_2(\aEv)$,
\item %\label{par-lambda2}
  if $\aEv\in\aEvs_1\cap\aEvs_2$ then $\labeling(\aEv) \in \fmerge{\labeling_1(\aEv)}{\labeling_2(\aEv)}$.
\end{enumerate}

\begin{definition}
  % Fences use the three point order:
  % \begin{math}
  %   \fREL \ltmode \mSC
  % \end{math}
  % and 
  % \begin{math}
  %   \fACQ \ltmode \mSC.
  % \end{math}
  \noindent    
  \begin{align*}
    \fmerge{\DR[\amode]{\aLoc}{\aVal}}{\DR[\bmode]{\aLoc}{\aVal}} &= \{ \DR[\amode\lubmode\bmode]{\aLoc}{\aVal} \}
    \\
    \fmerge{\DW[\amode]{\aLoc}{\aVal}}{\DW[\bmode]{\aLoc}{\bVal}} &= \{ \DW[\amode\lubmode\bmode]{\aLoc}{\bVal} \}
    \\
    \fmerge{\DF{\amode}}{\DF{\bmode}} &= \{ \DF{\amode\lubmode\bmode} \}
    \\
    \fmerge{\aAct}{\bAct} &= \emptyset, \textotherwise
  \end{align*}
\end{definition}  

\subsection{Merging Different labels}

Reordering and Merging:
\cite[\textsection7.1]{Kang19}
\cite[\textsection E]{DBLP:conf/cgo/ChakrabortyV17}

Examples of Unsafe Reorderings
\cite[\textsection D]{DBLP:conf/cgo/ChakrabortyV17}
See the slides for this paper...


We combine access and fence modes into a single order:
\begin{align*}
  \begin{tikzcenter}
    \node (rlx) at (0, 0) {$\mathstrut\mRLX$};
    \node (ra)  at (1, 0) {$\mathstrut\mRA$};
    \node (sc)  at (2, 0) {$\mathstrut\mSC$};
    \draw[->](rlx)to(ra);
    \draw[->](ra)to(sc);
  \end{tikzcenter}
  &&
  \begin{tikzcenter}
    \node (fsc) at (3, 0) {$\mathstrut\fSC$};
    \node (rel) at (2, -0.3) {$\mathstrut\fREL$};
    \node (acq) at (2,  0.3) {$\mathstrut\fACQ$};
    \draw[->](rel)to(fsc);
    \draw[->](acq)to(fsc);
  \end{tikzcenter}
\end{align*}

Note that for associativity, you have to take the join of modes.
\begin{definition}
  \label{def:compat}
  Define $\fmerge{}{}:\Act\times\Act\fun2^{\Act}$ as follows.  If
  $\aAct_0\in\fmerge{\aAct_1}{\aAct_2}$, then $\aAct_1$ and $\aAct_2$ can
  coalesce, resulting in $\aAct_0$.  This is useful for replacing
  $(\PW{x}{1}\SEMI \PW{x}{2})$ by $(\PW{x}{2})$.
  \begin{align*}
    \fmerge{\DR[\amode]{\aLoc}{\aVal}}{\DR[\bmode]{\aLoc}{\aVal}}
    &= \{ \DR[\amode\lubmode\bmode]{\aLoc}{\aVal} \}
    \\
    \fmerge{\DW[\amode]{\aLoc}{\aVal}}{\DW[\bmode]{\aLoc}{\bVal}}
    &= \{ \DW[\amode\lubmode\bmode]{\aLoc}{\bVal} \}
    \\
    \fmerge{\DR[\mRLX]{\aLoc}{\aVal}}{\DW[\amode]{\aLoc}{\aVal}}
    &= \{ \DW[\amode]{\aLoc}{\aVal} \}
    \\
    \fmerge{\DR[\amode\neq\mRLX]{\aLoc}{\aVal}}{\DW[\bmode]{\aLoc}{\aVal}}
    &= \{ \DW[\mSC]{\aLoc}{\aVal} \}
    \\
    \fmerge{\DF{\amode}}{\DF{\bmode}} &= \{ \DF{\amode\lubmode\bmode} \}
    \\
    \fmerge{\aAct}{\bAct} &= \emptyset, \textotherwise
  \end{align*}
\end{definition}  

\begin{enumerate}
\item \label{seq-lambda1}
  if $\aEv\in\aEvs_1\setminus\aEvs_2$ then $\labeling(\aEv)=\labeling_1(\aEv)$,
\item \label{seq-lambda2}
  if $\aEv\in\aEvs_2\setminus\aEvs_1$ then $\labeling(\aEv)=\labeling_2(\aEv)$,
\item \label{seq-lambda12}
  if $\aEv\in\aEvs_1\cap\aEvs_2$ then $\labeling(\aEv)\in
  \fmerge{\labeling_1(\aEv)}{\labeling_2(\aEv)}$, the first has no rf,
\end{enumerate}
  
\subsection{Read-Modify-Write Operations (\xRMW)}

Extend the syntax
\begin{align*}
  \aCmd
  \BNFDEF& \cdots 
  \BNFSEP \PCAS[\amode_1][\amode_2]{\REF{\cExp}}[\ascope]{\aReg}{\aExp}{\bExp}
  \BNFSEP \PFADD[\amode_1][\amode_2]{\REF{\cExp}}[\ascope]{\aReg}{\aExp}
  \BNFSEP \PEXCHG[\amode_1][\amode_2]{\REF{\cExp}}[\ascope]{\aReg}{\aExp}
\end{align*}
From the data model, we require an additional binary relation over
$\Act\times\Act$: $\roverlapsdef$.  For the actions in this paper, we say
$\aAct \roverlapsdef \bAct$ if they access the same location.


\RMW{} operations are formalized by adding a relation
${\xrmw}\subseteq\aEvs\times\aEvs$ that relates the read of a successful
\RMW{} to the succeeding write.
% Let two actions \emph{overlap} if they access the same location.
Extend the definition of a pomset as follows. % where two actions \emph{overlap} if they access the same location:
% \begin{enumerate}
% \item
%   ${\rrmw} \subseteq {\le}$ is a relation capturing
%   \emph{read-modify-write atomicity}, such that for any $\cEv$, $\bEv$, $\aEv\in\aEvs$,
%   where $\labeling(\cEv)$ and
%   $\labeling(\bEv)$ access the same location:
%   \begin{itemize}
%   \item if $\bEv \xrmw \aEv$ and $\cEv\le \aEv$ then $\cEv\le \bEv$,
%   \item if $\bEv \xrmw \aEv$ and $\bEv\le \cEv$ then $\aEv\le \cEv$.
%   \end{itemize}
% \end{enumerate}

\begin{enumerate}[,label=(\textsc{m}\arabic*),ref=\textsc{m}\arabic*]
  \setcounter{enumi}{-1}
\item \label{pom-rmw}
  ${\rrmw} : \Event\fun\Event$ is a partial function capturing
  read-modify-write \emph{atomicity}, such that
  \begin{enumerate}
    \item \label{pom-rmw-block}
      if $\bEv\xrmw\aEv$ then $\labeling(\aEv) \rblocks \labeling(\bEv)$,
  \item \label{pom-rmw-le}
    if $\bEv\xrmw\aEv$ then $\bEv \le \aEv$,    
  \item \label{pom-rmw-atomic}
    if $\labeling(\cEv) \roverlaps \labeling(\bEv)$ then
    \begin{enumerate}        
    \item \label{pom-rmw-atomic1}
      if $\bEv \xrmw \aEv$ then
      $\cEv\ledep \aEv$ implies $\cEv\le \bEv$,
    \item \label{pom-rmw-atomic2}
      if $\bEv \xrmw \aEv$ then
      $\bEv\ledep \cEv$ implies $\aEv\le \cEv$.
    \end{enumerate}
  \end{enumerate}
\end{enumerate}

Extend the definition of par, if, seq to include:
\begin{enumerate}
  \setcounter{enumi}{-1}
\item ${\rrmw}=\PBR{{\rrmw_1}\cup{\rrmw_2}}$,
\end{enumerate}

\begin{example}
  This definition ensures atomicity, disallowing executions such as
  \cite[Ex.~3.2]{DBLP:journals/pacmpl/PodkopaevLV19}:
  \begin{gather*}
    % \taglabel{RMW1}
    \begin{gathered}
      \PW{x}{0}\SEMI \PFADD[\mRLX][\mRLX]{x}{s}{1}
      \PAR
      \PW{x}{2}\SEMI \PR{x}{s}
      \\
      \hbox{\begin{tikzinline}[node distance=1.5em]
          \event{a2}{\DR{x}{0}}{}
          \event{a1}{\DW{x}{0}}{left=of a2}
          \rf{a1}{a2}
          \event{a3}{\DW{x}{2}}{right=of a2}
          \wk{a2}{a3}
          \event{b2}{\DW{x}{1}}{right=of a3}
          \event{b3}{\DR{x}{1}}{right=of b2}
          \rmw[out=-15,in=-165]{a2}[below]{b2}
          \wk{a3}{b2}
          \rf{b2}{b3}
        \end{tikzinline}}
    \end{gathered}
  \end{gather*}
  By \ref{pom-rmw-atomic1}, since $\DWP{x}{2}\xwk\DWP{x}{1}$, it must be that
  $\DWP{x}{2}\xwk\DRP{x}{0}$, creating a cycle.
\end{example}

\begin{example}
  \label{ex:rmw-33}
  Two successful \RMW{}s cannot see the same write:
  \begin{gather*}
    \begin{gathered}
      \PW{x}{0}\SEMI (\FADD^{\mRLX,\mRLX}(\aLoc,1)\PAR \FADD^{\mRLX,\mRLX}(\aLoc,1))
      \\
      \hbox{\begin{tikzinline}[node distance=1.5em]
          \event{i}{\DW{x}{0}}{}
          \event{a1}{a{:}\DR{x}{0}}{right=3em of i}
          \event{a2}{b{:}\DW{x}{1}}{right=of a1}
          \event{b1}{c{:}\DR{x}{0}}{right=3em of a2}
          \event{b2}{d{:}\DW{x}{1}}{right=of b1}
          \rmw{a1}{a2}
          \rmw{b1}{b2}
          \rf{i}{a1}
          \rf[out=15,in=165]{i}{b1}
          \wk[out=-15,in=-165]{a1}{b2}
          % \wk{a1}{b2}
          \wk{b1}{a2}
        \end{tikzinline}}
    \end{gathered}
  \end{gather*}
  The order from read-to-write is required by fulfillment.  
  Apply \ref{A1} to $a\xwk d$, we have that $a\xwk c$.  Subsequently
  applying \ref{A2}, we have $b \xwk c$, creating a cycle.
\end{example}

\begin{example}
  By using two actions rather than one, the definition allows examples such as the
  following, which is allowed by \armeight{} 
  \cite[Ex.~3.10]{DBLP:journals/pacmpl/PodkopaevLV19}:
  \begin{gather*}
    % \taglabel{RMW2}
    \begin{gathered}
      \PR{y}{r}\SEMI
      \PW{z}{r}
      \PAR
      \PR{z}{r}\SEMI
      \PW{x}{0}\SEMI
      \PFADD[\mRLX][\mREL]{x}{s}{1} \SEMI
      \PW{y}{s}{+}1
      \\[-1ex]
      \hbox{\begin{tikzinline}[node distance=1.5em]
          \event{a1}{\DR{y}{1}}{}
          \event{a2}{\DW{z}{1}}{right=of a1}
          \po{a1}{a2}
          \event{b1}{\DR{z}{1}}{right=3em of a2}
          \rf{a2}{b1}
          \event{b2}{\DW{x}{0}}{right=of b1}
          \event{b3}{\DR{x}{0}}{right=of b2}
          \rf{b2}{b3}
          \event{b4}{\DWRel{x}{1}}{right=2em of b3}
          \rmw{b3}{b4}
          \event{b5}{\DW{y}{1}}{right=of b4}
          \sync[out=-15,in=-165]{b1}{b4}
          \po[out=-20,in=-160]{b3}{b5}
          \rf[out=170,in=10]{b5}{a1}
        \end{tikzinline}}
    \end{gathered}
  \end{gather*}
\end{example}

\begin{example}
  \label{ex:rmw-dep}
  For \RMW{} operations, the independent case for a read should be the same as
  the empty case.  To see why, consider the semantics of local invariant
  reasoning (\xLIR) from \refdef{def:pomsets-lir}:
  \begin{enumerate}
  \item[\ref{S4})]
    $\aTr{\bEvs}{\bForm} \rimplies \bForm[\aExp/\aLoc]\land\aExp{=}\aVal$,
  \item[\ref{S5})]
    $\aTr{\cEvs}{\bForm} \rimplies \bForm[\aExp/\aLoc]$,
  \item[\ref{L4})]
    $\aTr{\bEvs}{\bForm} \rimplies \aVal{=}\aReg\limplies\bForm$, 
  \item[\ref{L5})]
    $\aTr{\cEvs}{\bForm} \rimplies (\aVal{=}\aReg\lor\aLoc{=}\aReg)\limplies\bForm$, when $\aEvs\neq\emptyset$,
  \item[\ref{L6})] 
    $\aTr{\dEvs}{\bForm} \rimplies \bForm$, when $\aEvs=\emptyset$.
  \end{enumerate}
  Consider the relaxed variant of the \textsc{cdrf} example from
  \cite{DBLP:conf/pldi/LeeCPCHLV20}, using a semantics for $\FADD$ that
  simply composes the rules for load and store above.
  \begin{gather*}
    \begin{gathered}
      \PW{x}{0}\SEMI
      \begin{aligned}[t]
        (&\PFADD[\mRLX][\mRLX]{x}{r}{1}\SEMI \IF{\BANG r}\THEN \IF{y}\THEN \PW{x}{0} \FI \FI \;\;\PAR
        \\&
        \PFADD[\mRLX][\mRLX]{x}{r}{1}\SEMI \IF{\BANG r}\THEN \PW{y}{1} \FI)
      \end{aligned}
      \\
      \hbox{\begin{tikzinlinesmall}[node distance=1em]
          \event{i}{\DW{x}{0}}{}
          \event{b0}{\DR{x}{0}}{right=2em of i}
          \event{b0b}{\DW{x}{1}}{right=1.5em of b0}
          \event{b1}{\DR{y}{1}}{right=of b0b}
          \event{b2}{\DW{x}{0}}{right=of b1}
          \event{a1}{\DR{x}{0}}{right=2em of b2}
          \event{a1b}{\DW{x}{1}}{right=1.5em of a1}
          \event{a2}{\DW{y}{1}}{right=of a1b}
          \rmw{a1}{a1b}
          \rmw{b0}{b0b}
          \rf{i}{b0}
          \rf[out=-165,in=-12]{a2}{b1}
          \wk[out=20,in=160]{b0b}{b2}
          % \sync{a1}{a2}
          % \sync{b0}{b1}
          \po{b1}{b2}
          \rf{b2}{a1}
        \end{tikzinlinesmall}}
    \end{gathered}
  \end{gather*}
  Looking at the independent transformers of the second thread and
  initializer, we have:
  \begin{align*}
    \begin{gathered}[t]
      \PW{x}{0}
      \\
      \hbox{\begin{tikzinline}[node distance=.5em and 1.5em]
          \event{a}{\DW{x}{0}}{}      
          \xform{xi}{\bForm[0/x]}{below=of a}
        \end{tikzinline}}    
    \end{gathered}
    &&
    \begin{gathered}[t]
      \smash{\FADD^{\mRLX,\mRLX}(x,1)}
      \\
      \hbox{\begin{tikzinline}[node distance=1em and .75em]
          \event{a0}{\DR{x}{0}}{}
          \node(a)[right=of a0]{};
          \event{a1}{\DW{x}{1}}{right=of a}
          \rmw{a0}{a1}
          \xform{xi}{(0{=}r\lor x{=}r)\limplies\bForm[1/x]}{below=of a}
        \end{tikzinline}}    
    \end{gathered}
    &&
    \begin{gathered}[t]
      \IF{\BANG r}\THEN \PW{y}{1} \FI
      \\
      \hbox{\begin{tikzinline}[node distance=.5em and 1.5em]
          \event{a2}{r{=}0\mid\DW{y}{1}}{}      
          \xform{xi}{\bForm[1/y]}{below=of a2}
        \end{tikzinline}}    
    \end{gathered}
  \end{align*}
  After sequencing, the precondition of $\DWP{y}{1}$ is a tautology:
  $(0{=}r\lor 0{=}r)\limplies r{=}0$.

  Here, local invariant reasoning is using the initializing write to $x$ to
  justify the independence of the write to $y$.  But this write is made
  unavailable by the first thread's successful \RMW{}.
\end{example}
As a result, we disallow the use of \ref{L5} when treating the read event in
an \RMW{}.

[Todo: write out the rules.]

\begin{example}
  Consider the \textsc{cdrf} example from \cite{DBLP:conf/pldi/LeeCPCHLV20}:
  \begin{gather*}
    \begin{gathered}
      \begin{aligned}
        &\PFADD[\mACQ][\mREL]{x}{r}{1}\SEMI \IF{r{=}0}\THEN \PW{y}{1} \FI
        \\\PAR\;\;&
        \PFADD[\mACQ][\mREL]{x}{r}{1}\SEMI \IF{r{=}0}\THEN \IF{y}\THEN \PW{x}{0} \FI \FI
      \end{aligned}
      \\
      \hbox{\footnotesize\begin{tikzinline}[node distance=1.5em]
          \raevent{a1}{\DR[\mACQ]{x}{0}}{}
          \raevent{a1b}{\DW[\mREL]{x}{1}}{right=of a1}
          \event{a2}{\DW{y}{1}}{right=of a1b}
          \raevent{b0}{\DR[\mACQ]{x}{0}}{right=3em of a2}
          \raevent{b0b}{\DW[\mREL]{x}{1}}{right=of b0}
          \event{b1}{\DR{y}{1}}{right=of b0b}
          \event{b2}{\DW{x}{0}}{right=of b1}
          \rmw{a1}{a1b}
          \rmw{b0}{b0b}
          \rf[out=-13,in=-163]{a2}{b1}
          \sync[out=20,in=160]{a1}{a2}
          \sync[out=20,in=160]{b0}{b1}
          \po{b1}{b2}
          \rf[out=-165,in=-12]{b2}{a1}
        \end{tikzinline}}
    \end{gathered}
  \end{gather*}
\end{example}
\begin{example}
  Consider this example from \cite[\textsection C]{DBLP:conf/pldi/LeeCPCHLV20}:
  \begin{gather*}
    \begin{gathered}
      \begin{aligned}
        &\PCAS[\mRLX][\mRLX]{x}{r}{0}{1}\SEMI \IF{r{\leq}1}\THEN \PW{y}{1} \FI
        \\\PAR\;\;&
        \PCAS[\mRLX][\mRLX]{x}{r}{0}{2}\SEMI \IF{r{=}0}\THEN \IF{y}\THEN \PW{x}{0} \FI \FI
      \end{aligned}
      \\
      \hbox{\footnotesize\begin{tikzinline}[node distance=1.5em]
          \event{a1}{\DR{x}{0}}{}
          \event{a1b}{\DW{x}{1}}{right=of a1}
          \event{a2}{\DW{y}{1}}{right=of a1b}
          \event{b0}{\DR{x}{0}}{right=3em of a2}
          \event{b0b}{\DW{x}{2}}{right=of b0}
          \event{b1}{\DR{y}{1}}{right=of b0b}
          \event{b2}{\DW{x}{0}}{right=of b1}
          \rmw{a1}{a1b}
          \rmw{b0}{b0b}
          \rf[out=-13,in=-163]{a2}{b1}
          \sync[out=20,in=160]{a1}{a2}
          \sync[out=20,in=160]{b0}{b1}
          \po{b1}{b2}
          \rf[out=-165,in=-12]{b2}{a1}
        \end{tikzinline}}
    \end{gathered}
  \end{gather*}
\end{example}
% \begin{example}
%   Let $\CAS$ return the value read, which is sufficient to determine whether
%   the $\CAS$ succeeded.
%   \begin{align*}
%     \begin{gathered}
%       \DW{x}{0}\SEMI(
%       \IF{\BANG \CAS(x,0,1)}\THEN \PW{y}{1} \FI
%       \PAR
%       \IF{\BANG \CAS(x,0,1)}\THEN \PW{z}{1} \FI
%       )
%       \\
%       \hbox{\begin{tikzinline}[node distance=1.5em]
%         \event{a1}{\DR{x}{0}}{}
%         \event{a2}{\DW{x}{1}}{right=of a1}
%         \event{a3}{\DW{y}{1}}{right=of a2}
%         \event{b1}{\DR{x}{1}}{right=4em of a3}
%         \event{b2}{\DW{z}{1}}{right=of b1}          
%         \event{i}{\DW{x}{0}}{left=4em of a1}          
%         \rmw{a1}{a2}
%         \rf{i}{a1}
%         \rf[out=-15,in=-165]{a2}{b1}
%       \end{tikzinline}}
%     \end{gathered}
%   \end{align*}
%   This clearly should not be allowed.
%   What's gone wrong here is that 
% \end{example}

\subsection{Fence Operations (\xFENCE)}
\label{sec:fence}
% Fence actions are pretty straightforward to handle in the semantics that uses
% independency for synchronization.  You just add dummy actions with the
% corresponding rules.  See \textsection\ref{sec:independency-ra}.

Syntactic fences $\FENCE^{\fmode}$ have corresponding actions:
$\DFP{\fmode}$.  The \emph{syntactic fence mode}
$(\fmode \!\!\BNFDEF\!\! \fREL \!\BNFSEP\! \fACQ \!\BNFSEP\! \fSC)$ is either
\emph{release}, \emph{acquire}, or \emph{sequentially-consistent}.



Formalizing this, $\QS{\aLoc}{\fREL}$ substitutes for $\Qw{*}$ in addition to
$\Qw{\aLoc}$, as in $\QS{\aLoc}{\mRA}$.  $\QL{\aLoc}{\fACQ}$ requires
$\Qr{*}$ in addition to $\Qw{\aLoc}$, as in $\QL{\aLoc}{\mRA}$.

\begin{definition}
  Extend \refdef{def:QS} and \refdef{def:DS}.
  \begin{align*}
    \QF{\aLoc}{\fREL}
    %=\QS{\aLoc}{\fREL}
    &=\Qr{*}\land\Qw{*} 
    &
    \QF{\aLoc}{\fACQ}
    %=\QL{\aLoc}{\fACQ}
    &=\Qw{\aLoc}\land \Qr{*}
    \\
    [\aForm/\QF{\aLoc}{\fREL}]
    %= [\aForm/\QS{\aLoc}{\fREL}]
    &= [\aForm/\Qw{*}]
    &{}
    [\aForm/\QF{\aLoc}{\fACQ}]
    %= [\aForm/\QL{\aLoc}{\fACQ}]
    &= [\aForm/\Qr{*},\aForm/\Qw{*}]
    % \\
    % \DS{\aLoc}{\fREL}&=[\FALSE/\D]
    % &\DL{\aLoc}{\fACQ}&=\Dx{\aLoc}%\land\D{}
  \end{align*}
  \begin{align*}
    \QF{\aLoc}{\fSC}
    % = \QS{\aLoc}{\fSC}
    % = \QL{\aLoc}{\fSC}
    &= \Qr{*}\land\Qw{*} \land\Qsc
    \\
    [\aForm/\QF{\aLoc}{\fSC}] 
    % = [\aForm/\QS{\aLoc}{\fSC}] 
    % = [\aForm/\QL{\aLoc}{\fACQ}]
    &= [\aForm/\Qr{*},\aForm/\Qw{*},\aForm/\Qsc]
  \end{align*}
\end{definition}

If $\aPS \in \sFENCE[\ascope]{\amode}$ then
$(\exists\aLocs\subseteq\Loc)$
%$(\exists\bmode\in\{\amode,\mRLX\})$
\begin{enumerate}[resume]
\item%[{\labeltext[F1]{(F1)}{F1}}]
  if $\bEv,\aEv\in\aEvs$ then $\bEv=\aEv$,
\item%[{\labeltext[F2]{(F2)}{F2}}]
  $\labelingAct(\aEv) = \DG[\ascope]{\amode}{\aLocs}$,
\item%[{\labeltext[F3]{(F3)}{F3}}]
  $\labelingForm(\aEv) \rimplies \bigwedge_{\aLoc\in (\Loc\setminus\aLocs)}\QF{\aLoc}{\amode}$,
\item%[{\labeltext[F4]{(F4)}{F4}}]
  $\aTr{\bEvs}{\bForm} \rimplies \bForm$, where $\bEvs\cap\aEvs\neq\emptyset$,
\item%[{\labeltext[F5]{(F5)}{F5}}]
  $\aTr{\cEvs}{\bForm} \rimplies \bigwedge_{\aLoc\in\aLocs}\Dx{\aLoc} \land\bForm[\FALSE/\QF{\aLoc}{}]$, where $\cEvs\cap\aEvs=\emptyset$.
\end{enumerate}

\begin{example}
  Extend \refex{ex:q1}.
  \begin{gather*}
    \begin{gathered}
      \begin{gathered}[t]
        \PW[\fREL]{x}{\aExp}
        \\
        \hbox{\begin{tikzinline}[node distance=.5em and 1.5em]
            \raevent{a}{\Qr{*}\land\Qw{*}\land\aExp{=}v\mid\DW[\mREL]{x}{v}}{}
            \xform{xi}{\bForm[\FALSE/\Qw{*}]}{above=of a}
            \xform{xd}{\bForm\land\aExp{=}\aVal}{below=of a}
            \xo{a}{xd}
          \end{tikzinline}}
      \end{gathered}
      \\
      \begin{gathered}[t]
        \PF{\fREL}
        \\
        \hbox{\begin{tikzinline}[node distance=.5em and 1.5em]
            \raevent{a}{\Qr{*}\land\Qw{*}\mid\DF{\fREL}}{}
            \xform{xi}{\bForm[\FALSE/\Qw{*}]}{above=of a}
            \xform{xd}{\bForm}{below=of a}
            \xo{a}{xd}
          \end{tikzinline}}
      \end{gathered}      
    \end{gathered}
    \qquad
    \begin{gathered}
      \begin{gathered}[t]
        \PR[\fACQ]{x}{r}
        \\
        \hbox{\begin{tikzinline}[node distance=.5em and 1.5em]
            \raevent{a}{\Qw{x}\land\Qr{*}\mid\DR[\fACQ]{x}{v}}{}
            \xform{xi}{\bForm[\FALSE/\Qr{*}][\FALSE/\Qw{*}]}{above=of a}
            \xform{xd}{v{=}r\limplies\bForm}{below=of a}
            \xo{a}{xd}
          \end{tikzinline}}
      \end{gathered}
      \\
      \begin{gathered}[t]
        \PF{\fACQ}
        \\
        \hbox{\begin{tikzinline}[node distance=.5em and 1.5em]
            \raevent{a}{\Qr{*}\mid\DF{\fACQ}}{}
            \xform{xi}{\bForm[\FALSE/\Qr{*}][\FALSE/\Qw{*}]}{above=of a}
            \xform{xd}{\bForm}{below=of a}
            \xo{a}{xd}
          \end{tikzinline}}
      \end{gathered}
    \end{gathered}
    \\
    \begin{gathered}[t]
      \PF{\fSC}
      \\
      \hbox{\begin{tikzinline}[node distance=.5em and 1.5em]
          \scevent{a}{\Qr{*}\land\Qw{*}\land\Qsc\mid\DF{\mSC}}{}
          \xform{xi}{\bForm[\FALSE/\Qr{*}][\FALSE/\Qw{*}][\FALSE/\Qsc]}{above=of a}
          \xform{xd}{\bForm}{below=of a}
          \xo{a}{xd}
        \end{tikzinline}}
    \end{gathered}
    \\
    \begin{gathered}[t]
      \PW[\fSC]{x}{\aExp}
      \\
      \hbox{\begin{tikzinline}[node distance=.5em and 1.5em]
          \scevent{a}{\Qr{*}\land\Qw{*}\land\Qsc\land\aExp{=}\aVal\mid\DW[\fSC]{x}{v}}{}
          \xform{xi}{\bForm[\FALSE/\Qr{*}][\FALSE/\Qw{*}][\FALSE/\Qsc]}{above=of a}
          \xform{xd}{\bForm\land\aExp{=}\aVal}{below=of a}
          \xo{a}{xd}
        \end{tikzinline}}
    \end{gathered}
    \;
    \begin{gathered}[t]
      \PR[\fSC]{x}{\aExp}
      \\
      \hbox{\begin{tikzinline}[node distance=.5em and 1.5em]
          \scevent{a}{\Qr{*}\land\Qw{*}\land\Qsc\mid\DR[\fSC]{x}{v}}{}
          \xform{xi}{\bForm[\FALSE/\Qr{*}][\FALSE/\Qw{*}][\FALSE/\Qsc]}{above=of a}
          \xform{xd}{v{=}r\limplies\bForm}{below=of a}
          \xo{a}{xd}
        \end{tikzinline}}
    \end{gathered}
  \end{gather*}
\end{example}



\subsection{Fence Actions with Downgrading Reads (\xFENCE/\xDGR)}
\label{sec:fence}

\begin{example}
  Revisiting \refex{ex:dgr2} using fences:
  % \begin{align*}
  %   \begin{gathered}[t]
  %     \PW{x}{2}
  %     \\
  %     \hbox{\begin{tikzinline}[node distance=0.5em and 1.5em]
  %       \event{a}{\DW{x}{2}}{}
  %       \xform{x1d}{\bForm[\TRUE/\Dx{\aLoc}]}{below=of a}
  %         %       \xform{x1i}{\lnot\Qw{x}\land\bForm[\TRUE/\Dx{x}]}{below=of x1d}
  %       \xo{a}{x1d}
  %     \end{tikzinline}}  
  %   \end{gathered}  
  %   &&
  %   \begin{gathered}[t]
  %     \PR[\mACQ]{x}{r}
  %     \\
  %     \hbox{\begin{tikzinline}[node distance=0.5em and 1.5em]
  %       \raevent{b}{\DR[\mACQ]{x}{2}}{}
  %         %       \xform{x1d}{2{=}r\limplies\bForm}{below=of b}
  %       \xform{x1i}{\Dx{x} \land\bForm[\FALSE/\Qr{x}]}{below=of b} %x1d}
  %         %       \xo{b}{x1d}
  %     \end{tikzinline}}  
  %   \end{gathered}  
  %   &&
  %   \begin{gathered}[t]
  %     \PW{y}{1}
  %     \\
  %     \hbox{\begin{tikzinline}[node distance=0.5em and 1.5em]
  %       \event{c}{\DW{y}{1}}{}
  %         %       \xform{x1d}{\bForm[\TRUE/\Dx{\aLoc}]}{below=of c}
  %         %       \xform{x1i}{\lnot\Qw{y}\land\bForm[\TRUE/\Dx{y}]}{below=of x1d}
  %         %       \xo{c}{x1d}
  %     \end{tikzinline}}  
  %   \end{gathered}  
  % \end{align*}
  % Associating right:
  % \begin{align*}
  %   \begin{gathered}[t]
  %     \PW{x}{2}
  %     \\
  %     \hbox{\begin{tikzinline}[node distance=0.5em and 1.5em]
  %       \event{a}{\DW{x}{2}}{}
  %       \xform{x1d}{\bForm[\TRUE/\Dx{\aLoc}]}{right=.7em of a}
  %         %       \xform{x1i}{\lnot\Qw{x}\land\bForm[\TRUE/\Dx{x}]}{below=of x1d}
  %       \xos{a}{x1d}
  %     \end{tikzinline}}  
  %   \end{gathered}  
  %   &&
  %   \begin{gathered}[t]
  %     \PR[\mACQ]{x}{r}
  %     \SEMI
  %     \PW{y}{1}
  %     \\
  %     \hbox{\begin{tikzinline}[node distance=0.5em and 1.5em]
  %       \raevent{b}{\DR[\mACQ]{x}{2}}{}
  %       \event{c}{\Dx{x}\mid\DW{y}{1}}{right=.7em of b}
  %     \end{tikzinline}}  
  %   \end{gathered}  
  % \end{align*}
  % Composing, we have, as desired:
  \begin{align*}
    \begin{gathered}[t]
      \PW{x}{2}
      \SEMI
      \PR{x}{r}
      \SEMI
      \PF{\fACQ}
      \SEMI
      \PW{y}{1}
      \\
      \hbox{\begin{tikzinline}[node distance=0.5em and 1.5em]
          \event{a}{\DW{x}{2}}{}
          \event{b}{\DR{x}{2}}{right=of a}
          \raevent{c}{\DF{\fACQ}}{right=of b}
          \event{d}{\DW{y}{1}}{right=of c}
          \rf{a}{b}
          \sync{b}{c}
          \sync{c}{d}
        \end{tikzinline}}  
    \end{gathered}  
  \end{align*}
  What we want is this:
  \begin{align*}
    \begin{gathered}[t]
      \hbox{\begin{tikzinline}[node distance=0.5em and 1.5em]
          \event{a}{\DW{x}{2}}{}
          \event{b}{\DR{x}{2}}{right=of a}
          \raevent{c}{\DF{\fACQ}}{right=of b}
          \event{d}{\DW{y}{1}}{right=of c}
          \rf{a}{b}
          \sync{c}{d}
        \end{tikzinline}}  
    \end{gathered}  
  \end{align*}
\end{example}

Let acquiring fence actions include a set: $\DGP{\fmode}{\aLocs}$.
The set is nondeterministically chosen by the semantics.
Idea is to downgrade the reads in $\aLocs$ and fence everything else.


\subsection{Extended Access Modes}

We can enrich read and write actions to use fence modes.
The resulting order is:
% Reads use the four point order:
% \begin{math}
%   \mRLX \ltmode \mRA \ltmode \fREL \ltmode \mSC.
% \end{math}
% Writes use the four point order:
% \begin{math}
%   \mRLX \ltmode \mRA \ltmode \fACQ \ltmode \mSC.
% \end{math}
\begin{displaymath}
  \begin{tikzpicture}
    \node (rlx) at (0, 0) {$\mathstrut\mRLX$};
    \node (ra)  at (1, 0) {$\mathstrut\mRA$};
    \node (sc)  at (2, 0) {$\mathstrut\mSC$};
    \node (fsc) at (3, 0) {$\mathstrut\fSC$};
    \node (rel) at (2, -0.3) {$\mathstrut\fREL$};
    \node (acq) at (2,  0.3) {$\mathstrut\fACQ$};
    \draw[->](rlx)to(ra);
    \draw[->](ra)to(sc);
    \draw[->](ra)to(rel);
    \draw[->](ra)to(acq);
    \draw[->](sc)to(fsc);
    \draw[->](rel)to(fsc);
    \draw[->](acq)to(fsc);
  \end{tikzpicture}
\end{displaymath}
We write $\amode\lemode\bmode$ for this order.
Let $\amode\lubmode\bmode$ denote the least upper bound of $\amode$ and $\bmode$.

Reads allow all annotations but $\fREL$.  Writes allow all annotations but
$\fACQ$.  Fences allow only the three annotations $\fREL$, $\fACQ$ and
$\fSC$.
% \begin{displaymath}
%   \begin{tikzcd}[column sep=.5em, row sep=.5ex]
%                              &                                                                 & \fREL \arrow[rd,sqsubset] &        \\
%     \mRLX \arrow[r,sqsubset] & \mRA \arrow[r,sqsubset] \arrow[ru,sqsubset] \arrow[rd,sqsubset] & \mSC  \arrow[r,sqsubset]  & \fSC  \\
%                              &                                                                 & \fACQ \arrow[ru,sqsubset] &          
%   \end{tikzcd}
% \end{displaymath}


\begin{definition}
  \showRAtrue
  \begin{align*}
    \fmerge{\DR[\amode]{\aLoc}{\aVal}}{\DR[\bmode]{\aLoc}{\aVal}}
    % = \fmerge{\DF{\amode}}{\DR[\bmode]{\aLoc}{\aVal}}
    % = \fmerge{\DR[\amode]{\aLoc}{\aVal}}{\DF{\bmode}}
    &= \{ \DR[\amode\lubmode\bmode]{\aLoc}{\aVal} \}
    \\
    \fmerge{\DW[\amode]{\aLoc}{\aVal}}{\DW[\bmode]{\aLoc}{\bVal}}
    % = \fmerge{\DF{\amode}}{\DW[\bmode]{\aLoc}{\bVal}}
    % = \fmerge{\DW[\amode]{\aLoc}{\bVal}}{\DF{\bmode}}
    &= \{ \DW[\amode\lubmode\bmode]{\aLoc}{\bVal} \}
    \\
    \fmerge{\DF{\amode}}{\DF{\bmode}} &= \{ \DF{\amode\lubmode\bmode} \}
    \\
    \fmerge{\DF{\amode}}{\DR[\bmode]{\aLoc}{\aVal}}
    = \fmerge{\DR[\amode]{\aLoc}{\aVal}}{\DF{\bmode}}
    &= \{ \DR[\amode\lubmode\bmode]{\aLoc}{\aVal} \}
    \\
    \fmerge{\DF{\amode}}{\DW[\bmode]{\aLoc}{\bVal}}
    = \fmerge{\DW[\amode]{\aLoc}{\bVal}}{\DF{\bmode}}
    &= \{ \DW[\amode\lubmode\bmode]{\aLoc}{\bVal} \}
    \\
    \fmerge{\aAct}{\bAct} &= \emptyset, \textotherwise
  \end{align*}
\end{definition}  
\section{Discussion}
\subsection{Closure Properties}

We would like the semantics to be closed with respect to \emph{augments} and
\emph{downsets}.

Augments include more order and stronger formulae; in examples, we typically
consider pomsets that are augment-minimal.  One intuitive reading of augment
closure is that adding order can only cause preconditions to weaken.
\begin{definition}
  \label{def:augment}
  $\aPS_2$ is an \emph{augment} of $\aPS_1$ if
  \begin{enumerate}
  \item $\aEvs_2=\aEvs_1$,
  \item $\labelingAct_2(\aEv)=\labelingAct_1(\aEv)$,
  \item $\labelingForm_2(\aEv) \rimplies \labelingForm_1(\aEv)$,
  \item $\aTr[2]{\bEvs}{\aEv} \rimplies \aTr[1]{\bEvs}{\aEv}$,
  \item if $\bEv\le_2\aEv$ then $\bEv\le_1\aEv$.
  \end{enumerate}
\end{definition}

\begin{proposition}
  % Suppose $\aPS_1\in\sem{\aCmd}$.
  If $\aPS_1\in\sem{\aCmd}$ and $\aPS_2$  augments $\aPS_1$ then $\aPS_2\in\sem{\aCmd}$.
  % \item If $\aPS_2$ is a downset of $\aPS_1$ then $\aPS_2\in\sem{\aCmd}$.
  % \end{enumerate}
\end{proposition}

Downsets include a subset of initial events, similar to \emph{prefixes} for
strings.
\begin{definition}
  \label{def:downset}
  $\aPS_2$ is an \emph{downset} of $\aPS_1$ if
  \begin{enumerate}
  \item $\aEvs_2\subseteq\aEvs_1$,
  \item $(\forall \aEv\in\aEvs_2)$ $\labelingAct_2(\aEv)=\labelingAct_1(\aEv)$,
  \item $(\forall \aEv\in\aEvs_2)$ $\labelingForm_2(\aEv)=\labelingForm_1(\aEv)$,
  \item $(\forall \aEv\in\aEvs_2)$ $\aTr[2]{\bEvs}{\aEv}=\aTr[1]{\bEvs}{\aEv}$,
  \item $(\forall \bEv\in\aEvs_2)$ $(\forall \aEv\in\aEvs_2)$ $\bEv\le_2\aEv$ if and only if $\bEv\le_1\aEv$,
  \item $(\forall \bEv\in\aEvs_1)$ $(\forall \aEv\in\aEvs_2)$ if $\bEv\le_1\aEv$ then $\bEv\in\aEvs_2$.
  \end{enumerate}
\end{definition}

Downset closure fails due to \xRRD{} and \xLIR{}.  The key property is that
the empty set transformer should behave the same as the independent
transformer.

\begin{example}
  For \xRRD{}, \refdef{def:pomsets-rr} states:
  \begin{enumerate}
  \item[\ref{L4})]
    $\aTr{\bEvs}{\bForm} \rimplies \aVal{=}\aReg\limplies\bForm$, 
  \item[\ref{L5})]
    $\aTr{\cEvs}{\bForm} \rimplies (\aVal{=}\aReg\lor\RW)\limplies\bForm$,
  \item[\ref{L6})] 
    $\aTr{\dEvs}{\bForm} \rimplies \bForm$, when $\aEvs=\emptyset$.
  \end{enumerate}
  This semantics is not downset closed due to the lack of read-read dependencies.
  In both cases, for subsequent writes, \ref{L5} is the same as \ref{L6}.  For
  subsequent reads, \ref{L5} is the same as \ref{L4}.
  Consider
  \begin{gather*}
    \begin{gathered}[t]
      \PR{x}{r}\SEMI\IF{\BANG r}\THEN\PR{y}{s}\FI
      \\
      \hbox{\begin{tikzinline}[node distance=.5em and 1.5em]
          \event{a}{\DR{x}{0}}{}
          \event{b}{\DR{y}{0}}{right=of a}
        \end{tikzinline}}
    \end{gathered}    
  \end{gather*}
  The semantics of this program includes the singleton pomset $\DRP{x}{0}$,
  but not the singleton pomset $\DRP{y}{0}$.
  To get $\DRP{x}{0}$, we combine:
  \begin{align*}
    \begin{gathered}[t]
      \PR{x}{r}
      \\
      \hbox{\begin{tikzinline}[node distance=.5em and 1.5em]
          \event{a}{\DR{x}{0}}{}
        \end{tikzinline}}
    \end{gathered}    
    &&
    \begin{gathered}[t]
      \IF{\BANG r}\THEN\PR{y}{s}\FI
      \\
      \emptyset
    \end{gathered}    
  \end{align*}
  Attempting to get $\DRP{y}{0}$, we instead get:
  \begin{align*}
    \begin{gathered}[t]
      \PR{x}{r}
      \\
      \emptyset
    \end{gathered}    
    &&
    \begin{gathered}[t]
      \IF{\BANG r}\THEN\PR{y}{s}\FI
      \\
      \hbox{\begin{tikzinline}[node distance=.5em and 1.5em]
          \event{b}{r\EQ0\mid\DR{y}{0}}{}
        \end{tikzinline}}
    \end{gathered}    
  \end{align*}
  Since $r$ appears only once in the program, this pomset cannot contribute
  to a top-level pomset.
\end{example}

\begin{example}
  \label{ex:downset-lir}
  For \xLIR{}, \refdef{def:pomsets-lir} states:
  \begin{enumerate}
  \item[\ref{L4})]
    $\aTr{\bEvs}{\bForm} \rimplies \aVal{=}\aReg\limplies\bForm$, 
  \item[\ref{L5})]
    $\aTr{\cEvs}{\bForm} \rimplies (\aVal{=}\aReg\lor\aLoc{=}\aReg)\limplies\bForm$, when $\aEvs\neq\emptyset$,
  \item[\ref{L6})] 
    $\aTr{\dEvs}{\bForm} \rimplies \bForm$, when $\aEvs=\emptyset$.
  \end{enumerate}
  This semantics is not downset closed: The independency reasoning of
  \ref{L5} is only applicable for pomsets where the ignored read is present!
  Revisiting \refex{ex:tc1-revisit}
  \begin{align*}
    \begin{gathered}[t]
      \PW{x}{0} 
      \\
      \hbox{\begin{tikzinline}[node distance=.5em and 1.5em]
          \event{a0}{\DW{x}{0}}{}
          \xform{xi}{\bForm[0/x]}{below=of a0}
        \end{tikzinline}}    
    \end{gathered}
    &&
    \begin{gathered}[t]
      \PR{x}{r} 
      \\
      \hbox{\begin{tikzinline}[node distance=.5em and 1.5em]
          \event{a1}{\DR{x}{1}}{}
          \xform{xi}{(1{=}r\lor x{=}r)\limplies\bForm}{below=of a1}
        \end{tikzinline}}    
    \end{gathered}
    &&
    \begin{gathered}[t]
      \IF{r{\geq}0}\THEN \PW{y}{1} \FI
      \\
      \hbox{\begin{tikzinline}[node distance=.5em and 1.5em]
          \event{a2}{r{\geq}0\mid\DW{y}{1}}{}      
        \end{tikzinline}}    
    \end{gathered}
  \end{align*}
  % Composing:
  \begin{align*}
    \begin{gathered}[t]
      \PW{x}{0} 
      \SEMI\PR{x}{r} 
      \SEMI\IF{r{\geq}0}\THEN \PW{y}{1} \FI
      \\
      \hbox{\begin{tikzinline}[node distance=.5em and 1.5em]
          \event{a0}{\DW{x}{0}}{}
          \event{a1}{\DR{x}{1}}{right=of a0}
          \event{a2}{(1{=}r\lor 0{=}r)\limplies r{\geq}0\mid\DW{y}{1}}{right=of a1}      
          \wk{a0}{a1}
        \end{tikzinline}}    
    \end{gathered}
  \end{align*}
  The precondition of $\DWP{y}{1}$ is a tautology.

  Taking the empty set for the read, however, we have:
  \begin{align*}
    \begin{gathered}[t]
      \PW{x}{0} 
      \SEMI\PR{x}{r} 
      \SEMI\IF{r{\geq}0}\THEN \PW{y}{1} \FI
      \\
      \hbox{\begin{tikzinline}[node distance=.5em and 1.5em]
          \event{a0}{\DW{x}{0}}{}
          % \event{a1}{\DR{x}{1}}{right=of a0}
          \event{a2}{r{\geq}0\mid\DW{y}{1}}{right=3em of a1}      
          % \wk{a0}{a1}
        \end{tikzinline}}    
    \end{gathered}
  \end{align*}
  The precondition of $\DWP{y}{1}$ is not a tautology.
\end{example}

\subsection{Comparison with Weakest Preconditions}

We compare traditional transformers to the dependent-case transformers of
\refdef{def:pomsets-lir}; thus we consider only totally ordered executions.
Because we only consider the dependent case, we drop the superscript $\aEvs$
on $\aTr{\aEvs}{}$ throughout this section.  We also assume that each
register appears at most once in a program, as we did throughout
\textsection\ref{sec:model}--\ref{sec:arm}.

Because of augment closure, we are not interested in isolating the
\emph{weakest} precondition.  Thus we think of transformers as Hoare triples.
In addition, all programs in our language are strongly normalizing, so we
need not distinguish strong and weak correctness.  In this setting, the Hoare
triple $\hoare{\aForm}{\aCmd}{\bForm}$ holds exactly when
$\aForm \limplies \fwp{\aCmd}{\bForm}$.

Hoare triples do not distinguish thread-local variables from shared
variables.  Thus, the assignment rule applies to all types of storage. The
rules can be written as follows:
\begin{align*}
  \fwp{\PW{\aLoc}{\aExp}}{\bForm} &= \bForm[\aExp/\aLoc]
  \\
  \fwp{\LET{\aReg}{\aExp}}{\bForm} &= \bForm[\aExp/\aReg]
  \\
  \fwp{\PR{\aLoc}{\aReg}}{\bForm} &= \aLoc{=}\aReg\limplies\bForm
\end{align*}
Here we have chosen an alternative formulation for the read rule, which is
equivalent the more traditional $\bForm[\aLoc/\aReg]$, as long as registers
occur at most once in a program.  In \refdef{def:pomsets-lir}, the
transformers for the dependent case are as follows:
\begin{align*}
  \trd{\PW{\aLoc}{\aExp}}{\bForm} &= \bForm[\aExp/\aLoc]
  \\
  \trd{\LET{\aReg}{\aExp}}{\bForm} &= \bForm[\aExp/\aReg]
  \\
  \trd{\PR{\aLoc}{\aReg}}{\bForm} &= \aVal{=}\aReg\limplies\bForm &&
  \textwhere \labelingAct(\aEv)=\DR{\aLoc}{\aVal}
\end{align*}
Only the read rule differs from the traditional one.


For programs where every register is bound and every read is fulfilled, our
dependent transformers are the same as the traditional ones.  In our
semantics, thus, we only consider totally-ordered executions where every read
could be fulfilled by prepending some writes.  For example, we ignore pomsets
of $\PW{x}{2}\SEMI\PR{x}{r}$ that read $1$ for $x$.

For example, let $\aCmd_i$ be defined:
% as follows.
\begin{align*}
  \aCmd_1&=\PR{x}{s}\SEMI\PW{x}{s{+}r}
  \\  
  \aCmd_2&=\PW{x}{t}\SEMI\aCmd_1
  \\  
  \aCmd_3&=\LET{t}{2}\SEMI\LET{r}{5}\SEMI\aCmd_2
\end{align*}
% \begin{itemize}
% \item
%   \begin{math}
%     \fwp{\LET{\aReg}{\aExp}}{\bForm} = \bForm[\aExp/\aReg]
%   \end{math}
% \item
%   \begin{math}
%     \fwp{\PR{\aLoc}{\aReg}\;\,}{\bForm} = %\bForm[x/r]
%     %     (\forall\bReg)
%     %     \aLoc{=}\bReg\limplies\bForm [\bReg/\aLoc]
%     \aLoc{=}\aReg\limplies\bForm
%   \end{math}
% \item
%   \begin{math}
%     \fwp{\PW{\aLoc}{\aExp}}{\bForm} = \bForm[\aExp/\aLoc]
%   \end{math}
% \end{itemize}
% General relation between Hoare triples and $\fwp{}{}$:
% \begin{itemize}
% \item $\hoare{\fwp{\aCmd}{\bForm}}{\aCmd}{\bForm}$,
% \item If $\hoare{\aForm}{\aCmd}{\bForm}$ and $\aCmd$ terminates when starting
%   in any state satisfying $\aForm$, then $\aForm \limplies \fwp{\aCmd}{\bForm}$.
% \end{itemize}
The following pomset appears in the semantics of $\aCmd_2$.  A pomset for
$\aCmd_3$ can be derived by substituting $[2/t,\allowbreak5/r]$.  A pomset
for $\aCmd_1$ can be derived by eliminating the initial write.
\begin{gather*}
  % \begin{gathered}[t]
  %   \PW{x}{3}
  %   \\
  %   \hbox{\begin{tikzinline}[node distance=.5em and 1.5em]
  %     \event{c}{\DW{x}{3}}{}
  %     \xform{xd}{\bForm}{below=of c}
  %     \xo{c}{xd}
  %   \end{tikzinline}}
  % \end{gathered}
  % \qquad\quad
  % \begin{gathered}[t]
  %   \PR{x}{s}\SEMI\PW{x}{s{+}r}
  %   \\
  %   \hbox{\begin{tikzinline}[node distance=.5em and 1.5em]
  %     \event{a}{\DR{x}{2}}{}
  %     \event{b}{2{=}s\limplies(s{+}r){=}7\mid\DW{x}{7}}{right=of a}%6.5em of a}
  %     \po{a}{b}
  %     \xform{xdd}{2{=}s \limplies \bForm[s{+}r/x]}{below right=.5em and -1em of a}
  %       %     \xform{xdd}{2{=}s \limplies \bForm[s{+}r/x]}{above=of a}
  %       %     \xform{xdi}{2{=}s \limplies \bForm[s{+}r/x]}{below=of a}
  %       %     \xform{xii}{(2{=}s\lor x{=}s)\limplies\bForm[s{+}r/x]}{above=of b}
  %       %     \xform{xid}{(2{=}s\lor x{=}s)\limplies\bForm[s{+}r/x]}{below=of b}
  %       %     \xo{a}{xdi}
  %       %     \xo{b}{xid}
  %     \xo{a}{xdd}
  %     \xo{b}{xdd}
  %   \end{tikzinline}}
  % \end{gathered}
  % \\[1ex]
  \begin{gathered}[t]
    % \LET{t}{2}\SEMI
    % \LET{r}{5}\SEMI
    \PW{x}{t}\SEMI
    \PR{x}{s}\SEMI\PW{x}{s{+}r}
    \\
    \hbox{\begin{tikzinline}[node distance=.5em and 1.5em]
        \event{a}{\DR{x}{2}}{}
        \event{b}{2{=}s\limplies(s{+}r){=}7\mid\DW{x}{7}}{right=of a}
        \event{c}{t{=}2\mid\DW{x}{2}}{left=of a}
        \xform{xdd}{2{=}s \limplies \bForm[s{+}r/x]}{below=of a}%below right=.5em and -1em of a}
        \xos{a}{xdd}
        \xos{b}{xdd}
        \xos{c}{xdd}
        \po{a}{b}
        \rf{c}{a}
      \end{tikzinline}}
  \end{gathered}
\end{gather*}
The predicate transformers are:
% \begin{align*}
%   \fwp{\aCmd_1}{\bForm} &= x{=}s\limplies\bForm[s{+}r/x] 
%   \\
%   \fwp{\aCmd_2}{\bForm} &= t\,{=}s\limplies\bForm[s{+}r/x] 
%   \\
%   \fwp{\aCmd_3}{\bForm} &= 2{=}s\limplies\bForm[s{+}5/x] 
%   \\
%   \trd{\aCmd_1}{\bForm} = \trd{\aCmd_2}{\bForm} &= 2{=}s\limplies\bForm[s{+}r/x] 
%   \\
%   \trd{\aCmd_3}{\bForm} &= 2{=}s\limplies\bForm[s{+}5/x] 
% \end{align*}
\begin{scope}
  \small
  \begin{align*}
    \fwp{\aCmd_1}{\bForm} &= x{=}s\limplies\bForm[s{+}r/x] 
    &
    \trd{\aCmd_1}{\bForm} &= 2{=}s\limplies\bForm[s{+}r/x] 
    \\
    \fwp{\aCmd_2}{\bForm} &= t\,{=}s\limplies\bForm[s{+}r/x] 
    &
    \trd{\aCmd_2}{\bForm} &= 2{=}s\limplies\bForm[s{+}r/x] 
    \\
    \fwp{\aCmd_3}{\bForm} &= 2{=}s\limplies\bForm[s{+}5/x] 
    &
    \trd{\aCmd_3}{\bForm} &= 2{=}s\limplies\bForm[s{+}5/x] 
  \end{align*}
\end{scope}

% % Let $\rho:\Reg\fun\Val$ and $\chi:\Loc\fun\Val$ be substitutions.
% Let $\aState$ and $\rho$ range over substitutions $(\Reg\cup\Loc)\fun\Val$.
% Treating substitutions as states, the big-step operational semantics of
% programs can be defined as a relation $\bigstep{\aState}{\aCmd}{\bState}$.
% % Ie, $\aForm\aState$ implies $\bForm\bState$.
% \begin{align}
%   \label{wp1}
%   \bigstep{[5/r,2/x]}{\aCmd_1&}{[5/r,2/s,7/x]}
%   \\
%   \label{wp2}
%   \bigstep{[\NEG5/r,2/x]}{\aCmd_1&}{[\NEG5/r,2/s,\NEG3/x]}
% \end{align}

% Then the semantics of Hoare triples guarantees that if
% $\aForm\limplies\fwp{\aCmd}{\bForm}$, $\bigstep{\aState}{\aCmd}{\rho}$ and
% $\aForm\aState$ is a tautology then $\bForm\bState$ is a tautology.
% \begin{align*}
%   \fwp{\aCmd_1}{x{>}0} &= (x{+}r{>}0) 
% \end{align*}
% In \eqref{wp1}, the pre- and post-conditions are satisfied.
% In \eqref{wp2}, they are not.


% \begin{itemize}  
% \item Suppose $\bigstep{\aState}{\aCmd}{\rho}$ and $\aForm\limplies\fwp{\aCmd}{\bForm}$.\\
%   If $\aForm\aState$ is a tautology then $\bForm\bState$ is a tautology.\\
%   Ie, $\aForm\aState$ implies $\bForm\bState$.
% \item Suppose $\bigstep{\aState}{\aCmd}{\rho}$ and $\hoare{\aForm}{\aCmd}{\bForm}$.\\
%   If $\aForm\aState$ is a tautology then $\bForm\bState$ is a tautology.\\
%   Ie, $\aForm\aState$ implies $\bForm\bState$.
% \item Suppose $\bigstep{\aState}{\aCmd}{\rho}$ and $\aForm=\fwp{\aCmd}{\bForm}$.\\
%   $\aForm\aState$ is a tautology if and only if $\bForm\bState$ is a tautology.\\
%   Ie, $\aForm\aState$ iff $\bForm\bState$.
% % \item Weakest: If $\aForm'\aState$ is a tautology, then $\aForm$ implies $\aForm'$.
% \end{itemize}
% Weakest preconditions are \emph{sound} in that if $\aForm$ holds in the
% initial state $\aState$, then $\bForm$ holds in the final state $\bState$.
% Formally, 


\begin{comment}
  If $\aPS\in\sem{\aCmd}$ is top-level and quiescent then 
  $\aTr{\aEvs}{\bForm}$ implies $\fwp{\aCmd}{\bForm}$.

  For any substitution $\aSub=[{v_1/r_1},\ldots, {v_n/r_n}]$ there is some
  $\aPS\in\sem{\aCmd}$ %that is top-level and quiescent
  such that all preconditions in $\aPS\aSub$ are tautologies then 
  $\fwp{\aCmd}{\bForm}\aSub$
\end{comment}


% For a language where all programs are
% terminating, we have for any statement $\aCmd$:
% \begin{align*}
%   \hoare{\aForm}{\aCmd}{\bForm} 
%   \;\;\Leftrightarrow\;\;
%   \aForm \textimplies \fwp{\aCmd}{\bForm}
% \end{align*}
% Interpretation is that if $\aState\models\fwp{\aCmd}{\bForm}$ and
% $\bigstep{\aState}{\aCmd}{\rho}$
% then $\bState\models\bForm$.

% Let $\aCmd_0$ be
% \begin{math}
%   \PW{\aLoc_1}{\aVal_1}\SEMI\cdots\SEMI \PW{\aLoc_n}{\aVal_n}, 
% \end{math}
% such that $\fwp{\aCmd_0}{\aForm}$ is a tautology, and $\aLoc_i=\aLoc_j$
% implies $i=j$.

% Let $\aSub_\aPS=[{\aVal_1/\aLoc_1},\ldots, {\aVal_n/\aLoc_n}]$ be the final
% state of $\aPS$.

% Let $\aState$ and $\rho$ range over substitutions $\Loc\fun\Exp$.
% If we leave the registers free, we have:
% \begin{align}
%   \label{wp1x}
%   \bigstep{[2/x]}{\aCmd&}{[6/x]}
% \end{align}

% Using \refdef{def:pomsets-trans}:
% \begin{align*}
%   \begin{gathered}[t]
%     %     \PR{x}{s}\SEMI\PW{x}{s{+}r}
%     \PR{x}{s}
%     \\
%     \hbox{\begin{tikzinline}[node distance=.5em and 1em]
%       \event{a}{\DR{x}{2}}{}
%       \xform{xi}{\bForm}{above=of a}
%       \xform{xd}{2{=}s \limplies \bForm}{below=of a}
%       \xo{a}{xd}
%     \end{tikzinline}}
%   \end{gathered}
%   &&
%   \begin{gathered}[t]
%     \PW{x}{s{+}r}
%     \\
%     \hbox{\begin{tikzinline}[node distance=.5em and 1em]
%       \event{a}{(s{+}r){=}7\mid\DW{x}{7}}{}
%       \xform{xi}{\bForm}{above=of a}
%       \xform{xd}{\bForm}{below=of a}
%       \xo{a}{xd}
%     \end{tikzinline}}
%   \end{gathered}
% \end{align*}
% Composing
% \begin{align*}
%   \begin{gathered}[t]
%     \PR{x}{s}\SEMI\PW{x}{s{+}r}
%     \\
%     \hbox{\begin{tikzinline}[node distance=.5em and 1em]
%       \event{a}{\DR{x}{2}}{}
%       \event{b}{(s{+}r){=}7\mid\DW{x}{7}}{right=of a}
%       \xform{xdd}{2{=}s \limplies \bForm}{above=of a}
%       \xform{xdi}{2{=}s \limplies \bForm}{below=of a}
%       \xform{xii}{\bForm}{above=of b}
%       \xform{xid}{\bForm}{below=of b}
%       \xo{a}{xdi}
%       \xo{b}{xid}
%       \xo{a}{xdd}
%       \xo{b}{xdd}
%     \end{tikzinline}}
%   \end{gathered}
% \end{align*}

% Using \refdef{def:pomsets-lir}:
% \begin{align*}
%   \begin{gathered}[t]
%     %     \PR{x}{s}\SEMI\PW{x}{s{+}r}
%     \PR{x}{s}
%     \\
%     \hbox{\begin{tikzinline}[node distance=.5em and 1em]
%       \event{a}{\DR{x}{2}}{}
%       \xform{xi}{(2{=}s\lor x{=}s)\limplies \bForm}{above=of a}
%       \xform{xd}{2{=}s \limplies \bForm}{below=of a}
%       \xo{a}{xd}
%     \end{tikzinline}}
%   \end{gathered}
%   &&
%   \begin{gathered}[t]
%     \PW{x}{s{+}r}
%     \\
%     \hbox{\begin{tikzinline}[node distance=.5em and 1em]
%       \event{a}{(s{+}r){=}7\mid\DW{x}{7}}{}
%       \xform{xi}{\bForm[s{+}r/x]}{above=of a}
%       \xform{xd}{\bForm[s{+}r/x]}{below=of a}
%       \xo{a}{xd}
%     \end{tikzinline}}
%   \end{gathered}
% \end{align*}
% Composing

% For example, let $\aCmd_1=\PR{x}{r}$ and $\aCmd_2=\PW{x}{r{+}1}$ and
% $\aCmd=\aCmd_1\SEMI \aCmd_2$.
% \begin{align*}
%   \fwp{\aCmd_2}{x{>}1}&=(r{+}1{>}1) = (r{>}0)
%   \\
%   \fwp{\aCmd_1}{r{>}0}=\fwp{\aCmd_0}{x{>}1}&=(x{>}0)
% \end{align*}
% Let $\aPS_i\in\sem{\aCmd_i}$.
% \begin{align*}
%   \aTr[2]{\aEvs_2}{x{>}1}&=(r{+}1{>}1) = (r{>}0)
%   \\
%   \aTr[0]{\aEvs_0}{x{>}1}&=(0{=}\aReg \limplies r{>}0)
%   \\
%   \aTr[0]{\aEvs_0}{x{>}1}&=(1{=}\aReg \limplies r{>}0)
%   \\
%   \aTr[0]{\aEvs_0}{x{>}1}&=(2{=}\aReg \limplies r{>}0)
% \end{align*}

% \begin{proposition}
%   If $\aPS\in\sem{\aCmd}$ is top-level and quiescent then 
%   $\aTr{\aEvs}{\aForm}$ implies $\fwp{\aCmd}{\aForm}$.

%   For any substitution $\aSub=[{\aVal_1/\aReg_1},\ldots, {\aVal_n/\aReg_n}]$ there is some
%   $\aPS\in\sem{\aCmd}$ %that is top-level and quiescent
%   such that all preconditions in $\aPS\aSub$ are tautologies then 
%   $\fwp{\aCmd}{\aForm}\aSub$
% \end{proposition}

% \subsection{Fences}
% \label{sec:fences}

\begin{figure*}[t]
  \showRAtrue
  \begin{center}
    \begin{minipage}{.91\textwidth}
      \renewcommand{\cEvs}{D}
\renewcommand{\dEvs}{D}
\noindent
If $\aPS \in \sSTORE[\amode]{\aLoc}{\aExp}$ then
$(\exists\aVal:\aEvs\fun\Val)$
$(\exists\cForm:\aEvs\fun\Formulae)$
\begin{enumerate}
\item[{\labeltext[S1]{S1)}{S1no-q-or-addr}}] 
  if $\cForm_\bEv\land\cForm_\aEv$ is satisfiable then $\bEv=\aEv$,
\item[{\labeltext[S2]{S2)}{S2no-q-or-addr}}] 
  $\labelingAct(\aEv) = \DW[\amode]{\aLoc}{\aVal_\aEv}$,
\item[{\labeltext[S3]{S3)}{S3no-q-or-addr}}] 
  $\labelingForm(\aEv)$ implies
  \begin{math}
    \cForm_\aEv
    \land \QS{}{\amode}
    \land \aExp{=}\aVal_\aEv
  \end{math},
  
  
\item[{\labeltext[S4]{S4)}{S4no-q-or-addr}}] 
  \begin{math}
    (\forall\aEv\in\aEvs\cap\bEvs)
  \end{math}
  $\aTr{\bEvs}{\bForm}$ implies 
  \begin{math}
    \cForm_\aEv
    \limplies {
      \bForm
      [\aExp/\aLoc]
      \DS{\aLoc}{\amode}
      [(\Q{}\land\aExp{=}\aVal_\aEv)/\Q{}]
    }
  \end{math},
\item[{\labeltext[S5]{S5)}{S5no-q-or-addr}}] 
  \begin{math}    
    (\forall\aEv\in\aEvs\setminus\cEvs)
  \end{math}
  $\aTr{\cEvs}{\bForm}$ implies
  \begin{math}
    \cForm_\aEv
    \limplies {
      \bForm
      [\aExp/\aLoc]
      \DS{\aLoc}{\amode}
      [\FALSE/\Q{}]
    }.
  \end{math}
% \item[{\labeltext[S6]{S6)}{S6no-q-or-addr}}] 
%   $\aTr{\dEvs}{\bForm}$ implies
%   \begin{math}
%     (\!\not\exists\aEv\in\aEvs \suchthat \cForm_\aEv)
%     \limplies {
%       \bForm
%       [\aExp/\aLoc]
%       \DS{\aLoc}{\amode}
%       [\FALSE/\Q{}]
%     }.
%   \end{math}
\end{enumerate}

\noindent
If $\aPS \in \sLOAD[\amode]{\aReg}{\aLoc}$ then
$(\exists\aVal:\aEvs\fun\Val)$
$(\exists\cForm:\aEvs\fun\Formulae)$
$(\exists\bmode\in\{\amode,\mRLX\})$

\begin{enumerate}
\item[{\labeltext[L1]{L1)}{L1no-q-or-addr}}] 
  if $\cForm_\bEv\land\cForm_\aEv$ is satisfiable then $\bEv=\aEv$,
\item[{\labeltext[L2]{L2)}{L2no-q-or-addr}}] 
  $\labelingAct(\aEv) = \DR[\bmode]{\aLoc}{\aVal_\aEv}$
\item[{\labeltext[L3]{L3)}{L3no-q-or-addr}}] 
  $\labelingForm(\aEv)$ implies
  \begin{math}
    \cForm_\aEv
    \land \QL{}{\amode}
  \end{math},
    
\item[{\labeltext[L4]{L4)}{L4no-q-or-addr}}] 
  \begin{math}
    (\forall\aEv\in\aEvs\cap\bEvs)
  \end{math}
  $\aTr{\bEvs}{\bForm}$ implies
  \begin{math}
    \cForm_\aEv
    \limplies \aVal_\aEv{=}\uReg{\aEv}
    \limplies \bForm[\uReg{\aEv}/\aReg]
  \end{math},
  
\item[{\labeltext[L5]{L5)}{L5no-q-or-addr}}] 
  \begin{math}
    (\forall\aEv\in\aEvs\setminus\cEvs)
  \end{math}
  $\aTr{\cEvs}{\bForm}$ implies
  \begin{math}
    \cForm_\aEv 
    \limplies
    \DLX{\aLoc}{\amode}{\bmode}
    \land
    \PBRbig{
      \ABRbig{
        \aVal_\aEv{=}\uReg{\aEv}
        \lor
        \PBR{
          \RW\land
          \aLoc{=}\uReg{\aEv}
        }
      }
      \limplies
      \bForm
      [\uReg{\aEv}/\aReg]
      [\FALSE/\Q{}]
    }    
  \end{math},
\item[{\labeltext[L6]{L6)}{L6no-q-or-addr}}] 
  \begin{math}
    (\forall\bReg)
  \end{math}
  $\aTr{\dEvs}{\bForm}$  implies 
  \begin{math}
    (\!\not\exists\aEv\in\aEvs \suchthat \cForm_\aEv)
    \limplies \PBR{        
      \DLX{\aLoc}{\amode}{\bmode} \land
      \bForm
      [\bReg/\aReg]
      [\FALSE/\Q{}]
    }.
  \end{math}  
\end{enumerate}  





















































    \end{minipage}
  \end{center}
  \caption{Simplified Quiescence Semantics w/o Address Calculation
    (See %\refdef{def:QSx} for $\QS{}{\amode}$, $\QL{}{\amode}$, and
    \refdef{def:dlx} for $\DLX{\aLoc}{\amode}{\bmode}$, $\DS{\aLoc}{\amode}$)
  } 
  \label{fig:no-q-or-addr}
\end{figure*}    

\subsection{Coherence/Synchronization via Reordering}
\label{sec:independency-ra}
\begin{scope}
  \showRAtrue

  In \textsection\ref{sec:sync}, we encoded coherence and synchronized access
  using quiescence symbols.  Building on the language with $\FORKJOIN{}$, it
  is possible to model these using reorderability
  (\textsection\ref{sec:pomsets}), rather than encoding them in the logic.
  With synchronization, the relationship becomes asymmetric.

  To capture completion, we use a single quiescence symbol: $\Q{}$.


  Update actions to include access modes: $\DWP[\amode]{\aLoc}{\aVal}$ and
  $\DRP[\amode]{\aLoc}{\aVal}$.
  Reorderability of two sequential actions is determined, in part, by the modes
  of the two actions, capturing synchronization:
  \begin{center}
    \begin{tabular}{c|ccc|ccc}
      &  \multicolumn{6}{|c}{$2^{\text{nd}}$} \\
      \hline
      $1^{\text{st}}$
      & $\DR[\mRLX]{}{}$ & $\DR[\mACQ]{}{}$ & $\DR[\mSC]{}{}$ & $\DW[\mRLX]{}{}$ & $\DW[\mREL]{}{}$ & $\DW[\mSC]{}{}$\\
      \hline
      $\DR[\mRLX]{}{}$ & \cmark           & \cmark          & \cmark          & \cmark           & \xmark          & \xmark         \\
      $\DR[\mACQ]{}{}$  & \xmark           & \xmark          & \xmark          & \xmark           & \xmark          & \xmark         \\
      $\DR[\mSC]{}{}$  & \xmark           & \xmark          & \xmark          & \xmark           & \xmark          & \xmark         \\
      \hline
      $\DW[\mRLX]{}{}$ & \cmark           & \cmark          & \cmark          & \cmark           & \xmark          & \xmark         \\
      $\DW[\mREL]{}{}$  & \cmark           & \cmark          & \cmark          & \cmark           & \xmark          & \xmark         \\
      $\DW[\mSC]{}{}$  & \cmark           & \cmark          & \xmark          & \cmark           & \xmark          & \xmark 
    \end{tabular}
  \end{center}

  % Least permissive for fences:
  \begin{comment}
    \showRAtrue
    \setlength{\tabcolsep}{4pt}
    \begin{tabular}{c|cccc|cccc|c}
      &  \multicolumn{9}{|c}{$2^{\text{nd}}$} \\
      \hline
      $1^{\text{st}}$
      & $\DR[\mRLX]{}{}$ & $\DR[\mACQ]{}{}$ & $\DR[\mSC]{}{}$ & $\DF{\fACQ}$ & $\DW[\mRLX]{}{}$ & $\DW[\mREL]{}{}$ & $\DW[\mSC]{}{}$ & $\DF{\fREL}$&$\DF{\mSC}$\\
      \hline                                                                                                                                                      
      $\DR[\mRLX]{}{}$ & \cmark           & \cmark          & \cmark          & \xmark       & \cmark           & \xmark          & \xmark          & \xmark      & \xmark    \\
      $\DR[\mACQ]{}{}$  & \xmark           & \xmark          & \xmark          & \xmark       & \xmark           & \xmark          & \xmark          & \xmark      & \xmark    \\
      $\DR[\mSC]{}{}$  & \xmark           & \xmark          & \xmark          & \xmark       & \xmark           & \xmark          & \xmark          & \xmark      & \xmark    \\
      $\DF{\fACQ}$     & \xmark           & \xmark          & \xmark          & \xmark       & \xmark           & \xmark          & \xmark          & \xmark      & \xmark    \\
      \hline                                                                                                                                                      
      $\DW[\mRLX]{}{}$ & \cmark           & \cmark          & \cmark          & \cmark       & \cmark           & \xmark          & \xmark          & \xmark      & \xmark    \\
      $\DW[\mREL]{}{}$  & \cmark           & \cmark          & \cmark          & \cmark       & \cmark           & \xmark          & \xmark          & \xmark      & \xmark    \\
      $\DW[\mSC]{}{}$  & \cmark           & \cmark          & \xmark          & \cmark       & \cmark           & \xmark          & \xmark          & \xmark      & \xmark    \\
      $\DF{\fREL}$     & \cmark           & \cmark          & \cmark          & \cmark       & \xmark           & \xmark          & \xmark          & \xmark      & \xmark    \\
      \hline                                                                                                                                                      
      $\DF{\mSC}$      & \xmark           & \xmark          & \xmark          & \xmark       & \xmark           & \xmark          & \xmark          & \xmark      & \xmark 
    \end{tabular}
  \end{comment}
  It seems that fences generally do not commute, except for road-motel.
  \begin{center}
    \showRAtrue
    \setlength{\tabcolsep}{4pt}
    \begin{tabular}{c|ccc|ccc|ccc}
      &  \multicolumn{9}{|c}{$2^{\text{nd}}$} \\
      \hline
      $1^{\text{st}}$
      & $\DR[\mRLX]{}{}$ & $\DR[\mACQ]{}{}$ & $\DR[\mSC]{}{}$ & $\DW[\mRLX]{}{}$ & $\DW[\mREL]{}{}$ & $\DW[\mSC]{}{}$ & $\DF{\fREL}$&$\DF{\fACQ}$ &$\DF{\fSC}$\\
      \hline                                                                                                                                                     
      $\DR[\mRLX]{}{}$ & \cmark           & \cmark          & \cmark          & \cmark           & \xmark          & \xmark          & \xmark      &\xmark       & \xmark    \\
      $\DR[\mACQ]{}{}$  & \xmark           & \xmark          & \xmark          & \xmark           & \xmark          & \xmark          & \xmark      &\xmark       & \xmark    \\
      $\DR[\mSC]{}{}$  & \xmark           & \xmark          & \xmark          & \xmark           & \xmark          & \xmark          & \xmark      &\xmark       & \xmark    \\
      \hline                                                                                                                                                     
      $\DW[\mRLX]{}{}$ & \cmark           & \cmark          & \cmark          & \cmark           & \xmark          & \xmark          & \xmark      &\cmark       & \xmark    \\
      $\DW[\mREL]{}{}$  & \cmark           & \cmark          & \cmark          & \cmark           & \xmark          & \xmark          & \xmark      &\cmark       & \xmark    \\
      $\DW[\mSC]{}{}$  & \cmark           & \cmark          & \xmark          & \cmark           & \xmark          & \xmark          & \xmark      &\cmark       & \xmark    \\
      \hline                                                                                                                                                     
      $\DF{\fREL}$     & \cmark           & \cmark          & \cmark          & \xmark           & \xmark          & \xmark          & \xmark      &\cmark       & \xmark    \\
      $\DF{\fACQ}$     & \xmark           & \xmark          & \xmark          & \xmark           & \xmark          & \xmark          & \xmark      &\xmark       & \xmark    \\
      $\DF{\fSC}$     & \xmark           & \xmark          & \xmark          & \xmark           & \xmark          & \xmark          & \xmark      &\xmark       & \xmark 
    \end{tabular}
  \end{center}
  \begin{definition}
    \label{def:reorderra}
    Including coherence, \emph{reorderability} is defined:
    \begin{align*}
      {\reorderco}
      &=
      \{(\DW{\aLoc}{}, \DR{\bLoc}{}) \mid \aLoc\neq\bLoc\}
      \cup\{(\DW{\aLoc}{}, \DW{\bLoc}{}) \mid \aLoc\neq\bLoc\}
      \\&
      \cup\{(\DR{\aLoc}{}, \DW{\bLoc}{}) \mid \aLoc\neq\bLoc\}
      \cup\{(\DR{\aLoc}{}, \DR{\bLoc}{}) \}
      \\
      {\reorderra}
      &=
      \{(\DW[\amode]{}{}, \DR[\bmode]{}{}) \mid \amode\neq\mSC \lor \bmode\neq\mSC\}
      \cup\{(\DW[\amode]{}{}, \DW[\mRLX]{}{} \} %\mid \amode\neq\fREL \}
      \\&
      \cup\{(\DR[\amode]{}{}, \DW[\bmode]{}{}) \mid \amode=\mRLX \land \bmode=\mRLX\}
      \cup\{(\DR[\mRLX]{}{},  \DR[\bmode]{}{}) \} %\mid \bmode\neq\fACQ\}
      \\&
      \cup\{(\DF{\fREL},      \DF{\fACQ}    ) \}
      \cup\{(\DF{\fREL},      \DR[\bmode]{}{}) \}%\mid \amode=\fREL \}
      \cup\{(\DW[\amode]{}{}, \DF{\fACQ}     ) \}%\mid \bmode=\fACQ \}
      \\
      {\reorderlt}
      &=
      {\reorderra}\cap{\reorderco} %\{(\aAct, \bAct) \mid \aAct\reorderco\bAct \land \aAct\reorderra\bAct\}
    \end{align*}  
  \end{definition}
  Here is the version with generalized modes for read and write:  
  \begin{scope}
    \small
    \begin{align*}
      {\reorderra}
      &=
      \{(\DW[\amode]{}{}, \DR[\bmode]{}{}) \mid \amode\not\gemode\mSC \lor \bmode\not\gemode\mSC\}
      \cup\{(\DW[\amode]{}{}, \DW[\mRLX]{}{} \mid \amode\not\gemode\fREL \}
      \\&
      \cup\{(\DR[\amode]{}{}, \DW[\bmode]{}{}) \mid \amode=\mRLX \land \bmode=\mRLX\}
      \cup\{(\DR[\mRLX]{}{},  \DR[\bmode]{}{}) \mid \bmode\not\gemode\fACQ \}
      \\&
      \cup\{(\DF{\fREL},      \DF{\fACQ}    ) \}
      \cup\{(\DF{\fREL},      \DR[\bmode]{}{}) \}%\mid \amode=\fREL \}
      \cup\{(\DW[\amode]{}{}, \DF{\fACQ}     ) \}%\mid \bmode=\fACQ \}
    \end{align*}  
  \end{scope}
  \begin{center}
    \footnotesize
    \showRAtrue
    \setlength{\tabcolsep}{4pt}
    \begin{tabular}{c|ccccc|ccccc}
      &  \multicolumn{9}{|c}{$2^{\text{nd}}$} \\
      \hline
      $1^{\text{st}}$
                       & $\DR[\mRLX]{}{}$  & $\DR[\mACQ]{}{}$ & $\DR[\fACQ]{}{}$& $\DR[\mSC]{}{}$  & $\DR[\fSC]{}{}$ & $\DW[\mRLX]{}{}$& $\DW[\mREL]{}{}$ & $\DW[\fREL]{}{}$& $\DW[\mSC]{}{}$& $\DW[\fSC]{}{}$\\% & $\DF{\fREL}$&$\DF{\fACQ}$ &$\DF{\fSC}$\\
      \hline                                                                                                                                                                                                                            
      $\DR[\mRLX]{}{}$ & \cmark            & \cmark          & \xmark          & \cmark           & \xmark           & \cmark          & \xmark          & \xmark          & \xmark         & \xmark         \\%  & \xmark      &\xmark       & \xmark    \\
      $\DR[\mACQ]{}{}$  & \xmark            & \xmark          & \xmark          & \xmark           & \xmark           & \xmark          & \xmark          & \xmark          & \xmark         & \xmark         \\%  & \xmark      &\xmark       & \xmark    \\
      $\DR[\fACQ]{}{}$ & \xmark            & \xmark          & \xmark          & \xmark           & \xmark           & \xmark          & \xmark          & \xmark          & \xmark         & \xmark         \\%  & \xmark      &\xmark       & \xmark    \\
      $\DR[\mSC]{}{}$  & \xmark            & \xmark          & \xmark          & \xmark           & \xmark           & \xmark          & \xmark          & \xmark          & \xmark         & \xmark         \\%  & \xmark      &\xmark       & \xmark    \\
      $\DR[\fSC]{}{}$ & \xmark            & \xmark          & \xmark          & \xmark           & \xmark           & \xmark          & \xmark          & \xmark          & \xmark         & \xmark         \\%  & \xmark      &\xmark       & \xmark    \\
      \hline                                                                                                                                                                                                                           
      $\DW[\mRLX]{}{}$ & \cmark            & \cmark          & \cmark          & \cmark           & \xmark           & \cmark          & \xmark          & \xmark          & \xmark         & \xmark         \\%  & \xmark      &\cmark       & \xmark    \\
      $\DW[\mREL]{}{}$  & \cmark            & \cmark          & \cmark          & \cmark           & \xmark           & \cmark          & \xmark          & \xmark          & \xmark         & \xmark         \\%  & \xmark      &\cmark       & \xmark    \\
      $\DW[\fREL]{}{}$ & \cmark            & \cmark          & \cmark          & \cmark           & \xmark           & \xmark          & \xmark          & \xmark          & \xmark         & \xmark         \\%  & \xmark      &\cmark       & \xmark    \\
      $\DW[\mSC]{}{}$  & \cmark            & \cmark          & \cmark          & \xmark           & \xmark           & \cmark          & \xmark          & \xmark          & \xmark         & \xmark         \\%  & \xmark      &\cmark       & \xmark    \\
      $\DW[\fSC]{}{}$ & \xmark            & \xmark          & \xmark          & \xmark           & \xmark           & \xmark          & \xmark          & \xmark          & \xmark         & \xmark         \\%  & \xmark      &\cmark       & \xmark    \\
      % \hline                                                                                                                                                                                       
      % $\DF{\fREL}$     & \cmark            & \cmark          & \cmark          & \cmark           & \xmark          & \xmark          & \xmark          & \xmark          & \xmark      &\cmark       & \xmark    \\
      % $\DF{\fACQ}$     & \xmark            & \xmark          & \xmark          & \xmark           & \xmark          & \xmark          & \xmark          & \xmark          & \xmark      &\xmark       & \xmark    \\
      % $\DF{\fSC}$     & \xmark            & \xmark          & \xmark          & \xmark           & \xmark          & \xmark          & \xmark          & \xmark          & \xmark      &\xmark       & \xmark 
    \end{tabular}
  \end{center}
  \begin{comment}
    \label{def:QSx}
    Let formulae $\QS{}{\amode}$ and $\QL{}{\amode}$ be defined:
    \begin{align*}
      \QS{}{\amode}&=
      \begin{cases}
        \Q{} & \textif \amode\neq\mRLX
        \\
        \TRUE & \textotherwise
      \end{cases}
      &
      \QL{}{\amode}&=
      \begin{cases}
        \Q{} & \textif \amode=\mSC
        \\
        \TRUE & \textotherwise
      \end{cases}
    \end{align*}
  \end{comment}
  
  \begin{definition}
    \label{def:independency-co}
    \noindent
    If $\aPS \in \sSEMI{\aPSS_1}{\aPSS_2}$ then $(\exists\aPS_1\in\aPSS_1)$
    $(\exists\aPS_2\in\aPSS_2)$
    % there are $\aPS_1\in\aPSS_1$ and $\aPS_2\in\aPSS_2$ such that
    % let
    % $\labelingForm'_2(\aEv)=\aTr[1]{{\downclose[0]{\aEv}}}{\labelingForm_2(\aEv)}$
    % let
    % $\labelingForm'_2(\aEv)=\aTr[1]{\{ \bEv \mid \bEv < \aEv
    % \}}{\labelingForm_2(\aEv})$
    \begin{enumerate}
      \setcounter{enumi}{\value{pomsetXSemiCount}}
    \item[1--\ref{seq-tau})] as for $\sSEMI{}{}$ in \refdef{def:pomsets-trans},
    \item
      \label{seq-reorder} if $\bEv\in\aEvs_1$ and $\aEv\in\aEvs_2$ either
      $\bEv\le\aEv$ or $a\reorderlt\labeling_2(\aEv)$.
    \end{enumerate}
    Update the read and write rules of \refdef{def:pomsets-trans} to: %and \ref{def:pomsets-fj} to: % (\ref{S4}/\ref{L4} unchanged):
    \begin{enumerate}
    \item[\ref{S2})]
      $\labelingAct(\aEv) = \DW[\amode]{\aLoc}{\aVal}$,
    \item[\ref{S3})] 
      $\labelingForm(\aEv) \rimplies \aExp{=}\aVal$,
    \item[\ref{S4})]
      $\aTr{\bEvs}{\bForm} \rimplies \bForm \land\aExp{=}\aVal$,
    \item[\ref{S5})]
      $\aTr{\cEvs}{\bForm} \rimplies \bForm[\FALSE/\Q{}]$,
    \item[\ref{L2})]
      $\labelingAct(\aEv) = \DR[\amode]{\aLoc}{\aVal}$,
    \item[\ref{L3})] 
      $\labelingForm(\aEv) \rimplies \TRUE$.
    \item[\ref{L4})]
      $\aTr{\bEvs}{\bForm} \rimplies \aVal{=}\aReg\limplies\bForm$, 
    \item[\ref{L5})]
      $\aTr{\cEvs}{\bForm} \rimplies \bForm[\FALSE/\Q{}]$.
    \end{enumerate}
  \end{definition}

  \begin{comment}
    The logic of quiescence is greatly simplified.  Compare the following to
    \refex{ex:q1}.
    \begin{align*}
      \begin{gathered}
        \begin{gathered}[t]
          \PW[\mRLX]{x}{\aExp}
          \\
          \hbox{\begin{tikzinline}[node distance=.5em and 1.5em]
              \event{a}{\aExp{=}v\mid\DW[\mRLX]{x}{v}}{}
              \xform{xi}{\bForm[\FALSE/\Q{}]}{above=of a}
              \xform{xd}{\bForm\land\aExp{=}\aVal}{below=of a}
              \xo{a}{xd}
            \end{tikzinline}}
        \end{gathered}
        \\
        \begin{gathered}[t]
          \PW[\amode\in\{\mREL, \mSC \}]{x}{\aExp}
          \\
          \hbox{\begin{tikzinline}[node distance=.5em and 1.5em]
              \event{a}{\Q{} \land \aExp{=}v\mid\DW[\amode]{x}{v}}{}
              \xform{xi}{\bForm[\FALSE/\Q{}]}{above=of a}
              \xform{xd}{\bForm\land\aExp{=}\aVal}{below=of a}
              \xo{a}{xd}
            \end{tikzinline}}
        \end{gathered}
      \end{gathered}
      &&
      \begin{gathered}
        \begin{gathered}[t]
          \PR[\amode\in\{\mRLX, \mACQ \}]{x}{r}
          \\
          \hbox{\begin{tikzinline}[node distance=.5em and 1.5em]
              \event{a}{\DR[\amode]{x}{v}}{}
              \xform{xi}{\bForm[\FALSE/\Q{}]}{above=of a}
              \xform{xd}{v{=}r\limplies\bForm}{below=of a}
              \xo{a}{xd}
            \end{tikzinline}}
        \end{gathered}
        \\
        \begin{gathered}[t]
          \PR[\mSC]{x}{r}
          \\
          \hbox{\begin{tikzinline}[node distance=.5em and 1.5em]
              \event{a}{\Q{} \mid \DR[\mSC]{x}{v}}{}
              \xform{xi}{\bForm[\FALSE/\Q{}]}{above=of a}
              \xform{xd}{v{=}r\limplies\bForm}{below=of a}
              \xo{a}{xd}
            \end{tikzinline}}
        \end{gathered}
      \end{gathered}
    \end{align*}
  \end{comment}
  \begin{example}
    The logic of quiescence is greatly simplified.  Compare the following to
    \refex{ex:q1}.
    \begin{align*}
      \begin{gathered}[t]
        \PW[\amode]{x}{\aExp}
        \\
        \hbox{\begin{tikzinline}[node distance=.5em and 1.5em]
            \event{a}{\Q{} \land \aExp{=}v\mid\DW[\amode]{x}{v}}{}
            \xform{xi}{\bForm[\FALSE/\Q{}]}{above=of a}
            \xform{xd}{\bForm\land\aExp{=}\aVal}{below=of a}
            \xo{a}{xd}
          \end{tikzinline}}
      \end{gathered}
      &&
      \begin{gathered}[t]
        \PR[\amode]{x}{r}
        \\
        \hbox{\begin{tikzinline}[node distance=.5em and 1.5em]
            \event{a}{\DR[\amode]{x}{v}}{}
            \xform{xi}{\bForm[\FALSE/\Q{}]}{above=of a}
            \xform{xd}{v{=}r\limplies\bForm}{below=of a}
            \xo{a}{xd}
          \end{tikzinline}}
      \end{gathered}
    \end{align*}
  \end{example}

  \reffig{fig:no-q-or-addr} shows the resulting full semantics of read and
  write, without address calculation.  \reffig{fig:no-q} shows the resulting
  full semantics with address calculation.

  \begin{definition}
    \label{def:dlx}
    % Let $[\aForm/\Dx{*}]$ substitute $\aForm$ for every $\Dx{\bLoc}$.

    \noindent
    Let substitution $\DS{\aLoc}{\amode}$ be defined:
    \begin{displaymath}
      \DS{\aLoc}{\amode}=
      \begin{cases}
        [\TRUE/\Dx{\aLoc}] & \textif \amode=\mRLX
        \\
        [\FALSE/\Dx{*}] & \textotherwise
      \end{cases}
    \end{displaymath}
    Let formula $\DLX{\aLoc}{\amode}{\bmode}$ be defined:
    \begin{displaymath}
      \DLX{\aLoc}{\amode}{\bmode}=
      \begin{cases}
        \TRUE & \textif \amode=\mRLX \textor \amode=\bmode
        \\
        \Dx{\aLoc} & \textotherwise
      \end{cases}
    \end{displaymath}
  \end{definition}
\end{scope}

\begin{figure*}
  \showRAtrue
  \begin{center}
    \begin{minipage}{.91\textwidth}
      \renewcommand{\cEvs}{D}
\renewcommand{\dEvs}{D}
\noindent
If $\aPS \in \sSTORE[\amode]{\cExp}{\aExp}$ then
$(\exists\cVal:\aEvs\fun\Val)$
$(\exists\aVal:\aEvs\fun\Val)$
$(\exists\cForm:\aEvs\fun\Formulae)$
\begin{enumerate}
\item[{\labeltext[S1]{S1)}{S1no-q}}] % [\ref{S1})]
  if $\cForm_\bEv\land\cForm_\aEv$ is satisfiable then $\bEv=\aEv$,
\item[{\labeltext[S2]{S2)}{S2no-q}}] %[\ref{S2})]
  $\labelingAct(\aEv) = \DWREF[\amode]{\cVal_\aEv}{\aVal_\aEv}$,
\item[{\labeltext[S3]{S3)}{S3no-q}}] %[\ref{S3})] 
  $\labelingForm(\aEv)$ implies
  \begin{math}
    \cForm_\aEv
    % \land \RW
    \land \cExp{=}\cVal_\aEv
    \land \aExp{=}\aVal_\aEv
  \end{math},
\item[{\labeltext[S4]{S4)}{S4no-q}}] %[\ref{S4})]
  \begin{math}
    (\forall\aEv\in\aEvs\cap\bEvs)
  \end{math}
  $\aTr{\bEvs}{\bForm}$ implies 
  \begin{math}
    \cForm_\aEv
    %\limplies (\cExp{=}\cVal_\aEv)
    \limplies {
      \bForm
      [\aExp/\REF{\cVal_\aEv}]
      \DS{\REF{\cVal_\aEv}}{\amode}
      [(\Q{}\land\aExp{=}\aVal_\aEv\land\cExp{=}\cVal_\aEv)/\Q{}]
    }
  \end{math},
\item[{\labeltext[S5]{S5)}{S5no-q}}] %[\ref{S5})] 
  \begin{math}    
    (\forall\aEv\in\aEvs\setminus\cEvs)
  \end{math}
  $\aTr{\cEvs}{\bForm}$ implies
  \begin{math}
    \cForm_\aEv
    %\limplies (\cExp{=}\cVal_\aEv)
    \limplies {
      \bForm
      [\aExp/\REF{\cVal_\aEv}]
      \DS{\REF{\cVal_\aEv}}{\amode}
      [\FALSE/\Q{}]
    },
  \end{math}
\item[{\labeltext[S6]{S6)}{S6no-q}}] %[S6)]%\ref{S6})] 
  \begin{math}
    (\forall\dVal)
  \end{math}
  $\aTr{\dEvs}{\bForm}$ implies
  \begin{math}
    (\!\not\exists\aEv\in\aEvs \suchthat \cForm_\aEv)
    %(\!\not\exists\aEv\in\aEvs\cap\cEvs \suchthat \cForm_\aEv)
    \limplies (\cExp{=}\dVal)
    \limplies {
      \bForm
      [\aExp/\REF{\dVal}]
      \DS{\REF{\dVal}}{\amode}
      [\FALSE/\Q{}]
    }.
  \end{math}
% \item[S5-6)]%\ref{S6})] 
%   \begin{math}
%     (\forall\dVal)
%   \end{math}
%   $\aTr{\cEvs}{\bForm}$ implies
%   \begin{math}
%     %(\!\not\exists\aEv\in\aEvs \suchthat \cForm_\aEv)
%     (\!\not\exists\aEv\in\aEvs\cap\cEvs \suchthat \cForm_\aEv)
%     \limplies (\cExp{=}\dVal)
%     \limplies \PBR{
%       \bForm
%       [\aExp/\REF{\dVal}]
%       \DS{\REF{\dVal}}{\amode}
%       [\FALSE/\QS{\REF{\dVal}}{\amode}]
%     }.
%   \end{math}
  % \\ where 
  % $\DS{}{\mRLX}{}=[\TRUE/\DxREF{\dVal}]$ and otherwise
  % $\DS{}{\amode}{}=[\FALSE/\D]$. % for $\amode\neq\mRLX$.
\end{enumerate}
% \item if $\amode=\mRLX$ then
%   $\labelingForm(\aEv)$ implies
%   \begin{math}
%     \cForm_\aEv
%     \land \cExp{=}\cVal_\aEv
%     \land \aExp{=}\aVal_\aEv
%     \land \RW
%     \land \QxREF{\cVal_\aEv},
%   \end{math}
% \item if $\amode\neq\mRLX$ then
%   $\labelingForm(\aEv)$ implies
%   \begin{math}
%     \cForm_\aEv
%     \land \cExp{=}\cVal_\aEv
%     \land \aExp{=}\aVal_\aEv
%     \land \RW
%     \land \Q{},
%   \end{math}
% \item if
%   $\aEv\in\bEvs$
%   and
%   $\amode=\mRLX$ then
%   \begin{math}
%     (\forall\dVal)
%   \end{math}
%   $\aTr{\bEvs}{\bForm}$ implies 
%   \begin{math}
%     \cForm_\aEv
%     \limplies (\cExp{=}\dVal)
%     \limplies \PBRbig{
%     (\QwREF{\dVal} \limplies \aExp{=}\aVal_\aEv)
%     \land \bForm[\aExp/\REF{\dVal}][\TRUE/\DxREF{\dVal}]
%   }
%   \end{math}
% \item if
%   $\aEv\in\bEvs$
%   and
%   $\amode\neq\mRLX$ then
%   \begin{math}
%     (\forall\dVal)
%   \end{math}
%   $\aTr{\bEvs}{\bForm}$ implies 
%   \begin{math}
%     \cForm_\aEv
%     \limplies (\cExp{=}\dVal)
%     \limplies \PBRbig{
%     (\QwREF{\dVal} \limplies \aExp{=}\aVal_\aEv)
%     \land \bForm[\aExp/\REF{\dVal}][\FALSE/\D]
%   }
%   \end{math}
% \item if 
%   \begin{math}
%     (\forall\aEv\in\bEvs)(\cForm \textimplies
%     \lnot\cForm_\aEv)
%   \end{math}
%   and $\amode=\mRLX$ 
%   then
%   \begin{math}
%     (\forall\dVal)
%   \end{math}
%   $\aTr{\bEvs}{\bForm}$ implies 
%   \begin{math}
%     \cForm
%     \limplies (\cExp{=}\dVal)
%     \limplies \PBRbig{
%     \lnot\QwREF{\dVal}
%     \land \bForm[\aExp/\REF{\dVal}][\TRUE/\DxREF{\dVal}]
%   }
%   \end{math}
% \item if 
%   \begin{math}
%     (\forall\aEv\in\bEvs)
%     (\cForm \textimplies \lnot\cForm_\aEv)
%   \end{math}
%   and $\amode\neq\mRLX$ 
%   then
%   \begin{math}
%     (\forall\dVal)
%   \end{math}
%   $\aTr{\bEvs}{\bForm}$ implies 
%   \begin{math}
%     \cForm
%     \limplies (\cExp{=}\dVal)
%     \limplies \PBRbig{
%     \lnot\QwREF{\dVal}
%     \land \bForm[\aExp/\REF{\dVal}][\FALSE/\D]
%   }
%   \end{math}

\noindent
If $\aPS \in \sLOAD[\amode]{\aReg}{\cExp}$ then
$(\exists\cVal:\aEvs\fun\Val)$
$(\exists\aVal:\aEvs\fun\Val)$
$(\exists\cForm:\aEvs\fun\Formulae)$
$(\exists\bmode\in\{\amode,\mRLX\})$
% $(\forall\uReg{\aEv}\in\uRegs{\aEvs})$
\begin{enumerate}
\item[{\labeltext[L1]{L1)}{L1no-q}}] %[\ref{L1})]
  if $\cForm_\bEv\land\cForm_\aEv$ is satisfiable then $\bEv=\aEv$,
\item[{\labeltext[L2]{L2)}{L2no-q}}] %[\ref{L2})]
  $\labelingAct(\aEv) = \DRREF[\bmode]{\cVal_\aEv}{\aVal_\aEv}$,
\item[{\labeltext[L3]{L3)}{L3no-q}}] %[\ref{L3})]
  $\labelingForm(\aEv)$ implies
  \begin{math}
    \cForm_\aEv
    % \land \RO
    \land \cExp{=}\cVal_\aEv
  \end{math},
  % where    
  % $\QL{}{\mSC}=\Q{\mSC}$ and otherwise $\QL{}{\amode}=\QwREF{\cVal_\aEv}$, % for $\amode\neq\mRLX$,
\item[{\labeltext[L4]{L4)}{L4no-q}}] %[\ref{L4})]
  \begin{math}
    (\forall\aEv\in\aEvs\cap\bEvs)
  \end{math}
  $\aTr{\bEvs}{\bForm}$ implies
  \begin{math}
    \cForm_\aEv
    \limplies (\cExp{=}\cVal_\aEv\limplies\aVal_\aEv{=}\uReg{\aEv})
    \limplies \bForm[\uReg{\aEv}/\aReg]
  \end{math},
  %\makebox[6.2cm]{}
\item[{\labeltext[L5]{L5)}{L5no-q}}] %[\ref{L5})] 
  \begin{math}
    (\forall\aEv\in\aEvs\setminus\cEvs)
  \end{math}
  $\aTr{\cEvs}{\bForm}$ implies
  \begin{math}
    \cForm_\aEv 
    \limplies
    \DLX{\REF{\cVal_\aEv}}{\amode}{\bmode}
    \land
    \PBRbig{
      \ABRbig{
        \PBR{\cExp{=}\cVal_\aEv\limplies\aVal_\aEv{=}\uReg{\aEv}}
        \lor
        \PBR{
          \RW\land
          \PBR{\cExp{=}\cVal_\aEv\limplies\REF{\cVal_\aEv}{=}\uReg{\aEv}}
        }
      }
      \limplies
      \bForm
      [\uReg{\aEv}/\aReg]
      [\FALSE/\Q{}]
    }    
  \end{math},
\item[{\labeltext[L6]{L6)}{L6no-q}}] %[\ref{L6})] 
  \begin{math}
    (\forall\dVal)
    (\forall\bReg)
  \end{math}
  $\aTr{\dEvs}{\bForm}$  implies 
  \begin{math}
    (\!\not\exists\aEv\in\aEvs \suchthat \cForm_\aEv)
    \limplies (\cExp{=}\dVal)
    \limplies \PBR{        
      \DLX{\REF{\dVal}}{\amode}{\bmode} \land
      \bForm
      [\bReg/\aReg]
      [\FALSE/\Q{}]
    }.
  \end{math}
  % \\ where $\DL{}{\mRLX}=\TRUE$ and otherwise $\DL{}{\amode}=\DxREF{\dVal}$.
  % Recall that $\uRegs{\bEvs}=\{\uReg{\aEv}\mid\aEv\in\bEvs\}$.
\end{enumerate}  
% \item if $\amode=\mRLX$ and $\bEv\notin\bEvs$ then
%   \begin{math}
%     (\forall\dVal)
%   \end{math}
%   $\aTr{\bEvs}{\bForm}$ implies
%   \begin{math}
%     \cForm_\bEv
%     \limplies (\cExp{=}\dVal)
%     \limplies \PBRbig{
%     (
%     \RW
%     \limplies (\aVal{=}\uReg{\bEv}\lor\aLoc{=}\uReg{\bEv}) 
%     \limplies \bForm[\uReg{\bEv}/\aReg][\uReg{\bEv}/\REF{\dVal}]
%     )
%     \land \lnot\QxREF{\dVal}
%   }
%     \phantom{\land\; \Dx{\dVal}}
%   \end{math}
% \item if $\amode\neq\mRLX$ and $\bEv\notin\bEvs$ then
%   \begin{math}
%     (\forall\dVal)
%   \end{math}
%   $\aTr{\bEvs}{\bForm}$ implies
%   \begin{math}
%     \cForm_\bEv
%     \limplies (\cExp{=}\dVal)
%     \limplies \PBRbig{
%     (
%     \RW
%     \limplies (\aVal{=}\uReg{\bEv}\lor\aLoc{=}\uReg{\bEv}) 
%     \limplies \bForm[\uReg{\bEv}/\aReg][\uReg{\bEv}/\REF{\dVal}]
%     )
%     \land \lnot\QxREF{\dVal}
%     \land \Dx{\dVal}
%   }
%   \end{math}

% \noindent
% If $\aPS \in \sTHREAD{\aPSS}$ then
% $(\exists\aPS_1\in\aPSS)$
% \begin{enumerate}
% \item[{\labeltext[T1]{T1)}{T1no-q}}] %[\ref{T1})]
%   $\aEvs=\aEvs_1$,
% \item[{\labeltext[T2]{T2)}{T2no-q}}] %[\ref{T2})]
%   $\labelingAct(\aEv) = \labelingAct_1(\aEv)$,
% \item[{\labeltext[T3]{T3)}{T3no-q}}] %[\ref{T3})]
%   $\labelingForm(\aEv)$ implies
%   $\labelingForm_1(\aEv) [\TRUE/\Q{}][\TRUE/\RW]$ if $\labelingAct_1(\aEv)$ is a write,
%   \\
%   $\labelingForm(\aEv)$ implies
%   $\labelingForm_1(\aEv) [\TRUE/\Q{}][\FALSE/\RW]$ otherwise.
% \end{enumerate}  

    \end{minipage}
  \end{center}
  \caption{Simplified Quiescence Semantics with Address Calculation
    (See %\refdef{def:QSx} for $\QS{}{\amode}$, $\QL{}{\amode}$, and
    \refdef{def:dlx} for $\DLX{\aLoc}{\amode}{\bmode}$, $\DS{\aLoc}{\amode}$)
  } 
  \label{fig:no-q}
\end{figure*}    


\subsection{Substitutions}
\label{sec:substitutions}

Recall the load rules from \textsection\ref{sec:tc1}: % (\refdef{def:pomsets-lir}):
\begin{enumerate}
\item[\ref{L4})]
  $\aTr{\bEvs}{\bForm} \rimplies \aVal{=}\aReg\limplies\bForm$, 
\item[\ref{L5})]
  $\aTr{\cEvs}{\bForm} \rimplies (\aVal{=}\aReg\lor\aLoc{=}\aReg)\limplies\bForm$, when $\aEvs\neq\emptyset$,
\item[\ref{L6})] 
  $\aTr{\dEvs}{\bForm}\; \rimplies \bForm$, when $\aEvs=\emptyset$.
\end{enumerate}
It is also possible to collapse $\aLoc$ and $\aReg$ when doing a load:
\begin{enumerate}
\item[\ref{L4})]
  $\aTr{\bEvs}{\bForm} \rimplies \aVal{=}\aReg\limplies\bForm[\aReg/\aLoc]$, 
\item[\ref{L5})]
  $\aTr{\cEvs}{\bForm} \rimplies (\aVal{=}\aReg\lor\aLoc{=}\aReg)\limplies\bForm[\aReg/\aLoc]$, when $\aEvs\neq\emptyset$.
\item[\ref{L6})] 
  $\aTr{\dEvs}{\bForm}\; \rimplies \bForm[\aReg/\aLoc]$, when $\aEvs=\emptyset$.
\end{enumerate}

Perhaps surprisingly, these two semantics are incomparable.  Consider the
following:
\begin{gather*}
  \IF{r\land s\;\mathsf{even}}\THEN \PW{y}{1}\FI\SEMI
  \IF{r\land s}\THEN \PW{z}{1}\FI
  \\
  \hbox{\begin{tikzinline}[node distance=0.5em and 1.5em]
      \event{a3}{r\land s\;\mathsf{even}\mid\DW{y}{1}}{}
      \event{a4}{r\land s\mid\DW{z}{1}}{below=of a3}
    \end{tikzinline}}
\end{gather*}
Prepending $\PRP{x}{s}$, we get the same result regardless of whether we
substitute $[s/x]$, since $x$ does not occur in either precondition.  Here
we show the independent case:
\begin{gather*}
  \PR{x}{s}\SEMI
  \IF{r\land s\;\mathsf{even}}\THEN \PW{y}{1}\FI\SEMI
  \IF{r\land s}\THEN \PW{z}{1}\FI
  \\
  \hbox{\begin{tikzinline}[node distance=0.5em and 1.5em]
      \event{a2}{\DR{x}{2}}{}
      \event{a3}{(2{=}s\lor x{=}s)\limplies (r\land s\;\mathsf{even})\mid\DW{y}{1}}{above right=of a2}
      \event{a4}{(2{=}s\lor x{=}s)\limplies (r\land s)\mid\DW{z}{1}}{below=of a3}
    \end{tikzinline}}
\end{gather*}
Prepending $\PRP{x}{r}$, we now get different results since the
preconditions mention $x$.
Without substitution:
\begin{gather*}
  \PR{x}{r}\SEMI
  \PR{x}{s}\SEMI
  \IF{r\land s\;\mathsf{even}}\THEN \PW{y}{1}\FI\SEMI
  \IF{r\land s}\THEN \PW{z}{1}\FI
  \\
  \hbox{\begin{tikzinline}[node distance=0.5em and 1.5em]
      \event{a1}{\DR{x}{1}}{}
      \event{a2}{\DR{x}{2}}{below=of a1}
      \event{a3}{1{=}r\limplies  (2{=}s\lor x{=}s)\limplies (r\land s\;\mathsf{even})\mid\DW{y}{1}}{right=of a1}
      \event{a4}{1{=}r\limplies  (2{=}s\lor x{=}s)\limplies (r\land s)\mid\DW{z}{1}}{below=of a3}
      \po{a1}{a3}
      \po[out=-20,in=177]{a1}{a4}
    \end{tikzinline}}
\end{gather*}
Prepending $\PWP{x}{0}$, which substitutes $[0/x]$, the precondition of
$\DWP{y}{1}$ becomes
$(1{=}r\limplies (2{=}s\lor0{=}s)\limplies (r\land s\;\mathsf{even}))$,
which is a tautology, whereas the precondition of $\DW{z}{1}$ becomes
$(1{=}r\limplies(2{=}s\lor0{=}s)\limplies (r\land s))$,
which is not.   In order to be top-level, $\DW{z}{1}$ must depend on
$\DR{x}{2}$; in this case the precondition becomes
$(1{=}r\limplies2{=}s\limplies (r\land s))$, which is a tautology.  
\begin{gather*}
  % \PW{x}{0}\SEMI
  % \PR{x}{r}\SEMI
  % \PR{x}{s}\SEMI
  % \IF{r\land s\;\mathsf{even}}\THEN \PW{y}{1}\FI\SEMI
  % \IF{r\land s}\THEN \PW{z}{1}\FI
  % \\
  \hbox{\begin{tikzinline}[node distance=1.5em]
      \event{a0}{\DW{x}{0}}{}
      \event{a1}{\DR{x}{1}}{right=of a0}
      \event{a2}{\DR{x}{2}}{right=of a1}
      \event{a3}{\DW{y}{1}}{right=of a2}
      \event{a4}{\DW{z}{1}}{right=of a3}
      % \wk{a0}{a1}
      % \wk[out=-20,in=-160]{a0}{a2}
      \po[out=20,in=160]{a1}{a3}
      \po[out=20,in=160]{a1}{a4}
      \po[out=-20,in=-160]{a2}{a4}
    \end{tikzinline}}
\end{gather*}
The situation reverses with the substitution $[r/x]$:
\begin{gather*}
  \PR{x}{r}\SEMI
  \PR{x}{s}\SEMI
  \IF{r\land s\;\mathsf{even}}\THEN \PW{y}{1}\FI\SEMI
  \IF{r\land s}\THEN \PW{z}{1}\FI
  \\
  \hbox{\begin{tikzinline}[node distance=0.5em and 1.5em]
      \event{a1}{\DR{x}{1}}{}
      \event{a2}{\DR{x}{2}}{below=of a1}
      \event{a3}{1{=}r\limplies  (2{=}s\lor r{=}s)\limplies (r\land s\;\mathsf{even})\mid\DW{y}{1}}{right=of a1}
      \event{a4}{1{=}r\limplies  (2{=}s\lor r{=}s)\limplies (r\land s)\mid\DW{z}{1}}{below=of a3}
      \po{a1}{a3}
      \po[out=-20,in=177]{a1}{a4}
    \end{tikzinline}}
\end{gather*}
Prepending $\PWP{x}{0}$:
\begin{gather*}
  % \PW{x}{0}\SEMI
  % \PR{x}{r}\SEMI
  % \PR{x}{s}\SEMI
  % \IF{r\land s\;\mathsf{even}}\THEN \PW{y}{1}\FI\SEMI
  % \IF{r\land s}\THEN \PW{z}{1}\FI
  % \\
  \hbox{\begin{tikzinline}[node distance=1.5em]
      \event{a0}{\DW{x}{0}}{}
      \event{a1}{\DR{x}{1}}{right=of a0}
      \event{a2}{\DR{x}{2}}{right=of a1}
      \event{a3}{\DW{y}{1}}{right=of a2}
      \event{a4}{\DW{z}{1}}{right=of a3}
      % \wk{a0}{a1}
      % \wk[out=-20,in=-160]{a0}{a2}
      \po[out=20,in=160]{a1}{a3}
      \po[out=20,in=160]{a1}{a4}
      \po{a2}{a3}
    \end{tikzinline}}
\end{gather*}
The dependency has changed from $\DRP{x}{2}\xpo\DWP{z}{1}$ to
$\DRP{x}{2}\xpo\DWP{y}{1}$.  The resulting sets of pomsets are
incomparable.


Thinking in terms of hardware, the difference is whether reads update the
cache, thus clobbering preceding writes.  With $[r/x]$, reads clobber the
cache, whereas without the substitution, they do not.  Since most caches work
this way, the model with $[r/x]$ is likely preferred for modeling hardware.
However, this substitution only makes sense in a model with read-read
coherence and dependency.  By leaving out the substitution, we also ensure
that downgraded reads are fulfilled by preceding writes, not reads.



\begin{comment}
  if in L6 we said [x/r], that says we know read the local version...  ignoring
  the value read...  Perhaps there is some intervening stuff that stops you
  from seeing the local state, such as release-acquire

  We could potentially get rid of [x/r] If you do two reads, your not allowed
  to be independent of the second based on the value that was read in the first
  read.

  x=0; r=x; if (r=1) { s=x; if (s=?) {y=1}}
  read 1 then 2.


  In order for the write to be independent of second read what does its
  precondition have to be.
  [r/x] then s==1
  no sub then s==0

  (s=? | Wy1)

  if (phi) z=1
  phi = s is even
  phi = s < 2

  With substitution you are saying you know that the ``local copy'' of x is the
  same as r.  Sitting in the local cache.  Read might have gone to main
  memory, but if it did it has updated the cache line so that the local copy is
  what I just read.

  If second read is a cache hit, then I know that I am seeing the same value.

  If we take substitution out then 
\end{comment}


\section{Differences with OOPSLA}
\label{sec:diff}

\myparagraph{Substitution}

\jjr{} uses substitution rather than Skolemizing.  Indeed our use of
Skolemization is motivated by disjunction closure for predicate transformers,
which do not appear in \jjr{}.
In \textsection\ref{sec:tc1} on local invariant
reasoning (\xLIR), % (\refdef{def:pomsets-lir}):
we gave the semantics of load for nonempty pomsets as:
\begin{enumerate}
\item[\ref{L4})]
  $\aTr{\bEvs}{\bForm} \rimplies \aVal{=}\aReg\limplies\bForm$, 
\item[\ref{L5})]
  $\aTr{\cEvs}{\bForm} \rimplies (\aVal{=}\aReg\lor\aLoc{=}\aReg)\limplies\bForm$.
  % , when $\aEvs\neq\emptyset$,
  % \item[\ref{L6})] 
  %   $\aTr{\dEvs}{\bForm}\; \rimplies \bForm$, when $\aEvs=\emptyset$.
\end{enumerate}
In \jjr{}, the definition is roughly as follows:
% (adding the case for $\ref{L6}$, which was missing):
\begin{enumerate}
\item[\ref{L4})]
  $\aTr{\bEvs}{\bForm} \rimplies \bForm[\aVal/\aReg][\aVal/\aLoc]$, 
\item[\ref{L5})]
  $\aTr{\cEvs}{\bForm}\; \rimplies \bForm[\aVal/\aReg][\aVal/\aLoc]\land\bForm[\aLoc/\aReg]$.
\end{enumerate}
The use of conjunction in \ref{L5} causes disjunction closure to fail
because the predicate transformer
% $\aTr{}{\bForm}=\bForm[\aVal/\aReg][\aVal/\aLoc]\land\bForm[\aLoc/\aReg]$ does not distribute through
% disjunction:
% \begin{math}
%   \aTr{}{\bForm_1\lor \bForm_2}=
%   (\bForm_1\lor \bForm_2)[\aVal/\aReg][\aVal/\aLoc]\land(\bForm_1\lor \bForm_2)[\aLoc/\aReg]
%   \neq
%   (\bForm_1[\aVal/\aReg][\aVal/\aLoc]\land\bForm_1[\aLoc/\aReg]) \lor
%   (\bForm_2[\aVal/\aReg][\aVal/\aLoc]\land\bForm_2[\aLoc/\aReg])
%   = \aTr{}{\bForm_1} \lor \aTr{}{\bForm_2}
% \end{math}
$\aTr{}{\bForm}=\bForm'\land\bForm''$ does not distribute through
disjunction, even assuming that the prime operations do:\footnote{%
  \begin{math}
    (\bForm_1\lor \bForm_2)'=(\bForm_1'\lor \bForm_2')
  \end{math}
  and
  \begin{math}
    (\bForm_1\lor \bForm_2)''=(\bForm_1''\lor \bForm_2'')
  \end{math}.
}
\begin{math}
  \aTr{}{\bForm_1\lor \bForm_2}=
  \href{https://www.wolframalpha.com/input/?i=\%28a+or+b\%29+and+\%28c+or+d\%29}{(\bForm_1'\lor \bForm_2')\land(\bForm_1''\lor \bForm_2'')}
  \neq
  \href{https://www.wolframalpha.com/input/?i=\%28a+and+c\%29+or+\%28b+and+d\%29}{(\bForm_1'\land\bForm_1'') \lor (\bForm_2'\land\bForm_2'')}
  = \aTr{}{\bForm_1} \lor \aTr{}{\bForm_2}
\end{math}.
% \begin{math}
%   (\bForm_{1}^{1}\lor \bForm_{1}^{2}) \land (\bForm_{2}^{1}\lor \bForm_{2}^{2})
%   \neq
%   (\bForm_{1}^{1}\land\bForm_{2}^{1}) \lor (\bForm_{1}^{2}\land\bForm_{1}^{2}).
% \end{math}
See also \refex{ex:skolem}.

The substitutions collapse $\aLoc$ and $\aReg$, allowing local invariant
reasoning, as in \textsection\ref{sec:tc1}.  Without Skolemizing it is
necessary to substitute $[\aLoc/\aReg]$, since the reverse substitution
$[\aReg/\aLoc]$ is useless when $\aReg$ is bound---compare with
\textsection\ref{sec:substitutions}.  As discussed below (\ref{p:downset}),
including this substitution affects the interaction of \xLIR{} and downset closure.

Removing the substitution of $[x/r]$ in the independent case has a technical
advantage: we no longer require \emph{extended} expressions (which include
memory references), since substitutions no longer introduce memory
references.

\begin{scope}
  The substitution $[x/r]$ does not work with Skolemization, even for the
  dependent case, since we lose the unique marker for each read.  In effect,
  this forces the reads to the same values.  To be concrete, the candidate
  definition would modify \ref{L4} to be:
  \begin{enumerate}
  \item[\ref{L4})]
    $\aTr{\bEvs}{\bForm} \rimplies \aVal{=}\aLoc\limplies\bForm[\aLoc/\aReg]$.
    % \item[\ref{L5})]
    %   $\aTr{\cEvs}{\bForm} \rimplies (\aVal{=}\aLoc\lor\TRUE)\limplies\bForm[\aLoc/\aReg]$. %, when $\aEvs\neq\emptyset$,
    % \item[\ref{L6})] 
    %   $\aTr{\dEvs}{\bForm}\; \rimplies \bForm$, when $\aEvs=\emptyset$.
  \end{enumerate}
  Using this definition, consider the following:
  \begin{gather*}
    \PR{x}{r}\SEMI
    \PR{x}{s}\SEMI
    \IF{r{<}s}\THEN \PW{y}{1}\FI 
    \\
    \hbox{\begin{tikzinline}[node distance=0.5em and 1.5em]
        \event{a1}{\DR{x}{1}}{}
        \event{a2}{\DR{x}{2}}{right=of a1}
        \event{a3}{1{=}x\limplies 2{=}x\limplies x{<} x\mid\DW{y}{1}}{right=of a2}
        \po[out=20,in=160]{a1}{a3}
        \po{a2}{a3}
      \end{tikzinline}}
  \end{gather*}
  Although the execution seems reasonable, the precondition on the write is
  not a tautology.
\end{scope}


% There, item \ref{loadpre-kappa2}  of $\sLOADPRE{}{}{}$ is written 
% \begin{enumerate}
% \item[] %[\ref{loadpre-kappa2})]
%   if $\aEv\in\aEvs_2\setminus\aEvs_1$ then either \\
%   $\labelingForm(\aEv) \rimplies \labelingForm_2(\aEv)[\aLoc/\aReg][\aVal/\aLoc]$ and $(\exists\bEv\in\aEvs_1)\bEv{<}\aEv$, or \\
%   $\labelingForm(\aEv) \rimplies \labelingForm_2(\aEv)[\aLoc/\aReg][\aVal/\aLoc] \land \labelingForm_2(\aEv)[\aLoc/\aReg]$.
% \end{enumerate}


% [Skolemization ensures disjunction closure, which is necessary
% for associativity. Show example.]

\myparagraph[p:downset]{Downset closure}

\jjr{} enforces downset closure in the prefixing rule.  Even without this,
downset closure would be different for the two semantics, due to the use of
substitution in \jjr{}.  Consider the final pomset of \refex{ex:downset-lir},
under the semantics of this paper, which elides the middle read event:
\begin{align*}
  \begin{gathered}[t]
    \PW{x}{0} 
    \SEMI\PR{x}{r} 
    \SEMI\IF{r{\geq}0}\THEN \PW{y}{1} \FI
    \\
    \hbox{\begin{tikzinline}[node distance=.5em and 1.5em]
        \event{a0}{\DW{x}{0}}{}
        % \event{a1}{\DR{x}{1}}{right=of a0}
        \event{a2}{r{\geq}0\mid\DW{y}{1}}{right=3em of a1}      
        % \wk{a0}{a1}
      \end{tikzinline}}    
  \end{gathered}
\end{align*}
In \jjr{}, the substitution $[x/r]$ is performed by the middle read
regardless of whether it is included in the pomset, with the subsequent
substitution of $[0/x]$ by the preceding write, we have $[x/r][0/x]$, which
is $[0/r][0/x]$, resulting in:
\begin{align*}
  \begin{gathered}[t]
    \hbox{\begin{tikzinline}[node distance=.5em and 1.5em]
        \event{a0}{\DW{x}{0}}{}
        % \event{a1}{\DR{x}{1}}{right=of a0}
        \event{a2}{0{\geq}0\mid\DW{y}{1}}{right=3em of a1}      
        % \wk{a0}{a1}
      \end{tikzinline}}    
  \end{gathered}
\end{align*}


\myparagraph{Consistency}
\jjr{} imposes \emph{consistency}, which requires that for every pomset
$\aPS$, $\bigwedge_{\aEv}\labelingForm(\aEv)$ is satisfiable.  
\begin{scope}
  Associativity requires that we allow pomsets with inconsistent
  preconditions.  Consider a variant of \refex{ex:if1} from
  \textsection\ref{sec:if}.
  \begin{scope}
    \footnotesize
    \begin{align*}
      \begin{gathered}
        \IF{\aExp}\THEN\PW{x}{1}\FI
        \\
        \hbox{\begin{tikzinline}[node distance=1em]
            \event{a}{\aExp\mid\DW{x}{1}}{}
          \end{tikzinline}}
      \end{gathered}
      &&
      \begin{gathered}
        \IF{\BANG\aExp}\THEN\PW{x}{1}\FI
        \\
        \hbox{\begin{tikzinline}[node distance=1em]
            \event{a}{\lnot\aExp\mid\DW{x}{1}}{}
          \end{tikzinline}}
      \end{gathered}
      &&
      \begin{gathered}
        \IF{\aExp}\THEN\PW{y}{1}\FI
        \\
        \hbox{\begin{tikzinline}[node distance=1em]
            \event{a}{\aExp\mid\DW{y}{1}}{}
          \end{tikzinline}}
      \end{gathered}
      &&
      \begin{gathered}
        \IF{\BANG\aExp}\THEN\PW{y}{1}\FI
        \\
        \hbox{\begin{tikzinline}[node distance=1em]
            \event{a}{\lnot\aExp\mid\DW{y}{1}}{}
          \end{tikzinline}}
      \end{gathered}
    \end{align*}
  \end{scope}
  Associating left and right, we have:
  \begin{scope}
    \footnotesize
    \begin{align*}
      \begin{gathered}
        \IF{\aExp}\THEN\PW{x}{1}\FI
        \SEMI
        \IF{\BANG\aExp}\THEN\PW{x}{1}\FI
        \\
        \hbox{\begin{tikzinline}[node distance=1em]
            \event{a}{\DW{x}{1}}{}
          \end{tikzinline}}
      \end{gathered}
      &&
      \begin{gathered}
        \IF{\aExp}\THEN\PW{y}{1}\FI
        \SEMI
        \IF{\BANG\aExp}\THEN\PW{y}{1}\FI
        \\
        \hbox{\begin{tikzinline}[node distance=1em]
            \event{a}{\DW{y}{1}}{}
          \end{tikzinline}}
      \end{gathered}
    \end{align*}
  \end{scope}  
  Associating into the middle, instead, we require:
  \begin{scope}
    \footnotesize
    \begin{align*}
      \begin{gathered}
        \IF{\aExp}\THEN\PW{x}{1}\FI
        \\
        \hbox{\begin{tikzinline}[node distance=1em]
            \event{a}{\aExp\mid\DW{x}{1}}{}
          \end{tikzinline}}
      \end{gathered}
      &&
      \begin{gathered}
        \IF{\BANG\aExp}\THEN\PW{x}{1}\FI
        \SEMI
        \IF{\aExp}\THEN\PW{y}{1}\FI
        \\
        \hbox{\begin{tikzinline}[node distance=1em]
            \event{a}{\lnot\aExp\mid\DW{x}{1}}{}
            \event{b}{\aExp\mid\DW{y}{1}}{right=of a}
          \end{tikzinline}}
      \end{gathered}
      &&
      \begin{gathered}
        \IF{\BANG\aExp}\THEN\PW{y}{1}\FI
        \\
        \hbox{\begin{tikzinline}[node distance=1em]
            \event{a}{\lnot\aExp\mid\DW{y}{1}}{}
          \end{tikzinline}}
      \end{gathered}
    \end{align*}
  \end{scope}
  Joining left and right, we have:
  \begin{scope}
    \footnotesize
    \begin{align*}
      \begin{gathered}
        \IF{\aExp}\THEN\PW{x}{1}\FI
        \SEMI
        \IF{\BANG\aExp}\THEN\PW{x}{1}\FI
        \SEMI
        \IF{\aExp}\THEN\PW{y}{1}\FI
        \SEMI
        \IF{\BANG\aExp}\THEN\PW{y}{1}\FI
        \\
        \hbox{\begin{tikzinline}[node distance=1em]
            \event{a}{\DW{x}{1}}{}
            \event{b}{\DW{y}{1}}{right=of a}
          \end{tikzinline}}
      \end{gathered}
    \end{align*}
  \end{scope}  
\end{scope}

\myparagraph{Causal Strengthening}
\labeltext[]{Causal Strengthening}{xCausal}
\jjr{} imposes \emph{causal strengthening}, which requires for every pomset
$\aPS$, if $\bEv\le\aEv$ then $\labelingForm(\aEv) \rimplies \labelingForm(\bEv)$. 
\begin{scope}
  Associativity requires that we allow pomsets without causal strengthening.
  Consider the following.
  \begin{align*}
    \begin{gathered}
      \IF{\aExp}\THEN\PR{x}{r}\FI
      \\
      \hbox{\begin{tikzinline}[node distance=1em]
          \event{a}{\aExp\mid\DR{x}{1}}{}
        \end{tikzinline}}
    \end{gathered}
    &&
    \begin{gathered}
      \PW{y}{r}
      \\
      \hbox{\begin{tikzinline}[node distance=1em]
          \event{a}{r{=}1\mid\DW{y}{1}}{}
        \end{tikzinline}}
    \end{gathered}
    &&
    \begin{gathered}
      \IF{\BANG\aExp}\THEN\PR{x}{s}\FI
      \\
      \hbox{\begin{tikzinline}[node distance=1em]
          \event{a}{\lnot\aExp\mid\DR{x}{1}}{}
        \end{tikzinline}}
    \end{gathered}
  \end{align*}
  Associating left, with causal strengthening:
  \begin{align*}
    \begin{gathered}
      \IF{\aExp}\THEN\PR{x}{r}\FI
      \SEMI
      \PW{y}{r}
      \\
      \hbox{\begin{tikzinline}[node distance=1em]
          \event{a}{\aExp\mid\DR{x}{1}}{}
          \event{b}{\aExp\mid\DW{y}{1}}{right=of a}
          \po{a}{b}
        \end{tikzinline}}
    \end{gathered}
    &&
    \begin{gathered}
      \IF{\BANG\aExp}\THEN\PR{x}{s}\FI
      \\
      \hbox{\begin{tikzinline}[node distance=1em]
          \event{a}{\lnot\aExp\mid\DR{x}{1}}{}
        \end{tikzinline}}
    \end{gathered}
  \end{align*}
  Finally, merging:
  \begin{align*}
    \begin{gathered}
      \IF{\aExp}\THEN\PR{x}{r}\FI
      \SEMI
      \PW{y}{r}
      \SEMI
      \IF{\BANG\aExp}\THEN\PR{x}{s}\FI
      \\
      \hbox{\begin{tikzinline}[node distance=1em]
          \event{a}{\DR{x}{1}}{}
          \event{b}{\aExp\mid\DW{y}{1}}{right=of a}
          \po{a}{b}
        \end{tikzinline}}
    \end{gathered}
  \end{align*}
  Instead, associating right:
  \begin{align*}
    \begin{gathered}
      \IF{\aExp}\THEN\PR{x}{r}\FI
      \\
      \hbox{\begin{tikzinline}[node distance=1em]
          \event{a}{\aExp\mid\DR{x}{1}}{}
        \end{tikzinline}}
    \end{gathered}
    &&
    \begin{gathered}
      \PW{y}{r}
      \SEMI
      \IF{\BANG\aExp}\THEN\PR{x}{s}\FI
      \\
      \hbox{\begin{tikzinline}[node distance=1em]
          \event{a}{\lnot\aExp\mid\DR{x}{1}}{}
          \event{b}{r{=}1\mid\DW{y}{1}}{left=of a}
        \end{tikzinline}}
    \end{gathered}
  \end{align*}
  Merging:
  \begin{align*}
    \begin{gathered}
      \IF{\aExp}\THEN\PR{x}{r}\FI
      \SEMI
      \PW{y}{r}
      \SEMI
      \IF{\BANG\aExp}\THEN\PR{x}{s}\FI
      \\
      \hbox{\begin{tikzinline}[node distance=1em]
          \event{a}{\DR{x}{1}}{}
          \event{b}{\DW{y}{1}}{right=of a}
          \po{a}{b}
        \end{tikzinline}}
    \end{gathered}
  \end{align*}
  With causal strengthening, the precondition of $\DW{y}{1}$ depends upon how
  we associate.  This is not an issue in \jjr{}, which always associates to
  the right.
\end{scope}

% \myparagraph{Causal Strengthening and Address Dependencies}
% \labeltext[]{Causal Strengthening and Address Dependencies}{xADDRxRRD}

\begin{scope}  
  One use of causal strengthening is to ensure that address dependencies do
  not introduce thin air reads.  
  Associating to the right, the intermediate state of \refex{ex:xADDRxRRD} is:
  \begin{align*}
    \begin{gathered}[t]
      \PR{\REF{r}}{s}
      \SEMI
      \PW{x}{s}
      \\
      \hbox{\begin{tikzinline}[node distance=.5em and 1.5em]
          \event{a2}{r\EQ2\mid\DR{\REF{2}}{1}}{}
          \event{a3}{(r\EQ2\limplies 1\EQ s) \limplies s\EQ1\mid\DW{x}{1}}{right=of a2}
          \po{a2}{a3}
        \end{tikzinline}}
    \end{gathered}
  \end{align*}
  In \jjr{}, we have, instead:
  \begin{gather*}
    % \begin{gathered}[t]
    %   \PW{x}{s}
    %   \\
    %   \hbox{\begin{tikzinline}[node distance=.5em and 1.5em]
    %     \event{b}{s\EQ1\mid\DW{x}{1}}{}
    %   \end{tikzinline}}
    % \end{gathered}
    % \\
    \begin{gathered}
      % \PR{y}{r}\SEMI
      \PR{\REF{r}}{s}\SEMI \PW{x}{s}
      \\
      \hbox{\begin{tikzinline}[node distance=.5em and 1.5em]
          % \event{a1}{\DR{y}{2}}{}
          \event{a2}{r\EQ2\mid\DR{\REF{2}}{1}}{}%right=of a1}
          \event{a3}{r\EQ2\land\REF{2}\EQ1\mid\DW{x}{1}}{right=of a2}
          \po{a2}{a3}
        \end{tikzinline}}
    \end{gathered}
  \end{gather*}
  Without causal strengthening, the precondition of $\DWP{x}{1}$ would be
  simply $\REF{2}\EQ1$.  The treatment in this paper, using implication
  rather than conjunction, is more precise.
\end{scope}

\myparagraph{Parallel Composition}

In \jjr{\textsection2.4}, parallel composition is defined allowing coalescing
of events.  Here we have forbidden coalescing.  This difference appears to be
arbitrary.  In \jjr{}, however, there is a mistake in the handling of
termination actions.  The predicates should be joined using $\land$, not
$\lor$.

\myparagraph{Read-Modify-Write Actions}

In \jjr{}, the atomicity axioms \ref{pom-rmw-atomic} erroneously applies only to
overlapping writes, not overlapping reads.  The difficulty can be seen in
\refex{ex:rmw-33}.

\jjr{} does not specify the calculation of dependency for \RMW{}s, as
discussed in \refex{ex:rmw-dep}.


\myparagraph{Downgrading Internal Acquiring Reads}

Shortly after publication, \citet{anton} noticed a shortcoming of the
implementation on \armeight{} in \jjr{\textsection 7}.  The proof given there
assumes that all internal reads can be dropped.  However, this is not the
case for acquiring reds.  For example, \jjr{} disallows the following
execution, which is allowed by \armeight{} and \tso{}.
\begin{gather*}
  \PW{x}{2}\SEMI 
  \PR[\mACQ]{x}{r}\SEMI
  \PR{y}{s} \PAR
  \PW{y}{2}\SEMI
  \PW[\mREL]{x}{1}
  \\
  \hbox{\begin{tikzinline}[node distance=1.5em]
      \event{a}{\DW{x}{2}}{}
      \raevent{b}{\DR[\mACQ]{x}{2}}{right=of a}
      \event{c}{\DR{y}{0}}{right=of b}
      \event{d}{\DW{y}{2}}{right=2.5em of c}
      \raevent{e}{\DW[\mREL]{x}{1}}{right=of d}
      \rf{a}{b}
      \sync{b}{c}
      \wk{c}{d}
      \sync{d}{e}
      \wk[out=-165,in=-15]{e}{a}
      % \rfi{a}{b}
      % \bob{b}{c}
      % \fre{c}{d}
      % \bob{d}{e}
      % \coe[out=-165,in=-15]{e}{a}
    \end{tikzinline}}
\end{gather*}
The solution we have adopted is to allow an acquiring read to be downgraded
to a relaxed read when it is preceded (sequentially) by a relaxed write that
could fulfill it.  This solution allows executions that are not allowed under
\armeight{} since we do not insist that the local relaxed write is actually
read from.  This may seem counterintuitive, but we don't see a local way to
be more precise.

As a result, we use a different proof strategy for \armeight{}
implementation, which does not rely on read elimination.  The proof idea uses
a recent alternative characterization of \armeight{}
\citep{alglave-git-alternate,arm-reference-manual}. %,armed-cats}.

\myparagraph{Redundant Read Elimination}

Contrary to the claim, redundant read elimination fails for \jjr{}.
We discussed redundant read elimination in \textsection\ref{sec:recycle}.
Consider JMM Causality Test Case 2, which we discussed there.
\begin{gather*}
  \PR{x}{r}\SEMI
  \PR{x}{s}\SEMI
  \IF{r{=}s}\THEN \PW{y}{1}\FI
  \PAR
  \PW{x}{y}
  \\
  \hbox{\begin{tikzinline}[node distance=1.5em]
      \event{a1}{\DR{x}{1}}{}
      \event{a2}{\DR{x}{1}}{right=of a1}
      \event{a3}{\DW{y}{1}}{right=of a2}
      \event{b1}{\DR{y}{1}}{right=3em of a3}
      \event{b2}{\DW{x}{1}}{right=of b1}
      \rf{a3}{b1}
      \po{b1}{b2}
      \rf[out=169,in=11]{b2}{a2}
      \rf[out=169,in=11]{b2}{a1}
    \end{tikzinline}}
\end{gather*}
Under the semantics of \jjr{}, we have
\begin{gather*}
  \PR{x}{r}\SEMI
  \PR{x}{s}\SEMI
  \IF{r{=}s}\THEN \PW{y}{1}\FI
  \\
  \hbox{\begin{tikzinline}[node distance=1.5em]
      \event{a1}{\DR{x}{1}}{}
      \event{a2}{\DR{x}{1}}{right=of a1}
      \event{a3}{1\EQ1\land1\EQ x \land x\EQ1 \land x=x\mid\DW{y}{1}}{right=of a2}
    \end{tikzinline}}
\end{gather*}
The precondition of $\DWP{y}{1}$ is \emph{not} a tautology, and therefore
redundant read elimination fails.
(It is a tautology in
\begin{math}
  \PR{x}{r}\SEMI
  \LET{s}{r}\SEMI
  \IF{r{=}s}\THEN \PW{y}{1}\FI
\end{math}.)
In \jjr{\textsection3.1}, we incorrectly stated that the precondition of
$\DWP{y}{1}$ was $1\EQ1\land x\EQ x$.  

\begin{comment}
  Precondition of $\DWP{y}{1}$ is $(r{=}s)$ in
  \begin{math}
    \sem{\IF{r{=}s}\THEN \PW{y}{1}\FI}.
  \end{math}
  Predicate transformers for $\emptyset$ in $\sem{\PR{x}{r}}$ and $\sem{\PR{x}{s}}$ are
  \begin{align*}
    \PREDP{(r{=}1 \lor r{=}x)\limplies\bForm[r/x]},
    \\
    \PREDP{(s{=}1 \lor s{=}x)\limplies\bForm[s/x]}.
  \end{align*}
  Combining the transformers, we have
  \begin{displaymath}
    \PREDP{(r{=}1 \lor r{=}x)\limplies(s{=}1 \lor s{=}r)\limplies\bForm[s/x]}.
  \end{displaymath}
  Applying this to $(r{=}s)$, we have
  \begin{displaymath}
    \PREDP{(r{=}1 \lor r{=}x)\limplies (s{=}1 \lor s{=}r)\limplies (r{=}s)},
  \end{displaymath}
  which is not a tautology.

  Same problem occurs \jjr{}, where we have:
  \begin{align*}
    \PREDP{\bForm[v/x,r] \land \bForm[x/r]},
    \\
    \PREDP{\bForm[v/x,s] \land \bForm[x/s]}.
  \end{align*}
  Combining the transformers, we have
  \begin{displaymath}
    \PREDP{\bForm[v/x,r,s] \land \bForm [v/x,r][x/s] \land \bForm[x/r][v/x,s] \land \bForm[x/r,s]}.
  \end{displaymath}
  Applying this to $(r{=}s)$, we have
  \begin{displaymath}
    \PREDP{v{=}v \land v{=}x \land x{=}v \land x{=}x},
  \end{displaymath}
  which is not a tautology.

  The semantics here allows this by coalescing:
  \begin{gather*}
    \PR{x}{r}\SEMI
    \PR{x}{s}\SEMI
    \IF{r{=}s}\THEN \PW{y}{1}\FI
    \PAR
    \PW{x}{y}
    \\
    \hbox{\begin{tikzinline}[node distance=1.5em]
        \event{a1}{\DR{x}{1}}{}
        \event{a3}{\DW{y}{1}}{right=of a1}
        \event{b1}{\DR{y}{1}}{right=3em of a3}
        \event{b2}{\DW{x}{1}}{right=of b1}
        \rf{a3}{b1}
        \po{b1}{b2}
        \rf[out=169,in=11]{b2}{a1}
      \end{tikzinline}}
  \end{gather*}

  In \jjr{\textsection2.6} the semantics of read is defined as follows:
  \begin{align*}
    \sem{\PR[\amode]{\aLoc}{\aReg}\SEMI \aCmd} & \eqdef \textstyle\bigcup_\aVal\;
    (\DRmode\aLoc\aVal) \prefix \sem{\aCmd} [\aLoc/\aReg]
  \end{align*}
  The definition of prefixing$((\aForm \mid \aAct) \prefix \aPSS)$ has several clauses.
  The most relevant are as follows, where $\bEv$ is the new event labeled with
  $(\aForm \mid \aAct)$ and $\aEv$ is an event from $\aPSS$:
  \begin{description}
  \item[{\labeltextsc[P4c]{(P4c)}{4c}}]
    If $\bEv$ reads $\aVal$ from $\aLoc$ then either $\aEv=\bEv$ or
    $\labelingForm'(\aEv) \rimplies \labelingForm(\aEv)[\aVal/\aLoc]$.
  \item[{\labeltextsc[P5a]{(P5a)}{5a}}]\labeltextsc[P5]{}{5}%
    If $\bEv$ reads and $\aEv$ writes then either $\labelingForm'(\aEv) \rimplies \labelingForm(\aEv)$ or $\bEv\le'\aEv$.
    % \item[{\labeltextsc[P5b]{(P5b)}{5b}}]
    %   If $\bEv$ and $\aEv$ are in conflict then $\bEv\le'\aEv$.
  \end{description}

  We have discovered two issues with this definition.

  The first issue concerns the substitution $[\aLoc/\aReg]$.  It should be
  $[\aReg/\aLoc]$.  We noticed this error while developing the alternative
  characterization presented here.  The error causes redundant read elimination
  to fail in \jjr{}.  As a result, common subexpression elimination also fails.
  The problem can be seen in \ref{TC2}.
  \begin{gather*}
    \taglabel{TC2}
    \PR{x}{r}\SEMI
    \PR{x}{s}\SEMI
    \IF{r{=}s}\THEN \PW{y}{1}\FI
    \PAR
    \PW{x}{y}
  \end{gather*}
  % In \jjr{\textsection3.1},
  We claimed that \ref{TC2} allowed the following
  execution:
  \begin{gather*}
    \hbox{\begin{tikzinline}[node distance=1.5em]
        \event{a1}{\DR{x}{1}}{}
        \event{a2}{\DR{x}{1}}{right=of a1}
        \event{a3}{\DW{y}{1}}{right=of a2}
        % \po{a2}{a3}
        % \po[out=15,in=165]{a1}{a3}
        \event{b1}{\DR{y}{1}}{right=3em of a3}
        \event{b2}{\DW{x}{1}}{right=of b1}
        \rf{a3}{b1}
        \po{b1}{b2}
        \rf[out=169,in=11]{b2}{a2}
        \rf[out=169,in=11]{b2}{a1}
      \end{tikzinline}}
  \end{gather*}
  But this execution is not possible using the semantics of \jjr{}:
  $\DWP{y}{1}$ has precondition $r{=}s$ in
  \begin{math}
    \sem{\IF{r{=}s}\THEN \PW{y}{1}\FI}.
  \end{math}
  Given the lack of order in the execution, the precondition of $\DWP{y}{1}$
  must entail $r{=}1\land r{=}x$ in 
  \begin{math}
    \sem{\PR{x}{s}\SEMI
      \IF{r{=}s}\THEN \PW{y}{1}\FI}.
  \end{math}
  \ref{4c} imposes $r{=}1$, and \ref{5a} imposes $r{=}x$.  Adding the second
  read, the precondition of $\DWP{y}{1}$ must entail both $1{=}1\land 1{=}x$
  and also $x{=}1\land x{=}x$.  This can be simplified to $x{=}1$.  This leaves
  a requirement that must be satisfied by a preceding write.  Since the
  preceding write is the initialization to $0$, the requirement cannot be
  satisfied, and the execution is impossible.\footnote{In \jjr{} we ignore the
    middle terms, mistakenly simplifying this to $1{=}1\land x{=}x$.
    Correcting the error, the attempted execution is:
    \begin{gather*}
      \hbox{\begin{tikzinline}[node distance=1.5em]
          \event{a1}{\DR{x}{1}}{}
          \event{a2}{\DR{x}{1}}{right=of a1}
          \event{a3}{\DW{y}{1}}{right=of a2}
          \po{a2}{a3}
          \po[out=-20,in=-160]{a1}{a3}
          \event{b1}{\DR{y}{1}}{right=3em of a3}
          \event{b2}{\DW{x}{1}}{right=of b1}
          \rf{a3}{b1}
          \po{b1}{b2}
          \rf[out=169,in=11]{b2}{a2}
          \rf[out=169,in=11]{b2}{a1}
        \end{tikzinline}}
    \end{gather*}}

  The substitution $[\aLoc/\aReg]$ leaves the obligation on $\aLoc$ to be
  fulfilled by the preceding write.  Thus, the read does not update the
  \emph{value} of $\aLoc$ in subsequent predicates.  The substitution
  $[\aReg/\aLoc]$, instead, does update the value of $\aLoc$, thus removing any
  obligation on $\aLoc$ for preceding code.

  In order to write this, we must update the definition of prefixing reads to
  include the register.  Then \ref{4c} becomes:
  \begin{description}
  \item[\textsc{(p4c)}] If $\bEv$ reads $\aVal$ from $\aLoc$ then either
    $\aEv=\bEv$ or $\labelingForm'(\aEv) \rimplies \labelingForm(\aEv)[\aVal/\aReg]$.
  \end{description}

  We can then reason with \ref{TC2} as follows: $\DWP{y}{1}$ has precondition
  $r{=}s$ in
  \begin{math}
    \sem{\IF{r{=}s}\THEN \PW{y}{1}\FI}.
  \end{math}
  To avoid introducing order in the execution, the precondition of $\DWP{y}{1}$
  must entail $r{=}1\land r{=}s$ in 
  \begin{math}
    \sem{\PR{x}{s}\SEMI
      \IF{r{=}s}\THEN \PW{y}{1}\FI}.
  \end{math}
  \ref{4c} imposes $r{=}1$, and \ref{5a} imposes $r{=}x$.  Adding the second
  read, the precondition of $\DWP{y}{1}$ must entail both $1{=}1\land 1{=}x$
  and also $x{=}1\land x{=}x$.  This can be simplified to $x{=}1$.  This leaves
  a requirement that must be satisfied by a preceding write.


  With read elimination, the rule for relaxed reads is as follows:
  \begin{align*}
    \sem{\PR{\aLoc}{\aReg} \SEMI \aCmd} &\eqdef
    \sem{\aCmd}[\aLoc/\aReg]
    \cup
    \textstyle\bigcup_\aVal\;
    \DRP{\aLoc}{\aVal} \prefix_{\aReg} %\Rdis{\aLoc}{\aVal}
    \sem{\aCmd}[\aReg/\aLoc]
  \end{align*}
  It is interesting to note that the substitution is $[\aLoc/\aReg]$ on
  eliminated reads, and $[\aReg/\aLoc]$ on non-eliminated reads.  Intuitively,
  the subsequent value of $\aLoc$ is fixed by an explicit read, but not for an
  eliminated read.  In the latter case, the value is fixed by some preceding
  action.  The preceding action may itself be a read. This gives rise to some
  fear that we might introduce thin-air reads, since we do not enforce
  read-read coherence.  But this is not the case.  Consider the following example:
  \begin{gather*}
    \PR{x}{r}\SEMI
    \PR{x}{s}\SEMI
    \PW{y}{s}
    \PAR
    \PW{x}{y}
    \\
    \hbox{\begin{tikzinline}[node distance=1.5em]
        \event{a1}{\DR{x}{1}}{}
        \event{a2}{\DR{x}{1}}{right=of a1}
        \event{a3}{\DW{y}{1}}{right=of a2}
        % \po{a2}{a3}
        \po[out=-20,in=-160]{a1}{a3}
        \event{b1}{\DR{y}{1}}{right=3em of a3}
        \event{b2}{\DW{x}{1}}{right=of b1}
        \rf{a3}{b1}
        \po{b1}{b2}
        \rf[out=169,in=11]{b2}{a2}
        \rf[out=169,in=11]{b2}{a1}
      \end{tikzinline}}
    \\
    \hbox{\begin{tikzinline}[node distance=1.5em]
        \event{a1}{\DR{x}{1}}{}
        \internal{a2}{\DR{x}{1}}{right=of a1}
        \event{a3}{\DW{y}{1}}{right=of a2}
        % \po{a2}{a3}
        \po[out=-20,in=-160]{a1}{a3}
        \event{b1}{\DR{y}{1}}{right=3em of a3}
        \event{b2}{\DW{x}{1}}{right=of b1}
        \rf{a3}{b1}
        \po{b1}{b2}
        % \rf[out=169,in=11]{b2}{a2}
        \rf[out=169,in=11]{b2}{a1}
      \end{tikzinline}}
  \end{gather*}
  But this is not a problem, since fulfillment requires that $\DWP{x}{1}$
  precede both reads of $x$.
\end{comment}

\myparagraph{Stupid Bug in LDRF}
Definition of race is wrong.
Should say that at least one is relaxed.

\clearpage
\section{More Stuff}

\subsection{A Note on Mixed-Mode Data Races}

In preparing this paper, we came across the following example, which appears
to invalidate Theorem 4.1 of \cite{DBLP:conf/ppopp/DongolJR19}.
\begin{gather}
  \nonumber
  \PW{x}{1}\SEMI
  \PW[\mREL]{y}{1}\SEMI
  \PR[\mACQ]{x}{r}
  \PAR
  \IF{\PR[\mACQ]{y}{}}\THEN \PW[\mREL]{x}{2}\FI
  \\
  \label{mix1}
  \hbox{\begin{tikzinline}[node distance=1.5em]
      \event{a1}{\DW{x}{1}}{}
      \raevent{a2}{\DW[\mREL]{y}{1}}{right=of a1}
      \raevent{a3}{\DR[\mACQ]{x}{1}}{right=of a2}
      \raevent{b1}{\DR[\mACQ]{y}{1}}{right=3em of a3}
      % \raevent{b1}{\DR[\mACQ]{y}{1}}{below=of a1}
      \raevent{b2}{\DW[\mREL]{x}{2}}{right=of b1}
      \sync{a1}{a2}
      \rf[out=20,in=160]{a1}{a3}
      \rf[out=20,in=160]{a2}{b1}
      \wk[out=-20,in=-160]{a3}{b2}
      \sync{b1}{b2}
      % \node(ai)[left=3em of a1]{};
      % \bgoval[yellow!50]{(ai)}{P}
      % \bgoval[pink!50]{(a1)(a2)(b1)(b2)}{P'\setminus P}
      % \bgoval[green!10]{(a3)}{P'''\setminus P'}
    \end{tikzinline}}
  \\
  \label{mix2}
  \hbox{\begin{tikzinline}[node distance=1.5em]
      \event{a1}{\DW{x}{1}}{}
      \raevent{a2}{\DW[\mREL]{y}{1}}{right=of a1}
      \raevent{a3}{\DR[\mACQ]{x}{2}}{right=of a2}
      \raevent{b1}{\DR[\mACQ]{y}{1}}{right=3em of a3}
      \raevent{b2}{\DW[\mREL]{x}{2}}{right=of b1}
      \sync{a1}{a2}
      \rf[out=20,in=160]{a2}{b1}
      \rf[out=160,in=20]{b2}{a3}
      \sync{b1}{b2}
    \end{tikzinline}}
\end{gather}
The program is data-race free.  The two executions shown are the only
top-level executions that include $\DWP[\mREL]{x}{2}$.

Theorem 4.1 of \cite{DBLP:conf/ppopp/DongolJR19} is stated by extending
execution sequences.  In the terminology of
\cite{DBLP:conf/ppopp/DongolJR19}, a read is \emph{$L$-weak} if it is
sequentially stale.  Let
\begin{math}
  \rho=\DWP{x}{1}
  \DWP[\mREL]{y}{1}
  \DRP[\mACQ]{y}{1}
  \DWP[\mREL]{x}{2}
\end{math}
be a sequence and
\begin{math}
  \alpha=\DRP[\mACQ]{x}{1}.
\end{math}
$\rho$ is $L$-sequential and $\alpha$ is $L$-weak in $\rho\alpha$.  But there
is no execution of this program that includes a data race, contradicting the
theorem.  The error seems to be in Lemma A.4 of
\cite{DBLP:conf/ppopp/DongolJR19}, which states that if $\alpha$ is $L$-weak
after an $L$-sequential $\rho$, then $\alpha$ must be in a data race.  That
is clearly false here, since $\DRP[\mACQ]{x}{1}$ is stale, but the program is
data race free.

In proving the SC-LDRF result in \jjr{\textsection8}, we noted that our proof
technique is more robust than that of \cite{DBLP:conf/ppopp/DongolJR19},
because it limits the prefixes that must be considered.  In \eqref{mix1}, the
induction hypothesis requires that we add $\DRP[\mACQ]{x}{1}$ before
$\DWP[\mREL]{x}{2}$ since $\DRP[\mACQ]{x}{1}\xwk\DWP[\mREL]{x}{2}$.  In
particular,
\begin{gather*}
  \hbox{\begin{tikzinline}[node distance=1.5em]
      \event{a1}{\DW{x}{1}}{}
      \raevent{a2}{\DW[\mREL]{y}{1}}{right=of a1}
      % \raevent{a3}{\DR[\mACQ]{x}{1}}{right=of a2}
      \raevent{b1}{\DR[\mACQ]{y}{1}}{right=3em of a3}
      % \raevent{b1}{\DR[\mACQ]{y}{1}}{below=of a1}
      \raevent{b2}{\DW[\mREL]{x}{2}}{right=of b1}
      \sync{a1}{a2}
      % \rf[out=20,in=160]{a1}{a3}
      \rf[out=20,in=160]{a2}{b1}
      % \wk[out=-20,in=-160]{a3}{b2}
      \sync{b1}{b2}
      % \node(ai)[left=3em of a1]{};
      % \bgoval[yellow!50]{(ai)}{P}
      % \bgoval[pink!50]{(a1)(a2)(b1)(b2)}{P'\setminus P}
      % \bgoval[green!10]{(a3)}{P'''\setminus P'}
    \end{tikzinline}}
\end{gather*}
is not a downset of \eqref{mix1}, because
$\DRP[\mACQ]{x}{1}\xwk\DWP[\mREL]{x}{2}$.  As we noted in \jjr{\textsection8},
this affects the inductive order in which we move across pomsets, but does
not affect the set of pomsets that are considered.  In particular,
\begin{gather*}
  \hbox{\begin{tikzinline}[node distance=1.5em]
      \event{a1}{\DW{x}{1}}{}
      \raevent{a2}{\DW[\mREL]{y}{1}}{right=of a1}
      % \raevent{a3}{\DR[\mACQ]{x}{1}}{right=of a2}
      \raevent{b1}{\DR[\mACQ]{y}{1}}{right=3em of a3}
      % \raevent{b1}{\DR[\mACQ]{y}{1}}{below=of a1}
      % \raevent{b2}{\DW[\mREL]{x}{2}}{right=of b1}
      \sync{a1}{a2}
      % \rf[out=20,in=160]{a1}{a3}
      \rf[out=20,in=160]{a2}{b1}
      % \wk[out=-20,in=-160]{a3}{b2}
      % \sync{b1}{b2}
      % \node(ai)[left=3em of a1]{};
      % \bgoval[yellow!50]{(ai)}{P}
      % \bgoval[pink!50]{(a1)(a2)(b1)(b2)}{P'\setminus P}
      % \bgoval[green!10]{(a3)}{P'''\setminus P'}
    \end{tikzinline}}
\end{gather*}
is a downset of \eqref{mix1}.


\subsection{Downgraded Reads}
\label{sec:dgr}

We allow downgrades in executions that are not be allowed by \armeight{}.
\begin{gather*}
  \PW{x}{2}\SEMI 
  \PR[\mACQ]{x}{r}\SEMI
  \PW{y}{1} \PAR
  \PW{y}{2}\SEMI
  \PW[\mREL]{x}{1} \PAR
  \PW{x}{3}
  \\
  \hbox{\begin{tikzinline}[node distance=1.5em]
      \event{a}{\DW{x}{2}}{}
      \raevent{b}{\DR[\mACQ]{x}{3}}{right=of a}
      \event{c}{\DW{y}{1}}{right=of b}
      \event{d}{\DW{y}{2}}{right=2.5em of c}
      \raevent{e}{\DW[\mREL]{x}{1}}{right=of d}
      \event{f}{\DW{x}{3}}{right=2.5em of e}
      \wk[out=15,in=165]{a}{f}
      \rf[out=-165,in=-15]{f}{b}
      % \sync{b}{c}
      \wk{c}{d}
      \sync{d}{e}
      \wk[out=-165,in=-15]{e}{a}
    \end{tikzinline}}  
\end{gather*}
\armeight{} disallows this because the acquiring read is fulfilled by an
external write.
\begin{gather*}
  % \taglabel{data-rfi-rfe-rfe}
  \PW{x}{\PR{z}{}} \SEMI
  \PR[\mACQ]{x}{r}\SEMI
  \PW{y}{1} \PAR
  \PW{z}{\PR{y}{}}
  \\
  % \tag{\xmark\armeight}
  \hbox{\begin{tikzinline}[node distance=1.5em]
      \event{a}{\DR{z}{1}}{}
      \event{b}{\DW{x}{1}}{right=of a}
      \raevent{c}{\DR[\mACQ]{x}{1}}{right=of b}
      \event{d}{\DW{y}{1}}{right=of c}
      \event{e}{\DW{y}{1}}{right=2.5em of d}
      \event{f}{\DW{z}{1}}{right=of e}
      \po{a}{b}
      \rf{b}{c}
      % \bob{c}{d}
      \po{e}{f}
      \rf[out=-165,in=-15]{f}{a}
      \rf{d}{e}
    \end{tikzinline}}
\end{gather*}
\armeight{} disallows this because data and control dependencies change  acquiring read is fulfilled by an
external write.

\subsection{If Closure and Address Dependencies}
\label{sec:addr}

An optimization ($p$/$q$ are registers):
\begin{displaymath}
  \PRREF{p}{r}\SEMI
  \PRREF{q}{s}
\end{displaymath}
vs
\begin{displaymath}
  \PRREF{p}{r}\SEMI
  \IF{p{=}q}\THEN \LET{s}{r} \ELSE \PRREF{q}{s}\FI
\end{displaymath}

\begin{displaymath}
  \LET{r}{\mathtt{new}}\SEMI
  \PW{\REF{r}}{42}\SEMI
  \PR{\REF{r}}{s}\SEMI
  \PW{x}{r}
  \PAR
  \PR{x}{r}\SEMI
  \PW{\REF{r}}{7}
\end{displaymath}

If closure is at odds with Java Final field semantics.

Do sequencing and if commute?

\subsection{About Arm}
Hypothesis: gcb cannot contradict (poloc minus RxR).


\subsection{Using Independency for Coherence}
\label{sec:independency-co}

It is also possible to use independency only to capture coherence, but the
results are less interesting.

\begin{figure*}[t]
  \begin{subfigure}{.5\textwidth}
    \centering
    \begin{align*}
  \begin{gathered}
    \begin{gathered}[t]
      \PW{x}{\aExp}
      \\
      \hbox{\begin{tikzinline}[node distance=.5em and 1.5em]
          \event{a}{\aExp{=}v\land\Qr{x}\land\Qw{x}\mid\DW{x}{v}}{}
          \xform{xi}{\bForm[\FALSE/\Qw{x}]}{above=of a}
          \xform{xd}{\bForm[(\Qw{x}\land\aExp{=}\aVal)/\Qw{x}]}{below=of a}
          \xo{a}{xd}
        \end{tikzinline}}
    \end{gathered}
    \\[1ex]
    \begin{gathered}[t]
      \PW[\mRA]{x}{\aExp}
      \\
      \hbox{\begin{tikzinline}[node distance=.5em and 1.5em]
          \raevent{a}{\aExp{=}v\land\Qr{*}\land\Qw{*}\mid\DW[\mRA]{x}{v}}{}
          \xform{xi}{\bForm[\FALSE/\Qw{x}]}{above=of a}
          \xform{xd}{\bForm[(\Qw{x}\land\aExp{=}\aVal)/\Qw{x}]}{below=of a}
          \xo{a}{xd}
        \end{tikzinline}}
    \end{gathered}
    \\[1ex]
    \begin{gathered}[t]
      \PW[\mSC]{x}{\aExp}
      \\
      \hbox{\begin{tikzinline}[node distance=.5em and 1.5em]
          \scevent{a}{\aExp{=}v\land\Qr{*}\land\Qw{*}\land\Qsc\mid\DW[\mSC]{x}{v}}{}
          \xform{xi}{\bForm[\FALSE/\Qw{x}][\FALSE/\Qsc]}{above=of a}
          \xform{xd}{\bForm[(\Qw{x}\land\aExp{=}\aVal)/\Qw{x}]}{below=of a}
          \xo{a}{xd}
        \end{tikzinline}}
    \end{gathered}
  \end{gathered}
  &&
  \begin{gathered}
    \begin{gathered}[t]
      \PR{x}{r}
      \\
      \hbox{\begin{tikzinline}[node distance=.5em and 1.5em]
          \event{a}{\Qw{x}\mid\DR{x}{v}}{}
          \xform{xi}{\bForm[\FALSE/\Qr{x}]}{above=of a}
          \xform{xd}{v{=}r\limplies\bForm}{below=of a}
          \xo{a}{xd}
        \end{tikzinline}}
    \end{gathered}
    \\[1ex]
    \begin{gathered}[t]
      \PR[\mRA]{x}{r}
      \\
      \hbox{\begin{tikzinline}[node distance=.5em and 1.5em]
          \raevent{a}{\Qw{x}\mid\DR[\mRA]{x}{v}}{}
          \xform{xi}{\bForm[\FALSE/\Qr{*}][\FALSE/\Qw{*}]}{above=of a}
          \xform{xd}{v{=}r\limplies\bForm}{below=of a}
          \xo{a}{xd}
        \end{tikzinline}}
    \end{gathered}
    \\[1ex]
    \begin{gathered}[t]
      \PR[\mSC]{x}{r}
      \\
      \hbox{\begin{tikzinline}[node distance=.5em and 1.5em]
          \scevent{a}{\Qw{x}\land\Qsc\mid\DR[\mSC]{x}{v}}{}
          \xform{xi}{\bForm[\FALSE/\Qr{*}][\FALSE/\Qw{*}][\FALSE/\Qsc]}{above=of a}
          \xform{xd}{v{=}r\limplies\bForm}{below=of a}
          \xo{a}{xd}
        \end{tikzinline}}
    \end{gathered}
  \end{gathered}
\end{align*}
    \caption{Quiescence Examples (\textsection\ref{sec:sync})}
    \label{fig:q1}
  \end{subfigure}  
  \begin{subfigure}{.5\textwidth}
    \centering
    \begin{align*}
  \begin{gathered}
    \begin{gathered}[t]
      \PW{x}{\aExp}
      \\
      \hbox{\begin{tikzinline}[node distance=.5em and 1.5em]
          \event{a}{\Qra \land\aExp{=}v\mid\DW{x}{v}}{}
          \xform{xi}{\bForm[\FALSE/\Qrlx]}{above=of a}
          \xform{xd}{\bForm\land \aExp{=}\aVal}{below=of a}
          %\xform{xd}{\bForm[(\Qrlx\land\aExp{=}\aVal)/\Qrlx]}{below=of a}
          \xo[xright]{a}{xd}
        \end{tikzinline}}
    \end{gathered}
    \\[1ex]
    \begin{gathered}[t]
      \PW[\mREL]{x}{\aExp}
      \\
      \hbox{\begin{tikzinline}[node distance=.5em and 1.5em]
          \raevent{a}{\Qra\land\Qrlx \land\aExp{=}v\mid\DW[\mREL]{x}{v}}{}
          \xform{xi}{\bForm[\FALSE/\Qrlx]}{above=of a}
          \xform{xd}{\bForm\land \aExp{=}\aVal}{below=of a}
          %\xform{xd}{\bForm[(\Qrlx\land\aExp{=}\aVal)/\Qrlx]}{below=of a}
          \xo[xright]{a}{xd}
        \end{tikzinline}}
    \end{gathered}
    \\[1ex]
    \begin{gathered}[t]
      \PW[\mSC]{x}{\aExp}
      \\
      \hbox{\begin{tikzinline}[node distance=.5em and 1.5em]
          \scevent{a}{\Qra\land\Qrlx\land\Qsc \land\aExp{=}v\mid\DW[\mSC]{x}{v}}{}
          \xform{xi}{\bForm[\FALSE/\Qrlx][\FALSE/\Qsc]}{above=of a}
          \xform{xd}{\bForm\land \aExp{=}\aVal}{below=of a}
          %\xform{xd}{\bForm[(\Qrlx\land\aExp{=}\aVal)/\Qrlx]}{below=of a}
          \xo[xright]{a}{xd}
        \end{tikzinline}}
    \end{gathered}
  \end{gathered}
  &&
  \begin{gathered}
    \begin{gathered}[t]
      \PR{x}{r}
      \\
      \hbox{\begin{tikzinline}[node distance=.5em and 1.5em]
          \event{a}{\Qra\mid\DR{x}{v}}{}
          \xform{xi}{\bForm[\FALSE/\Qrlx]}{above=of a}
          \xform{xd}{v{=}r\limplies\bForm}{below=of a}
          \xo[xright]{a}{xd}
        \end{tikzinline}}
    \end{gathered}
    \\[1ex]
    \begin{gathered}[t]
      \PR[\mACQ]{x}{r}
      \\
      \hbox{\begin{tikzinline}[node distance=.5em and 1.5em]
          \raevent{a}{\Qra\mid\DR[\mACQ]{x}{v}}{}
          \xform{xi}{\bForm[\FALSE/\Qrlx][\FALSE/\Qra]}{above=of a}
          \xform{xd}{v{=}r\limplies\bForm}{below=of a}
          \xo[xright]{a}{xd}
        \end{tikzinline}}
    \end{gathered}
    \\[1ex]
    \begin{gathered}[t]
      \PR[\mSC]{x}{r}
      \\
      \hbox{\begin{tikzinline}[node distance=.5em and 1.5em]
          \scevent{a}{\Qra\land\Qsc\mid\DR[\mSC]{x}{v}}{}
          \xform{xi}{\bForm[\FALSE/\Qrlx][\FALSE/\Qra][\FALSE/\Qsc]}{above=of a}
          \xform{xd}{v{=}r\limplies\bForm}{below=of a}
          \xo[xright]{a}{xd}
        \end{tikzinline}}
    \end{gathered}
  \end{gathered}
\end{align*}

    \caption{Quiescence Examples (\textsection\ref{sec:independency-co})}
    \label{fig:q2}
  \end{subfigure}
  \caption{Quiescence Examples for Coherence}
  \label{fig:q-co}
\end{figure*}
% \begin{figure}[t!]
%   \centering
%   \begin{align*}
  \begin{gathered}
    \begin{gathered}[t]
      \PW{x}{\aExp}
      \\
      \hbox{\begin{tikzinline}[node distance=.5em and 1.5em]
          \event{a}{\aExp{=}v\land\Qr{x}\land\Qw{x}\mid\DW{x}{v}}{}
          \xform{xi}{\bForm[\FALSE/\Qw{x}]}{above=of a}
          \xform{xd}{\bForm[(\Qw{x}\land\aExp{=}\aVal)/\Qw{x}]}{below=of a}
          \xo{a}{xd}
        \end{tikzinline}}
    \end{gathered}
    \\[1ex]
    \begin{gathered}[t]
      \PW[\mRA]{x}{\aExp}
      \\
      \hbox{\begin{tikzinline}[node distance=.5em and 1.5em]
          \raevent{a}{\aExp{=}v\land\Qr{*}\land\Qw{*}\mid\DW[\mRA]{x}{v}}{}
          \xform{xi}{\bForm[\FALSE/\Qw{x}]}{above=of a}
          \xform{xd}{\bForm[(\Qw{x}\land\aExp{=}\aVal)/\Qw{x}]}{below=of a}
          \xo{a}{xd}
        \end{tikzinline}}
    \end{gathered}
    \\[1ex]
    \begin{gathered}[t]
      \PW[\mSC]{x}{\aExp}
      \\
      \hbox{\begin{tikzinline}[node distance=.5em and 1.5em]
          \scevent{a}{\aExp{=}v\land\Qr{*}\land\Qw{*}\land\Qsc\mid\DW[\mSC]{x}{v}}{}
          \xform{xi}{\bForm[\FALSE/\Qw{x}][\FALSE/\Qsc]}{above=of a}
          \xform{xd}{\bForm[(\Qw{x}\land\aExp{=}\aVal)/\Qw{x}]}{below=of a}
          \xo{a}{xd}
        \end{tikzinline}}
    \end{gathered}
  \end{gathered}
  &&
  \begin{gathered}
    \begin{gathered}[t]
      \PR{x}{r}
      \\
      \hbox{\begin{tikzinline}[node distance=.5em and 1.5em]
          \event{a}{\Qw{x}\mid\DR{x}{v}}{}
          \xform{xi}{\bForm[\FALSE/\Qr{x}]}{above=of a}
          \xform{xd}{v{=}r\limplies\bForm}{below=of a}
          \xo{a}{xd}
        \end{tikzinline}}
    \end{gathered}
    \\[1ex]
    \begin{gathered}[t]
      \PR[\mRA]{x}{r}
      \\
      \hbox{\begin{tikzinline}[node distance=.5em and 1.5em]
          \raevent{a}{\Qw{x}\mid\DR[\mRA]{x}{v}}{}
          \xform{xi}{\bForm[\FALSE/\Qr{*}][\FALSE/\Qw{*}]}{above=of a}
          \xform{xd}{v{=}r\limplies\bForm}{below=of a}
          \xo{a}{xd}
        \end{tikzinline}}
    \end{gathered}
    \\[1ex]
    \begin{gathered}[t]
      \PR[\mSC]{x}{r}
      \\
      \hbox{\begin{tikzinline}[node distance=.5em and 1.5em]
          \scevent{a}{\Qw{x}\land\Qsc\mid\DR[\mSC]{x}{v}}{}
          \xform{xi}{\bForm[\FALSE/\Qr{*}][\FALSE/\Qw{*}][\FALSE/\Qsc]}{above=of a}
          \xform{xd}{v{=}r\limplies\bForm}{below=of a}
          \xo{a}{xd}
        \end{tikzinline}}
    \end{gathered}
  \end{gathered}
\end{align*}
%   \caption{Quiescence Examples (Old)}
%   \label{fig:q1}
% \end{figure}
% \begin{figure}[t!]
%   \centering
%   \begin{align*}
  \begin{gathered}
    \begin{gathered}[t]
      \PW{x}{\aExp}
      \\
      \hbox{\begin{tikzinline}[node distance=.5em and 1.5em]
          \event{a}{\Qra \land\aExp{=}v\mid\DW{x}{v}}{}
          \xform{xi}{\bForm[\FALSE/\Qrlx]}{above=of a}
          \xform{xd}{\bForm\land \aExp{=}\aVal}{below=of a}
          %\xform{xd}{\bForm[(\Qrlx\land\aExp{=}\aVal)/\Qrlx]}{below=of a}
          \xo[xright]{a}{xd}
        \end{tikzinline}}
    \end{gathered}
    \\[1ex]
    \begin{gathered}[t]
      \PW[\mREL]{x}{\aExp}
      \\
      \hbox{\begin{tikzinline}[node distance=.5em and 1.5em]
          \raevent{a}{\Qra\land\Qrlx \land\aExp{=}v\mid\DW[\mREL]{x}{v}}{}
          \xform{xi}{\bForm[\FALSE/\Qrlx]}{above=of a}
          \xform{xd}{\bForm\land \aExp{=}\aVal}{below=of a}
          %\xform{xd}{\bForm[(\Qrlx\land\aExp{=}\aVal)/\Qrlx]}{below=of a}
          \xo[xright]{a}{xd}
        \end{tikzinline}}
    \end{gathered}
    \\[1ex]
    \begin{gathered}[t]
      \PW[\mSC]{x}{\aExp}
      \\
      \hbox{\begin{tikzinline}[node distance=.5em and 1.5em]
          \scevent{a}{\Qra\land\Qrlx\land\Qsc \land\aExp{=}v\mid\DW[\mSC]{x}{v}}{}
          \xform{xi}{\bForm[\FALSE/\Qrlx][\FALSE/\Qsc]}{above=of a}
          \xform{xd}{\bForm\land \aExp{=}\aVal}{below=of a}
          %\xform{xd}{\bForm[(\Qrlx\land\aExp{=}\aVal)/\Qrlx]}{below=of a}
          \xo[xright]{a}{xd}
        \end{tikzinline}}
    \end{gathered}
  \end{gathered}
  &&
  \begin{gathered}
    \begin{gathered}[t]
      \PR{x}{r}
      \\
      \hbox{\begin{tikzinline}[node distance=.5em and 1.5em]
          \event{a}{\Qra\mid\DR{x}{v}}{}
          \xform{xi}{\bForm[\FALSE/\Qrlx]}{above=of a}
          \xform{xd}{v{=}r\limplies\bForm}{below=of a}
          \xo[xright]{a}{xd}
        \end{tikzinline}}
    \end{gathered}
    \\[1ex]
    \begin{gathered}[t]
      \PR[\mACQ]{x}{r}
      \\
      \hbox{\begin{tikzinline}[node distance=.5em and 1.5em]
          \raevent{a}{\Qra\mid\DR[\mACQ]{x}{v}}{}
          \xform{xi}{\bForm[\FALSE/\Qrlx][\FALSE/\Qra]}{above=of a}
          \xform{xd}{v{=}r\limplies\bForm}{below=of a}
          \xo[xright]{a}{xd}
        \end{tikzinline}}
    \end{gathered}
    \\[1ex]
    \begin{gathered}[t]
      \PR[\mSC]{x}{r}
      \\
      \hbox{\begin{tikzinline}[node distance=.5em and 1.5em]
          \scevent{a}{\Qra\land\Qsc\mid\DR[\mSC]{x}{v}}{}
          \xform{xi}{\bForm[\FALSE/\Qrlx][\FALSE/\Qra][\FALSE/\Qsc]}{above=of a}
          \xform{xd}{v{=}r\limplies\bForm}{below=of a}
          \xo[xright]{a}{xd}
        \end{tikzinline}}
    \end{gathered}
  \end{gathered}
\end{align*}

%   \caption{Quiescence Examples (New)}
%   \label{fig:q2}
% \end{figure}

In the logic, we remove the symbols $\Qw{\aLoc}$ and $\Qr{\aLoc}$.
Previously, we had given the semantics of $\mRA$ access using $\Qw{*}$ and
$\Qr{*}$, which were encoded using $\Qw{\aLoc}$ and $\Qr{\aLoc}$.  With these
gone, we introduce the quiescence symbol $\Qrlx$ and $\Qra$.  Thus, the only
quiescence symbols required are $\Qrlx$, $\Qra$ and $\Qsc$.
\reffig{fig:q-co} shows the difference with the semantics of \textsection\ref{sec:sync}.
% \reffig{fig:q1}.  In the new
% interpretation, see \reffig{fig:q2}.

% To see how the quiescence symbols are used, consider the following examples:

\begin{comment}
  Let formulae $\QS{\aLoc}{\amode}$ and $\QL{\aLoc}{\amode}$ be defined:
  \begin{align*}
    \QS{\aLoc}{\mRLX}&=\Qr{\aLoc}\land\Qw{\aLoc}
    &\QL{\aLoc}{\mRLX}&=\Qw{\aLoc}
    \\
    \QS{\aLoc}{\mRA}&=
    \Qr{*}\land\Qw{*} %\textstyle\bigwedge_\bLoc \Qr{\bLoc}\land\Qw{\bLoc}
    &\QL{\aLoc}{\mRA}&=\Qw{\aLoc}
    \\
    \QS{\aLoc}{\mSC}&=
    \Qr{*}\land\Qw{*} %\textstyle\bigwedge_\bLoc \Qr{\bLoc}\land\Qw{\bLoc}
    \land \Qsc
    &\QL{\aLoc}{\mSC}&=\Qw{\aLoc}\land\Qsc
  \end{align*}
  Let substitutions $[\aForm/\QS{\aLoc}{\amode}]$ and  $[\aForm/\QL{\aLoc}{\amode}]$ be defined:
  \begin{align*}
    [\aForm/\QS{\aLoc}{\mRLX}] &= [\aForm/\Qw{\aLoc}]
    &{} [\aForm/\QL{\aLoc}{\mRLX}] &= [\aForm/\Qr{\aLoc}]
    \\
    [\aForm/\QS{\aLoc}{\mRA}] &= [\aForm/\Qw{\aLoc}]
    &{} [\aForm/\QL{\aLoc}{\mRA}] &= [\aForm/\Qr{*},\aForm/\Qw{*}]
    \\
    [\aForm/\QS{\aLoc}{\mSC}] &= [\aForm/\Qw{\aLoc},\aForm/\Qsc]
    &{} [\aForm/\QL{\aLoc}{\mSC}] &= [\aForm/\Qr{*},\aForm/\Qw{*},\aForm/\Qsc]
  \end{align*}
  Update \refdef{def:pomsets-trans} from: % (\ref{S4}/\ref{L4} unchanged):
  \begin{enumerate}
  \item[\ref{S3})]
    $\labelingForm(\aEv) \rimplies \aExp{=}\aVal\land\QS{\aLoc}{\amode}$,
  \item[\ref{L3})]
    $\labelingForm(\aEv) \rimplies \QL{\aLoc}{\amode}$,
  \item[\ref{T3})]
    $\labelingForm(\aEv) \rimplies \labelingForm_1(\aEv)[\TRUE/\Qr{*}][\TRUE/\Qw{*}][\TRUE/\Qsc]$,
  \end{enumerate}
  \begin{enumerate}
  \item[\ref{S4})]
    $\aTr{\bEvs}{\bForm} \rimplies \bForm \land\aExp{=}\aVal$,
  \item[\ref{S5})]
    $\aTr{\cEvs}{\bForm} \rimplies \bForm[\FALSE/\QS{\aLoc}{\amode}]$,
  \item[\ref{L4})]
    $\aTr{\bEvs}{\bForm} \rimplies \aVal{=}\aReg\limplies\bForm$, 
  \item[\ref{L5})]
    $\aTr{\cEvs}{\bForm} \rimplies \bForm[\FALSE/\QL{\aLoc}{\amode}]$.
  \end{enumerate}
\end{comment}

\begin{definition}
  Let formulae $\QS{}{\amode}$ and $\QL{}{\amode}$ be defined:
  \begin{scope}
    \small
    \begin{align*}
      \QS{}{\mRLX}&=\Qra
      &\QL{}{\mRLX}&=\Qra
      \\
      \QS{}{\mRA}&=\Qra\land\Qrlx
      &\QL{}{\mRA}&=\Qra
      \\
      \QS{}{\mSC}&=\Qra\land \Qrlx \land \Qsc
      &\QL{}{\mSC}&=\Qra\land\Qsc
    \end{align*}
  \end{scope}
  Let substitutions $[\aForm/\QS{}{\amode}]$ and  $[\aForm/\QL{}{\amode}]$ be defined:
  \begin{scope}
    \small
    \begin{align*}
      [\aForm/\QS{}{\mRLX}] &= [\aForm/\Qrlx]
      &{} [\aForm/\QL{}{\mRLX}] &= [\aForm/\Qrlx]
      \\
      [\aForm/\QS{}{\mRA}] &= [\aForm/\Qrlx]
      &{} [\aForm/\QL{}{\mRA}] &= [\aForm/\Qrlx,\aForm/\Qra]
      \\
      [\aForm/\QS{}{\mSC}] &= [\aForm/\Qrlx,\aForm/\Qsc]
      &{} [\aForm/\QL{}{\mSC}] &= [\aForm/\Qrlx,\aForm/\Qra,\aForm/\Qsc]
    \end{align*}
  \end{scope}
\end{definition}
\begin{definition}%[\xCO/\xRASC]
  Update \refdef{def:pomsets-trans} to: %and \ref{def:pomsets-fj} to: % (\ref{S4}/\ref{L4} unchanged):
  \begin{enumerate}
  \item[\ref{S3})]
    $\labelingForm(\aEv) \rimplies \QS{}{\amode}\land\aExp{=}\aVal$,
  \item[\ref{L3})]
    $\labelingForm(\aEv) \rimplies \QL{}{\amode}$,
    % \item[\ref{F3})]
    %   $\labelingForm(\aEv) \rimplies \Qrlx\land\Qra\land\Qsc\land\labelingForm_1(\aEv)$, 
    % \item[\ref{T3})]
    %   $\labelingForm(\aEv) \rimplies \labelingForm_1(\aEv)[\TRUE/\Q{}]$, %[\TRUE/\Qrlx][\TRUE/\Qra][\TRUE/\Qsc]$,
  \end{enumerate}
  \begin{enumerate}
  \item[\ref{S4})]
    $\aTr{\bEvs}{\bForm} \rimplies \bForm \land\aExp{=}\aVal$,
  \item[\ref{S5})]
    $\aTr{\cEvs}{\bForm} \rimplies \bForm[\FALSE/\QS{}{\amode}]$,
  \item[\ref{L4})]
    $\aTr{\bEvs}{\bForm} \rimplies \aVal{=}\aReg\limplies\bForm$, 
  \item[\ref{L5})]
    $\aTr{\cEvs}{\bForm} \rimplies \bForm[\FALSE/\QL{}{\amode}]$.
  \end{enumerate}
\end{definition}

The most interesting examples in \reffig{fig:q2} concern $\mRA$ access.
Every independent transformer substitutes $[\FALSE/\Qrlx]$.  $\Qrlx$ is a
precondition for any releasing write $\aEv$, ensuring that all preceding
events must are ordered before $\aEv$.  Conversely, $\Qra$ is a precondition
of every event.  The independent transformer for any acquiring read $\aEv$
substitutes $[\FALSE/\Qra]$, ensuring that all following events must be
ordered after $\aEv$.

% As before, the substitution in \ref{S4} ensures that left merges are not
% quiescent (\refex{ex:left-merge}).

Item \ref{seq-reorder} of \refdef{def:independency-co} ensures
coherence.  This definition is incompatible with asynchronous $\FORK{}$
parallelism of \refdef{def:pomsets-group}, where we expect executions such
as:
\begin{gather*}
  \FORK{\THREAD{\PR{x}{r}}}\SEMI \PW{x}{1}
  \\
  \hbox{\begin{tikzinline}[node distance=0.5em and 1.5em]
      \event{a}{\DR{x}{1}}{}
      \event{b}{\DW{x}{1}}{right=3em of a}
      \rf{b}{a}
    \end{tikzinline}}
\end{gather*}
Item \ref{seq-reorder} would require $\DRP{x}{1} \xwk \DWP{x}{1}$, forbidding
this.

% , since \ref{T3full} substitutes
% $\TRUE$ for every quiescence symbol.  Preconditions of augment-minimal
% pomsets in $\sem{\FORK{\THREAD{\aCmd}}}$ contain no quiescence symbols.
% Instead, preconditions of augment-minimal pomsets in
% $\sem{\FORKJOIN{\THREAD{\aCmd}}}$ are saturated with quiescence symbols.

% One must be careful, however, due to \emph{inconsistency}.  Consider that
% \texttt{x=0;x=1} should not have completed pomset with only $\DWP{x}{0}$.

% \eqref{seq-reorder} does not do the right thing with fork either.  If you
% want to enforce coherence this way then you need to use fork-join as the
% sequential combinator, rather than fork.


% [We drop $\reorder$ because incompatible with $\sFORK{}$.  If you want to
% use $\reorder$, then you need to use fork-join as the sequential
% combinator, rather than fork.]

% We can then encode coherence as follows.
% \begin{enumerate}
%   \setcounter{enumi}{\value{pomsetXSemiCount}}
% \item if $\bEv\in\aEvs_1$ and $\aEv\in\aEvs_2$ either $\bEv<\aEv$ or
%   $a\reorder\labeling_2(\aEv)$.
% \end{enumerate}


% Access modes can be encoded in the independency relation, indexing labels by
% $\amode$, but the extra flexibility of the logic is necessary for \armeight{}
% (see \textsection\ref{sec:downgrade}).  Using independency, one would also
% need another way to define completed pomsets.  Finally, this use of
% independency is incompatible with fork (see \textsection\ref{sec:co}).


% If we move coherence to independency (and use fork-join), we have the
% following, assuming that each register occurs at most once.
% \begin{align*}
%   \QS{}{\mSC}&=\Q{\mSC}
%   &\QS{}{\mRA}&=\Q{\mRA}
%   &\QS{}{\mRLX}&=\Qx{\aLoc}
%   \\
%   \QL{}{\mSC}&=\Q{\mSC}
%   &\QL{}{\mRA}&=\Qw{\aLoc}
%   &\QL{}{\mRLX}&=\Qw{\aLoc}
%   \\
%   \DS{\aLoc}{\mSC}{\bForm}&=\bForm[\FALSE/\D]
%   &\DS{\aLoc}{\mRA}{\bForm}&=\bForm[\FALSE/\D]
%   &\DS{\aLoc}{\mRLX}{\bForm}&=\bForm[\TRUE/\Dx{\aLoc}] 
%   \\
%   \DL{\aLoc}{\mSC}&=\Dx{\aLoc}
%   &\DL{\aLoc}{\mRA}&=\Dx{\aLoc}
%   &\DL{\aLoc}{\mRLX}&=\TRUE
% \end{align*}

% % $\QS{}{\mRLX}=\TRUE$ and otherwise $\QS{}{\amode}=\Q{\amode}$.

% % $\QL{}{\mSC}=\Q{\mSC}$ and otherwise $\QL{}{\amode}=\TRUE$.

% % $\DS{\aLoc}{\mRLX}{\bForm}=\bForm[\TRUE/\Dx{\aLoc}]$ and otherwise
% % $\DS{\aLoc}{\amode}{\bForm}=\bForm[\FALSE/\D]$.

% % $\DL{\aLoc}{\mRLX}=\TRUE$ and otherwise $\DL{\aLoc}{\amode}=\Dx{\aLoc}$.

% % \begin{definition}$\phantom{\;}$\par
% %   $\QS{}{\mRLX}=\TRUE$ and otherwise $\QS{}{\amode}=\Q{\amode}$.

% %   $\QL{}{\mSC}=\Q{\mSC}$ and otherwise $\QL{}{\amode}=\TRUE$.

%   \noindent
%   \begin{enumerate}
%   \item[\ref{S3})] $\labelingForm(\aEv)$ implies
%     \begin{math}
%       \aExp{=}\aVal \land \RW \land \QS{}{\amode}
%     \end{math},
%   \item[\ref{S4})] $\aTr{\bEvs}{\bForm}$ implies
%     \begin{math}
%       \aExp{=}\aVal \land \DS{\aLoc}{\amode}{\bForm[\aExp/{\aLoc}]}
%     \end{math},
%   \item[\ref{S5})] $\aTr{\emptyset}{\bForm}$ implies
%     \begin{math}
%       \lnot\Q{\mRA} \land \DS{\aLoc}{\amode}{\bForm[\aExp/{\aLoc}]}
%     \end{math}
%   \end{enumerate}

%   \noindent
%   \begin{enumerate}
%   \item[\ref{L3})] $\labelingForm(\aEv)$ implies
%     \begin{math}
%       \RO \land \QL{}{\amode}
%     \end{math},
%   \item[\ref{L4})] $\aTr{\bEvs}{\bForm}$ implies
%     \begin{math}
%       (\aVal{=}\aReg) \limplies \bForm[\aReg/{\aLoc}]
%     \end{math}
%   \item[\ref{L5})] $\aTr{\emptyset}{\bForm}$ implies
%     \begin{math}
%       \DL{\aLoc}{\amode} \land \lnot\Q{\mRA} \land (\RW \limplies
%       (\aVal{=}\aReg\lor\aLoc{=}\aReg) \limplies \bForm[\aReg/{\aLoc}] ).
%     \end{math}
%   \end{enumerate}

\begin{comment}
  \subsection{Completed Pomsets and Fork}
  \label{sec:fork}

  It is sometimes useful to distinguish \emph{terminated} or \emph{completed}
  executions from partial executions.  For example in
  \begin{math}
    \sem{\PW{x}{1}\SEMI\PW{y}{1}},
  \end{math}
  we expect completed executions to include two write actions.  Note that this
  is different from being downset-maximal.
  \begin{gather}
    \nonumber
    \PW{x}{0} \SEMI \PW{x}{1}
    \PAR
    \PR{x}{r}\SEMI\PR{x}{s}\SEMI\IF{s}\THEN\PW{y}{1}\FI
    \\
    \label{down1}
    \hbox{\begin{tikzinline}[node distance=0.5em and 1.5em]
        \event{a}{\DW{x}{0}}{}
        \event{b}{\DW{x}{1}}{right=of a}
        \event{c}{\DR{x}{1}}{right=3em of b}
        \event{d}{\DR{x}{0}}{right=of c}
        \wk{a}{b}
        \rf{b}{c}
        \rf[out=-20,in=-160]{a}{d}
      \end{tikzinline}}
    \\
    \label{down2}
    \hbox{\begin{tikzinline}[node distance=0.5em and 1.5em]
        \event{a}{\DW{x}{0}}{}
        \event{b}{\DW{x}{1}}{right=of a}
        \event{c}{\DR{x}{0}}{right=3em of b}
        \event{d}{\DR{x}{1}}{right=of c}
        \event{e}{\DW{y}{1}}{right=of d}
        \po{d}{e}
        \wk{a}{b}
        \rf[out=-20,in=-160]{a}{c}
        \rf[out=-20,in=-160]{b}{d}
      \end{tikzinline}}
  \end{gather}
  \eqref{down1} is a downset of \eqref{down2}, but both are completed. 

  For pomsets with predicate transformers, we identify \emph{completion} with
  \emph{quiescence.}
  \begin{definition}
    \label{def:completed}
    A pomset with predicate transformers $\aPS$ is \emph{completed} if
    $\aTr{\aEvs}{\aSym} \rimplies \aSym$, for every quiescence symbol $\aSym$.
  \end{definition}
  For example, there are no pomsets in $\sem{\ABORT}$ that are completed,
  whereas the augment-minimal pomset of $\sem{\SKIP}$ (which has the identity
  transformer) is completed.

  % While this definition is sensible for single \emph{threads}, it is less
  % satisfying for thread \emph{groups}.  To see why, consider that in
  % $\sem{\FORK{\THREAD{\aCmd}}}$:
  % \begin{itemize}
  % \item by \ref{T3full}, quiescence symbols and the symbol $\RW$ have been
  %   substituted out of preconditions $\labelingForm(\aEv)$,
  % \item by \ref{F4}, every predicate transformer $\aTr{\bEvs}{}$ is the
  %   identity function. %, for any $\bEvs$.
  % \end{itemize}
  % Every pomset in $\sem{\FORK{\aGrp}}$ is completed, by definition. As a
  % result, in general, $\sem{\FORK{\THREAD{\aCmd}}}\neq\sem{\aCmd}$.

  The $\FORK{}$ operation is asynchronous: In
  $\sem{\aCmd_1\SEMI\aCmd_2\SEMI\FORK{\aGrp}\SEMI \aCmd_3}$, the only order enforced
  between $\sem{\aCmd_3}$ and $\sem{\aGrp}$ comes from
  quiescence preconditions in $\sem{\aGrp}$; the transformer of
  $\sem{\FORK{\aGrp}}$ is the identity transformer, thus $\sem{\aGrp}$ runs
  concurrently with $\sem{\aCmd_3}$.  In addition, if $\aCmd_2$ includes no
  releases and the locations of $\aCmd_2$ are disjoint from those of $\aGrp$,
  then $\sem{\aGrp}$ run concurrently with $\sem{\aCmd_2\SEMI\aCmd_3}$.
  % $\FORK{}$ %(\textsection\ref{sec:pomsets-trans})
  % does not introduce barriers:
  \begin{gather*}
    \PW{x}{1}\SEMI \PW{y}{1}\SEMI\FORK{\THREAD{\PW{x}{2}}}
    \\
    \hbox{\begin{tikzinline}[node distance=0.5em and 1.5em]
        \event{a}{\DW{x}{1}}{}
        \event{b}{\DW{y}{1}}{right=of a}
        \event{c}{\DW{x}{2}}{right=3em of b}
        \wk[out=-20,in=-160]{a}{c}
      \end{tikzinline}}
  \end{gather*}
  % In fact, perhaps surprisingly,
  % \begin{math}
  %   \sem{\PR{x}{r}\SEMI\FORK{\THREAD{\PW{x}{1}}}} = \sem{\FORK{\THREAD{\PW{x}{1}}}\SEMI\PR{x}{r}}.
  % \end{math}
  % Order between the threads
  % can be enforced using synchronization.  For example, the ``backwards'' read
  % above is forbidden in:
  % \begin{gather*}
  %   \PR{x}{r}\SEMI\PW[\mREL]{z}{1}\SEMI\FORK{\THREAD{\IF{\PR[\mACQ]{z}{}}\THEN\PW{x}{1}\FI}}
  %   \\
  %   \hbox{\begin{tikzinline}[node distance=0.5em and 1.5em]
  %     \event{a}{\DR{x}{1}}{}
  %     \event{b}{\DW{z}{1}}{right=of a}
  %     \event{c}{\DR{z}{1}}{right=3em of b}
  %     \event{d}{\DW{x}{1}}{right=of c}
  %     \rf{b}{c}
  %     \sync{a}{b}
  %     \sync{c}{d}
  %     \rf[out=-160,in=-20]{d}{a}
  %   \end{tikzinline}}
  % \end{gather*}

  % [Proposal: add level of syntax for full programs and do all thread
  % substitutions there... have one level of syntax around for each semantic
  % categories.]

  \subsection{Fork-Join Parallelism}

  In this subsection, we model a variant of our language that removes the
  asynchronous $\FORK{}$ operation and adds a synchronous \emph{fork-join},
  which we define using an asymmetric operator for parallel composition, as in
  \cite{DBLP:conf/icfp/FerreiraHJ96}.

  We remove thread groups as a separate category in the syntax and semantics.
  The transformer for the \emph{left composition} $\aCmd_1\LPAR\aCmd_2$ takes
  the register state from the left, but quiescence from both sides.
  \begin{align*}
    \aCmd
    \BNFDEF& \ABORT
    \BNFSEP \SKIP
    \BNFSEP \LET{\aReg}{\aExp}
    % \BNFSEP \PR[\amode]{\aLoc}{\aReg}
    % \BNFSEP \PW[\amode]{\aLoc}{\aExp}
    \BNFSEP \PRREF[\amode]{\cExp}{\aReg}
    \BNFSEP \PWREF[\amode]{\cExp}{\aExp}
    % \BNFSEP \PA{\aLoc}{\aExp} 
    \\[-.5ex]
    \BNFSEP& \aCmd_1 \LPAR \aCmd_2
    \BNFSEP \aCmd_1 \SEMI \aCmd_2
    \BNFSEP \IF{\aExp} \THEN \aCmd_1 \ELSE \aCmd_2 \FI  
  \end{align*}

  \begin{definition}
    If $\aPS \in (\aPSS_1\sLPAR\aPSS_2)$ then
    $(\exists\aPS_1\in\aPSS_1)$ $(\exists\aPS_2\in\aPSS_2)$
    % there are $\aPS_1\in\aPSS_1$ and $\aPS_2\in\aPSS_2$ such that
    \begin{enumerate}
      \setcounter{enumi}{\value{pomsetPreParCount}}
    \item[\ref{par-E}--\ref{par-kappa2})]
      as for $\sPAR{}{}$ in \refdef{def:pomsets-pre},
    \item \label{par-tau1}
      $\aTr{\bEvs}{\bForm} \rimplies \aTr[1]{\bEvs}{\bForm}$,
    \item \label{par-tau2}
      $\aTr{\bEvs}{\aSym} \rimplies \aTr[2]{\bEvs}{\aSym}$,
      for every quiescence symbol $\aSym$.
    \end{enumerate}
  \end{definition}

  We now interpret pomsets with predicate transformers directly as
  \emph{top-level} (rather than passing through the intermediate interpretation
  of thread groups as pomsets with preconditions).  We only consider
  \emph{completed} pomsets with predicate transformers to be top-level.
  \begin{definition}
    \label{def:pomsets-top2}
    \labeltext[\ensuremath{\sTOP{}}]{}{sTOP2}
    \noindent
    If $\aPS \in \sTOP{\aPSS}$ then
    $(\exists\aPS_1\in\aPSS)$
    \begin{enumerate}
    \item[1-5)] as in \refdef{def:pomsets-top},
      % \item  \label{top2-E} % [{\labeltext[T1]{T1)}{T1}}]
      %   $\aEvs=\aEvs_1$,
      % \item  \label{top2-lambda} % [{\labeltext[T2]{T2)}{T2}}]
      %   $\labelingAct(\aEv) = \labelingAct_1(\aEv)$,
      % \item  \label{top2-le} % [{\labeltext[T2]{T2)}{T2}}]
      %   if $\bEv\le_1\aEv$ then $\bEv\le\aEv$, 
      % \item  \label{top2-kappa-write} % [{\labeltext[T3]{T3)}{T3}}]
      %   if $\labelingAct_1(\aEv)$ is a write, $\labelingForm_1(\aEv) [\TRUE/\Q{}][\TRUE/\RW]$ is a tautology,
      % \item  \label{top2-kappa-read} % [{\labeltext[T3]{T3)}{T3}}]
      %   if $\labelingAct_1(\aEv)$ is a read,
      %   $\labelingForm_1(\aEv) [\TRUE/\Q{}][\FALSE/\RW]$ is a tautology and
      %   $\aEv$ is fulfilled (\refdef{def:fulfilled}),
    \item[6)]  \label{top2-tau} % [{\labeltext[T3]{T3)}{T3}}]
      $\aTr[1]{\aEvs_1}{\aSym} \rimplies \aSym$, for every quiescence symbol $\aSym$.
    \end{enumerate}  
  \end{definition}

  \subsection{Fork-Join Parallelism 2}
  \label{sec:join2}

  In this subsection, we model a variant of our language that removes the
  asynchronous $\FORK{}$ operation and adds a synchronous $\FORKJOIN{}$.
  \begin{align*}
    \aCmd
    \BNFDEF& \ABORT
    \BNFSEP \SKIP
    \BNFSEP \LET{\aReg}{\aExp}
    % \BNFSEP \PR[\amode]{\aLoc}{\aReg}
    % \BNFSEP \PW[\amode]{\aLoc}{\aExp}
    \BNFSEP \PRREF[\amode]{\cExp}{\aReg}
    \BNFSEP \PWREF[\amode]{\cExp}{\aExp}
    % \BNFSEP \PA{\aLoc}{\aExp} 
    \\[-.5ex]
    \BNFSEP& \FORKJOIN{\aGrp}
    \BNFSEP \aCmd_1 \SEMI \aCmd_2
    \BNFSEP \IF{\aExp} \THEN \aCmd_1 \ELSE \aCmd_2 \FI
  \end{align*}
  In $\PBR{\aCmd_1\SEMI\FORKJOIN{\aGrp}\SEMI\aCmd_2}$, ${\aCmd_1}$ must
  complete before ${\aGrp}$ begins, and threads in ${\aGrp}$ must complete
  before ${\aCmd_2}$ begins.  Thus $\PBR{\FORKJOIN{\THREAD{\PR{x}{r}}}}$ acts
  like a full fence.
  % 
  As modeled here, however, if $\aGrp$ is empty, no order is imposed between
  ${\aCmd_1}$ and ${\aCmd_2}$.  Thus
  $\sem{\FORKJOIN{\THREAD{\SKIP}}}=\sem{\SKIP}$.

  To model $\FORKJOIN{}$, we give the semantics of thread groups using pomsets
  with preconditions \emph{and termination}.
  \begin{definition}
    \label{def:pomsets-term}
    A \emph{pomset with preconditions and termination} is a pomset with
    preconditions (\refdef{def:pomsets-pre}) together with a termination
    predicate (notation $\TICK$).
  \end{definition}

  % The definition is a small change relative to that of
  % \textsection\ref{sec:pomsets-trans}.

  % Define $\sTHREAD{}$ to transform a pomset with predicate transformers into
  % a pomset with preconditions and termination by dropping the predicate
  % transformer and setting $\TICK$ to indicate whether the pomset was
  % completed.

  % Extend the definition of $\sNIL$ so that $\TICK$ is true.

  % Extend the definition of $\sPAR{}{}$ to handle for $\TICK$ by adding the
  % following.
  % \begin{enumerate}
  %   \setcounter{enumi}{\value{pomsetPreParCount}}
  % \item \label{par-tick} if $\TICK$ then $\TICK_1$ and $\TICK_2$.
  % \end{enumerate}

  % Similarly, $\sFORKJOIN{}$ extends $\sFORK{}$ by adding the following.
  % % \noindent
  % % If $\aPS \in \sFORKJOIN{\aPSS}$ then
  % % $(\exists\aPS_1\in\aPSS)$
  % \begin{enumerate}
  %   \setcounter{enumi}{\value{pomsetXForkCount}}
  % \item $\TICK_1$.
  % \end{enumerate}

  \begin{definition}%$\phantom{\;}$\par
    \label{def:pomsets-fj}
    % \noindent
    % If $\aPS\in\sNIL$ then $\aEvs = \emptyset$ and $\TICK$.
    \noindent
    If $\aPS \in \sTHREAD{\aPSS}$ then $(\exists\aPS_1\in\aPSS)$
    \begin{enumerate}
      \setcounter{enumi}{\value{pomsetXThreadCount}}
    \item[\ref{thread-E}--\ref{thread-kappa})] as for $\sTHREAD{}$ in \refdef{def:pomsets-group}, %, repeating 3 below,% \reffig{fig:full},    
      % \item[\ref{T3full})]
      %   $\labelingForm(\aEv)$ implies
      %   $\labelingForm_1(\aEv) [\TRUE/\Q{}][\TRUE/\RW]$ if $\labelingAct_1(\aEv)$ is a write,
      %   \\
      %   $\labelingForm(\aEv)$ implies
      %   $\labelingForm_1(\aEv) [\TRUE/\Q{}][\FALSE/\RW]$ otherwise.
      
    \item %[{\labeltext[T4]{T4)}{T4}}]
      if $\TICK$ then $\aPS$ is completed (\refdef{def:completed}).
      % $\aTr{\aEvs}{\Q{}} \rimplies \Q{}$.
    \end{enumerate}

    \noindent
    If $\aPS \in \sPAR{\aPSS_1}{\aPSS_2}$ then $(\exists\aPS_1\in\aPSS_1)$
    $(\exists\aPS_2\in\aPSS_2)$
    \begin{enumerate}
      \setcounter{enumi}{\value{pomsetPreParCount}}
    \item[\ref{par-E}--\ref{par-kappa2})] as for $\sPAR{}{}$ in
      \refdef{def:pomsets-pre},
    \item \label{par-tick} $\TICK \rimplies \TICK_1\land\TICK_2$.
    \end{enumerate}

    \noindent
    If $\aPS \in \sFORKJOIN{\aPSS}$ then $(\exists\aPS_1\in\aPSS)$
    \begin{enumerate}
      \setcounter{enumi}{\value{pomsetXForkCount}}
    \item[\ref{fork-E}--\ref{fork-le})] as for $\sFORK{}$ in \refdef{def:pomsets-group},
    \item
      $\labelingForm(\aEv) \rimplies \labelingForm_1(\aEv)$,    
    \item 
      $\labelingForm(\aEv) \rimplies \aSym$, for every quiescence symbol $\aSym$,
    \item
      $\aTr{\bEvs}{\bForm} \rimplies \bForm$, if $\bEvs=\aEvs$ and $\TICK_1$,
    \item %[{\labeltext[F5]{F5)}{F5}}]
      $\aTr{\bEvs}{\bForm} \rimplies \bForm[\FALSE/\Q{}]$, otherwise.
    \end{enumerate}
  \end{definition}
  \begin{definition}
    Update \refdef{def:sem-funs} to include:
    \begin{align*}
      \sem{\FORKJOIN{\aGrp}} = \sFORKJOIN{}\sem{\aGrp}  
    \end{align*}
  \end{definition}

  We embed pomsets with predicate transformers into pomsets with preconditions
  and termination using {completion}.  The rules for thread groups keep track
  of the termination predicate.
  As noted in \textsection\ref{sec:fork}, every pomset in $\sem{\FORK{\aGrp}}$ is
  completed.  In contrast, a pomset in $\sem{\FORKJOIN{\aGrp}}$ is completed
  only if every thread in $\aGrp$ is completed.

  Top-level thread groups do not need quiescence symbols; thus, $\sTHREAD{}$
  removes all quiescence symbols by substitution.  However, $\sFORKJOIN{\aPSS}$
  adds every possible quiescence symbol as a precondition to the events of
  $\aPSS$.  For example, the preconditions of $\sem{\THREAD{\aCmd}\PAR\NIL}$ do
  not contain quiescence symbols.  Instead, the preconditions of
  $\sem{\FORKJOIN{\THREAD{\aCmd}\PAR\NIL}}$ are saturated with them.  As a
  result, in completed top-level pomsets of
  $\sem{\aCmd_1\SEMI\FORKJOIN{\aGrp}}$, all of the events from
  $\sem{\aCmd_1}$ must precede those of $\sem{\aGrp}$.

  A similar thing happens with predicate transformers.  Thread groups in
  $\sem{\THREAD{\aCmd}\PAR\NIL}$ do not contain predicate transformers.
  Instead, all of the independent predicate transformers of
  $\sem{\FORKJOIN{\THREAD{\aCmd}\PAR\NIL}}$ take $\bForm$ to
  $\bForm[\FALSE/\Q{}]$.  As a result, in completed top-level pomsets of
  $\sem{\FORKJOIN{\aGrp}\SEMI\aCmd_2}$, all of the events from $\sem{\aGrp}$
  must precede those of $\sem{\aCmd_2}$.


  % The $\JOIN$ operation requires a full synchronization, but $\FORK{}$ does
  % not.  The following execution is allowed.
  % \begin{gather*}
  %   \PR{x}{r}\SEMI\FORKJOIN{\THREAD{\PW{x}{1}}}\SEMI\PW{y}{1}
  %   \\
  %   \hbox{\begin{tikzinline}[node distance=0.5em and 1.5em]      
  %     \event{a}{\DW{x}{1}}{}
  %     \event{b}{\DW{y}{1}}{right=3em of a}
  %     \event{c}{\DR{x}{1}}{left=3em of a}
  %     \rf{a}{c}
  %     \sync{a}{b}
  %     \sync[out=-20,in=-160]{c}{b}
  %   \end{tikzinline}}
  % \end{gather*}
  % Synchronization can be added to 
  % This asymmetry arises naturally when using pomsets with preconditions to
  % model thread groups.
\end{comment}
