\section{Discussion}
\label{sec:discussion}

\subsection{Combining Address Calculation and If-Closure}
\label{sec:semcaaddr}

\refdef{def:semaddr} is naive with respect to merging events.  Consider the
following example:
\begin{align*}
  \begin{gathered}
    \PW{\REF{r}}{0}\SEMI \PW{\REF{0}}{\BANG r}
    \\
    \hbox{\begin{tikzinline}[node distance=1.5em]
        \eventl{c}{a}{r\EQ1\mathbin{\mid}\DW{\REF{1}}{0}}{}
        \eventl{d}{b}{r\EQ1\mathbin{\mid}\DW{\REF{0}}{0}}{right=of a}
      \end{tikzinline}}
  \end{gathered}
  &&
  \begin{gathered}
    \PW{\REF{r}}{0}\SEMI \PW{\REF{0}}{\BANG r}
    \\
    \hbox{\begin{tikzinline}[node distance=1.5em]
        \eventl{d}{a}{r\EQ0\mathbin{\mid}\DW{\REF{0}}{0}}{}
        \eventl{e}{b}{r\EQ0\mathbin{\mid}\DW{\REF{0}}{1}}{right=of a}
        \wki{a}{b}
      \end{tikzinline}}
  \end{gathered}
\end{align*}
Merging, we have:
% Thus, the disjunction closure also includes both of the following: % By using \!$\PAR$\!, it also includes:
\begin{align*}
  \begin{gathered}
    \IF{\aExp}\THEN
    \PW{\REF{r}}{0}\SEMI \PW{\REF{0}}{\BANG r}
    \ELSE
    \PW{\REF{r}}{0}\SEMI \PW{\REF{0}}{\BANG r}
    \FI
    \\
    \hbox{\begin{tikzinline}[node distance=1.5em]
        \eventl{c}{a}{r\EQ1\mathbin{\mid}\DW{\REF{1}}{0}}{}
        \eventl{d}{b}{r\EQ0\lor r\EQ1\mathbin{\mid}\DW{\REF{0}}{0}}{right=of a}
        \eventl{e}{c}{r\EQ0\mathbin{\mid}\DW{\REF{0}}{1}}{right=of b}
        \wki{b}{c}
      \end{tikzinline}}
  \end{gathered}
\end{align*}
% \begin{align*}
%   \begin{gathered}
%     \IF{\aExp}\THEN
%     \PW{\REF{r}}{0}\SEMI \PW{\REF{0}}{\BANG r}
%     \ELSE
%     \PW{\REF{r}}{0}\SEMI \PW{\REF{0}}{\BANG r}
%     \FI
%     \\
%     \hbox{\begin{tikzinline}[node distance=1em]
%       \eventl{c}{a}{r\EQ1\mathbin{\mid}\DW{\REF{1}}{0}}{}
%       \eventl{d}{b}{r\EQ0\lor r\EQ1\mathbin{\mid}\DW{\REF{0}}{0}}{right=of a}
%     \end{tikzinline}}
%   \end{gathered}
% \end{align*}
The precondition of $\DWREF{0}{0}$ is a tautology; however, this is not
possible for $(\PW{\REF{r}}{0}\SEMI \PW{\REF{0}}{\BANG r})$ alone, using \refdef{def:semaddr}.

\refdef{def:semcaaddr}, enables this execution using if-closure.  Under this
semantics, we have:
\begin{align*}
  \begin{gathered}
    \PW{\REF{r}}{0}
    \\
    \hbox{\begin{tikzinline}[node distance=1.5em]
        \eventl{c}{a}{r\EQ1\mathbin{\mid}\DW{\REF{1}}{0}}{}
        \eventl{d}{b}{r\EQ0\mathbin{\mid}\DW{\REF{0}}{0}}{right=of a}
      \end{tikzinline}}
  \end{gathered}
  &&
  \begin{gathered}
    \PW{\REF{0}}{\BANG r}
    \\
    \hbox{\begin{tikzinline}[node distance=1.5em]
        \eventl{d}{b}{r\EQ1\mathbin{\mid}\DW{\REF{0}}{0}}{}
        \eventl{e}{c}{r\EQ0\mathbin{\mid}\DW{\REF{0}}{1}}{right=of b}
      \end{tikzinline}}
  \end{gathered}
\end{align*}
% These pomsets contain inconsistent preconditions.  This is disallowed in
% \jjr{}, but allowed here.
Sequencing and merging: 
\begin{align*}
  \begin{gathered}
    \PW{\REF{r}}{0}
    \SEMI
    \PW{\REF{0}}{\BANG r}
    \\
    \hbox{\begin{tikzinline}[node distance=1.5em]
        \eventl{c}{a}{r\EQ1\mathbin{\mid}\DW{\REF{1}}{0}}{}
        \eventl{d}{b}{r\EQ0\lor r\EQ1\mathbin{\mid}\DW{\REF{0}}{0}}{right=of a}
        \eventl{e}{c}{r\EQ0\mathbin{\mid}\DW{\REF{0}}{1}}{right=of b}
        \wki{b}{c}
      \end{tikzinline}}
  \end{gathered}
\end{align*}
The precondition of $\DWP{\REF{0}}{0}$ is a tautology, as required.

\begin{definition}
  \label{def:semcaaddr}
  Let $\semcaaddr{}$ be defined as in \reffig{fig:seq}, changing
  $\sSTORE{}{}$ and $\sLOAD{}{}$:

  \noindent
  If $\aPS \in \sSTORE[\amode]{\cExp}[\ascope]{\aExp}[\aThrd]$ then
  $(\exists\cVal:\aEvs\fun\Val)$
  $(\exists\aVal:\aEvs\fun\Val)$
  $(\exists\cForm:\aEvs\fun\Formulae)$
  \begin{multicols}{2}
    \begin{enumerate}[topsep=0pt,label=(\textsc{w}\arabic*),ref=\textsc{w}\arabic*]
    \item \label{write-E-ca-addr}
      if $\cForm_\bEv\land\cForm_\aEv$ is satisfiable then $\bEv=\aEv$,
    \item \label{write-lambda-ca-addr}
      $\labelingAct(\aEv) = \DW[\amode]{\REF{\cVal}}[\ascope]{\aVal_\aEv}[\aThrd]$,
    \item \label{write-kappa-ca-addr}
      \begin{math}
        \labelingForm(\aEv) \riff
        \cForm_\aEv
        \land \cExp{=}\cVal_\aEv
        \land \aExp{=}\aVal_\aEv
      \end{math},      
    \item
    \begin{math}
      (\forall\dVal)
      \begin{aligned}[t]
        \aTr{\bEvs}{\bForm} \riff
        &\textstyle\bigwedge_{\aEv\in\aEvs}
        \cForm_\aEv
        \limplies (\cExp{=}\cVal)
        \limplies 
        \bForm[\aExp/\aLoc][\aExp{=}\aVal_\aEv/\Q{\aLoc}]
        \\[-.5ex]
        \land
        &\textstyle (\bigwedge_{\aEv\in\aEvs}\lnot\cForm_\aEv)
        \limplies (\cExp{=}\dVal)
        \limplies 
        \bForm[\aExp/\aLoc][\FALSE/\Q{\aLoc}]
      \end{aligned}
    \end{math}
    \columnbreak
    % \stepcounter{enumi}
    % \item[] \labeltext[\textsc{w}4]{}{write-tau-ca-addr}
      % \begin{enumerate}[leftmargin=0pt]
      % \item \label{write-tau-dep-ca-addr}
      %   \begin{math}
      %     \aTr{\bEvs}{\bForm} \riff 
      %     \cForm_\aEv
      %     \limplies (\cExp{=}\cVal)
      %     \limplies 
      %     \bForm[\aExp/\REF{\cVal}]
      %   \end{math},
      % \item \label{write-tau-empty-ca-addr}
      %   \begin{math}
      %     (\forall\dVal)
      %   \end{math}
      %   \begin{math}
      %     \aTr{\bEvs}{\bForm} \riff 
      %     % (\!\not\exists\aEv\in\aEvs \suchthat \cForm_\aEv)
      %     (\bigwedge_{\aEv\in\aEvs}\lnot\cForm_\aEv)
      %     \limplies (\cExp{=}\dVal)
      %     \limplies 
      %     \bForm
      %     [\aExp/\REF{\dVal}]
      %   \end{math}  
      % \end{enumerate}  
      \stepcounter{enumi}
    \item[] \labeltext[\textsc{w}5]{}{write-term-ca-addr}
      \begin{enumerate}[leftmargin=0pt]
      \item \label{write-term-nonempty-ca-addr}
        $\aTerm \riff \cForm_\aEv \limplies \cExp{=}\cVal_\aEv \land \aExp{=}\aVal_\aEv$,
      \item \label{write-term-empty-ca-addr}
        $\aTerm \riff \bigvee_{\aEv\in\aEvs}\cForm_\aEv$.
      \end{enumerate}
    \end{enumerate}
  \end{multicols}

  \medskip
  \noindent
  If $\aPS \in \sLOAD[\amode]{\aReg}[\ascope]{\cExp}[\aThrd]$ then
  $(\exists\cVal:\aEvs\fun\Val)$
  $(\exists\aVal:\aEvs\fun\Val)$
  $(\exists\cForm:\aEvs\fun\Formulae)$ 
  \begin{multicols}{2}
  \begin{enumerate}[topsep=0pt,label=(\textsc{r}\arabic*),ref=\textsc{r}\arabic*]
  \item \label{read-E-ca-addr}
    if $\cForm_\bEv\land\cForm_\aEv$ is satisfiable then $\bEv=\aEv$,
  \item \label{read-lambda-ca-addr}
    $\labelingAct(\aEv) = \DR[\amode]{\REF{\cVal}}[\ascope]{\aVal_\aEv}[\aThrd]$
  \item \label{read-kappa-ca-addr}
    \begin{math}
      \labelingForm(\aEv) \riff
      \cForm_\aEv
      \land \cExp{=}\cVal_\aEv
      \land \Q{\REF{\cVal}}
    \end{math},
    \stepcounter{enumi}
  \item \label{read-tau-ca-addr}
    \begin{math}
      \begin{aligned}[t]
        (\forall\bReg)
        \aTr{\bEvs}{\bForm} \riff
        &\textstyle\bigwedge_{\aEv\in\aEvs\cap\bEvs}
        \cForm_\aEv
        \limplies (\cExp{=}\cVal_\aEv\limplies\aVal_\aEv{=}\uReg{\aEv})
        \limplies \bForm[\uReg{\aEv}/\aReg]
        \\[-.5ex]
        \land
        &\textstyle\bigwedge_{\aEv\in\aEvs\setminus\bEvs}
        \cForm_\aEv 
        \limplies
        \PBR{(\cExp{=}\cVal_\aEv\limplies\aVal_\aEv{=}\uReg{\aEv}) \lor (\cExp{=}\cVal_\aEv\limplies\REF{\cVal}{=}\uReg{\aEv})}
        \limplies
        \bForm[\uReg{\aEv}/\aReg]
        \\[-.5ex]
        \land
        &\textstyle (\bigwedge_{\aEv\in\aEvs}\lnot\cForm_\aEv)
        \limplies 
        \bForm[\bReg/\aReg],
      \end{aligned}
    \end{math}
  % \item[] \labeltext[\textsc{r}4]{}{read-tau-ca-addr}
  %   \begin{enumerate}[leftmargin=0pt]
  %   \item \label{read-tau-dependent-ca-addr}
  %     \begin{math}
  %       (\forall\aEv\in\aEvs\cap\bEvs)
  %     \end{math}
  %     \begin{math}
  %       \aTr{\bEvs}{\bForm} \riff
  %       \cForm_\aEv
  %       \limplies (\cExp{=}\cVal_\aEv\limplies\aVal_\aEv{=}\uReg{\aEv})
  %       \limplies \bForm[\uReg{\aEv}/\aReg]
  %     \end{math},      
  %   \item \label{read-tau-independent-ca-addr}
  %     \begin{math}
  %       (\forall\aEv\in\aEvs\setminus\bEvs)
  %     \end{math}
  %     \begin{math}
  %       \aTr{\bEvs}{\bForm} \riff
  %       \cForm_\aEv 
  %       \limplies
  %       \PBR{(\cExp{=}\cVal_\aEv\limplies\aVal_\aEv{=}\uReg{\aEv}) \lor (\cExp{=}\cVal_\aEv\limplies\REF{\cVal}{=}\uReg{\aEv})}
  %       \limplies
  %       \bForm[\uReg{\aEv}/\aReg]
  %     \end{math},      
  %   \item \label{read-tau-empty-ca-addr}
  %     \begin{math}
  %       (\forall\bReg)
  %     \end{math}
  %     \begin{math}
  %       \aTr{\bEvs}{\bForm} \riff 
  %       (\bigwedge_{\aEv\in\aEvs}\lnot\cForm_\aEv)
  %       \limplies 
  %       \bForm[\bReg/\aReg],
  %     \end{math}  
  %   \end{enumerate}  
    \columnbreak
  \item \label{read-term-ca-addr}
    if $\aEvs=\emptyset$ and $\amode\neq\mRLX$ then $\aTerm \riff \FALSE$. 
  \end{enumerate}
  \end{multicols}
  % \medskip Similarly, let $\frf{\semaddr{}}$ be defined as for $\frf{\semrr{}}$
  % in \refdef{def:sem:frf}, with these definitions of $\sSTORE{}{}$ and
  % $\sLOAD{}{}$.
\end{definition}

\subsection{Comparison to ``A Promising Semantics 2.1'' [POPL 2017]}
\label{sec:promising}

\todo{Write this.}

Case analysis gives very weak results when combined with thread inlining.
See \cite[\textsection B.1]{DBLP:journals/pacmpl/ChakrabortyV19appendix}.
These happen by performing transformations that: 
(1) introduce conditionals,
(2) inline two threads on both sides of the introduced conditional,
(3) choose different orders for the two threads for the two sides of the conditional.

Case analysis gives very weak results when combined with read introduction.
See \cite{promising-ldrf}.
These happen by performing transformations that: 
(1) introduce reads,
(2) introduce conditionals,
(3) choose different values for the reads on the two sides of the conditional.


The fact that the semantics is not verifiable a posteriori is something it
shares with \weakestmo{}, where the justification relation must be built
inductively.

\weakestmo{} admits FADD, but \PS{} does not.
\PS{} admits CohCYC, but \weakestmo{} does not.



\subsection{Comparison to ``Pomsets with Preconditions'' [OOPSLA 2020]}
\label{sec:diff}

\PwTmca{} is closely related to \PwP{} model of
\citep{DBLP:journals/pacmpl/JagadeesanJR20}.  The major difference is that
\PwTmca{} supports sequential composition.  In the remainder of this section,
we discuss other differences.  We also point out some errors in
\cite{DBLP:journals/pacmpl/JagadeesanJR20}, all of which have been confirmed
by the authors.

\myparagraph{Substitution}

\jjr{} uses substitution rather than Skolemizing.  Indeed our use of
Skolemization is motivated by disjunction closure for predicate transformers,
which do not appear in \jjr{}.  In \reffig{fig:seq}, 
we gave the semantics of read for nonempty pomsets as:
\begin{enumerate}
\item[{\labeltext[\textsc{r}4a]{(\textsc{r}4a)}{read-tau-dep-oopsla}}]
  if $(\aEvs\cap\bEvs)\neq\emptyset$ then
  \begin{math}
    \aTr{\bEvs}{\bForm} \riff
    \aVal{=}\aReg
    \limplies \bForm
  \end{math},    
\item[{\labeltext[\textsc{r}4b]{(\textsc{r}4b)}{read-tau-ind-oopsla}}]
  if $(\aEvs\cap\bEvs)=\emptyset$ then
  \begin{math}
   \aTr{\bEvs}{\bForm} \riff
    \PBR{\aVal{=}\aReg \lor \aLoc{=}\aReg} \limplies
    \bForm.
  \end{math}
\end{enumerate}
In \jjr{}, the definition is roughly as follows:
% (adding the case for $\ref{L6}$, which was missing):
\begin{enumerate}
\item[{\labeltext[\textsc{r}4a$'$]{(\textsc{r}4a$'$)}{read-tau-dep-oopsla-sub}}]
  if $(\aEvs\cap\bEvs)\neq\emptyset$ then
  \begin{math}
    \aTr{\bEvs}{\bForm} \riff
    \bForm[\aVal/\aReg][\aVal/\aLoc]
    % \aVal{=}\aReg
    % \limplies \bForm[\aReg/\aLoc]
  \end{math},    
\item[{\labeltext[\textsc{r}4b$'$]{(\textsc{r}4b$'$)}{read-tau-ind-oopsla-sub}}]
  if $(\aEvs\cap\bEvs)=\emptyset$ then
  \begin{math}
    \aTr{\bEvs}{\bForm} \riff
    \bForm[\aVal/\aReg][\aVal/\aLoc]\land\bForm[\aLoc/\aReg]
  \end{math}
\end{enumerate}
The use of conjunction in \ref{read-tau-ind-oopsla-sub} causes disjunction closure to fail
because the predicate transformer
% $\aTr{}{\bForm}=\bForm[\aVal/\aReg][\aVal/\aLoc]\land\bForm[\aLoc/\aReg]$ does not distribute through
% disjunction:
% \begin{math}
%   \aTr{}{\bForm_1\lor \bForm_2}=
%   (\bForm_1\lor \bForm_2)[\aVal/\aReg][\aVal/\aLoc]\land(\bForm_1\lor \bForm_2)[\aLoc/\aReg]
%   \neq
%   (\bForm_1[\aVal/\aReg][\aVal/\aLoc]\land\bForm_1[\aLoc/\aReg]) \lor
%   (\bForm_2[\aVal/\aReg][\aVal/\aLoc]\land\bForm_2[\aLoc/\aReg])
%   = \aTr{}{\bForm_1} \lor \aTr{}{\bForm_2}
% \end{math}
$\aTr{}{\bForm}=\bForm'\land\bForm''$ does not distribute through
disjunction, even assuming that the prime operations do:\footnote{%
  \begin{math}
    (\bForm_1\lor \bForm_2)'=(\bForm_1'\lor \bForm_2')
  \end{math}
  and
  \begin{math}
    (\bForm_1\lor \bForm_2)''=(\bForm_1''\lor \bForm_2'')
  \end{math}.
}
\begin{math}
  \aTr{}{\bForm_1\lor \bForm_2}=
  \href{https://www.wolframalpha.com/input/?i=\%28a+or+b\%29+and+\%28c+or+d\%29}{(\bForm_1'\lor \bForm_2')\land(\bForm_1''\lor \bForm_2'')}
  \neq
  \href{https://www.wolframalpha.com/input/?i=\%28a+and+c\%29+or+\%28b+and+d\%29}{(\bForm_1'\land\bForm_1'') \lor (\bForm_2'\land\bForm_2'')}
  = \aTr{}{\bForm_1} \lor \aTr{}{\bForm_2}
\end{math}.
% \begin{math}
%   (\bForm_{1}^{1}\lor \bForm_{1}^{2}) \land (\bForm_{2}^{1}\lor \bForm_{2}^{2})
%   \neq
%   (\bForm_{1}^{1}\land\bForm_{2}^{1}) \lor (\bForm_{1}^{2}\land\bForm_{1}^{2}).
% \end{math}
See also \textsection\ref{sec:ex:assoc}.

The substitutions collapse $\aLoc$ and $\aReg$, allowing local invariant
reasoning (\xLIR{}), as required by causality test case 1, discussed at the end of
\textsection\ref{sec:ex:control}.  Without Skolemizing it is necessary to
substitute $[\aLoc/\aReg]$, since the reverse substitution $[\aReg/\aLoc]$ is
useless when $\aReg$ is bound---compare with
\textsection\ref{sec:substitutions}.  As discussed below (\ref{p:downset}),
including this substitution affects the interaction of \xLIR{} and downset
closure.

Removing the substitution of $[x/r]$ in the independent case has a technical
advantage: we no longer require \emph{extended} expressions (which include
memory references), since substitutions no longer introduce memory
references.

\begin{scope}
  The substitution $[x/r]$ does not work with Skolemization, even for the
  dependent case, since we lose the unique marker for each read.  In effect,
  this forces all reads of a location to see the same values.
  % To be concrete, the candidate
  % definition would modify \ref{L4} to be:
  % \begin{enumerate}
  % \item[\ref{L4})]
  %   $\aTr{\bEvs}{\bForm} \riff \aVal{=}\aLoc\limplies\bForm[\aLoc/\aReg]$.
  %   % \item[\ref{L5})]
  %   %   $\aTr{\cEvs}{\bForm} \riff (\aVal{=}\aLoc\lor\TRUE)\limplies\bForm[\aLoc/\aReg]$. %, when $\aEvs\neq\emptyset$,
  %   % \item[\ref{L6})] 
  %   %   $\aTr{\dEvs}{\bForm}\; \riff \bForm$, when $\aEvs=\emptyset$.
  % \end{enumerate}
  Using this definition, consider the following:
  \begin{gather*}
    \PR{x}{r}\SEMI
    \PR{x}{s}\SEMI
    \IF{r{<}s}\THEN \PW{y}{1}\FI 
    \\[-1ex]
    \hbox{\begin{tikzinline}[node distance=0.5em and 1.5em]
        \event{a1}{\DR{x}{1}}{}
        \event{a2}{\DR{x}{2}}{right=of a1}
        \event{a3}{1{=}x\limplies 2{=}x\limplies x{<} x\mid\DW{y}{1}}{right=of a2}
        \po[out=20,in=160]{a1}{a3}
        \po{a2}{a3}
      \end{tikzinline}}
  \end{gather*}
  Although the execution seems reasonable, the precondition on the write is
  not a tautology.
\end{scope}


% There, item \ref{loadpre-kappa2}  of $\sLOADPRE{}{}{}$ is written 
% \begin{enumerate}
% \item[] %[\ref{loadpre-kappa2})]
%   if $\aEv\in\aEvs_2\setminus\aEvs_1$ then either \\
%   $\labelingForm(\aEv) \riff \labelingForm_2(\aEv)[\aLoc/\aReg][\aVal/\aLoc]$ and $(\exists\bEv\in\aEvs_1)\bEv{<}\aEv$, or \\
%   $\labelingForm(\aEv) \riff \labelingForm_2(\aEv)[\aLoc/\aReg][\aVal/\aLoc] \land \labelingForm_2(\aEv)[\aLoc/\aReg]$.
% \end{enumerate}


% [Skolemization ensures disjunction closure, which is necessary
% for associativity. Show example.]

\myparagraph[p:downset]{Downset closure}

\jjr{} enforces downset closure in the prefixing rule.  Even without this,
downset closure would be different for the two semantics, due to the use of
substitution in \jjr{}.  Consider the final pomset in the last example of
\textsection\ref{sec:downset} under the semantics of this paper, which elides
the middle read event:
\begin{align*}
  \begin{gathered}[t]
    \PW{x}{0} 
    \SEMI\PR{x}{r} 
    \SEMI\IF{r{\geq}0}\THEN \PW{y}{1} \FI
    \\
    \hbox{\begin{tikzinline}[node distance=.5em and 1.5em]
        \event{a0}{\DW{x}{0}}{}
        % \event{a1}{\DR{x}{1}}{right=of a0}
        \event{a2}{r{\geq}0\mid\DW{y}{1}}{right=3em of a1}      
        % \wk{a0}{a1}
      \end{tikzinline}}    
  \end{gathered}
\end{align*}
In \jjr{}, the substitution $[x/r]$ is performed by the middle read
regardless of whether it is included in the pomset, with the subsequent
substitution of $[0/x]$ by the preceding write, we have $[x/r][0/x]$, which
is $[0/r][0/x]$, resulting in:
\begin{align*}
  \begin{gathered}[t]
    \hbox{\begin{tikzinline}[node distance=.5em and 1.5em]
        \event{a0}{\DW{x}{0}}{}
        % \event{a1}{\DR{x}{1}}{right=of a0}
        \event{a2}{0{\geq}0\mid\DW{y}{1}}{right=3em of a1}      
        % \wk{a0}{a1}
      \end{tikzinline}}    
  \end{gathered}
\end{align*}


\myparagraph{Augmentation of Preconditions}
\jjr{} allows augmentation of preconditions.
As discussed in \textsection\ref{sec:delay}, this causes associativity to fail
for $\rdelay$, at least when attempting to validate \reflem{lem:if}\eqref{lem:ifelse:if:if1}--\eqref{lem:ifelse:if:if2}
Thus, we use \emph{weakest} preconditions, rather than general preconditions.
As a result, we fail to validate the following
refinement:
\begin{math}
  \aPSS_1
  \not\supseteq
  \xIFTHEN{\aForm}{\aPSS_1}{}.
\end{math}

\myparagraph{Consistency}
\jjr{} imposes \emph{consistency}, which requires that for every pomset
$\aPS$, $\bigwedge_{\aEv}\labelingForm(\aEv)$ is satisfiable.  
\begin{scope}
  Associativity requires that we allow pomsets with inconsistent
  preconditions.  Consider a variant of the example from \textsection\ref{sec:semca}.
  \begin{scope}
    \footnotesize
    \begin{align*}
      \begin{gathered}
        \IF{\aExp}\THEN\PW{x}{1}\FI
        \\
        \hbox{\begin{tikzinline}[node distance=1em]
            \event{a}{\aExp\mid\DW{x}{1}}{}
          \end{tikzinline}}
      \end{gathered}
      &&
      \begin{gathered}
        \IF{\BANG\aExp}\THEN\PW{x}{1}\FI
        \\
        \hbox{\begin{tikzinline}[node distance=1em]
            \event{a}{\lnot\aExp\mid\DW{x}{1}}{}
          \end{tikzinline}}
      \end{gathered}
      &&
      \begin{gathered}
        \IF{\aExp}\THEN\PW{y}{1}\FI
        \\
        \hbox{\begin{tikzinline}[node distance=1em]
            \event{a}{\aExp\mid\DW{y}{1}}{}
          \end{tikzinline}}
      \end{gathered}
      &&
      \begin{gathered}
        \IF{\BANG\aExp}\THEN\PW{y}{1}\FI
        \\
        \hbox{\begin{tikzinline}[node distance=1em]
            \event{a}{\lnot\aExp\mid\DW{y}{1}}{}
          \end{tikzinline}}
      \end{gathered}
    \end{align*}
  \end{scope}
  Associating left and right, we have:
  \begin{scope}
    \footnotesize
    \begin{align*}
      \begin{gathered}
        \IF{\aExp}\THEN\PW{x}{1}\FI
        \SEMI
        \IF{\BANG\aExp}\THEN\PW{x}{1}\FI
        \\
        \hbox{\begin{tikzinline}[node distance=1em]
            \event{a}{\DW{x}{1}}{}
          \end{tikzinline}}
      \end{gathered}
      &&
      \begin{gathered}
        \IF{\aExp}\THEN\PW{y}{1}\FI
        \SEMI
        \IF{\BANG\aExp}\THEN\PW{y}{1}\FI
        \\
        \hbox{\begin{tikzinline}[node distance=1em]
            \event{a}{\DW{y}{1}}{}
          \end{tikzinline}}
      \end{gathered}
    \end{align*}
  \end{scope}  
  Associating into the middle, instead, we require:
  \begin{scope}
    \footnotesize
    \begin{align*}
      \begin{gathered}
        \IF{\aExp}\THEN\PW{x}{1}\FI
        \\
        \hbox{\begin{tikzinline}[node distance=1em]
            \event{a}{\aExp\mid\DW{x}{1}}{}
          \end{tikzinline}}
      \end{gathered}
      &&
      \begin{gathered}
        \IF{\BANG\aExp}\THEN\PW{x}{1}\FI
        \SEMI
        \IF{\aExp}\THEN\PW{y}{1}\FI
        \\
        \hbox{\begin{tikzinline}[node distance=1em]
            \event{a}{\lnot\aExp\mid\DW{x}{1}}{}
            \event{b}{\aExp\mid\DW{y}{1}}{right=of a}
          \end{tikzinline}}
      \end{gathered}
      &&
      \begin{gathered}
        \IF{\BANG\aExp}\THEN\PW{y}{1}\FI
        \\
        \hbox{\begin{tikzinline}[node distance=1em]
            \event{a}{\lnot\aExp\mid\DW{y}{1}}{}
          \end{tikzinline}}
      \end{gathered}
    \end{align*}
  \end{scope}
  Joining left and right, we have:
  \begin{scope}
    \footnotesize
    \begin{align*}
      \begin{gathered}
        \IF{\aExp}\THEN\PW{x}{1}\FI
        \SEMI
        \IF{\BANG\aExp}\THEN\PW{x}{1}\FI
        \SEMI
        \IF{\aExp}\THEN\PW{y}{1}\FI
        \SEMI
        \IF{\BANG\aExp}\THEN\PW{y}{1}\FI
        \\
        \hbox{\begin{tikzinline}[node distance=1em]
            \event{a}{\DW{x}{1}}{}
            \event{b}{\DW{y}{1}}{right=of a}
          \end{tikzinline}}
      \end{gathered}
    \end{align*}
  \end{scope}  
\end{scope}

\myparagraph{Causal Strengthening}
% \labeltext[]{Causal Strengthening}{xCausal}
\jjr{} imposes \emph{causal strengthening}, which requires for every pomset
$\aPS$, if $\bEv\le\aEv$ then $\labelingForm(\aEv) \rimplies \labelingForm(\bEv)$. 
\begin{scope}
  Associativity requires that we allow pomsets without causal strengthening.
  Consider the following.
  \begin{align*}
    \begin{gathered}
      \IF{\aExp}\THEN\PR{x}{r}\FI
      \\
      \hbox{\begin{tikzinline}[node distance=1em]
          \event{a}{\aExp\mid\DR{x}{1}}{}
        \end{tikzinline}}
    \end{gathered}
    &&
    \begin{gathered}
      \PW{y}{r}
      \\
      \hbox{\begin{tikzinline}[node distance=1em]
          \event{a}{r{=}1\mid\DW{y}{1}}{}
        \end{tikzinline}}
    \end{gathered}
    &&
    \begin{gathered}
      \IF{\BANG\aExp}\THEN\PR{x}{s}\FI
      \\
      \hbox{\begin{tikzinline}[node distance=1em]
          \event{a}{\lnot\aExp\mid\DR{x}{1}}{}
        \end{tikzinline}}
    \end{gathered}
  \end{align*}
  Associating left, with causal strengthening:
  \begin{align*}
    \begin{gathered}
      \IF{\aExp}\THEN\PR{x}{r}\FI
      \SEMI
      \PW{y}{r}
      \\
      \hbox{\begin{tikzinline}[node distance=1em]
          \event{a}{\aExp\mid\DR{x}{1}}{}
          \event{b}{\aExp\mid\DW{y}{1}}{right=of a}
          \po{a}{b}
        \end{tikzinline}}
    \end{gathered}
    &&
    \begin{gathered}
      \IF{\BANG\aExp}\THEN\PR{x}{s}\FI
      \\
      \hbox{\begin{tikzinline}[node distance=1em]
          \event{a}{\lnot\aExp\mid\DR{x}{1}}{}
        \end{tikzinline}}
    \end{gathered}
  \end{align*}
  Finally, merging:
  \begin{align*}
    \begin{gathered}
      \IF{\aExp}\THEN\PR{x}{r}\FI
      \SEMI
      \PW{y}{r}
      \SEMI
      \IF{\BANG\aExp}\THEN\PR{x}{s}\FI
      \\
      \hbox{\begin{tikzinline}[node distance=1em]
          \event{a}{\DR{x}{1}}{}
          \event{b}{\aExp\mid\DW{y}{1}}{right=of a}
          \po{a}{b}
        \end{tikzinline}}
    \end{gathered}
  \end{align*}
  Instead, associating right:
  \begin{align*}
    \begin{gathered}
      \IF{\aExp}\THEN\PR{x}{r}\FI
      \\
      \hbox{\begin{tikzinline}[node distance=1em]
          \event{a}{\aExp\mid\DR{x}{1}}{}
        \end{tikzinline}}
    \end{gathered}
    &&
    \begin{gathered}
      \PW{y}{r}
      \SEMI
      \IF{\BANG\aExp}\THEN\PR{x}{s}\FI
      \\
      \hbox{\begin{tikzinline}[node distance=1em]
          \event{a}{\lnot\aExp\mid\DR{x}{1}}{}
          \event{b}{r{=}1\mid\DW{y}{1}}{left=of a}
        \end{tikzinline}}
    \end{gathered}
  \end{align*}
  Merging:
  \begin{align*}
    \begin{gathered}
      \IF{\aExp}\THEN\PR{x}{r}\FI
      \SEMI
      \PW{y}{r}
      \SEMI
      \IF{\BANG\aExp}\THEN\PR{x}{s}\FI
      \\
      \hbox{\begin{tikzinline}[node distance=1em]
          \event{a}{\DR{x}{1}}{}
          \event{b}{\DW{y}{1}}{right=of a}
          \po{a}{b}
        \end{tikzinline}}
    \end{gathered}
  \end{align*}
  With causal strengthening, the precondition of $\DW{y}{1}$ depends upon how
  we associate.  This is not an issue in \jjr{}, which always associates to
  the right.
\end{scope}

% \myparagraph{Causal Strengthening and Address Dependencies}
% \labeltext[]{Causal Strengthening and Address Dependencies}{xADDRxRRD}

\begin{scope}  
  One use of causal strengthening is to ensure that address dependencies do
  not introduce thin air reads.  Associating to the right, the intermediate
  state of the example in \textsection\ref{sec:addr} is:
  \begin{align*}
    \begin{gathered}[t]
      \PR{\REF{r}}{s}
      \SEMI
      \PW{x}{s}
      \\
      \hbox{\begin{tikzinline}[node distance=.5em and 1.5em]
          \event{a2}{r\EQ2\mid\DR{\REF{2}}{1}}{}
          \event{a3}{(r\EQ2\limplies 1\EQ s) \limplies s\EQ1\mid\DW{x}{1}}{right=of a2}
          \po{a2}{a3}
        \end{tikzinline}}
    \end{gathered}
  \end{align*}
  In \jjr{}, we have, instead:
  \begin{gather*}
    % \begin{gathered}[t]
    %   \PW{x}{s}
    %   \\
    %   \hbox{\begin{tikzinline}[node distance=.5em and 1.5em]
    %     \event{b}{s\EQ1\mid\DW{x}{1}}{}
    %   \end{tikzinline}}
    % \end{gathered}
    % \\
    \begin{gathered}
      % \PR{y}{r}\SEMI
      \PR{\REF{r}}{s}\SEMI \PW{x}{s}
      \\
      \hbox{\begin{tikzinline}[node distance=.5em and 1.5em]
          % \event{a1}{\DR{y}{2}}{}
          \event{a2}{r\EQ2\mid\DR{\REF{2}}{1}}{}%right=of a1}
          \event{a3}{r\EQ2\land\REF{2}\EQ1\mid\DW{x}{1}}{right=of a2}
          \po{a2}{a3}
        \end{tikzinline}}
    \end{gathered}
  \end{gather*}
  Without causal strengthening, the precondition of $\DWP{x}{1}$ would be
  simply $\REF{2}\EQ1$.  The treatment in this paper, using implication
  rather than conjunction, is more precise.
\end{scope}

\myparagraph{Internal Acquiring Reads}

The proof of compilation to Arm in \jjr{} assumes that all internal reads can
be eliminated.
% Shortly after publication, \citet{anton}
% noticed a shortcoming of the implementation on \armeight{} in
% \jjr{\textsection 7}.  The proof given there assumes that all internal reads
% can be dropped.
However, this is not the case for acquiring reads.  For example, \jjr{}
disallows the following execution, where the final values of $x$ is $2$ and
the final value of $y$ is $2$.  This execution is allowed by \armeight{} and
\tso{}.
\begin{gather*}
  \PW{x}{2}\SEMI 
  \PR[\mACQ]{x}{r}\SEMI
  \PR{y}{s} \PAR
  \PW{y}{2}\SEMI
  \PW[\mREL]{x}{1}
  \\
  \hbox{\begin{tikzinline}[node distance=1.5em]
      \event{a}{\DW{x}{2}}{}
      \raevent{b}{\DR[\mACQ]{x}{2}}{right=of a}
      \event{c}{\DR{y}{0}}{right=of b}
      \event{d}{\DW{y}{2}}{right=2.5em of c}
      \raevent{e}{\DW[\mREL]{x}{1}}{right=of d}
      \rf{a}{b}
      \sync{b}{c}
      \wk{c}{d}
      \sync{d}{e}
      \wk[out=-165,in=-15]{e}{a}
      % \rfi{a}{b}
      % \bob{b}{c}
      % \fre{c}{d}
      % \bob{d}{e}
      % \coe[out=-165,in=-15]{e}{a}
    \end{tikzinline}}
\end{gather*}
We discussed two approaches to this problem in \textsection\ref{sec:arm}.
% The solution we have adopted is to allow an acquiring read to be downgraded
% to a relaxed read when it is preceded (sequentially) by a relaxed write that
% could fulfill it.  This solution allows executions that are not allowed under
% \armeight{} since we do not insist that the local relaxed write is actually
% read from.  This may seem counterintuitive, but we don't see a local way to
% be more precise.

% As a result, we use a different proof strategy for \armeight{}
% implementation, which does not rely on read elimination.  The proof idea uses
% a recent alternative characterization of \armeight{}
% \citep{alglave-git-alternate,arm-reference-manual}. %,armed-cats}.

\myparagraph{Redundant Read Elimination}

Contrary to the claim, redundant read elimination fails for \jjr{}.
We discussed redundant read elimination in \textsection\ref{sec:semreg}.
Consider JMM Causality Test Case 2, which we discussed there.
\begin{gather*}
  \PR{x}{r}\SEMI
  \PR{x}{s}\SEMI
  \IF{r{=}s}\THEN \PW{y}{1}\FI
  \PAR
  \PW{x}{y}
  \\
  \hbox{\begin{tikzinline}[node distance=1.5em]
      \event{a1}{\DR{x}{1}}{}
      \event{a2}{\DR{x}{1}}{right=of a1}
      \event{a3}{\DW{y}{1}}{right=of a2}
      \event{b1}{\DR{y}{1}}{right=3em of a3}
      \event{b2}{\DW{x}{1}}{right=of b1}
      \rf{a3}{b1}
      \po{b1}{b2}
      \rf[out=169,in=11]{b2}{a2}
      \rf[out=169,in=11]{b2}{a1}
    \end{tikzinline}}
\end{gather*}
Under the semantics of \jjr{}, we have
\begin{gather*}
  \PR{x}{r}\SEMI
  \PR{x}{s}\SEMI
  \IF{r{=}s}\THEN \PW{y}{1}\FI
  \\
  \hbox{\begin{tikzinline}[node distance=1.5em]
      \event{a1}{\DR{x}{1}}{}
      \event{a2}{\DR{x}{1}}{right=of a1}
      \event{a3}{1\EQ1\land1\EQ x \land x\EQ1 \land x=x\mid\DW{y}{1}}{right=of a2}
    \end{tikzinline}}
\end{gather*}
The precondition of $\DWP{y}{1}$ is \emph{not} a tautology, and therefore
redundant read elimination fails.
(It is a tautology in
\begin{math}
  \PR{x}{r}\SEMI
  \LET{s}{r}\SEMI
  \IF{r{=}s}\THEN \PW{y}{1}\FI
\end{math}.)
\jjr{\textsection3.1} incorrectly stated that the precondition of
$\DWP{y}{1}$ was $1\EQ1\land x\EQ x$.  

\myparagraph{Parallel Composition}

In \jjr{\textsection2.4}, parallel composition is defined allowing coalescing
of events.  Here we have forbidden coalescing.  This difference appears to be
arbitrary.  In \jjr{}, however, there is a mistake in the handling of
termination actions.  The predicates should be joined using $\land$, not
$\lor$.

\myparagraph{Read-Modify-Write Actions}

In \jjr{}, the atomicity axioms \ref{pom-rmw-atomic} erroneously applies only to
overlapping writes, not overlapping reads.  The difficulty can be seen in
\refex{ex:rmw-33}.

In addition, \jjr{} uses $\sLOAD{}{}$ instead of $\sLOADP{}{}$ when
calculating of dependency for \RMW{}s.  For a discussion, see the example at
the end of \textsection\ref{sec:rmw}.

\myparagraph{Data Race Freedom}

The definition of data race is wrong in \jjr{}.  It should require that that
at least one action is relaxed.

Note that the definition of \emph{$L$-stable} applies in the case that
conflicting writes are totally ordered.  This gives a result more in the
spirit of \cite{Dolan:2018:BDR:3192366.3192421}.  In particular, this special
case of the theorem clarifies the discussion of the \textsc{past} example
in \jjr{};











\endinput




Precondition of $\DWP{y}{1}$ is $(r{=}s)$ in
\begin{math}
  \sem{\IF{r{=}s}\THEN \PW{y}{1}\FI}.
\end{math}
Predicate transformers for $\emptyset$ in $\sem{\PR{x}{r}}$ and $\sem{\PR{x}{s}}$ are
\begin{align*}
  \PREDP{(r{=}1 \lor r{=}x)\limplies\bForm[r/x]},
  \\
  \PREDP{(s{=}1 \lor s{=}x)\limplies\bForm[s/x]}.
\end{align*}
Combining the transformers, we have
\begin{displaymath}
  \PREDP{(r{=}1 \lor r{=}x)\limplies(s{=}1 \lor s{=}r)\limplies\bForm[s/x]}.
\end{displaymath}
Applying this to $(r{=}s)$, we have
\begin{displaymath}
  \PREDP{(r{=}1 \lor r{=}x)\limplies (s{=}1 \lor s{=}r)\limplies (r{=}s)},
\end{displaymath}
which is not a tautology.

Same problem occurs \jjr{}, where we have:
\begin{align*}
  \PREDP{\bForm[v/x,r] \land \bForm[x/r]},
  \\
  \PREDP{\bForm[v/x,s] \land \bForm[x/s]}.
\end{align*}
Combining the transformers, we have
\begin{displaymath}
  \PREDP{\bForm[v/x,r,s] \land \bForm [v/x,r][x/s] \land \bForm[x/r][v/x,s] \land \bForm[x/r,s]}.
\end{displaymath}
Applying this to $(r{=}s)$, we have
\begin{displaymath}
  \PREDP{v{=}v \land v{=}x \land x{=}v \land x{=}x},
\end{displaymath}
which is not a tautology.

The semantics here allows this by coalescing:
\begin{gather*}
  \PR{x}{r}\SEMI
  \PR{x}{s}\SEMI
  \IF{r{=}s}\THEN \PW{y}{1}\FI
  \PAR
  \PW{x}{y}
  \\
  \hbox{\begin{tikzinline}[node distance=1.5em]
      \event{a1}{\DR{x}{1}}{}
      \event{a3}{\DW{y}{1}}{right=of a1}
      \event{b1}{\DR{y}{1}}{right=3em of a3}
      \event{b2}{\DW{x}{1}}{right=of b1}
      \rf{a3}{b1}
      \po{b1}{b2}
      \rf[out=169,in=11]{b2}{a1}
    \end{tikzinline}}
\end{gather*}

In \jjr{\textsection2.6} the semantics of read is defined as follows:
\begin{align*}
  \sem{\PR[\amode]{\aLoc}{\aReg}\SEMI \aCmd} & \eqdef \textstyle\bigcup_\aVal\;
  (\DRmode\aLoc\aVal) \prefix \sem{\aCmd} [\aLoc/\aReg]
\end{align*}
The definition of prefixing$((\aForm \mid \aAct) \prefix \aPSS)$ has several clauses.
The most relevant are as follows, where $\bEv$ is the new event labeled with
$(\aForm \mid \aAct)$ and $\aEv$ is an event from $\aPSS$:
\begin{description}
\item[{\labeltextsc[P4c]{(P4c)}{4c}}]
  If $\bEv$ reads $\aVal$ from $\aLoc$ then either $\aEv=\bEv$ or
  $\labelingForm'(\aEv) \rimplies \labelingForm(\aEv)[\aVal/\aLoc]$.
\item[{\labeltextsc[P5a]{(P5a)}{5a}}]\labeltextsc[P5]{}{5}%
  If $\bEv$ reads and $\aEv$ writes then either $\labelingForm'(\aEv) \rimplies \labelingForm(\aEv)$ or $\bEv\le'\aEv$.
  % \item[{\labeltextsc[P5b]{(P5b)}{5b}}]
  %   If $\bEv$ and $\aEv$ are in conflict then $\bEv\le'\aEv$.
\end{description}

We have discovered two issues with this definition.

The first issue concerns the substitution $[\aLoc/\aReg]$.  It should be
$[\aReg/\aLoc]$.  We noticed this error while developing the alternative
characterization presented here.  The error causes redundant read elimination
to fail in \jjr{}.  As a result, common subexpression elimination also fails.
The problem can be seen in \ref{TC2}.
\begin{gather*}
  \taglabel{TC2}
  \PR{x}{r}\SEMI
  \PR{x}{s}\SEMI
  \IF{r{=}s}\THEN \PW{y}{1}\FI
  \PAR
  \PW{x}{y}
\end{gather*}
% In \jjr{\textsection3.1},
We claimed that \ref{TC2} allowed the following
execution:
\begin{gather*}
  \hbox{\begin{tikzinline}[node distance=1.5em]
      \event{a1}{\DR{x}{1}}{}
      \event{a2}{\DR{x}{1}}{right=of a1}
      \event{a3}{\DW{y}{1}}{right=of a2}
      % \po{a2}{a3}
      % \po[out=15,in=165]{a1}{a3}
      \event{b1}{\DR{y}{1}}{right=3em of a3}
      \event{b2}{\DW{x}{1}}{right=of b1}
      \rf{a3}{b1}
      \po{b1}{b2}
      \rf[out=169,in=11]{b2}{a2}
      \rf[out=169,in=11]{b2}{a1}
    \end{tikzinline}}
\end{gather*}
But this execution is not possible using the semantics of \jjr{}:
$\DWP{y}{1}$ has precondition $r{=}s$ in
\begin{math}
  \sem{\IF{r{=}s}\THEN \PW{y}{1}\FI}.
\end{math}
Given the lack of order in the execution, the precondition of $\DWP{y}{1}$
must entail $r{=}1\land r{=}x$ in 
\begin{math}
  \sem{\PR{x}{s}\SEMI
    \IF{r{=}s}\THEN \PW{y}{1}\FI}.
\end{math}
\ref{4c} imposes $r{=}1$, and \ref{5a} imposes $r{=}x$.  Adding the second
read, the precondition of $\DWP{y}{1}$ must entail both $1{=}1\land 1{=}x$
and also $x{=}1\land x{=}x$.  This can be simplified to $x{=}1$.  This leaves
a requirement that must be satisfied by a preceding write.  Since the
preceding write is the initialization to $0$, the requirement cannot be
satisfied, and the execution is impossible.\footnote{In \jjr{} we ignore the
  middle terms, mistakenly simplifying this to $1{=}1\land x{=}x$.
  Correcting the error, the attempted execution is:
  \begin{gather*}
    \hbox{\begin{tikzinline}[node distance=1.5em]
        \event{a1}{\DR{x}{1}}{}
        \event{a2}{\DR{x}{1}}{right=of a1}
        \event{a3}{\DW{y}{1}}{right=of a2}
        \po{a2}{a3}
        \po[out=-20,in=-160]{a1}{a3}
        \event{b1}{\DR{y}{1}}{right=3em of a3}
        \event{b2}{\DW{x}{1}}{right=of b1}
        \rf{a3}{b1}
        \po{b1}{b2}
        \rf[out=169,in=11]{b2}{a2}
        \rf[out=169,in=11]{b2}{a1}
      \end{tikzinline}}
  \end{gather*}}

The substitution $[\aLoc/\aReg]$ leaves the obligation on $\aLoc$ to be
fulfilled by the preceding write.  Thus, the read does not update the
\emph{value} of $\aLoc$ in subsequent predicates.  The substitution
$[\aReg/\aLoc]$, instead, does update the value of $\aLoc$, thus removing any
obligation on $\aLoc$ for preceding code.

In order to write this, we must update the definition of prefixing reads to
include the register.  Then \ref{4c} becomes:
\begin{description}
\item[\textsc{(p4c)}] If $\bEv$ reads $\aVal$ from $\aLoc$ then either
  $\aEv=\bEv$ or $\labelingForm'(\aEv) \rimplies \labelingForm(\aEv)[\aVal/\aReg]$.
\end{description}

We can then reason with \ref{TC2} as follows: $\DWP{y}{1}$ has precondition
$r{=}s$ in
\begin{math}
  \sem{\IF{r{=}s}\THEN \PW{y}{1}\FI}.
\end{math}
To avoid introducing order in the execution, the precondition of $\DWP{y}{1}$
must entail $r{=}1\land r{=}s$ in 
\begin{math}
  \sem{\PR{x}{s}\SEMI
    \IF{r{=}s}\THEN \PW{y}{1}\FI}.
\end{math}
\ref{4c} imposes $r{=}1$, and \ref{5a} imposes $r{=}x$.  Adding the second
read, the precondition of $\DWP{y}{1}$ must entail both $1{=}1\land 1{=}x$
and also $x{=}1\land x{=}x$.  This can be simplified to $x{=}1$.  This leaves
a requirement that must be satisfied by a preceding write.


With read elimination, the rule for relaxed reads is as follows:
\begin{align*}
  \sem{\PR{\aLoc}{\aReg} \SEMI \aCmd} &\eqdef
  \sem{\aCmd}[\aLoc/\aReg]
  \cup
  \textstyle\bigcup_\aVal\;
  \DRP{\aLoc}{\aVal} \prefix_{\aReg} %\Rdis{\aLoc}{\aVal}
  \sem{\aCmd}[\aReg/\aLoc]
\end{align*}
It is interesting to note that the substitution is $[\aLoc/\aReg]$ on
eliminated reads, and $[\aReg/\aLoc]$ on non-eliminated reads.  Intuitively,
the subsequent value of $\aLoc$ is fixed by an explicit read, but not for an
eliminated read.  In the latter case, the value is fixed by some preceding
action.  The preceding action may itself be a read. This gives rise to some
fear that we might introduce thin-air reads, since we do not enforce
read-read coherence.  But this is not the case.  Consider the following example:
\begin{gather*}
  \PR{x}{r}\SEMI
  \PR{x}{s}\SEMI
  \PW{y}{s}
  \PAR
  \PW{x}{y}
  \\
  \hbox{\begin{tikzinline}[node distance=1.5em]
      \event{a1}{\DR{x}{1}}{}
      \event{a2}{\DR{x}{1}}{right=of a1}
      \event{a3}{\DW{y}{1}}{right=of a2}
      % \po{a2}{a3}
      \po[out=-20,in=-160]{a1}{a3}
      \event{b1}{\DR{y}{1}}{right=3em of a3}
      \event{b2}{\DW{x}{1}}{right=of b1}
      \rf{a3}{b1}
      \po{b1}{b2}
      \rf[out=169,in=11]{b2}{a2}
      \rf[out=169,in=11]{b2}{a1}
    \end{tikzinline}}
  \\
  \hbox{\begin{tikzinline}[node distance=1.5em]
      \event{a1}{\DR{x}{1}}{}
      \internal{a2}{\DR{x}{1}}{right=of a1}
      \event{a3}{\DW{y}{1}}{right=of a2}
      % \po{a2}{a3}
      \po[out=-20,in=-160]{a1}{a3}
      \event{b1}{\DR{y}{1}}{right=3em of a3}
      \event{b2}{\DW{x}{1}}{right=of b1}
      \rf{a3}{b1}
      \po{b1}{b2}
      % \rf[out=169,in=11]{b2}{a2}
      \rf[out=169,in=11]{b2}{a1}
    \end{tikzinline}}
\end{gather*}
But this is not a problem, since fulfillment requires that $\DWP{x}{1}$
precede both reads of $x$.





\subsection{Substitutions}
\label{sec:substitutions}

In $\sLOAD{}{}$, it is also possible to collapse $\aLoc$ and $\aReg$ via substitution:
\begin{enumerate}
\item[{\labeltext[\textsc{r}4a$'$]{(\textsc{r}4a$'$)}{read-tau-dep-sub}}]
  if $(\aEvs\cap\bEvs)\neq\emptyset$ then
  \begin{math}
    \aTr{\bEvs}{\bForm} \riff
    \aVal{=}\aReg
    \limplies \bForm[\aReg/\aLoc]
  \end{math},    
\item[{\labeltext[\textsc{r}4b$'$]{(\textsc{r}4b$'$)}{read-tau-ind-sub}}]
  if $\aEvs\neq\emptyset$ and $(\aEvs\cap\bEvs)=\emptyset$ then
  \begin{math}
    \aTr{\bEvs}{\bForm} \riff
    \PBR{\aVal{=}\aReg \lor \aLoc{=}\aReg} \limplies
    \bForm[\aReg/\aLoc],
  \end{math}
\item[{\labeltext[\textsc{r}4c$'$]{(\textsc{r}4c$'$)}{read-tau-empty-sub}}]
  if $\aEvs=\emptyset$ then
  \begin{math}
    \aTr{\bEvs}{\bForm} \riff
    % \PBR{\aVal{=}\aReg \lor \aLoc{=}\aReg} \limplies
    \bForm[\aReg/\aLoc],
  \end{math}
\end{enumerate}
Perhaps surprisingly, this semantics is incomparable with that of
\reffig{fig:seq}.  Consider the following:
\begin{gather*}
  \IF{r\land s\;\mathsf{even}}\THEN \PW{y}{1}\FI\SEMI
  \IF{r\land s}\THEN \PW{z}{1}\FI
  \\
  \hbox{\begin{tikzinline}[node distance=0.5em and 1.5em]
      \event{a3}{r\land s\;\mathsf{even}\mid\DW{y}{1}}{}
      \event{a4}{r\land s\mid\DW{z}{1}}{right=of a3}
    \end{tikzinline}}
\end{gather*}
Prepending $\PRP{x}{s}$, we get the same result regardless of whether we
substitute $[s/x]$, since $x$ does not occur in either precondition.  Here
we show the independent case:
\begin{gather*}
  \PR{x}{s}\SEMI
  \IF{r\land s\;\mathsf{even}}\THEN \PW{y}{1}\FI\SEMI
  \IF{r\land s}\THEN \PW{z}{1}\FI
  \\
  \hbox{\begin{tikzinline}[node distance=0.5em and 1.5em]
      \event{a2}{\DR{x}{2}}{}
      \event{a3}{(2{=}s\lor x{=}s)\limplies (r\land s\;\mathsf{even})\mid\DW{y}{1}}{right=of a2}
      \event{a4}{(2{=}s\lor x{=}s)\limplies (r\land s)\mid\DW{z}{1}}{right=of a3}
    \end{tikzinline}}
\end{gather*}
Since the preconditions mention $x$, prepending $\PRP{x}{r}$, we now get
different results depending on whether we perform the substitution.  Without
any substitution, we have:
\begin{gather*}
  \PR{x}{r}\SEMI
  \PR{x}{s}\SEMI
  \IF{r\land s\;\mathsf{even}}\THEN \PW{y}{1}\FI\SEMI
  \IF{r\land s}\THEN \PW{z}{1}\FI
  \\[-2ex]
  \hbox{\begin{tikzinline}[node distance=0.5em and 1.5em]
      \event{a1}{\DR{x}{1}}{}
      \event{a2}{\DR{x}{2}}{right=of a1}
      \event{a3}{1{=}r\limplies  (2{=}s\lor x{=}s)\limplies (r\land s\;\mathsf{even})\mid\DW{y}{1}}{right=of a2}
      \event{a4}{1{=}r\limplies  (2{=}s\lor x{=}s)\limplies (r\land s)\mid\DW{z}{1}}{right=of a3}
      \po[out=12,in=168]{a1}{a3}
      \po[out=10,in=170]{a1}{a4}
    \end{tikzinline}}
\end{gather*}
Prepending $\PWP{x}{0}$, which substitutes $[0/x]$, the precondition of
$\DWP{y}{1}$ becomes
$(1{=}r\limplies (2{=}s\lor0{=}s)\limplies (r\land s\;\mathsf{even}))$, which
is a tautology, whereas the precondition of $\DW{z}{1}$ becomes
$(1{=}r\limplies(2{=}s\lor0{=}s)\limplies (r\land s))$, which is not.  In
order to be top-level, $\DWP{z}{1}$ must be dependency ordered after
$\DRP{x}{2}$; in this case the precondition becomes
$(1{=}r\limplies2{=}s\limplies (r\land s))$, which is a tautology.
\begin{gather*}
  % \PW{x}{0}\SEMI
  % \PR{x}{r}\SEMI
  % \PR{x}{s}\SEMI
  % \IF{r\land s\;\mathsf{even}}\THEN \PW{y}{1}\FI\SEMI
  % \IF{r\land s}\THEN \PW{z}{1}\FI
  % \\
  \hbox{\begin{tikzinline}[node distance=1.5em]
      \event{a0}{\DW{x}{0}}{}
      \event{a1}{\DR{x}{1}}{right=of a0}
      \event{a2}{\DR{x}{2}}{right=of a1}
      \event{a3}{\DW{y}{1}}{right=of a2}
      \event{a4}{\DW{z}{1}}{right=of a3}
      % \wk{a0}{a1}
      % \wk[out=-20,in=-160]{a0}{a2}
      \po[out=20,in=160]{a1}{a3}
      \po[out=20,in=160]{a1}{a4}
      \po[out=-20,in=-160]{a2}{a4}
    \end{tikzinline}}
\end{gather*}
The situation reverses with the substitution $[r/x]$:
\begin{gather*}
  \PR{x}{r}\SEMI
  \PR{x}{s}\SEMI
  \IF{r\land s\;\mathsf{even}}\THEN \PW{y}{1}\FI\SEMI
  \IF{r\land s}\THEN \PW{z}{1}\FI
  \\[-2ex]
  \hbox{\begin{tikzinline}[node distance=0.5em and 1.5em]
      \event{a1}{\DR{x}{1}}{}
      \event{a2}{\DR{x}{2}}{right=of a1}
      \event{a3}{1{=}r\limplies  (2{=}s\lor r{=}s)\limplies (r\land s\;\mathsf{even})\mid\DW{y}{1}}{right=of a2}
      \event{a4}{1{=}r\limplies  (2{=}s\lor r{=}s)\limplies (r\land s)\mid\DW{z}{1}}{right=of a3}
      \po[out=12,in=168]{a1}{a3}
      \po[out=10,in=170]{a1}{a4}
    \end{tikzinline}}
\end{gather*}
Prepending $\PWP{x}{0}$:
%\vspace{-.5\baselineskip}
\begin{gather*}
  % \PW{x}{0}\SEMI
  % \PR{x}{r}\SEMI
  % \PR{x}{s}\SEMI
  % \IF{r\land s\;\mathsf{even}}\THEN \PW{y}{1}\FI\SEMI
  % \IF{r\land s}\THEN \PW{z}{1}\FI
  % \\
  \hbox{\begin{tikzinline}[node distance=1.5em]
      \event{a0}{\DW{x}{0}}{}
      \event{a1}{\DR{x}{1}}{right=of a0}
      \event{a2}{\DR{x}{2}}{right=of a1}
      \event{a3}{\DW{y}{1}}{right=of a2}
      \event{a4}{\DW{z}{1}}{right=of a3}
      % \wk{a0}{a1}
      % \wk[out=-20,in=-160]{a0}{a2}
      \po[out=20,in=160]{a1}{a3}
      \po[out=20,in=160]{a1}{a4}
      \po{a2}{a3}
    \end{tikzinline}}
\end{gather*}
The dependency has changed from $\DRP{x}{2}\xpo\DWP{z}{1}$ to
$\DRP{x}{2}\xpo\DWP{y}{1}$.  The resulting sets of pomsets are
incomparable.


Thinking in terms of hardware, the difference is whether reads update the
cache, thus clobbering preceding writes.  With $[r/x]$, reads clobber the
cache, whereas without the substitution, they do not.  Since most caches work
this way, the model with $[r/x]$ is likely preferred for modeling hardware.
However, this substitution only makes sense in a model with read-read
coherence and read-read dependencies, which we will see is not the case for Arm.  By
leaving out the substitution, we also ensure that downgraded reads are
fulfilled by preceding writes, not reads.





% \begin{figure*}[t]
%   \showRAtrue
%   \begin{center}
%     \begin{minipage}{.91\textwidth}
%       \renewcommand{\cEvs}{D}
\renewcommand{\dEvs}{D}
\noindent
If $\aPS \in \sSTORE[\amode]{\aLoc}{\aExp}$ then
$(\exists\aVal:\aEvs\fun\Val)$
$(\exists\cForm:\aEvs\fun\Formulae)$
\begin{enumerate}
\item[{\labeltext[S1]{S1)}{S1no-q-or-addr}}] 
  if $\cForm_\bEv\land\cForm_\aEv$ is satisfiable then $\bEv=\aEv$,
\item[{\labeltext[S2]{S2)}{S2no-q-or-addr}}] 
  $\labelingAct(\aEv) = \DW[\amode]{\aLoc}{\aVal_\aEv}$,
\item[{\labeltext[S3]{S3)}{S3no-q-or-addr}}] 
  $\labelingForm(\aEv)$ implies
  \begin{math}
    \cForm_\aEv
    \land \QS{}{\amode}
    \land \aExp{=}\aVal_\aEv
  \end{math},
  
  
\item[{\labeltext[S4]{S4)}{S4no-q-or-addr}}] 
  \begin{math}
    (\forall\aEv\in\aEvs\cap\bEvs)
  \end{math}
  $\aTr{\bEvs}{\bForm}$ implies 
  \begin{math}
    \cForm_\aEv
    \limplies {
      \bForm
      [\aExp/\aLoc]
      \DS{\aLoc}{\amode}
      [(\Q{}\land\aExp{=}\aVal_\aEv)/\Q{}]
    }
  \end{math},
\item[{\labeltext[S5]{S5)}{S5no-q-or-addr}}] 
  \begin{math}    
    (\forall\aEv\in\aEvs\setminus\cEvs)
  \end{math}
  $\aTr{\cEvs}{\bForm}$ implies
  \begin{math}
    \cForm_\aEv
    \limplies {
      \bForm
      [\aExp/\aLoc]
      \DS{\aLoc}{\amode}
      [\FALSE/\Q{}]
    }.
  \end{math}
% \item[{\labeltext[S6]{S6)}{S6no-q-or-addr}}] 
%   $\aTr{\dEvs}{\bForm}$ implies
%   \begin{math}
%     (\!\not\exists\aEv\in\aEvs \suchthat \cForm_\aEv)
%     \limplies {
%       \bForm
%       [\aExp/\aLoc]
%       \DS{\aLoc}{\amode}
%       [\FALSE/\Q{}]
%     }.
%   \end{math}
\end{enumerate}

\noindent
If $\aPS \in \sLOAD[\amode]{\aReg}{\aLoc}$ then
$(\exists\aVal:\aEvs\fun\Val)$
$(\exists\cForm:\aEvs\fun\Formulae)$
$(\exists\bmode\in\{\amode,\mRLX\})$

\begin{enumerate}
\item[{\labeltext[L1]{L1)}{L1no-q-or-addr}}] 
  if $\cForm_\bEv\land\cForm_\aEv$ is satisfiable then $\bEv=\aEv$,
\item[{\labeltext[L2]{L2)}{L2no-q-or-addr}}] 
  $\labelingAct(\aEv) = \DR[\bmode]{\aLoc}{\aVal_\aEv}$
\item[{\labeltext[L3]{L3)}{L3no-q-or-addr}}] 
  $\labelingForm(\aEv)$ implies
  \begin{math}
    \cForm_\aEv
    \land \QL{}{\amode}
  \end{math},
    
\item[{\labeltext[L4]{L4)}{L4no-q-or-addr}}] 
  \begin{math}
    (\forall\aEv\in\aEvs\cap\bEvs)
  \end{math}
  $\aTr{\bEvs}{\bForm}$ implies
  \begin{math}
    \cForm_\aEv
    \limplies \aVal_\aEv{=}\uReg{\aEv}
    \limplies \bForm[\uReg{\aEv}/\aReg]
  \end{math},
  
\item[{\labeltext[L5]{L5)}{L5no-q-or-addr}}] 
  \begin{math}
    (\forall\aEv\in\aEvs\setminus\cEvs)
  \end{math}
  $\aTr{\cEvs}{\bForm}$ implies
  \begin{math}
    \cForm_\aEv 
    \limplies
    \DLX{\aLoc}{\amode}{\bmode}
    \land
    \PBRbig{
      \ABRbig{
        \aVal_\aEv{=}\uReg{\aEv}
        \lor
        \PBR{
          \RW\land
          \aLoc{=}\uReg{\aEv}
        }
      }
      \limplies
      \bForm
      [\uReg{\aEv}/\aReg]
      [\FALSE/\Q{}]
    }    
  \end{math},
\item[{\labeltext[L6]{L6)}{L6no-q-or-addr}}] 
  \begin{math}
    (\forall\bReg)
  \end{math}
  $\aTr{\dEvs}{\bForm}$  implies 
  \begin{math}
    (\!\not\exists\aEv\in\aEvs \suchthat \cForm_\aEv)
    \limplies \PBR{        
      \DLX{\aLoc}{\amode}{\bmode} \land
      \bForm
      [\bReg/\aReg]
      [\FALSE/\Q{}]
    }.
  \end{math}  
\end{enumerate}  





















































%     \end{minipage}
%   \end{center}
%   \caption{Simplified Quiescence Semantics w/o Address Calculation
%     (See %\refdef{def:QSx} for $\QS{}{\amode}$, $\QL{}{\amode}$, and
%     \refdef{def:dlx} for $\DLX{\aLoc}{\amode}{\bmode}$, $\DS{\aLoc}{\amode}$)
%   } 
%   \label{fig:no-q-or-addr}
% \end{figure*}    
% \begin{figure*}
%   \begin{center}
%     \begin{minipage}{.91\textwidth}
%       \noindent
If $\aPS \in \sSTORE[\amode]{\cExp}{\aExp}$ then
$(\exists\cVal:\aEvs\fun\Val)$
$(\exists\aVal:\aEvs\fun\Val)$
$(\exists\cForm:\aEvs\fun\Formulae)$
\begin{enumerate}
\item[\ref{S1})] if $\cForm_\bEv\land\cForm_\aEv$ is satisfiable then $\bEv=\aEv$,
\item[\ref{S2})] $\labelingAct(\aEv) = \DWREF{\cVal_\aEv}{\aVal_\aEv}$,
\item[\ref{S3})] 
  $\labelingForm(\aEv)$ implies
  \begin{math}
    \cForm_\aEv
    \land \QS{\REF{\cVal_\aEv}}{\amode}
    % \land \RW
    \land \cExp{=}\cVal_\aEv
    \land \aExp{=}\aVal_\aEv
  \end{math},
  % where
  % $\QS{}{\mRLX}=\QxREF{\cVal_\aEv}$ and otherwise $\QS{}{\amode}=\Q{\amode}$, % for $\amode\neq\mRLX$,
\item[\ref{S4})]
  \begin{math}
    (\forall\dVal)
    (\forall\aEv\in\aEvs\cap\bEvs)
  \end{math}
  $\aTr{\bEvs}{\bForm}$ \;implies \,
  \begin{math}
    \cForm_\aEv
    \limplies (\cExp{=}\dVal)
    \limplies \PBR{
      %(\QwREF{\dVal} \limplies \aExp{=}\aVal_\aEv) \land
      \bForm
      [\aExp/\REF{\dVal}]
      \DS{\REF{\dVal}}{\amode}
      [(\QwREF{\dVal}\land\aExp{=}\aVal)/\QwREF{\dVal}]
    }
  \end{math},
\item[\ref{S5})] %if 
  % \begin{math}
  %   (\forall\aEv\in\bEvs)(\cForm \textimplies
  %   \lnot\cForm_\aEv)
  % \end{math}
  % then
  \begin{math}
    (\forall\dVal)
  \end{math}
  $\aTr{\cEvs}{\bForm}$ implies
  \begin{math}
    (\!\not\exists\aEv\in\aEvs\cap\cEvs \suchthat \cForm_\aEv)
    \limplies (\cExp{=}\dVal)
    \limplies \PBR{
      % \lnot\QwREF{\dVal} \land
      \bForm
      [\aExp/\REF{\dVal}]
      \DS{\REF{\dVal}}{\amode}
      [\FALSE/\QS{\REF{\dVal}}{\amode}]
    }.
  \end{math}
  % \\ where 
  % $\DS{}{\mRLX}{}=[\TRUE/\DxREF{\dVal}]$ and otherwise
  % $\DS{}{\amode}{}=[\FALSE/\D]$. % for $\amode\neq\mRLX$.
\end{enumerate}
% \item if $\amode=\mRLX$ then
%   $\labelingForm(\aEv)$ implies
%   \begin{math}
%     \cForm_\aEv
%     \land \cExp{=}\cVal_\aEv
%     \land \aExp{=}\aVal_\aEv
%     \land \RW
%     \land \QxREF{\cVal_\aEv},
%   \end{math}
% \item if $\amode\neq\mRLX$ then
%   $\labelingForm(\aEv)$ implies
%   \begin{math}
%     \cForm_\aEv
%     \land \cExp{=}\cVal_\aEv
%     \land \aExp{=}\aVal_\aEv
%     \land \RW
%     \land \Q{},
%   \end{math}
% \item if
%   $\aEv\in\bEvs$
%   and
%   $\amode=\mRLX$ then
%   \begin{math}
%     (\forall\dVal)
%   \end{math}
%   $\aTr{\bEvs}{\bForm}$ implies 
%   \begin{math}
%     \cForm_\aEv
%     \limplies (\cExp{=}\dVal)
%     \limplies \PBRbig{
%     (\QwREF{\dVal} \limplies \aExp{=}\aVal_\aEv)
%     \land \bForm[\aExp/\REF{\dVal}][\TRUE/\DxREF{\dVal}]
%   }
%   \end{math}
% \item if
%   $\aEv\in\bEvs$
%   and
%   $\amode\neq\mRLX$ then
%   \begin{math}
%     (\forall\dVal)
%   \end{math}
%   $\aTr{\bEvs}{\bForm}$ implies 
%   \begin{math}
%     \cForm_\aEv
%     \limplies (\cExp{=}\dVal)
%     \limplies \PBRbig{
%     (\QwREF{\dVal} \limplies \aExp{=}\aVal_\aEv)
%     \land \bForm[\aExp/\REF{\dVal}][\FALSE/\D]
%   }
%   \end{math}
% \item if 
%   \begin{math}
%     (\forall\aEv\in\bEvs)(\cForm \textimplies
%     \lnot\cForm_\aEv)
%   \end{math}
%   and $\amode=\mRLX$ 
%   then
%   \begin{math}
%     (\forall\dVal)
%   \end{math}
%   $\aTr{\bEvs}{\bForm}$ implies 
%   \begin{math}
%     \cForm
%     \limplies (\cExp{=}\dVal)
%     \limplies \PBRbig{
%     \lnot\QwREF{\dVal}
%     \land \bForm[\aExp/\REF{\dVal}][\TRUE/\DxREF{\dVal}]
%   }
%   \end{math}
% \item if 
%   \begin{math}
%     (\forall\aEv\in\bEvs)
%     (\cForm \textimplies \lnot\cForm_\aEv)
%   \end{math}
%   and $\amode\neq\mRLX$ 
%   then
%   \begin{math}
%     (\forall\dVal)
%   \end{math}
%   $\aTr{\bEvs}{\bForm}$ implies 
%   \begin{math}
%     \cForm
%     \limplies (\cExp{=}\dVal)
%     \limplies \PBRbig{
%     \lnot\QwREF{\dVal}
%     \land \bForm[\aExp/\REF{\dVal}][\FALSE/\D]
%   }
%   \end{math}

\noindent
If $\aPS \in \sLOAD[\amode]{\aReg}{\cExp}$ then
$(\exists\cVal:\aEvs\fun\Val)$
$(\exists\aVal:\aEvs\fun\Val)$
$(\exists\cForm:\aEvs\fun\Formulae)$
% $(\forall\uReg{\aEv}\in\uRegs{\aEvs})$
\begin{enumerate}
\item[\ref{L1})] if $\cForm_\bEv\land\cForm_\aEv$ is satisfiable then $\bEv=\aEv$,
\item[\ref{L2})] $\labelingAct(\aEv) = \DRREF{\cVal_\aEv}{\aVal_\aEv}$,
\item[\ref{L3})] $\labelingForm(\aEv)$ implies
  \begin{math}
    \cForm_\aEv
    \land \QL{\REF{\cVal_\aEv}}{\amode}
    % \land \RO
    \land \cExp{=}\cVal_\aEv
  \end{math},
  % where    
  % $\QL{}{\mSC}=\Q{\mSC}$ and otherwise $\QL{}{\amode}=\QwREF{\cVal_\aEv}$, % for $\amode\neq\mRLX$,
\item[\ref{L4})]
  \begin{math}
    (\forall\dVal)
    (\forall\aEv\in\aEvs\cap\bEvs)
  \end{math}
  $\aTr{\bEvs}{\bForm}$ implies
  \begin{math}
    \cForm_\aEv
    \limplies (\cExp{=}\dVal)
    \limplies (\aVal_\aEv{=}\uReg{\aEv})
    \limplies \bForm[\uReg{\aEv}/\aReg]%[\uReg{\aEv}/\REF{\dVal}]
  \end{math},
  \makebox[5.75cm]{}
\item[\ref{L5})] 
  \begin{math}
    (\forall\dVal)
    (\forall\aEv\in\aEvs\setminus\cEvs)
  \end{math}
  $\aTr{\cEvs}{\bForm}$ implies
  \begin{math}
    \cForm_\aEv
    \limplies (\cExp{=}\dVal)
    \limplies \PBR{        
      %\lnot\QxREF{\dVal}\land
      \DL{\REF{\dVal}}{\amode} \land
      (\RW
      \limplies (\aVal_\aEv{=}\uReg{\aEv}\lor\REF{\dVal}{=}\uReg{\aEv}) 
      \limplies
      \bForm
      [\uReg{\aEv}/\aReg]%[\uReg{\aEv}/\REF{\dVal}]
      [\FALSE/\QL{\REF{\dVal}}{\amode}]
      )
    }      
  \end{math},
\item[\ref{L6})] % if 
  % \begin{math}
  %   (\forall\aEv\in\bEvs)(\cForm \textimplies
  %   \lnot\cForm_\aEv)
  % \end{math}
  % then
  \begin{math}
    (\forall\dVal)
    (\forall\bReg)
  \end{math}
  $\aTr{\dEvs}{\bForm}$  implies 
  \begin{math}
    (\!\not\exists\aEv\in\aEvs \suchthat \cForm_\aEv)
    \limplies (\cExp{=}\dVal)
    \limplies \PBR{        
      %\lnot\QxREF{\dVal} \land
      \DL{\REF{\dVal}}{\amode} \land
      \bForm
      [\bReg/\aReg]%[\bReg/\REF{\dVal}]
      [\FALSE/\QL{\REF{\dVal}}{\amode}]
    }.
  \end{math}
  % \\ where $\DL{}{\mRLX}=\TRUE$ and otherwise $\DL{}{\amode}=\DxREF{\dVal}$.
  % Recall that $\uRegs{\bEvs}=\{\uReg{\aEv}\mid\aEv\in\bEvs\}$.
\end{enumerate}  
% \item if $\amode=\mRLX$ and $\bEv\notin\bEvs$ then
%   \begin{math}
%     (\forall\dVal)
%   \end{math}
%   $\aTr{\bEvs}{\bForm}$ implies
%   \begin{math}
%     \cForm_\bEv
%     \limplies (\cExp{=}\dVal)
%     \limplies \PBRbig{
%     (
%     \RW
%     \limplies (\aVal{=}\uReg{\bEv}\lor\aLoc{=}\uReg{\bEv}) 
%     \limplies \bForm[\uReg{\bEv}/\aReg][\uReg{\bEv}/\REF{\dVal}]
%     )
%     \land \lnot\QxREF{\dVal}
%   }
%     \phantom{\land\; \Dx{\dVal}}
%   \end{math}
% \item if $\amode\neq\mRLX$ and $\bEv\notin\bEvs$ then
%   \begin{math}
%     (\forall\dVal)
%   \end{math}
%   $\aTr{\bEvs}{\bForm}$ implies
%   \begin{math}
%     \cForm_\bEv
%     \limplies (\cExp{=}\dVal)
%     \limplies \PBRbig{
%     (
%     \RW
%     \limplies (\aVal{=}\uReg{\bEv}\lor\aLoc{=}\uReg{\bEv}) 
%     \limplies \bForm[\uReg{\bEv}/\aReg][\uReg{\bEv}/\REF{\dVal}]
%     )
%     \land \lnot\QxREF{\dVal}
%     \land \Dx{\dVal}
%   }
%   \end{math}

\noindent
If $\aPS \in \sTHREAD{\aPSS}$ then
$(\exists\aPS_1\in\aPSS)$
\begin{enumerate}
\item[\ref{T1})]
  $\aEvs=\aEvs_1$,
\item[\ref{T2})]
  $\labelingAct(\aEv) = \labelingAct_1(\aEv)$,
\item[\ref{T3})]
  $\labelingForm(\aEv)$ implies
  $\labelingForm_1(\aEv) [\TRUE/\Qr{*}][\TRUE/\Qw{*}][\TRUE/\Qsc][\TRUE/\RW]$ if $\labelingAct_1(\aEv)$ is a write,
  \\
  $\labelingForm(\aEv)$ implies
  $\labelingForm_1(\aEv) [\TRUE/\Qr{*}][\TRUE/\Qw{*}][\TRUE/\Qsc][\FALSE/\RW]$ otherwise.
\end{enumerate}  

%       % \noindent
If $\aPS \in \sSTORE[\amode]{\cExp}{\aExp}$ then
$(\exists\cVal:\aEvs\fun\Val)$
$(\exists\aVal:\aEvs\fun\Val)$
$(\exists\cForm:\aEvs\fun\Formulae)$
\begin{enumerate}
\item if $\cForm_\bEv\land\cForm_\aEv$ is satisfiable then $\bEv=\aEv$,
\item $\labelingAct(\aEv) = \DWREFP{\cVal_\aEv}{\aVal_\aEv}$,
\item 
  $\labelingForm(\aEv)$ implies
  \begin{math}
    \cForm_\aEv
    \land \cExp{=}\cVal_\aEv
    \land \aExp{=}\aVal_\aEv
    \land \RW
    \land \Qmode{\amode}
  \end{math},
  where
  $\Qmode{\mRLX}=\QxREF{\cVal_\aEv}$ and otherwise $\Qmode{\amode}=\Q{\amode}$, % for $\amode\neq\mRLX$,
\item
  \begin{math}
    (\forall\dVal)
  \end{math}
  if
  $\bEv\in\bEvs$
  then
  $\aTr{\bEvs}{\aForm}$ implies 
  \begin{math}
    \cForm_\bEv
    \limplies (\cExp{=}\dVal)
    \limplies \PBRbig{
      (\QwREF{\dVal} \limplies \aExp{=}\aVal_\bEv)
      \land \aForm [\aExp/\REF{\dVal}]\Dmode{\amode}
    }
  \end{math},
\item %if 
  % \begin{math}
  %   (\forall\bEv\in\bEvs)(\cForm \textimplies
  %   \lnot\cForm_\bEv)
  % \end{math}
  % then
  \begin{math}
    (\forall\dVal)
  \end{math}
  $\aTr{\bEvs}{\aForm}$ implies 
  \begin{math}
    (\not\exists\bEv\in\bEvs.\; \cForm_\bEv)
    \limplies (\cExp{=}\dVal)
    \limplies \PBR{
      \lnot\QwREF{\dVal}
      \land \aForm [\aExp/\REF{\dVal}]\Dmode{\amode}
    }
  \end{math},
  \\ where 
  $\Dmode{\mRLX}=[\TRUE/\DxREF{\dVal}]$ and otherwise
  $\Dmode{\amode}=[\FALSE/\D]$. % for $\amode\neq\mRLX$.
\end{enumerate}
% \item if $\amode=\mRLX$ then
%   $\labelingForm(\aEv)$ implies
%   \begin{math}
%     \cForm_\aEv
%     \land \cExp{=}\cVal_\aEv
%     \land \aExp{=}\aVal_\aEv
%     \land \RW
%     \land \QxREF{\cVal_\aEv},
%   \end{math}
% \item if $\amode\neq\mRLX$ then
%   $\labelingForm(\aEv)$ implies
%   \begin{math}
%     \cForm_\aEv
%     \land \cExp{=}\cVal_\aEv
%     \land \aExp{=}\aVal_\aEv
%     \land \RW
%     \land \Q{},
%   \end{math}
% \item if
%   $\bEv\in\bEvs$
%   and
%   $\amode=\mRLX$ then
%   \begin{math}
%     (\forall\dVal)
%   \end{math}
%   $\aTr{\bEvs}{\aForm}$ implies 
%   \begin{math}
%     \cForm_\bEv
%     \limplies (\cExp{=}\dVal)
%     \limplies \PBRbig{
%     (\QwREF{\dVal} \limplies \aExp{=}\aVal_\bEv)
%     \land \aForm[\aExp/\REF{\dVal}][\TRUE/\DxREF{\dVal}]
%   }
%   \end{math}
% \item if
%   $\bEv\in\bEvs$
%   and
%   $\amode\neq\mRLX$ then
%   \begin{math}
%     (\forall\dVal)
%   \end{math}
%   $\aTr{\bEvs}{\aForm}$ implies 
%   \begin{math}
%     \cForm_\bEv
%     \limplies (\cExp{=}\dVal)
%     \limplies \PBRbig{
%     (\QwREF{\dVal} \limplies \aExp{=}\aVal_\bEv)
%     \land \aForm[\aExp/\REF{\dVal}][\FALSE/\D]
%   }
%   \end{math}
% \item if 
%   \begin{math}
%     (\forall\bEv\in\bEvs)(\cForm \textimplies
%     \lnot\cForm_\bEv)
%   \end{math}
%   and $\amode=\mRLX$ 
%   then
%   \begin{math}
%     (\forall\dVal)
%   \end{math}
%   $\aTr{\bEvs}{\aForm}$ implies 
%   \begin{math}
%     \cForm
%     \limplies (\cExp{=}\dVal)
%     \limplies \PBRbig{
%     \lnot\QwREF{\dVal}
%     \land \aForm[\aExp/\REF{\dVal}][\TRUE/\DxREF{\dVal}]
%   }
%   \end{math}
% \item if 
%   \begin{math}
%     (\forall\bEv\in\bEvs)
%     (\cForm \textimplies \lnot\cForm_\bEv)
%   \end{math}
%   and $\amode\neq\mRLX$ 
%   then
%   \begin{math}
%     (\forall\dVal)
%   \end{math}
%   $\aTr{\bEvs}{\aForm}$ implies 
%   \begin{math}
%     \cForm
%     \limplies (\cExp{=}\dVal)
%     \limplies \PBRbig{
%     \lnot\QwREF{\dVal}
%     \land \aForm[\aExp/\REF{\dVal}][\FALSE/\D]
%   }
%   \end{math}

\noindent
If $\aPS \in \sLOAD[\amode]{\aReg}{\cExp}$ then
$(\exists\cVal:\aEvs\fun\Val)$
$(\exists\aVal:\aEvs\fun\Val)$
$(\exists\cForm:\aEvs\fun\Formulae)$
% $(\forall\uReg{\aEv}\in\uRegs{\aEvs})$
\begin{enumerate}
\item if $\cForm_\bEv\land\cForm_\aEv$ is satisfiable then $\bEv=\aEv$,
\item $\labelingAct(\aEv) = \DRREFP{\cVal_\aEv}{\aVal_\aEv}$,
\item $\labelingForm(\aEv)$ implies
  \begin{math}
    \cForm_\aEv
    \land \cExp{=}\cVal_\aEv
    \land \RO
    \land \Qmode{\amode}
  \end{math},
  where    
  $\Qmode{\mSC}=\Q{\mSC}$ and otherwise $\Qmode{\amode}=\QwREF{\cVal_\aEv}$, % for $\amode\neq\mRLX$,
\item
  \begin{math}
    (\forall\dVal)
  \end{math}
  if $\bEv\in\bEvs$ then
  $\aTr{\bEvs}{\aForm}$ implies
  \begin{math}
    \cForm_\bEv
    \limplies (\cExp{=}\dVal)
    \limplies (\aVal{=}\uReg{\bEv})
    \limplies \aForm[\uReg{\bEv}/\aReg][\uReg{\bEv}/\REF{\dVal}]
  \end{math},
  \makebox[4.4cm]{}
\item 
  \begin{math}
    (\forall\dVal)
  \end{math}
  if $\bEv\notin\bEvs$ then
  $\aTr{\bEvs}{\aForm}$ implies
  \begin{math}
    \cForm_\bEv
    \limplies (\cExp{=}\dVal)
    \limplies \PBRbig{        
      \Dmode{\amode}
      \land \lnot\QxREF{\dVal}
      \land
      (\RW
      \limplies (\aVal{=}\uReg{\bEv}\lor\aLoc{=}\uReg{\bEv}) 
      \limplies \aForm[\uReg{\bEv}/\aReg][\uReg{\bEv}/\REF{\dVal}]
      )
    }      
  \end{math},
\item % if 
  % \begin{math}
  %   (\forall\bEv\in\bEvs)(\cForm \textimplies
  %   \lnot\cForm_\bEv)
  % \end{math}
  % then
  \begin{math}
    (\forall\dVal)
    (\forall\bReg)
  \end{math}
  $\aTr{\bEvs}{\aForm}$ implies 
  \begin{math}
    (\not\exists\bEv\in\bEvs.\; \cForm_\bEv)
    \limplies (\cExp{=}\dVal)
    \limplies \PBR{        
      \Dmode{\amode}
      \land \lnot\QxREF{\dVal}
      \land
      \limplies \aForm[\bReg/\aReg][\bReg/\REF{\dVal}]
    }      
  \end{math},
  \\ where $\Dmode{\mRLX}=\TRUE$ and otherwise $\Dmode{\amode}=\Dx{\dVal}$.
  Recall that $\uRegs{\bEvs}=\{\uReg{\bEv}\mid\bEv\in\bEvs\}$.
\end{enumerate}  
% \item if $\amode=\mRLX$ and $\bEv\notin\bEvs$ then
%   \begin{math}
%     (\forall\dVal)
%   \end{math}
%   $\aTr{\bEvs}{\aForm}$ implies
%   \begin{math}
%     \cForm_\bEv
%     \limplies (\cExp{=}\dVal)
%     \limplies \PBRbig{
%     (
%     \RW
%     \limplies (\aVal{=}\uReg{\bEv}\lor\aLoc{=}\uReg{\bEv}) 
%     \limplies \aForm[\uReg{\bEv}/\aReg][\uReg{\bEv}/\REF{\dVal}]
%     )
%     \land \lnot\QxREF{\dVal}
%   }
%     \phantom{\land\; \Dx{\dVal}}
%   \end{math}
% \item if $\amode\neq\mRLX$ and $\bEv\notin\bEvs$ then
%   \begin{math}
%     (\forall\dVal)
%   \end{math}
%   $\aTr{\bEvs}{\aForm}$ implies
%   \begin{math}
%     \cForm_\bEv
%     \limplies (\cExp{=}\dVal)
%     \limplies \PBRbig{
%     (
%     \RW
%     \limplies (\aVal{=}\uReg{\bEv}\lor\aLoc{=}\uReg{\bEv}) 
%     \limplies \aForm[\uReg{\bEv}/\aReg][\uReg{\bEv}/\REF{\dVal}]
%     )
%     \land \lnot\QxREF{\dVal}
%     \land \Dx{\dVal}
%   }
%   \end{math}

%       % \noindent
If $\aPS \in \sSTORE[\amode]{\cExp}{\aExp}$ then
$(\exists\cVal:\aEvs\fun\Val)$
$(\exists\aVal:\aEvs\fun\Val)$
$(\exists\bForm:\aEvs\fun\Formulae)$
\begin{enumerate}
\item if $\bForm_\bEv\land\bForm_\aEv$ is satisfiable then $\bEv=\aEv$,
\item $\labelingAct(\aEv) = \DWREFP{\cVal_\aEv}{\aVal_\aEv}$,
\item 
  $\labelingForm(\aEv)$ implies
  \begin{math}
    \bForm_\aEv
    \land \cExp{=}\cVal_\aEv
    \land \aExp{=}\aVal_\aEv
    \land \RW
    \land \QS{}{\amode}
  \end{math},
\item
  \begin{math}
    (\forall\dVal)
  \end{math}
  if
  $\bEv\in\bEvs$
  then
  $\aTr[\bEvs](\aForm)$ implies 
  \begin{math}
    \bForm_\bEv
    \limplies (\cExp{=}\dVal)
    \limplies \PBRbig{
      \aExp{=}\aVal_\bEv
      \land \DS{\REF{\dVal}}{\amode}{\aForm[\aExp/\REF{\dVal}]}
    }
  \end{math},
\item 
  \begin{math}
    (\forall\dVal)
  \end{math}
  $\aTr[\bEvs](\aForm)$ implies 
  \begin{math}
    (\not\exists\bEv\in\bEvs.\; \bForm_\bEv)
    \limplies (\cExp{=}\dVal)
    \limplies \PBR{
      \lnot\Q{\mRA}
      \land \DS{\REF{\dVal}}{\amode}{\aForm[\aExp/\REF{\dVal}]}
    }.
  \end{math}
\end{enumerate}

\noindent
If $\aPS \in \sLOAD[\amode]{\aReg}{\cExp}$ then
$(\exists\cVal:\aEvs\fun\Val)$
$(\exists\aVal:\aEvs\fun\Val)$
$(\exists\bForm:\aEvs\fun\Formulae)$
\begin{enumerate}
\item if $\bForm_\bEv\land\bForm_\aEv$ is satisfiable then $\bEv=\aEv$,
\item $\labelingAct(\aEv) = \DRREFP{\cVal_\aEv}{\aVal_\aEv}$,
\item $\labelingForm(\aEv)$ implies
  \begin{math}
    \bForm_\aEv
    \land \cExp{=}\cVal_\aEv
    \land \RO
    \land \QL{}{\amode}
  \end{math},
\item
  \begin{math}
    (\forall\dVal)
  \end{math}
  if $\bEv\in\bEvs$ then
  $\aTr[\bEvs](\aForm)$ implies
  \begin{math}
    \bForm_\bEv
    \limplies (\cExp{=}\dVal)
    \limplies (\aVal{=}\uReg{\bEv})
    \limplies \aForm[\uReg{\bEv}/\aReg][\uReg{\bEv}/\REF{\dVal}]
  \end{math},
  \makebox[4.8cm]{}
\item 
  \begin{math}
    (\forall\dVal)
  \end{math}
  if $\bEv\notin\bEvs$ then
  $\aTr[\bEvs](\aForm)$ implies
  \begin{math}
    \bForm_\bEv
    \limplies (\cExp{=}\dVal)
    \limplies \PBRbig{        
      \DL{\REF{\dVal}}{\amode}
      \land \lnot\Q{\mRA}
      \land
      (\RW
      \limplies (\aVal{=}\uReg{\bEv}\lor\aLoc{=}\uReg{\bEv}) 
      \limplies \aForm[\uReg{\bEv}/\aReg][\uReg{\bEv}/\REF{\dVal}]
      )
    }      
  \end{math},
\item 
  \begin{math}
    (\forall\dVal)
    (\forall\bReg)
  \end{math}
  $\aTr[\bEvs](\aForm)$ implies 
  \begin{math}
    (\not\exists\bEv\in\bEvs.\; \bForm_\bEv)
    \limplies (\cExp{=}\dVal)
    \limplies \PBR{        
      \DL{\REF{\dVal}}{\amode}
      \land \lnot\Q{\mRA}
      \land
      \limplies \aForm[\bReg/\aReg][\bReg/\REF{\dVal}]
    }.
  \end{math}
\end{enumerate}  

%     \end{minipage}
%   \end{center}
%   \caption{Full Semantics with Address Calculation
%     (See \refdef{def:QS} for $\QS{\aLoc}{\amode}$, $\QL{\aLoc}{\amode}$
%     and \refdef{def:DS} for $\DL{\aLoc}{\amode}$, $\DS{\aLoc}{\amode}$)
%   }
%   \label{fig:full}
% \end{figure*}    

%\section{Discussion}
\subsection{Downset Closure}
\label{sec:downset}

% We would like the semantics to be closed with respect to \emph{augments} and
% \emph{downsets}.

% Augments include more order and stronger formulae; in examples, we typically
% consider pomsets that are augment-minimal.  One intuitive reading of augment
% closure is that adding order can only cause preconditions to weaken.
% \begin{definition}
%   \label{def:augment}
%   $\aPS_2$ is an \emph{augment} of $\aPS_1$ if
%   \begin{enumerate}
%   \item $\aEvs_2=\aEvs_1$,
%   \item $\labelingAct_2(\aEv)=\labelingAct_1(\aEv)$,
%   \item $\labelingForm_2(\aEv) \rimplies \labelingForm_1(\aEv)$,
%   \item $\aTr[2]{\bEvs}{\aEv} \rimplies \aTr[1]{\bEvs}{\aEv}$,
%   \item if $\bEv\le_2\aEv$ then $\bEv\le_1\aEv$.
%   \end{enumerate}
% \end{definition}

% \begin{proposition}
%   %   Suppose $\aPS_1\in\sem{\aCmd}$.
%   If $\aPS_1\in\sem{\aCmd}$ and $\aPS_2$  augments $\aPS_1$ then $\aPS_2\in\sem{\aCmd}$.
%   % \item If $\aPS_2$ is a downset of $\aPS_1$ then $\aPS_2\in\sem{\aCmd}$.
%   % \end{enumerate}
% \end{proposition}

We would like the semantics to be closed with respect to \emph{downsets}.
Downsets include a subset of initial events, similar to \emph{prefixes} for
strings.
\begin{definition}
  \label{def:downset}
  $\aPS_2$ is an \emph{downset} of $\aPS_1$ if
  \begin{multicols}{2}
    \begin{enumerate}
    \item $\aEvs_2\subseteq\aEvs_1$,
    \item $(\forall \aEv\in\aEvs_2)$ $\labelingAct_2(\aEv)=\labelingAct_1(\aEv)$,
    \item $(\forall \aEv\in\aEvs_2)$ $\labelingForm_2(\aEv)\riff\labelingForm_1(\aEv)$,
    \item $(\forall \aEv\in\aEvs_2)$ $\aTr[2]{\bEvs}{\aEv}\riff\aTr[1]{\bEvs}{\aEv}$,
    \item $\aTerm[2] \rimplies \aTerm[1]$,
      \stepcounter{enumi}
    \item[] 
      \begin{enumerate}[leftmargin=0pt]
      \item $(\forall \bEv\in\aEvs_2)$ $(\forall \aEv\in\aEvs_2)$ $\bEv\le_2\aEv$ iff $\bEv\le_1\aEv$,
      \item $(\forall \bEv\in\aEvs_1)$ $(\forall \aEv\in\aEvs_2)$ if
        $\bEv\le_1\aEv$ then $\bEv\in\aEvs_2$,
      \end{enumerate}
    \item $(\forall \bEv\in\aEvs_2)$ $(\forall \aEv\in\aEvs_2)$ $\bEv\rrfx_2\aEv$ iff $\bEv\rrfx_1\aEv$.
    \end{enumerate}
  \end{multicols}
\end{definition}

Downset closure fails due to for two reasons.  The key property is that the
empty set transformer should behave the same as the independent transformer.

First, downset closure fails for read-read independency \textsection\ref{sec:read-read}.
  % For \xRRD{}, \refdef{def:pomsets-rr} states:
  % \begin{enumerate}
  % \item[\ref{L4})]
  %   $\aTr{\bEvs}{\bForm} \rimplies \aVal{=}\aReg\limplies\bForm$, 
  % \item[\ref{L5})]
  %   $\aTr{\cEvs}{\bForm} \rimplies (\aVal{=}\aReg\lor\RW)\limplies\bForm$,
  % \item[\ref{L6})] 
  %   $\aTr{\dEvs}{\bForm} \rimplies \bForm$, when $\aEvs=\emptyset$.
  % \end{enumerate}
  % This semantics is not downset closed due to the lack of read-read dependencies.
  % In both cases, for subsequent writes, \ref{L5} is the same as \ref{L6}.  For
  % subsequent reads, \ref{L5} is the same as \ref{L4}.
Consider
\begin{gather*}
  \begin{gathered}[t]
    \PR{x}{r}\SEMI\IF{\BANG r}\THEN\PR{y}{s}\FI
    \\
    \hbox{\begin{tikzinline}[node distance=.5em and 1.5em]
        \event{a}{\DR{x}{0}}{}
        \event{b}{\DR{y}{0}}{right=of a}
      \end{tikzinline}}
  \end{gathered}    
\end{gather*}
The semantics of this program includes the singleton pomset $\DRP{x}{0}$,
but not the singleton pomset $\DRP{y}{0}$.
To get $\DRP{x}{0}$, we combine:
\begin{align*}
  \begin{gathered}[t]
    \PR{x}{r}
    \\
    \hbox{\begin{tikzinline}[node distance=.5em and 1.5em]
        \event{a}{\DR{x}{0}}{}
      \end{tikzinline}}
  \end{gathered}    
  &&
  \begin{gathered}[t]
    \IF{\BANG r}\THEN\PR{y}{s}\FI
    \\
    \emptyset
  \end{gathered}    
\end{align*}
Attempting to get $\DRP{y}{0}$, we instead get:
\begin{align*}
  \begin{gathered}[t]
    \PR{x}{r}
    \\
    \emptyset
  \end{gathered}    
  &&
  \begin{gathered}[t]
    \IF{\BANG r}\THEN\PR{y}{s}\FI
    \\
    \hbox{\begin{tikzinline}[node distance=.5em and 1.5em]
        \event{b}{r\EQ0\mid\DR{y}{0}}{}
      \end{tikzinline}}
  \end{gathered}    
\end{align*}
Since $r$ appears only once in the program, this pomset cannot contribute
to a top-level pomset.


Second, the semantics is not downset closed because the independency reasoning of
\ref{read-tau-ind} is only applicable for pomsets where the ignored read is present!
Revisiting \jmm{} causality test case 1 from the end of \textsection\ref{sec:ex:control}:
\begin{align*}
  \begin{gathered}[t]
    \PW{x}{0} 
    \\
    \hbox{\begin{tikzinline}[node distance=.5em and 1.5em]
        \event{a0}{\DW{x}{0}}{}
        \xform{xi}{\bForm[0/x]}{below=of a0}
      \end{tikzinline}}    
  \end{gathered}
  &&
  \begin{gathered}[t]
    \PR{x}{r} 
    \\
    \hbox{\begin{tikzinline}[node distance=.5em and 1.5em]
        \event{a1}{\DR{x}{1}}{}
        \xform{xi}{(1{=}r\lor x{=}r)\limplies\bForm}{below=of a1}
      \end{tikzinline}}    
  \end{gathered}
  &&
  \begin{gathered}[t]
    \IF{r{\geq}0}\THEN \PW{y}{1} \FI
    \SEMI
    \PW{z}{r}
    \\
    \hbox{\begin{tikzinline}[node distance=.5em and 1.5em]
        \event{a2}{r{\geq}0\mid\DW{y}{1}}{}      
        \event{a3}{r{=}1\mid\DW{z}{1}}{right=of a2}      
      \end{tikzinline}}    
  \end{gathered}
\end{align*}
% Composing:
\begin{align*}
  \begin{gathered}[t]
    \PW{x}{0} 
    \SEMI\PR{x}{r} 
    \SEMI\IF{r{\geq}0}\THEN \PW{y}{1} \FI
    \SEMI
    \PW{z}{r}
    \\
    \hbox{\begin{tikzinline}[node distance=.5em and 1.5em]
        \event{a0}{\DW{x}{0}}{}
        \event{a1}{\DR{x}{1}}{right=of a0}
        \event{a2}{(1{=}r\lor 0{=}r)\limplies r{\geq}0\mid\DW{y}{1}}{right=of a1}      
        \event{a3}{1{=}r\limplies r{=}1\mid\DW{z}{1}}{right=of a2}
        \po[out=-15,in=-165]{a1}{a3}
        \wki{a0}{a1}
      \end{tikzinline}}    
  \end{gathered}
\end{align*}
The precondition of $\DWP{y}{1}$ is a tautology.

Taking the empty set for the read, however,
the precondition of $\DWP{y}{1}$ is not a tautology:
\begin{align*}
  \begin{gathered}[t]
    \PW{x}{0} 
    \SEMI\PR{x}{r} 
    \SEMI\IF{r{\geq}0}\THEN \PW{y}{1} \FI
    \SEMI
    \PW{z}{r}
    \\
    \hbox{\begin{tikzinline}[node distance=.5em and 1.5em]
        \event{a0}{\DW{x}{0}}{}
        % \event{a1}{\DR{x}{1}}{right=of a0}
        \event{a2}{r{\geq}0\mid\DW{y}{1}}{right=6em of a0}      
        \event{a3}{r{=}1\mid\DW{z}{1}}{right=of a2}
        % \wk{a0}{a1}
      \end{tikzinline}}    
  \end{gathered}
\end{align*}
One way to deal with the second issue would be to allow general access
elimination to merge $\DWP{x}{0}$ and $\DRP{x}{0}$:
\begin{align*}
  \begin{gathered}[t]
    \PW{x}{0} 
    \SEMI\PR{x}{r} 
    \SEMI\IF{r{\geq}0}\THEN \PW{y}{1} \FI
    \SEMI
    \PW{z}{r}
    \\
    \hbox{\begin{tikzinline}[node distance=.5em and 1.5em]
        \event{a0}{\DW{x}{0}}{}
        %\event{a1}{\DR{x}{1}}{right=of a0}
        \event{a2}{(0{=}r\lor 0{=}r)\limplies r{\geq}0\mid\DW{y}{1}}{right=6em of a0}      
        \event{a3}{r{=}1\mid\DW{z}{1}}{right=of a2}
        %\po[out=-15,in=-165]{a1}{a3}
        %\wki{a0}{a1}
      \end{tikzinline}}    
  \end{gathered}
\end{align*}
We leave the elaboration of this idea to future work.

\begin{comment}
  if in L6 we said [x/r], that says we know read the local version...  ignoring
  the value read...  Perhaps there is some intervening stuff that stops you
  from seeing the local state, such as release-acquire

  We could potentially get rid of [x/r] If you do two reads, your not allowed
  to be independent of the second based on the value that was read in the first
  read.

  x=0; r=x; if (r=1) { s=x; if (s=?) {y=1}}
  read 1 then 2.


  In order for the write to be independent of second read what does its
  precondition have to be.
  [r/x] then s==1
  no sub then s==0

  (s=? | Wy1)

  if (phi) z=1
  phi = s is even
  phi = s < 2

  With substitution you are saying you know that the ``local copy'' of x is the
  same as r.  Sitting in the local cache.  Read might have gone to main
  memory, but if it did it has updated the cache line so that the local copy is
  what I just read.

  If second read is a cache hit, then I know that I am seeing the same value.

  If we take substitution out then 
\end{comment}

\subsection{Logical Encoding of Delay for \PwTmcaTITLE{}}
\label{sec:delay}

\PwTmca{} satisfies one direction of
\reflem{lem:if}\eqref{lem:ifelse:if:if1}--\eqref{lem:ifelse:if:if2}
\begin{enumerate}
\item[\eqref{lem:ifelse:if:if1}]
  \begin{math} 
    \xIFTHEN{\aForm}{\aPSS_1}{\aPSS_2}
    \supseteq
    \xSEMI{
      \xIFTHEN{\aForm}{\aPSS_1}{}
    }{
      \xIFTHEN{\lnot\aForm}{\aPSS_2}{}
    }.
  \end{math}
  
\item[\eqref{lem:ifelse:if:if2}]
  \begin{math} 
    \xIFTHEN{\aForm}{\aPSS_1}{\aPSS_2}
    \supseteq
    \xSEMI{
      \xIFTHEN{\lnot\aForm}{\aPSS_2}{}
    }{
      \xIFTHEN{\aForm}{\aPSS_1}{}
    }.
  \end{math}
\end{enumerate}
In order to validate the reverse inclusions, we could require that
\ref{seq-le-delays} not impose order when
$\labelingForm_1(\bEv) \land \labelingForm_2(\aEv)$ is unsatisfiable.
Thus, following on \textsection\ref{sec:false}, we would also like this:
\begin{enumerate}
\item[{\labeltext[\textsc{s}6b$'$]{(\textsc{s}6b$'$)}{seq-le-delays'}}] if
  $\labeling_1(\bEv) \rdelays \labeling_2(\aEv)$ and
  $\labelingForm_1(\bEv) \land \labelingForm_2'(\aEv)$ is
  $\labeling$-consistent then $\bEv\le\aEv$.
\end{enumerate}

However, \eqref{seq-le-delays'} fails associativity.
Example where $\cForm_\labeling=(r{=}0)$
\begin{align*}
  \begin{gathered}    
    \PR{y}{r}
    \\
    \hbox{\begin{tikzinline}[node distance=1.5em]
        \event{a}{\DR{y}{0}}{}
      \end{tikzinline}}
  \end{gathered}  
  &&
  \begin{gathered}    
    \IF{r\OR s}\THEN\PW{x}{1}\FI
    \\
    \hbox{\begin{tikzinline}[node distance=1.5em]
        \event{b}{r{\neq}0\lor s{\neq}0\mid\DW{x}{1}}{}
      \end{tikzinline}}
  \end{gathered}    
  &&
  \begin{gathered}    
    \IF{\BANG s}\THEN\PW{x}{2}\FI
    \\
    \hbox{\begin{tikzinline}[node distance=1.5em]
        \event{c}{s{=}0\mid\DW{x}{2}}{}
      \end{tikzinline}}
  \end{gathered}    
\end{align*}
Associating right, order is required since
$((r{\neq}0 \lor s{\neq}0)\land s{=}0)$ is satisfiable (take $r{=}1$ and $s{=}0$):
\begin{align*}
  \begin{gathered}    
    \PR{y}{r}
    \\
    \hbox{\begin{tikzinline}[node distance=1.5em]
        \event{a}{\DR{y}{0}}{}
      \end{tikzinline}}
  \end{gathered}    
  &&
  \begin{gathered}    
    \IF{r\OR s}\THEN\PW{x}{1}\FI
    \SEMI
    \IF{\BANG s}\THEN\PW{x}{2}\FI
    \\
    \hbox{\begin{tikzinline}[node distance=1.5em]
        \event{b}{r{\neq}0\lor s{\neq}0\mid\DW{x}{1}}{}
        \event{c}{s{=}0\mid\DW{x}{2}}{right=of b}
        \wki{b}{c}
      \end{tikzinline}}
  \end{gathered}    
\end{align*}
\begin{align*}
  \begin{gathered}    
    \PR{y}{r}
    \SEMI
    \IF{r\OR s}\THEN\PW{x}{1}\FI
    \SEMI
    \IF{\BANG s}\THEN\PW{x}{2}\FI
    \\
    \hbox{\begin{tikzinline}[node distance=1.5em]
        \event{a}{\DR{y}{0}}{}
        \event{b}{r{=}0\limplies (r{\neq}0\lor s{\neq}0)\mid\DW{x}{1}}{right=of a}
        \event{c}{s{=}0\mid\DW{x}{2}}{right=of b}
        \po{a}{b}
        \wki{b}{c}
      \end{tikzinline}}
  \end{gathered}    
\end{align*}
Associating left, order is not required between the writes since
$(s{\neq}0\land s{=}0)$ is unsatisfiable:
\begin{align*}
  \begin{gathered}    
    \PR{y}{r}
    \SEMI
    \IF{r\OR s}\THEN\PW{x}{1}\FI
    \\
    \hbox{\begin{tikzinline}[node distance=1.5em]
        \event{a}{\DR{y}{0}}{}
        \event{b}{r{=}0\limplies (r{\neq}0\lor s{\neq}0)\mid\DW{x}{1}}{right=of a}
        \po{a}{b}
      \end{tikzinline}}
  \end{gathered}    
  &&
  \begin{gathered}    
    \IF{\BANG s}\THEN\PW{x}{2}\FI
    \\
    \hbox{\begin{tikzinline}[node distance=1.5em]
        \event{c}{s{=}0\mid\DW{x}{2}}{}
      \end{tikzinline}}
  \end{gathered}    
\end{align*}
\begin{align*}
  \begin{gathered}    
    \PR{y}{r}
    \SEMI
    \IF{r\OR s}\THEN\PW{x}{1}\FI
    \SEMI
    \IF{\BANG s}\THEN\PW{x}{2}\FI
    \\
    \hbox{\begin{tikzinline}[node distance=1.5em]
        \event{a}{\DR{y}{0}}{}
        \event{b}{r{=}0\limplies (r{\neq}0\lor s{\neq}0)\mid\DW{x}{1}}{right=of a}
        \event{c}{s{=}0\mid\DW{x}{2}}{right=of b}
        \po{a}{b}
      \end{tikzinline}}
  \end{gathered}    
\end{align*}

This motivates the logic-based presentation of $\rdelay$.

In the data model, we require additional symbols: $\Q{\mSC}$, $\Qr{\aLoc}$,
and $\Qw{\aLoc}$.  We refer to these collectively as \emph{quiescence
  symbols}.

We update the \refdef{def:pomset} of complete pomset to substitute true for every
quiescence symbol:
\begin{definition}
  A \PwT{} is \emph{complete} if
  \begin{multicols}{2}
    \begin{enumerate}[,label=(\textsc{c}\arabic*),ref=\textsc{c}\arabic*]
      \setcounter{enumi}{\value{Bkappa}}
    \item \label{top-kappa-q}
      $\labelingForm(\aEv)[\TRUE/\Q{}]$ is a tautology,
      \setcounter{enumi}{\value{Bterm}}
    \item \label{top-term-q} $\aTerm{}[\TRUE/\Q{}]$ is a tautology.
    \end{enumerate}
  \end{multicols}
\end{definition}

We define some helper notation:
\begin{definition}
  \label{def:QS}
  Let $\Qr{*}=\textstyle\bigwedge_\bLoc \Qr{\bLoc}$, and similarly for $\Qw{*}$.\\
  Let formulae $\QS{\aLoc}{\amode}$, $\QL{\aLoc}{\amode}$, and $\QF{}{\amode}$ be defined:
  \begin{scope}
    \small
    \begin{align*}
      \QS{\aLoc}{\mRLX}&=\Qr{\aLoc}\land\Qw{\aLoc}
      &\QL{\aLoc}{\mRLX}&=\Qw{\aLoc}
      &\QF{}{\fREL}&=\Qr{*}\land\Qw{*} 
      \\
      \QS{\aLoc}{\mREL}&= \Qr{*}\land\Qw{*} 
      &\QL{\aLoc}{\mACQ}&=\Qw{\aLoc}
      &\QF{}{\fACQ} &=\Qr{*}
      \\
      \QS{\aLoc}{\mSC}&= \Qr{*}\land\Qw{*} \land \Qsc
      &\QL{\aLoc}{\mSC}&=\Qw{\aLoc}\land\Qsc      
      &\QF{}{\fSC} &= \Qr{*}\land\Qw{*} \land\Qsc
    \end{align*}
  \end{scope}
  % \end{definition}
% \begin{definition}
  Let $[\aForm/\Qr{*}]$ substitute $\aForm$ for every $\Qr{\bLoc}$, and similarly for $\Qw{*}$.\\
  Let substitutions $[\aForm/\QS{\aLoc}{\amode}]$, $[\aForm/\QL{\aLoc}{\amode}]$, and  $[\aForm/\QF{}{\amode}]$ be defined:
  \begin{scope}
    \small
    \begin{align*}
      [\aForm/\QS{\aLoc}{\mRLX}] &= [\aForm/\Qw{\aLoc}]
      &{} [\aForm/\QL{\aLoc}{\mRLX}] &= [\aForm/\Qr{\aLoc}]
      &{} [\aForm/\QF{}{\fREL}] &= [\aForm/\Qw{*}]
      \\
      [\aForm/\QS{\aLoc}{\mREL}] &= [\aForm/\Qw{\aLoc}]
      &{} [\aForm/\QL{\aLoc}{\mACQ}] &= [\aForm/\Qr{*},\aForm/\Qw{*}]
      &{} [\aForm/\QF{}{\fACQ}] &= [\aForm/\Qr{*},\aForm/\Qw{*}]
      \\
      [\aForm/\QS{\aLoc}{\mSC}] &= [\aForm/\Qw{\aLoc},\aForm/\Qsc]
      &{} [\aForm/\QL{\aLoc}{\mSC}] &= [\aForm/\Qr{*},\aForm/\Qw{*},\aForm/\Qsc]
      &{} [\aForm/\QF{}{\fSC}] &= [\aForm/\Qr{*},\aForm/\Qw{*},\aForm/\Qsc]
    \end{align*}
  \end{scope}
\end{definition}
Update the following rules from \reffig{fig:seq}.
(The change is similar for address calculation and if-closure.)
\begin{enumerate}[topsep=0pt,label=(\textsc{w}\arabic*),ref=\textsc{w}\arabic*]
  \setcounter{enumi}{\value{Bkappa}}
\item \label{write-kappa-q}
  \begin{math}
    \labelingForm(\aEv) \riff
    \aExp{=}\aVal
    \land
    \QS{\aLoc}{\amode}
  \end{math},    
  \stepcounter{enumi}      
\item[] \labeltext[\textsc{w}5]{}{write-tau-q}
  \begin{enumerate}[leftmargin=0pt]
  \item \label{write-tau-nonempty-q}
    if $\aEvs\neq\emptyset$ then 
    \makebox[0cm][l]{%
      \begin{math}
        \aTr{\bEvs}{\bForm} \riff 
        \bForm
        [\aExp/\aLoc][\aExp{=}\aVal/\QS{\aLoc}{\amode}],
      \end{math}}
  \item \label{write-tau-empty-q}
    if $\aEvs=\emptyset$ then 
    \begin{math}
      \aTr{\bEvs}{\bForm} \riff 
      \bForm
      [\aExp/\aLoc][\FALSE/\QS{\aLoc}{\amode}],
    \end{math}
  \end{enumerate}
\end{enumerate}
\begin{enumerate}[topsep=0pt,label=(\textsc{r}\arabic*),ref=\textsc{r}\arabic*]
  \setcounter{enumi}{\value{Bkappa}}
\item \label{read-kappa-q}
  \begin{math}
    \labelingForm(\aEv) \riff \QL{\aLoc}{\amode},
  \end{math}
  \stepcounter{enumi}
\item[] \labeltext[\textsc{r}4]{}{read-tau-q}
  \begin{enumerate}[leftmargin=0pt]
  \item \label{read-tau-dep-q}
    if $\aEvs\neq\emptyset$ and $(\aEvs\cap\bEvs)\neq\emptyset$ then
    \makebox[0pt][l]{\begin{math}
        \aTr{\bEvs}{\bForm} \riff
        \aVal{=}\aReg
        \limplies \bForm,
      \end{math}}
  \item \label{read-tau-ind-q}
    if $\aEvs\neq\emptyset$ and $(\aEvs\cap\bEvs)=\emptyset$ then
    \makebox[0pt][l]{\begin{math}
        \aTr{\bEvs}{\bForm} \riff
        \PBR{\aVal{=}\aReg \lor \aLoc{=}\aReg}
        \limplies \bForm [\FALSE/\QL{\aLoc}{\amode}],
      \end{math}}
  \item \label{read-tau-empty-q}
    if $\aEvs=\emptyset$ then
    \begin{math}
      \aTr{\bEvs}{\bForm} \riff
      \bForm [\FALSE/\QL{\aLoc}{\amode}],
    \end{math}
  \end{enumerate}
\end{enumerate}    
\begin{enumerate}[topsep=0pt,label=(\textsc{f}\arabic*),ref=\textsc{f}\arabic*]
  \setcounter{enumi}{\value{Bkappa}}
\item \label{fence-kappa-q}
  $\labelingForm(\aEv) \riff \QF{\aLoc}{\amode}$,
  \stepcounter{enumi}
\item[] \labeltext[\textsc{f}4]{}{fence-tau-q}
  \begin{enumerate}[leftmargin=0pt]
  \item \label{fence-tau-dep-q}
    if $\aEvs\neq\emptyset$ then
    \begin{math}
      \aTr{\bEvs}{\bForm} \riff
      \bForm,
    \end{math}
  \item \label{fence-tau-ind-q}
    if $\aEvs=\emptyset$ then
    \begin{math}
      \aTr{\bEvs}{\bForm} \riff
      \bForm [\FALSE/\QF{\aLoc}{\amode}].
    \end{math}
  \end{enumerate}
\end{enumerate}
  
The quiescence formulae indicate what must precede an event.
For example, all preceding accesses must be ordered before a releasing write,
whereas only writes on $x$ must be ordered before a releasing read on $x$.

The quiescence substitutions update quiescence symbols in subsequent code.
For subsequent independent code, $\ref{write-kappa-q}$ and $\ref{read-kappa-q}$ substitute false.
In complete pomsets, we substitute true for .
%
For example, we substitute $\FALSE$ for $\QS{\aLoc}{\mREL}$ in the independent
case for a releasing write; this ensures that subsequent writes to $\aLoc$
follow the releasing write in top-level pomsets.  Similarly, we substitute
$\FALSE$ for $\QL{\aLoc}{\mACQ}$ in the independent case for an acquiring
write; this ensures that all subsequent accesses follow the acquiring read in
top-level pomsets.

\todo{Fix these examples}
\begin{example}
  \label{ex:q1}
  The following pomsets show the effect of quiescence for each access mode.
  \begin{align*}
  \begin{gathered}
    \begin{gathered}[t]
      \PW{x}{\aExp}
      \\
      \hbox{\begin{tikzinline}[node distance=.5em and 1.5em]
          \event{a}{\aExp{=}v\land\Qr{x}\land\Qw{x}\mid\DW{x}{v}}{}
          \xform{xi}{\bForm[\FALSE/\Qw{x}]}{above=of a}
          \xform{xd}{\bForm[(\Qw{x}\land\aExp{=}\aVal)/\Qw{x}]}{below=of a}
          \xo{a}{xd}
        \end{tikzinline}}
    \end{gathered}
    \\[1ex]
    \begin{gathered}[t]
      \PW[\mRA]{x}{\aExp}
      \\
      \hbox{\begin{tikzinline}[node distance=.5em and 1.5em]
          \raevent{a}{\aExp{=}v\land\Qr{*}\land\Qw{*}\mid\DW[\mRA]{x}{v}}{}
          \xform{xi}{\bForm[\FALSE/\Qw{x}]}{above=of a}
          \xform{xd}{\bForm[(\Qw{x}\land\aExp{=}\aVal)/\Qw{x}]}{below=of a}
          \xo{a}{xd}
        \end{tikzinline}}
    \end{gathered}
    \\[1ex]
    \begin{gathered}[t]
      \PW[\mSC]{x}{\aExp}
      \\
      \hbox{\begin{tikzinline}[node distance=.5em and 1.5em]
          \scevent{a}{\aExp{=}v\land\Qr{*}\land\Qw{*}\land\Qsc\mid\DW[\mSC]{x}{v}}{}
          \xform{xi}{\bForm[\FALSE/\Qw{x}][\FALSE/\Qsc]}{above=of a}
          \xform{xd}{\bForm[(\Qw{x}\land\aExp{=}\aVal)/\Qw{x}]}{below=of a}
          \xo{a}{xd}
        \end{tikzinline}}
    \end{gathered}
  \end{gathered}
  &&
  \begin{gathered}
    \begin{gathered}[t]
      \PR{x}{r}
      \\
      \hbox{\begin{tikzinline}[node distance=.5em and 1.5em]
          \event{a}{\Qw{x}\mid\DR{x}{v}}{}
          \xform{xi}{\bForm[\FALSE/\Qr{x}]}{above=of a}
          \xform{xd}{v{=}r\limplies\bForm}{below=of a}
          \xo{a}{xd}
        \end{tikzinline}}
    \end{gathered}
    \\[1ex]
    \begin{gathered}[t]
      \PR[\mRA]{x}{r}
      \\
      \hbox{\begin{tikzinline}[node distance=.5em and 1.5em]
          \raevent{a}{\Qw{x}\mid\DR[\mRA]{x}{v}}{}
          \xform{xi}{\bForm[\FALSE/\Qr{*}][\FALSE/\Qw{*}]}{above=of a}
          \xform{xd}{v{=}r\limplies\bForm}{below=of a}
          \xo{a}{xd}
        \end{tikzinline}}
    \end{gathered}
    \\[1ex]
    \begin{gathered}[t]
      \PR[\mSC]{x}{r}
      \\
      \hbox{\begin{tikzinline}[node distance=.5em and 1.5em]
          \scevent{a}{\Qw{x}\land\Qsc\mid\DR[\mSC]{x}{v}}{}
          \xform{xi}{\bForm[\FALSE/\Qr{*}][\FALSE/\Qw{*}][\FALSE/\Qsc]}{above=of a}
          \xform{xd}{v{=}r\limplies\bForm}{below=of a}
          \xo{a}{xd}
        \end{tikzinline}}
    \end{gathered}
  \end{gathered}
\end{align*}
\end{example}

\begin{example}
  The definition enforces publication.  Consider:
  \begin{align*}
    \begin{gathered}
      \PW{x}{1}
      \\
      \hbox{\begin{tikzinline}[node distance=.5em and 1.5em]
          \event{a1}{1{=}1\land\QS{\aLoc}{\mRLX}\mid\DW{x}{1}}{}
          \xform{x1d}{1{=}1\land\bForm}{below=of a1}
          \xform{x1i}{\bForm[\FALSE/\QS{\aLoc}{\mRLX}]}{below=of x1d}
          \xos{a1}{x1d}
        \end{tikzinline}}
    \end{gathered}
    &&
    \begin{gathered}
      \PW[\mREL]{y}{1}
      \\
      \hbox{\begin{tikzinline}[node distance=.5em and 1.5em]
          \raevent{a2}{1{=}1\land\QS{\bLoc}{\mREL}\mid\DW{y}{2}}{}
          \xform{x2d}{1{=}1\land\bForm}{below=of a2}
          \xform{x2i}{\bForm[\FALSE/\QS{\bLoc}{\mREL}]}{below=of x2d}
          \xos{a2}{x2d}
        \end{tikzinline}}
    \end{gathered}
  \end{align*}
  Since $\QS{\bLoc}{\mREL}[\FALSE/\QS{\aLoc}{\mRLX}]$ is $\FALSE$,
  composing these without order simplifies to:
  \begin{gather*}
    \PW{x}{1}\SEMI \PW[\mREL]{y}{1}
    \\
    \hbox{\begin{tikzinline}[node distance=.5em and 1.5em]
          \event{a1}{\QS{\aLoc}{\mRLX}\mid\DW{x}{1}}{}
          \xform{x1d}{\bForm}{below right=of a1}
          \xform{x2i}{\bForm[\FALSE/\QS{\bLoc}{\mREL}]}{below=of a1}
          \xo{a1}{x1d}
          \raevent{a2}{\FALSE\mid\DW{\bLoc}{1}}{above right=of x1d}
          %\xform{x2d}{\bForm}{below left=of a2}
          \xform{x1i}{\bForm[\FALSE/\QS{\aLoc}{\mRLX}]}{below=of a2}
          \xform{xii}{\bForm[\FALSE/\QS{\bLoc}{\mREL}][\FALSE/\QS{\aLoc}{\mRLX}]}{below right=of a2}
          \xo{a2}{x1d}
          \xos[xleft]{a1}{x2i}
          \xos{a2}{x1i}
        \end{tikzinline}}
  \end{gather*}
  In order to get a satisfiable precondition for $\DWP{y}{1}$, we must
  introduce order:
  \begin{gather*}
    % \PW{x}{1}\SEMI \PW[\mREL]{y}{1}
    % \\
    \hbox{\begin{tikzinline}[node distance=.5em and 1.5em]
          \event{a1}{\QS{\aLoc}{\mRLX}\mid\DW{x}{1}}{}
          \xform{x1d}{\bForm}{below right=of a1}
          \xform{x2i}{\bForm[\FALSE/\QS{\bLoc}{\mREL}]}{below=of a1}
          \xo{a1}{x1d}
          \raevent{a2}{\QS{\bLoc}{\mREL}\mid\DW{\bLoc}{1}}{above right=of x1d}
          %\xform{x2d}{\bForm}{below left=of a2}
          \xform{x1i}{\bForm[\FALSE/\QS{\aLoc}{\mRLX}]}{below=of a2}
          \xform{xii}{\bForm[\FALSE/\QS{\bLoc}{\mREL}][\FALSE/\QS{\aLoc}{\mRLX}]}{below right=of a2}
          \xo{a2}{x1d}
          \xos[xleft]{a1}{x2i}
          \xos{a2}{x1i}
          \sync{a1}{a2}
        \end{tikzinline}}
  \end{gather*}
\end{example}

\begin{example}
  \label{ex:subscription}
  The definition enforces subscription.  Consider:
  \begin{align*}
    \begin{gathered}
      \PR[\mACQ]{y}{r}
      \\
      \hbox{\begin{tikzinline}[node distance=.5em and .5em]
          \raevent{a1}{\QL{\bLoc}{\mACQ}\mid\DR{y}{1}}{}
          \xform{x1d}{r{=}1\limplies\bForm}{below=of a1}
          \xform{x1i}{\bForm[\FALSE/\QL{\bLoc}{\mACQ}]}{right=of x1d}
          \xos[xleft]{a1}{x1d}
        \end{tikzinline}}
    \end{gathered}
    &&
    \begin{gathered}
      \PR{x}{s}
      \\
      \hbox{\begin{tikzinline}[node distance=.5em and .5em]
          \event{a2}{\QL{\aLoc}{\mRLX}\mid\DR{x}{1}}{}
          \xform{x2d}{s{=}1\limplies\bForm}{below=of a2}
          \xform{x2i}{\bForm[\FALSE/\QL{\aLoc}{\mRLX}]}{right=of x2d}
          \xos[xleft]{a2}{x2d}
        \end{tikzinline}}
    \end{gathered}
  \end{align*}
  Since $\QL{\aLoc}{\mRLX}[\FALSE/\QL{\bLoc}{\mACQ}]$ is $\FALSE$,
  composing these without order simplifies to:
  \begin{gather*}
    \PR[\mACQ]{y}{r}\SEMI \PR{x}{s}
    \\
    \hbox{\begin{tikzinline}[node distance=.5em and 1.5em]
          \raevent{a1}{\QL{\bLoc}{\mACQ}\mid\DR{y}{1}}{}
          \xform{x1d}{r{=}1\limplies\bForm[\FALSE/\QL{\aLoc}{\mRLX}]}{below=of a1}
          \xform{xdd}{r{=}1\limplies s{=}1\limplies\bForm}{right=of x1d}
          \xform{xii}{\bForm[\FALSE/\QL{\bLoc}{\mACQ}][\FALSE/\QL{\aLoc}{\mRLX}]}{above=of xdd}
          \xform{x2d}{s{=}1\limplies\bForm[\FALSE/\QL{\bLoc}{\mACQ}]}{right=of xdd}
          \event{a2}{\FALSE\mid\DR{x}{1}}{above=of x2d}
          \xos[xleft]{a1}{x1d}
          \xos{a2}{x2d}
          \xo{a1}{xdd}
          \xo{a2}{xdd}
        \end{tikzinline}}
  \end{gather*}
  In order to get a satisfiable precondition for $\DRP{x}{1}$, we must
  introduce order:
  \begin{gather*}
    % \PR[\mACQ]{y}{r}\SEMI \PR{x}{s}
    % \\
    \hbox{\begin{tikzinline}[node distance=.5em and 1.5em]
          \raevent{a1}{\QL{\bLoc}{\mACQ}\mid\DR{y}{1}}{}
          \xform{x1d}{r{=}1\limplies\bForm[\FALSE/\QL{\aLoc}{\mRLX}]}{below=of a1}
          \xform{xdd}{r{=}1\limplies s{=}1\limplies\bForm}{right=of x1d}
          \xform{xii}{\bForm[\FALSE/\QL{\bLoc}{\mACQ}][\FALSE/\QL{\aLoc}{\mRLX}]}{above=of xdd}
          \xform{x2d}{s{=}1\limplies\bForm[\FALSE/\QL{\bLoc}{\mACQ}]}{right=of xdd}
          \event{a2}{\QL{\aLoc}{\mRLX}\mid\DR{x}{1}}{above=of x2d}
          \xos[xleft]{a1}{x1d}
          \xos{a2}{x2d}
          \xo{a1}{xdd}
          \xo{a2}{xdd}
          \sync[out=15,in=165]{a1}{a2}
        \end{tikzinline}}
  \end{gather*}
\end{example}

\begin{example}
Even in its logical form, \ref{seq-le-delays'} is incompatible with the
ability to strengthen preconditions using augment closure, which is allowed
in \cite{DBLP:journals/pacmpl/JagadeesanJR20}.  Consider the following.
\begin{align*}
  \begin{gathered}[t]
    \IF{r}\THEN\PW{x}{2}\FI
    \\
    \hbox{\begin{tikzinline}[node distance=.5em and 1.5em]
        \event{a1}{r{\neq}0\mid\DW{x}{2}}{}
      \end{tikzinline}}    
  \end{gathered}
  &&
  \begin{gathered}[t]
    \PW{x}{1}
    \\
    \hbox{\begin{tikzinline}[node distance=.5em and 1.5em]
        \event{a2}{            \DW{x}{1}}{}
      \end{tikzinline}}    
  \end{gathered}
  &&
  \begin{gathered}[t]
    \PW{x}{2}
    \\
    \hbox{\begin{tikzinline}[node distance=.5em and 1.5em]
        \event{a3}{            \DW{x}{2}}{}
      \end{tikzinline}}    
  \end{gathered}
  &&
  \begin{gathered}[t]
    \IF{\BANG r}\THEN\PW{x}{1}\FI
    \\
    \hbox{\begin{tikzinline}[node distance=.5em and 1.5em]
        \event{a4}{r{=}0   \mid\DW{x}{1}}{}
      \end{tikzinline}}    
  \end{gathered}
\end{align*}
% \begin{align*}
%   \begin{gathered}[t]
%     \IF{r}\THEN\PW{x}{2}\FI
%     \SEQ
%     \PW{x}{1}
%     \SEQ
%     \PW{x}{2}
%     \SEQ
%     \IF{\BANG r}\THEN\PW{x}{1}\FI
%     \\
%     \hbox{\begin{tikzinline}[node distance=.5em and 1.5em]
%         \event{a1}{r{\neq}0\mid\DW{x}{2}}{}
%         \event{a2}{            \DW{x}{1}}{right=of a1}
%         \event{a3}{            \DW{x}{2}}{right=of a2}
%         \event{a4}{r{=}0   \mid\DW{x}{1}}{right=of a3}
%       \end{tikzinline}}    
%   \end{gathered}
% \end{align*}
Augmenting the middle preconditions and then using sequential composition, we have:
\begin{align*}
  \begin{gathered}[t]
    \IF{r}\THEN\PW{x}{2}\FI
    \\
    \hbox{\begin{tikzinline}[node distance=.5em and 1.5em]
        \event{a1}{r{\neq}0\mid\DW{x}{2}}{}
      \end{tikzinline}}    
  \end{gathered}
  &&
  \begin{gathered}[t]
    \PW{x}{1}
    \SEQ
    \PW{x}{2}
    \\
    \hbox{\begin{tikzinline}[node distance=.5em and 1.5em]
        \event{a2}{r{\neq}0\mid\DW{x}{1}}{}
        \event{a3}{r{=}0   \mid\DW{x}{2}}{right=of a1}
      \end{tikzinline}}    
  \end{gathered}
  &&
  \begin{gathered}[t]
    \IF{\BANG r}\THEN\PW{x}{1}\FI
    \\
    \hbox{\begin{tikzinline}[node distance=.5em and 1.5em]
        \event{a4}{r{=}0   \mid\DW{x}{1}}{}
      \end{tikzinline}}    
  \end{gathered}
\end{align*}
Note that \ref{seq-le-delays'} does not require any order between the two
writes of the middle pomset.
% \begin{align*}
%   \begin{gathered}[t]
%     \hbox{\begin{tikzinline}[node distance=.5em and 1.5em]
%         \event{a1}{r{\neq}0\mid\DW{x}{2}}{}
%         \event{a2}{r{=}0   \mid\DW{x}{1}}{right=of a1}
%         \event{a3}{r{\neq}0\mid\DW{x}{2}}{right=of a2}
%         \event{a4}{r{=}0   \mid\DW{x}{1}}{right=of a3}
%       \end{tikzinline}}    
%   \end{gathered}
% \end{align*}
Merging left and right, we have:
\begin{align*}
  \begin{gathered}[t]
    \IF{r}\THEN\PW{x}{2}\FI
    \SEQ
    \PW{x}{1}
    \SEQ
    \PW{x}{2}
    \SEQ
    \IF{\BANG r}\THEN\PW{x}{1}\FI
    \\
    \hbox{\begin{tikzinline}[node distance=.5em and 1.5em]
        \event{a1}{\DW{x}{2}}{}
        \event{a4}{\DW{x}{1}}{right=of a1}
        \wki{a1}{a4}
      \end{tikzinline}}    
  \end{gathered}
\end{align*}
As shown by the following single-threaded code, allowing this outcome would violate \drfsc{}.
\begin{align*}
  \begin{gathered}[t]
    \PW{y}{1}
    \SEQ
    \PR{y}{r}
    \SEQ
    \IF{r}\THEN\PW{x}{2}\FI
    \SEQ
    \PW{x}{1}
    \SEQ
    \PW{x}{2}
    \SEQ
    \IF{\BANG r}\THEN\PW{x}{1}\FI
    \\
    \hbox{\begin{tikzinline}[node distance=.5em and 1.5em]
        \event{a1}{\DW{x}{2}}{}
        \event{a4}{\DW{x}{1}}{right=of a1}
        \wki{a1}{a4}
        \event{b2}{\DR{y}{1}}{left=of a1}
        \event{b1}{\DW{y}{1}}{left=of b2}
        \rf{b1}{b2}
      \end{tikzinline}}    
  \end{gathered}
\end{align*}

It is for this reason that we use \emph{weakest} preconditions, rather than
preconditions.
% Note that as a result, we fail to validate the following
% refinement:
% \begin{math}
%   \aPSS_1
%   \not\supseteq
%   \xIFTHEN{\aForm}{\aPSS_1}{}.
% \end{math}
\end{example}





% \subsection{Post-Hoc Verification of Fulfillment for \PwTmcaTITLE{2}}
% \label{sec:post-hoc}

