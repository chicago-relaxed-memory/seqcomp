\section{Discussion}
\subsection{Closure properties}

The semantics is closed with respect to \emph{augments} and
\emph{downsets}.
Augments include more order and stronger formulae; in examples, we typically
consider pomsets that are augment-minimal.
Downsets include a subset of initial events, similar to \emph{prefixes} for
strings.
\begin{definition}
  \label{def:augment}
  $\aPS_2$ is an \emph{augment} of $\aPS_1$ if
  \begin{enumerate}
  \item $\aEvs_2=\aEvs_1$,
  \item $\labelingAct_2(\aEv)=\labelingAct_1(\aEv)$,
  \item $\labelingForm_2(\aEv)$ implies $\labelingForm_1(\aEv)$,
  \item $\aTr[2]{\bEvs}{\aEv}$ implies $\aTr[1]{\bEvs}{\aEv}$,
  \item if $\bEv\le_2\aEv$ then $\bEv\le_1\aEv$.
  \end{enumerate}
\end{definition}

\begin{definition}
  \label{def:downset}
  $\aPS_2$ is an \emph{downset} of $\aPS_1$ if
  \begin{enumerate}
  \item $\aEvs_2\subseteq\aEvs_1$,
  \item $(\forall \aEv\in\aEvs_2)$ $\labelingAct_2(\aEv)=\labelingAct_1(\aEv)$,
  \item $(\forall \aEv\in\aEvs_2)$ $\labelingForm_2(\aEv)=\labelingForm_1(\aEv)$,
  \item $(\forall \aEv\in\aEvs_2)$ $\aTr[2]{\bEvs}{\aEv}=\aTr[1]{\bEvs}{\aEv}$,
  \item $(\forall \bEv\in\aEvs_2)$ $(\forall \aEv\in\aEvs_2)$ $\bEv\le_2\aEv$ if and only if $\bEv\le_1\aEv$,
  \item $(\forall \bEv\in\aEvs_1)$ $(\forall \aEv\in\aEvs_2)$ if $\bEv\le_1\aEv$ then $\bEv\in\aEvs_2$.
  \end{enumerate}
\end{definition}

\begin{proposition}
  Suppose $\aPS_1\in\sem{\aCmd}$.
  \begin{enumerate}
  \item If $\aPS_2$ is an augment of $\aPS_1$ then $\aPS_2\in\sem{\aCmd}$.
  \item If $\aPS_2$ is a downset of $\aPS_1$ then $\aPS_2\in\sem{\aCmd}$.
  \end{enumerate}
\end{proposition}


\subsection{Comparison with Weakest Preconditions}

We compare traditional transformers to the dependent-case transformers of
\refdef{def:pomsets-lir}; thus we consider only totally ordered executions.
Because we only consider the dependent case, we drop the superscript $\aEvs$
on $\aTr{\aEvs}{}$ throughout this section.  We also assume that each
register appears at most once in a program, as we did throughout
\textsection\ref{sec:model}--\ref{sec:arm}.

Because of augment closure, we are not interested in isolating the
\emph{weakest} precondition.  Thus we think of transformers as Hoare triples.
In addition, all programs in our language are strongly normalizing, so we
need not distinguish strong and weak correctness.  In this setting, the Hoare
triple $\hoare{\aForm}{\aCmd}{\bForm}$ holds exactly when
$\aForm \limplies \fwp{\aCmd}{\bForm}$.

Hoare triples do not distinguish thread-local variables from shared
variables.  Thus, the assignment rule applies to all types of storage. The
rules can be written as follows:
\begin{align*}
  % \fwp{x\GETS \aExp}{\bForm} &= (\forall y)\; y{=}\aExp \limplies \bForm[y/x] &&\text{ where $y$ is fresh}
  % \\
  \fwp{\PW{\aLoc}{\aExp}}{\bForm} &= \bForm[\aExp/\aLoc]
  \\
  \fwp{\LET{\aReg}{\aExp}}{\bForm} &= \bForm[\aExp/\aReg]
  \\
  \fwp{\PR{\aLoc}{\aReg}}{\bForm} &= \aLoc{=}\aReg\limplies\bForm
\end{align*}
Here we have chosen an alternative formulation for the read rule, which is
equivalent the more traditional $\bForm[\aLoc/\aReg]$, as long as registers
occur at most once in a program.  In \refdef{def:pomsets-lir}, the
transformers for the dependent case are as follows:
\begin{align*}
  \trd{\PW{\aLoc}{\aExp}}{\bForm} &= \bForm[\aExp/\aLoc]
  \\
  \trd{\LET{\aReg}{\aExp}}{\bForm} &= \bForm[\aExp/\aReg]
  \\
  \trd{\PR{\aLoc}{\aReg}}{\bForm} &= \aVal{=}\aReg\limplies\bForm &&
  \textwhere \labelingAct(\aEv)=\DR{\aLoc}{\aVal}
\end{align*}
Only the read rule differs from the traditional one.


For programs where every register is bound and every read is fulfilled, our
dependent transformers are the same as the traditional ones.  In our
semantics, thus, we only consider totally-ordered executions where every read
could be fulfilled by prepending some writes.  For example, we ignore pomsets
of $\PW{x}{2}\SEMI\PR{x}{r}$ that read $1$ for $x$.

For example, let $\aCmd_i$ be defined:
% as follows.
\begin{align*}
  \aCmd_1&=\PR{x}{s}\SEMI\PW{x}{s{+}r}
  \\  
  \aCmd_2&=\PW{x}{t}\SEMI\aCmd_1
  \\  
  \aCmd_3&=\LET{t}{2}\SEMI\LET{r}{5}\SEMI\aCmd_2
\end{align*}
% \begin{itemize}
% \item
%   \begin{math}
%     \fwp{\LET{\aReg}{\aExp}}{\bForm} = \bForm[\aExp/\aReg]
%   \end{math}
% \item
%   \begin{math}
%     \fwp{\PR{\aLoc}{\aReg}\;\,}{\bForm} = %\bForm[x/r]
%     %     (\forall\bReg)
%     %     \aLoc{=}\bReg\limplies\bForm [\bReg/\aLoc]
%     \aLoc{=}\aReg\limplies\bForm
%   \end{math}
% \item
%   \begin{math}
%     \fwp{\PW{\aLoc}{\aExp}}{\bForm} = \bForm[\aExp/\aLoc]
%   \end{math}
% \end{itemize}
% General relation between Hoare triples and $\fwp{}{}$:
% \begin{itemize}
% \item $\hoare{\fwp{\aCmd}{\bForm}}{\aCmd}{\bForm}$,
% \item If $\hoare{\aForm}{\aCmd}{\bForm}$ and $\aCmd$ terminates when starting
%   in any state satisfying $\aForm$, then $\aForm \limplies \fwp{\aCmd}{\bForm}$.
% \end{itemize}
The following pomset appears in the semantics of $\aCmd_2$.  A pomset for
$\aCmd_3$ can be derived by substituting $[2/t,\allowbreak5/r]$.  A pomset
for $\aCmd_1$ can be derived by eliminating the initial write.
\begin{gather*}
  % \begin{gathered}[t]
  %   \PW{x}{3}
  %   \\
  %   \hbox{\begin{tikzinline}[node distance=.5em and 1.5em]
  %     \event{c}{\DW{x}{3}}{}
  %     \xform{xd}{\bForm}{below=of c}
  %     \xo{c}{xd}
  %   \end{tikzinline}}
  % \end{gathered}
  % \qquad\quad
  % \begin{gathered}[t]
  %   \PR{x}{s}\SEMI\PW{x}{s{+}r}
  %   \\
  %   \hbox{\begin{tikzinline}[node distance=.5em and 1.5em]
  %     \event{a}{\DR{x}{2}}{}
  %     \event{b}{2{=}s\limplies(s{+}r){=}7\mid\DW{x}{7}}{right=of a}%6.5em of a}
  %     \po{a}{b}
  %     \xform{xdd}{2{=}s \limplies \bForm[s{+}r/x]}{below right=.5em and -1em of a}
  %       %     \xform{xdd}{2{=}s \limplies \bForm[s{+}r/x]}{above=of a}
  %       %     \xform{xdi}{2{=}s \limplies \bForm[s{+}r/x]}{below=of a}
  %       %     \xform{xii}{(2{=}s\lor x{=}s)\limplies\bForm[s{+}r/x]}{above=of b}
  %       %     \xform{xid}{(2{=}s\lor x{=}s)\limplies\bForm[s{+}r/x]}{below=of b}
  %       %     \xo{a}{xdi}
  %       %     \xo{b}{xid}
  %     \xo{a}{xdd}
  %     \xo{b}{xdd}
  %   \end{tikzinline}}
  % \end{gathered}
  % \\[1ex]
  \begin{gathered}[t]
    % \LET{t}{2}\SEMI
    % \LET{r}{5}\SEMI
    \PW{x}{t}\SEMI
    \PR{x}{s}\SEMI\PW{x}{s{+}r}
    \\
    \hbox{\begin{tikzinline}[node distance=.5em and 1.5em]
        \event{a}{\DR{x}{2}}{}
        \event{b}{2{=}s\limplies(s{+}r){=}7\mid\DW{x}{7}}{right=of a}
        \event{c}{t{=}2\mid\DW{x}{2}}{left=of a}
        \xform{xdd}{2{=}s \limplies \bForm[s{+}r/x]}{below=of a}%below right=.5em and -1em of a}
        \xo{a}{xdd}
        \xo{b}{xdd}
        \xo{c}{xdd}
        \po{a}{b}
        \rf{c}{a}
      \end{tikzinline}}
  \end{gathered}
\end{gather*}
The predicate transformers are:
% \begin{align*}
%   \fwp{\aCmd_1}{\bForm} &= x{=}s\limplies\bForm[s{+}r/x] 
%   \\
%   \fwp{\aCmd_2}{\bForm} &= t\,{=}s\limplies\bForm[s{+}r/x] 
%   \\
%   \fwp{\aCmd_3}{\bForm} &= 2{=}s\limplies\bForm[s{+}5/x] 
%   \\
%   \trd{\aCmd_1}{\bForm} = \trd{\aCmd_2}{\bForm} &= 2{=}s\limplies\bForm[s{+}r/x] 
%   \\
%   \trd{\aCmd_3}{\bForm} &= 2{=}s\limplies\bForm[s{+}5/x] 
% \end{align*}
\begin{scope}
  \small
  \begin{align*}
    \fwp{\aCmd_1}{\bForm} &= x{=}s\limplies\bForm[s{+}r/x] 
    &
    \trd{\aCmd_1}{\bForm} &= 2{=}s\limplies\bForm[s{+}r/x] 
    \\
    \fwp{\aCmd_2}{\bForm} &= t\,{=}s\limplies\bForm[s{+}r/x] 
    &
    \trd{\aCmd_2}{\bForm} &= 2{=}s\limplies\bForm[s{+}r/x] 
    \\
    \fwp{\aCmd_3}{\bForm} &= 2{=}s\limplies\bForm[s{+}5/x] 
    &
    \trd{\aCmd_3}{\bForm} &= 2{=}s\limplies\bForm[s{+}5/x] 
  \end{align*}
\end{scope}

% % Let $\rho:\Reg\fun\Val$ and $\chi:\Loc\fun\Val$ be substitutions.
% Let $\aState$ and $\rho$ range over substitutions $(\Reg\cup\Loc)\fun\Val$.
% Treating substitutions as states, the big-step operational semantics of
% programs can be defined as a relation $\bigstep{\aState}{\aCmd}{\bState}$.
% % Ie, $\aForm\aState$ implies $\bForm\bState$.
% \begin{align}
%   \label{wp1}
%   \bigstep{[5/r,2/x]}{\aCmd_1&}{[5/r,2/s,7/x]}
%   \\
%   \label{wp2}
%   \bigstep{[\NEG5/r,2/x]}{\aCmd_1&}{[\NEG5/r,2/s,\NEG3/x]}
% \end{align}

% Then the semantics of Hoare triples guarantees that if
% $\aForm\limplies\fwp{\aCmd}{\bForm}$, $\bigstep{\aState}{\aCmd}{\rho}$ and
% $\aForm\aState$ is a tautology then $\bForm\bState$ is a tautology.
% \begin{align*}
%   \fwp{\aCmd_1}{x{>}0} &= (x{+}r{>}0) 
% \end{align*}
% In \eqref{wp1}, the pre- and post-conditions are satisfied.
% In \eqref{wp2}, they are not.


% \begin{itemize}  
% \item Suppose $\bigstep{\aState}{\aCmd}{\rho}$ and $\aForm\limplies\fwp{\aCmd}{\bForm}$.\\
%   If $\aForm\aState$ is a tautology then $\bForm\bState$ is a tautology.\\
%   Ie, $\aForm\aState$ implies $\bForm\bState$.
% \item Suppose $\bigstep{\aState}{\aCmd}{\rho}$ and $\hoare{\aForm}{\aCmd}{\bForm}$.\\
%   If $\aForm\aState$ is a tautology then $\bForm\bState$ is a tautology.\\
%   Ie, $\aForm\aState$ implies $\bForm\bState$.
% \item Suppose $\bigstep{\aState}{\aCmd}{\rho}$ and $\aForm=\fwp{\aCmd}{\bForm}$.\\
%   $\aForm\aState$ is a tautology if and only if $\bForm\bState$ is a tautology.\\
%   Ie, $\aForm\aState$ iff $\bForm\bState$.
% % \item Weakest: If $\aForm'\aState$ is a tautology, then $\aForm$ implies $\aForm'$.
% \end{itemize}
% Weakest preconditions are \emph{sound} in that if $\aForm$ holds in the
% initial state $\aState$, then $\bForm$ holds in the final state $\bState$.
% Formally, 


\begin{comment}
  If $\aPS\in\sem{\aCmd}$ is top-level and quiescent then 
  $\aTr{\aEvs}{\bForm}$ implies $\fwp{\aCmd}{\bForm}$.

  For any substitution $\aSub=[{v_1/r_1},\ldots, {v_n/r_n}]$ there is some
  $\aPS\in\sem{\aCmd}$ %that is top-level and quiescent
  such that all preconditions in $\aPS\aSub$ are tautologies then 
  $\fwp{\aCmd}{\bForm}\aSub$
\end{comment}


% For a language where all programs are
% terminating, we have for any statement $\aCmd$:
% \begin{align*}
%   \hoare{\aForm}{\aCmd}{\bForm} 
%   \;\;\Leftrightarrow\;\;
%   \aForm \textimplies \fwp{\aCmd}{\bForm}
% \end{align*}
% Interpretation is that if $\aState\models\fwp{\aCmd}{\bForm}$ and
% $\bigstep{\aState}{\aCmd}{\rho}$
% then $\bState\models\bForm$.

% Let $\aCmd_0$ be
% \begin{math}
%   \PW{\aLoc_1}{\aVal_1}\SEMI\cdots\SEMI \PW{\aLoc_n}{\aVal_n}, 
% \end{math}
% such that $\fwp{\aCmd_0}{\aForm}$ is a tautology, and $\aLoc_i=\aLoc_j$
% implies $i=j$.

% Let $\aSub_\aPS=[{\aVal_1/\aLoc_1},\ldots, {\aVal_n/\aLoc_n}]$ be the final
% state of $\aPS$.

% Let $\aState$ and $\rho$ range over substitutions $\Loc\fun\Exp$.
% If we leave the registers free, we have:
% \begin{align}
%   \label{wp1x}
%   \bigstep{[2/x]}{\aCmd&}{[6/x]}
% \end{align}

% Using \refdef{def:pomsets-trans}:
% \begin{align*}
%   \begin{gathered}[t]
%     %     \PR{x}{s}\SEMI\PW{x}{s{+}r}
%     \PR{x}{s}
%     \\
%     \hbox{\begin{tikzinline}[node distance=.5em and 1em]
%       \event{a}{\DR{x}{2}}{}
%       \xform{xi}{\bForm}{above=of a}
%       \xform{xd}{2{=}s \limplies \bForm}{below=of a}
%       \xo{a}{xd}
%     \end{tikzinline}}
%   \end{gathered}
%   &&
%   \begin{gathered}[t]
%     \PW{x}{s{+}r}
%     \\
%     \hbox{\begin{tikzinline}[node distance=.5em and 1em]
%       \event{a}{(s{+}r){=}7\mid\DW{x}{7}}{}
%       \xform{xi}{\bForm}{above=of a}
%       \xform{xd}{\bForm}{below=of a}
%       \xo{a}{xd}
%     \end{tikzinline}}
%   \end{gathered}
% \end{align*}
% Composing
% \begin{align*}
%   \begin{gathered}[t]
%     \PR{x}{s}\SEMI\PW{x}{s{+}r}
%     \\
%     \hbox{\begin{tikzinline}[node distance=.5em and 1em]
%       \event{a}{\DR{x}{2}}{}
%       \event{b}{(s{+}r){=}7\mid\DW{x}{7}}{right=of a}
%       \xform{xdd}{2{=}s \limplies \bForm}{above=of a}
%       \xform{xdi}{2{=}s \limplies \bForm}{below=of a}
%       \xform{xii}{\bForm}{above=of b}
%       \xform{xid}{\bForm}{below=of b}
%       \xo{a}{xdi}
%       \xo{b}{xid}
%       \xo{a}{xdd}
%       \xo{b}{xdd}
%     \end{tikzinline}}
%   \end{gathered}
% \end{align*}

% Using \refdef{def:pomsets-lir}:
% \begin{align*}
%   \begin{gathered}[t]
%     %     \PR{x}{s}\SEMI\PW{x}{s{+}r}
%     \PR{x}{s}
%     \\
%     \hbox{\begin{tikzinline}[node distance=.5em and 1em]
%       \event{a}{\DR{x}{2}}{}
%       \xform{xi}{(2{=}s\lor x{=}s)\limplies \bForm}{above=of a}
%       \xform{xd}{2{=}s \limplies \bForm}{below=of a}
%       \xo{a}{xd}
%     \end{tikzinline}}
%   \end{gathered}
%   &&
%   \begin{gathered}[t]
%     \PW{x}{s{+}r}
%     \\
%     \hbox{\begin{tikzinline}[node distance=.5em and 1em]
%       \event{a}{(s{+}r){=}7\mid\DW{x}{7}}{}
%       \xform{xi}{\bForm[s{+}r/x]}{above=of a}
%       \xform{xd}{\bForm[s{+}r/x]}{below=of a}
%       \xo{a}{xd}
%     \end{tikzinline}}
%   \end{gathered}
% \end{align*}
% Composing

% For example, let $\aCmd_1=\PR{x}{r}$ and $\aCmd_2=\PW{x}{r{+}1}$ and
% $\aCmd=\aCmd_1\SEMI \aCmd_2$.
% \begin{align*}
%   \fwp{\aCmd_2}{x{>}1}&=(r{+}1{>}1) = (r{>}0)
%   \\
%   \fwp{\aCmd_1}{r{>}0}=\fwp{\aCmd_0}{x{>}1}&=(x{>}0)
% \end{align*}
% Let $\aPS_i\in\sem{\aCmd_i}$.
% \begin{align*}
%   \aTr[2]{\aEvs_2}{x{>}1}&=(r{+}1{>}1) = (r{>}0)
%   \\
%   \aTr[0]{\aEvs_0}{x{>}1}&=(0{=}\aReg \limplies r{>}0)
%   \\
%   \aTr[0]{\aEvs_0}{x{>}1}&=(1{=}\aReg \limplies r{>}0)
%   \\
%   \aTr[0]{\aEvs_0}{x{>}1}&=(2{=}\aReg \limplies r{>}0)
% \end{align*}

% \begin{proposition}
%   If $\aPS\in\sem{\aCmd}$ is top-level and quiescent then 
%   $\aTr{\aEvs}{\aForm}$ implies $\fwp{\aCmd}{\aForm}$.

%   For any substitution $\aSub=[{\aVal_1/\aReg_1},\ldots, {\aVal_n/\aReg_n}]$ there is some
%   $\aPS\in\sem{\aCmd}$ %that is top-level and quiescent
%   such that all preconditions in $\aPS\aSub$ are tautologies then 
%   $\fwp{\aCmd}{\aForm}\aSub$
% \end{proposition}

% \subsection{Fences}
% \label{sec:fences}

\subsection{Completed Pomsets and Fork}
\label{sec:fork}

It is sometimes useful to distinguish \emph{terminated} or \emph{completed}
executions from partial executions.  For example in
\begin{math}
  \sem{\PW{x}{1}\SEMI\PW{y}{1}},
\end{math}
we expect completed executions to include two write actions.  Note that this
is different from being downset-maximal.
\begin{gather}
  \nonumber
  \PW{x}{0} \SEMI \PW{x}{1}
  \PAR
  \PR{x}{r}\SEMI\PR{x}{s}\SEMI\IF{s}\THEN\PW{y}{1}\FI
  \\
  \label{down1}
  \hbox{\begin{tikzinline}[node distance=0.5em and 1.5em]
      \event{a}{\DW{x}{0}}{}
      \event{b}{\DW{x}{1}}{right=of a}
      \event{c}{\DR{x}{1}}{right=3em of b}
      \event{d}{\DR{x}{0}}{right=of c}
      \wk{a}{b}
      \rf{b}{c}
      \rf[out=-20,in=-160]{a}{d}
    \end{tikzinline}}
  \\
  \label{down2}
  \hbox{\begin{tikzinline}[node distance=0.5em and 1.5em]
      \event{a}{\DW{x}{0}}{}
      \event{b}{\DW{x}{1}}{right=of a}
      \event{c}{\DR{x}{0}}{right=3em of b}
      \event{d}{\DR{x}{1}}{right=of c}
      \event{e}{\DW{y}{1}}{right=of d}
      \po{d}{e}
      \wk{a}{b}
      \rf[out=-20,in=-160]{a}{c}
      \rf[out=-20,in=-160]{b}{d}
    \end{tikzinline}}
\end{gather}
\eqref{down1} is a downset of \eqref{down2}, but both are completed. 

For pomsets with predicate transformers, we identify \emph{completion} with
\emph{quiescence.}
\begin{definition}
  \label{def:completed}
  A pomset with predicate transformers $\aPS$ is \emph{completed} if, for
  every quiescence symbol $\aSym$, $\aTr{\aEvs}{\aSym}$ implies $\aSym$.
\end{definition}
For example, there are no pomsets in $\sem{\ABORT}$ that are completed,
whereas the augment-minimal pomset of $\sem{\SKIP}$ is completed.

While this definition is sensible for single \emph{threads}, it is less
satisfying for thread \emph{groups}.  To see why, consider that in
$\sem{\FORK{\THREAD{\aCmd}}}$:
\begin{itemize}
\item by \ref{T3full}, quiescence symbols and the symbol $\RW$ have been
  substituted out of preconditions $\labelingForm(\aEv)$,
\item by \ref{F4}, every predicate transformer $\aTr{\bEvs}{}$ is the
  identity function. %, for any $\bEvs$.
\end{itemize}
Every pomset in $\sem{\FORK{\aGrp}}$ is completed, by definition. As a
result, in general, $\sem{\FORK{\THREAD{\aCmd}}}\neq\sem{\aCmd}$.

The $\FORK{}$ operation is asynchronous: In
$\sem{\aCmd_1\SEMI\FORK{\aGrp}\SEMI \aCmd_2}$, the threads in $\sem{\aGrp}$
run concurrently with $\sem{\aCmd_1\SEMI\aCmd_2}$.  % $\FORK{}$ %(\textsection\ref{sec:pomsets-trans})
% does not introduce barriers:
\begin{gather*}
  \PR{x}{r}\SEMI\FORK{\THREAD{\PW{x}{1}}}
  \\
  \hbox{\begin{tikzinline}[node distance=0.5em and 1.5em]
      \event{a}{\DR{x}{1}}{}
      \event{b}{\DW{x}{1}}{right=3em of a}
      \rf{b}{a}
    \end{tikzinline}}
\end{gather*}
In fact, perhaps surprisingly,
\begin{math}
  \sem{\PR{x}{r}\SEMI\FORK{\THREAD{\PW{x}{1}}}} = \sem{\FORK{\THREAD{\PW{x}{1}}}\SEMI\PR{x}{r}}.
\end{math}
Order between the threads
can be enforced using synchronization.  For example, the ``backwards'' read
above is forbidden in:
\begin{gather*}
  \PR{x}{r}\SEMI\PW[\mRA]{z}{1}\SEMI\FORK{\THREAD{\IF{\PR[\mRA]{z}{}}\THEN\PW{x}{1}\FI}}
  \\
  \hbox{\begin{tikzinline}[node distance=0.5em and 1.5em]
      \event{a}{\DR{x}{1}}{}
      \event{b}{\DW{z}{1}}{right=of a}
      \event{c}{\DR{z}{1}}{right=3em of b}
      \event{d}{\DW{x}{1}}{right=of c}
      \rf{b}{c}
      \sync{a}{b}
      \sync{c}{d}
      \rf[out=-160,in=-20]{d}{a}
    \end{tikzinline}}
\end{gather*}

\subsection{Fork-Join}
\label{sec:join}

In this subsection, we model a variant of our language that removes the
asynchronous $\FORK{}$ operation and adds a synchronous $\FORKJOIN{}$.
\begin{align*}
  \aCmd
  \BNFDEF& \ABORT
  \BNFSEP \SKIP
  \BNFSEP \LET{\aReg}{\aExp}
  % \BNFSEP \PR[\amode]{\aLoc}{\aReg}
  % \BNFSEP \PW[\amode]{\aLoc}{\aExp}
  \BNFSEP \PRREF[\amode]{\cExp}{\aReg}
  \BNFSEP \PWREF[\amode]{\cExp}{\aExp}
  % \BNFSEP \PA{\aLoc}{\aExp} 
  \\[-.5ex]
  \BNFSEP& \FORKJOIN{\aGrp}
  \BNFSEP \aCmd_1 \SEMI \aCmd_2
  \BNFSEP \IF{\aExp} \THEN \aCmd_1 \ELSE \aCmd_2 \FI
\end{align*}
In $\PBR{\aCmd_1\SEMI\FORKJOIN{\aGrp}\SEMI\aCmd_2}$, ${\aCmd_1}$ must
complete before ${\aGrp}$ begins, and threads in ${\aGrp}$ must complete
before ${\aCmd_2}$ begins.  Thus $\PBR{\FORKJOIN{\THREAD{\PR{x}{r}}}}$ acts
like a full fence.
% 
As modeled here, however, if $\aGrp$ is empty, no order is imposed between
${\aCmd_1}$ and ${\aCmd_2}$.  Thus
$\sem{\FORKJOIN{\THREAD{\SKIP}}}=\sem{\SKIP}$.

To model $\FORKJOIN{}$, we give the semantics of thread groups using pomsets
with preconditions \emph{and termination}.
\begin{definition}
  \label{def:pomsets-term}
  A \emph{pomset with preconditions and termination} is a pomset with
  preconditions (\refdef{def:pomsets-pre}) together with a termination
  predicate (notation $\TICK$).
\end{definition}

% The definition is a small change relative to that of
% \textsection\ref{sec:pomsets-trans}.

% Define $\sTHREAD{}$ to transform a pomset with predicate transformers into
% a pomset with preconditions and termination by dropping the predicate
% transformer and setting $\TICK$ to indicate whether the pomset was
% completed.

% Extend the definition of $\sNIL$ so that $\TICK$ is true.

% Extend the definition of $\sPAR$ to handle for $\TICK$ by adding the
% following.
% \begin{enumerate}
%   \setcounter{enumi}{\value{pomsetPreParCount}}
% \item \label{par-tick} if $\TICK$ then $\TICK_1$ and $\TICK_2$.
% \end{enumerate}

% Similarly, $\sFORKJOIN{}$ extends $\sFORK{}$ by adding the following.
% % \noindent
% % If $\aPS \in \sFORKJOIN{\aPSS}$ then
% % $(\exists\aPS_1\in\aPSS)$
% \begin{enumerate}
%   \setcounter{enumi}{\value{pomsetXForkCount}}
% \item $\TICK_1$.
% \end{enumerate}

\begin{definition}%$\phantom{\;}$\par
  \label{def:pomsets-fj}
  % \noindent
  % If $\aPS\in\sNIL$ then $\aEvs = \emptyset$ and $\TICK$.
  \noindent
  If $\aPS \in \sTHREAD{\aPSS}$ then $(\exists\aPS_1\in\aPSS)$
  \begin{enumerate}
  \item[1--2)] as for $\sTHREAD{}$ in \refdef{def:pomsets-arm},% \reffig{fig:full},    
  \item[\ref{T3full})]
    $\labelingForm(\aEv)$ implies
    $\labelingForm_1(\aEv) [\TRUE/\Q{}][\TRUE/\RW]$ if $\labelingAct_1(\aEv)$ is a write,
    \\
    $\labelingForm(\aEv)$ implies
    $\labelingForm_1(\aEv) [\TRUE/\Q{}][\FALSE/\RW]$ otherwise.
    
  \item[{\labeltext[T4]{T4)}{T4}}] if $\TICK$ then $\aPS$ is completed (\refdef{def:completed}).
    % $\aTr{\aEvs}{\Q{}}$ implies $\Q{}$.
  \end{enumerate}

  \noindent
  If $\aPS \in (\aPSS_1\sPAR\aPSS_2)$ then $(\exists\aPS_1\in\aPSS_1)$
  $(\exists\aPS_2\in\aPSS_2)$
  \begin{enumerate}
    \setcounter{enumi}{\value{pomsetPreParCount}}
  \item[\ref{par-E}--\ref{par-kappa2})] as for $\sPAR$ in
    \refdef{def:pomsets-pre},
  \item \label{par-tick} $\TICK$ implies $\TICK_1\land\TICK_2$.
  \end{enumerate}

  \noindent
  If $\aPS \in \sFORKJOIN{\aPSS}$ then $(\exists\aPS_1\in\aPSS)$
  \begin{enumerate}
  \item[1--2)] as for $\sFORK{}$ in \refdef{def:pomsets-group},
  \item[\ref{F3})]
    $\labelingForm(\aEv)$ implies $\Qr{*}\land\Qw{*} \land \Qsc\land \labelingForm_1(\aEv)$,     
  \item[\ref{F4})]
    $\aTr{\bEvs}{\bForm}$ implies $\bForm$, if $\bEvs=\aEvs$ and $\TICK_1$,
  \item[{\labeltext[F5]{F5)}{F5}}]
    $\aTr{\bEvs}{\bForm}$ implies $\bForm[\FALSE/\Q{}]$, otherwise.
  \end{enumerate}
\end{definition}
\begin{definition}
  Update \refdef{def:sem-funs} to include:
  \begin{align*}
    \sem{\FORKJOIN{\aGrp}} = \sFORKJOIN{}\sem{\aGrp}  
  \end{align*}
\end{definition}

We embed pomsets with predicate transformers into pomsets with preconditions
and termination using {completion}.  The rules for thread groups keep track
of the termination predicate.
As noted in \textsection\ref{sec:fork}, every pomset in $\sem{\FORK{\aGrp}}$ is
completed.  In contrast, a pomset in $\sem{\FORKJOIN{\aGrp}}$ is completed
only if every thread in $\aGrp$ is completed.

Top-level thread groups do not need quiescence symbols; thus, $\sTHREAD{}$
removes all quiescence symbols by substitution.  However, $\sFORKJOIN{\aPSS}$
adds every possible quiescence symbol as a precondition to the events of
$\aPSS$.  For example, the preconditions of $\sem{\THREAD{\aCmd}\PAR\NIL}$ do
not contain quiescence symbols.  Instead, the preconditions of
$\sem{\FORKJOIN{\THREAD{\aCmd}\PAR\NIL}}$ are saturated with them.  As a
result, in completed top-level pomsets of
$\sem{\aCmd_1\SEMI\FORKJOIN{\aGrp}}$, all of the events from
$\sem{\aCmd_1}$ must precede those of $\sem{\aGrp}$.

A similar thing happens with predicate transformers.  Thread groups in
$\sem{\THREAD{\aCmd}\PAR\NIL}$ do not contain predicate transformers.
Instead, all of the independent predicate transformers of
$\sem{\FORKJOIN{\THREAD{\aCmd}\PAR\NIL}}$ take $\bForm$ to
$\bForm[\FALSE/\Q{}]$.  As a result, in completed top-level pomsets of
$\sem{\FORKJOIN{\aGrp}\SEMI\aCmd_2}$, all of the events from $\sem{\aGrp}$
must precede those of $\sem{\aCmd_2}$.


% The $\JOIN$ operation requires a full synchronization, but $\FORK{}$ does
% not.  The following execution is allowed.
% \begin{gather*}
%   \PR{x}{r}\SEMI\FORKJOIN{\THREAD{\PW{x}{1}}}\SEMI\PW{y}{1}
%   \\
%   \hbox{\begin{tikzinline}[node distance=0.5em and 1.5em]      
%     \event{a}{\DW{x}{1}}{}
%     \event{b}{\DW{y}{1}}{right=3em of a}
%     \event{c}{\DR{x}{1}}{left=3em of a}
%     \rf{a}{c}
%     \sync{a}{b}
%     \sync[out=-20,in=-160]{c}{b}
%   \end{tikzinline}}
% \end{gather*}
% Synchronization can be added to 
% This asymmetry arises naturally when using pomsets with preconditions to
% model thread groups.

\subsection{Using Independency for Coherence}
\label{sec:independency-coherence}

\begin{figure*}[t]
  \begin{subfigure}{.5\textwidth}
    \centering
    \begin{align*}
  \begin{gathered}
    \begin{gathered}[t]
      \mathstrut\PW{x}{\aExp}
      \\
      \hbox{\begin{tikzinlinesmall}[node distance=.5em and 1.5em]
          \event{a}{\aExp{=}v\land\Qr{x}\land\Qw{x}\bigmid\DW{x}{v}}{}
          \xform{xi}{\bForm[\aExp/\aLoc][\FALSE/\Qw{\aLoc}]}{above=of a}
          \xform{xd}{\bForm[\aExp/\aLoc][(\aExp{=}\aVal\land \Qw{\aLoc})/\Qw{\aLoc}]}{below=of a}
          \xos{a}{xd}
        \end{tikzinlinesmall}}
    \end{gathered}
    \\[1ex]
    \begin{gathered}[t]
      \PW[\mREL]{x}{\aExp}
      \\
      \hbox{\begin{tikzinlinesmall}[node distance=.5em and 1.5em]
          \raevent{a}{\aExp{=}v\land\Qr{*}\land\Qw{*}\bigmid\DW[\mREL]{x}{v}}{}
          \xform{xi}{\bForm[\aExp/\aLoc][\FALSE/\Qw{x}]}{above=of a}
          \xform{xd}{\bForm[\aExp/\aLoc][(\aExp{=}\aVal\land \Qw{\aLoc})/\Qw{x}]}{below=of a}
          \xos{a}{xd}
        \end{tikzinlinesmall}}
    \end{gathered}
    \\[1ex]
    \begin{gathered}[t]
      \PW[\mSC]{x}{\aExp}
      \\
      \hbox{\begin{tikzinlinesmall}[node distance=.5em and 1.5em]
          \scevent{a}{\aExp{=}v\land\Qr{*}\land\Qw{*}\land\Qsc \bigmid\DW[\mSC]{x}{v}}{}
          \xform{xi}{\bForm[\aExp/\aLoc][\FALSE/\Qw{x}][\FALSE/\Qsc]}{above=of a}
          \xform{xd}{\bForm[\aExp/\aLoc][(\aExp{=}\aVal\land \Qw{\aLoc})/\Qw{x}][(\aExp{=}\aVal\land \Qsc)/\Qsc]}{below=of a}
          \xos{a}{xd}
        \end{tikzinlinesmall}}
    \end{gathered}
  \end{gathered}
  &&
  \mkern-60mu
  \begin{gathered}
    \begin{gathered}[t]
      \mathstrut\PR{x}{r}
      \\
      \hbox{\begin{tikzinlinesmall}[node distance=.5em and 1.5em]
          \event{a}{\Qw{x}\bigmid\DR{x}{v}}{}
          \xform{xi}{\PBR{\aVal{=}\aReg \lor \aLoc{=}\aReg} \limplies \bForm[\FALSE/\Qr{x}]}{above=of a}
          \xform{xd}{v{=}r\limplies\bForm}{below=of a}
          \xos{a}{xd}
        \end{tikzinlinesmall}}
    \end{gathered}
    \\[1ex]
    \begin{gathered}[t]
      \PR[\mACQ]{x}{r}
      \\
      \hbox{\begin{tikzinlinesmall}[node distance=.5em and 1.5em]
          \raevent{a}{\Qw{x}\bigmid\DR[\mACQ]{x}{v}}{}
          \xform{xi}{\PBR{\aVal{=}\aReg \lor \aLoc{=}\aReg} \limplies \bForm[\FALSE/\Qr{*}][\FALSE/\Qw{*}]}{above=of a}
          \xform{xd}{v{=}r\limplies\bForm}{below=of a}
          \xos{a}{xd}
        \end{tikzinlinesmall}}
    \end{gathered}
    \\[1ex]
    \begin{gathered}[t]
      \PR[\mSC]{x}{r}
      \\
      \hbox{\begin{tikzinlinesmall}[node distance=.5em and 1.5em]
          \scevent{a}{\Qw{x}\land\Qsc\bigmid\DR[\mSC]{x}{v}}{}
          \xform{xi}{\PBR{\aVal{=}\aReg \lor \aLoc{=}\aReg} \limplies \bForm[\FALSE/\Qr{*}][\FALSE/\Qw{*}][\FALSE/\Qsc]}{above=of a}
          \xform{xd}{v{=}r\limplies\bForm}{below=of a}
          \xos{a}{xd}
        \end{tikzinlinesmall}}
    \end{gathered}
  \end{gathered}
  &&
  \begin{gathered}
      \begin{gathered}[t]
        \mathstrut\smash{\PF{\fREL}}
        \\
        \hbox{\begin{tikzinlinesmall}[node distance=.5em and 1.5em]
            \raevent{a}{\Qr{*}\land\Qw{*}\bigmid\DF{\fREL}}{}
            \xform{xi}{\bForm[\FALSE/\Qw{*}]}{above=of a}
            \xform{xd}{\bForm}{below=of a}
            \xos[xright]{a}{xd}
          \end{tikzinlinesmall}}
      \end{gathered}      
    \\[1ex]
      \begin{gathered}[t]
        \PF{\fACQ}
        \\
        \hbox{\begin{tikzinlinesmall}[node distance=.5em and 1.5em]
            \raevent{a}{\Qr{*}\bigmid\DR[\fACQ]{x}{v}}{}
            \xform{xi}{\bForm[\FALSE/\Qr{*}][\FALSE/\Qw{*}]}{above=of a}
            \xform{xd}{\bForm}{below=of a}
            \xos[xright]{a}{xd}
          \end{tikzinlinesmall}}
      \end{gathered}
    \\[1ex]
    \begin{gathered}[t]
      \PF{\fSC}
      \\
      \hbox{\begin{tikzinlinesmall}[node distance=.5em and 1.5em]
          \scevent{a}{\Qr{*}\land\Qw{*}\land\Qsc\bigmid\DF{\mSC}}{}
          \xform{xi}{\bForm[\FALSE/\Qr{*}][\FALSE/\Qw{*}][\FALSE/\Qsc]}{above=of a}
          \xform{xd}{\bForm}{below=of a}
          \xos[xright]{a}{xd}
        \end{tikzinlinesmall}}
    \end{gathered}
  \end{gathered}
\end{align*}

    \caption{Quiescence Examples (\textsection\ref{sec:sync})}
    \label{fig:q1}
  \end{subfigure}  
  \begin{subfigure}{.5\textwidth}
    \centering
    \begin{align*}
  \begin{gathered}
    \begin{gathered}[t]
      \PW{x}{\aExp}
      \\
      \hbox{\begin{tikzinline}[node distance=.5em and 1.5em]
          \event{a}{\Qra \land\aExp{=}v\mid\DW{x}{v}}{}
          \xform{xi}{\bForm[\FALSE/\Qrlx]}{above=of a}
          \xform{xd}{\bForm\land \aExp{=}\aVal}{below=of a}
          %\xform{xd}{\bForm[(\Qrlx\land\aExp{=}\aVal)/\Qrlx]}{below=of a}
          \xo{a}{xd}
        \end{tikzinline}}
    \end{gathered}
    \\[1ex]
    \begin{gathered}[t]
      \PW[\mREL]{x}{\aExp}
      \\
      \hbox{\begin{tikzinline}[node distance=.5em and 1.5em]
          \raevent{a}{\Qra\land\Qrlx \land\aExp{=}v\mid\DW[\mREL]{x}{v}}{}
          \xform{xi}{\bForm[\FALSE/\Qrlx]}{above=of a}
          \xform{xd}{\bForm\land \aExp{=}\aVal}{below=of a}
          %\xform{xd}{\bForm[(\Qrlx\land\aExp{=}\aVal)/\Qrlx]}{below=of a}
          \xo{a}{xd}
        \end{tikzinline}}
    \end{gathered}
    \\[1ex]
    \begin{gathered}[t]
      \PW[\mSC]{x}{\aExp}
      \\
      \hbox{\begin{tikzinline}[node distance=.5em and 1.5em]
          \scevent{a}{\Qra\land\Qrlx\land\Qsc \land\aExp{=}v\mid\DW[\mSC]{x}{v}}{}
          \xform{xi}{\bForm[\FALSE/\Qrlx][\FALSE/\Qsc]}{above=of a}
          \xform{xd}{\bForm\land \aExp{=}\aVal}{below=of a}
          %\xform{xd}{\bForm[(\Qrlx\land\aExp{=}\aVal)/\Qrlx]}{below=of a}
          \xo{a}{xd}
        \end{tikzinline}}
    \end{gathered}
  \end{gathered}
  &&
  \begin{gathered}
    \begin{gathered}[t]
      \PR{x}{r}
      \\
      \hbox{\begin{tikzinline}[node distance=.5em and 1.5em]
          \event{a}{\Qra\mid\DR{x}{v}}{}
          \xform{xi}{\bForm[\FALSE/\Qrlx]}{above=of a}
          \xform{xd}{v{=}r\limplies\bForm}{below=of a}
          \xo{a}{xd}
        \end{tikzinline}}
    \end{gathered}
    \\[1ex]
    \begin{gathered}[t]
      \PR[\mACQ]{x}{r}
      \\
      \hbox{\begin{tikzinline}[node distance=.5em and 1.5em]
          \raevent{a}{\Qra\mid\DR[\mACQ]{x}{v}}{}
          \xform{xi}{\bForm[\FALSE/\Qrlx][\FALSE/\Qra]}{above=of a}
          \xform{xd}{v{=}r\limplies\bForm}{below=of a}
          \xo{a}{xd}
        \end{tikzinline}}
    \end{gathered}
    \\[1ex]
    \begin{gathered}[t]
      \PR[\mSC]{x}{r}
      \\
      \hbox{\begin{tikzinline}[node distance=.5em and 1.5em]
          \scevent{a}{\Qra\land\Qsc\mid\DR[\mSC]{x}{v}}{}
          \xform{xi}{\bForm[\FALSE/\Qrlx][\FALSE/\Qra][\FALSE/\Qsc]}{above=of a}
          \xform{xd}{v{=}r\limplies\bForm}{below=of a}
          \xo{a}{xd}
        \end{tikzinline}}
    \end{gathered}
  \end{gathered}
\end{align*}

    \caption{Quiescence Examples (\textsection\ref{sec:independency-coherence})}
    \label{fig:q2}
  \end{subfigure}
  \caption{Quiescence Examples}
  \label{fig:q}
\end{figure*}
% \begin{figure}[t!]
%   \centering
%   \begin{align*}
  \begin{gathered}
    \begin{gathered}[t]
      \mathstrut\PW{x}{\aExp}
      \\
      \hbox{\begin{tikzinlinesmall}[node distance=.5em and 1.5em]
          \event{a}{\aExp{=}v\land\Qr{x}\land\Qw{x}\bigmid\DW{x}{v}}{}
          \xform{xi}{\bForm[\aExp/\aLoc][\FALSE/\Qw{\aLoc}]}{above=of a}
          \xform{xd}{\bForm[\aExp/\aLoc][(\aExp{=}\aVal\land \Qw{\aLoc})/\Qw{\aLoc}]}{below=of a}
          \xos{a}{xd}
        \end{tikzinlinesmall}}
    \end{gathered}
    \\[1ex]
    \begin{gathered}[t]
      \PW[\mREL]{x}{\aExp}
      \\
      \hbox{\begin{tikzinlinesmall}[node distance=.5em and 1.5em]
          \raevent{a}{\aExp{=}v\land\Qr{*}\land\Qw{*}\bigmid\DW[\mREL]{x}{v}}{}
          \xform{xi}{\bForm[\aExp/\aLoc][\FALSE/\Qw{x}]}{above=of a}
          \xform{xd}{\bForm[\aExp/\aLoc][(\aExp{=}\aVal\land \Qw{\aLoc})/\Qw{x}]}{below=of a}
          \xos{a}{xd}
        \end{tikzinlinesmall}}
    \end{gathered}
    \\[1ex]
    \begin{gathered}[t]
      \PW[\mSC]{x}{\aExp}
      \\
      \hbox{\begin{tikzinlinesmall}[node distance=.5em and 1.5em]
          \scevent{a}{\aExp{=}v\land\Qr{*}\land\Qw{*}\land\Qsc \bigmid\DW[\mSC]{x}{v}}{}
          \xform{xi}{\bForm[\aExp/\aLoc][\FALSE/\Qw{x}][\FALSE/\Qsc]}{above=of a}
          \xform{xd}{\bForm[\aExp/\aLoc][(\aExp{=}\aVal\land \Qw{\aLoc})/\Qw{x}][(\aExp{=}\aVal\land \Qsc)/\Qsc]}{below=of a}
          \xos{a}{xd}
        \end{tikzinlinesmall}}
    \end{gathered}
  \end{gathered}
  &&
  \mkern-60mu
  \begin{gathered}
    \begin{gathered}[t]
      \mathstrut\PR{x}{r}
      \\
      \hbox{\begin{tikzinlinesmall}[node distance=.5em and 1.5em]
          \event{a}{\Qw{x}\bigmid\DR{x}{v}}{}
          \xform{xi}{\PBR{\aVal{=}\aReg \lor \aLoc{=}\aReg} \limplies \bForm[\FALSE/\Qr{x}]}{above=of a}
          \xform{xd}{v{=}r\limplies\bForm}{below=of a}
          \xos{a}{xd}
        \end{tikzinlinesmall}}
    \end{gathered}
    \\[1ex]
    \begin{gathered}[t]
      \PR[\mACQ]{x}{r}
      \\
      \hbox{\begin{tikzinlinesmall}[node distance=.5em and 1.5em]
          \raevent{a}{\Qw{x}\bigmid\DR[\mACQ]{x}{v}}{}
          \xform{xi}{\PBR{\aVal{=}\aReg \lor \aLoc{=}\aReg} \limplies \bForm[\FALSE/\Qr{*}][\FALSE/\Qw{*}]}{above=of a}
          \xform{xd}{v{=}r\limplies\bForm}{below=of a}
          \xos{a}{xd}
        \end{tikzinlinesmall}}
    \end{gathered}
    \\[1ex]
    \begin{gathered}[t]
      \PR[\mSC]{x}{r}
      \\
      \hbox{\begin{tikzinlinesmall}[node distance=.5em and 1.5em]
          \scevent{a}{\Qw{x}\land\Qsc\bigmid\DR[\mSC]{x}{v}}{}
          \xform{xi}{\PBR{\aVal{=}\aReg \lor \aLoc{=}\aReg} \limplies \bForm[\FALSE/\Qr{*}][\FALSE/\Qw{*}][\FALSE/\Qsc]}{above=of a}
          \xform{xd}{v{=}r\limplies\bForm}{below=of a}
          \xos{a}{xd}
        \end{tikzinlinesmall}}
    \end{gathered}
  \end{gathered}
  &&
  \begin{gathered}
      \begin{gathered}[t]
        \mathstrut\smash{\PF{\fREL}}
        \\
        \hbox{\begin{tikzinlinesmall}[node distance=.5em and 1.5em]
            \raevent{a}{\Qr{*}\land\Qw{*}\bigmid\DF{\fREL}}{}
            \xform{xi}{\bForm[\FALSE/\Qw{*}]}{above=of a}
            \xform{xd}{\bForm}{below=of a}
            \xos[xright]{a}{xd}
          \end{tikzinlinesmall}}
      \end{gathered}      
    \\[1ex]
      \begin{gathered}[t]
        \PF{\fACQ}
        \\
        \hbox{\begin{tikzinlinesmall}[node distance=.5em and 1.5em]
            \raevent{a}{\Qr{*}\bigmid\DR[\fACQ]{x}{v}}{}
            \xform{xi}{\bForm[\FALSE/\Qr{*}][\FALSE/\Qw{*}]}{above=of a}
            \xform{xd}{\bForm}{below=of a}
            \xos[xright]{a}{xd}
          \end{tikzinlinesmall}}
      \end{gathered}
    \\[1ex]
    \begin{gathered}[t]
      \PF{\fSC}
      \\
      \hbox{\begin{tikzinlinesmall}[node distance=.5em and 1.5em]
          \scevent{a}{\Qr{*}\land\Qw{*}\land\Qsc\bigmid\DF{\mSC}}{}
          \xform{xi}{\bForm[\FALSE/\Qr{*}][\FALSE/\Qw{*}][\FALSE/\Qsc]}{above=of a}
          \xform{xd}{\bForm}{below=of a}
          \xos[xright]{a}{xd}
        \end{tikzinlinesmall}}
    \end{gathered}
  \end{gathered}
\end{align*}

%   \caption{Quiescence Examples (Old)}
%   \label{fig:q1}
% \end{figure}
% \begin{figure}[t!]
%   \centering
%   \begin{align*}
  \begin{gathered}
    \begin{gathered}[t]
      \PW{x}{\aExp}
      \\
      \hbox{\begin{tikzinline}[node distance=.5em and 1.5em]
          \event{a}{\Qra \land\aExp{=}v\mid\DW{x}{v}}{}
          \xform{xi}{\bForm[\FALSE/\Qrlx]}{above=of a}
          \xform{xd}{\bForm\land \aExp{=}\aVal}{below=of a}
          %\xform{xd}{\bForm[(\Qrlx\land\aExp{=}\aVal)/\Qrlx]}{below=of a}
          \xo{a}{xd}
        \end{tikzinline}}
    \end{gathered}
    \\[1ex]
    \begin{gathered}[t]
      \PW[\mREL]{x}{\aExp}
      \\
      \hbox{\begin{tikzinline}[node distance=.5em and 1.5em]
          \raevent{a}{\Qra\land\Qrlx \land\aExp{=}v\mid\DW[\mREL]{x}{v}}{}
          \xform{xi}{\bForm[\FALSE/\Qrlx]}{above=of a}
          \xform{xd}{\bForm\land \aExp{=}\aVal}{below=of a}
          %\xform{xd}{\bForm[(\Qrlx\land\aExp{=}\aVal)/\Qrlx]}{below=of a}
          \xo{a}{xd}
        \end{tikzinline}}
    \end{gathered}
    \\[1ex]
    \begin{gathered}[t]
      \PW[\mSC]{x}{\aExp}
      \\
      \hbox{\begin{tikzinline}[node distance=.5em and 1.5em]
          \scevent{a}{\Qra\land\Qrlx\land\Qsc \land\aExp{=}v\mid\DW[\mSC]{x}{v}}{}
          \xform{xi}{\bForm[\FALSE/\Qrlx][\FALSE/\Qsc]}{above=of a}
          \xform{xd}{\bForm\land \aExp{=}\aVal}{below=of a}
          %\xform{xd}{\bForm[(\Qrlx\land\aExp{=}\aVal)/\Qrlx]}{below=of a}
          \xo{a}{xd}
        \end{tikzinline}}
    \end{gathered}
  \end{gathered}
  &&
  \begin{gathered}
    \begin{gathered}[t]
      \PR{x}{r}
      \\
      \hbox{\begin{tikzinline}[node distance=.5em and 1.5em]
          \event{a}{\Qra\mid\DR{x}{v}}{}
          \xform{xi}{\bForm[\FALSE/\Qrlx]}{above=of a}
          \xform{xd}{v{=}r\limplies\bForm}{below=of a}
          \xo{a}{xd}
        \end{tikzinline}}
    \end{gathered}
    \\[1ex]
    \begin{gathered}[t]
      \PR[\mACQ]{x}{r}
      \\
      \hbox{\begin{tikzinline}[node distance=.5em and 1.5em]
          \raevent{a}{\Qra\mid\DR[\mACQ]{x}{v}}{}
          \xform{xi}{\bForm[\FALSE/\Qrlx][\FALSE/\Qra]}{above=of a}
          \xform{xd}{v{=}r\limplies\bForm}{below=of a}
          \xo{a}{xd}
        \end{tikzinline}}
    \end{gathered}
    \\[1ex]
    \begin{gathered}[t]
      \PR[\mSC]{x}{r}
      \\
      \hbox{\begin{tikzinline}[node distance=.5em and 1.5em]
          \scevent{a}{\Qra\land\Qsc\mid\DR[\mSC]{x}{v}}{}
          \xform{xi}{\bForm[\FALSE/\Qrlx][\FALSE/\Qra][\FALSE/\Qsc]}{above=of a}
          \xform{xd}{v{=}r\limplies\bForm}{below=of a}
          \xo{a}{xd}
        \end{tikzinline}}
    \end{gathered}
  \end{gathered}
\end{align*}

%   \caption{Quiescence Examples (New)}
%   \label{fig:q2}
% \end{figure}

In \textsection\ref{sec:sync}, we encoded coherence and synchronized access
using quiescence symbols.  Building on the language with $\FORKJOIN{}$, it is
possible to model coherence using independency
(\textsection\ref{sec:pomsets}), rather than encoding it in the logic.
\begin{definition}
  \label{def:independency-coherence}
  \noindent
  If $\aPS \in (\aPSS_1\sSEMI\aPSS_2)$ then $(\exists\aPS_1\in\aPSS_1)$
  $(\exists\aPS_2\in\aPSS_2)$
  % there are $\aPS_1\in\aPSS_1$ and $\aPS_2\in\aPSS_2$ such that
  % let
  % $\labelingForm'_2(\aEv)=\aTr[1]{{\downclose[0]{\aEv}}}{\labelingForm_2(\aEv)}$
  % let
  % $\labelingForm'_2(\aEv)=\aTr[1]{\{ \bEv \mid \bEv < \aEv
  % \}}{\labelingForm_2(\aEv})$
  \begin{enumerate}
    \setcounter{enumi}{\value{pomsetXSemiCount}}
  \item[1--\ref{seq-kappa12})] as for $\sSEMI$ in \refdef{def:pomsets-trans},
  \item
    \label{seq-reorder} if $\bEv\in\aEvs_1$ and $\aEv\in\aEvs_2$ either
    $\bEv<\aEv$ or $a\reorder\labeling_2(\aEv)$.
  \end{enumerate}
\end{definition}

In the logic, we remove the symbols $\Qw{\aLoc}$ and $\Qr{\aLoc}$.
Previously, we had given the semantics of $\mRA$ access using $\Qw{*}$ and
$\Qr{*}$, which were encoded using $\Qw{\aLoc}$ and $\Qr{\aLoc}$.  With these
gone, we introduce the quiescence symbol $\Qrlx$ and $\Qra$.  Thus, the only
quiescence symbols required are $\Qrlx$, $\Qra$ and $\Qsc$.
\reffig{fig:q} shows the difference with the semantics of \textsection\ref{sec:sync}.
% \reffig{fig:q1}.  In the new
% interpretation, see \reffig{fig:q2}.

% To see how the quiescence symbols are used, consider the following examples:

\begin{comment}
  Let formulae $\QS{\aLoc}{\amode}$ and $\QL{\aLoc}{\amode}$ be defined:
  \begin{align*}
    \QS{\aLoc}{\mRLX}&=\Qr{\aLoc}\land\Qw{\aLoc}
    &\QL{\aLoc}{\mRLX}&=\Qw{\aLoc}
    \\
    \QS{\aLoc}{\mRA}&=
    \Qr{*}\land\Qw{*} %\textstyle\bigwedge_\bLoc \Qr{\bLoc}\land\Qw{\bLoc}
    &\QL{\aLoc}{\mRA}&=\Qw{\aLoc}
    \\
    \QS{\aLoc}{\mSC}&=
    \Qr{*}\land\Qw{*} %\textstyle\bigwedge_\bLoc \Qr{\bLoc}\land\Qw{\bLoc}
    \land \Qsc
    &\QL{\aLoc}{\mSC}&=\Qw{\aLoc}\land\Qsc
  \end{align*}
  Let substitutions $[\aForm/\QS{\aLoc}{\amode}]$ and  $[\aForm/\QL{\aLoc}{\amode}]$ be defined:
  \begin{align*}
    [\aForm/\QS{\aLoc}{\mRLX}] &= [\aForm/\Qw{\aLoc}]
    &{} [\aForm/\QL{\aLoc}{\mRLX}] &= [\aForm/\Qr{\aLoc}]
    \\
    [\aForm/\QS{\aLoc}{\mRA}] &= [\aForm/\Qw{\aLoc}]
    &{} [\aForm/\QL{\aLoc}{\mRA}] &= [\aForm/\Qr{*},\aForm/\Qw{*}]
    \\
    [\aForm/\QS{\aLoc}{\mSC}] &= [\aForm/\Qw{\aLoc},\aForm/\Qsc]
    &{} [\aForm/\QL{\aLoc}{\mSC}] &= [\aForm/\Qr{*},\aForm/\Qw{*},\aForm/\Qsc]
  \end{align*}
  Update \refdef{def:pomsets-trans} from: % (\ref{S4}/\ref{L4} unchanged):
  \begin{enumerate}
  \item[\ref{S3})]
    $\labelingForm(\aEv)$ implies $\aExp{=}\aVal\land\QS{\aLoc}{\amode}$,
  \item[\ref{L3})]
    $\labelingForm(\aEv)$ implies $\QL{\aLoc}{\amode}$,
  \item[\ref{T3})]
    $\labelingForm(\aEv)$ implies $\labelingForm_1(\aEv)[\TRUE/\Qr{*}][\TRUE/\Qw{*}][\TRUE/\Qsc]$,
  \end{enumerate}
  \begin{enumerate}
  \item[\ref{S4})]
    $\aTr{\bEvs}{\bForm}$ implies $\bForm[(\Qw{\aLoc}\land\aExp{=}\aVal)/\Qw{\aLoc}]$,
  \item[\ref{S5})]
    $\aTr{\cEvs}{\bForm}$ implies $\bForm[\FALSE/\QS{\aLoc}{\amode}]$,
  \item[\ref{L4})]
    $\aTr{\bEvs}{\bForm}$ implies $\aVal{=}\aReg\limplies\bForm$, 
  \item[\ref{L5})]
    $\aTr{\cEvs}{\bForm}$ implies $\bForm[\FALSE/\QL{\aLoc}{\amode}]$.
  \end{enumerate}
\end{comment}

\begin{definition}
  Let formulae $\QS{}{\amode}$ and $\QL{}{\amode}$ be defined:
  \begin{align*}
    \QS{}{\mRLX}&=\Qra
    &\QL{}{\mRLX}&=\Qra
    \\
    \QS{}{\mRA}&=\Qra\land\Qrlx
    &\QL{}{\mRA}&=\Qra
    \\
    \QS{}{\mSC}&=\Qra\land \Qrlx \land \Qsc
    &\QL{}{\mSC}&=\Qra\land\Qsc
  \end{align*}
  Let substitutions $[\aForm/\QS{}{\amode}]$ and  $[\aForm/\QL{}{\amode}]$ be defined:
  \begin{align*}
    [\aForm/\QS{}{\mRLX}] &= [\aForm/\Qrlx]
    &{} [\aForm/\QL{}{\mRLX}] &= [\aForm/\Qrlx]
    \\
    [\aForm/\QS{}{\mRA}] &= [\aForm/\Qrlx]
    &{} [\aForm/\QL{}{\mRA}] &= [\aForm/\Qrlx,\aForm/\Qra]
    \\
    [\aForm/\QS{}{\mSC}] &= [\aForm/\Qrlx,\aForm/\Qsc]
    &{} [\aForm/\QL{}{\mSC}] &= [\aForm/\Qrlx,\aForm/\Qra,\aForm/\Qsc]
  \end{align*}
\end{definition}
\begin{definition}%[\xCO/\xRASC]
  Update \refdef{def:pomsets-trans} and \ref{def:pomsets-fj} to: % (\ref{S4}/\ref{L4} unchanged):
  \begin{enumerate}
  \item[\ref{S3})]
    $\labelingForm(\aEv)$ implies $\aExp{=}\aVal\land\QS{}{\amode}$,
  \item[\ref{L3})]
    $\labelingForm(\aEv)$ implies $\QL{}{\amode}$,
  \item[\ref{F3})]
    $\labelingForm(\aEv)$ implies $\Qrlx\land\Qra\land\Qsc\land\labelingForm_1(\aEv)$, 
  % \item[\ref{T3})]
  %   $\labelingForm(\aEv)$ implies $\labelingForm_1(\aEv)[\TRUE/\Q{}]$, %[\TRUE/\Qrlx][\TRUE/\Qra][\TRUE/\Qsc]$,
  \end{enumerate}
  \begin{enumerate}
  \item[\ref{S4})]
    $\aTr{\bEvs}{\bForm}$ implies $\bForm[(\Qrlx\land\aExp{=}\aVal)/\Qrlx]$,
  \item[\ref{S5})]
    $\aTr{\cEvs}{\bForm}$ implies $\bForm[\FALSE/\QS{}{\amode}]$,
  \item[\ref{L4})]
    $\aTr{\bEvs}{\bForm}$ implies $\aVal{=}\aReg\limplies\bForm$, 
  \item[\ref{L5})]
    $\aTr{\cEvs}{\bForm}$ implies $\bForm[\FALSE/\QL{}{\amode}]$.
  \end{enumerate}
\end{definition}

The most interesting examples in \reffig{fig:q2} concern $\mRA$ access.
Every independent transformer substitutes $[\FALSE/\Qrlx]$.  $\Qrlx$ is a
precondition for any releasing write $\aEv$, ensuring that all preceding
events must are ordered before $\aEv$.  Conversely, $\Qra$ is a precondition
of every event.  The independent transformer for any acquiring read $\aEv$
substitutes $[\FALSE/\Qra]$, ensuring that all following events must be
ordered after $\aEv$.

As before, the substitution in \ref{S4} ensures that left merges are not
quiescent (\refex{ex:left-merge}).

Item \ref{seq-reorder} of \refdef{def:independency-coherence} ensures
coherence.  This definition is incompatible with asynchronous $\FORK{}$
parallelism of \refdef{def:pomsets-group}, where we expect executions such
as:
\begin{gather*}
  \FORK{\THREAD{\PR{x}{r}}}\SEMI \PW{x}{1}
  \\
  \hbox{\begin{tikzinline}[node distance=0.5em and 1.5em]
      \event{a}{\DR{x}{1}}{}
      \event{b}{\DW{x}{1}}{right=3em of a}
      \rf{b}{a}
    \end{tikzinline}}
\end{gather*}
Item \ref{seq-reorder} would require $\DRP{x}{1} \xwk \DWP{x}{1}$, forbidding
this.

% , since \ref{T3full} substitutes
% $\TRUE$ for every quiescence symbol.  Preconditions of augment-minimal
% pomsets in $\sem{\FORK{\THREAD{\aCmd}}}$ contain no quiescence symbols.
% Instead, preconditions of augment-minimal pomsets in
% $\sem{\FORKJOIN{\THREAD{\aCmd}}}$ are saturated with quiescence symbols.

% One must be careful, however, due to \emph{inconsistency}.  Consider that
% \texttt{x=0;x=1} should not have completed pomset with only $\DWP{x}{0}$.

% \eqref{seq-reorder} does not do the right thing with fork either.  If you
% want to enforce coherence this way then you need to use fork-join as the
% sequential combinator, rather than fork.


% [We drop $\reorder$ because incompatible with $\sFORK{}$.  If you want to
% use $\reorder$, then you need to use fork-join as the sequential
% combinator, rather than fork.]

% We can then encode coherence as follows.
% \begin{enumerate}
%   \setcounter{enumi}{\value{pomsetXSemiCount}}
% \item if $\bEv\in\aEvs_1$ and $\aEv\in\aEvs_2$ either $\bEv<\aEv$ or
%   $a\reorder\labeling_2(\aEv)$.
% \end{enumerate}


% Access modes can be encoded in the independency relation, indexing labels by
% $\amode$, but the extra flexibility of the logic is necessary for \armeight{}
% (see \textsection\ref{sec:internal}).  Using independency, one would also
% need another way to define completed pomsets.  Finally, this use of
% independency is incompatible with fork (see \textsection\ref{sec:co}).


% If we move coherence to independency (and use fork-join), we have the
% following, assuming that each register occurs at most once.
% \begin{align*}
%   \QS{}{\mSC}&=\Q{\mSC}
%   &\QS{}{\mRA}&=\Q{\mRA}
%   &\QS{}{\mRLX}&=\Qx{\aLoc}
%   \\
%   \QL{}{\mSC}&=\Q{\mSC}
%   &\QL{}{\mRA}&=\Qw{\aLoc}
%   &\QL{}{\mRLX}&=\Qw{\aLoc}
%   \\
%   \DS{\aLoc}{\mSC}{\bForm}&=\bForm[\FALSE/\D]
%   &\DS{\aLoc}{\mRA}{\bForm}&=\bForm[\FALSE/\D]
%   &\DS{\aLoc}{\mRLX}{\bForm}&=\bForm[\TRUE/\Dx{\aLoc}] 
%   \\
%   \DL{\aLoc}{\mSC}&=\Dx{\aLoc}
%   &\DL{\aLoc}{\mRA}&=\Dx{\aLoc}
%   &\DL{\aLoc}{\mRLX}&=\TRUE
% \end{align*}

% % $\QS{}{\mRLX}=\TRUE$ and otherwise $\QS{}{\amode}=\Q{\amode}$.

% % $\QL{}{\mSC}=\Q{\mSC}$ and otherwise $\QL{}{\amode}=\TRUE$.

% % $\DS{\aLoc}{\mRLX}{\bForm}=\bForm[\TRUE/\Dx{\aLoc}]$ and otherwise
% % $\DS{\aLoc}{\amode}{\bForm}=\bForm[\FALSE/\D]$.

% % $\DL{\aLoc}{\mRLX}=\TRUE$ and otherwise $\DL{\aLoc}{\amode}=\Dx{\aLoc}$.

% % \begin{definition}$\phantom{\;}$\par
% %   $\QS{}{\mRLX}=\TRUE$ and otherwise $\QS{}{\amode}=\Q{\amode}$.

% %   $\QL{}{\mSC}=\Q{\mSC}$ and otherwise $\QL{}{\amode}=\TRUE$.

% \noindent
% \begin{enumerate}
% \item[\ref{S3})] $\labelingForm(\aEv)$ implies
%   \begin{math}
%     \aExp{=}\aVal \land \RW \land \QS{}{\amode}
%   \end{math},
% \item[\ref{S4})] $\aTr{\bEvs}{\bForm}$ implies
%   \begin{math}
%     \aExp{=}\aVal \land \DS{\aLoc}{\amode}{\bForm[\aExp/{\aLoc}]}
%   \end{math},
% \item[\ref{S5})] $\aTr{\emptyset}{\bForm}$ implies
%   \begin{math}
%     \lnot\Q{\mRA} \land \DS{\aLoc}{\amode}{\bForm[\aExp/{\aLoc}]}
%   \end{math}
% \end{enumerate}

% \noindent
% \begin{enumerate}
% \item[\ref{L3})] $\labelingForm(\aEv)$ implies
%   \begin{math}
%     \RO \land \QL{}{\amode}
%   \end{math},
% \item[\ref{L4})] $\aTr{\bEvs}{\bForm}$ implies
%   \begin{math}
%     (\aVal{=}\aReg) \limplies \bForm[\aReg/{\aLoc}]
%   \end{math}
% \item[\ref{L5})] $\aTr{\emptyset}{\bForm}$ implies
%   \begin{math}
%     \DL{\aLoc}{\amode} \land \lnot\Q{\mRA} \land (\RW \limplies
%     (\aVal{=}\aReg\lor\aLoc{=}\aReg) \limplies \bForm[\aReg/{\aLoc}] ).
%   \end{math}
% \end{enumerate}

\subsection{Substitutions}
\label{sec:substitutions}

Recall the load rules from \textsection\ref{sec:tc1}: % (\refdef{def:pomsets-lir}):
\begin{enumerate}
\item[\ref{L4})]
  $\aTr{\bEvs}{\bForm}$ implies $\aVal{=}\aReg\limplies\bForm$, 
\item[\ref{L5})]
  $\aTr{\cEvs}{\bForm}$ implies
  $(\aVal{=}\aReg\lor\aLoc{=}\aReg)\limplies\bForm$, when $\aEvs\neq\emptyset$,
\item[\ref{L6})] 
  $\aTr{\dEvs}{\bForm}\;$ implies $\bForm$, when $\aEvs=\emptyset$.
\end{enumerate}
It is also possible to collapse $\aLoc$ and $\aReg$ when doing a load:
\begin{enumerate}
\item[\ref{L4})]
  $\aTr{\bEvs}{\bForm}$ implies $\aVal{=}\aReg\limplies\bForm[\aReg/\aLoc]$, 
\item[\ref{L5})]
  $\aTr{\cEvs}{\bForm}$ implies
  $(\aVal{=}\aReg\lor\aLoc{=}\aReg)\limplies\bForm[\aReg/\aLoc]$, when $\aEvs\neq\emptyset$.
\item[\ref{L6})] 
  $\aTr{\dEvs}{\bForm}\;$ implies $\bForm[\aReg/\aLoc]$, when $\aEvs=\emptyset$.
\end{enumerate}

Perhaps surprisingly, these two semantics are incomparable.  Consider the
following:
\begin{gather*}
  \IF{r\land s\;\mathsf{even}}\THEN \PW{y}{1}\FI\SEMI
  \IF{r\land s}\THEN \PW{z}{1}\FI
  \\
  \hbox{\begin{tikzinline}[node distance=0.5em and 1.5em]
      \event{a3}{r\land s\;\mathsf{even}\mid\DW{y}{1}}{}
      \event{a4}{r\land s\mid\DW{z}{1}}{below=of a3}
    \end{tikzinline}}
\end{gather*}
Prepending $\PRP{x}{s}$, we get the same result regardless of whether we
substitute $[s/x]$, since $x$ does not occur in either precondition.  Here
we show the independent case:
\begin{gather*}
  \PR{x}{s}\SEMI
  \IF{r\land s\;\mathsf{even}}\THEN \PW{y}{1}\FI\SEMI
  \IF{r\land s}\THEN \PW{z}{1}\FI
  \\
  \hbox{\begin{tikzinline}[node distance=0.5em and 1.5em]
      \event{a2}{\DR{x}{2}}{}
      \event{a3}{(2{=}s\lor x{=}s)\limplies (r\land s\;\mathsf{even})\mid\DW{y}{1}}{above right=of a2}
      \event{a4}{(2{=}s\lor x{=}s)\limplies (r\land s)\mid\DW{z}{1}}{below=of a3}
    \end{tikzinline}}
\end{gather*}
Prepending $\PRP{x}{r}$, we now get different results since the
preconditions mention $x$.
Without substitution:
\begin{gather*}
  \PR{x}{r}\SEMI
  \PR{x}{s}\SEMI
  \IF{r\land s\;\mathsf{even}}\THEN \PW{y}{1}\FI\SEMI
  \IF{r\land s}\THEN \PW{z}{1}\FI
  \\
  \hbox{\begin{tikzinline}[node distance=0.5em and 1.5em]
      \event{a1}{\DR{x}{1}}{}
      \event{a2}{\DR{x}{2}}{below=of a1}
      \event{a3}{1{=}r\limplies  (2{=}s\lor x{=}s)\limplies (r\land s\;\mathsf{even})\mid\DW{y}{1}}{right=of a1}
      \event{a4}{1{=}r\limplies  (2{=}s\lor x{=}s)\limplies (r\land s)\mid\DW{z}{1}}{below=of a3}
      \po{a1}{a3}
      \po[out=-20,in=177]{a1}{a4}
    \end{tikzinline}}
\end{gather*}
Prepending $\PWP{x}{0}$, which substitutes $[0/x]$, the precondition of
$\DWP{y}{1}$ becomes
$(1{=}r\limplies (2{=}s\lor0{=}s)\limplies (r\land s\;\mathsf{even}))$,
which is a tautology, whereas the precondition of $\DW{z}{1}$ becomes
$(1{=}r\limplies(2{=}s\lor0{=}s)\limplies (r\land s))$,
which is not.   In order to be top-level, $\DW{z}{1}$ must depend on
$\DR{x}{2}$; in this case the precondition becomes
$(1{=}r\limplies2{=}s\limplies (r\land s))$, which is a tautology.  
\begin{gather*}
  % \PW{x}{0}\SEMI
  % \PR{x}{r}\SEMI
  % \PR{x}{s}\SEMI
  % \IF{r\land s\;\mathsf{even}}\THEN \PW{y}{1}\FI\SEMI
  % \IF{r\land s}\THEN \PW{z}{1}\FI
  % \\
  \hbox{\begin{tikzinline}[node distance=1.5em]
      \event{a0}{\DW{x}{0}}{}
      \event{a1}{\DR{x}{1}}{right=of a0}
      \event{a2}{\DR{x}{2}}{right=of a1}
      \event{a3}{\DW{y}{1}}{right=of a2}
      \event{a4}{\DW{z}{1}}{right=of a3}
      % \wk{a0}{a1}
      % \wk[out=-20,in=-160]{a0}{a2}
      \po[out=20,in=160]{a1}{a3}
      \po[out=20,in=160]{a1}{a4}
      \po[out=-20,in=-160]{a2}{a4}
    \end{tikzinline}}
\end{gather*}
The situation reverses with the substitution $[r/x]$:
\begin{gather*}
  \PR{x}{r}\SEMI
  \PR{x}{s}\SEMI
  \IF{r\land s\;\mathsf{even}}\THEN \PW{y}{1}\FI\SEMI
  \IF{r\land s}\THEN \PW{z}{1}\FI
  \\
  \hbox{\begin{tikzinline}[node distance=0.5em and 1.5em]
      \event{a1}{\DR{x}{1}}{}
      \event{a2}{\DR{x}{2}}{below=of a1}
      \event{a3}{1{=}r\limplies  (2{=}s\lor r{=}s)\limplies (r\land s\;\mathsf{even})\mid\DW{y}{1}}{right=of a1}
      \event{a4}{1{=}r\limplies  (2{=}s\lor r{=}s)\limplies (r\land s)\mid\DW{z}{1}}{below=of a3}
      \po{a1}{a3}
      \po[out=-20,in=177]{a1}{a4}
    \end{tikzinline}}
\end{gather*}
Prepending $\PWP{x}{0}$:
\begin{gather*}
  % \PW{x}{0}\SEMI
  % \PR{x}{r}\SEMI
  % \PR{x}{s}\SEMI
  % \IF{r\land s\;\mathsf{even}}\THEN \PW{y}{1}\FI\SEMI
  % \IF{r\land s}\THEN \PW{z}{1}\FI
  % \\
  \hbox{\begin{tikzinline}[node distance=1.5em]
      \event{a0}{\DW{x}{0}}{}
      \event{a1}{\DR{x}{1}}{right=of a0}
      \event{a2}{\DR{x}{2}}{right=of a1}
      \event{a3}{\DW{y}{1}}{right=of a2}
      \event{a4}{\DW{z}{1}}{right=of a3}
      % \wk{a0}{a1}
      % \wk[out=-20,in=-160]{a0}{a2}
      \po[out=20,in=160]{a1}{a3}
      \po[out=20,in=160]{a1}{a4}
      \po{a2}{a3}
    \end{tikzinline}}
\end{gather*}
The dependency has changed from $\DRP{x}{2}\xpo\DWP{z}{1}$ to
$\DRP{x}{2}\xpo\DWP{y}{1}$.  The resulting sets of pomsets are
incomparable.


Thinking in terms of hardware, the difference is whether reads update the
cache, thus clobbering preceding writes.  With $[r/x]$, reads clobber the
cache, whereas without the substitution, they do not.  Since most caches work
this way, the model with $[r/x]$ is likely preferred for modeling hardware.
In a software model, however, we see no reason to prefer one of these over
the other.


\begin{comment}
  if in L6 we said [x/r], that says we know read the local version...  ignoring
  the value read...  Perhaps there is some intervening stuff that stops you
  from seeing the local state, such as release-acquire

  We could potentially get rid of [x/r] If you do two reads, your not allowed
  to be independent of the second based on the value that was read in the first
  read.

  x=0; r=x; if (r=1) { s=x; if (s=?) {y=1}}
  read 1 then 2.


  In order for the write to be independent of second read what does its
  precondition have to be.
  [r/x] then s==1
  no sub then s==0

  (s=? | Wy1)

  if (phi) z=1
  phi = s is even
  phi = s < 2

  With substitution you are saying you know that the ``local copy'' of x is the
  same as r.  Sitting in the local cache.  Read might have gone to main
  memory, but if it did it has updated the cache line so that the local copy is
  what I just read.

  If second read is a cache hit, then I know that I am seeing the same value.

  If we take substitution out then 
\end{comment}


\section{Differences with OOPSLA}
\label{sec:diff}

\subsubsection*{Substitution}

\jjr{} uses substitution rather than Skolemizing.  Indeed our use of
Skolemization is motivated by disjunction closure for predicate transformers,
which do not appear in \jjr{}; see \textsection\ref{sec:pomsets-trans}.

In \textsection\ref{sec:tc1}, % (\refdef{def:pomsets-lir}):
we give the semantics of load for nonempty pomsets as:
\begin{enumerate}
\item[\ref{L4})]
  $\aTr{\bEvs}{\bForm}$ implies $\aVal{=}\aReg\limplies\bForm$, 
\item[\ref{L5})]
  $\aTr{\cEvs}{\bForm}$ implies
  $(\aVal{=}\aReg\lor\aLoc{=}\aReg)\limplies\bForm$.
  % , when $\aEvs\neq\emptyset$,
  % \item[\ref{L6})] 
  %   $\aTr{\dEvs}{\bForm}\;$ implies $\bForm$, when $\aEvs=\emptyset$.
\end{enumerate}
In \jjr{}, the definition is roughly as follows:
% (adding the case for $\ref{L6}$, which was missing):
\begin{enumerate}
\item[\ref{L4})]
  $\aTr{\bEvs}{\bForm}$ implies $\bForm[\aVal/\aReg][\aVal/\aLoc]$, 
\item[\ref{L5})]
  $\aTr{\cEvs}{\bForm}\;$ implies $\bForm[\aVal/\aReg][\aVal/\aLoc]\land\bForm[\aLoc/\aReg]$.
\end{enumerate}
These substitutions collapse $\aLoc$ and $\aReg$, allowing local invariant
reasoning, as in \textsection\ref{sec:tc1}.  Without Skolemizing it is
necessary to substitute $[\aLoc/\aReg]$, since the reverse substitution
$[\aReg/\aLoc]$ is useless when $\aReg$ is bound.

Removing the substitution of $[x/r]$ in the independent case has a small
technical advantage: we no longer require \emph{extended} expressions (which
include memory references), since substitutions no longer introduce memory
references.

\begin{scope}
  The substitution $[x/r]$ does not work with Skolemization, even for the
  dependent case, since we lose the unique marker for each read.  In effect,
  this forces the reads to the same values.  To be concrete, the candidate
  definition would modify \ref{L4} to be:
  \begin{enumerate}
  \item[\ref{L4})]
    $\aTr{\bEvs}{\bForm}$ implies $\aVal{=}\aLoc\limplies\bForm[\aLoc/\aReg]$.
    % \item[\ref{L5})]
    %   $\aTr{\cEvs}{\bForm}$ implies
    %   $(\aVal{=}\aLoc\lor\TRUE)\limplies\bForm[\aLoc/\aReg]$. %, when $\aEvs\neq\emptyset$,
    % \item[\ref{L6})] 
    %   $\aTr{\dEvs}{\bForm}\;$ implies $\bForm$, when $\aEvs=\emptyset$.
  \end{enumerate}
  Using this definition, consider the following:
  \begin{gather*}
    \PR{x}{r}\SEMI
    \PR{x}{s}\SEMI
    \IF{r{<}s}\THEN \PW{y}{1}\FI 
    \\
    \hbox{\begin{tikzinline}[node distance=0.5em and 1.5em]
        \event{a1}{\DR{x}{1}}{}
        \event{a2}{\DR{x}{2}}{right=of a1}
        \event{a3}{1{=}x\limplies 2{=}x\limplies x{<} x\mid\DW{y}{1}}{right=of a2}
        \po[out=20,in=160]{a1}{a3}
        \po{a2}{a3}
      \end{tikzinline}}
  \end{gather*}
  Although the execution seems reasonable, the precondition on the write is
  not a tautology.
\end{scope}


% There, item \ref{loadpre-kappa2}  of $\sLOADPRE{}{}{}$ is written 
% \begin{enumerate}
% \item[] %[\ref{loadpre-kappa2})]
%   if $\aEv\in\aEvs_2\setminus\aEvs_1$ then either \\
%   $\labelingForm(\aEv)$ implies $\labelingForm_2(\aEv)[\aLoc/\aReg][\aVal/\aLoc]$ and $(\exists\bEv\in\aEvs_1)\bEv{<}\aEv$, or \\
%   $\labelingForm(\aEv)$ implies
%   $\labelingForm_2(\aEv)[\aLoc/\aReg][\aVal/\aLoc] \land \labelingForm_2(\aEv)[\aLoc/\aReg]$.
% \end{enumerate}


% [Skolemization ensures disjunction closure, which is necessary
% for associativity. Show example.]

\subsubsection*{Consistency}
\jjr{} imposes \emph{consistency}, which requires that for every pomset
$\aPS$, $\bigwedge_{\aEv}\labelingForm(\aEv)$ is satisfiable.  
\begin{scope}
  Associativity requires that we allow pomsets with inconsistent
  preconditions.  Consider a variant of \refex{ex:if1} from
  \textsection\ref{sec:if}.
  \begin{scope}
    \footnotesize
    \begin{align*}
      \begin{gathered}
        \IF{\aExp}\THEN\PW{x}{1}\FI
        \\
        \hbox{\begin{tikzinline}[node distance=1em]
            \event{a}{\aExp\mid\DW{x}{1}}{}
          \end{tikzinline}}
      \end{gathered}
      &&
      \begin{gathered}
        \IF{\BANG\aExp}\THEN\PW{x}{1}\FI
        \\
        \hbox{\begin{tikzinline}[node distance=1em]
            \event{a}{\lnot\aExp\mid\DW{x}{1}}{}
          \end{tikzinline}}
      \end{gathered}
      &&
      \begin{gathered}
        \IF{\aExp}\THEN\PW{y}{1}\FI
        \\
        \hbox{\begin{tikzinline}[node distance=1em]
            \event{a}{\aExp\mid\DW{y}{1}}{}
          \end{tikzinline}}
      \end{gathered}
      &&
      \begin{gathered}
        \IF{\BANG\aExp}\THEN\PW{y}{1}\FI
        \\
        \hbox{\begin{tikzinline}[node distance=1em]
            \event{a}{\lnot\aExp\mid\DW{y}{1}}{}
          \end{tikzinline}}
      \end{gathered}
    \end{align*}
  \end{scope}
  Associating left and right, we have:
  \begin{scope}
    \footnotesize
    \begin{align*}
      \begin{gathered}
        \IF{\aExp}\THEN\PW{x}{1}\FI
        \SEMI
        \IF{\BANG\aExp}\THEN\PW{x}{1}\FI
        \\
        \hbox{\begin{tikzinline}[node distance=1em]
            \event{a}{\DW{x}{1}}{}
          \end{tikzinline}}
      \end{gathered}
      &&
      \begin{gathered}
        \IF{\aExp}\THEN\PW{y}{1}\FI
        \SEMI
        \IF{\BANG\aExp}\THEN\PW{y}{1}\FI
        \\
        \hbox{\begin{tikzinline}[node distance=1em]
            \event{a}{\DW{y}{1}}{}
          \end{tikzinline}}
      \end{gathered}
    \end{align*}
  \end{scope}  
  Associating into the middle, instead, we require:
  \begin{scope}
    \footnotesize
    \begin{align*}
      \begin{gathered}
        \IF{\aExp}\THEN\PW{x}{1}\FI
        \\
        \hbox{\begin{tikzinline}[node distance=1em]
            \event{a}{\aExp\mid\DW{x}{1}}{}
          \end{tikzinline}}
      \end{gathered}
      &&
      \begin{gathered}
        \IF{\BANG\aExp}\THEN\PW{x}{1}\FI
        \SEMI
        \IF{\aExp}\THEN\PW{y}{1}\FI
        \\
        \hbox{\begin{tikzinline}[node distance=1em]
            \event{a}{\lnot\aExp\mid\DW{x}{1}}{}
            \event{b}{\aExp\mid\DW{y}{1}}{right=of a}
          \end{tikzinline}}
      \end{gathered}
      &&
      \begin{gathered}
        \IF{\BANG\aExp}\THEN\PW{y}{1}\FI
        \\
        \hbox{\begin{tikzinline}[node distance=1em]
            \event{a}{\lnot\aExp\mid\DW{y}{1}}{}
          \end{tikzinline}}
      \end{gathered}
    \end{align*}
  \end{scope}
  Joining left and right, we have:
  \begin{scope}
    \footnotesize
    \begin{align*}
      \begin{gathered}
        \IF{\aExp}\THEN\PW{x}{1}\FI
        \SEMI
        \IF{\BANG\aExp}\THEN\PW{x}{1}\FI
        \SEMI
        \IF{\aExp}\THEN\PW{y}{1}\FI
        \SEMI
        \IF{\BANG\aExp}\THEN\PW{y}{1}\FI
        \\
        \hbox{\begin{tikzinline}[node distance=1em]
            \event{a}{\DW{x}{1}}{}
            \event{b}{\DW{y}{1}}{right=of a}
          \end{tikzinline}}
      \end{gathered}
    \end{align*}
  \end{scope}  
\end{scope}

\subsubsection*{Causal Strengthening}
\jjr{} imposes \emph{causal strengthening}, which requires for every pomset
$\aPS$, if $\bEv\le\aEv$ then $\labelingForm(\aEv)$ implies
$\labelingForm(\bEv)$. 
\begin{scope}
  Associativity requires that we allow pomsets without causal strengthening.
  Consider the following.
  \begin{align*}
    \begin{gathered}
      \IF{\aExp}\THEN\PR{x}{r}\FI
      \\
      \hbox{\begin{tikzinline}[node distance=1em]
          \event{a}{\aExp\mid\DR{x}{1}}{}
        \end{tikzinline}}
    \end{gathered}
    &&
    \begin{gathered}
      \PW{y}{r}
      \\
      \hbox{\begin{tikzinline}[node distance=1em]
          \event{a}{r{=}1\mid\DW{y}{1}}{}
        \end{tikzinline}}
    \end{gathered}
    &&
    \begin{gathered}
      \IF{\BANG\aExp}\THEN\PR{x}{s}\FI
      \\
      \hbox{\begin{tikzinline}[node distance=1em]
          \event{a}{\lnot\aExp\mid\DR{x}{1}}{}
        \end{tikzinline}}
    \end{gathered}
  \end{align*}
  Associating left, with causal strengthening:
  \begin{align*}
    \begin{gathered}
      \IF{\aExp}\THEN\PR{x}{r}\FI
      \SEMI
      \PW{y}{r}
      \\
      \hbox{\begin{tikzinline}[node distance=1em]
          \event{a}{\aExp\mid\DR{x}{1}}{}
          \event{b}{\aExp\mid\DW{y}{1}}{right=of a}
          \po{a}{b}
        \end{tikzinline}}
    \end{gathered}
    &&
    \begin{gathered}
      \IF{\BANG\aExp}\THEN\PR{x}{s}\FI
      \\
      \hbox{\begin{tikzinline}[node distance=1em]
          \event{a}{\lnot\aExp\mid\DR{x}{1}}{}
        \end{tikzinline}}
    \end{gathered}
  \end{align*}
  Finally, merging:
  \begin{align*}
    \begin{gathered}
      \IF{\aExp}\THEN\PR{x}{r}\FI
      \SEMI
      \PW{y}{r}
      \SEMI
      \IF{\BANG\aExp}\THEN\PR{x}{s}\FI
      \\
      \hbox{\begin{tikzinline}[node distance=1em]
          \event{a}{\DR{x}{1}}{}
          \event{b}{\aExp\mid\DW{y}{1}}{right=of a}
          \po{a}{b}
        \end{tikzinline}}
    \end{gathered}
  \end{align*}
  Instead, associating right:
  \begin{align*}
    \begin{gathered}
      \IF{\aExp}\THEN\PR{x}{r}\FI
      \\
      \hbox{\begin{tikzinline}[node distance=1em]
          \event{a}{\aExp\mid\DR{x}{1}}{}
        \end{tikzinline}}
    \end{gathered}
    &&
    \begin{gathered}
      \PW{y}{r}
      \SEMI
      \IF{\BANG\aExp}\THEN\PR{x}{s}\FI
      \\
      \hbox{\begin{tikzinline}[node distance=1em]
          \event{a}{\lnot\aExp\mid\DR{x}{1}}{}
          \event{b}{r{=}1\mid\DW{y}{1}}{left=of a}
        \end{tikzinline}}
    \end{gathered}
  \end{align*}
  Merging:
  \begin{align*}
    \begin{gathered}
      \IF{\aExp}\THEN\PR{x}{r}\FI
      \SEMI
      \PW{y}{r}
      \SEMI
      \IF{\BANG\aExp}\THEN\PR{x}{s}\FI
      \\
      \hbox{\begin{tikzinline}[node distance=1em]
          \event{a}{\DR{x}{1}}{}
          \event{b}{\DW{y}{1}}{right=of a}
          \po{a}{b}
        \end{tikzinline}}
    \end{gathered}
  \end{align*}
  With causal strengthening, the precondition of $\DW{y}{1}$ depends upon how
  we associate.  This is not an issue in \jjr{}, which always associates to
  the right.
\end{scope}

\subsubsection*{Causal Strengthening and Address Dependencies}
\begin{scope}
  In order to guarantee that address calculation does not introduce thin-air
  executions, the predicate transformer for address calculation must be
  chosen carefully.
  Combing \refdef{def:pomsets-rr} and \refdef{def:pomsets-addr} we have:

  \begin{enumerate}
    % \item[\ref{L4})]
    %   $\aTr{\bEvs}{\bForm}$ implies $\aVal{=}\aReg\limplies\bForm$, 
    % \item[\ref{L5})]
    %   $\aTr{\cEvs}{\bForm}$ implies $(\aVal{=}\aReg\lor\RW)\limplies\bForm$,
    
  \item[\ref{L4})]
    $\aTr{\bEvs}{\bForm}$ implies
    \begin{math}
      (\cExp{=}\cVal\limplies\aVal{=}\aReg)\limplies\bForm,
    \end{math}
  \item[\ref{L5})]
    $\aTr{\cEvs}{\bForm}$ implies
    \begin{math}
      ((\cExp{=}\cVal\limplies\aVal{=}\aReg)\lor\RW)\limplies\bForm.
    \end{math}
  \end{enumerate}  
  
  Consider the following program, from \jjr{\textsection5}, where initially $x=0$, $y=0$, $\REF{0}=0$,
  $\REF{1}=2$, and $\REF{2}=1$.  It should only be possible to read $0$,
  disallowing the attempted execution below:
  \begin{gather*}
    \begin{gathered}
      r\GETS y\SEMI s\GETS \REF{r}\SEMI x\GETS s
      \PAR
      r\GETS x\SEMI s\GETS \REF{r}\SEMI y\GETS s
      \\
      \hbox{\begin{tikzinline}[node distance=1.5em]
          \event{a1}{\DR{y}{2}}{}
          \event{a2}{\DR{\REF{2}}{1}}{right=of a1}
          \event{a3}{\DW{x}{1}}{right=of a2}
          \po{a2}{a3}
          \po[out=10,in=170]{a1}{a3}
          \event{b1}{\DR{x}{1}}{right=3em of a3}
          \event{b2}{\DR{\REF{1}}{2}}{right=of b1}
          \event{b3}{\DW{y}{2}}{right=of b2}
          \po{b2}{b3}
          \po[out=10,in=170]{b1}{b3}
          \rf[out=-170,in=-10]{b3}{a1}
          \rf{a3}{b1}
        \end{tikzinline}}
    \end{gathered}
  \end{gather*}
  Looking at the left thread:
  \begin{align*}
    \begin{gathered}[t]
      r\GETS y
      \\
      \hbox{\begin{tikzinline}[node distance=.5em and 1.5em]
          \event{a}{\DR{y}{2}}{}
          \xform{xi}{(2{=}r\lor\RW)\limplies\bForm}{above=of a}
          \xform{xd}{2{=}r\limplies\bForm}{below=of a}
          \xo{a}{xd}
        \end{tikzinline}}
    \end{gathered}
    &&
    \begin{gathered}[t]
      s\GETS \REF{r}
      \\
      \hbox{\begin{tikzinline}[node distance=.5em and 1.5em]
          \event{b}{r\EQ2\mid\DR{\REF{2}}{1}}{}
          \xform{xd}{(r\EQ2\limplies 1\EQ s) \limplies\bForm}{below=of b}
          \xform{xi}{((r\EQ2\limplies 1\EQ s)\lor\RW) \limplies\bForm}{above=of b}
          \xo{b}{xd}
        \end{tikzinline}}
    \end{gathered}
    &&
    \begin{gathered}[t]
      x\GETS s
      \\
      \hbox{\begin{tikzinline}[node distance=.5em and 1.5em]
          \event{b}{s\EQ1\mid\DW{x}{1}}{}
          \xform{xd}{\bForm}{below=of b}
          \xform{xi}{\bForm}{above=of b}
          \xo{b}{xd}
        \end{tikzinline}}
    \end{gathered}
  \end{align*}
  Composing, we have:
  \begin{gather*}
    \begin{gathered}
      r\GETS y\SEMI s\GETS \REF{r}\SEMI x\GETS s
      \\
      \hbox{\begin{tikzinline}[node distance=.5em and 1.5em]
          \event{a1}{\DR{y}{2}}{}
          \event{a2}{(2{=}r\lor\RW)\limplies r\EQ2\mid\DR{\REF{2}}{1}}{right=of a1}
          \event{a3}{(2{=}r\lor\RW)\limplies (r\EQ2\limplies 1\EQ s)
            \limplies s\EQ1\mid\DW{x}{1}}{below right=.5em and -8em of a2}
          \po{a2}{a3}
        \end{tikzinline}}
    \end{gathered}
  \end{gather*}  
  Substituting for $\RW$:
  \begin{gather*}
    \begin{gathered}
      r\GETS y\SEMI s\GETS \REF{r}\SEMI x\GETS s
      \\
      \hbox{\begin{tikzinline}[node distance=.5em and 1.5em]
          \event{a1}{\DR{y}{2}}{}
          \event{a2}{(2{=}r\lor\TRUE)\limplies r\EQ2\mid\DR{\REF{2}}{1}}{right=of a1}
          \event{a3}{(2{=}r\lor\FALSE)\limplies (r\EQ2\limplies 1\EQ s)
            \limplies s\EQ1\mid\DW{x}{1}}{below right=.5em and -8em of a2}
          \po{a2}{a3}
        \end{tikzinline}}
    \end{gathered}
  \end{gather*}
  The precondition of $\DRP{\REF{2}}{1}$ is a tautology, but the precondition
  of $\DWP{x}{1}$ is not.  This forces a dependency:
  \begin{gather*}
    \begin{gathered}
      r\GETS y\SEMI s\GETS \REF{r}\SEMI x\GETS s
      \\
      \hbox{\begin{tikzinline}[node distance=.5em and 1.5em]
          \event{a1}{\DR{y}{2}}{}
          \event{a2}{(2{=}r\lor\TRUE)\limplies r\EQ2\mid\DR{\REF{2}}{1}}{right=of a1}
          \event{a3}{2{=}r\limplies (r\EQ2\limplies 1\EQ s)
            \limplies s\EQ1\mid\DW{x}{1}}{below right=.5em and -5em of a2}
          \po[out=-20,in=180]{a1}{a3}
          \po{a2}{a3}
        \end{tikzinline}}
    \end{gathered}
  \end{gather*}
  All the preconditions are now tautologies.
\end{scope}

\subsubsection*{Parallel Composition}

In \jjr{\textsection2.4}, parallel composition is defined allowing coalescing
of events.  Here we have forbidden coalescing.  This difference appears to be
arbitrary.  In \jjr{}, however, there is a mistake in the handling of
termination actions.  The predicates should be joined using $\land$, not
$\lor$.

\subsubsection*{Internal Acquiring Reads}

Shortly after publication, \citet{anton} noticed a shortcoming of the
implementation on \armeight{} in \jjr{\textsection 7}.  The proof given there
assumes that all internal reads can be dropped.  However, this is not the
case for acquiring reds.  For example, \jjr{} disallows the following
execution, which is allowed by \armeight{} and \tso{}.
\begin{gather*}
  \PW{x}{2}\SEMI 
  \PR[\mRA]{x}{r}\SEMI
  \PR{y}{s} \PAR
  \PW{y}{2}\SEMI
  \PW[\mRA]{x}{1}
  \\
  \hbox{\begin{tikzinline}[node distance=1.5em]
      \event{a}{\DW{x}{2}}{}
      \raevent{b}{\DR[\mRA]{x}{2}}{right=of a}
      \event{c}{\DR{y}{0}}{right=of b}
      \event{d}{\DW{y}{2}}{right=2.5em of c}
      \raevent{e}{\DW[\mRA]{x}{1}}{right=of d}
      \rf{a}{b}
      \sync{b}{c}
      \wk{c}{d}
      \sync{d}{e}
      \wk[out=-165,in=-15]{e}{a}
      % \rfi{a}{b}
      % \bob{b}{c}
      % \fre{c}{d}
      % \bob{d}{e}
      % \coe[out=-165,in=-15]{e}{a}
    \end{tikzinline}}
\end{gather*}
The solution we have adopted is to allow an acquiring read to be downgraded
to a relaxed read when it is preceded (sequentially) by a relaxed write that
could fulfill it.  This solution allows executions that are not allowed under
\armeight{} since we do not insist that the local relaxed write is actually
read from.  This may seem counterintuitive, but we don't see a local way to
be more precise.

As a result, we use a different proof strategy for \armeight{}
implementation, which does not rely on read elimination.  The proof idea uses
a recent alternative characterization of \armeight{}
\citep{alglave-git-alternate,arm-reference-manual}. %,armed-cats}.

\subsubsection*{Redundant Read Elimination}

Contrary to the claim, redundant read elimination fails for \jjr{}.
We discussed redundant read elimination in \textsection\ref{sec:recycle}.
Consider JMM Causality Test Case 2, which we discussed there.
\begin{gather*}
  \PR{x}{r}\SEMI
  \PR{x}{s}\SEMI
  \IF{r{=}s}\THEN \PW{y}{1}\FI
  \PAR
  x\GETS y
  \\
  \hbox{\begin{tikzinline}[node distance=1.5em]
      \event{a1}{\DR{x}{1}}{}
      \event{a2}{\DR{x}{1}}{right=of a1}
      \event{a3}{\DW{y}{1}}{right=of a2}
      \event{b1}{\DR{y}{1}}{right=3em of a3}
      \event{b2}{\DW{x}{1}}{right=of b1}
      \rf{a3}{b1}
      \po{b1}{b2}
      \rf[out=169,in=11]{b2}{a2}
      \rf[out=169,in=11]{b2}{a1}
    \end{tikzinline}}
\end{gather*}
Under the semantics of \jjr{}, we have
\begin{gather*}
  \PR{x}{r}\SEMI
  \PR{x}{s}\SEMI
  \IF{r{=}s}\THEN \PW{y}{1}\FI
  \\
  \hbox{\begin{tikzinline}[node distance=1.5em]
      \event{a1}{\DR{x}{1}}{}
      \event{a2}{\DR{x}{1}}{right=of a1}
      \event{a3}{1\EQ1\land1\EQ x \land x\EQ1 \land x=x\mid\DW{y}{1}}{right=of a2}
    \end{tikzinline}}
\end{gather*}
The precondition of $\DWP{y}{1}$ is \emph{not} a tautology, and therefore
redundant read elimination fails.
(It is a tautology in
\begin{math}
  \PR{x}{r}\SEMI
  \LET{s}{r}\SEMI
  \IF{r{=}s}\THEN \PW{y}{1}\FI
\end{math}.)
In \jjr{\textsection3.1}, we incorrectly stated that the precondition of
$\DWP{y}{1}$ was $1\EQ1\land x\EQ x$.  

\begin{comment}
  Precondition of $\DWP{y}{1}$ is $(r{=}s)$ in
  \begin{math}
    \sem{\IF{r{=}s}\THEN y\GETS 1\FI}.
  \end{math}
  Predicate transformers for $\emptyset$ in $\sem{\PR{x}{r}}$ and $\sem{\PR{x}{s}}$ are
  \begin{align*}
    \PREDP{(r{=}1 \lor r{=}x)\limplies\bForm[r/x]},
    \\
    \PREDP{(s{=}1 \lor s{=}x)\limplies\bForm[s/x]}.
  \end{align*}
  Combining the transformers, we have
  \begin{displaymath}
    \PREDP{(r{=}1 \lor r{=}x)\limplies(s{=}1 \lor s{=}r)\limplies\bForm[s/x]}.
  \end{displaymath}
  Applying this to $(r{=}s)$, we have
  \begin{displaymath}
    \PREDP{(r{=}1 \lor r{=}x)\limplies (s{=}1 \lor s{=}r)\limplies (r{=}s)},
  \end{displaymath}
  which is not a tautology.

  Same problem occurs \jjr{}, where we have:
  \begin{align*}
    \PREDP{\bForm[v/x,r] \land \bForm[x/r]},
    \\
    \PREDP{\bForm[v/x,s] \land \bForm[x/s]}.
  \end{align*}
  Combining the transformers, we have
  \begin{displaymath}
    \PREDP{\bForm[v/x,r,s] \land \bForm [v/x,r][x/s] \land \bForm[x/r][v/x,s] \land \bForm[x/r,s]}.
  \end{displaymath}
  Applying this to $(r{=}s)$, we have
  \begin{displaymath}
    \PREDP{v{=}v \land v{=}x \land x{=}v \land x{=}x},
  \end{displaymath}
  which is not a tautology.

  The semantics here allows this by coalescing:
  \begin{gather*}
    r\GETS x\SEMI
    s\GETS x\SEMI
    \IF{r{=}s}\THEN y\GETS 1\FI
    \PAR
    x\GETS y
    \\
    \hbox{\begin{tikzinline}[node distance=1.5em]
        \event{a1}{\DR{x}{1}}{}
        \event{a3}{\DW{y}{1}}{right=of a1}
        \event{b1}{\DR{y}{1}}{right=3em of a3}
        \event{b2}{\DW{x}{1}}{right=of b1}
        \rf{a3}{b1}
        \po{b1}{b2}
        \rf[out=169,in=11]{b2}{a1}
      \end{tikzinline}}
  \end{gather*}

  In \jjr{\textsection2.6} the semantics of read is defined as follows:
  \begin{align*}
    \sem{\aReg\GETS\aLoc^\amode\SEMI \aCmd} & \eqdef \textstyle\bigcup_\aVal\;
    (\DRmode\aLoc\aVal) \prefix \sem{\aCmd} [\aLoc/\aReg]
  \end{align*}
  The definition of prefixing$((\aForm \mid \aAct) \prefix \aPSS)$ has several clauses.
  The most relevant are as follows, where $\bEv$ is the new event labeled with
  $(\aForm \mid \aAct)$ and $\aEv$ is an event from $\aPSS$:
  \begin{description}
  \item[{\labeltextsc[P4c]{(P4c)}{4c}}]
    If $\bEv$ reads $\aVal$ from $\aLoc$ then either $\aEv=\bEv$ or
    $\labelingForm'(\aEv)$ implies $\labelingForm(\aEv)[\aVal/\aLoc]$.
  \item[{\labeltextsc[P5a]{(P5a)}{5a}}]\labeltextsc[P5]{}{5}%
    If $\bEv$ reads and $\aEv$ writes then either $\labelingForm'(\aEv)$
    implies $\labelingForm(\aEv)$ or $\bEv\le'\aEv$.
    % \item[{\labeltextsc[P5b]{(P5b)}{5b}}]
    %   If $\bEv$ and $\aEv$ are in conflict then $\bEv\le'\aEv$.
  \end{description}

  We have discovered two issues with this definition.

  The first issue concerns the substitution $[\aLoc/\aReg]$.  It should be
  $[\aReg/\aLoc]$.  We noticed this error while developing the alternative
  characterization presented here.  The error causes redundant read elimination
  to fail in \jjr{}.  As a result, common subexpression elimination also fails.
  The problem can be seen in \ref{TC2}.
  \begin{gather*}
    \taglabel{TC2}
    r\GETS x\SEMI
    s\GETS x\SEMI
    \IF{r{=}s}\THEN y\GETS 1\FI
    \PAR
    x\GETS y
  \end{gather*}
  % In \jjr{\textsection3.1},
  We claimed that \ref{TC2} allowed the following
  execution:
  \begin{gather*}
    \hbox{\begin{tikzinline}[node distance=1.5em]
        \event{a1}{\DR{x}{1}}{}
        \event{a2}{\DR{x}{1}}{right=of a1}
        \event{a3}{\DW{y}{1}}{right=of a2}
        % \po{a2}{a3}
        % \po[out=15,in=165]{a1}{a3}
        \event{b1}{\DR{y}{1}}{right=3em of a3}
        \event{b2}{\DW{x}{1}}{right=of b1}
        \rf{a3}{b1}
        \po{b1}{b2}
        \rf[out=169,in=11]{b2}{a2}
        \rf[out=169,in=11]{b2}{a1}
      \end{tikzinline}}
  \end{gather*}
  But this execution is not possible using the semantics of \jjr{}:
  $\DWP{y}{1}$ has precondition $r{=}s$ in
  \begin{math}
    \sem{\IF{r{=}s}\THEN y\GETS 1\FI}.
  \end{math}
  Given the lack of order in the execution, the precondition of $\DWP{y}{1}$
  must entail $r{=}1\land r{=}x$ in 
  \begin{math}
    \sem{s\GETS x\SEMI
      \IF{r{=}s}\THEN y\GETS 1\FI}.
  \end{math}
  \ref{4c} imposes $r{=}1$, and \ref{5a} imposes $r{=}x$.  Adding the second
  read, the precondition of $\DWP{y}{1}$ must entail both $1{=}1\land 1{=}x$
  and also $x{=}1\land x{=}x$.  This can be simplified to $x{=}1$.  This leaves
  a requirement that must be satisfied by a preceding write.  Since the
  preceding write is the initialization to $0$, the requirement cannot be
  satisfied, and the execution is impossible.\footnote{In \jjr{} we ignore the
    middle terms, mistakenly simplifying this to $1{=}1\land x{=}x$.
    Correcting the error, the attempted execution is:
    \begin{gather*}
      \hbox{\begin{tikzinline}[node distance=1.5em]
          \event{a1}{\DR{x}{1}}{}
          \event{a2}{\DR{x}{1}}{right=of a1}
          \event{a3}{\DW{y}{1}}{right=of a2}
          \po{a2}{a3}
          \po[out=-20,in=-160]{a1}{a3}
          \event{b1}{\DR{y}{1}}{right=3em of a3}
          \event{b2}{\DW{x}{1}}{right=of b1}
          \rf{a3}{b1}
          \po{b1}{b2}
          \rf[out=169,in=11]{b2}{a2}
          \rf[out=169,in=11]{b2}{a1}
        \end{tikzinline}}
    \end{gather*}}

  The substitution $[\aLoc/\aReg]$ leaves the obligation on $\aLoc$ to be
  fulfilled by the preceding write.  Thus, the read does not update the
  \emph{value} of $\aLoc$ in subsequent predicates.  The substitution
  $[\aReg/\aLoc]$, instead, does update the value of $\aLoc$, thus removing any
  obligation on $\aLoc$ for preceding code.

  In order to write this, we must update the definition of prefixing reads to
  include the register.  Then \ref{4c} becomes:
  \begin{description}
  \item[\textsc{(p4c)}] If $\bEv$ reads $\aVal$ from $\aLoc$ then either
    $\aEv=\bEv$ or $\labelingForm'(\aEv)$ implies
    $\labelingForm(\aEv)[\aVal/\aReg]$.
  \end{description}

  We can then reason with \ref{TC2} as follows: $\DWP{y}{1}$ has precondition
  $r{=}s$ in
  \begin{math}
    \sem{\IF{r{=}s}\THEN y\GETS 1\FI}.
  \end{math}
  To avoid introducing order in the execution, the precondition of $\DWP{y}{1}$
  must entail $r{=}1\land r{=}s$ in 
  \begin{math}
    \sem{s\GETS x\SEMI
      \IF{r{=}s}\THEN y\GETS 1\FI}.
  \end{math}
  \ref{4c} imposes $r{=}1$, and \ref{5a} imposes $r{=}x$.  Adding the second
  read, the precondition of $\DWP{y}{1}$ must entail both $1{=}1\land 1{=}x$
  and also $x{=}1\land x{=}x$.  This can be simplified to $x{=}1$.  This leaves
  a requirement that must be satisfied by a preceding write.


  With read elimination, the rule for relaxed reads is as follows:
  \begin{align*}
    \sem{\PR{\aLoc}{\aReg} \SEMI \aCmd} &\eqdef
    \sem{\aCmd}[\aLoc/\aReg]
    \cup
    \textstyle\bigcup_\aVal\;
    \DRP{\aLoc}{\aVal} \prefix_{\aReg} %\Rdis{\aLoc}{\aVal}
    \sem{\aCmd}[\aReg/\aLoc]
  \end{align*}
  It is interesting to note that the substitution is $[\aLoc/\aReg]$ on
  eliminated reads, and $[\aReg/\aLoc]$ on non-eliminated reads.  Intuitively,
  the subsequent value of $\aLoc$ is fixed by an explicit read, but not for an
  eliminated read.  In the latter case, the value is fixed by some preceding
  action.  The preceding action may itself be a read. This gives rise to some
  fear that we might introduce thin-air reads, since we do not enforce
  read-read coherence.  But this is not the case.  Consider the following example:
  \begin{gather*}
    r\GETS x\SEMI
    s\GETS x\SEMI
    y\GETS s
    \PAR
    x\GETS y
    \\
    \hbox{\begin{tikzinline}[node distance=1.5em]
        \event{a1}{\DR{x}{1}}{}
        \event{a2}{\DR{x}{1}}{right=of a1}
        \event{a3}{\DW{y}{1}}{right=of a2}
        % \po{a2}{a3}
        \po[out=-20,in=-160]{a1}{a3}
        \event{b1}{\DR{y}{1}}{right=3em of a3}
        \event{b2}{\DW{x}{1}}{right=of b1}
        \rf{a3}{b1}
        \po{b1}{b2}
        \rf[out=169,in=11]{b2}{a2}
        \rf[out=169,in=11]{b2}{a1}
      \end{tikzinline}}
    \\
    \hbox{\begin{tikzinline}[node distance=1.5em]
        \event{a1}{\DR{x}{1}}{}
        \internal{a2}{\DR{x}{1}}{right=of a1}
        \event{a3}{\DW{y}{1}}{right=of a2}
        % \po{a2}{a3}
        \po[out=-20,in=-160]{a1}{a3}
        \event{b1}{\DR{y}{1}}{right=3em of a3}
        \event{b2}{\DW{x}{1}}{right=of b1}
        \rf{a3}{b1}
        \po{b1}{b2}
        % \rf[out=169,in=11]{b2}{a2}
        \rf[out=169,in=11]{b2}{a1}
      \end{tikzinline}}
  \end{gather*}
  But this is not a problem, since fulfillment requires that $\DWP{x}{1}$
  precede both reads of $x$.
\end{comment}

\subsubsection*{A Note on Mixed Races}

In preparing this paper, we came across the following example, which appears
to invalidate Theorem 4.1 of \cite{DBLP:conf/ppopp/DongolJR19}.
\begin{gather}
  \nonumber
  \PW{x}{1}\SEMI
  \PW[\mRA]{y}{1}\SEMI
  \PR[\mRA]{x}{r}
  \PAR
  \IF{\PR[\mRA]{y}{}}\THEN \PW[\mRA]{x}{2}\FI
  \\
  \label{mix1}
  \hbox{\begin{tikzinline}[node distance=1.5em]
      \event{a1}{\DW{x}{1}}{}
      \raevent{a2}{\DW[\mRA]{y}{1}}{right=of a1}
      \raevent{a3}{\DR[\mRA]{x}{1}}{right=of a2}
      \raevent{b1}{\DR[\mRA]{y}{1}}{right=3em of a3}
      % \raevent{b1}{\DR[\mRA]{y}{1}}{below=of a1}
      \raevent{b2}{\DW[\mRA]{x}{2}}{right=of b1}
      \sync{a1}{a2}
      \rf[out=20,in=160]{a1}{a3}
      \rf[out=20,in=160]{a2}{b1}
      \wk[out=-20,in=-160]{a3}{b2}
      \sync{b1}{b2}
      % \node(ai)[left=3em of a1]{};
      % \bgoval[yellow!50]{(ai)}{P}
      % \bgoval[pink!50]{(a1)(a2)(b1)(b2)}{P'\setminus P}
      % \bgoval[green!10]{(a3)}{P'''\setminus P'}
    \end{tikzinline}}
  \\
  \label{mix2}
  \hbox{\begin{tikzinline}[node distance=1.5em]
      \event{a1}{\DW{x}{1}}{}
      \raevent{a2}{\DW[\mRA]{y}{1}}{right=of a1}
      \raevent{a3}{\DR[\mRA]{x}{2}}{right=of a2}
      \raevent{b1}{\DR[\mRA]{y}{1}}{right=3em of a3}
      \raevent{b2}{\DW[\mRA]{x}{2}}{right=of b1}
      \sync{a1}{a2}
      \rf[out=20,in=160]{a2}{b1}
      \rf[out=160,in=20]{b2}{a3}
      \sync{b1}{b2}
    \end{tikzinline}}
\end{gather}
The program is data-race free.  The two executions shown are the only
top-level executions that include $\DWP[\mRA]{x}{2}$.

Theorem 4.1 of \cite{DBLP:conf/ppopp/DongolJR19} is stated by extending
execution sequences.  In the terminology of
\cite{DBLP:conf/ppopp/DongolJR19}, a read is \emph{$L$-weak} if it is
sequentially stale.  Let
\begin{math}
  \rho=\DWP{x}{1}
  \DWP[\mRA]{y}{1}
  \DRP[\mRA]{y}{1}
  \DWP[\mRA]{x}{2}
\end{math}
be a sequence and
\begin{math}
  \alpha=\DRP[\mRA]{x}{1}.
\end{math}
$\rho$ is $L$-sequential and $\alpha$ is $L$-weak in $\rho\alpha$.  But there
is no execution of this program that includes a data race, contradicting the
theorem.  The error seems to be in Lemma A.4 of
\cite{DBLP:conf/ppopp/DongolJR19}, which states that if $\alpha$ is $L$-weak
after an $L$-sequential $\rho$, then $\alpha$ must be in a data race.  That
is clearly false here, since $\DRP[\mRA]{x}{1}$ is stale, but the program is
data race free.

In proving the SC-LDRF result in \jjr{\textsection8}, we noted that our proof
technique is more robust than that of \cite{DBLP:conf/ppopp/DongolJR19},
because it limits the prefixes that must be considered.  In \eqref{mix1}, the
induction hypothesis requires that we add $\DRP[\mRA]{x}{1}$ before
$\DWP[\mRA]{x}{2}$ since $\DRP[\mRA]{x}{1}\xwk\DWP[\mRA]{x}{2}$.  In
particular,
\begin{gather*}
  \hbox{\begin{tikzinline}[node distance=1.5em]
      \event{a1}{\DW{x}{1}}{}
      \raevent{a2}{\DW[\mRA]{y}{1}}{right=of a1}
      % \raevent{a3}{\DR[\mRA]{x}{1}}{right=of a2}
      \raevent{b1}{\DR[\mRA]{y}{1}}{right=3em of a3}
      % \raevent{b1}{\DR[\mRA]{y}{1}}{below=of a1}
      \raevent{b2}{\DW[\mRA]{x}{2}}{right=of b1}
      \sync{a1}{a2}
      % \rf[out=20,in=160]{a1}{a3}
      \rf[out=20,in=160]{a2}{b1}
      % \wk[out=-20,in=-160]{a3}{b2}
      \sync{b1}{b2}
      % \node(ai)[left=3em of a1]{};
      % \bgoval[yellow!50]{(ai)}{P}
      % \bgoval[pink!50]{(a1)(a2)(b1)(b2)}{P'\setminus P}
      % \bgoval[green!10]{(a3)}{P'''\setminus P'}
    \end{tikzinline}}
\end{gather*}
is not a downset of \eqref{mix1}, because
$\DRP[\mRA]{x}{1}\xwk\DWP[\mRA]{x}{2}$.  As we noted in \jjr{\textsection8},
this affects the inductive order in which we move across pomsets, but does
not affect the set of pomsets that are considered.  In particular,
\begin{gather*}
  \hbox{\begin{tikzinline}[node distance=1.5em]
      \event{a1}{\DW{x}{1}}{}
      \raevent{a2}{\DW[\mRA]{y}{1}}{right=of a1}
      % \raevent{a3}{\DR[\mRA]{x}{1}}{right=of a2}
      \raevent{b1}{\DR[\mRA]{y}{1}}{right=3em of a3}
      % \raevent{b1}{\DR[\mRA]{y}{1}}{below=of a1}
      % \raevent{b2}{\DW[\mRA]{x}{2}}{right=of b1}
      \sync{a1}{a2}
      % \rf[out=20,in=160]{a1}{a3}
      \rf[out=20,in=160]{a2}{b1}
      % \wk[out=-20,in=-160]{a3}{b2}
      % \sync{b1}{b2}
      % \node(ai)[left=3em of a1]{};
      % \bgoval[yellow!50]{(ai)}{P}
      % \bgoval[pink!50]{(a1)(a2)(b1)(b2)}{P'\setminus P}
      % \bgoval[green!10]{(a3)}{P'''\setminus P'}
    \end{tikzinline}}
\end{gather*}
is a downset of \eqref{mix1}.

