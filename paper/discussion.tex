\section{Relation to Traditional Predicate Transformers}

[Todo.]

\begin{comment}
\begin{proposition}
  If $\aPS\in\sem{\aCmd}$ is top-level and quiescent then 
  $\aTr{\aEvs}{\bForm}$ implies $\fwp{\aCmd}{\bForm}$.

  For any substitution $\aSub=[{v_1/r_1},\ldots, {v_n/r_n}]$ there is some
  $\aPS\in\sem{\aCmd}$ %that is top-level and quiescent
  such that all preconditions in $\aPS\aSub$ are tautologies then 
  $\fwp{\aCmd}{\bForm}\aSub$
\end{proposition}

For a language where all programs are
terminating, we have for any statement $\aCmd$:
\begin{align*}
  \hoare{\aForm}{\aCmd}{\bForm} 
  \;\;\Leftrightarrow\;\;
  \aForm \textimplies \fwp{\aCmd}{\bForm}
\end{align*}
Interpretation is that if $\aState\models\fwp{\aCmd}{\bForm}$ and
$(\aState,\aCmd)\Downarrow\bState$
then $\bState\models\bForm$.

Let $\aCmd_0$ be
\begin{math}
  \PW{\aLoc_1}{\aVal_1}\SEMI\cdots\SEMI \PW{\aLoc_n}{\aVal_n}, 
\end{math}
such that $\fwp{\aCmd_0}{\aForm}$ is a tautology, and $\aLoc_i=\aLoc_j$
implies $i=j$.

Let $\aSub_\aPS=[{\aVal_1/\aLoc_1},\ldots, {\aVal_n/\aLoc_n}]$ be the final
state of $\aPS$.


For example, let $\aCmd_1=\PR{x}{r}$ and $\aCmd_2=\PW{x}{r{+}1}$ and
$\aCmd=\aCmd_1\SEMI \aCmd_2$.
\begin{align*}
  \fwp{\aCmd_2}{x{>}1}&=(r{+}1{>}1) = (r{>}0)
  \\
  \fwp{\aCmd_1}{r{>}0}=\fwp{\aCmd_0}{x{>}1}&=(x{>}0)
\end{align*}
Let $\aPS_i\in\sem{\aCmd_i}$.
\begin{align*}
  \aTr[2]{\aEvs_2}{x{>}1}&=(r{+}1{>}1) = (r{>}0)
  \\
  \aTr[0]{\aEvs_0}{x{>}1}&=(0{=}\aReg \limplies r{>}0)
  \\
  \aTr[0]{\aEvs_0}{x{>}1}&=(1{=}\aReg \limplies r{>}0)
  \\
  \aTr[0]{\aEvs_0}{x{>}1}&=(2{=}\aReg \limplies r{>}0)
\end{align*}

\begin{proposition}
  If $\aPS\in\sem{\aCmd}$ is top-level and quiescent then 
  $\aTr{\aEvs}{\aForm}$ implies $\fwp{\aCmd}{\aForm}$.

  For any substitution $\aSub=[{\aVal_1/\aReg_1},\ldots, {\aVal_n/\aReg_n}]$ there is some
  $\aPS\in\sem{\aCmd}$ %that is top-level and quiescent
  such that all preconditions in $\aPS\aSub$ are tautologies then 
  $\fwp{\aCmd}{\aForm}\aSub$
\end{proposition}
\end{comment}

\section{Completed Pomsets and Fork-Join}
\label{sec:join}

[Todo.]
\begin{comment}
  \begin{definition}
    A pomset with predicate transformers $\aPS$ is \emph{completed} if, for
    every quiescence symbol $\aSym$, $\aTr{\aEvs}{\aSym}$ implies $\aSym$.
  \end{definition}


  It is also possible to put coherence in the independency relation, in which
  case, the semantics of $;$ includes the following.
  \begin{enumerate}
    \setcounter{enumi}{\value{pomsetXSemiCount}}
  \item
    \label{seq-reorder} if $\bEv\in\aEvs_1$ and $\aEv\in\aEvs_2$ either
    $\bEv<\aEv$ or $a\reorder\labeling_2(\aEv)$.
  \end{enumerate}
  One must be careful, however, due to \emph{inconsistency}.  Consider that
  \texttt{x=0;x=1} should not have completed pomset with only $\DWP{x}{0}$.

  \eqref{seq-reorder} does not do the right thing with fork either.  If you
  want to enforce coherence this way then you need to use fork-join as the
  sequential combinator, rather than fork.


  [We drop $\reorder$ because incompatible with $\sFORK{}$.  If you want to
  use $\reorder$, then you need to use fork-join as the sequential
  combinator, rather than fork.]

\begin{definition}
  A \emph{pomset with preconditions and termination} is a pomset with
  preconditions together with a predicate $\TICK$.
\end{definition}

% Define $\sTHREAD{}$ to transform a pomset with predicate transformers into
% a pomset with preconditions and termination by dropping the predicate
% transformer and setting $\TICK$ to indicate whether the pomset was
% completed.

% Extend the definition of $\sNIL$ so that $\TICK$ is true.

% Extend the definition of $\sPAR$ to handle for $\TICK$ by adding the
% following.
% \begin{enumerate}
%   \setcounter{enumi}{\value{pomsetPreParCount}}
% \item \label{par-tick} if $\TICK$ then $\TICK_1$ and $\TICK_2$.
% \end{enumerate}

% Similarly, $\sFORKJOIN{}$ extends $\sFORK{}$ by adding the following.
% % \noindent
% % If $\aPS \in \sFORKJOIN{\aPSS}$ then
% % $(\exists\aPS_1\in\aPSS)$
% \begin{enumerate}
%   \setcounter{enumi}{\value{pomsetXForkCount}}
% \item $\TICK_1$.
% \end{enumerate}

\begin{definition}$\phantom{\;}$\par
  % \noindent
  % If $\aPS\in\sNIL$ then $\aEvs = \emptyset$ and $\TICK$.

  \noindent
  If $\aPS \in (\aPSS_1\sPAR\aPSS_2)$ then $(\exists\aPS_1\in\aPSS_1)$
  $(\exists\aPS_2\in\aPSS_2)$
  \begin{enumerate}
    \setcounter{enumi}{\value{pomsetPreParCount}}
  \item[\ref{par-E}--\ref{par-kappa2})] as for $\sPAR$ in
    Definition~\ref{def:pomsets-pre},
  \item \label{par-tick} $\TICK$ implies $\TICK_1\land\TICK_2$.
  \end{enumerate}

  \noindent
  If $\aPS \in \sTHREAD{\aPSS}$ then $(\exists\aPS_1\in\aPSS)$
  \begin{enumerate}
  \item[1--3)] as for $\sTHREAD{}$ in \refdef{def:pomsets-group},
  \item[{\labeltext[T4]{T4)}{T4}}] if $\TICK$ then $\aTr{\aEvs}{\Q{}}$
    implies $\Q{}$.
  \end{enumerate}

  \noindent
  If $\aPS \in \sFORKJOIN{\aPSS}$ then $(\exists\aPS_1\in\aPSS)$
  \begin{enumerate}
  \item[1--4)] as for $\sFORK{}$ in \refdef{def:pomsets-group},
  \item[{\labeltext[F5]{F5)}{F5}}] $\TICK_1$.
  \end{enumerate}
\end{definition}

\begin{align*}
  \sem{\FORKJOIN{\aGrp}} &= \sFORKJOIN{}\sem{\aGrp}  
\end{align*}

We can then encode coherence as follows.
\begin{enumerate}
  \setcounter{enumi}{\value{pomsetXSemiCount}}
\item if $\bEv\in\aEvs_1$ and $\aEv\in\aEvs_2$ either $\bEv<\aEv$ or
  $a\reorder\labeling_2(\aEv)$.
\end{enumerate}


Access modes can be encoded in the independency relation, indexing labels by
$\amode$, but the extra flexibility of the logic is necessary for \armeight{}
(see \textsection\ref{sec:internal}).  Using independency, one would also
need another way to define completed pomsets.  Finally, this use of
independency is incompatible with fork (see \textsection\ref{sec:co}).


If we move coherence to independency (and use fork-join), we have the
following, assuming that each register occurs at most once.
\begin{align*}
  \QS{}{\mSC}&=\Q{\mSC}
  &\QS{}{\mRA}&=\Q{\mRA}
  &\QS{}{\mRLX}&=\Qx{\aLoc}
  \\
  \QL{}{\mSC}&=\Q{\mSC}
  &\QL{}{\mRA}&=\Qw{\aLoc}
  &\QL{}{\mRLX}&=\Qw{\aLoc}
  \\
  \DS{\aLoc}{\mSC}{\bForm}&=\bForm[\FALSE/\D]
  &\DS{\aLoc}{\mRA}{\bForm}&=\bForm[\FALSE/\D]
  &\DS{\aLoc}{\mRLX}{\bForm}&=\bForm[\TRUE/\Dx{\aLoc}] 
  \\
  \DL{\aLoc}{\mSC}&=\Dx{\aLoc}
  &\DL{\aLoc}{\mRA}&=\Dx{\aLoc}
  &\DL{\aLoc}{\mRLX}&=\TRUE
\end{align*}

% $\QS{}{\mRLX}=\TRUE$ and otherwise $\QS{}{\amode}=\Q{\amode}$.

% $\QL{}{\mSC}=\Q{\mSC}$ and otherwise $\QL{}{\amode}=\TRUE$.

% $\DS{\aLoc}{\mRLX}{\bForm}=\bForm[\TRUE/\Dx{\aLoc}]$ and otherwise
% $\DS{\aLoc}{\amode}{\bForm}=\bForm[\FALSE/\D]$.

% $\DL{\aLoc}{\mRLX}=\TRUE$ and otherwise $\DL{\aLoc}{\amode}=\Dx{\aLoc}$.

% \begin{definition}$\phantom{\;}$\par
%   $\QS{}{\mRLX}=\TRUE$ and otherwise $\QS{}{\amode}=\Q{\amode}$.

%   $\QL{}{\mSC}=\Q{\mSC}$ and otherwise $\QL{}{\amode}=\TRUE$.

\noindent
\begin{enumerate}
\item[\ref{S3})] $\labelingForm(\aEv)$ implies
  \begin{math}
    \aExp{=}\aVal \land \RW \land \QS{}{\amode}
  \end{math},
\item[\ref{S4})] $\aTr{\bEvs}{\bForm}$ implies
  \begin{math}
    \aExp{=}\aVal \land \DS{\aLoc}{\amode}{\bForm[\aExp/{\aLoc}]}
  \end{math},
\item[\ref{S5})] $\aTr{\emptyset}{\bForm}$ implies
  \begin{math}
    \lnot\Q{\mRA} \land \DS{\aLoc}{\amode}{\bForm[\aExp/{\aLoc}]}
  \end{math}
\end{enumerate}

\noindent
\begin{enumerate}
\item[\ref{L3})] $\labelingForm(\aEv)$ implies
  \begin{math}
    \RO \land \QL{}{\amode}
  \end{math},
\item[\ref{L4})] $\aTr{\bEvs}{\bForm}$ implies
  \begin{math}
    (\aVal{=}\aReg) \limplies \bForm[\aReg/{\aLoc}]
  \end{math}
\item[\ref{L5})] $\aTr{\emptyset}{\bForm}$ implies
  \begin{math}
    \DL{\aLoc}{\amode} \land \lnot\Q{\mRA} \land (\RW \limplies
    (\aVal{=}\aReg\lor\aLoc{=}\aReg) \limplies \bForm[\aReg/{\aLoc}] ).
  \end{math}
\end{enumerate}
\end{comment}

\section{Substitutions}

It is also possible to collapse $\aLoc$ and $\aReg$ in the semantics of
this paper:
\begin{enumerate}
\item[\ref{L4})]
  $\aTr{\bEvs}{\bForm}$ implies $\aVal{=}\aReg\limplies\bForm[\aReg/\aLoc]$, 
\item[\ref{L5})]
  $\aTr{\cEvs}{\bForm}$ implies
  $(\aVal{=}\aReg\lor\aLoc{=}\aReg)\limplies\bForm[\aReg/\aLoc]$. %, when $\aEvs\neq\emptyset$,
  % \item[\ref{L6})] 
  %   $\aTr{\dEvs}{\bForm}\;$ implies $\bForm$, when $\aEvs=\emptyset$.
\end{enumerate}
This semantics is incomparable to that the version without the substitution
of $[r/x]$.

\begin{example}
  Consider the following:
  \begin{gather*}
    \IF{r\land s\;\mathsf{even}}\THEN \PW{y}{1}\FI\SEMI
    \IF{r\land s}\THEN \PW{z}{1}\FI
    \\
    \hbox{\begin{tikzinline}[node distance=0.5em and 1.5em]
        \event{a3}{r\land s\;\mathsf{even}\mid\DW{y}{1}}{}
        \event{a4}{r\land s\mid\DW{z}{1}}{below=of a3}
      \end{tikzinline}}
  \end{gather*}
  Prepending $\PRP{x}{s}$, we get the same result regardless of whether we
  substitute $[s/x]$, since $x$ does not occur in either precondition.  Here
  we show the independent case:
  \begin{gather*}
    \PR{x}{s}\SEMI
    \IF{r\land s\;\mathsf{even}}\THEN \PW{y}{1}\FI\SEMI
    \IF{r\land s}\THEN \PW{z}{1}\FI
    \\
    \hbox{\begin{tikzinline}[node distance=0.5em and 1.5em]
        \event{a2}{\DR{x}{2}}{}
        \event{a3}{(2{=}s\lor x{=}s)\limplies (r\land s\;\mathsf{even})\mid\DW{y}{1}}{above right=of a2}
        \event{a4}{(2{=}s\lor x{=}s)\limplies (r\land s)\mid\DW{z}{1}}{below=of a3}
      \end{tikzinline}}
  \end{gather*}
  Prepending $\PRP{x}{r}$, we now get different results since the
  preconditions mention $x$.
  Without substitution:
  \begin{gather*}
    \PR{x}{r}\SEMI
    \PR{x}{s}\SEMI
    \IF{r\land s\;\mathsf{even}}\THEN \PW{y}{1}\FI\SEMI
    \IF{r\land s}\THEN \PW{z}{1}\FI
    \\
    \hbox{\begin{tikzinline}[node distance=0.5em and 1.5em]
        \event{a1}{\DR{x}{1}}{}
        \event{a2}{\DR{x}{2}}{below=of a1}
        \event{a3}{1{=}r\limplies  (2{=}s\lor x{=}s)\limplies (r\land s\;\mathsf{even})\mid\DW{y}{1}}{right=of a1}
        \event{a4}{1{=}r\limplies  (2{=}s\lor x{=}s)\limplies (r\land s)\mid\DW{z}{1}}{below=of a3}
        \po{a1}{a3}
        \po[out=-20,in=177]{a1}{a4}
      \end{tikzinline}}
  \end{gather*}
  Prepending $\PWP{x}{0}$, which substitutes $[0/x]$, the precondition of
  $\DWP{y}{1}$ becomes
  $(1{=}r\limplies (2{=}s\lor0{=}s)\limplies (r\land s\;\mathsf{even}))$,
  which is a tautology, whereas the precondition of $\DW{z}{1}$ becomes
  $(1{=}r\limplies(2{=}s\lor0{=}s)\limplies (r\land s))$,
  which is not.   In order to be top-level, $\DW{z}{1}$ must depend on
  $\DR{x}{2}$; in this case the precondition becomes
  $(1{=}r\limplies2{=}s\limplies (r\land s))$, which is a tautology.  
  \begin{gather*}
    % \PW{x}{0}\SEMI
    % \PR{x}{r}\SEMI
    % \PR{x}{s}\SEMI
    % \IF{r\land s\;\mathsf{even}}\THEN \PW{y}{1}\FI\SEMI
    % \IF{r\land s}\THEN \PW{z}{1}\FI
    % \\
    \hbox{\begin{tikzinline}[node distance=1.5em]
        \event{a0}{\DW{x}{0}}{}
        \event{a1}{\DR{x}{1}}{right=of a0}
        \event{a2}{\DR{x}{2}}{right=of a1}
        \event{a3}{\DW{y}{1}}{right=of a2}
        \event{a4}{\DW{z}{1}}{right=of a3}
        % \wk{a0}{a1}
        % \wk[out=-20,in=-160]{a0}{a2}
        \po[out=20,in=160]{a1}{a3}
        \po[out=20,in=160]{a1}{a4}
        \po[out=-20,in=-160]{a2}{a4}
      \end{tikzinline}}
  \end{gather*}
  The situation reverses with the substitution $[r/x]$:
  \begin{gather*}
    \PR{x}{r}\SEMI
    \PR{x}{s}\SEMI
    \IF{r\land s\;\mathsf{even}}\THEN \PW{y}{1}\FI\SEMI
    \IF{r\land s}\THEN \PW{z}{1}\FI
    \\
    \hbox{\begin{tikzinline}[node distance=0.5em and 1.5em]
        \event{a1}{\DR{x}{1}}{}
        \event{a2}{\DR{x}{2}}{below=of a1}
        \event{a3}{1{=}r\limplies  (2{=}s\lor r{=}s)\limplies (r\land s\;\mathsf{even})\mid\DW{y}{1}}{right=of a1}
        \event{a4}{1{=}r\limplies  (2{=}s\lor r{=}s)\limplies (r\land s)\mid\DW{z}{1}}{below=of a3}
        \po{a1}{a3}
        \po[out=-20,in=177]{a1}{a4}
      \end{tikzinline}}
  \end{gather*}
  Prepending $\PWP{x}{0}$:
  \begin{gather*}
    % \PW{x}{0}\SEMI
    % \PR{x}{r}\SEMI
    % \PR{x}{s}\SEMI
    % \IF{r\land s\;\mathsf{even}}\THEN \PW{y}{1}\FI\SEMI
    % \IF{r\land s}\THEN \PW{z}{1}\FI
    % \\
    \hbox{\begin{tikzinline}[node distance=1.5em]
        \event{a0}{\DW{x}{0}}{}
        \event{a1}{\DR{x}{1}}{right=of a0}
        \event{a2}{\DR{x}{2}}{right=of a1}
        \event{a3}{\DW{y}{1}}{right=of a2}
        \event{a4}{\DW{z}{1}}{right=of a3}
        % \wk{a0}{a1}
        % \wk[out=-20,in=-160]{a0}{a2}
        \po[out=20,in=160]{a1}{a3}
        \po[out=20,in=160]{a1}{a4}
        \po{a2}{a3}
      \end{tikzinline}}
  \end{gather*}
  The dependency has changed from $\DRP{x}{2}\xpo\DWP{z}{1}$ to
  $\DRP{x}{2}\xpo\DWP{y}{1}$.  The resulting sets of pomsets are
  incomparable.
\end{example}

Thinking in terms of hardware, the difference is whether reads update the
cache, thus clobbering preceding writes.  With $[r/x]$, reads clobber the
cache, whereas without the substitution, they do not.  Since most caches work
this way, the model with $[r/x]$ is likely preferred for modeling hardware.
In a software model, however, we see no reason to prefer one of these over
the other.


\begin{comment}
  if in L6 we said [x/r], that says we know read the local version...  ignoring
  the value read...  Perhaps there is some intervening stuff that stops you
  from seeing the local state, such as release-acquire

  We could potentially get rid of [x/r] If you do two reads, your not allowed
  to be independent of the second based on the value that was read in the first
  read.

  x=0; r=x; if (r=1) { s=x; if (s=?) {y=1}}
  read 1 then 2.


  In order for the write to be independent of second read what does its
  precondition have to be.
  [r/x] then s==1
  no sub then s==0

  (s=? | Wy1)

  if (phi) z=1
  phi = s is even
  phi = s < 2

  With substitution you are saying you know that the ``local copy'' of x is the
  same as r.  Sitting in the local cache.  Read might have gone to main
  memory, but if it did it has updated the cache line so that the local copy is
  what I just read.

  If second read is a cache hit, then I know that I am seeing the same value.

  If we take substitution out then 
\end{comment}


\section{Differences with OOPSLA}
\label{sec:diff}

\subsubsection*{Substitution}

\jjr{} uses substitution rather than Skolemizing.  Indeed our use of
Skolemization is motivated by disjunction closure for predicate transformers,
which do not appear in \jjr{}; see \textsection\ref{sec:pomsets-trans}.

In \textsection\ref{sec:tc1}, we give the semantics of load for nonempty
pomsets as:
\begin{enumerate}
\item[\ref{L4})]
  $\aTr{\bEvs}{\bForm}$ implies $\aVal{=}\aReg\limplies\bForm$, 
\item[\ref{L5})]
  $\aTr{\cEvs}{\bForm}$ implies
  $(\aVal{=}\aReg\lor\aLoc{=}\aReg)\limplies\bForm$. %, when $\aEvs\neq\emptyset$,
  % \item[\ref{L6})] 
  %   $\aTr{\dEvs}{\bForm}\;$ implies $\bForm$, when $\aEvs=\emptyset$.
\end{enumerate}
In \jjr{}, the definition is roughly as follows:
% (adding the case for $\ref{L6}$, which was missing):
\begin{enumerate}
\item[\ref{L4})]
  $\aTr{\bEvs}{\bForm}$ implies $\bForm[\aVal/\aReg][\aVal/\aLoc]$, 
\item[\ref{L5})]
  $\aTr{\cEvs}{\bForm}\;$ implies $\bForm[\aVal/\aReg][\aVal/\aLoc]\land\bForm[\aLoc/\aReg]$. %, when $\aEvs\neq\emptyset$,
  % \item[\ref{L6})]
  %   $\aTr{\dEvs}{\bForm}\;$ implies $(\forall\aReg)\bForm$, when $\aEvs=\emptyset$.
\end{enumerate}
These substitutions collapse $\aLoc$ and $\aReg$, allowing local invariant
reasoning, as in \textsection\ref{sec:tc1}.  Without Skolemizing it is
necessary to substitute $[\aLoc/\aReg]$, since the reverse substitution
$[\aReg/\aLoc]$ is useless when $\aReg$ is bound.

Removing the substitution of $[x/r]$ in the independent case has a small
technical advantage: we no longer require \emph{extended} expressions (which
include memory references), since substitutions no longer introduce memory
references.

\begin{example}
  The substitution $[x/r]$ does not work with Skolemization, even for the
  dependent case.  It forces the reads to the same values.
  \begin{enumerate}
  \item[\ref{L4})]
    $\aTr{\bEvs}{\bForm}$ implies $\aVal{=}\aLoc\limplies\bForm[\aLoc/\aReg]$, 
    % \item[\ref{L5})]
    %   $\aTr{\cEvs}{\bForm}$ implies
    %   $(\aVal{=}\aLoc\lor\TRUE)\limplies\bForm[\aLoc/\aReg]$. %, when $\aEvs\neq\emptyset$,
    % \item[\ref{L6})] 
    %   $\aTr{\dEvs}{\bForm}\;$ implies $\bForm$, when $\aEvs=\emptyset$.
  \end{enumerate}
  For example
  \begin{gather*}
    \PR{x}{r}\SEMI
    \PR{x}{s}\SEMI
    \IF{r{<}s}\THEN \PW{y}{1}\FI 
    \\
    \hbox{\begin{tikzinline}[node distance=0.5em and 1.5em]
        \event{a1}{\DR{x}{1}}{}
        \event{a2}{\DR{x}{2}}{right=of a1}
        \event{a3}{1{=}x\limplies 2{=}x\limplies x{<} x\mid\DW{y}{1}}{right=of a2}
        \po[out=20,in=160]{a1}{a3}
        \po{a2}{a3}
      \end{tikzinline}}
  \end{gather*}
\end{example}


% There, item \ref{loadpre-kappa2}  of $\sLOADPRE{}{}{}$ is written 
% \begin{enumerate}
% \item[] %[\ref{loadpre-kappa2})]
%   if $\aEv\in\aEvs_2\setminus\aEvs_1$ then either \\
%   $\labelingForm(\aEv)$ implies $\labelingForm_2(\aEv)[\aLoc/\aReg][\aVal/\aLoc]$ and $(\exists\bEv\in\aEvs_1)\bEv{<}\aEv$, or \\
%   $\labelingForm(\aEv)$ implies
%   $\labelingForm_2(\aEv)[\aLoc/\aReg][\aVal/\aLoc] \land \labelingForm_2(\aEv)[\aLoc/\aReg]$.
% \end{enumerate}


% [Skolemization ensures disjunction closure, which is necessary
% for associativity. Show example.]

\subsubsection*{Consistency}
\jjr{} imposes \emph{consistency} and \emph{causal strengthening}.  To see
that we must allow inconsistent preconditions, see \textsection\ref{sec:if}.

\begin{example}
  The lack of causal strengthening does not cause thin-air executions, as it
  would in \jjr{}, but the predicate transformers are quite delicate.
  
  Consider the following program, from \jjr{\textsection5}, where initially $x=0$, $y=0$, $\REF{0}=0$,
  $\REF{1}=2$, and $\REF{2}=1$.  It should only be possible to read $0$,
  disallowing the attempted execution below:
  \begin{gather*}
    \begin{gathered}
      r\GETS y\SEMI s\GETS \REF{r}\SEMI x\GETS s
      \PAR
      r\GETS x\SEMI s\GETS \REF{r}\SEMI y\GETS s
      \\
      \hbox{\begin{tikzinline}[node distance=1.5em]
          \event{a1}{\DR{y}{2}}{}
          \event{a2}{\DR{\REF{2}}{1}}{right=of a1}
          \event{a3}{\DW{x}{1}}{right=of a2}
          \po{a2}{a3}
          \po[out=10,in=170]{a1}{a3}
          \event{b1}{\DR{x}{1}}{right=3em of a3}
          \event{b2}{\DR{\REF{1}}{2}}{right=of b1}
          \event{b3}{\DW{y}{2}}{right=of b2}
          \po{b2}{b3}
          \po[out=10,in=170]{b1}{b3}
          \rf[out=-170,in=-10]{b3}{a1}
          \rf{a3}{b1}
        \end{tikzinline}}
    \end{gathered}
  \end{gather*}
  Using \refdef{def:pomsets-rr} with \refdef{def:pomsets-addr} and looking at the left thread:
  \begin{align*}
    \begin{gathered}[t]
      r\GETS y
      \\
      \hbox{\begin{tikzinline}[node distance=.5em and 1.5em]
          \event{a}{\DR{y}{2}}{}
          \xform{xa}{(2{=}r\lor\RW)\limplies\bForm}{below=of a}
        \end{tikzinline}}
    \end{gathered}
    &&
    \begin{gathered}[t]
      s\GETS \REF{r}
      \\
      \hbox{\begin{tikzinline}[node distance=.5em and 1.5em]
          \event{b}{r\EQ2\mid\DR{\REF{2}}{1}}{}
          \xform{xb}{(r\EQ2\limplies 1\EQ s) \limplies\bForm}{below=of b}
          \xo{b}{xb}
        \end{tikzinline}}
    \end{gathered}
    &&
    \begin{gathered}[t]
      x\GETS s
      \\
      \hbox{\begin{tikzinline}[node distance=.5em and 1.5em]
          \event{b}{s\EQ1\mid\DW{x}{1}}{}
        \end{tikzinline}}
    \end{gathered}
  \end{align*}
  Composing, we have:
  \begin{gather*}
    \begin{gathered}
      r\GETS y\SEMI s\GETS \REF{r}\SEMI x\GETS s
      \\
      \hbox{\begin{tikzinline}[node distance=.5em and 1.5em]
          \event{a1}{\DR{y}{2}}{}
          \event{a2}{(2{=}r\lor\RW)\limplies r\EQ2\mid\DR{\REF{2}}{1}}{right=of a1}
          \event{a3}{(2{=}r\lor\RW)\limplies (r\EQ2\limplies 1\EQ s)
            \limplies s\EQ1\mid\DW{x}{1}}{below right=.5em and -8em of a2}
          \po{a2}{a3}
        \end{tikzinline}}
    \end{gathered}
  \end{gather*}  
\end{example}

\subsubsection*{Parallel Composition}

In \jjr{\textsection2.4}, parallel composition is defined allowing coalescing
of events.  Here we have forbidden coalescing.  This difference appears to be
arbitrary.  In \jjr{}, however, there is a mistake in the handling of
termination actions.  The predicates should be joined using $\land$, not
$\lor$.

\subsubsection*{Internal Acquiring Reads}

Shortly after publication, \citet{anton} noticed a shortcoming of the
implementation on \armeight{} in \jjr{\textsection 7}.  The proof given there
assumes that all internal reads can be dropped.  However, this is not the
case for acquiring reds.  For example, \jjr{} disallows the following
execution, which is allowed by \armeight{} and \tso{}.
\begin{gather*}
  \PW{x}{2}\SEMI 
  \PR[\mRA]{x}{r}\SEMI
  \PR{y}{s} \PAR
  \PW{y}{2}\SEMI
  \PW[\mRA]{x}{1}
  \\
  \hbox{\begin{tikzinline}[node distance=1.5em]
      \event{a}{\DW{x}{2}}{}
      \raevent{b}{\DR[\mRA]{x}{2}}{right=of a}
      \event{c}{\DR{y}{0}}{right=of b}
      \event{d}{\DW{y}{2}}{right=2.5em of c}
      \raevent{e}{\DW[\mRA]{x}{1}}{right=of d}
      \rf{a}{b}
      \sync{b}{c}
      \wk{c}{d}
      \sync{d}{e}
      \wk[out=-165,in=-15]{e}{a}
      % \rfi{a}{b}
      % \bob{b}{c}
      % \fre{c}{d}
      % \bob{d}{e}
      % \coe[out=-165,in=-15]{e}{a}
    \end{tikzinline}}
\end{gather*}
The solution we have adopted is to allow an acquiring read to be downgraded
to a relaxed read when it is preceded (sequentially) by a relaxed write that
could fulfill it.  This solution allows executions that are not allowed under
\armeight{} since we do not insist that the local relaxed write is actually
read from.  This may seem counterintuitive, but we don't see a local way to
be more precise.

As a result, we use a different proof strategy for \armeight{}
implementation, which does not rely on read elimination.  The proof idea uses
a recent alternative characterization of \armeight{}
\citep{alglave-git-alternate,arm-reference-manual}. %,armed-cats}.

\subsubsection*{Redundant Read Elimination}

Contrary to the claim, redundant read elimination fails for \jjr{}.
We discussed redundant read elimination in \textsection\ref{sec:recycle}.
Consider JMM Causality Test Case 2, which we discussed there.
\begin{gather*}
  \PR{x}{r}\SEMI
  \PR{x}{s}\SEMI
  \IF{r{=}s}\THEN \PW{y}{1}\FI
  \PAR
  x\GETS y
  \\
  \hbox{\begin{tikzinline}[node distance=1.5em]
      \event{a1}{\DR{x}{1}}{}
      \event{a2}{\DR{x}{1}}{right=of a1}
      \event{a3}{\DW{y}{1}}{right=of a2}
      \event{b1}{\DR{y}{1}}{right=3em of a3}
      \event{b2}{\DW{x}{1}}{right=of b1}
      \rf{a3}{b1}
      \po{b1}{b2}
      \rf[out=169,in=11]{b2}{a2}
      \rf[out=169,in=11]{b2}{a1}
    \end{tikzinline}}
\end{gather*}
Under the semantics of \jjr{}, we have
\begin{gather*}
  \PR{x}{r}\SEMI
  \PR{x}{s}\SEMI
  \IF{r{=}s}\THEN \PW{y}{1}\FI
  \\
  \hbox{\begin{tikzinline}[node distance=1.5em]
      \event{a1}{\DR{x}{1}}{}
      \event{a2}{\DR{x}{1}}{right=of a1}
      \event{a3}{1\EQ1\land1\EQ x \land x\EQ1 \land x=x\mid\DW{y}{1}}{right=of a2}
    \end{tikzinline}}
\end{gather*}
The precondition of $\DWP{y}{1}$ is \emph{not} a tautology, and therefore
redundant read elimination fails.
(It is a tautology in
\begin{math}
  \PR{x}{r}\SEMI
  \LET{s}{r}\SEMI
  \IF{r{=}s}\THEN \PW{y}{1}\FI
\end{math}.)
In \jjr{\textsection3.1}, we incorrectly stated that the precondition of
$\DWP{y}{1}$ was $1\EQ1\land x\EQ x$.  

\begin{comment}
  Precondition of $\DWP{y}{1}$ is $(r{=}s)$ in
  \begin{math}
    \sem{\IF{r{=}s}\THEN y\GETS 1\FI}.
  \end{math}
  Predicate transformers for $\emptyset$ in $\sem{\PR{x}{r}}$ and $\sem{\PR{x}{s}}$ are
  \begin{align*}
    \PREDP{(r{=}1 \lor r{=}x)\limplies\bForm[r/x]},
    \\
    \PREDP{(s{=}1 \lor s{=}x)\limplies\bForm[s/x]}.
  \end{align*}
  Combining the transformers, we have
  \begin{displaymath}
    \PREDP{(r{=}1 \lor r{=}x)\limplies(s{=}1 \lor s{=}r)\limplies\bForm[s/x]}.
  \end{displaymath}
  Applying this to $(r{=}s)$, we have
  \begin{displaymath}
    \PREDP{(r{=}1 \lor r{=}x)\limplies (s{=}1 \lor s{=}r)\limplies (r{=}s)},
  \end{displaymath}
  which is not a tautology.

  Same problem occurs \jjr{}, where we have:
  \begin{align*}
    \PREDP{\bForm[v/x,r] \land \bForm[x/r]},
    \\
    \PREDP{\bForm[v/x,s] \land \bForm[x/s]}.
  \end{align*}
  Combining the transformers, we have
  \begin{displaymath}
    \PREDP{\bForm[v/x,r,s] \land \bForm [v/x,r][x/s] \land \bForm[x/r][v/x,s] \land \bForm[x/r,s]}.
  \end{displaymath}
  Applying this to $(r{=}s)$, we have
  \begin{displaymath}
    \PREDP{v{=}v \land v{=}x \land x{=}v \land x{=}x},
  \end{displaymath}
  which is not a tautology.

  The semantics here allows this by coalescing:
  \begin{gather*}
    r\GETS x\SEMI
    s\GETS x\SEMI
    \IF{r{=}s}\THEN y\GETS 1\FI
    \PAR
    x\GETS y
    \\
    \hbox{\begin{tikzinline}[node distance=1.5em]
        \event{a1}{\DR{x}{1}}{}
        \event{a3}{\DW{y}{1}}{right=of a1}
        \event{b1}{\DR{y}{1}}{right=3em of a3}
        \event{b2}{\DW{x}{1}}{right=of b1}
        \rf{a3}{b1}
        \po{b1}{b2}
        \rf[out=169,in=11]{b2}{a1}
      \end{tikzinline}}
  \end{gather*}

  In \jjr{\textsection2.6} the semantics of read is defined as follows:
  \begin{align*}
    \sem{\aReg\GETS\aLoc^\amode\SEMI \aCmd} & \eqdef \textstyle\bigcup_\aVal\;
    (\DRmode\aLoc\aVal) \prefix \sem{\aCmd} [\aLoc/\aReg]
  \end{align*}
  The definition of prefixing$((\aForm \mid \aAct) \prefix \aPSS)$ has several clauses.
  The most relevant are as follows, where $\bEv$ is the new event labeled with
  $(\aForm \mid \aAct)$ and $\aEv$ is an event from $\aPSS$:
  \begin{description}
  \item[{\labeltextsc[P4c]{(P4c)}{4c}}]
    If $\bEv$ reads $\aVal$ from $\aLoc$ then either $\aEv=\bEv$ or
    $\labelingForm'(\aEv)$ implies $\labelingForm(\aEv)[\aVal/\aLoc]$.
  \item[{\labeltextsc[P5a]{(P5a)}{5a}}]\labeltextsc[P5]{}{5}%
    If $\bEv$ reads and $\aEv$ writes then either $\labelingForm'(\aEv)$
    implies $\labelingForm(\aEv)$ or $\bEv\le'\aEv$.
    % \item[{\labeltextsc[P5b]{(P5b)}{5b}}]
    %   If $\bEv$ and $\aEv$ are in conflict then $\bEv\le'\aEv$.
  \end{description}

  We have discovered two issues with this definition.

  The first issue concerns the substitution $[\aLoc/\aReg]$.  It should be
  $[\aReg/\aLoc]$.  We noticed this error while developing the alternative
  characterization presented here.  The error causes redundant read elimination
  to fail in \jjr{}.  As a result, common subexpression elimination also fails.
  The problem can be seen in \ref{TC2}.
  \begin{gather*}
    \taglabel{TC2}
    r\GETS x\SEMI
    s\GETS x\SEMI
    \IF{r{=}s}\THEN y\GETS 1\FI
    \PAR
    x\GETS y
  \end{gather*}
  % In \jjr{\textsection3.1},
  We claimed that \ref{TC2} allowed the following
  execution:
  \begin{gather*}
    \hbox{\begin{tikzinline}[node distance=1.5em]
        \event{a1}{\DR{x}{1}}{}
        \event{a2}{\DR{x}{1}}{right=of a1}
        \event{a3}{\DW{y}{1}}{right=of a2}
        % \po{a2}{a3}
        % \po[out=15,in=165]{a1}{a3}
        \event{b1}{\DR{y}{1}}{right=3em of a3}
        \event{b2}{\DW{x}{1}}{right=of b1}
        \rf{a3}{b1}
        \po{b1}{b2}
        \rf[out=169,in=11]{b2}{a2}
        \rf[out=169,in=11]{b2}{a1}
      \end{tikzinline}}
  \end{gather*}
  But this execution is not possible using the semantics of \jjr{}:
  $\DWP{y}{1}$ has precondition $r{=}s$ in
  \begin{math}
    \sem{\IF{r{=}s}\THEN y\GETS 1\FI}.
  \end{math}
  Given the lack of order in the execution, the precondition of $\DWP{y}{1}$
  must entail $r{=}1\land r{=}x$ in 
  \begin{math}
    \sem{s\GETS x\SEMI
      \IF{r{=}s}\THEN y\GETS 1\FI}.
  \end{math}
  \ref{4c} imposes $r{=}1$, and \ref{5a} imposes $r{=}x$.  Adding the second
  read, the precondition of $\DWP{y}{1}$ must entail both $1{=}1\land 1{=}x$
  and also $x{=}1\land x{=}x$.  This can be simplified to $x{=}1$.  This leaves
  a requirement that must be satisfied by a preceding write.  Since the
  preceding write is the initialization to $0$, the requirement cannot be
  satisfied, and the execution is impossible.\footnote{In \jjr{} we ignore the
    middle terms, mistakenly simplifying this to $1{=}1\land x{=}x$.
    Correcting the error, the attempted execution is:
    \begin{gather*}
      \hbox{\begin{tikzinline}[node distance=1.5em]
          \event{a1}{\DR{x}{1}}{}
          \event{a2}{\DR{x}{1}}{right=of a1}
          \event{a3}{\DW{y}{1}}{right=of a2}
          \po{a2}{a3}
          \po[out=-20,in=-160]{a1}{a3}
          \event{b1}{\DR{y}{1}}{right=3em of a3}
          \event{b2}{\DW{x}{1}}{right=of b1}
          \rf{a3}{b1}
          \po{b1}{b2}
          \rf[out=169,in=11]{b2}{a2}
          \rf[out=169,in=11]{b2}{a1}
        \end{tikzinline}}
    \end{gather*}}

  The substitution $[\aLoc/\aReg]$ leaves the obligation on $\aLoc$ to be
  fulfilled by the preceding write.  Thus, the read does not update the
  \emph{value} of $\aLoc$ in subsequent predicates.  The substitution
  $[\aReg/\aLoc]$, instead, does update the value of $\aLoc$, thus removing any
  obligation on $\aLoc$ for preceding code.

  In order to write this, we must update the definition of prefixing reads to
  include the register.  Then \ref{4c} becomes:
  \begin{description}
  \item[\textsc{(p4c)}] If $\bEv$ reads $\aVal$ from $\aLoc$ then either
    $\aEv=\bEv$ or $\labelingForm'(\aEv)$ implies
    $\labelingForm(\aEv)[\aVal/\aReg]$.
  \end{description}

  We can then reason with \ref{TC2} as follows: $\DWP{y}{1}$ has precondition
  $r{=}s$ in
  \begin{math}
    \sem{\IF{r{=}s}\THEN y\GETS 1\FI}.
  \end{math}
  To avoid introducing order in the execution, the precondition of $\DWP{y}{1}$
  must entail $r{=}1\land r{=}s$ in 
  \begin{math}
    \sem{s\GETS x\SEMI
      \IF{r{=}s}\THEN y\GETS 1\FI}.
  \end{math}
  \ref{4c} imposes $r{=}1$, and \ref{5a} imposes $r{=}x$.  Adding the second
  read, the precondition of $\DWP{y}{1}$ must entail both $1{=}1\land 1{=}x$
  and also $x{=}1\land x{=}x$.  This can be simplified to $x{=}1$.  This leaves
  a requirement that must be satisfied by a preceding write.


  With read elimination, the rule for relaxed reads is as follows:
  \begin{align*}
    \sem{\PR{\aLoc}{\aReg} \SEMI \aCmd} &\eqdef
    \sem{\aCmd}[\aLoc/\aReg]
    \cup
    \textstyle\bigcup_\aVal\;
    \DRP{\aLoc}{\aVal} \prefix_{\aReg} %\Rdis{\aLoc}{\aVal}
    \sem{\aCmd}[\aReg/\aLoc]
  \end{align*}
  It is interesting to note that the substitution is $[\aLoc/\aReg]$ on
  eliminated reads, and $[\aReg/\aLoc]$ on non-eliminated reads.  Intuitively,
  the subsequent value of $\aLoc$ is fixed by an explicit read, but not for an
  eliminated read.  In the latter case, the value is fixed by some preceding
  action.  The preceding action may itself be a read. This gives rise to some
  fear that we might introduce thin-air reads, since we do not enforce
  read-read coherence.  But this is not the case.  Consider the following example:
  \begin{gather*}
    r\GETS x\SEMI
    s\GETS x\SEMI
    y\GETS s
    \PAR
    x\GETS y
    \\
    \hbox{\begin{tikzinline}[node distance=1.5em]
        \event{a1}{\DR{x}{1}}{}
        \event{a2}{\DR{x}{1}}{right=of a1}
        \event{a3}{\DW{y}{1}}{right=of a2}
        % \po{a2}{a3}
        \po[out=-20,in=-160]{a1}{a3}
        \event{b1}{\DR{y}{1}}{right=3em of a3}
        \event{b2}{\DW{x}{1}}{right=of b1}
        \rf{a3}{b1}
        \po{b1}{b2}
        \rf[out=169,in=11]{b2}{a2}
        \rf[out=169,in=11]{b2}{a1}
      \end{tikzinline}}
    \\
    \hbox{\begin{tikzinline}[node distance=1.5em]
        \event{a1}{\DR{x}{1}}{}
        \internal{a2}{\DR{x}{1}}{right=of a1}
        \event{a3}{\DW{y}{1}}{right=of a2}
        % \po{a2}{a3}
        \po[out=-20,in=-160]{a1}{a3}
        \event{b1}{\DR{y}{1}}{right=3em of a3}
        \event{b2}{\DW{x}{1}}{right=of b1}
        \rf{a3}{b1}
        \po{b1}{b2}
        % \rf[out=169,in=11]{b2}{a2}
        \rf[out=169,in=11]{b2}{a1}
      \end{tikzinline}}
  \end{gather*}
  But this is not a problem, since fulfillment requires that $\DWP{x}{1}$
  precede both reads of $x$.
\end{comment}

\subsubsection*{Triangular Races}

The notion of data-race for local data-race freedom is incorrect in \jjr{}.
\begin{gather*}
  \PW{x}{1}\SEMI
  \PW[\mRA]{y}{1}\SEMI
  \PR[\mRA]{x}{r}
  \PAR
  \IF{\PR[\mRA]{y}{}}\THEN \PW[\mRA]{x}{2}\FI
  \\
  \hbox{\begin{tikzinline}[node distance=1.5em]
      \event{a1}{\DW{x}{1}}{}
      \raevent{a2}{\DW[\mRA]{y}{1}}{right=of a1}
      \raevent{a3}{\DR[\mRA]{x}{1}}{right=of a2}
      \raevent{b1}{\DR[\mRA]{y}{1}}{right=3em of a3}
      \raevent{b2}{\DW[\mRA]{x}{2}}{right=of b1}
      \sync{a1}{a2}
      \rf[out=20,in=160]{a1}{a3}
      \rf[out=20,in=160]{a2}{b1}
      \sync{b1}{b2}
    \end{tikzinline}}
\end{gather*}
Consider the prefix with everything but the read of $x$.  When extending this
prefix as above, the read sees a stale value.  The local DRF theorem requires
that whenever such a stale read is possible there is an SC execution that can
be derived simply by changing the read, and further, that this SC execution
has a data-race. In this example, there is an SC execution, but it does not
exhibit a data-race by the definition of \jjr{}.
\begin{gather*}
  \hbox{\begin{tikzinline}[node distance=1.5em]
      \event{a1}{\DW{x}{1}}{}
      \raevent{a2}{\DW[\mRA]{y}{1}}{right=of a1}
      \raevent{a3}{\DR[\mRA]{x}{2}}{right=of a2}
      \raevent{b1}{\DR[\mRA]{y}{1}}{right=3em of a3}
      \raevent{b2}{\DW[\mRA]{x}{2}}{right=of b1}
      \sync{a1}{a2}
      \rf[out=20,in=160]{a2}{b1}
      \rf[out=160,in=20]{b2}{a3}
      \sync{b1}{b2}
    \end{tikzinline}}
\end{gather*}
This error is inherited from \citep[Lemma A.4]{DBLP:conf/ppopp/DongolJR19},
which assumes that $\DRP[\mRA]{x}{1}$ and $\DWP[\mRA]{x}{2}$ are racing in
the first execution because they are not ordered by happens-before.  But this
is false since neither is plain.

We believe that the fix to consider a broader notion of data race when
stating local DRF with mixed-mode access.


