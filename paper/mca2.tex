%\subsection{\PwTmcaTITLE{2}: Pomsets with Predicate Transformers for MCA (Part 2)}
\subsection{\PwTmcaTITLE{2}}
\label{sec:mca2}

Lowering \PwTmcaTITLE{1} to \armeight{} requires a full fence after every
acquiring read.  To see why, consider the following attempted
execution, where the final values of both $x$ and $y$ are $2$.
\begin{gather*}
  \PW{x}{2}\SEMI 
  \PR[\mACQ]{x}{r}\SEMI
  \PW{y}{r{-}1} \PAR
  \PW{y}{2}\SEMI
  \PW[\mREL]{x}{1}
  \\
  \hbox{\begin{tikzinline}[node distance=1.5em]
      \event{a}{\DW{x}{2}}{}
      \raevent{b}{\DR[\mACQ]{x}{2}}{right=of a}
      \event{c}{\DW{y}{1}}{right=of b}
      \event{d}{\DW{y}{2}}{right=2.5em of c}
      \raevent{e}{\DW[\mREL]{x}{1}}{right=of d}
      \rfint{a}{b}
      \sync{b}{c}
      \wk{c}{d}
      \sync{d}{e}
      \wk[out=-165,in=-15]{e}{a}
    \end{tikzinline}}
\end{gather*}
The execution is  allowed by \armeight, but disallowed by \PwTmca{1}, due to
the cycle.

\armeight{} allows the execution because the read of $x$ is internal to the
thread.  This aspect of \armeight{} semantics is difficult to model locally.
To capture this, we found it necessary to drop \ref{pom-rf-le} and relax
\ref{seq-le-delays}, %from
% \refdef{def:pwt:mca1},
adding local constraints on $\rrfx$ to $\sLPAR{}{}$, $\sSEMI{}{}$ and
$\sIF{}{}{}$.
Rather than ensuring that there is no
\emph{global} blocker for a sequentially fulfilled read \eqref{pom-rf-le}, we
require only that there is no \emph{thread-local} blocker
\eqref{seq-le-rf-rf}.
For \PwTmca{2}, internal reads don't necessarily contribute to order, and
thus the above execution is allowed.

\begin{definition}
  \label{def:pwt:mca2}
  A \PwTmca{2} is a \PwT{} (\refdef{def:pomset}) equipped with an injective
  relation $\rrfx$ that satisfies requirements \ref{pom-rf-match} and
  \ref{pom-rf-block} of \refdef{def:pwt:mca1}.

  A \PwTmca{2} is \emph{complete} if it satisfies
  \ref{top-kappa}, \ref{top-term}, and \ref{top-rf}---this is the same as for
  \PwTmca{1}. 

  \input{fig-mca2.tex}
\end{definition}
% We write $\semmca{2}{}$ for the semantic function.
A \PwTmca{2} need not satisfy requirement \ref{pom-rf-le}, and thus we may
have $\bEv\xrfx\aEv$ and $\aEv\le\bEv$.

% 
% The use of $\rextendsdef{}{}$ in \ref{if-rf-extends}
% ensures that no $\rrfx$ is introduced between events in
% $\aEvs_1\cap\aEvs_2$ when coalescing.

% In the second semantics, we weaken the relationship between $\rrfx$
% and $\le$ in \ref{pom-rf-le}.  
% This change both allows and requires us to weaken the definition of
% \emph{delays} to drop write-to-read order from $\eqreorderco$.

\todo{Example using \ref{seq-le-delays-rf} and \ref{seq-le-rf-rf}. To make
  space, \reflem{lem:mca2} could move to appendix.}

With the weakening of \ref{seq-le-delays}, we must be careful not to allow
spurious pairs to be added to the $\rrfx$ relation.  Thus we add
\ref{par-rf-extends},
\ref{seq-rf-extends}, and
\ref{if-rf-extends}.
For example, \ref{if-rf-extends} ensure that
\begin{math}
  \frf{\semrr{\IF{b}\THEN\PR{x}{r}\PAR\PW{x}{1}\ELSE\PR{x}{r}\SEMI\PW{x}{1}\FI}}
\end{math}
does not include 
\begin{math}
  \smash{\hbox{\begin{tikzinlinesmall}[node distance=1.5em]
        \event{a}{\DR{x}{1}}{}
        \event{b}{\DW{x}{1}}{right=of a}
        \rfint[out=165,in=15]{b}{a}
        \wki{a}{b}
      \end{tikzinlinesmall}}}
\end{math}, taking $\rrfx$ from the left and $\le$ from the right.

% \PwTmca{2} does not enforce \ref{pom-rf-le}: $\bEv\xrfx\aEv$ may not imply
% $\bEv\le\aEv$ when $\bEv$ and $\aEv$ come from different sides of a
% sequential composition.  This means that
As a consequence of dropping \ref{pom-rf-le}, sequential $\rrfx$ must be validated during
pomset construction, rather than post-hoc.  In \textsection\ref{sec:c11}, we
show how to construct program order ($\rpox$) for complete pomsets using
phantom events ($\fmrg{}$).  Using this construction, the following lemma
gives a post-hoc verification technique for $\rrfx$.
\begin{lemma}
  \label{lem:mca2}
  If $\aPS\in\frf{\semmca{2}{\aCmd}}$ is complete, then
  for every $\bEv\xrfx\aEv$ either
  \begin{itemize}
  \item external fulfillment:
    $\bEv\le\aEv$ and if $\labeling(\cEv) \rblocks\labeling(\aEv)$ then either $\cEv\le\bEv$ or $\aEv\le\cEv$, or
  \item internal fulfillment:
    $(\exists\bEv'\in\fmrginv{\bEv})$
    $(\exists\aEv'\in\fmrginv{\aEv})$
    $\bEv'\xpox\aEv'$ and $(\not\exists\cEv')$
    $\labelingForm(\cEv)$ is a tautology and
    $\labeling(\cEv) \rblocks \labeling(\aEv)$ and $\bEv'\xpox\cEv\xpox\aEv'$.
  \end{itemize}
\end{lemma}
These mimic the \emph{external consistency} requirements of \armeight{}
\cite{armed}.

% \begin{enumerate}[,label=(\textsc{m}\arabic*),ref=\textsc{m}\arabic*]
  %   \setcounter{enumi}{\value{Brf}}
  % \item \label{pom-rf'} ${\rrfx} \subseteq \aEvs\times\aEvs$
  %   is an injective relation capturing \emph{reads-from}, such that
  %   % \end{enumerate}
  %   % A pomset is a \emph{candidate} if there is an injective relation
  %   % ${\rrfx} : \aEvs\times\aEvs$, capturing \emph{reads-from}, such that
  %   \begin{multicols}{2}    
  %     \begin{enumerate}
  %       % \begin{enumerate}[,label=(\textsc{i}\arabic*),ref=\textsc{i}\arabic*]
  %       % \item \label{rf-injective}
  %       %   if $\bEv\xrfx\aEv$ and $\cEv\xrfx\aEv$ then $\bEv=\cEv$, that is,
  %       %   ${\rrfx}$ is injective,
  %     \item[\eqref{pom-rf-match}]
  %       \eqref{pom-rf-block}\;
  %       as in \refdef{def:pwt:mca1},
  %       \setcounter{enumii}{2}
  %     \item \label{pom-rf-le'} 
  %       if $\bEv\xrfx\aEv$ then either $\bEv\le\aEv$ or $\aEv\le\bEv$.
  %     \end{enumerate}
  %   \end{multicols}
  % \end{enumerate}

% Derive $\rdelayspdef$ from $\rdelaysdef$ by replacing $\eqreorderco$ with
% \begin{math}
%   {\reorderlws}
%   =
%   \{(\DW{\aLoc}{}, \DW{\aLoc}{}),\;(\DR{\aLoc}{}, \DW{\aLoc}{})\}.
% \end{math}
% The acronym $\textsf{lws}$ is adopted from \armeight.  It stands for
% \emph{Local Write Successor}.

%The semantic rules are given in \reffig{fig:mca2}.
