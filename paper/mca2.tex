%\subsection{\PwTmcaTITLE{2}: Pomsets with Predicate Transformers for MCA (Part 2)}
\subsection{\PwTmcaTITLE{2}}
\label{sec:mca2}

%
The use of $\rextendsdef{}{}$ in \ref{if-rf-extends}
ensures that no $\rrfx$ is introduced between events in
$\aEvs_1\cap\aEvs_2$ when coalescing.

In the second semantics, we weaken the relationship between $\rrfx$
and $\le$ in \ref{pom-rf-le}.  Rather than ensuring that there is no
\emph{global} blocker for a sequentially fulfilled read \eqref{pom-rf-le}, we
require only that there is no \emph{thread-local} blocker
\eqref{seq-rf-le-rf}.
% This change both allows and requires us to weaken the definition of
% \emph{delays} to drop write-to-read order from $\eqreorderco$.

With the weakening of \ref{seq-rf-le-rf}, we must be careful not to allow
spurious pairs to be added to the $\rrfx$ relation.  Thus we add
\ref{par-rf-extends},
\ref{seq-rf-extends}, and
\ref{if-rf-extends}.
For example, \ref{if-rf-extends} ensure that
\begin{math}
  \frf{\semrr{\IF{b}\THEN\PR{x}{r}\PAR\PW{x}{1}\ELSE\PR{x}{r}\SEMI\PW{x}{1}\FI}}
\end{math}
does not include 
\begin{math}
  \smash{\hbox{\begin{tikzinlinesmall}[node distance=1.5em]
        \event{a}{\DR{x}{1}}{}
        \event{b}{\DW{x}{1}}{right=of a}
        \rfint[out=165,in=15]{b}{a}
        \wki{a}{b}
      \end{tikzinlinesmall}}}
\end{math}, taking $\rrfx$ from the left and $\le$ from the right.

\begin{definition}
  \label{def:pwt:mca2}
  A \PwTmca{2} is a \PwT{} (\refdef{def:pomset}) equipped with an injective
  relation $\rrfx$ that satisfies requirements \ref{pom-rf-match} and
  \ref{pom-rf-block} of \refdef{def:pwt:mca1}.

  A A \PwTmca{2} is \emph{complete} if it satisfies
  \ref{top-kappa}, \ref{top-term}, and \ref{top-rf}.
  % \begin{enumerate}[,label=(\textsc{m}\arabic*),ref=\textsc{m}\arabic*]
  %   \setcounter{enumi}{\value{Brf}}
  % \item \label{pom-rf'} ${\rrfx} \subseteq \aEvs\times\aEvs$
  %   is an injective relation capturing \emph{reads-from}, such that
  %   % \end{enumerate}
  %   % A pomset is a \emph{candidate} if there is an injective relation
  %   % ${\rrfx} : \aEvs\times\aEvs$, capturing \emph{reads-from}, such that
  %   \begin{multicols}{2}    
  %     \begin{enumerate}
  %       % \begin{enumerate}[,label=(\textsc{i}\arabic*),ref=\textsc{i}\arabic*]
  %       % \item \label{rf-injective}
  %       %   if $\bEv\xrfx\aEv$ and $\cEv\xrfx\aEv$ then $\bEv=\cEv$, that is,
  %       %   ${\rrfx}$ is injective,
  %     \item[\eqref{pom-rf-match}]
  %       \eqref{pom-rf-block}\;
  %       as in \refdef{def:pwt:mca1},
  %       \setcounter{enumii}{2}
  %     \item \label{pom-rf-le'} 
  %       if $\bEv\xrfx\aEv$ then either $\bEv\le\aEv$ or $\aEv\le\bEv$.
  %     \end{enumerate}
  %   \end{multicols}
  % \end{enumerate}
\end{definition}
A \PwTmca{2} need not satisfy requirement \ref{pom-rf-le}, and thus we may
have $\bEv\xrfx\aEv$ and $\aEv\le\bEv$.

% Derive $\rdelayspdef$ from $\rdelaysdef$ by replacing $\eqreorderco$ with
% \begin{math}
%   {\reorderlws}
%   =
%   \{(\DW{\aLoc}{}, \DW{\aLoc}{}),\;(\DR{\aLoc}{}, \DW{\aLoc}{})\}.
% \end{math}
% The acronym $\textsf{lws}$ is adopted from \armeight.  It stands for
% \emph{Local Write Successor}.

The semantic rules are given in \reffig{fig:mca2}.
We write $\semmca{2}{}$ for the semantic function.
