%\subsection{\PwTmcaTITLE{2}: Pomsets with Predicate Transformers for MCA (Part 2)}
\subsection{\PwTmcaTITLE{2}}
\label{sec:mca2}

Lowering \PwTmcaTITLE{1} to \armeight{} requires a full fence after every
acquiring read.  To see why, consider the following attempted
execution, where the final values of both $x$ and $y$ are $2$.
\begin{gather*}
  \PW{x}{2}\SEMI 
  \PR[\mACQ]{x}{r}\SEMI
  \PW{y}{r{-}1} \PAR
  \PW{y}{2}\SEMI
  \PW[\mREL]{x}{1}
  \\
  \hbox{\begin{tikzinline}[node distance=1.5em]
      \event{a}{\DW{x}{2}}{}
      \raevent{b}{\DR[\mACQ]{x}{2}}{right=of a}
      \event{c}{\DW{y}{1}}{right=of b}
      \event{d}{\DW{y}{2}}{right=2.5em of c}
      \raevent{e}{\DW[\mREL]{x}{1}}{right=of d}
      \rfint{a}{b}
      \sync{b}{c}
      \wk{c}{d}
      \sync{d}{e}
      \wk[out=-165,in=-15]{e}{a}
    \end{tikzinline}}
\end{gather*}
The execution is  allowed by \armeight, but disallowed by \PwTmca{1}, due to
the cycle.

\armeight{} allows the execution because the read of $x$ is internal to the
thread.  This aspect of \armeight{} semantics is difficult to model locally.
To capture this, we found it necessary to drop \ref{pom-rf-le} and relax
\ref{seq-le-delays}, %from
% \refdef{def:pwt:mca1},
adding local constraints on $\rrfx$ to $\sLPAR{}{}$, $\sSEMI{}{}$ and
$\sIF{}{}{}$.
Rather than ensuring that there is no
\emph{global} blocker for a sequentially fulfilled read \eqref{pom-rf-le}, we
require only that there is no \emph{thread-local} blocker
\eqref{seq-rf-le-rf}.
For \PwTmca{2}, internal reads don't necessarily contribute to order, and
thus the above execution is allowed.

\begin{definition}
  \label{def:pwt:mca2}
  A \PwTmca{2} is a \PwT{} (\refdef{def:pomset}) equipped with an injective
  relation $\rrfx$ that satisfies requirements \ref{pom-rf-match} and
  \ref{pom-rf-block} of \refdef{def:pwt:mca1}.

  A A \PwTmca{2} is \emph{complete} if it satisfies
  \ref{top-kappa}, \ref{top-term}, and \ref{top-rf}.

  % \begin{figure}
%   \bookmark{\PwTmcaTITLE{2} Semantics}
%   \raggedright
  % \noindent
  % Suppose $\rrfx_i$ is a relation in $\aEvs_i\times\aEvs_i$. % and $\rrfx$ is a relation over $\aEvs\supseteq\aEvs_i$.  %\subseteq\aEvs_i\times\aEvs_i$. % and $\rrfx_2\subseteq\aEvs_2\times\aEvs_2$.
  % We say ${\rrfx} \rextendsdef{\rrfx_i}{}$ %{\rrfx_2}$
  % if
  % % $\rrfx\supseteq (\rrfx_i\cup \rrfx_2)$ and
  % ${\rrfx}\cap(\aEvs_i\times \aEvs_i) = {\rrfx_i}$.
  % %and $\rrfx\cap(\aEvs_2\times \aEvs_2) = \rrfx_2$.
  % \smallskip
  
  % \noindent
  % If $\aPS\in\sSKIP$ then $\aEvs = \emptyset$ and
  % $\aTr{\bEvs}{\bForm} \riff \bForm$.
  % \smallskip

  \noindent
  If $\aPS \in \sLPAR{\aPSS_1}{\aPSS_2}$ then  
  $(\exists\aPS_1\in\aPSS_1)$ $(\exists\aPS_2\in\aPSS_2)$
  \begin{multicols}{2}
    \begin{enumerate}[topsep=0pt,label=(\textsc{p}\arabic*),ref=\textsc{p}\arabic*]
    \item[\eqref{par-E}]
      \eqref{par-lambda}\,
      \eqref{par-kappa}\,
      \eqref{par-tau}\, 
      \eqref{par-term}\, 
      \eqref{par-le}\,
      \eqref{par-rf}
      as in
      \refdef{def:pwt:mca1},
      %\reffig{fig:mca1},
      \setcounter{enumi}{\value{le}}
    % \item \label{par-E}
    %   $\aEvs = (\aEvs_1\uplus\aEvs_2)$,
    % \item \label{par-lambda}
    %   ${\labeling}=\PBR{{\labeling_1}\cup {\labeling_2}}$, 
    %   \stepcounter{enumi}
    % \item[] \labeltext[\textsc{p}3]{}{par-kappa}
    %   \begin{enumerate}[leftmargin=0pt]
    %   \item \label{par-kappa1}
    %     if $\aEv\in\aEvs_1$ then $\labelingForm(\aEv) \riff \labelingForm_1(\aEv)$,
    %   \item \label{par-kappa2}
    %     if $\aEv\in\aEvs_2$ then $\labelingForm(\aEv) \riff \labelingForm_2(\aEv)$,
    %   \end{enumerate}
    % \item \label{par-tau}
    %   $\aTr{\bEvs}{\bForm} \riff \aTr[1]{\bEvs}{\bForm}$,
    % \item \label{par-term}
    %   $\aTerm \riff \aTerm[1]\land\aTerm[2]$,
    % \item \label{par-le}
    %     ${\le}\rextends{\le_1}{\le_2}$,
    \item[] \labeltext[\textsc{p}6]{}{par-rf-le}
      \begin{enumerate}[leftmargin=0pt]
        % \label{par-le-extends}
        % ${\le}\rextends{\le_1}{\le_2}$,
      \item \label{par-rf-le1}
        % ${\rrfx}\subseteq\PBR{{\rrfx_1}\cup{\rrfx_2}\cup{\le}}$.
        % if $\bEv\xrfx\aEv$ then either $\bEv\xrfx_1\aEv$, $\bEv\xrfx_2\aEv$, or $\bEv\le\aEv$.
        if $\bEv\in\aEvs_1$, $\aEv\in\aEvs_2$ and $\bEv\xrfx\aEv$ then $\bEv\le\aEv$,
      \item \label{par-rf-le2}
        if $\bEv\in\aEvs_1$, $\aEv\in\aEvs_2$ and $\aEv\xrfx\bEv$ then $\aEv\le\bEv$,
      \end{enumerate}
      \stepcounter{enumi}
    \item[] \labeltext[\textsc{p}7]{}{par-rf2}
      \begin{enumerate}[leftmargin=0pt]
      \item
        \label{par-rf-extends}
        ${\rrfx_i}={\rrfx}\cap(\aEvs_i\times \aEvs_i)$, for $i\in\{1,2\}$.
        %${\rrfx}\rextends{\rrfx_1}{\rrfx_2}$.
      \end{enumerate}
    \end{enumerate}
  \end{multicols}
  \smallskip

  \noindent
  If $\aPS \in \sSEMI{\aPSS_1}{\aPSS_2}$ then $(\exists\aPS_1\in\aPSS_1)$
  $(\exists\aPS_2\in\aPSS_2)$
  % let $\aTerm[1](\aEv) = \aTerm[1]$ if $\labeling_2(\aEv)$ is a $\srelease$
  % and $\aTerm[1](\aEv) = \TRUE$ otherwise,\\
  % let $\labelingForm'_2(\aEv)=\aTr[1]{\Cdown{\aEv}}{\labelingForm_2(\aEv})$,
  % where
  % $\Cdown{\aEv}=\{ \cEv \mid \cEv \lt \aEv \}$%, if $\labeling(\aEv)$ is a write, and $\Cdown{\aEv}=\aEvs_1$, otherwise
  %
  % \begin{math}
  %   \Cdown{\aEv}=
  %   \begin{cases}
  %     \{ \cEv \mid \cEv \lt \aEv \} & \textif \labeling(\aEv) \;\text{is a write}
  %     \\
  %     \aEvs_1 & \textotherwise
  %   \end{cases}
  % \end{math}
  % let
  % \begin{math}
  %   \smash{
  %   \downarrow\aEv=\begin{cases}
  %     \{ \cEv \mid \cEv \lt \aEv \} &\textif \labeling(\aEv) \;\text{is a write}\\
  %     \aEvs_1 &\textotherwise
  %   \end{cases}}
  % \end{math}
  \begin{multicols}{2}
    \begin{enumerate}[topsep=0pt,label=(\textsc{s}\arabic*),ref=\textsc{s}\arabic*]
    \item[\eqref{seq-E}]
      \eqref{seq-lambda}\,
      \eqref{seq-kappa}\,
      \eqref{seq-tau}\,
      \eqref{seq-term}\,
      \eqref{seq-le}\, 
      \eqref{seq-rf}
      as in
      \refdef{def:pwt:mca1},
      %\reffig{fig:mca1},
      \setcounter{enumi}{\value{le}}
    % \item \label{seq-E}
    %   $\aEvs = (\aEvs_1\cup\aEvs_2)$,
    % \item \label{seq-lambda}
    %   ${\labeling}=\PBR{{\labeling_1}\cup {\labeling_2}}$, 
    %   \stepcounter{enumi}
    % \item[] \labeltext[\textsc{s}3]{}{seq-kappa}
    %   \begin{enumerate}[leftmargin=0pt]
    %   \item \label{seq-kappa1}
    %     if $\aEv\in\aEvs_1\setminus\aEvs_2$ then $\labelingForm(\aEv) \riff \labelingForm_1(\aEv)$,
    %   \item \label{seq-kappa2}
    %     if $\aEv\in\aEvs_2\setminus\aEvs_1$ then $\labelingForm(\aEv) \riff \labelingForm'_2(\aEv) \land \aTerm[1](\aEv)$,
    %   \item \label{seq-kappa12}
    %     % if $\aEv{\in}\aEvs_1{\cap}\aEvs_2$ then $\labelingForm(\aEv) \riff \labelingForm_1(\aEv){\lor}\labelingForm'_2(\aEv)$,\\
    %     if $\aEv\in\aEvs_1\cap\aEvs_2$ then $\labelingForm(\aEv) \riff (\labelingForm_1(\aEv)\lor\labelingForm'_2(\aEv)) \land \aTerm[1](\aEv)$,
    %     % where
    %     % $\labelingForm'_2(\aEv)=\aTr[1]{\Cdown{\aEv}}{\labelingForm_2(\aEv})$,
    %     % where $\Cdown{\aEv}=\{ \cEv \mid \cEv \lt \aEv \}$ if $\labeling(\aEv)$
    %     % is a write, and $\Cdown{\aEv}=\aEvs_1$, otherwise,
    %   % \item \label{seq-kappa-release}
    %   %   if $\labeling_2(\aEv)$ is a $\srelease$ then $\labelingForm(\aEv) \riff \aTerm[1]$,
    %   \end{enumerate}
    % \item \label{seq-tau}
    %   $\aTr{\bEvs}{\bForm} \riff \aTr[1]{\bEvs}{\aTr[2]{\bEvs}{\bForm}}$,
    % \item \label{seq-term}
    %   $\aTerm \riff \aTerm[1]\land\aTr[1]{}{\aTerm[2]}$,
    %   \stepcounter{enumi}
    \item[]
      \labeltext[\textsc{s}6]{}{seq-le-mca2}
      \begin{enumerate}[leftmargin=0pt]
      % \item \label{seq-le-extends}
      %   ${\le}\rextends{\le_1}{\le_2}$, 
      % \item \label{seq-le-delays-rf}
      %   if $\labeling_1(\bEv) \rdelaysp \labeling_2(\aEv)$ and
      %   %$\labelingForm_1(\bEv) \land \labelingForm_2(\aEv)$ is satisfiable then
      %   $\bEv\le\aEv$,
      %   % either $\bEv\le\aEv$ or $\labelingForm(\bEv)\land\labelingForm(\aEv)$
      %   % is unsatisfiable.
      % \item \label{seq-le-blocks}
      %   if $\labeling_1(\bEv) \rblocks \labeling_2(\aEv)$ then %either
      %   %$\labelingForm_1(\bEv) \land \labelingForm_2(\aEv)$ is satisfiable then
      %   \makebox[0cm][l]{$\bEv\le\aEv$
      %   or
      %   $\aEv\le\bEv$,}
      %   % either $\bEv\le\aEv$ or $\labelingForm(\bEv)\land\labelingForm(\aEv)$
      %   % is unsatisfiable.
      % \item \label{seq-rf-le-rf}
      %   %if $\bEv\in\aEvs_1$, $\aEv\in\aEvs_2$ and $\aEv\le\bEv$ then $\bEv\xrfx\aEv$.
      %   if $\labeling_1(\cEv) \rblocks \labeling_2(\aEv)$ and $\bEv\xrfx\aEv$
      %   then $\cEv\le\bEv$,
      \item \label{seq-le-delays-rf}
        if $\labeling_1(\bEv) \rdelays \labeling_2(\aEv)$ then either
        $\bEv\xrfx\aEv$ or $\bEv\le\aEv$,
      \item \label{seq-le-rf-rf}
        %if $\bEv\in\aEvs_1$, $\aEv\in\aEvs_2$ and $\aEv\le\bEv$ then $\bEv\xrfx\aEv$.
        if $\labeling_1(\cEv) \rblocks \labeling_2(\aEv)$ and $\bEv\xrfx\aEv$
        \\then $\cEv\le\bEv$,
      \end{enumerate}
      \stepcounter{enumi}
    \item[]
      %\labeltext[\textsc{s}7]{}{seq-rf}
      \begin{enumerate}[leftmargin=0pt]
      \item \label{seq-rf-extends}
        %${\rrfx}\rextends{\rrfx_1}{\rrfx_2}$.
        ${\rrfx_i}={\rrfx}\cap(\aEvs_i\times \aEvs_i)$, for $i\in\{1,2\}$.
      \end{enumerate}
      %   % if $\cEv\in\aEvs_1$ and $(\bEv,\aEv)\in\PBR{{\rrfx}\cap(\aEvs_1\times \aEvs_2)}$ then 
      %   % either $\labeling(\cEv) \reorderca \labeling(\bEv)$
      %   % or $\cEv\le\bEv$,
    \end{enumerate}
  \end{multicols}
  \medskip

  \noindent
  \begin{minipage}{1.0\linewidth}
  If $\aPS \in \sIF{\aForm}\sTHEN\aPSS_1\sELSE\aPSS_2\sFI$ then
  $(\exists\aPS_1\in\aPSS_1)$ $(\exists\aPS_2\in\aPSS_2)$
  \begin{multicols}{2}
    \begin{enumerate}[topsep=0pt,label=(\textsc{i}\arabic*),ref=\textsc{i}\arabic*]
    \item[\eqref{if-E}]
      \eqref{if-lambda}\,
      \eqref{if-kappa}\,
      \eqref{if-tau}\,
      \eqref{if-term}\,
      \eqref{if-le}\, 
      \eqref{if-rf}
      as in
      \refdef{def:pwt:mca1},
      %\reffig{fig:mca1},
      \setcounter{enumi}{\value{le}}
    % \item \label{if-E}
    %   $\aEvs = (\aEvs_1\cup\aEvs_2)$,
    % \item \label{if-lambda}
    %   ${\labeling}=\PBR{{\labeling_1}\cup {\labeling_2}}$, 
    %   % \item[(\ref{par-E}--\ref{par-le})] as for $\sLPAR{}{}$,
    %   \stepcounter{enumi}
    % \item[] \labeltext[\textsc{i}3]{}{if-kappa}
    %   \begin{enumerate}[leftmargin=0pt]
    %   \item \label{if-kappa1}
    %     if $\aEv\in\aEvs_1\setminus\aEvs_2$ then $\labelingForm(\aEv) \riff \aForm\land\labelingForm_1(\aEv)$,
    %   \item \label{if-kappa2}
    %     if $\aEv\in\aEvs_2\setminus\aEvs_1$ then $\labelingForm(\aEv) \riff \neg\aForm\land\labelingForm_2(\aEv)$, 
    %   \item \label{if-kappa12}
    %     if $\aEv\in\aEvs_1\cap\aEvs_2$\\ then
    %     % $\labelingForm(\aEv) \riff (\aForm\limplies\labelingForm_1(\aEv))\land(\neg\aForm\limplies\labelingForm_2(\aEv))$,
    %     $\labelingForm(\aEv) \riff (\aForm\land\labelingForm_1(\aEv))\lor(\neg\aForm\land\labelingForm_2(\aEv))$,
    %   \end{enumerate}
    % \item \label{if-tau}
    %   % $\aTr{\bEvs}{\bForm} \riff (\aForm\limplies\aTr[1]{\bEvs}{\bForm})\land(\neg\aForm\limplies\aTr[2]{\bEvs}{\bForm})$,
    %   $\aTr{\bEvs}{\bForm} \riff (\aForm\land\aTr[1]{\bEvs}{\bForm})\lor(\neg\aForm\land\aTr[2]{\bEvs}{\bForm})$,
    % \item \label{if-term}
    %   % $\aTerm \riff (\aForm\limplies\aTerm[1])\land(\neg\aForm\limplies\aTerm[2])$.
    %   $\aTerm \riff (\aForm\land\aTerm[1])\lor(\neg\aForm\land\aTerm[2])$.
    %   \stepcounter{enumi}
    % \item[]
    %   % \labeltext[\textsc{i}6]{}{if-le}
    %   \begin{enumerate}[leftmargin=0pt]
    %     % \item \label{if-le-rf}
    %     %   if $\bEv\in\aEvs_1$ and $\aEv\in\aEvs_2$ and $\bEv\xrfx\aEv$ then $\bEv\le\aEv$,
    %   \item \label{if-le-extends}
    %     ${\le}\rextends{\le_1}{\le_2}$,
    %   \item \label{if-le-subset}
    %     ${\le}\rsubset{\le_1}{\le_2}$,
    %   \end{enumerate}
    %  \stepcounter{enumi}
    % \item[]
    %   %\labeltext[\textsc{i}7]{}{if-le}
    %   \begin{enumerate}[leftmargin=0pt]
    %   \item \label{if-le-extends}
    %     ${\le}\rextends{\le_1}{\le_2}$.
    %   \end{enumerate}
      \stepcounter{enumi}
    \item[]
      %\labeltext[\textsc{i}7]{}{if-rf}
      \begin{enumerate}[leftmargin=0pt]
      \item \label{if-rf-extends}
        ${\rrfx_i}={\rrfx}\cap(\aEvs_i\times \aEvs_i)$, for $i\in\{1,2\}$.
        %${\rrfx}\rextends{\rrfx_1}{\rrfx_2}$.
      % \item \label{if-rf-le}
      %   ${\rrfx}\rsubset{\rrfx_1}{\rrfx_2}$.
      \end{enumerate}
    \end{enumerate}
  \end{multicols}
  \end{minipage}
  \smallskip

  % \noindent
  % If $\aPS\in\sLET{\aReg}{\aExp}$ then $\aEvs = \emptyset$ and
  % $\aTr{\bEvs}{\bForm} \riff \bForm[\aExp/\aReg]$.
  % \smallskip

  % \noindent
  % If $\aPS \in \sLOAD[\amode]{\aReg}[\ascope]{\aLoc}[\aThrd]$ then
  % $(\exists\aVal\in\Val)$
  % \begin{multicols}{2}
  %   \begin{enumerate}[topsep=0pt,label=(\textsc{r}\arabic*),ref=\textsc{r}\arabic*]
  %   \item \label{read-E}
  %     if $\bEv,\aEv\in\aEvs$ then $\bEv=\aEv$,
  %   \item \label{read-lambda}
  %     $\labelingAct(\aEv) = \DR[\amode]{\aLoc}[\ascope]{\aVal}[\aThrd]$,
  %   \item \label{read-kappa}
  %     $\labelingForm(\aEv) \riff \Q{\aLoc}$,
  %     %$\labelingForm(\aEv) \riff \TRUE$,
  %     \stepcounter{enumi}
  %   \item[] \labeltext[\textsc{r}4]{}{read-tau}
  %     \begin{enumerate}[leftmargin=0pt]
  %     \item \label{read-tau-dep}
  %       if $\aEvs\neq\emptyset$ and $(\aEvs\cap\bEvs)\neq\emptyset$ then
  %       \begin{math}
  %         \aTr{\bEvs}{\bForm} \riff
  %         \aVal{=}\aReg
  %         \limplies \bForm
  %       \end{math},    
  %     \item \label{read-tau-ind}
  %       if $\aEvs\neq\emptyset$ and $(\aEvs\cap\bEvs)=\emptyset$ then
  %       \begin{math}
  %         \aTr{\bEvs}{\bForm} \riff
  %         \PBR{\aVal{=}\aReg \lor \aLoc{=}\aReg} \limplies
  %         \bForm,
  %       \end{math}
  %     \item \label{read-tau-empty}
  %       if $\aEvs=\emptyset$ then
  %       \begin{math}
  %         \aTr{\bEvs}{\bForm} \riff
  %         % \PBR{\aVal{=}\aReg \lor \aLoc{=}\aReg} \limplies
  %         \bForm,
  %       \end{math}
  %     \end{enumerate}
  %     \stepcounter{enumi}
  %   \item[] \labeltext[\textsc{r}5]{}{read-term}
  %     \begin{enumerate}[leftmargin=0pt]
  %     \item \label{read-term-nonempty}
  %       if $\aEvs\neq\emptyset$ or $\amode\lemode\mRLX$ then $\aTerm \riff \TRUE$. 
  %     \item \label{read-term-empty}
  %       if $\aEvs=\emptyset$ and $\amode\gemode\mACQ$ then $\aTerm \riff \FALSE$. 
  %     \end{enumerate}      
  %   % \item \label{read-term}
  %   %   if $\aEvs=\emptyset$ and $\amode\gemode\mACQ$ then $\aTerm \riff \FALSE$. 
  %   \end{enumerate}
  % \end{multicols}
  % \smallskip

  % \noindent
  % If $\aPS \in \sSTORE[\amode]{\aLoc}[\ascope]{\aExp}[\aThrd]$ then
  % $(\exists\aVal\in\Val)$
  % \begin{multicols}{2}
  %   \begin{enumerate}[topsep=0pt,label=(\textsc{w}\arabic*),ref=\textsc{w}\arabic*]
  %   \item \label{write-E}
  %     if $\bEv,\aEv\in\aEvs$ then $\bEv=\aEv$,
  %   \item \label{write-lambda}
  %     $\labelingAct(\aEv) = \DW[\amode]{\aLoc}[\ascope]{\aVal}[\aThrd]$,
  %   \item \label{write-kappa}
  %     \begin{math}
  %       \labelingForm(\aEv) \riff
  %       \aExp{=}\aVal
  %     \end{math},    
  %   % \item \label{write-tau}
  %   %   % if $(\aEvs\cap\bEvs)\neq\emptyset$ then
  %   %   \begin{math}
  %   %     \aTr{\bEvs}{\bForm} \riff 
  %   %     \bForm
  %   %     [\aExp/\aLoc]
  %   %     % \land\aExp{=}\aVal
  %   %   \end{math},
  %     \stepcounter{enumi}      
  %   \item[] \labeltext[\textsc{w}5]{}{write-tau}
  %     \begin{enumerate}[leftmargin=0pt]
  %     \item \label{write-tau-nonempty}
  %       if $\aEvs\neq\emptyset$ then 
  %       \makebox[0cm][l]{%
  %       \begin{math}
  %         \aTr{\bEvs}{\bForm} \riff 
  %         \bForm
  %         [\aExp/\aLoc][\aExp{=}\aVal/\Q{\aLoc}]
  %       \end{math}}
  %     \item \label{write-tau-empty}
  %       if $\aEvs=\emptyset$ then 
  %       \begin{math}
  %         \aTr{\bEvs}{\bForm} \riff 
  %         \bForm
  %         [\aExp/\aLoc][\FALSE/\Q{\aLoc}]
  %       \end{math}
  %     \end{enumerate}
  %     \stepcounter{enumi}      
  %   \item[] \labeltext[\textsc{w}5]{}{write-term}
  %     \begin{enumerate}[leftmargin=0pt]
  %     \item \label{write-term-nonempty}
  %       if $\aEvs\neq\emptyset$ then $\aTerm \riff \aExp{=}\aVal$,
  %     \item \label{write-term-empty}
  %       if $\aEvs=\emptyset$ then $\aTerm \riff \FALSE$.
  %     \end{enumerate}
  %   \end{enumerate}
  % \end{multicols}
  % \smallskip

  % \noindent
  % If $\aPS \in \sFENCE[\ascope]{\amode}[\aThrd]$ then
  % % $(\exists\aLocs\subseteq\Loc)$
  % % $(\exists\bmode\in\{\amode,\mRLX\})$
  % \begin{multicols}{2}
  %   \begin{enumerate}[topsep=0pt,label=(\textsc{f}\arabic*),ref=\textsc{f}\arabic*]
  %   \item \label{fence-E}
  %     % $\aEvs\neq\emptyset$ and
  %     if $\bEv,\aEv\in\aEvs$ then $\bEv=\aEv$,
  %   \item \label{fence-lambda}
  %     $\labelingAct(\aEv) = \DF[\ascope]{\amode}[\aThrd]$,
  %   \item \label{fence-kappa}
  %     $\labelingForm(\aEv) \riff \TRUE$,
  %   \item \label{fence-tau}
  %     $\aTr{\bEvs}{\bForm} \riff \bForm$,
  %     \stepcounter{enumi}      
  %   \item[] \labeltext[\textsc{f}5]{}{fence-term}
  %     \begin{enumerate}[leftmargin=0pt]
  %     \item \label{fence-term-nonempty}
  %       if $\aEvs\neq\emptyset$ then $\aTerm \riff \TRUE$,
  %     \item \label{fence-term-empty}
  %       if $\aEvs=\emptyset$ then $\aTerm \riff \FALSE$.
  %     \end{enumerate}
  %   % \item \label{fence-term}
  %   %   if $\aEvs=\emptyset$ then $\aTerm \riff \FALSE$.
  %   \end{enumerate}
  % \end{multicols}

%   \caption{\PwTmcaTITLE{2} Semantics} 
%   \label{fig:mca2}
% \end{figure}

\end{definition}
% We write $\semmca{2}{}$ for the semantic function.
A \PwTmca{2} need not satisfy requirement \ref{pom-rf-le}, and thus we may
have $\bEv\xrfx\aEv$ and $\aEv\le\bEv$.

% 
% The use of $\rextendsdef{}{}$ in \ref{if-rf-extends}
% ensures that no $\rrfx$ is introduced between events in
% $\aEvs_1\cap\aEvs_2$ when coalescing.

% In the second semantics, we weaken the relationship between $\rrfx$
% and $\le$ in \ref{pom-rf-le}.  
% This change both allows and requires us to weaken the definition of
% \emph{delays} to drop write-to-read order from $\eqreorderco$.

With the weakening of \ref{seq-rf-le-rf}, we must be careful not to allow
spurious pairs to be added to the $\rrfx$ relation.  Thus we add
\ref{par-rf-extends},
\ref{seq-rf-extends}, and
\ref{if-rf-extends}.
For example, \ref{if-rf-extends} ensure that
\begin{math}
  \frf{\semrr{\IF{b}\THEN\PR{x}{r}\PAR\PW{x}{1}\ELSE\PR{x}{r}\SEMI\PW{x}{1}\FI}}
\end{math}
does not include 
\begin{math}
  \smash{\hbox{\begin{tikzinlinesmall}[node distance=1.5em]
        \event{a}{\DR{x}{1}}{}
        \event{b}{\DW{x}{1}}{right=of a}
        \rfint[out=165,in=15]{b}{a}
        \wki{a}{b}
      \end{tikzinlinesmall}}}
\end{math}, taking $\rrfx$ from the left and $\le$ from the right.

\PwTmca{2} does not enforce \ref{pom-rf-le}: $\bEv\xrfx\aEv$ may not imply
$\bEv\le\aEv$ when $\bEv$ and $\aEv$ come from different sides of a
sequential composition.  This means that $\rrfx$ must be verified during
pomset construction, rather than post-hoc.  In \textsection\ref{sec:c11}, we
show how to construct program order ($\rpox$) for complete pomsets using
phantom events ($\fmrg{}$).  Using this construction, the following lemma
gives a post-hoc verification technique for $\rrfx$.
\begin{lemma}
  If $\aPS\in\frf{\semmca{2}{\aCmd}}$ is complete, then
  for every $\bEv\xrfx\aEv$ either
  \begin{itemize}
  \item external fulfillment:
    $\bEv\le\aEv$ and if $\labeling(\cEv) \rblocks\labeling(\aEv)$ then either $\cEv\le\bEv$ or $\aEv\le\cEv$, or
  \item internal fulfillment:
    $(\exists\bEv'\in\fmrginv{\bEv})$
    $(\exists\aEv'\in\fmrginv{\aEv})$
    $\bEv'\xpox\aEv'$ and $(\not\exists\cEv')$
    $\labelingForm(\cEv)$ is a tautology and
    $\labeling(\cEv) \rblocks \labeling(\aEv)$ and $\bEv'\xpox\cEv\xpox\aEv'$.
  \end{itemize}
\end{lemma}
These mimic the \emph{external consistency} requirements of \armeight{}
\cite{armed}.

% \begin{enumerate}[,label=(\textsc{m}\arabic*),ref=\textsc{m}\arabic*]
  %   \setcounter{enumi}{\value{Brf}}
  % \item \label{pom-rf'} ${\rrfx} \subseteq \aEvs\times\aEvs$
  %   is an injective relation capturing \emph{reads-from}, such that
  %   % \end{enumerate}
  %   % A pomset is a \emph{candidate} if there is an injective relation
  %   % ${\rrfx} : \aEvs\times\aEvs$, capturing \emph{reads-from}, such that
  %   \begin{multicols}{2}    
  %     \begin{enumerate}
  %       % \begin{enumerate}[,label=(\textsc{i}\arabic*),ref=\textsc{i}\arabic*]
  %       % \item \label{rf-injective}
  %       %   if $\bEv\xrfx\aEv$ and $\cEv\xrfx\aEv$ then $\bEv=\cEv$, that is,
  %       %   ${\rrfx}$ is injective,
  %     \item[\eqref{pom-rf-match}]
  %       \eqref{pom-rf-block}\;
  %       as in \refdef{def:pwt:mca1},
  %       \setcounter{enumii}{2}
  %     \item \label{pom-rf-le'} 
  %       if $\bEv\xrfx\aEv$ then either $\bEv\le\aEv$ or $\aEv\le\bEv$.
  %     \end{enumerate}
  %   \end{multicols}
  % \end{enumerate}

% Derive $\rdelayspdef$ from $\rdelaysdef$ by replacing $\eqreorderco$ with
% \begin{math}
%   {\reorderlws}
%   =
%   \{(\DW{\aLoc}{}, \DW{\aLoc}{}),\;(\DR{\aLoc}{}, \DW{\aLoc}{})\}.
% \end{math}
% The acronym $\textsf{lws}$ is adopted from \armeight.  It stands for
% \emph{Local Write Successor}.

%The semantic rules are given in \reffig{fig:mca2}.
