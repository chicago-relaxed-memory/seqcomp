\section{Complications}

[I have a note: TC1: Track local state ???]

\subsection{Must Allow Inconsistent Preconditions}

\subsection{Release, Acquire, and Sequentially Consistent Access}

We use $\Q{\mRA}$ and $\Q{\mSC}$.

$\Q{\mSC}$ implies $\Q{\mRA}$.

\begin{definition}
  $\aPS$ is \emph{completed} if $\aTr[\aEvs](\Q{\mSC})$ implies $\Q{\mSC}$.
\end{definition}

Access modes can be encoded in the independency relation, indexing labels by
$\amode$, but the extra flexibility of the logic is necessary for \armeight{}
(see \textsection\ref{sec:internal}).  Using independency, one would also
need another way to define completed pomsets.  Finally, this use of
independency is incompatible with fork (see \textsection\ref{sec:co}).

\subsection{Coherence}
\label{sec:co}

$\Q{\mSC}$ implies $\Q{\mRA}$ implies $\Qx{\aLoc}$ implies $\Qw{\aLoc}$

\begin{itemize}
\item Coherence respects program order: $\Qx{\aLoc}$
\item Drop read-read coherence: $\Qw{\aLoc}$ (Required for CSE without
  alias analysis over read only code, not required by hardware)
\end{itemize}

It is also possible to put coherence in the independency relation, in which
case, the semantics of $;$ includes the following.
\begin{enumerate}
  \setcounter{enumi}{\value{pomsetXSemiCount}}
\item
  \label{seq-reorder} if $\bEv\in\aEvs_1$ and $\aEv\in\aEvs_2$ either $\bEv<\SB0\aEv$ or $a\reorder\labeling_2(\aEv)$.
\end{enumerate}
One must be careful, however, due to \emph{inconsistency}.

Consider

\eqref{seq-reorder} does not do the right thing with fork either.  If you
want to enforce coherence this way then you need to use fork-join as the
sequential combinator, rather than fork.

\begin{figure*}
  \begin{center}
  \begin{minipage}{0.905\textwidth}
    \renewcommand{\cEvs}{D}
\renewcommand{\dEvs}{D}
\noindent
If $\aPS \in \sSTORE[\amode]{\cExp}{\aExp}$ then
$(\exists\cVal:\aEvs\fun\Val)$
$(\exists\aVal:\aEvs\fun\Val)$
$(\exists\cForm:\aEvs\fun\Formulae)$
\begin{enumerate}
\item[{\labeltext[S1]{S1)}{S1full}}] % [\ref{S1})]
  if $\cForm_\bEv\land\cForm_\aEv$ is satisfiable then $\bEv=\aEv$,
\item[{\labeltext[S2]{S2)}{S2full}}] %[\ref{S2})]
  $\labelingAct(\aEv) = \DWREF{\cVal_\aEv}{\aVal_\aEv}$,
\item[{\labeltext[S3]{S3)}{S3full}}] %[\ref{S3})] 
  $\labelingForm(\aEv)$ implies
  \begin{math}
    \cForm_\aEv
    \land \QS{\REF{\cVal_\aEv}}{\amode}
    % \land \RW
    \land \cExp{=}\cVal_\aEv
    \land \aExp{=}\aVal_\aEv
  \end{math},
  % where
  % $\QS{}{\mRLX}=\QxREF{\cVal_\aEv}$ and otherwise $\QS{}{\amode}=\Q{\amode}$, % for $\amode\neq\mRLX$,
\item[{\labeltext[S4]{S4)}{S4full}}] %[\ref{S4})]
  \begin{math}
    (\forall\aEv\in\aEvs\cap\bEvs)
  \end{math}
  $\aTr{\bEvs}{\bForm}$ implies 
  \begin{math}
    \cForm_\aEv
    %\limplies (\cExp{=}\cVal_\aEv)
    \limplies {
      \bForm
      [\aExp/\REF{\cVal_\aEv}]
      \DS{\REF{\cVal_\aEv}}{\amode}
      [(\QwREF{\cVal_\aEv}\land\aExp{=}\aVal_\aEv\land\cExp{=}\cVal_\aEv)/\QwREF{\cVal_\aEv}]
    }
  \end{math},
\item[{\labeltext[S5]{S5)}{S5full}}] %[\ref{S5})] 
  \begin{math}    
    (\forall\aEv\in\aEvs\setminus\cEvs)
  \end{math}
  $\aTr{\cEvs}{\bForm}$ implies
  \begin{math}
    \cForm_\aEv
    %\limplies (\cExp{=}\cVal_\aEv)
    \limplies {
      \bForm
      [\aExp/\REF{\cVal_\aEv}]
      \DS{\REF{\cVal_\aEv}}{\amode}
      [\FALSE/\QS{\REF{\cVal_\aEv}}{\amode}]
    },
  \end{math}
\item[{\labeltext[S6]{S6)}{S6full}}] %[S6)]%\ref{S6})] 
  \begin{math}
    (\forall\dVal)
  \end{math}
  $\aTr{\dEvs}{\bForm}$ implies
  \begin{math}
    (\!\not\exists\aEv\in\aEvs \suchthat \cForm_\aEv)
    %(\!\not\exists\aEv\in\aEvs\cap\cEvs \suchthat \cForm_\aEv)
    \limplies (\cExp{=}\dVal)
    \limplies {
      \bForm
      [\aExp/\REF{\dVal}]
      \DS{\REF{\dVal}}{\amode}
      [\FALSE/\QS{\REF{\dVal}}{\amode}]
    }.
  \end{math}
% \item[S5-6)]%\ref{S6})] 
%   \begin{math}
%     (\forall\dVal)
%   \end{math}
%   $\aTr{\cEvs}{\bForm}$ implies
%   \begin{math}
%     %(\!\not\exists\aEv\in\aEvs \suchthat \cForm_\aEv)
%     (\!\not\exists\aEv\in\aEvs\cap\cEvs \suchthat \cForm_\aEv)
%     \limplies (\cExp{=}\dVal)
%     \limplies \PBR{
%       \bForm
%       [\aExp/\REF{\dVal}]
%       \DS{\REF{\dVal}}{\amode}
%       [\FALSE/\QS{\REF{\dVal}}{\amode}]
%     }.
%   \end{math}
  % \\ where 
  % $\DS{}{\mRLX}{}=[\TRUE/\DxREF{\dVal}]$ and otherwise
  % $\DS{}{\amode}{}=[\FALSE/\D]$. % for $\amode\neq\mRLX$.
\end{enumerate}
% \item if $\amode=\mRLX$ then
%   $\labelingForm(\aEv)$ implies
%   \begin{math}
%     \cForm_\aEv
%     \land \cExp{=}\cVal_\aEv
%     \land \aExp{=}\aVal_\aEv
%     \land \RW
%     \land \QxREF{\cVal_\aEv},
%   \end{math}
% \item if $\amode\neq\mRLX$ then
%   $\labelingForm(\aEv)$ implies
%   \begin{math}
%     \cForm_\aEv
%     \land \cExp{=}\cVal_\aEv
%     \land \aExp{=}\aVal_\aEv
%     \land \RW
%     \land \Q{},
%   \end{math}
% \item if
%   $\aEv\in\bEvs$
%   and
%   $\amode=\mRLX$ then
%   \begin{math}
%     (\forall\dVal)
%   \end{math}
%   $\aTr{\bEvs}{\bForm}$ implies 
%   \begin{math}
%     \cForm_\aEv
%     \limplies (\cExp{=}\dVal)
%     \limplies \PBRbig{
%     (\QwREF{\dVal} \limplies \aExp{=}\aVal_\aEv)
%     \land \bForm[\aExp/\REF{\dVal}][\TRUE/\DxREF{\dVal}]
%   }
%   \end{math}
% \item if
%   $\aEv\in\bEvs$
%   and
%   $\amode\neq\mRLX$ then
%   \begin{math}
%     (\forall\dVal)
%   \end{math}
%   $\aTr{\bEvs}{\bForm}$ implies 
%   \begin{math}
%     \cForm_\aEv
%     \limplies (\cExp{=}\dVal)
%     \limplies \PBRbig{
%     (\QwREF{\dVal} \limplies \aExp{=}\aVal_\aEv)
%     \land \bForm[\aExp/\REF{\dVal}][\FALSE/\D]
%   }
%   \end{math}
% \item if 
%   \begin{math}
%     (\forall\aEv\in\bEvs)(\cForm \textimplies
%     \lnot\cForm_\aEv)
%   \end{math}
%   and $\amode=\mRLX$ 
%   then
%   \begin{math}
%     (\forall\dVal)
%   \end{math}
%   $\aTr{\bEvs}{\bForm}$ implies 
%   \begin{math}
%     \cForm
%     \limplies (\cExp{=}\dVal)
%     \limplies \PBRbig{
%     \lnot\QwREF{\dVal}
%     \land \bForm[\aExp/\REF{\dVal}][\TRUE/\DxREF{\dVal}]
%   }
%   \end{math}
% \item if 
%   \begin{math}
%     (\forall\aEv\in\bEvs)
%     (\cForm \textimplies \lnot\cForm_\aEv)
%   \end{math}
%   and $\amode\neq\mRLX$ 
%   then
%   \begin{math}
%     (\forall\dVal)
%   \end{math}
%   $\aTr{\bEvs}{\bForm}$ implies 
%   \begin{math}
%     \cForm
%     \limplies (\cExp{=}\dVal)
%     \limplies \PBRbig{
%     \lnot\QwREF{\dVal}
%     \land \bForm[\aExp/\REF{\dVal}][\FALSE/\D]
%   }
%   \end{math}

\noindent
If $\aPS \in \sLOAD[\amode]{\aReg}{\cExp}$ then
$(\exists\cVal:\aEvs\fun\Val)$
$(\exists\aVal:\aEvs\fun\Val)$
$(\exists\cForm:\aEvs\fun\Formulae)$
% $(\forall\uReg{\aEv}\in\uRegs{\aEvs})$
\begin{enumerate}
\item[{\labeltext[L1]{L1)}{L1full}}] %[\ref{L1})]
  if $\cForm_\bEv\land\cForm_\aEv$ is satisfiable then $\bEv=\aEv$,
\item[{\labeltext[L2]{L2)}{L2full}}] %[\ref{L2})]
  $\labelingAct(\aEv) = \DRREF{\cVal_\aEv}{\aVal_\aEv}$,
\item[{\labeltext[L3]{L3)}{L3full}}] %[\ref{L3})]
  $\labelingForm(\aEv)$ implies
  \begin{math}
    \cForm_\aEv
    \land \QL{\REF{\cVal_\aEv}}{\amode}
    % \land \RO
    \land \cExp{=}\cVal_\aEv
  \end{math},
  % where    
  % $\QL{}{\mSC}=\Q{\mSC}$ and otherwise $\QL{}{\amode}=\QwREF{\cVal_\aEv}$, % for $\amode\neq\mRLX$,
\item[{\labeltext[L4]{L4)}{L4full}}] %[\ref{L4})]
  \begin{math}
    (\forall\aEv\in\aEvs\cap\bEvs)
  \end{math}
  $\aTr{\bEvs}{\bForm}$ implies
  \begin{math}
    \cForm_\aEv
    \limplies (\cExp{=}\cVal_\aEv\limplies\aVal_\aEv{=}\uReg{\aEv})
    \limplies \bForm[\uReg{\aEv}/\aReg]
  \end{math},
  %\makebox[6.2cm]{}
\item[{\labeltext[L5]{L5)}{L5full}}] %[\ref{L5})] 
  \begin{math}
    (\forall\aEv\in\aEvs\setminus\cEvs)
  \end{math}
  $\aTr{\cEvs}{\bForm}$ implies
  \begin{math}
    \cForm_\aEv 
    \limplies
    \DL{\REF{\cVal_\aEv}}{\amode}
    \land
    \PBRbig{
      \ABRbig{
        \PBR{\cExp{=}\cVal_\aEv\limplies\aVal_\aEv{=}\uReg{\aEv}}
        \lor
        \PBR{
          \RW\lor
          \PBR{\cExp{=}\cVal_\aEv\limplies\REF{\cVal_\aEv}{=}\uReg{\aEv}}
        }
      }
      \limplies
      \bForm
      [\uReg{\aEv}/\aReg]
      [\FALSE/\QL{\REF{\cVal_\aEv}}{\amode}]
    }    
  \end{math},
\item[{\labeltext[L6]{L6)}{L6full}}] %[\ref{L6})] 
  \begin{math}
    (\forall\dVal)
    (\forall\bReg)
  \end{math}
  $\aTr{\dEvs}{\bForm}$  implies 
  \begin{math}
    (\!\not\exists\aEv\in\aEvs \suchthat \cForm_\aEv)
    \limplies (\cExp{=}\dVal)
    \limplies \PBR{        
      \DL{\REF{\dVal}}{\amode} \land
      \bForm
      [\bReg/\aReg]
      [\FALSE/\QL{\REF{\dVal}}{\amode}]
    }.
  \end{math}
  % \\ where $\DL{}{\mRLX}=\TRUE$ and otherwise $\DL{}{\amode}=\DxREF{\dVal}$.
  % Recall that $\uRegs{\bEvs}=\{\uReg{\aEv}\mid\aEv\in\bEvs\}$.
\end{enumerate}  
% \item if $\amode=\mRLX$ and $\bEv\notin\bEvs$ then
%   \begin{math}
%     (\forall\dVal)
%   \end{math}
%   $\aTr{\bEvs}{\bForm}$ implies
%   \begin{math}
%     \cForm_\bEv
%     \limplies (\cExp{=}\dVal)
%     \limplies \PBRbig{
%     (
%     \RW
%     \limplies (\aVal{=}\uReg{\bEv}\lor\aLoc{=}\uReg{\bEv}) 
%     \limplies \bForm[\uReg{\bEv}/\aReg][\uReg{\bEv}/\REF{\dVal}]
%     )
%     \land \lnot\QxREF{\dVal}
%   }
%     \phantom{\land\; \Dx{\dVal}}
%   \end{math}
% \item if $\amode\neq\mRLX$ and $\bEv\notin\bEvs$ then
%   \begin{math}
%     (\forall\dVal)
%   \end{math}
%   $\aTr{\bEvs}{\bForm}$ implies
%   \begin{math}
%     \cForm_\bEv
%     \limplies (\cExp{=}\dVal)
%     \limplies \PBRbig{
%     (
%     \RW
%     \limplies (\aVal{=}\uReg{\bEv}\lor\aLoc{=}\uReg{\bEv}) 
%     \limplies \bForm[\uReg{\bEv}/\aReg][\uReg{\bEv}/\REF{\dVal}]
%     )
%     \land \lnot\QxREF{\dVal}
%     \land \Dx{\dVal}
%   }
%   \end{math}

\noindent
If $\aPS \in \sTHREAD{\aPSS}$ then
$(\exists\aPS_1\in\aPSS)$
\begin{enumerate}
\item[{\labeltext[T1]{T1)}{T1full}}] %[\ref{T1})]
  $\aEvs=\aEvs_1$,
\item[{\labeltext[T2]{T2)}{T2full}}] %[\ref{T2})]
  $\labelingAct(\aEv) = \labelingAct_1(\aEv)$,
\item[{\labeltext[T3]{T3)}{T3full}}] %[\ref{T3})]
  $\labelingForm(\aEv)$ implies
  $\labelingForm_1(\aEv) [\TRUE/\Q{}][\TRUE/\RW]$ if $\labelingAct_1(\aEv)$ is a write,
  \\
  $\labelingForm(\aEv)$ implies
  $\labelingForm_1(\aEv) [\TRUE/\Q{}][\FALSE/\RW]$ otherwise.
\end{enumerate}  

    % \noindent
If $\aPS \in \sSTORE[\amode]{\cExp}{\aExp}$ then
$(\exists\cVal:\aEvs\fun\Val)$
$(\exists\aVal:\aEvs\fun\Val)$
$(\exists\bForm:\aEvs\fun\Formulae)$
\begin{enumerate}
\item if $\bForm_\bEv\land\bForm_\aEv$ is satisfiable then $\bEv=\aEv$,
\item $\labelingAct(\aEv) = \DWREFP{\cVal_\aEv}{\aVal_\aEv}$,
\item 
  $\labelingForm(\aEv)$ implies
  \begin{math}
    \bForm_\aEv
    \land \cExp{=}\cVal_\aEv
    \land \aExp{=}\aVal_\aEv
    \land \RW
    \land \Qmode{\amode}
  \end{math},
  where
  $\Qmode{\mRLX}=\QxREF{\cVal_\aEv}$ and otherwise $\Qmode{\amode}=\Q{\amode}$, % for $\amode\neq\mRLX$,
\item
  \begin{math}
    (\forall\dVal)
  \end{math}
  if
  $\bEv\in\bEvs$
  then
  $\aTr[\bEvs](\aForm)$ implies 
  \begin{math}
    \bForm_\bEv
    \limplies (\cExp{=}\dVal)
    \limplies \PBRbig{
      (\QwREF{\dVal} \limplies \aExp{=}\aVal_\bEv)
      \land \aForm [\aExp/\REF{\dVal}]\Dmode{\amode}
    }
  \end{math},
\item %if 
  % \begin{math}
  %   (\forall\bEv\in\bEvs)(\cForm \textimplies
  %   \lnot\bForm_\bEv)
  % \end{math}
  % then
  \begin{math}
    (\forall\dVal)
  \end{math}
  $\aTr[\bEvs](\aForm)$ implies 
  \begin{math}
    (\not\exists\bEv\in\bEvs.\; \bForm_\bEv)
    \limplies (\cExp{=}\dVal)
    \limplies \PBR{
      \lnot\QwREF{\dVal}
      \land \aForm [\aExp/\REF{\dVal}]\Dmode{\amode}
    }
  \end{math},
  \\ where 
  $\Dmode{\mRLX}=[\TRUE/\DxREF{\dVal}]$ and otherwise
  $\Dmode{\amode}=[\FALSE/\D]$. % for $\amode\neq\mRLX$.
\end{enumerate}
% \item if $\amode=\mRLX$ then
%   $\labelingForm(\aEv)$ implies
%   \begin{math}
%     \bForm_\aEv
%     \land \cExp{=}\cVal_\aEv
%     \land \aExp{=}\aVal_\aEv
%     \land \RW
%     \land \QxREF{\cVal_\aEv},
%   \end{math}
% \item if $\amode\neq\mRLX$ then
%   $\labelingForm(\aEv)$ implies
%   \begin{math}
%     \bForm_\aEv
%     \land \cExp{=}\cVal_\aEv
%     \land \aExp{=}\aVal_\aEv
%     \land \RW
%     \land \Q{},
%   \end{math}
% \item if
%   $\bEv\in\bEvs$
%   and
%   $\amode=\mRLX$ then
%   \begin{math}
%     (\forall\dVal)
%   \end{math}
%   $\aTr[\bEvs](\aForm)$ implies 
%   \begin{math}
%     \bForm_\bEv
%     \limplies (\cExp{=}\dVal)
%     \limplies \PBRbig{
%     (\QwREF{\dVal} \limplies \aExp{=}\aVal_\bEv)
%     \land \aForm[\aExp/\REF{\dVal}][\TRUE/\DxREF{\dVal}]
%   }
%   \end{math}
% \item if
%   $\bEv\in\bEvs$
%   and
%   $\amode\neq\mRLX$ then
%   \begin{math}
%     (\forall\dVal)
%   \end{math}
%   $\aTr[\bEvs](\aForm)$ implies 
%   \begin{math}
%     \bForm_\bEv
%     \limplies (\cExp{=}\dVal)
%     \limplies \PBRbig{
%     (\QwREF{\dVal} \limplies \aExp{=}\aVal_\bEv)
%     \land \aForm[\aExp/\REF{\dVal}][\FALSE/\D]
%   }
%   \end{math}
% \item if 
%   \begin{math}
%     (\forall\bEv\in\bEvs)(\cForm \textimplies
%     \lnot\bForm_\bEv)
%   \end{math}
%   and $\amode=\mRLX$ 
%   then
%   \begin{math}
%     (\forall\dVal)
%   \end{math}
%   $\aTr[\bEvs](\aForm)$ implies 
%   \begin{math}
%     \cForm
%     \limplies (\cExp{=}\dVal)
%     \limplies \PBRbig{
%     \lnot\QwREF{\dVal}
%     \land \aForm[\aExp/\REF{\dVal}][\TRUE/\DxREF{\dVal}]
%   }
%   \end{math}
% \item if 
%   \begin{math}
%     (\forall\bEv\in\bEvs)
%     (\cForm \textimplies \lnot\bForm_\bEv)
%   \end{math}
%   and $\amode\neq\mRLX$ 
%   then
%   \begin{math}
%     (\forall\dVal)
%   \end{math}
%   $\aTr[\bEvs](\aForm)$ implies 
%   \begin{math}
%     \cForm
%     \limplies (\cExp{=}\dVal)
%     \limplies \PBRbig{
%     \lnot\QwREF{\dVal}
%     \land \aForm[\aExp/\REF{\dVal}][\FALSE/\D]
%   }
%   \end{math}

\noindent
If $\aPS \in \sLOAD[\amode]{\aReg}{\cExp}$ then
$(\exists\cVal:\aEvs\fun\Val)$
$(\exists\aVal:\aEvs\fun\Val)$
$(\exists\bForm:\aEvs\fun\Formulae)$
% $(\forall\uReg{\aEv}\in\uRegs{\aEvs})$
\begin{enumerate}
\item if $\bForm_\bEv\land\bForm_\aEv$ is satisfiable then $\bEv=\aEv$,
\item $\labelingAct(\aEv) = \DRREFP{\cVal_\aEv}{\aVal_\aEv}$,
\item $\labelingForm(\aEv)$ implies
  \begin{math}
    \bForm_\aEv
    \land \cExp{=}\cVal_\aEv
    \land \RO
    \land \Qmode{\amode}
  \end{math},
  where    
  $\Qmode{\mSC}=\Q{\mSC}$ and otherwise $\Qmode{\amode}=\QwREF{\cVal_\aEv}$, % for $\amode\neq\mRLX$,
\item
  \begin{math}
    (\forall\dVal)
  \end{math}
  if $\bEv\in\bEvs$ then
  $\aTr[\bEvs](\aForm)$ implies
  \begin{math}
    \bForm_\bEv
    \limplies (\cExp{=}\dVal)
    \limplies (\aVal{=}\uReg{\bEv})
    \limplies \aForm[\uReg{\bEv}/\aReg][\uReg{\bEv}/\REF{\dVal}]
  \end{math},
  \makebox[4.4cm]{}
\item 
  \begin{math}
    (\forall\dVal)
  \end{math}
  if $\bEv\notin\bEvs$ then
  $\aTr[\bEvs](\aForm)$ implies
  \begin{math}
    \bForm_\bEv
    \limplies (\cExp{=}\dVal)
    \limplies \PBRbig{        
      \Dmode{\amode}
      \land \lnot\QxREF{\dVal}
      \land
      (\RW
      \limplies (\aVal{=}\uReg{\bEv}\lor\aLoc{=}\uReg{\bEv}) 
      \limplies \aForm[\uReg{\bEv}/\aReg][\uReg{\bEv}/\REF{\dVal}]
      )
    }      
  \end{math},
\item % if 
  % \begin{math}
  %   (\forall\bEv\in\bEvs)(\cForm \textimplies
  %   \lnot\bForm_\bEv)
  % \end{math}
  % then
  \begin{math}
    (\forall\dVal)
    (\forall\bReg)
  \end{math}
  $\aTr[\bEvs](\aForm)$ implies 
  \begin{math}
    (\not\exists\bEv\in\bEvs.\; \bForm_\bEv)
    \limplies (\cExp{=}\dVal)
    \limplies \PBR{        
      \Dmode{\amode}
      \land \lnot\QxREF{\dVal}
      \land
      \limplies \aForm[\bReg/\aReg][\bReg/\REF{\dVal}]
    }      
  \end{math},
  \\ where $\Dmode{\mRLX}=\TRUE$ and otherwise $\Dmode{\amode}=\Dx{\dVal}$.
  Recall that $\uRegs{\bEvs}=\{\uReg{\bEv}\mid\bEv\in\bEvs\}$.
\end{enumerate}  
% \item if $\amode=\mRLX$ and $\bEv\notin\bEvs$ then
%   \begin{math}
%     (\forall\dVal)
%   \end{math}
%   $\aTr[\bEvs](\aForm)$ implies
%   \begin{math}
%     \bForm_\bEv
%     \limplies (\cExp{=}\dVal)
%     \limplies \PBRbig{
%     (
%     \RW
%     \limplies (\aVal{=}\uReg{\bEv}\lor\aLoc{=}\uReg{\bEv}) 
%     \limplies \aForm[\uReg{\bEv}/\aReg][\uReg{\bEv}/\REF{\dVal}]
%     )
%     \land \lnot\QxREF{\dVal}
%   }
%     \phantom{\land\; \Dx{\dVal}}
%   \end{math}
% \item if $\amode\neq\mRLX$ and $\bEv\notin\bEvs$ then
%   \begin{math}
%     (\forall\dVal)
%   \end{math}
%   $\aTr[\bEvs](\aForm)$ implies
%   \begin{math}
%     \bForm_\bEv
%     \limplies (\cExp{=}\dVal)
%     \limplies \PBRbig{
%     (
%     \RW
%     \limplies (\aVal{=}\uReg{\bEv}\lor\aLoc{=}\uReg{\bEv}) 
%     \limplies \aForm[\uReg{\bEv}/\aReg][\uReg{\bEv}/\REF{\dVal}]
%     )
%     \land \lnot\QxREF{\dVal}
%     \land \Dx{\dVal}
%   }
%   \end{math}

  \end{minipage}
  \end{center}
  \caption{Full Semantics of Load and Store}
  \label{fig:full}
\end{figure*}    


\subsection{ARM Compilation: Internal Acquires}
\label{sec:internal}
Downgrading acquires/Anton example: $\Dx{\aLoc}$

%$\D$ implies $\Dx{\aLoc}$

We write $[\aForm/\D]$ for the substitution that performs
$[\aForm/\Dx{\aLoc}]$ for every $\aLoc$.

\subsection{ARM Compilation: Read-read dependencies}
$\RW$/$\RO$ (control dependencies into reads as in MP with
release on right and control dependency on left)

$\RW$ implies $\lnot\RO$ and 
$\RO$ implies $\lnot\RW$.


\subsection{Putting it together}


Combining the features defined thus far, we have the following, assuming that
each register occurs at most once.

\begin{definition}$\phantom{\;}$\par
  
  $\QS{\aLoc}{\mRLX}=\Qx{\aLoc}$ and otherwise $\QS{\aLoc}{\amode}=\Q{\amode}$.

  $\QL{\aLoc}{\mSC}=\Q{\mSC}$ and otherwise $\QL{\aLoc}{\amode}=\Qw{\aLoc}$.

  $\DS{\aLoc}{\mRLX}{\aForm}=\aForm[\TRUE/\Dx{\aLoc}]$ and otherwise
  $\DS{\aLoc}{\amode}{\aForm}=\aForm[\FALSE/\D]$. 

  $\DL{\aLoc}{\mRLX}=\TRUE$ and otherwise $\DL{\aLoc}{\amode}=\Dx{\aLoc}$.
  \smallskip
  
  \noindent
If $\aPS \in \sSTORE[\amode]{\aLoc}{\aExp}$ then
\begin{enumerate}
\item if $\bEv\in\aEvs$ and $\aEv\in\aEvs$ then $\bEv=\aEv$,
\item $\labelingAct(\aEv) = \DWP{\aLoc}{\aVal}$,
\item 
  $\labelingForm(\aEv)$ implies
  \begin{math}
    \aExp{=}\aVal
    \land \RW
    \land \QS{\aLoc}{\amode}
  \end{math},
\item
  $\aTr[\bEvs](\aForm)$ implies 
  \begin{math}
    (\Qw{\aLoc} \limplies \aExp{=}\aVal)
    \land \DS{\aLoc}{\amode}{\aForm[\aExp/{\aLoc}]}
  \end{math},
\item 
  $\aTr[\emptyset](\aForm)$ implies 
  \begin{math}
    \lnot\Qw{\aLoc}
    \land \DS{\aLoc}{\amode}{\aForm[\aExp/{\aLoc}]}.
  \end{math}
\end{enumerate}

\noindent
If $\aPS \in \sLOAD[\amode]{\aLoc}{\aReg}$ then
\begin{enumerate}
\item if $\bEv\in\aEvs$ and $\aEv\in\aEvs$ then $\bEv=\aEv$,
\item $\labelingAct(\aEv) = \DRP{\aLoc}{\aVal}$,
\item $\labelingForm(\aEv)$ implies
  \begin{math}
    \RO
    \land \QL{\aLoc}{\amode}
  \end{math},
\item
  $\aTr[\bEvs](\aForm)$ implies
  \begin{math}
    (\aVal{=}\aReg)
    \limplies \aForm[\aReg/{\aLoc}]
  \end{math}
\item 
  $\aTr[\emptyset](\aForm)$ implies
  \begin{math}
    \DL{\aLoc}{\amode}
    \land \lnot\Qx{\aLoc}
  % \end{math}
  % \\
  % \begin{math}
    % {}
    \land 
    (\RW
    \limplies (\aVal{=}\aReg\lor\aLoc{=}\aReg) 
    \limplies \aForm[\aReg/{\aLoc}]
    ).
  \end{math}
  % \item 
  %   $\aTr[\bEvs](\aForm)$ implies 
  %   \begin{math}
  %     (\not\exists\bEv\in\bEvs.\; \bForm)
  %     \limplies \PBR{        
  %     \DL{\aLoc}{\amode}
  %     \land \lnot\Qx{\aLoc}
  %     \land
  %     \limplies \aForm[\aReg/{\aLoc}]
  %   }      
  %   \end{math}
\end{enumerate}  

\end{definition}

If we move coherence to independency (and use fork-join), we have the
following, assuming that each register occurs at most once.
\begin{definition}$\phantom{\;}$\par
  $\QS{}{\mRLX}=\TRUE$ and otherwise $\QS{}{\amode}=\Q{\amode}$.

  $\QL{}{\mSC}=\Q{\mSC}$ and otherwise $\QL{}{\amode}=\TRUE$.

  \smallskip

  \noindent
If $\aPS \in \sSTORE[\amode]{\aLoc}{\aExp}$ then
\begin{enumerate}
\item if $\bEv\in\aEvs$ and $\aEv\in\aEvs$ then $\bEv=\aEv$,
\item $\labelingAct(\aEv) = \DWP{\aLoc}{\aVal}$,
\item 
  $\labelingForm(\aEv)$ implies
  \begin{math}
    \aExp{=}\aVal
    \land \RW
    \land \QS{}{\amode}
  \end{math},
\item
  $\aTr[\bEvs](\aForm)$ implies 
  \begin{math}
    \aExp{=}\aVal
    \land \DS{\aLoc}{\amode}{\aForm[\aExp/{\aLoc}]}
  \end{math},
\item 
  $\aTr[\emptyset](\aForm)$ implies 
  \begin{math}
    \lnot\Q{\mRA}
    \land \DS{\aLoc}{\amode}{\aForm[\aExp/{\aLoc}]}
  \end{math}
\end{enumerate}

\noindent
If $\aPS \in \sLOAD[\amode]{\aLoc}{\aReg}$ then
\begin{enumerate}
\item if $\bEv\in\aEvs$ and $\aEv\in\aEvs$ then $\bEv=\aEv$,
\item $\labelingAct(\aEv) = \DRP{\aLoc}{\aVal}$,
\item $\labelingForm(\aEv)$ implies
  \begin{math}
    \RO
    \land \QL{}{\amode}
  \end{math},
\item
  $\aTr[\bEvs](\aForm)$ implies
  \begin{math}
    (\aVal{=}\aReg)
    \limplies \aForm[\aReg/{\aLoc}]
  \end{math}
\item 
  $\aTr[\emptyset](\aForm)$ implies
  \begin{math}
    \DL{\aLoc}{\amode}
    \land \lnot\Q{\mRA}
    \land
    (\RW
    \limplies (\aVal{=}\aReg\lor\aLoc{=}\aReg) 
    \limplies \aForm[\aReg/{\aLoc}]
    ).
  \end{math}
\end{enumerate}  

\end{definition}



\section{Further Complications}

\subsection{Redundant Read Elimination}

Requires indexing to resolve nondeterminism.

\begin{gather*}
  \taglabel{TC2}
  \PR{x}{r}\SEMI
  \PR{x}{s}\SEMI
  \IF{r{=}s}\THEN \PW{y}{1}\FI
  \PAR
  x\GETS y
  \\
  \hbox{\begin{tikzinline}[node distance=1.5em]
  \event{a1}{\DR{x}{1}}{}
  \event{a2}{\DR{x}{1}}{right=of a1}
  \event{a3}{\DW{y}{1}}{right=of a2}
  % \po{a2}{a3}
  % \po[out=-20,in=-160]{a1}{a3}
  \event{b1}{\DR{y}{1}}{right=3em of a3}
  \event{b2}{\DW{x}{1}}{right=of b1}
  \rf{a3}{b1}
  \po{b1}{b2}
  \rf[out=169,in=11]{b2}{a2}
  \rf[out=169,in=11]{b2}{a1}
    \end{tikzinline}}
\end{gather*}
Precondition of $\DWP{y}{1}$ is $(r{=}s)$ in
\begin{math}
  \sem{\IF{r{=}s}\THEN y\GETS 1\FI}.
\end{math}
Predicate transformers for $\emptyset$ in $\sem{\PR{x}{r}}$ and $\sem{\PR{x}{s}}$ are
\begin{align*}
  \PREDP{(r{=}1 \lor r{=}x)\limplies\aForm[r/x]},
  \\
  \PREDP{(s{=}1 \lor s{=}x)\limplies\aForm[s/x]}.
\end{align*}
Combining the transformers, we have
\begin{displaymath}
  \PREDP{(r{=}1 \lor r{=}x)\limplies(s{=}1 \lor s{=}r)\limplies\aForm[s/x]}.
\end{displaymath}
Applying this to $(r{=}s)$, we have
\begin{displaymath}
  \PREDP{(r{=}1 \lor r{=}x)\limplies (s{=}1 \lor s{=}r)\limplies (r{=}s)},
\end{displaymath}
which is not a tautology.

Same problem occurs oopsla, where we have:
\begin{align*}
  \PREDP{\aForm[v/x,r] \land \aForm[x/r]},
  \\
  \PREDP{\aForm[v/x,s] \land \aForm[x/s]}.
\end{align*}
Combining the transformers, we have
\begin{displaymath}
  \PREDP{\aForm[v/x,r,s] \land \aForm [v/x,r][x/s] \land \aForm[x/r][v/x,s] \land \aForm[x/r,s]}.
\end{displaymath}
Applying this to $(r{=}s)$, we have
\begin{displaymath}
  \PREDP{v{=}v \land v{=}x \land x{=}v \land x{=}x},
\end{displaymath}
which is not a tautology.

The semantics here allows this by coalescing:
\begin{gather*}
  r\GETS x\SEMI
  s\GETS x\SEMI
  \IF{r{=}s}\THEN y\GETS 1\FI
  \PAR
  x\GETS y
  \\
  \hbox{\begin{tikzinline}[node distance=1.5em]
      \event{a1}{\DR{x}{1}}{}
      \event{a3}{\DW{y}{1}}{right=of a1}
      \event{b1}{\DR{y}{1}}{right=3em of a3}
      \event{b2}{\DW{x}{1}}{right=of b1}
      \rf{a3}{b1}
      \po{b1}{b2}
      \rf[out=169,in=11]{b2}{a1}
    \end{tikzinline}}
\end{gather*}

\subsection{If Closure}
Requires indexing to resolve nondeterminism.

IF closure/case analysis: $\psi_e$

\subsection{Address Calculation}

Do this after if closure, because problem with punning badly.

\begin{definition}
  \noindent
  If $\aPS\SB0 \in \sSTORE{\cExp}{\aExp}$ then
  $(\exists\aVal,\,\cVal\in\Val)$
  \begin{enumerate}
  \item if $\bEv,\aEv\in\aEvs\SB0$ then $\bEv=\aEv$.
  \item $\labelingAct\SB0(\aEv) = \DWP{\REF\cVal}{\aVal}$,
  \item $\labelingForm\SB0(\aEv)$ implies $(\cExp{=}\cVal \land \aExp{=}\aVal)$,
  \item $\aTr[\emptyset]\SB0(\aForm)$ implies $(\cExp{=}\cVal) \limplies \aForm[\aExp/\REF{\cVal}]$,
  \item $\aTr[\bEvs]\SB0(\aForm)$ implies $(\cExp{=}\cVal) \limplies (\aExp{=}\aVal) \land \aForm[\aExp/\REF{\cVal}]$, 
  \end{enumerate}

  \noindent
  If $\aPS\SB0 \in \sLOAD{\cExp}{\aReg}$ then
  $(\exists\aVal,\,\cVal\in\Val)$
  \begin{enumerate}
  \item if $\bEv,\aEv\in\aEvs\SB0$ then $\bEv=\aEv$.
  \item $\labelingAct\SB0(\aEv) = \DRP{\REF{\cVal}}{\aVal}$,
  \item $\labelingForm\SB0(\aEv)$ implies $(\cExp{=}\cVal)$,
  \item $\aTr[\emptyset]\SB0(\aForm)$ implies
    $(\cExp{=}\cVal) \limplies (\aReg{=}\aVal\lor\aReg{=}\REF{\cVal})\limplies\aForm[\aReg/\REF{\cVal}]$,
  \item $\aTr[\bEvs]\SB0(\aForm)$ implies
    $(\cExp{=}\cVal) \limplies (\aReg{=}\aVal)\limplies\aForm[\aReg/\REF{\cVal}]$, 
  \end{enumerate}  
\end{definition}

\subsection{Putting it together}

The full semantics of load and store is given in Figure \ref{fig:full}.
Recall that $\uRegs{\bEvs}=\{\uReg{\bEv}\mid\bEv\in\bEvs\}$.

% \subsection{Agda}
% \begin{figure*}
%   \includegraphics[width=\textwidth]{agda.png}
% \end{figure*}
