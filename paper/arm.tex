\section{Efficient Implementation on ARMv8}
\label{sec:arm}

We discuss \armeight{} using \emph{external global completion} (\EGC{})
\cite{alglave-git-alternate} %,armed-cats},
\cite[\textsection B2.3.6]{arm-reference-manual}
which is very close to our model.

\subsection{Downgraded Reads (\xDGR)}
\label{sec:downgrade}


\begin{example}
  \label{ex:dgr1}
  The following example is from \citeauthor{anton} \cite{anton}.  It is
  allowed by \armeight{}, but disallowed by \refdef{def:pomsets-ra}.  The
  coherence order between the writes can be witnessed by a separate thread,
  which we have elided.
  \begin{gather*}
    \PW{x}{2}\SEMI 
    \PR[\mRA]{x}{r}\SEMI
    \PW{y}{1} \PAR
    \PW{y}{2}\SEMI
    \PW[\mRA]{x}{1}
    \\
    \hbox{\begin{tikzinline}[node distance=1.5em]
        \event{a}{\DW{x}{2}}{}
        \raevent{b}{\DR[\mRA]{x}{2}}{right=of a}
        \event{c}{\DW{y}{1}}{right=of b}
        \event{d}{\DW{y}{2}}{right=2.5em of c}
        \raevent{e}{\DW[\mRA]{x}{1}}{right=of d}
        \rf{a}{b}
        \sync{b}{c}
        \wk{c}{d}
        \sync{d}{e}
        \wk[out=-165,in=-15]{e}{a}
      \end{tikzinline}}
  \end{gather*}
  Under \EGC{}, this is explained by dropping the order
  $\DRP[\mRA]{x}{2}\xsync\DWP{y}{1}$, because $\DRP[\mRA]{x}{2}$ is fulfilled
  by a relaxed write in the same thread.
  \begin{gather*}
    \hbox{\begin{tikzinline}[node distance=1.5em]
        \event{a}{\DW{x}{2}}{}
        \raevent{b}{\DR[\mRA]{x}{2}}{right=of a}
        \event{c}{\DW{y}{1}}{right=of b}
        \event{d}{\DW{y}{2}}{right=2.5em of c}
        \raevent{e}{\DW[\mRA]{x}{1}}{right=of d}
        \rf{a}{b}
        % \sync{b}{c}
        \wk{c}{d}
        \sync{d}{e}
        \wk[out=-165,in=-15]{e}{a}
      \end{tikzinline}}  
  \end{gather*}

  More generally, this can be understood as a compiler optimization that
  downgrades a read from $\mRA$ to $\mRLX$ when it can be fulfilled by a
  relaxed write in the same thread.
\end{example}

To model such \emph{downgraded reads}, we use the uninterpreted symbols
$\Dx{\aLoc}$.  

\begin{definition}
  \label{def:DS}
  % Let $[\FALSE/\D]$ be the substitution that performs $[\FALSE/\Dx{\aLoc}]$
  % for every $\aLoc$.
  Let $[\aForm/\Dx{*}]$ substitute $\aForm$ for every $\Dx{\bLoc}$.\\
  Let formula $\DL{\aLoc}{\amode}$ and substitution $\DS{\aLoc}{\amode}$ be defined:
  \begin{align*}
    \DL{\aLoc}{\amode}&=
    \begin{cases}
      \Dx{\aLoc} & \textif \amode\neq\mRLX
      \\
      \TRUE& \textotherwise
    \end{cases}
    &
    \DS{\aLoc}{\amode}&=
    \begin{cases}
      [\TRUE/\Dx{\aLoc}]  & \textif \amode=\mRLX
      \\
      [\FALSE/\D] & \textotherwise
    \end{cases}
  \end{align*}
  % \begin{align*}
  %   \DS{\aLoc}{\mRLX}&=[\TRUE/\Dx{\aLoc}] 
  %   &\DL{\aLoc}{\mRLX}&=\TRUE
  %   \\
  %   \DS{\aLoc}{\mRA}&=[\FALSE/\D]
  %   &\DL{\aLoc}{\mRA}&=\Dx{\aLoc}%\land\D{}
  %   \\
  %   \DS{\aLoc}{\mSC}&=[\FALSE/\D]
  %   &\DL{\aLoc}{\mSC}&=\Dx{\aLoc}%\land\D{}
  % \end{align*}
\end{definition}
\begin{definition}[\xCO/\xRASC/\xDGR]
  \label{def:pomsets-down}
  Update \refdef{def:pomsets-ra} to: % (\ref{L4} unchanged):
  \begin{enumerate}
  \item[\ref{S4})]
    $\aTr{\bEvs}{\bForm}$ implies
    \begin{math}
      \bForm
      \DS{\aLoc}{\amode} \land\aExp{=}\aVal,
      %[(\Qw{\aLoc}\land\aExp{=}\aVal)/\Qw{\aLoc}],
    \end{math}
  \item[\ref{S5})]
    $\aTr{\cEvs}{\bForm}$ implies
    \begin{math}
      \bForm
      \DS{\aLoc}{\amode}
      [\FALSE/\QS{\aLoc}{\amode}],
    \end{math}
  % \item[\ref{L4})]
  %   $\aTr{\bEvs}{\bForm}$ implies
  %   \begin{math}
  %     \aVal{=}\aReg\limplies\bForm,
  %   \end{math}
  \item[\ref{L5})]
    $\aTr{\cEvs}{\bForm}$ implies
    \begin{math}
      \DL{\aLoc}{\amode}\land \bForm[\FALSE/\QL{\aLoc}{\amode}].
    \end{math}
  \end{enumerate}
\end{definition}
Load actions that require downgrading introduce $\Dx{\aLoc}$.
Relaxed stores on $\aLoc$ substitute true for $\Dx{\aLoc}$, whereas
synchronizing stores substitute false for $\Dx{\aLoc}$.

\begin{example}
  \label{ex:dgr2}
  Revisiting \refex{ex:dgr1} and eliding irrelevant transformers:
  \begin{align*}
    \begin{gathered}[t]
      \PW{x}{2}
      \\
      \hbox{\begin{tikzinline}[node distance=0.5em and 1.5em]
          \event{a}{\DW{x}{2}}{}
          \xform{x1d}{\bForm[\TRUE/\Dx{\aLoc}]}{below=of a}
          % \xform{x1i}{\lnot\Qw{x}\land\bForm[\TRUE/\Dx{x}]}{below=of x1d}
          \xo{a}{x1d}
        \end{tikzinline}}  
    \end{gathered}  
    &&
    \begin{gathered}[t]
      \PR[\mRA]{x}{r}
      \\
      \hbox{\begin{tikzinline}[node distance=0.5em and 1.5em]
          \raevent{b}{\DR[\mRA]{x}{2}}{}
          % \xform{x1d}{2{=}r\limplies\bForm}{below=of b}
          \xform{x1i}{\Dx{x} \land\bForm[\FALSE/\Qr{x}]}{below=of b} %x1d}
          % \xo{b}{x1d}
        \end{tikzinline}}  
    \end{gathered}  
    &&
    \begin{gathered}[t]
      \PW{y}{1}
      \\
      \hbox{\begin{tikzinline}[node distance=0.5em and 1.5em]
          \event{c}{\DW{y}{1}}{}
          % \xform{x1d}{\bForm[\TRUE/\Dx{\aLoc}]}{below=of c}
          % \xform{x1i}{\lnot\Qw{y}\land\bForm[\TRUE/\Dx{y}]}{below=of x1d}
          % \xo{c}{x1d}
        \end{tikzinline}}  
    \end{gathered}  
  \end{align*}
  Associating right:
  \begin{align*}
    \begin{gathered}[t]
      \PW{x}{2}
      \\
      \hbox{\begin{tikzinline}[node distance=0.5em and 1.5em]
          \event{a}{\DW{x}{2}}{}
          \xform{x1d}{\bForm[\TRUE/\Dx{\aLoc}]}{right=.7em of a}
          % \xform{x1i}{\lnot\Qw{x}\land\bForm[\TRUE/\Dx{x}]}{below=of x1d}
          \xos{a}{x1d}
        \end{tikzinline}}  
    \end{gathered}  
    &&
    \begin{gathered}[t]
      \PR[\mRA]{x}{r}
      \SEMI
      \PW{y}{1}
      \\
      \hbox{\begin{tikzinline}[node distance=0.5em and 1.5em]
          \raevent{b}{\DR[\mRA]{x}{2}}{}
          \event{c}{\Dx{x}\mid\DW{y}{1}}{right=.7em of b}
        \end{tikzinline}}  
    \end{gathered}  
  \end{align*}
  Composing, we have, as desired:
  \begin{align*}
    \begin{gathered}[t]
      \PW{x}{2}
      \SEMI
      \PR[\mRA]{x}{r}
      \SEMI
      \PW{y}{1}
      \\
      \hbox{\begin{tikzinline}[node distance=0.5em and 1.5em]
          \event{a}{\DW{x}{2}}{}
          \raevent{b}{\DR[\mRA]{x}{2}}{right=of a}
          \event{c}{\DW{y}{1}}{right=of b}
          \rf{a}{b}
        \end{tikzinline}}  
    \end{gathered}  
  \end{align*}
\end{example}

% The following execution is allowed both by \armeight{} and \tso{}, but
% disallowed by \refdef{def:pomsets-ra}.
% \begin{gather*}
%   \PW{x}{2}\SEMI 
%   \PR[\mRA]{x}{r}\SEMI
%   \PR{y}{s} \PAR
%   \PW{y}{2}\SEMI
%   \PW[\mRA]{x}{1}
%   \\
%   \hbox{\begin{tikzinline}[node distance=1.5em]
%     \event{a}{\DW{x}{2}}{}
%     \raevent{b}{\DR[\mRA]{x}{2}}{right=of a}
%     \event{c}{\DR{y}{0}}{right=of b}
%     \event{d}{\DW{y}{2}}{right=2.5em of c}
%     \raevent{e}{\DW[\mRA]{x}{1}}{right=of d}
%     \rf{a}{b}
%     \sync{b}{c}
%     \wk{c}{d}
%     \sync{d}{e}
%     \wk[out=-165,in=-15]{e}{a}
%   \end{tikzinline}}
% \end{gather*}

\begin{example}
  One might worry that our model is too permissive for $\mSC$ access, but
  \armeight{} itself allows some very counterintuitive results for $\mSC$
  access.  In the following execution we elide the initializing write
  $\DWP{y}{0}$.
  \begin{gather*}
    \IF{\PR{x}{}}\THEN \PW{x}{2} \FI\SEMI
    \PR[\mSC]{x}{r}\SEMI
    \PR[\mSC]{y}{s} \PAR
    \PW[\mSC]{y}{2}\SEMI
    \PW[\mSC]{x}{1}
    \\
    \hbox{\begin{tikzinline}[node distance=1.5em]
        \event{a}{\DR{x}{1}}{}
        \event{b}{\DW{x}{2}}{right=of a}
        \scevent{c}{\DR[\mSC]{x}{2}}{right=of b}
        \scevent{d}{\DR[\mSC]{y}{0}}{right=of c}
        \scevent{e}{\DW[\mSC]{y}{2}}{right=2.5em of d}
        \scevent{f}{\DW[\mSC]{x}{1}}{right=of e}
        \po{a}{b}
        \rf{b}{c}
        % \sync{c}{d}
        \sync{e}{f}
        \wk{d}{e}
        \rf[out=-165,in=-15]{f}{a}
      \end{tikzinline}}
  \end{gather*}
  Under \EGC{}, this is explained by dropping the order
  $\DRP[\mSC]{x}{2}\xsync\DRP{y}{0}$, because $\DRP[\mSC]{x}{2}$ is fulfilled
  by a relaxed write in the same thread.
\end{example}

\subsection{Removing Read-Read dependencies (\xRRD)}
\label{sec:rrd}

\begin{example}
  \label{ex:rrd1}
  The following execution is allowed by \armeight{}, but disallowed by
  \refdef{def:pomsets-ra}.  
  \begin{gather*}
    \begin{gathered}
      x\GETS0\SEMI x\GETS 1\SEMI y^\mRA\GETS1
      \PAR
      r\GETS y\SEMI \IF{r}\THEN s\GETS x\FI
      \\[-.4ex]
      \hbox{\begin{tikzinline}[node distance=1.5em]
          \event{wx0}{\DW{x}{0}}{}
          \event{wx1}{\DW{x}{1}}{right=of wx0}
          \raevent{wy1}{\DW[\mRA]{y}{1}}{right=of wx1}
          \event{ry1}{\DR{y}{1}}{right=2.5em of wy1}
          \event{rx0}{\DR{x}{0}}{right=of ry1}
          \sync{wx1}{wy1}
          \po{ry1}{rx0}
          \rf{wy1}{ry1}
          \wk[in=-15,out=-165]{rx0}{wx1}
          \wk{wx0}{wx1}
        \end{tikzinline}}
    \end{gathered}
  \end{gather*}
  Under \EGC{}, this is explained by dropping the order
  $\DRP{y}{1}\xpo\DRP{x}{0}$, because \armeight{} does not include control
  dependencies between reads in the locally-ordered-before relation.
  \begin{gather*}
    \hbox{\begin{tikzinline}[node distance=1.5em]
        \event{wx0}{\DW{x}{0}}{}
        \event{wx1}{\DW{x}{1}}{right=of wx0}
        \raevent{wy1}{\DWRel{y}{1}}{right=of wx1}
        \event{ry1}{\DR{y}{1}}{right=2.5em of wy1}
        \event{rx0}{\DR{x}{0}}{right=of ry1}
        \sync{wx1}{wy1}
        % \po{ry1}{rx0}
        \rf{wy1}{ry1}
        \wk[in=-15,out=-165]{rx0}{wx1}
        \wk{wx0}{wx1}
      \end{tikzinline}}  
  \end{gather*}
\end{example}

Since we do not distinguish control dependencies from other dependencies, we
are forced to drop all dependencies between reads.  In order to do so, we use
the uninterpreted symbol $\RW$.

\begin{definition}[\xRRD]
  \label{def:pomsets-rr}
  Update \refdef{def:pomsets-trans} to:
  % \begin{enumerate}
  % \item[\ref{T3})]
  %   $\labelingForm(\aEv)$ implies
  %   $\labelingForm_1(\aEv)[\TRUE/\RW]$ if $\labelingAct_1(\aEv)$ is a write,
  %   \\
  %   $\labelingForm(\aEv)$ implies
  %   $\labelingForm_1(\aEv)[\FALSE/\RW]$ otherwise.
  % \end{enumerate}
  \begin{enumerate}
  \item[\ref{L4})]
    $\aTr{\bEvs}{\bForm}$ implies $\aVal{=}\aReg\limplies\bForm$, 
  \item[\ref{L5})]
    % $\aTr{\cEvs}{\bForm}$ implies $\RW\limplies(\aVal{=}\aReg\lor\aLoc{=}\aReg)\limplies\bForm$,
    $\aTr{\cEvs}{\bForm}$ implies $(\aVal{=}\aReg\lor\RW)\limplies\bForm$, when $\aEvs\neq\emptyset$,
  \item[{\labeltext[L6]{L6)}{L6}}] 
    $\aTr{\dEvs}{\bForm}$ implies $\bForm$, when $\aEvs=\emptyset$.
  \end{enumerate}  
\end{definition}
\ref{L4} is the \emph{dependent} case,
\ref{L5} the \emph{independent} case, and
\ref{L6} the \emph{empty} case. 
 Recall the substitutions performed by $\ref{sTOP}$
(\refdef{def:pomsets-top}).
For writes, \ref{L5} is the same as \ref{L6} (since $\RW$
is $\TRUE$).  For reads, \ref{L5} is the
same as \ref{L4} (since $\RW$ is $\FALSE$).
% For reads, the dependent and independent predicate transformers are identical
% .  For writes, \ref{L5} requires
% $\aTr{\cEvs}{\bForm}$ implies $\bForm$ for the independent case.

One reading of \ref{L5} is that when satisfying a precondition $\aForm$ it is
safe to use the value of the read, as long as the action is not a write.  If
the pomset is empty, however, there is no value $\aVal$ to use.  Therefore
the best we can do is to emulate skip, as in \ref{L6}.  % In order to
% eventually arrive at a top-level pomset, this means that subsequent code must
% be independent of $\aReg$.

\begin{example}
  Revisiting \refex{ex:rrd1} and eliding irrelevant transformers:
  \begin{align*}
    \begin{gathered}[t]
      \PR{y}{r}
      \\
      \hbox{\begin{tikzinline}[node distance=0.5em and .5em]
          \event{b}{\DR{y}{1}}{}
          \xform{x1i}{(1{=}r\lor\RW)\limplies\bForm}{right=of b} %x1d}
        \end{tikzinline}}  
    \end{gathered}  
    &&
    \begin{gathered}[t]
      \IF{r}\THEN s\GETS x\FI
      \\
      \hbox{\begin{tikzinline}[node distance=0.5em and 1.5em]
          \event{c}{r\mid\DR{x}{0}}{}
          % \xform{x1d}{\bForm[\TRUE/\Dx{\aLoc}]}{below=of c}
          % \xform{x1i}{\lnot\Qw{y}\land\bForm[\TRUE/\Dx{y}]}{below=of x1d}
          % \xo{c}{x1d}
        \end{tikzinline}}  
    \end{gathered}  
  \end{align*}
  Composing sequentially:
  \begin{align*}
    \begin{gathered}[t]
      \PR{y}{r}
      \SEMI
      \IF{r}\THEN s\GETS x\FI
      \\
      \hbox{\begin{tikzinline}[node distance=0.5em and 1.5em]
          \event{b}{\DR{y}{1}}{}
          \event{c}{(1{=}r\lor\RW)\limplies r\mid\DR{x}{0}}{right=of b}
          %\xform{x1i}{\RW\limplies\bForm}{right=of b} %x1d}
        \end{tikzinline}}  
    \end{gathered}  
  \end{align*}
  At top-level, $\ref{sTOP}$ yields:
  \begin{align*}
    \begin{gathered}[t]
      % \PR{y}{r}
      % \SEMI
      % \IF{r}\THEN s\GETS x\FI
      % \\
      \hbox{\begin{tikzinline}[node distance=0.5em and 1.5em]
          \event{b}{\DR{y}{1}}{}
          \event{c}{(1{=}r\lor\FALSE)\limplies r\mid\DR{x}{0}}{right=of b}
          %\xform{x1i}{\RW\limplies\bForm}{right=of b} %x1d}
        \end{tikzinline}}  
    \end{gathered}  
  \end{align*}
  The precondition of $\DRP{x}{0}$ is a tautology, as required.
\end{example}

\subsection{Full semantics for ARM}

\refdef{def:pomsets-arm} combines all of the features of
\textsection\ref{sec:q}--\ref{sec:arm}.  
% Only \ref{L5} and \ref{T3} are changed from \refdef{def:pomsets-down}.

\begin{definition}[\xCO/\xRASC/\xDGR/\xRRD]
  \label{def:pomsets-arm}
  Let $[\aForm/\Q{}]$ be the substitution $[\aForm/\Qr{*}][\aForm/\Qw{*}][\aForm/\Qsc]$.  
  Update \refdef{def:pomsets-trans} to: % (\ref{L4} unchanged):
  \begin{enumerate}
  \item[\ref{S3})]
    $\labelingForm(\aEv)$ implies $\QS{\aLoc}{\amode}\land\aExp{=}\aVal$,
  \item[\ref{L3})]
    $\labelingForm(\aEv)$ implies $\QL{\aLoc}{\amode}$.
  % \item[\ref{T3})]
  %   $\labelingForm(\aEv)$ implies
  %   $\labelingForm_1(\aEv)[\TRUE/\Q{}][\TRUE/\RW]$ if $\labelingAct_1(\aEv)$ is a write,
  %   \\
  %   $\labelingForm(\aEv)$ implies
  %   $\labelingForm_1(\aEv)[\TRUE/\Q{}][\FALSE/\RW]$ otherwise.
  \end{enumerate}
  \begin{enumerate}
  \item[\ref{S4})]
    $\aTr{\bEvs}{\bForm}$ implies
    \begin{math}
      %(\Qw{\aLoc}\limplies\aExp{\neq}\aVal)\land
      \bForm 
      \DS{\aLoc}{\amode}
      \land\aExp{=}\aVal %\noSUB{[\aExp/\aLoc]},
    \end{math}
  \item[\ref{S5})]
    $\aTr{\cEvs}{\bForm}$ implies
    \begin{math}
      %\lnot\Qw{\aLoc}\land 
      \bForm %\noSUB{[\aExp/\aLoc]}.
      \DS{\aLoc}{\amode}
      [\FALSE/\QS{\aLoc}{\amode}]
    \end{math}
  \item[\ref{L4})]
    $\aTr{\bEvs}{\bForm}$ implies $\aVal{=}\aReg\limplies\bForm$, 
  \item[\ref{L5})]
    % $\aTr{\cEvs}{\bForm}$ implies $\lnot\Qx{\aLoc} \land \DL{\aLoc}{\amode} \land(\RW\limplies(\aVal{=}\aReg\lor\aLoc{=}\aReg)\limplies\bForm)$,
    % \item[\ref{L6})]
    $\aTr{\cEvs}{\bForm}$ implies
    \begin{math}
      % \lnot\Qx{\aLoc} \land
      \DL{\aLoc}{\amode}{} \land
      ((\aVal{=}\aReg\lor\RW)\limplies\bForm[\FALSE/\QL{\aLoc}{\amode}]),
      %\textwhen \aEvs\neq\emptyset
    \end{math}
  \item[\ref{L6})] 
    $\aTr{\dEvs}{\bForm}$ implies
    \begin{math}
      \DL{\aLoc}{\amode}{} \land
      \bForm,
    \end{math}
    when $\aEvs=\emptyset$.
  \end{enumerate}  
\end{definition}

\begin{example}
  \xRRD{} does not adversely affect subscription (\refex{ex:subscription}).
  \begin{align*}
    \begin{gathered}[t]
      \PR[\mRA]{y}{r}
      \\
      \hbox{\begin{tikzinline}[node distance=0.5em and .5em]
          \raevent{b}{\QL{y}{\mRA}\mid\DR[\mRA]{y}{1}}{}
          \xform{x1i}{\Dx{y}\land(\RW\limplies\bForm[\FALSE/\QL{\aLoc}{\mRA}])}{right=of b} %x1d}
        \end{tikzinline}}  
    \end{gathered}  
    &&
    \begin{gathered}[t]
      \PR{x}{s}
      \\
      \hbox{\begin{tikzinline}[node distance=0.5em and 1.5em]
          \event{c}{\QL{y}{\mRA}\mid\DW{x}{0}}{}
        \end{tikzinline}}  
    \end{gathered}  
  \end{align*}
  Since $\QL{y}{\mRA}[\FALSE/\QL{\aLoc}{\mRA}]$ is false, we have:
  \begin{align*}
    \begin{gathered}[t]
      \PR[\mRA]{y}{r}
      \SEMI
      \PR{x}{s}
      \\
      \hbox{\begin{tikzinline}[node distance=0.5em and 1.5em]
          \raevent{b}{\QL{y}{\mRA}\mid\DR[\mRA]{y}{1}}{}
          \event{c}{\Dx{y}\land(\RW\limplies\FALSE)\mid\DW{x}{0}}{right=of b}
        \end{tikzinline}}  
    \end{gathered}  
  \end{align*}
  Applying $\ref{sTOP}$ yields:
  \begin{align*}
    \begin{gathered}[t]
      % \PR[\mRA]{y}{r}
      % \SEMI
      % \PR{x}{s}
      % \\
      \hbox{\begin{tikzinline}[node distance=0.5em and 1.5em]
          \raevent{b}{\TRUE\mid\DR[\mRA]{y}{1}}{}
          \event{c}{\Dx{y}\land(\FALSE\limplies\FALSE)\mid\DW{x}{0}}{right=of b}
        \end{tikzinline}}  
    \end{gathered}  
  \end{align*}
  Although $(\FALSE\limplies\FALSE)$ is a tautology, $\Dx{y}$ is not.  Thus,
  the pomset can only contribute to a top-level pomset if the thread is
  preceded by a relaxed write to $y$, allowing $\DRP[\mRA]{y}{1}$ to be
  downgraded to a relaxed read.
\end{example}

Every \armeight{} execution is allowed by \refdef{def:pomsets-arm}.  The
proof of this fact is simplified by the recent characterization of
\armeight{} in terms of \emph{external global completion} (\EGC{})
\cite[\textsection B2.3.6]{arm-reference-manual}.  Under \EGC{}, an
\armeight{} execution is a linearization of per-location program order and a
subset of local-order.  Every such linearization is also a valid pomset under
\refdef{def:pomsets-arm}.

% \begin{definition}
%   Let $[\aForm/\RW{}]$ be the substitution that replaces $\RW{}$ by $\aForm$.
% \end{definition}


% \subsection{Proof Strategy}


\begin{comment}
  Internal visibility:
  If $\bEv\xpoloc\aEv$ then either
  \begin{itemize}
  \item %$\bEv\xca\aEv$:
    $\bEv:\DWP{x}{1}\xco\aEv:\DWP{x}{2}$ 
  \item %$\bEv\xrfx\aEv$:
    $\bEv:\DWP{x}{1}\xrfx\aEv:\DRP{x}{1}$
  \item %$\bEv\mathrel{\xco\xrfx}\aEv$:
    $\bEv:\DWP{x}{1}\xco\DWP{x}{2}\xrfx\aEv:\DRP{x}{2}$
  \item
    $\bEv:\DRP{x}{0}\xfr\aEv:\DWP{x}{1}$\\
    $\bEv:\DRP{x}{0}\xrfxinv\DWP{x}{0}\xco\aEv:\DWP{x}{1}$
  \item %$\bEv\mathrel{\xfr\xrfx}\aEv$:
    $\bEv:\DRP{x}{0}\xfr\DWP{x}{1}\xrfx\aEv:\DRP{x}{1}$\\
    $\bEv:\DRP{x}{0}\xrfxinv\DWP{x}{0}\xco\DWP{x}{1}\xrfx\aEv:\DRP{x}{1}$
  \item %$\bEv\mathrel{\xrfxinv\xrfx}\aEv$:
    $\bEv:\DRP{x}{1}\xrfxinv\DWP{x}{1}\xrfx\aEv:\DRP{x}{1}$
    % \begin{tikzinline}
    %   \node(a){$\DWP{x}{1}$};
    %   \node(b)[right=1.5em of a]{$\DRP{x}{1}$};
    %   \node(c)[right=1.5em of b]{$\DRP{x}{1}$};
    %   \rfx{a}{b}
    %   \rfx[out=15,in=165]{a}{c}
    % \end{tikzinline}
  \end{itemize}
  If $\bEv\xpoloc\aEv$ then the following are impossible:
  \begin{itemize}
  \item $\bEv:\DWP{x}{2}\xcoinv\aEv:\DWP{x}{1}$ 
  \item $\bEv:\DWP{x}{1}\xfrinv\aEv:\DRP{x}{0}$ 
  \item $\bEv:\DRP{x}{1}\xrfxinv\aEv:\DWP{x}{1}$ 
  \item $\bEv:\DRP{x}{2}\xrfxinv\DWP{x}{2}\xcoinv\aEv:\DWP{x}{1}$
  \item $\bEv:\DRP{x}{1}\xrfxinv\DWP{x}{1}\xfrinv\aEv:\DRP{x}{0}$
  \end{itemize}
\end{comment}












\endinput


\begin{figure*}
\begin{verbatim}
let IM0 = loc & ((IW * (M\IW)) | ((W\FW) * FW))

(* Local read successor *)
let lrs = [W]; po-loc \ intervening-write(po-loc); [R]

(* Local write successor *)
let lws = po-loc; [W]

(* Dependency-ordered-before *)
let dob =  addr | data               
        | (addr | data); lrs
        | addr; po; [W]         |  addr; po; [ISB]; po; [R]
        | ctrl;     [W]         |  ctrl;     [ISB]; po; [R]
our dob = addr; [W]
        | data; [W]
        | ctrl; [W]

(* Barrier-ordered-before *)
let bob = [A | Q]; po            | po; [dmb.full]; po          | po; ([A];amo;[L]); po
        | po; [L]                | [R]; po; [dmb.ld]; po
        | [L]; po; [A]           | [W]; po; [dmb.st]; po; [W]

(* Locally-ordered-before *)                let aob = rmw 
let rec lob = lws | dob | bob | lob; lob            | [W & range(rmw)]; lrs; [A | Q] 

(* Internal visibility requirement *)
acyclic po-loc | fr | co | rf as internal

(* External global completion requirement *)
let gc-req = (W * _) | (R * _) & ((range(rfe) * _) | (rfi^-1; lob))
let preorder-gcb = IM0 | lob & gc-req

with gcb from linearisations(M, preorder-gcb)
~empty gcb

let gcbl = gcb & loc
let rf-gcb = (W * R) & gcbl \ intervening-write(gcbl)
let co-gcb = (W * W) & gcbl

call equal(rf, rf-gcb)
call equal(co, co-gcb)

James:
let po-conflict = po-loc \ (R * R)
Claim we can add po-conflict to preorder-gcb without changing things
let preorder-gcb = IM0 | lob & gc-req | po-conflict
\end{verbatim}
  \caption{ARM}
  \label{fig:arm}
\end{figure*}

\textcolor{red}{[This is out of date, from an old submission]}


In this section, we consider the fragment of our language without
restriction.  For simplicity, we allow release and acquire synchronization
but ban fences.  We assume that all memory locations are initialized to $0$
and parallel-composition occurs only at top level.  We take the set of memory
locations to be finite.  In other words, we assume that programs have the
form
\begin{displaymath}
  {\aLoc_1}\GETS{0}\SEMI
  \cdots\SEMI
  {\aLoc_m}\GETS{0}\SEMI
  (\aCmd^1 \PAR \cdots \PAR \aCmd^n)
\end{displaymath}
where $\aCmd^1$, \ldots, $\aCmd^n$ do not include composition, restriction or
fence operations.

Our language can be translated to ARM using \texttt{ldr} for relaxed read,
\texttt{ldar} for acquiring read, \texttt{str} for relaxed write, and
\texttt{stlr} for releasing write.  Relative to the ARM specification, we
have removed loops and read-modify-write (RMW) operations, in addition to
fences\footnote{We leave out fences for simplicity.  Following
  \citet{DBLP:journals/pacmpl/PodkopaevLV19}, our $\FENCE$ instruction can be
  translated to \texttt{dmb.sy}, since it has release-acquire semantics.
  Acquire fences map to \texttt{dmb.ld}, and release fences to
  \texttt{dmb.sy} --- \texttt{dmb.st} does not provide order to prior
  reads.}.

We show that any ARM-consistent execution graph for this sublanguage can be
considered an execution of our semantics.  
% 
% Syntactically, we drop the superscript \textsf{rlx} on relaxed reads and
% writes; in addition, we use structured conditionals rather than the more
% general \textsf{goto}.  We refer to this sublanguage as $\muIMM$.
% (Because the source language lacks RMW operations, the ``is
% exclusive'' flag on every read will be \textsf{not-ex} and the RMW mode on
% every write will be \textsf{normal}.)
% 
Due to space limitations, we do not include a full description ARM
consistency in the main text .
Here we provide a birds eye view of the details, drawing on the intuitions gleaned from~\citep{DBLP:journals/pacmpl/PulteFDFSS18}.  
Interested readers should see \textsection\ref{sec:arm:proof}
for further details.

An ARM execution graph $G$ defines many relations, including program order
($\rpox$), reads-from ($\rrfx$), coherence ($\rco$) and several dependency
orders.  From these are derived:
\begin{itemize}
\item ${\rpoloc}$, which is the subrelation of $\rpox$ that only relates
  actions on the same location,
\item ${\rob}$, which is required to be acyclic (\ref{external}), and
\item $\reco$, with the requirement that ${\rpoloc}\cup{\reco}$ be acyclic (\ref{sc-per-loc}).
\end{itemize}
% Let $G$ be an execution graph satisfying the ARM consistency 
% requirements.
Given an execution graph $G$, we say that $\aEv$ is an \emph{internal read} if $\aEv\in\fcodom({\rpox}\cap {\rrfx})$.

The ${\rob}$ order is an acyclic global order on events, agreed upon by all threads, reflecting the progress of time in an \armeight\ execution.  The cross thread component of the ordering is induced by the ordering on conflicting actions on the same location from different threads.    The intra thread component of the ordering is induced by barrier ordering and data ordering.  Notably, these dependencies  are determined syntactically.  In particular, $\rob$
may not necessarily include the intra thread component of $\rpoloc$ ordering.  

This motivates the translation of an \armeight\ execution into our setting.  In our setting, the progress of time is given by $\lt$.   We accommodate intra-thread reordering by internal read actions, thus excusing us from the obligation of placing them on the global $\lt$-timeline.  

Formally, from $G$ we construct a candidate pomset $\aPS$ as follows:
\begin{itemize}
\item $\Event= \textsf{E}$,
\item $\labelingAct(\aEv)=\tau \mathsf{lab}(e)$, if $\aEv$ is a relaxed
  internal read, 
\item $\labelingAct(\aEv)=\mathsf{lab}(e)$, if $\aEv$ is not a relaxed
  internal read,
\item $\labelingForm(\aEv)=\TRUE$,
  % \item ${\le} = {\rob}^?$, where $?$ denotes reflexive closure, and
\item ${\gtN} = ({\rob} \cup {\reco})^*$, where $*$ denotes reflexive and transitive closure.
\end{itemize}

\begin{theorem}
  If $G$ is ARM consistent, the constructed candidate satisfies the
  requirements for a top-level memory model pomset.
\end{theorem}
Any $\lt$-ordering imposed in our model
is enforced by \armeight, since our notion of semantic dependency is more
permissive than \armeight 's syntactic dependency.  So, the heart of the proof is showing the acyclicity of $({\rob} \cup {\reco})^*$ for the events under consideration.  Since the cross thread portion of $\reco$ ($\rcoe,\rfre,\rrfe$) is included in $\rob$, this result is really about the influence of $\reco \cap \rpox$.  Our translation of \armeight 's \rrfi\  as silent internal actions removes them from order considerations.   Consequently, we only have to consider the suborder of ${\rob}$ derived without ever using  $\rrfi$ for the following key property demonstrated in \textsection\ref{sec:arm:proof}.
\begin{lemma}\label{extendob}
  Let $\aEv, \bEv$ be distinct events and $\bEv'\ (\xob\cap \xpox) \ \bEv\ ((\xeco\cap \xpox) \setminus \xrfi) \  \aEv\ (\xob \cap \xpox)  \ \aEv'$.  Then $\bEv' \xob \aEv'$.
\end{lemma}



\begin{remark}[Proof for TSO]
  The proof for compilation into \tso\ is very similar.  In particular the facts listed above hold for \tso, where $\rob$ is replaced by (the
  transitive closure of) the propagation relation defined for \tso\ 
  \citep{alglave}.
\end{remark}

\section{Proof of compilation for ARMv8}
\label{sec:arm:proof}
\textcolor{red}{[This is out of date, from an old submission]}

In this section, we develop the proof of correctness of compilation to \armeight.  In order to ease readability, we reproduce the definitions from the main text. 



Given a relation $R$, $R^?$ denotes reflexive closure, $R^+$ denotes
transitive closure and $R^*$ denotes reflexive and transitive closure.  Given relations $R$ and $S$, $R;S$ denotes composition.


The ARMv8 model is described using the following relations.
\begin{itemize}
\item $\IDR$, $\IDW$, $\IDAcq$, $\IDRel$: identity on reads, writes, acquires
  and releases.
  % \item $\IDR$ identity on reads
  % \item $\IDW$: identity on writes
  % \item $\IDAcq$: identity on acquires
  % \item $\IDRel$: identity on releases
\item $\IDLoc$: relates any two events that touch the same location.
\item $\rpox$: program order.
\item $\rdata$, $\rctrl$, $\raddr$: data, control and address dependencies.
\item $\rrfx$: reads-from. $\rrfx^{-1}$ relates each read to a matching write
  on the same location.
\item $\rco$: coherence, which is a total order on the writes to a single
  location.
\item ${\rfr}\eqdef{\rco};\rrfx^{-1}$: from-read, which relates reads to
  subsequent writes.
\end{itemize}
For any relation, the cross-thread subrelation is denoted by appending $e$;
the intra-thread subrelation is denoted by appending $i$.  For example,
${\rrfe}\eqdef{\rrfx}\setminus{\rpox}$ and ${\rrfi}\eqdef{\rrfx}\cap{\rpox}$.
The subrelation restriction attention to actions on the same location is
given by appending $\mathsf{loc}$.  For example, ${\rpoloc}\eqdef{\rpox}\cap{\IDLoc}$.

The ARMv8 model defines the following relations.
In our presentation, we have elided rules concerning fences and RMW operations.
\begin{align*}
  \tag{Extended coherence}
  {\reco} &\eqdef {\rrf} \cup {\rfr} \cup {\rco}
  \\
  \tag{Observed externally}
  {\robs} &\eqdef \smash{
    {\rrfe} \cup {\rfre} \cup {\rcoe}
  }
  \\
  \tag{Dependency order}
  {\rdob} &\eqdef\smash{
    ({\raddr}\cup{\rdata}); {\rrfi}^?
    \cup ({\rctrl}\cup{\rdata}); {\IDW}; {\rcoi}^?
    \cup {\raddr}; {\rpox}; {\IDW}
  }
  \\
  \tag{Barrier order}
  {\rbob} &\eqdef\smash{
    {\IDAcq}; {\rpox}
    \cup {\rpox};{\IDRel}; {\rcoi}^?
  }
  \\
  \tag{Acyclic order}
  {\rob} &\eqdef\smash{
    ({\robs} \cup {\rdob} \cup {\rbob})^+
  }
\end{align*}
\begin{definition}
  An RMW-free and fence-free execution is \emph{ARM-consistent} if
  \begin{align*}&
    \tag{\textsc{$\rrfx$-completeness}}\label{rf-comp}
    \fcodom(\rrfx)=\fdom(\rreads)
    \\[-1ex]&
    \tag{\textsc{$\rco$-totality}}\label{co-tot}
    \text{For every location $\aLoc$, $\rco$ totally orders the writes of $\aLoc$}  
    \\[-1ex]&
    \tag{\textsc{sc-per-loc}}\label{sc-per-loc}
    {\rpoloc} \cup {\rrfx} \cup {\rfr} \cup {\rco}\;\text{is acyclic}
    \\[-1ex]&
    \tag{\textsc{external}}\label{external}
    {\rob}\;\text{is acyclic}
  \end{align*}
\end{definition}

% Use these to refer to the rules in text:
% \ref{rf-comp} 
% \ref{co-tot}
% \ref{sc-per-loc}
% \ref{external}


Given an execution graph $G$, we say that $\aEv$ is an \emph{internal read} if
$\aEv\in\fcodom(\mathsf{po}\cap \mathsf{rf})$.    We are going to translate internal reads of execution graphs into internal reads of the semantics.  

From $G$ we construct a candidate pomset $\aPS$ as follows:
\begin{itemize}
\item $\Event= \textsf{E}$,
\item $\labelingAct(\aEv)=\tau \mathsf{lab}(e)$, if $\aEv$ is a relaxed
  internal read, 
\item $\labelingAct(\aEv)=\mathsf{lab}(e)$, if $\aEv$ is not a relaxed
  internal read,
\item $\labelingForm(\aEv)=\TRUE$,
  % \item ${\le} = {\rob}$, and
\item ${\gtN} = ({\rob} \cup {\reco})^*$
\end{itemize}
To reempphasize, in this candidate pomset, $\rob$ is calculated by considering the definition of $\rob$ without $\rrfi$, ie.:
\begin{align*}
  \tag{Dependency order}
  {\rdob} &\eqdef\smash{
    ({\raddr}\cup{\rdata});
    \cup ({\rctrl}\cup{\rdata}); {\IDW}; {\rcoi}^?
    \cup {\raddr}; {\rpox}; {\IDW}
  }
  \\
  \tag{Barrier order}
  {\rbob} &\eqdef\smash{
    {\IDAcq}; {\rpox}
    \cup {\rpox};{\IDRel}; {\rcoi}^?
  }
  \\
  \tag{Acyclic order}
  {\rob} &\eqdef\smash{
    ({\robs} \cup {\rdob} \cup {\rbob})^+
  }
\end{align*}


We show that $\aPS$ is a top-level pomset, reasoning as follows.
% We establish the criteria for a top-level memory-model pomset:
\begin{itemize}
\item ${\le}$ is a partial order.  This holds since $G.{\rar}$ is acyclic.
  % \item If $\bEv \le \aEv$ then $\bEv \gtN \aEv$.  By construction.
  % \item If $\bEv \le \aEv$ and $\aEv \gtN \bEv$ then $\bEv = \aEv$.  Proved below.
  % \item If $\cEv \le \bEv \gtN \aEv$ or $\cEv \gtN \bEv \le \aEv$ then
  %   $\cEv \gtN \aEv$. By construction.
  % \end{itemize}

  % Next, we establish the criteria for a 3-valued pomset with preconditions (Definition~\ref{def:3pre}).
  % \begin{itemize}
\item $\labelingForm(\aEv)$ implies $\labelingForm(\bEv)$ whenever
  $\bEv\le\aEv$.   Trivial, since every formula is $\TRUE$.
  % \item $\aPS$ is $\aLoc$-coherent; that is, when restricted to events that
  %   read or write $\aLoc$, $\gtN$ forms a partial order.
  % \end{itemize}

  % Finally, we establish the criteria for a top-level pomset
  % (Definition~\ref{def:x-closed}).
  % \begin{itemize}
\item $\aEv$ is location independent. Trivial, since every formula is $\TRUE$.
\item If $\aEv$ reads $\aVal$ from $\aLoc$, then there is some $\bEv$ such that
  \begin{itemize}
  \item $\bEv \lt \aEv$,  
  \item $\bEv$ writes $\aVal$ to $\aLoc$, and
  \item if $\cEv$ writes to $\aLoc$
    then either $\cEv \gtN \bEv$ or $\aEv \gtN \cEv$.
  \end{itemize}    
\end{itemize}

\subsection{Proof that  $({\rob} \cup {\reco})^*$ is irreflexive. }

\paragraph*{Proof of lemma~\ref{extendob}. } 


Let $\aEv, \bEv$ be distinct events and $\bEv'\ (\xob\cap \xpox) \ \bEv\ ((\xeco\cap \xpox) \setminus \xrfi) \  \aEv\ (\xob \cap \xpox)  \ \aEv'$.  Then $\bEv' \xob \aEv'$.

\begin{proof}
  If $\bEv'$ is an acquire,  or $\aEv$ is an release, or $\aEv'$ is a release, result is immediate.

  We next consider the case where $\aEv$ is a read.  In this case,  $\bEv$ is a write.  Since $\bEv\ ((\xeco\cap \xpox) \setminus \xrfi) \  \aEv$, there is a write $\bEv_1$ such that $ \bEv \xcoe\ \bEv_1 \ \xrfe\ \aEv' $.  So, $\bEv \xob \aEv$ and result follows in this case. 


  So, it suffices to prove the following assuming that $\bEv'$ is not an acquire and $\aEv'$ is not a release and $\aEv$ is not a release or a read and $\aEv, \bEv$ are distinct.
  \begin{itemize}
  \item If $\bEv'\ (\xob\cap \xpox)  \ \bEv(\xeco\cap\xpox)\aEv$ then $\bEv'\xob\aEv$.
  \item If $\bEv\ (\xeco\cap\xpox) \ \aEv(\xob\cap\xpox)\aEv'$ then $\bEv\xob\aEv'$.
  \end{itemize}


  We first prove that if $\bEv'\ (\xob \cap \xpox) \ \bEv\ (\xeco \cap \xpox) \ \aEv$ then $\bEv'\xob\aEv$.   Proof proceeds by cases on the witness for $\bEv'\ (\xob\cap \xpox) \ \bEv$. 
  \begin{itemize}
  \item  If $\bEv' \xbob  \bEv$, then: 
    \[ \bEv'\ (\smash{
        {\IDAcq}; {\rpox}
        \cup {\rpox};{\IDRel}; {\rcoi}^?) \ 
      }
      \bEv
    \]
    Since $\bEv'$ is not an acquire, $\bEv' ({\rpox};{\IDRel}; {\rcoi}^?) \bEv$, so $\bEv$ is a write.  Since $\aEv$ is not a read,  $\bEv \xcoi\ \aEv$. Thus, result follows.

  \item If $\bEv' \xdob  \bEv$, then: 
    \[ \bEv'\ 
      \smash{
        ( ({\rctrl}\cup{\rdata}); {\IDW}; {\rcoi}^?
        \cup {\raddr}; {\rpox}; {\IDW}
      } \
      \bEv
    \]
    So, $\bEv$ is a write.  Since $\aEv$ is also a write, we deduce that 
    \[ \bEv'\ 
      \smash{
        ( ({\rctrl}\cup{\rdata}); {\IDW}; {\rcoi}^?
        \cup {\raddr}; {\rpox}; {\IDW}
      } \
      \aEv
    \]
  \end{itemize}


  We next prove  that if $\bEv\ (\xeco \cap \xpox) \ \aEv\ (\xob\cap \xpox) \ \aEv'$ then $\bEv\xob\aEv'$, under the assumptions that  $\aEv'$ is not a release and $\aEv$ is not a release or a read and $\aEv, \bEv$ are distinct.


  Proof proceeds by cases on the witness for $\aEv (\xob\cap \xpox) \aEv'$.  

  \begin{itemize}
  \item  If $\aEv \xbob  \aEv'$, then: 
    \[ \aEv\ (\smash{
        {\IDAcq}; {\rpox}
        \cup {\rpox};{\IDRel}; {\rcoi}^?) \ 
      }
      \aEv'
    \]
    Since $\aEv$ is not a read, $\aEv ({\rpox};{\IDRel}; {\rcoi}^?) \aEv'$.  Result follows since  $\bEv \xpox\ \aEv$.


  \item If $\aEv \xdob  \aEv'$, then $\aEv$ is a read.  \qedhere

  \end{itemize}
\end{proof}


\begin{lemma}\label{obeco1}
  If $\bEv\xob\aEv$ then $\lnot(\aEv\xeco\bEv)$.

  \begin{proof}
    Proof by contradiction.  Let 
    \[ \aEv \xob \aEv' \xeco \bEv' \xob \bEv \xob \cEv \xob \cEv' \xob \aEv \]
    where $\aEv' \xpox \bEv'$.

    By lemma~\ref{extendob}, if $\aEv \not=\aEv'$, we deduce $\aEv \xob \bEv'$, and thus $\aEv \xob \bEv$.  If $\bEv \not=\bEv'$, we deduce $\aEv' \xob \bEv$ and thus $\aEv \xob \bEv$.

    Thus, if $\aEv \not=\aEv'$ or $\bEv \not=\bEv'$, then there is a cycle $\aEv \xob \bEv \xob \cEv \xob \cEv' \xob \aEv$.  

    So we can assume that  $\aEv' = \aEv$, $\bEv' = \bEv$ and 
    \[ \aEv  \xeco \bEv \xob \cEv \xob \cEv' \xeco \aEv \]
    where all of $\aEv, \bEv, \cEv, \cEv'$ access the same location and at least one of $\aEv,\bEv$ is a write, at least one of $\aEv,\cEv'$ is a write, and at least one of $\bEv,\cEv$ is a write.

    We reason by cases.
    \begin{itemize}
    \item If $\cEv'$ is a write or both $(\aEv, \bEv)$ are writes.

      We deduce that $\bEv \xeco \cEv' \xeco \aEv$ and thus $\bEv \xeco \aEv$.

    \item $\cEv'$ is a read.  $\aEv$ is a write.  $\bEv$ is a read.

      In this case $\cEv$ is a write.  From $\cEv \xob \aEv$, we deduce $\cEv \xeco \aEv$. Combining with $\bEv \xeco \cEv$, we deduce that $\bEv \xeco \aEv$.  


    \end{itemize}
    In either case, there is a contradiction $\aEv \xeco \bEv \xeco \aEv$.
  \end{proof}
\end{lemma}


\begin{lemma}\label{obeco2}
  $({\rob} \cup {\reco})^*$ is irrreflexive.
\end{lemma}
\begin{proof}
  The simple case that $\rob; \reco$ is irreflexive is proved above.  The full proof by contradiction.  

  Let $n \geq 1$ be the minimum such that:
  \begin{align*} 
    &\aEv^0_0 \xob \aEv^0_1 \xeco \bEv^0_0 \xob \bEv^0_1  \\
    (\xeco \cap \xob) &  \   \aEv^1_0 \xob \aEv^1_1 \xeco \bEv^1_0 \xob \bEv^1_1 \\
    (\xeco \cap \xob) & \ \ldots \\
    & \ldots \bEv^n_1 \\
    (\xeco \cap \xob) & \  \aEv^0_0
  \end{align*}
  where  for all $i$, we have:
  \[ \aEv^i_0 \xpox \aEv^i_1 (\xeco \cap \xpox) \bEv^i_0 \xpox \bEv^i_1\] and 
  \[ \neg (\bEv^i_1 \xpox (\aEv^{(i+1) \mod n}_0 \]

  For any $i$, if $\aEv^i_0 \not= \aEv^i_1$ or $\bEv^i_0 \xpox \bEv^i_1$, via lemma~\ref{extendob}, we deduce that $\aEv^i_0  \xob \bEv^i_1$, contradicting minimality of $n$.  

  So, we can assume that $n \geq 1$ is such that:
  \begin{align*} 
    &\ \aEv^0 \xeco  \bEv^0 \\
    (\xeco \cap \xob) &  \   \aEv^1  \xeco \bEv^1 \\
    (\xeco \cap \xob) & \ \ldots \\
    & \ldots \bEv^n \\
    (\xeco \cap \xob) & \ \aEv^0
  \end{align*}
  which is a contradiction since it is a cycle in $\xeco$ and since at least one of $\aEv^i ,\bEv^i$ is a write for all $i$. 
\end{proof}

