\section{Future Work}

This paper is the first to present a direct denotational semantics for
sequential composition in a relaxed memory model which can be
efficiently compiled to modern CPUs. There is, as usual, more research
to be done.

We have not treated loops in this model, though we expect that the usual
approach of showing continuity for all the semantic operations with respect to set inclusion
would go through. \citet{DBLP:conf/esop/PaviottiCPWOB20} use step-indexing to account for
loops; a similar approach could be applied here.

In \S\ref{sec:arm1} we presented a compilation strategy to \armeight{}
for a simplified model, but which introduces fences to acquiring
reads. These fences are not required in \S\ref{sec:arm2}, but at the
cost of model complexity. It would be illuminating to find out what
the performance penalty is for these fences.

An earlier version of this paper has been mechanized in Agda; it would be
reassuring to update the mechanization to bring it in line with the current state.

We don't handle access elimination.

% We have presented the first model of relaxed memory that treats sequential
% composition as a first-class citizen. The model builds directly on \jjr{}.

% For sequential composition, parallel composition and the conditional, we
% believe that the definition is \emph{natural}, even \emph{canonical}.
% For stores and loads, instead, the definition in \reffig{fig:no-addr} is a
% Frankenstein's monster of features.  This complexity is \emph{essential},
% however, not just an accident of our poor choices.  Relaxed memory models must
% please many audiences: compiler writers want one thing, hardware designers
% another, and programmers yet another still.  The result is inevitably full of
% compromise.

% Given that \emph{complexity} cannot be eliminated from relaxed memory models,
% the best one can do is attempt to understand its causes.  We have broken the
% problem into seven manageable pieces, discussed throughout
% \textsection\ref{sec:q}--\ref{sec:complications}.  \refdef{def:pomsets-arm}
% summarizes all the features necessary for efficient implementation on
% \armeight{}.  We discuss address calculation, read-modify-write operations
% and fences in the appendix.

% {Logic} is the thread that sews these features together.

% %A unique feature of our model is the centrality of logic: 

% % study each feature in isolation, as a small delta to the base definition
% % given in \textsection\ref{sec:model}.
% % In we studied eight such
% % features.


% % seem staggeringly complex, we argue that this complexity is unavoidable if
% % one wants all the features that it embodies.  By breaking the definition into
% % its constituent parts, we have shown how each of eight


% \subsection*{Acknowledgements}
% % This paper has been greatly improved by the comments of the anonymous reviewers.
% Riely was supported by the National Science Foundation under
% grant No.~CCR-1617175.

% % It is based upon work supported by the National Science Foundation under
% % Grant No. CCR-1617175. Any opinions, findings, and conclusions or
% % recommendations expressed in this material are those of the author and do
% % not necessarily reflect the views of the NSF.}

