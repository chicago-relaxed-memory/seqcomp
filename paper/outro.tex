\section{Conclusions}

We have presented the first model of relaxed memory that treats sequential
composition as a first-class citizen. The model builds directly on \jjr{}.

For sequential composition, parallel composition and the conditional, we
believe that the definition is \emph{natural}, even \emph{canonical}.
For stores and loads, instead, the definition in \reffig{fig:no-addr} is a
Frankenstein's monster of features.  This complexity is \emph{essential},
however, not just an accident of our poor choices.  Relaxed memory models must
please many audiences: compiler writers want one thing, hardware designers
another, and programmers yet another still.  The result is inevitably full of
compromise.

Given that \emph{complexity} cannot be eliminated from relaxed memory models,
the best one can do is attempt to understand its causes.  We have broken the
problem into seven manageable pieces, discussed throughout
\textsection\ref{sec:q}--\ref{sec:complications}.  \refdef{def:pomsets-arm}
summarizes all the features necessary for efficient implementation on
\armeight{}.  We discuss address calculation, read-modify-write operations
and fences in the appendix.

{Logic} is the thread that sews these features together.

%A unique feature of our model is the centrality of logic: 

% study each feature in isolation, as a small delta to the base definition
% given in \textsection\ref{sec:model}.
% In we studied eight such
% features.


% seem staggeringly complex, we argue that this complexity is unavoidable if
% one wants all the features that it embodies.  By breaking the definition into
% its constituent parts, we have shown how each of eight
