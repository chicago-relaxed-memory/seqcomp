\section{Introduction}
\label{sec:intro}

This paper is about the interaction of two of the fundamental building blocks
of computing: sequential composition and mutable state. One would like to
think that these are well-worn topics, where every issue has been settled,
but this is not the case.

\subsection{Sequential Composition} %Predicate Transformers}

Introductory programmers are taught \emph{sequential abstraction}: that the
program $\aCmd_1\SEMI \aCmd_2$ executes $\aCmd_1$ before $\aCmd_2$.  Since
the late 1960s, we've been able to explain this using logic
\citep{Hoare:1969:ABC:363235.363259}.  In
\citeauthor{DBLP:journals/cacm/Dijkstra75}'s
[\citeyear{DBLP:journals/cacm/Dijkstra75}] formulation, we think of programs
as \emph{predicate transformers}, where predicates describe the state of
memory in the system.  In the calculus of weakest preconditions, programs map
postconditions to preconditions.  We recall the definition of
$\fwp{\aCmd}{\bForm}$ for loop-free code below (where $\aReg$--$\bReg$ range over thread-local \emph{registers}
and $\aExp$--$\bExp$ range over side-effect free \emph{expressions}). 
%\begin{multicols}{2}
  \begin{enumerate}[,label=(\textsc{d}\arabic*),ref=\textsc{d}\arabic*]
  \item
    \begin{math}
      \fwp{\SKIP}{\bForm} = \bForm
    \end{math}
  \item \label{wp-let}
    $\fwp{\LET{\aReg}{\aExp}}{\bForm} = \bForm[\aExp/\aReg]$
    % \item 
    %   \begin{math}
    %     \fwp{\ABORT}{\bForm} = \FALSE
    %   \end{math}
    %   \stepcounter{enumi}
    % \item[] \labeltext[\textsc{d}2]{}{wp-assign}
    %   \begin{enumerate}[leftmargin=0pt]
    %   \item \label{wp-write}
    %     $\fwp{\PW{\aLoc}{\aExp}}{\bForm} = \bForm[\aExp/\aLoc]$
    %   \item \label{wp-let}
    %     $\fwp{\LET{\aReg}{\aExp}}{\bForm} = \bForm[\aExp/\aReg]$
    %   \item \label{wp-read}
    %     $\fwp{\PR{\aLoc}{\aReg}}{\bForm} = \aLoc{=}\aReg\limplies\bForm$ %\bForm[\aLoc/\aReg]$ %
    %   \end{enumerate}
    % \item 
    %   \begin{math}
    %     \fwp{\LET{x}{\aExp}}{\bForm} = \bForm[\aExp/x]
    %   \end{math}
  \item
    \begin{math}
      \fwp{\aCmd_1;\aCmd_2}{\bForm} = \fwp{\aCmd_1}{\fwp{\aCmd_2}{\bForm}}
    \end{math}
  \item
    \begin{math}
      \fwp{\IF{\aExp}\THEN \aCmd_1\ELSE \aCmd_2\FI}{\bForm}= {}
    \end{math}
    \begin{math}
      ((\aExp{\ne}0) \limplies \fwp{\aCmd_1}{\bForm}) \land
      ((\aExp{=}0) \limplies \fwp{\aCmd_2}{\bForm})
    \end{math}
  \end{enumerate}
%\end{multicols}
For this language, the Hoare triple
$\hoare{\aForm}{\aCmd}{\bForm}$ holds exactly when $\aForm \limplies
\fwp{\aCmd}{\bForm}$.
%% We have split
%% \citeauthor{DBLP:journals/cacm/Dijkstra75}'s rule for assignment
%% \eqref{wp-assign} into three cases. In our notation,
%% $\aReg$--$\bReg$ range over thread-local registers, which may be assigned at
%% most once,
%% $\aLoc$--$\cLoc$ range over shared memory references, and
%% $\aExp$--$\bExp$ range over thread-local expressions, which do \emph{not}
%% include
%% $\aLoc$--$\cLoc$.\footnote{Under these assumptions, \eqref{wp-read} is
%%   equivalent to $\fwp{\PR{\aLoc}{\aReg}}{\bForm} = \bForm[\aLoc/\aReg]$.}
%%
This is an elegant explanation of sequential computation in a sequential
context. Note that \ref{wp-let} is sound because a read from a thread-local
register must be fulfilled by a preceding write in the same thread.
In a concurrent context, with shared variables
($\aLoc$--$\cLoc$), the obvious generalizations
\begin{multicols}{2}
\begin{itemize}
\item[{\labeltext[\textsc{d}2b]{(\textsc{d}2b)}{wp-write}}] $\fwp{\LET{\aLoc}{\aExp}}{\bForm} = \bForm[\aExp/\aLoc]$
\item[{\labeltext[\textsc{d}2c]{(\textsc{d}2c)}{wp-read}}] $\fwp{\LET{\aReg}{\aLoc}}{\bForm} = \bForm[\aLoc/\aReg]$
\end{itemize}
\end{multicols}
\noindent
are unsound!  In particular, a read from a shared memory location may be fulfilled by a write
in another thread, invalidating \ref{wp-read}.  (We assume that expressions
% $\aExp$--$\bExp$
do \emph{not} include shared variables.) % $\aLoc$--$\cLoc$.)

Existing approaches to sequential composition in the concurrent context
either assume exclusive access, as in concurrent separation logic
\cite{OHearn:2007:RCL:1235896.1236121}, or abandon the logical approach
altogether, as in the pomset model of
\citet{DBLP:journals/corr/abs-1804-04214}---this model uses syntactic
dependencies and thus dramatically limits compiler optimization.  This leaves
open the question of how to apply logic to racy programs without
overconstraining the implementation.  To understand the solution, one must first
understand the constraints imposed by hardware and compilers.


% Assuming all variables are initialized to $0$,
% $\PW{x}{1}\SEMI\PW{y}{1}$ can be distinguished from $\PW{y}{1}\SEMI\PW{x}{1}$
% by the concurrent observer $\PR{x}{r}\SEMI\PR{y}{s}$

% Microprocessors and compilers do not execute this way, but they attempt to
% preserve the abstraction, at least for sequential code.  For concurrent code,
% all bets are off.

% \citet{DBLP:conf/snapl/MarinoMMNS15} argue that the ``silently shifting
% semicolon'' is problematic for programmers, and thus concurrent languages
% should guarantee sequential abstraction.  But processor and language
% implementors balk at the prospect, due to significant costs.

% These leaves programmers in a bit of pickle: What does that semicolon mean?

% The last decade has seen a slew of research in ``relaxed memory models,''
% which attempt to explain concurrent execution.  The models are mostly
% operational, sometimes with an additional series of axioms to eliminate
% ``bad'' executions.  These models have limited modularity properties,
% typically requiring whole program analysis.

% In this paper, we attempt to 





\subsection{Memory Models}

For single-threaded programs, memory can be thought of as you might
expect: programs write to, and read from, memory references.
This can be thought of as a total order of reads and writes (black arrows),
where each read has a matching \emph{fulfilling} write (green arrows),
for example:
\begin{gather*}
  \THREAD{\PW{x}{0}\SEMI \PW{x}{1}\SEMI \PW{y}{2}\SEMI
    \PR{y}{r}\SEMI \PR{x}{s}}
  \\[-.4ex]
  \nonumber
  \hbox{\begin{tikzinline}[node distance=1.5em]
      \event{wx0}{\DW{x}{0}}{}
      \event{wx1}{\DW{x}{1}}{right=of wx0}
      \event{wy2}{\DW{y}{2}}{right=of wx1}
      \event{ry2}{\DR{y}{2}}{right=3em of wy2}
      \event{rx1}{\DR{x}{1}}{right=of ry2}
      \rf[out=15,in=165]{wy2}{ry2}
      \rf[out=15,in=165]{wx1}{rx1}
      \po{wx0}{wx1}
      \po{wx1}{wy2}
      \po{wy2}{ry2}
      \po{ry2}{rx1}
    \end{tikzinline}}
\end{gather*}

% (In examples, $\aReg$--$\bReg$ range over thread-local registers and $\aLoc$-$\cLoc$
% range over shared memory references.)

This model naturally extends to the case of shared-memory concurrency, leading to a \emph{sequentially consistent}
semantics \cite{Lamport:1979:MMC:1311099.1311750}, in which \emph{program order} inside a thread implies
a total \emph{causal order} between read and write events, for example:
\begin{gather*}
  \THREAD{\PW{x}{0}\SEMI \PW{x}{1}\SEMI \PW{y}{2}}
  \PAR
  \THREAD{\PR{y}{r}\SEMI \PR{x}{s}}
  \\[-.4ex]
  \nonumber
  \hbox{\begin{tikzinline}[node distance=1.5em]
      \event{wx0}{\DW{x}{0}}{}
      \event{wx1}{\DW{x}{1}}{right=of wx0}
      \event{wy2}{\DW{y}{2}}{right=of wx1}
      \event{ry2}{\DR{y}{2}}{right=3em of wy2}
      \event{rx1}{\DR{x}{1}}{right=of ry2}
      \rf[out=15,in=165]{wy2}{ry2}
      \rf[out=15,in=165]{wx1}{rx1}
      \po{wx0}{wx1}
      \po{wx1}{wy2}
      \po{ry2}{rx1}
    \end{tikzinline}}
\end{gather*}

Unfortunately, this model does not compile efficiently to commodity
hardware, resulting in a 37--73\% increase in CPU time on Arm8~\cite{Liu:2019:ASC:3314221.3314611} and,
hence, in power consumption.  Developers of software and compilers have
therefore been faced with a difficult trade-off, between an elegant
model of memory, and its impact on resource usage (such as size of
data centers, electricity bills and carbon footprint). Unsurprisingly,
many have chosen to prioritize efficiency over elegance.

This has led to \emph{relaxed memory models}, in which the requirement of
sequential consistency is weakened to only apply \emph{per-location} and not globally
over the whole program. This allows executions which
are inconsistent with program order, such as:
\begin{gather*}
  \THREAD{\PW{x}{0}\SEMI \PW{x}{1}\SEMI \PW{y}{2}}
  \PAR
  \THREAD{\PR{y}{r}\SEMI \PR{x}{s}}
  \\[-.4ex]
  \nonumber
  \hbox{\begin{tikzinline}[node distance=1.5em]
      \event{wx0}{\DW{x}{0}}{}
      \event{wx1}{\DW{x}{1}}{right=of wx0}
      \event{wy2}{\DW{y}{2}}{right=of wx1}
      \event{ry2}{\DR{y}{2}}{right=3em of wy2}
      \event{rx0}{\DR{x}{0}}{right=of ry2}
      \rf[out=15,in=165]{wy2}{ry2}
      \rf[out=15,in=165]{wx0}{rx0}
      \po{wx0}{wx1}
      \wk[out=-165,in=-15]{rx0}{wx1}
    \end{tikzinline}}
\end{gather*}

In such models, the causal order between events is important,
and includes control and data dependencies, to avoid
paradoxical ``out of thin air'' examples such as:
\begin{gather*}
  \THREAD{\PR{x}{r}\SEMI \IF r \THEN \PW{y}{1} \FI}
  \PAR
  \THREAD{\PR{y}{s}\SEMI \PW{x}{s}}
  \\[-.4ex]
  \nonumber
  \hbox{\begin{tikzinline}[node distance=1.5em]
      \event{rx1}{\DR{x}{1}}{}
      \event{wy1}{\DW{y}{1}}{right=of rx1}
      \event{ry1}{\DR{y}{1}}{right=3em of wy1}
      \event{wx1}{\DW{x}{1}}{right=of ry1}
      \rf[out=15,in=165]{wy1}{ry1}
      \rf[out=165,in=15]{wx1}{rx1}
      \po{rx1}{wy1}
      \po{ry1}{wx1}
    \end{tikzinline}}
\end{gather*}
This candidate execution forms a cycle in causal order, so is disallowed,
but this depends crucially on the control dependency
from $(\DR{x}{1})$ to $(\DW{y}{1})$, and the data dependency
from $(\DR{y}{1})$ to $(\DW{x}{1})$. If either is missing, then this execution
is acyclic and hence allowed. For example dropping the control dependency
results in:
\begin{gather*}
  \THREAD{\PR{x}{r}\SEMI \PW{y}{1}}
  \PAR
  \THREAD{\PR{y}{s}\SEMI \PW{x}{s}}
  \\[-.4ex]
  \nonumber
  \hbox{\begin{tikzinline}[node distance=1.5em]
      \event{rx1}{\DR{x}{1}}{}
      \event{wy1}{\DW{y}{1}}{right=of rx1}
      \event{ry1}{\DR{y}{1}}{right=3em of wy1}
      \event{wx1}{\DW{x}{1}}{right=of ry1}
      \rf[out=15,in=165]{wy1}{ry1}
      \rf[out=165,in=15]{wx1}{rx1}
      \po{ry1}{wx1}
    \end{tikzinline}}
\end{gather*}

While syntactic dependency calculation suffices for hardware models, it is
not preserved by common compiler optimizations. For example, if we calculate
control dependencies syntactically, then there is a dependency from
$(\DR{x}{1})$ to $(\DW{y}{1})$, and therefore a cycle in, the candidate
execution:
\begin{gather*}
  \THREAD{\PR{x}{r}\SEMI \IF r \THEN \PW{y}{1} \ELSE \PW{y}{1} \FI}
  \PAR
  \THREAD{\PR{y}{s}\SEMI \PW{x}{s}}
  \\[-.4ex]
  \nonumber
  \hbox{\begin{tikzinline}[node distance=1.5em]
      \event{rx1}{\DR{x}{1}}{}
      \event{wy1}{\DW{y}{1}}{right=of rx1}
      \event{ry1}{\DR{y}{1}}{right=3em of wy1}
      \event{wx1}{\DW{x}{1}}{right=of ry1}
      \rf[out=15,in=165]{wy1}{ry1}
      \rf[out=165,in=15]{wx1}{rx1}
      \po{rx1}{wy1}
      \po{ry1}{wx1}
    \end{tikzinline}}
\end{gather*}
A compiler may lift the assignment $\PW{y}{1}$ out of the conditional,
thus removing the dependency.

To address this, \citet{DBLP:journals/pacmpl/JagadeesanJR20} introduced
\emph{Pomsets with Preconditions}, where events are labeled with logical
formulae.  Nontrivial preconditions are introduced by store actions (modeling
data dependencies) and conditionals (modeling control dependencies):
\begin{gather*}
  \IF{s{<}1} \THEN \PW{z}{r{*}s} \FI
  \\
  \nonumber
  \hbox{\begin{tikzinline}[node distance=1.5em]
      \event{wz0}{(s {<} 1) \land (r{*}s){=}0 \mid \DW{z}{0}}{}
    \end{tikzinline}}
\end{gather*}
Preconditions are discharged by being ordered after a read:
\begin{gather*}
  \tag{$\dagger$}\label{eq1}
  \PR{x}{r}\SEMI \PR{y}{s}\SEMI \IF{s{<}1} \THEN \PW{z}{r{*}s} \FI
  \\
  \nonumber
  \hbox{\begin{tikzinline}[node distance=1.5em]
      \event{rx0}{\DR{x}{0}}{}
      \event{ry0}{\DR{y}{0}}{right=of rx0}
      \event{wz0}{(0{=}s) \limplies (s {<} 1) \land (r{*}s){=}0 \mid \DW{z}{0}}{right=of ry0}
      \po{ry0}{wz0}
    \end{tikzinline}}
\end{gather*}
Note that there is dependency order from $(\DR{y}{0})$ to $(\DW{z}{0})$
so the precondition for $(\DW{z}{0})$ only has to be satisfied assuming the hypothesis
$(0{=}s)$. There is no matching order from $(\DR{x}{0})$ to $(\DW{z}{0})$
which is why we do not assume the hypothesis $(0{=}r)$. Nonetheless, the precondition on
$(\DW{z}{0})$ is a tautology, and so can be elided in the diagram:
\begin{gather*}
  % \PR{x}{r}\SEMI \PR{y}{s}\SEMI \IF{s{<}1} \THEN \PW{z}{r{*}s} \FI
  % \\
  \nonumber
  \hbox{\begin{tikzinline}[node distance=1.5em]
      \event{rx0}{\DR{x}{0}}{}
      \event{ry0}{\DR{y}{0}}{right=of rx0}
      \event{wz0}{\DW{z}{0}}{right=of ry0}
      \po{ry0}{wz0}
    \end{tikzinline}}
\end{gather*}

\subsection{Predicate Transformers For Relaxed Memory}

Pomsets with Preconditions show how the logical approach to sequential
dependency calculation can be mixed into a relaxed memory model.  However,
\citeauthor{DBLP:journals/pacmpl/JagadeesanJR20} do not provide a model of
sequential composition.  Instead, their model uses \emph{prefixing}, which
requires that the model is built from right to left: events are prepended one
at a time, with perfect knowledge of the future.  This makes reasoning about
sequential program fragments difficult.  For example,
\citeauthor{DBLP:journals/pacmpl/JagadeesanJR20} state the equivalence
allowing reordering independent writes as follows,
\begin{align*}
  \sem{\aLoc \GETS \aExp \SEMI \bLoc  \GETS \bExp\SEMI\aCmd} &=
  \sem{\bLoc  \GETS \bExp\SEMI \aLoc \GETS \aExp\SEMI\aCmd} \;\;\text{if}\; \aLoc\neq\bLoc
\end{align*}
where $\aCmd$ is the entire future computation!  By formalizing sequential
composition, we can show:
\begin{align*}
  \sem{\aLoc \GETS \aExp \SEMI \bLoc  \GETS \bExp} &=
  \sem{\bLoc  \GETS \bExp\SEMI \aLoc \GETS \aExp} \;\;\text{if}\; \aLoc\neq\bLoc
\end{align*}
Then the equivalence holds in any
%% (sequential)
context.


% While existing models of relaxed memory have detailed treatments of parallel composition,
% they often give sequential composition little attention, either ignoring it altogether,
% or treating it operationally with its usual small-step semantics. This paper
% investigates how existing models of sequential composition interact with relaxed memory.

% \begin{figure*}
%   \begin{align*}
  \fwp(\SKIP,\bForm) &= \bForm
  \\
  \fwp(\ABORT,\bForm) &= \FALSE
  \\
  \fwp(x\GETS \aExp,\bForm) &= (\forall y)\; y{=}\aExp \limplies \bForm[y/x] \text{, where $y$ is fresh}
  \\
  \fwp(x\GETS \aExp,\bForm) &= \bForm[\aExp/x] 
  \\
  \fwp(r\GETS \aLoc,\bForm) &= \bForm[x/r] 
  \\
  \fwp(\aCmd;\bCmd,\bForm) &= \fwp(\aCmd,\fwp(\bCmd,\bForm))
  \\
  \fwp(\IF{\aExp}\THEN \aCmd\ELSE \bCmd\FI,\bForm) &=
  (\aExp\limplies \fwp(\aCmd,\bForm)) \land (\lnot\aExp\limplies \fwp(\bCmd,\bForm))
\end{align*}

\begin{align*}
  \fsp(\SKIP,\aForm) &= \aForm
  \\
  \fsp(x\GETS \aExp,\aForm) &= (\exists y)\;x{=}\aExp[y/x]  \land \aForm[y/x] \text{, where $y$ is fresh}
  \\
  \fsp(\aCmd;\bCmd,\aForm) &= \fsp(\bCmd,\fsp(\aCmd,\aForm))
  \\
  \fsp(\IF{\aExp}\THEN \aCmd\ELSE \bCmd\FI,\aForm) &=
  \fsp(\aCmd, (\aExp\land \aForm)) \land \fsp(\bCmd, (\lnot\aExp\land \aForm))
\end{align*}

\begin{align*}
  \fsp(\aCmd,\aForm) \textimplies \bForm
  \;\;\textiff\;\;
  \hoare{\aForm}{\aCmd}{\bForm} %\;\text{is provable}
  \;\;\Leftrightarrow\;\;
  \aForm \textimplies \fwlp(\aCmd,\bForm)
\end{align*}

% \begin{align*}
%   \hoare{\aTr(\aCmd,\aForm)}{\bCmd}{\bForm} %\;\text{is provable}
%   \;\;\Leftrightarrow\;\;
%   \hoare{\aForm}{\aCmd\SEMI\bCmd}{\bForm} %\;\text{is provable}
% \end{align*}

%   \caption{Weakest precondition semantics}
%   \label{fig:sp}
% \end{figure*}

% Our approach follows that of weakest precondition semantics of
% \citet{DBLP:journals/cacm/Dijkstra75}, which provides an alternative
% characterization of Hoare logic \citep{Hoare:1969:ABC:363235.363259} by
% mapping postconditions to preconditions. We recall the definition of
% $\fwp{\aCmd}{\bForm}$ for loop-free code below. % in Figure~\ref{fig:sp}
% \begin{itemize}
% \item
%   \begin{math}
%     \fwp{\SKIP}{\bForm} = \bForm
%   \end{math}
% % \item 
% %   \begin{math}
% %     \fwp{\ABORT}{\bForm} = \FALSE
% %   \end{math}
% \item
%   \begin{math}
%     \fwp{\LET{r}{\aExp}}{\bForm} = \bForm[\aExp/r]
%   \end{math}
% \item
%   \begin{math}
%     \fwp{\aCmd_1;\aCmd_2}{\bForm} = \fwp{\aCmd_1}{\fwp{\aCmd_2}{\bForm}}
%   \end{math}
% \item
%   \begin{math}
%     \fwp{\IF{\aExp}\THEN \aCmd_1\ELSE \aCmd_2\FI}{\bForm}= {}
%   \end{math}
%   \\
%   \begin{math}
%     ((\aExp{\ne}0) \limplies \fwp{\aCmd_1}{\bForm}) \land
%     ((\aExp{=}0) \limplies \fwp{\aCmd_2}{\bForm})
%   \end{math}
% \end{itemize}
% The rule we are most interested
% in is the one for sequential composition, which maps sequential composition of programs
% to function composition of predicate transformers.

Predicate transformers are a good fit for logical models of dependency
calculation, since both are concerned with preconditions and how they are
transformed by sequential composition. Our first attempt is to associate a
predicate transformer with each pomset. We visualize this in diagrams by
showing how $\bForm$ is transformed, for example:
\begin{align*}
  \begin{gathered}
    \PR{x}{r}
    \\
    \hbox{\begin{tikzinline}[node distance=1ex and 1.5em]
        \event{rx0}{\DR{x}{0}}{}
        \xform{rx0d}{(0{=}r) \limplies \bForm}{right=of rx0}
        \xo{rx0}{rx0d}
      \end{tikzinline}}
  \end{gathered}
  &&
  \begin{gathered}
    \PR{y}{s}
    \\
    \hbox{\begin{tikzinline}[node distance=1ex and 1.5em]
        \event{ry0}{\DR{y}{0}}{}
        \xform{ry0d}{(0{=}s) \limplies \bForm}{right=of ry0}
        \xo{ry0}{ry0d}
      \end{tikzinline}}
  \end{gathered}
  &&
  \begin{gathered}
    \IF{s{<}1} \THEN \PW{z}{r{*}s} \FI
    \\
    \hbox{\begin{tikzinline}[node distance=1ex and 1.5em]
        \event{wz0}{(s {<} 1) \land (r{*}s){=}0 \mid \DW{z}{0}}{}
        \xform{wz0d}{\bForm[r{*}s/z]}{right=of wz0}
        \xo{wz0}{wz0d}
      \end{tikzinline}}
  \end{gathered}
\end{align*}
The predicate transformer from the write matches
\citeauthor{DBLP:journals/cacm/Dijkstra75}'s \ref{wp-write}.  For the reads,
however, \ref{wp-read} defines the transformer of $\PR{x}{r}$ to be
$\bForm[x/r]$, which is equivalent to $(x{=}r) \limplies \bForm$ under the
assumption that registers are assigned at most once.
Instead, we use $(0{=}r) \limplies \bForm$, reflecting the fact that $0$ may
come from a concurrent write.  The obligation to find a matching write is
moved from the sequential semantics of \emph{substitution} and
\emph{implication} to the concurrent semantics of \emph{fulfillment}.

% In the rightmost program above, the write to $z$ affects the shared store, not the
% local state of the thread, therefore we assign it the identity transformer.

For a sequentially consistent semantics, sequential composition is
straightforward: we apply each predicate transformer to the preconditions of
subsequent events, composing the predicate transformers.  (In subsequent
diagrams, we only show predicate transformers for reads.)
\begin{gather*}
  \PR{x}{r}\SEMI \PR{y}{s}\SEMI \IF{s{<}1} \THEN \PW{z}{r{*}s} \FI
  \\[-1.5ex]
  \nonumber
  \hbox{\begin{tikzinline}[node distance=1ex and 1.5em]
      \event{rx0}{\DR{x}{0}}{}
      \event{ry0}{\DR{y}{0}}{right=of rx0}
      \event{wz0}{(0{=}r) \limplies (0{=}s) \limplies (s {<} 1) \land (r{*}s){=}0 \mid \DW{z}{0}}{right=of ry0}
      \xform{rx0ry0d}{(0{=}r) \limplies(0{=}s) \limplies \bForm}{left=of rx0}
      \po{rx0}{ry0}
      \po{ry0}{wz0}
      \xo{rx0}{rx0ry0d}
      \xo[out=155,in=15]{ry0}{rx0ry0d}
      % \xo{wz0}{rx0ry0d}
    \end{tikzinline}}
\end{gather*}
This model works for the sequentially consistent case, but needs to be
weakened for the relaxed case. The key observation of this paper is
that rather than working with one predicate transformer, we should
work with a \emph{family} of predicate transformers, indexed by sets
of events.

For example, for single-event pomsets, there are two predicate
transformers, since there are two subsets of any one-element set.
The \emph{independent}
transformer is indexed by the empty set, whereas the \emph{dependent}
transformer is indexed by the singleton.
We visualize this by including more than one transformed predicate,
with an edge leading to the dependent one. For example:
\begin{align*}
  \begin{gathered}
    \PR{x}{r}
    \\
    \hbox{\begin{tikzinline}[node distance=1ex and 1.5em]
        \event{rx0}{\DR{x}{0}}{}
        \xform{rx0i}{\bForm}{left=.4em of rx0}
        \xform{rx0d}{(0{=}r) \limplies \bForm}{right=of rx0}
        \xo{rx0}{rx0d}
      \end{tikzinline}}
  \end{gathered}
  &&
  \begin{gathered}
    \PR{y}{s}
    \\
    \hbox{\begin{tikzinline}[node distance=1ex and 1.5em]
        \event{ry0}{\DR{y}{0}}{}
        \xform{ry0i}{\bForm}{left=.4em of ry0}
        \xform{ry0d}{(0{=}s) \limplies \bForm}{right=of ry0}
        \xo{ry0}{ry0d}
      \end{tikzinline}}
  \end{gathered}
  % &&
  % \begin{gathered}
  %   \IF{s{<}1} \THEN \PW{z}{r{*}s} \FI
  %   \\
  %   \hbox{\begin{tikzinline}[node distance=1ex and1.5em]
  %     \event{wz0}{(s {<} 1) \land (r{*}s){=}0 \mid \DW{z}{0}}{}
  %     \xform{wz0i}{\bForm}{left=.4em of wz0}
  %     \xform{wz0d}{\bForm}{right=of wz0}
  %     \xo{wz0}{wz0d}
  %   \end{tikzinline}}
  % \end{gathered}
\end{align*}
The model of sequential composition then picks which
predicate transformer to apply to an event's precondition by picking
the one indexed by all the events before it in causal order.

For example, we can recover the expected semantics
for \eqref{eq1} by choosing
the predicate transformer which is independent of $(\DR x0)$
but dependent on $(\DR y0)$, which is the transformer
which maps $\bForm$ to $(0{=}s) \limplies \bForm$.
\begin{gather*}
  \PR{x}{r}\SEMI \PR{y}{s}\SEMI \IF{s{<}1} \THEN \PW{z}{r{*}s} \FI
  \\[-2ex]
  \nonumber
  \hbox{\begin{tikzinline}[node distance=1ex and 1.2em]
      \xform{rx0d}{(0{=}r) \limplies \bForm}{}
      \event{rx0}{\DR{x}{0}}{right=of rx0d}
      \xform{rx0ry0d}{(0{=}r) \limplies(0{=}s) \limplies \bForm}{right=of rx0}
      \event{ry0}{\DR{y}{0}}{right=of rx0ry0d}
      \xform{ry0d}{(0{=}s) \limplies \bForm}{right=of ry0}
      \event{wz0}{(0{=}s) \limplies (s {<} 1) \land (r{*}s){=}0 \mid \DW{z}{0}}{right=of ry0d}
      \xform{wz0i}{\bForm}{left=of rx0d}
      \po[out=20,in=170]{ry0}{wz0}
      \xo{rx0}{rx0d}
      \xo{ry0}{ry0d}
      \xo{rx0}{rx0ry0d}
      \xo{ry0}{rx0ry0d}
    \end{tikzinline}}
\end{gather*}
As a sanity check, we can see that sequential composition is
associative in this case, since it does not matter whether we
associate to the left, with intermediate step:
\begin{gather*}
  \PR{x}{r}\SEMI \PR{y}{s}
  \\
  \nonumber
  \hbox{\begin{tikzinline}[node distance=1ex and 1.5em]
      \xform{rx0d}{(0{=}r) \limplies \bForm}{}
      \event{rx0}{\DR{x}{0}}{right=of rx0d}
      \xform{rx0ry0d}{(0{=}r) \limplies(0{=}s) \limplies \bForm}{right=of rx0}
      \event{ry0}{\DR{y}{0}}{right=of rx0ry0d}
      \xform{ry0d}{(0{=}s) \limplies \bForm}{right=of ry0}
      \xform{wz0i}{\bForm}{left=of rx0d}
      \xo{rx0}{rx0d}
      \xo{ry0}{ry0d}
      \xo{rx0}{rx0ry0d}
      \xo{ry0}{rx0ry0d}
    \end{tikzinline}}
\end{gather*}
or to the right, with intermediate step:
\begin{gather*}
  \PR{y}{s}\SEMI \IF{s{<}1} \THEN \PW{z}{r{*}s} \FI
  \\
  \nonumber
  \hbox{\begin{tikzinline}[node distance=1ex and 1.5em]
      \event{ry0}{\DR{y}{0}}{}
      \event{wz0}{(0{=}s) \limplies (s {<} 1) \land (r{*}s){=}0 \mid \DW{z}{0}}{right=of ry0}
      \xform{ry0d}{(0{=}s) \limplies \bForm}{left=of ry0}
      \xform{wz0i}{\bForm}{left=of ry0d}
      \po{ry0}{wz0}
      \xo{ry0}{ry0d}
    \end{tikzinline}}
\end{gather*}
This is an instance of the general result that sequential composition forms a monoid.

\subsection{Related Work}

\citet{DBLP:conf/snapl/MarinoMMNS15} argue that the ``silently shifting
semicolon'' is sufficiently problematic for programmers that concurrent
languages should guarantee sequential abstraction, despite the performance
penalties.  In this paper, we have taken the opposite approach.  We have
attempted to find the most intellectually tractable model that encompasses
all of the messiness of relaxed memory.

There are few prior studies of relaxed memory that include sequential
composition and/or precise calculation of semantic dependencies.
\citet{DBLP:conf/esop/PaviottiCPWOB20} give a denotational semantics,
calculating dependencies using event structures rather than logic.  They give
the semantics of sequential composition in continuation passing style,
whereas we give it in direct style.
%
\citet{DBLP:journals/corr/abs-1804-04214} define a semantics
using pomsets without preconditions. Instead, their model uses syntactic
dependencies, thus invalidating many compiler optimizations.  They also
require a fence after every relaxed read on \armeight{}.
%
\citet{Pichon-Pharabod:2016:CSR:2837614.2837616} use event structures to
calculate dependencies, combined with an operational semantics that
incorporates program transformations.  This approach seems to require
whole-program analysis.

Other studies of relaxed memory can be categorized by their approach to
dependency calculation.  Hardware models use syntactic dependencies
\cite{alglave}.  Many software models do not bother with dependencies at all
\cite{Batty:2011:MCC:1926385.1926394, DBLP:journals/pacmpl/WattRP19,
  DBLP:conf/pldi/WattPPBDFPG20, goMM}.  Others have strong dependencies that
disallow compiler optimizations and efficient implementation, typically
requiring fences for every relaxed read on Arm
\cite{Lamport:1979:MMC:1311099.1311750, DBLP:conf/pldi/LahavVKHD17,
  Dolan:2018:BDR:3192366.3192421, DBLP:conf/lics/JeffreyR16,
  Boehm:2014:OGA:2618128.2618134}. %, DBLP:journals/corr/abs-1804-04214}.
%
Many of the most prominent models are operational, whole-program models based
on speculative execution \cite{Manson:2005:JMM:1047659.1040336,
  DBLP:conf/esop/JagadeesanPR10,
  DBLP:conf/popl/KangHLVD17,DBLP:journals/pacmpl/ChakrabortyV19,DBLP:conf/pldi/LeeCPCHLV20,promising-ldrf}.

\citet{DBLP:journals/pacmpl/JagadeesanJR20} note that the speculative models
listed above all, including \cite{DBLP:conf/popl/KangHLVD17}, fail to
validate compositional reasoning of temporal properties---see their examples
OOTA4, OOTA5, and \citep[Fig.~8]{DBLP:journals/toplas/Lochbihler13}).  The
difference with our model can be understood in terms of the valid program
transformations.  The speculative models allow reads to be introduced, with
subsequent case analysis on the value read---effectively, this can turn one
read into two, with different conditional branches taken for the two copies
of the read.  Our model invalidates this transformation.  In return, our
model enjoys compositionality for temporal safety properties.

We provide a detailed comparison with
\cite{DBLP:journals/pacmpl/JagadeesanJR20} in \textsection\ref{sec:diff}.

\subsection{Contributions}

We show how predicate transformers~\cite{DBLP:journals/cacm/Dijkstra75} can
be added to pomsets with
preconditions~\cite{DBLP:journals/pacmpl/JagadeesanJR20} to create a
compositional semantics for sequential composition.
\begin{itemize}
\item \S\ref{sec:model} presents the basic model, which satisfies many
  desiderata, but not all.
\item \S\ref{sec:arm} shows two approaches for efficient implementation on
  Arm.  The first uses a suboptimal lowering for acquiring reads.  The second
  uses an optimal lowering, but requires a nontrivial change to the
  definition of sequential composition.
\item \S\ref{sec:additional} generalizes the basic semantics of read and write
  to validate compiler optimizations.
\end{itemize}
% Acknowledgements go here, once we're not double-blinded.
% The definitions in this paper have been formalized in Agda.
Because it is closely related, we expect that the memory-model results of
\cite{DBLP:journals/pacmpl/JagadeesanJR20} apply to our model, including
compositional reasoning for temporal safety properties and {local} \drfsc{}
as in \cite{Dolan:2018:BDR:3192366.3192421,DBLP:conf/ppopp/DongolJR19,promising-ldrf}.
% In \textsection\ref{sec:arm}, we provide an alternative proof strategy for
% efficient compilation to \armeight{}, which improves upon that of
% \cite{DBLP:journals/pacmpl/JagadeesanJR20} by using a recent alternative
% characterization of \armeight{}.

