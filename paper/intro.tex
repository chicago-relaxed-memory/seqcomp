\section{Introduction}

% \begin{figure*}
%   \begin{align*}
  \fwp(\SKIP,\bForm) &= \bForm
  \\
  \fwp(\ABORT,\bForm) &= \FALSE
  \\
  \fwp(x\GETS \aExp,\bForm) &= (\forall y)\; y{=}\aExp \limplies \bForm[y/x] \text{, where $y$ is fresh}
  \\
  \fwp(x\GETS \aExp,\bForm) &= \bForm[\aExp/x] 
  \\
  \fwp(r\GETS \aLoc,\bForm) &= \bForm[x/r] 
  \\
  \fwp(\aCmd;\bCmd,\bForm) &= \fwp(\aCmd,\fwp(\bCmd,\bForm))
  \\
  \fwp(\IF{\aExp}\THEN \aCmd\ELSE \bCmd\FI,\bForm) &=
  (\aExp\limplies \fwp(\aCmd,\bForm)) \land (\lnot\aExp\limplies \fwp(\bCmd,\bForm))
\end{align*}

\begin{align*}
  \fsp(\SKIP,\aForm) &= \aForm
  \\
  \fsp(x\GETS \aExp,\aForm) &= (\exists y)\;x{=}\aExp[y/x]  \land \aForm[y/x] \text{, where $y$ is fresh}
  \\
  \fsp(\aCmd;\bCmd,\aForm) &= \fsp(\bCmd,\fsp(\aCmd,\aForm))
  \\
  \fsp(\IF{\aExp}\THEN \aCmd\ELSE \bCmd\FI,\aForm) &=
  \fsp(\aCmd, (\aExp\land \aForm)) \land \fsp(\bCmd, (\lnot\aExp\land \aForm))
\end{align*}

\begin{align*}
  \fsp(\aCmd,\aForm) \textimplies \bForm
  \;\;\textiff\;\;
  \hoare{\aForm}{\aCmd}{\bForm} %\;\text{is provable}
  \;\;\Leftrightarrow\;\;
  \aForm \textimplies \fwlp(\aCmd,\bForm)
\end{align*}

% \begin{align*}
%   \hoare{\aTr(\aCmd,\aForm)}{\bCmd}{\bForm} %\;\text{is provable}
%   \;\;\Leftrightarrow\;\;
%   \hoare{\aForm}{\aCmd\SEMI\bCmd}{\bForm} %\;\text{is provable}
% \end{align*}

%   \caption{Weakest Precondition and Strongest Postcondition \cite{}}
% \end{figure*}


Our approach follows that of weakest precondition semantics of
\citet{DBLP:journals/cacm/Dijkstra75}, which provides an alternative
characterization of Hoare logic \citep{Hoare:1969:ABC:363235.363259} by
mapping postconditions to preconditions.

% Our transformers $\aTr$ map preconditions to postconditions, in order to
% apply to preconditions.
% \begin{align*}
%   \hoare{\aForm}{\bCmd}{\bForm}
%   \;\;\Leftrightarrow\;\;
%   \hoare{\aTr(\aCmd,\aForm)}{\aCmd\SEMI\bCmd}{\bForm}
% \end{align*}
% For example
% \begin{gather*}
%   \hoare{r{=}1}{\PW{y}{r}}{y{=}1}
%   \\
%   \hoare{x{=}1}{\PR{x}{r}\SEMI\PW{y}{r}}{y{=}1}
% \end{gather*}


\endinput

This paper builds on
\cite{2019-sp} and
\cite{10.1145/3428262}.


Pomsets are a nice model of concurrent computation.
Predicate transformers are a nice model of sequential computation.
We show how these interact.

We do this by attaching predicate transformers to \emph{configurations}.

Thus the amount of transformation depends on which read events are seen in
the configuration.

We develop the theory of pomsets with predicate transformers, then apply the
theory to model relaxed memory.

Other things we could model include Map/Reduce and Fork/Join.
