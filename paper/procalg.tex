\section{Process Algebra}

We begin with a warm-up exercise, presenting the key ideas of the
model in the simplified setting of a process algebra.

Process algebras were developed as a simple model of message-passing
concurrency~\cite{DBLP:books/daglib/0067019,DBLP:books/ph/Hoare85,DBLP:books/daglib/0069083}
and have been given many different semantics, notably traces~\cite{???},
labelled transition systems~\cite{???}, equational~\cite{???}, and event
structures~\cite{???}.  In this section, we recap the \emph{pomset}
semantics, and show how it can be used to model a simple process language
based on the asynchronous $\pi$-calculus~\cite{???} and value-passing
CCS~\cite{???}.

The key idea of this paper is a relaxed treatment of sequential
composition, where causality is not given by program order, but
instead is given by control and data dependencies.

% Note to self: Process algebras typically focus on channel-based
% communication.  Here we adopt communication via shared memory.
% Characteristics of this mode of communication:
% \begin{itemize}
% \item It is asynchronous.
% \item Reads are nondestructive.
% \end{itemize}
% So it is similar to an asynchronous process calculus with broadcast
% communication where reads don't kill.

\subsection{Pomsets}

\begin{definition}
  A \emph{poimet} with alphabet $\Alphabet$ is a tuple
  $(\Event, {\le}, \labeling)$ where
  \begin{enumerate}
  \item $\Event$ is a set of \emph{events}
  \item
    ${\le} \subseteq (\Event\times\Event)$ is the \emph{causality} partial order, and
  \item
    $\labeling: \Event \fun \Alphabet$ is a \emph{labeling}.
  \end{enumerate}
\end{definition}


\begin{definition}
  If $\aPS_0\in\sSTOP$ then:
  \begin{enumerate}
    \item $\aEvs_0 = \emptyset$.
  \end{enumerate}

  \noindent
  If $\aPS_0 \in (\aPSS_1\PAR\aPSS_2)$ then  
  $(\exists\aPS_1\in\aPSS_1)$ $(\exists\aPS_2\in\aPSS_2)$:
  %there are $\aPS_1\in\aPSS_1$ and $\\\\\0PS_2\in\aPSS_2$ such that:
  \begin{enumerate}
  \item $\aEvs_0 = (\aEvs_1\cup\aEvs_2)$,
  \item if $\aEv\in\aEvs_1$ then $\labeling_0(\aEv) = \labeling_1(\aEv)$, 
  \item if $\aEv\in\aEvs_2$ then $\labeling_0(\aEv) = \labeling_2(\aEv)$,
  \item if $\bEv\le_1\aEv$ then $\bEv\le_0\aEv$, 
  \item if $\bEv\le_2\aEv$ then $\bEv\le_0\aEv$, and
    \newcounter{pomsetParXount}
    \setcounter{pomsetParXount}{\value{enumi}}
  \item $\aEvs_1$ and $\aEvs_2$ are disjoint.
    \newcounter{pomsetParDisjointXount}
    \setcounter{pomsetParDisjointXount}{\value{enumi}}
  \end{enumerate}

  % \noindent
  % If $\aPS_0 \in (\aAct\sPREFIX\aPSS)$ then
  % $(\exists\aPS_2\in\aPSS)$:
  % \begin{enumerate}
  % \item $\aEvs_0=(\aEvs_1 \cup \aEvs_2)$,
  % \item if $\aEv\in\aEvs_1$ then $\labelingAct_0(\aEv) = \aAct$,
  % \item if $\aEv\in\aEvs_2$ then $\labelingAct_0(\aEv) = \labelingAct_2(\aEv)$,
  % \item if $\bEv,\aEv\in\aEvs_1$ then $\bEv=\aEv$,
  % \item if $\bEv\le_2\aEv$ then $\bEv\le_0\aEv$, and
  %   \newcounter{pomsetPrefixXount}
  %   \setcounter{pomsetPrefixXount}{\value{enumi}}
  % \item if $\bEv\in\aEvs_1$ and $\aEv\in\aEvs_2$ then $\bEv<_0\aEv$.
  %   \newcounter{pomsetPrefixOrderXount}
  %   \setcounter{pomsetPrefixOrderXount}{\value{enumi}}
  % \end{enumerate}

  \noindent
  If $\aPS_0 \in (\aAct\wPREFIX\aPSS)$ then
  $(\exists\aPS_2\in\aPSS)$:
  %there is $\aPS_2\in\aPSS$ such that:
  \begin{enumerate}
  \item $\aEvs_0=(\aEvs_1 \cup \aEvs_2)$,
  \item if $\aEv\in\aEvs_1$ then $\labelingAct_0(\aEv) = \aAct$,
  \item if $\aEv\in\aEvs_2$ then $\labelingAct_0(\aEv) = \labelingAct_2(\aEv)$,
  \item if $\bEv,\aEv\in\aEvs_1$ then $\bEv=\aEv$,
  \item if $\bEv\le_2\aEv$ then $\bEv\le_0\aEv$, and
    \newcounter{pomsetPrefixXount}
    \setcounter{pomsetPrefixXount}{\value{enumi}}
  %   \setcounter{enumi}{\value{pomsetPrefixXount}}
  % \item[1--\thepomsetPrefixXount)] as for $(\aAct\sPREFIX\aPSS)$ in
  %   Definition~\ref{tab:pomsets},
  \item if $\bEv\in\aEvs_1$ and $\aEv\in\aEvs_2$, either $\bEv<_0\aEv$ or $a\reorder\labeling_2(\aEv)$.
    \newcounter{pomsetPrefixOrderXount}
    \setcounter{pomsetPrefixOrderXount}{\value{enumi}}
  \end{enumerate}
%\caption{Process algebra as sets of pomsets}
\label{tab:pomsets}
\end{definition}

[No communication between threads, but this can be added as a condition on pomsets,
  for instance a simple model where every receive event has a matching send.]

[Introduce ``blocker'', and ``fulfillment'' on $\Alphabet$, and lift up to events.]

For example, for latches:
\begin{itemize}
\item $\set{\aChan}{}$ fulfills $\get{\aChan}{}$, and
\item there are no blockers.
\end{itemize}
for condition variables:
\begin{itemize}
\item $\notify{\aChan}{}$ fulfills $\notified{\aChan}{}$, and
\item $\wait{\aChan}{}$ blocks $\notified{\aChan}{}$.
\end{itemize}
for mutexes:
\begin{itemize}
\item $\init{\aChan}$ and $\release{\aChan}$ fulfil $\acquire{\aChan}{}$,
\item $\acquire{\aChan}$ fulfills $\release{\aChan}$, and
\item $\acquire{\aChan}$ and $\release{\aChan}$ block each other and themselves. 
\end{itemize}
  
\begin{definition}
  A pomset $\aPS$ is closed if
  for every $\aEv$ which can be fulfilled,
  there is a $\bEv\le\aEv$ which fulfills it,
  and for any $\cEv$ which can block $\aEv$, either $\cEv\le\bEv$ or $\aEv\le\cEv$. 
\end{definition}

[Do some process algebra examples.]

[Use refinement (that is subset order) as notion of compiler optimization.]

For some examples, we want to allow reorderings:
\[
  (\aAct \wPREFIX \bAct \wPREFIX \aPS)
\subseteq
  (\bAct \wPREFIX \aAct \wPREFIX \aPS)
\]
or mumbling:
\[
  (\aAct \wPREFIX \aAct \wPREFIX \aPS)
\subseteq
  (\aAct \wPREFIX \aPS)
\]
[Talk about Mazurkiewicz traces.]

[Introduce $\aAct\reorder\bAct$ for when reordering is valid,
  and  $\aAct\reorder\aAct$ for when mumbling is valid.]

[Give examples: latches support mumbling, mutexes support roach motel reorderings.]

\subsection{Pomsets with preconditions}

This is essentially the model of \cite{10.1145/3428262}.

\begin{definition}
  A \emph{pomset with preconditions} is
  a pomset together with $\labelingForm:\aEvs\fun\Formulae$.
\end{definition}

\begin{definition}
  If $\aPS_0\in\sSTOP$ then:
  \begin{enumerate}
  \item $\aEvs_0 = \emptyset$.
  \end{enumerate}

  \noindent
  If $\aPS_0 \in (\aPSS_1\PAR\aPSS_2)$ then
  $(\exists\aPS_1\in\aPSS_1)$ $(\exists\aPS_2\in\aPSS_2)$:
  %there are $\aPS_1\in\aPSS_1$ and $\aPS_2\in\aPSS_2$ such that:
  \begin{enumerate}
     \setcounter{enumi}{\value{pomsetParDisjointXount}}
  \item[1--\thepomsetParDisjointXount)] as for $(\aPSS_1\PAR\aPSS_2)$ in Definition~\ref{tab:pomsets},
  \item if $\aEv\in\aEvs_1$ then $\labelingForm_0(\aEv)$ implies $\labelingForm_1(\aEv)$,
  \item if $\aEv\in\aEvs_2$ then $\labelingForm_0(\aEv)$ implies $\labelingForm_2(\aEv)$,
    \newcounter{pomsetPreParXount}
    \setcounter{pomsetPreParXount}{\value{enumi}}
  \end{enumerate}

  \noindent
  If $\aPS_0 \in \sIF{\bForm}\sTHEN\aPSS_1\sELSE\aPSS_2\sFI$ then
  $(\exists\aPS_1\in\aPSS_1)$ $(\exists\aPS_2\in\aPSS_2)$:
  %there are $\aPS_1\in\aPSS_1$ and $\aPS_2\in\aPSS_2$ such that:
  \begin{enumerate}
  \setcounter{enumi}{\value{pomsetParXount}}
  \item[1--\thepomsetParXount)] as for $(\aPSS_1\PAR\aPSS_2)$  in
    Definition~\ref{tab:pomsets} (ignoring \thepomsetParDisjointXount),
  \item if $\aEv\in\aEvs_1$ and $\aEv\not\in\aEvs_1$ then\\ $\labelingForm_0(\aEv)$ implies $\bForm\land\labelingForm_1(\aEv)$,
  \item if $\aEv\not\in\aEvs_1$ and $\aEv\in\aEvs_2$ then\\ $\labelingForm_0(\aEv)$ implies $\neg\bForm\land\labelingForm_2(\aEv)$, and
  \item if $\aEv\in\aEvs_1$ and $\aEv\in\aEvs_2$ then\\ $\labelingForm_0(\aEv)$ implies $(\bForm\land\labelingForm_1(\aEv))\lor(\neg\bForm\land\labelingForm_2(\aEv))$.
    \newcounter{pomsetPreIfXount}
    \setcounter{pomsetPreIfXount}{\value{enumi}}
  \end{enumerate}

  \noindent
  If $\aPS_0 \in (\aAct[\aVal/\aLoc]\wPREFIX\aPSS)$ then
  $(\exists\aPS_2\in\aPSS_2)$:
  %there is $\aPS_2\in\aPSS$ such that:
  \begin{enumerate}
     \setcounter{enumi}{\value{pomsetPrefixXount}}
  \item[1--\thepomsetPrefixXount)] as for $(\aAct\wPREFIX\aPSS)$ in Definition~\ref{tab:pomsets} (ignoring \thepomsetPrefixOrderXount),
  \item if $\aEv\in\aEvs_1$ then $\labelingForm_0(\aEv)$ implies $\preForm_\aAct$,
  \item\label{skolem} if $\aEv\in\aEvs_2$ then $\labelingForm_0(\aEv)$ implies $(\aLoc=\aVal) \limplies \labelingForm_2(\aEv)$,
  \item if $\bEv\in\aEvs_1$ and $\aEv\in\aEvs_2$ then either $\bEv<_0\aEv$ or\\ 
    $\labelingForm_0(\aEv)$ implies $\reorderIndForm_\aAct\land\labelingForm_2(\aEv)$.
  \end{enumerate}
% \caption{Process algebra as sets of pomsets with preconditions}
\label{tab:pomsets-pre}
\end{definition}

\begin{remark}
Note that in \ref{skolem} for prefixing, we use
$(\aLoc=\aVal) \limplies \labelingForm_2(\aEv)$ rather than
$\labelingForm_2(\aEv) [\aVal/\aLoc]$.
This use of skolemization  ensures disjunction closure for nonlinear uses of
the argument.

% \begin{center}
%   \includegraphics[width=90mm]{boards/skolem1.jpg}

%   Xounterexample:
%   \includegraphics[width=90mm]{boards/skolem2.jpg}

%   Skolem:
%   \includegraphics[width=90mm]{boards/skolem3.jpg}
%   This will be easier if we just require registers to be unique. Then we can use r for s.
% \end{center}
\end{remark}

[Introduce data models, with formulae.]

Example of value-passing on unordered channels which allow
message duplication:
\begin{eqnarray*}
  \aForm\sGUARD\aPSS & = &
  \sIF{\aForm}\sTHEN\aPSS\sELSE\sSTOP\sFI
\\
  \sSEND\aChan\aExp\wPREFIX\aPSS & = &
  \textstyle\bigcup_\aVal (\aExp=\aVal) \sGUARD \send\aChan{\aVal} \wPREFIX \aPSS
\\
  \sRECV\aChan\aLoc\wPREFIX\aPSS & = &
  \textstyle\bigcup_\aVal \recv\aChan{\aVal}[\aVal/\aLoc] \wPREFIX \aPSS
\end{eqnarray*}

Unfortunately we can't play the same trick with reordering based on
label, since the label doesn't track control or data dependencies. But
we can code up reorderings in the logic. Assume formulae
$\preForm_\aAct$ (the precondition for $\aAct$) and
$\reorderIndForm_\aAct$ (the precondition for events which can be reordered with $\aAct$).

For example, we introduce a new uninterpreted formula $\quietForm$
and have for relaxed $\aAct$:
\[
  \preForm_{\aAct} = \neg\quietForm
\qquad
  \reorderIndForm_{\aAct} = \neg\quietForm
\]
for release $\aAct$:
\[
  \preForm_{\aAct} = \quietForm
\qquad
  \reorderIndForm_{\aAct} = \neg\quietForm
\]
and for acquire $\aAct$:
\[
  \preForm_{\aAct} = \neg\quietForm
\qquad
  \reorderIndForm_{\aAct} = \FALSE
\]

[Stuff about conditionals and merging events.]

\begin{definition}
  A pomset with precondition is \emph{top level} if for any $\aEv\in\aEvs$,
  $\preForm_{\labelingAct(\aEv)}$ implies $\labelingForm(\aEv)$.
\end{definition}


\subsection{A pomset with preconditions and predicate transformers}

[The problem with the previous section is that there's no story for sequential composition.]

\begin{definition}
  A \emph{predicate transformer} is a monotone function
  $\aTr:\Formulae\fun\Formulae$ such that
  $\aTr(\FALSE)$ is $\FALSE$,
  $\aTr(\aForm\land\bForm)$ is $\aTr(\aForm)\land\aTr(\bForm)$, and
  $\aTr(\aForm\lor\bForm)$ is $\aTr(\aForm)\lor\aTr(\bForm)$.
\end{definition}

\begin{definition}
  A \emph{family of predicate transformers}
  indexed by subsets of $\aEvs$
  consists of predicate transformers
  $\aTr{\bEvs}{}$ for each set of events $\bEvs$,
  such that if $(\cEvs\cap\aEvs) \subseteq \bEvs$
  then $\aTr{\cEvs}{\aForm}$ implies $\aTr{\bEvs}{\aForm}$.
\end{definition}

\begin{definition}
  A pomset with predicate tansformers is a
  pomset with preconditions, together with a family of predicate transformers $\aTr$
  indexed by subsets of $\aEvs$.
\end{definition}

% \begin{definition}
%   For pomset $\aPS_0$ and $\aEv\in\aEvs_0$, let
%   $\downclose[0]{\aEv} = \{ \bEv\in\aEvs_0 \mid \bEv <_0 \aEv \}$.
% \end{definition}

Process algebra as sets of pomsets with preconditions and predicate
transformers
\begin{definition}
  If $\aPS_0\in\sSTOP$ then:
  \begin{enumerate}
    \item $\aEvs_0 = \emptyset$, and
  \item $\aTr[0]{\bEvs}{\aForm}$ implies $\FALSE$.
  \end{enumerate}

  \noindent
  If $\aPS_0\in\sSKIP$ then:
  \begin{enumerate}
  \item $\aEvs_0 = \emptyset$, and
  \item $\aTr[0]{\bEvs}{\aForm}$ implies $\aForm$.
  \end{enumerate}

  \noindent
  If $\aPS_0 \in \sIF{\bForm}\sTHEN\aPSS_1\sELSE\aPSS_2\sFI$ then
  $(\exists\aPS_1\in\aPSS_1)$ $(\exists\aPS_2\in\aPSS_2)$:
  %there are $\aPS_1\in\aPSS_1$ and $\aPS_2\in\aPSS_2$ such that:
  \begin{enumerate}
  \setcounter{enumi}{\value{pomsetPreIfXount}}
  \item[1--\thepomsetPreIfXount)] as for $(\sIF{\bForm}\sTHEN\aPSS_1\sELSE\aPSS_2\sFI)$ in Definition~\ref{tab:pomsets-pre},
  \item $\aTr[0]{\bEvs}{\aForm}$ implies $(\bForm\land\aTr[1]{\bEvs}{\aEv})\lor(\neg\bForm\land\aTr[2]{\bEvs}{\aForm})$.
  \end{enumerate}

  \noindent
  If $\aPS_0 \in (\aPSS_1\wSEMI\aPSS_2)$ then
  $(\exists\aPS_1\in\aPSS_1)$ $(\exists\aPS_2\in\aPSS_2)$,\\
  %there are $\aPS_1\in\aPSS_1$ and $\aPS_2\in\aPSS_2$ such that:
  %let $\labelingForm'_2(\aEv)=\aTr[1]{\downclose[0]{\aEv}}{\labelingForm_2(\aEv)}$:  
  %let $\labelingForm'_2(\aEv)=\aTr[1]{\{ \bEv \mid \bEv <_0 \aEv \}}{\labelingForm_2(\aEv})$:  
  let $\labelingForm'_2(\aEv)=\aTr[1]{\cEvs}{\labelingForm_2(\aEv})$, where $\cEvs=\{ \cEv \mid \cEv <_0 \aEv \}$:
  \begin{enumerate}  
  \setcounter{enumi}{\value{pomsetParXount}}
  \item[1--\thepomsetParXount)] as for $(\aPSS_1\PAR\aPSS_2)$  in
    Definition~\ref{tab:pomsets} (ignoring \thepomsetParDisjointXount),
  \item if $\aEv\in\aEvs_1$ and $\aEv\not\in\aEvs_1$ then $\labelingForm_0(\aEv)$ implies $\labelingForm_1(\aEv)$,
  \item if $\aEv\not\in\aEvs_1$ and $\aEv\in\aEvs_2$ then $\labelingForm_0(\aEv)$ implies $\labelingForm'_2(\aEv)$,
  \item if $\aEv\in\aEvs_1$ and $\aEv\in\aEvs_2$ then\\ $\labelingForm_0(\aEv)$ implies $\labelingForm_1(\aEv)\lor\labelingForm'_2(\aEv)$, and
  \item $\aTr[0]{\bEvs}{\aForm}$ implies $\aTr[2]{\bEvs}{\aTr[1]{\bEvs}{\aForm}}$.
  \end{enumerate}
  % where we define $\labelingForm'_2(\aEv)$ to be $\aTr[1]{\{ \bEv \mid \bEv <_0 \aEv \}}{\labelingForm_2(\aEv})$.

  \noindent
  If $\aPS_0 \in (\aAct[\aVal/\aLoc])$ then:
  \begin{enumerate}
  \item if $\bEv,\aEv\in\aEvs_0$ then $\bEv=\aEv$,
  \item $\labelingAct_0(\aEv) = \aAct$,
  \item $\labelingForm_0(\aEv)$ implies $\preForm_\aAct$,
  \item $\aTr[0]{\bEvs}{\aForm}$ implies $(\aLoc=\aVal)\limplies\aForm$, and
  \item $\aTr[0]{\emptyset}{\aForm}$ implies $\reorderIndForm_\aAct\land\aForm$.
  \end{enumerate}
% \caption{Process algebra as sets of pomsets with preconditions and predicate transformers}
\end{definition}

Embedding pomsets with preconditions semantics in pomsets with preconditions and predicate transformers semantics  
\begin{definition}
  If $\aPS_0 \in \wFORK\aPSS$ then
  $(\exists\aPS_1\in\aPSS)$:
  %there is $\aPS_1\in\aPSS$ such that:
  \begin{enumerate}
  \item $\aEvs_0=\aEvs_1$,
  \item $\labelingAct_0(\aEv) = \labelingAct_1(\aEv)$,
  \item $\labelingForm_0(\aEv)$ implies $\labelingForm_1(\aEv)$, and
  \item $\aTr[0]{\bEvs}{\aForm}$ implies $\aForm$.
  \end{enumerate}
% \caption{Embedding pomsets with precodnitions semantics in pomsets with preconditions and predicate transformers semantics}
\end{definition}

[Examples.]

\begin{eqnarray*}
  \sSEND\aChan\aExp & = &
  \textstyle\bigcup_\aVal (\aExp=\aVal) \sGUARD \send\aChan{\aVal}
\\
  \sRECV\aChan\aLoc & = &
  \textstyle\bigcup_\aVal \recv\aChan{\aVal}[\aVal/\aLoc]
\end{eqnarray*}

