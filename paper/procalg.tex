\section{Process Algebra}

We begin with a warm-up exercise, presenting the key ideas of the
model in the simplified setting of a process algebra.

Process algebras were developed as a simple model of message-passing
concurrency~\cite{CCS,CSP,ACP} and have been given many different
semantics, notably traces~\cite{???}, labelled transition
systems~\cite{???}, equational~\cite{???}, and event
structures~\cite{???}.  In this section, we recap the \emph{pomset}
semantics, and show how it can be used to model a simple process
language based on the asynchronous $\pi$-calculus~\cite{???} and
value-passing CCS~\cite{???}.

The key idea of this paper is a relaxed treatment of sequential
composition, where causality is not given by program order, but
instead is given by control and data dependencies.

\subsection{A pomset model of condition variables}

\begin{definition}
  A \emph{pomset} with alphabet $\Alphabet$ is a tuple
  $(\Event, {\le}, \labeling)$ where
  \begin{enumerate}
  \item $\Event$ is a set of \emph{events}
  \item
    ${\le} \subseteq (\Event\times\Event)$ is the \emph{causality} partial order, and
  \item
    $\labeling: \Event \fun \Alphabet$ is a \emph{labeling}.
  \end{enumerate}
\end{definition}

The set of pomsets $\sSKIP$ contains just the empty pomset.

\begin{definition}
  If $\aPS_0\in\sSKIP$ then
  $\aEvs_0 = \emptyset$.
\end{definition}

The set of pomsets $(\aPSS_1\sPAR\aPSS_2)$ represents the \emph{parallel composition} of pomsets from
$\aPSS_1$ and pomsets from $\aPSS_2$.
\begin{definition}
  \label{defn:pomset-par}
  If $\aPS_0 \in (\aPSS_1\sPAR\aPSS_2)$ then
  there are $\aPS_1\in\aPSS_1$ and $\aPS_2\in\aPSS_2$ where:
  \begin{enumerate}
  \item $\aEvs_0 = (\aEvs_1\cup\aEvs_2)$,
  \item $\aEvs_1$ and  $\aEvs_2$ are disjoint,
  \item if $\aEv\in\aEvs_1$ then $\labeling_0(\aEv) = \labeling_1(\aEv)$, 
  \item if $\aEv\in\aEvs_2$ then $\labeling_0(\aEv) = \labeling_2(\aEv)$,
  \item if $\bEv\le_1\aEv$ then $\bEv\le_0\aEv$, and
  \item if $\bEv\le_2\aEv$ then $\bEv\le_0\aEv$.
    \newcounter{pomsetParCount}
    \setcounter{pomsetParCount}{\value{enumi}}
  \end{enumerate}
\end{definition}
 
The set of pomsets $(\aPSS_1\sSEMI\aPSS_2)$ represents the \emph{strong sequential composition} of pomsets from
$\aPSS_1$ followed by pomsets from $\aPSS_2$.
\begin{definition}
  If $\aPS_0 \in (\aPSS_1\sSEMI\aPSS_2)$ then
  there are $\aPS_1\in\aPSS_1$ and $\aPS_2\in\aPSS_2$ where:
  \begin{enumerate}
    \setcounter{enumi}{\value{pomsetParCount}}
  \item[1--\thepomsetParCount)] As in Definition~\ref{defn:pomset-par}, and
  \item if $\bEv\in\aEvs_1$ and $\aEv\in\aEvs_2$ then $\bEv\le_0\aEv$.
  \end{enumerate}
\end{definition}

Write $\aAct$ for the set of pomsets where there at most one event, labelled $\aAct$
\begin{definition}
  If $\aPS_0 \in \aAct$ then
  \begin{enumerate}
  \item if $\bEv,\aEv\in\aEvs$ then $\bEv=\aEv$,
  \item if $\aEv\in\aEvs$ then $\labeling(\aEv)=\aAct$.
  \end{enumerate}
\end{definition}

[Define the alphabet of set, unset and wait on condition variables.]

\begin{definition}
  A pomset $\aPS$ is $\aChan$-closed if
  for every $\aEv$ where $\labeling(\aEv)=(\wait\aChan)$:
  \begin{enumerate}
  \item there is a $\bEv\le\aEv$ where $\labeling(\bEv)=(\set\aChan)$
  \item for any $\cEv$ where $\labeling(\cEv)=(\unset\aChan)$,
    either $\cEv\le\bEv$ or $\aEv\le\cEv$.
  \end{enumerate}
\end{definition}

\subsection{A pomset with preconditions model of Value-passing Asynchronus CCS}

This is essentially the model of \cite{10.1145/3428262}.

[Define a data model. Define the alphabet of send and recv of values over channels.]

\begin{definition}
  A \emph{pomset with preconditions} is
  a pomset with alphabet $(\Formulae \times \Act)$.
\end{definition}

The set of pomsets $\sSEND\aChan\aExp\sPREFIX\aPSS_1$ represents sending a value computed from expression $\aExp$ on channel $\aChan$,
followed by a pomset from $\aPSS_1$.
\begin{definition}
  If $\aPS_0 \in (\sSEND\aChan\aExp\sPREFIX\aPSS_1)$ then either $\aEvs_0=\emptyset$, or there is $\aPS_1\in\aPSS_1$ such that:
  \begin{enumerate}
  \item $\aEvs_0 = \{\cEv\} \cup \aEvs_1$,
  \item $\cEv\not\in\aEvs_1$,
  \item $\labelingAct_0(\cEv) = \send\aChan\aVal$
  \item $\labelingForm_0(\cEv)$ implies $(\aExp=\aVal)$
  \item if $\aEv\in\aEvs_1$ then $\labelingAct_0(\aEv) = \labelingAct_1(\aEv)$,
  \item if $\aEv\in\aEvs_1$ then $\labelingForm_0(\aEv)$ implies $\labelingForm_1(\aEv)$,
  \item if $\bEv\le_1\aEv$ then $\bEv\le_0\aEv$, and
  \item if $\aEv\in\aEvs_1$ then $\cEv\le_0\aEv$.
  \end{enumerate}
\end{definition}

The set of pomsets $\sRECV\aChan\aVar\sPREFIX\aPSS_1$ represents receiving a value bound to variable $\aVar$ on channel $\aChan$,
fopllowed by a pomset from $\aPSS_1$.
\begin{definition}
  If $\aPS_0 \in (\sRECV\aChan\aVar\sPREFIX\aPSS_1)$ then either $\aEvs_0=\emptyset$, or there is $\aPS_1\in\aPSS_1$ such that:
  \begin{enumerate}
  \item $\aEvs_0 = \{\cEv\} \cup \aEvs_1$,
  \item $\cEv\not\in\aEvs_1$,
  \item $\labelingAct_0(\cEv) = \recv\aChan\aVal$
  \item if $\aEv\in\aEvs_1$ then $\labelingAct_0(\aEv) = \labelingAct_1(\aEv)$,
  \item if $\aEv\in\aEvs_1$ then $\labelingForm_0(\aEv)$ implies $\labelingForm_1(\aEv)[\aVal/\aVar]$
  \item if $\bEv\le_1\aEv$ then $\bEv\le_0\aEv$, and
  \item if $\aEv\in\aEvs_1$ then $\cEv\le_0\aEv$.
  \end{enumerate}
\end{definition}

The set of pomsets $\aForm\sGUARD\aPSS_1$ represents a guarded execution of $\aPSS_1$ whenever $\aForm$ is satisfied.
\begin{definition}
  \label{defn:pomset-guard}
  If $\aPS_0 \in (\aForm\sGUARD\aPSS_1)$ then:
  \begin{enumerate}
  \item $\aEvs_0 = \aEvs_1$,
  \item $\labelingAct_0(\aEv) = \labelingAct_1(\aEv)$,
  \item $\labelingForm_0(\aEv)$ implies $\aForm\land\labelingForm_1(\aEv)$, and
  \item if $\bEv\le_1\aEv$ then $\bEv\le_0\aEv$.
    \newcounter{pomsetGuardCount}
    \setcounter{pomsetGuardCount}{\value{enumi}}
  \end{enumerate}
\end{definition}

\subsection{A pomset with preconditions and predicate transformers model of Value-passing Asynchronus CCS}

[The problem with the previous section is that there's no story for sequential composition.]

\begin{definition}
  A \emph{predicate transformer} is a monotone function
  $\aTr:\Formulae\fun\Formulae$ such that
  $\aTr(\FALSE)$ is $\FALSE$,
  $\aTr(\aForm\land\bForm)$ is $\aTr(\aForm)\land\aTr(\bForm)$, and
  $\aTr(\aForm\lor\bForm)$ is $\aTr(\aForm)\lor\aTr(\bForm)$.
\end{definition}

\begin{definition}
  A pomset with one predicate tansformer is a possibly incomplete
  pomset with preconditions, together with a predicate transformer $\aTr$.
\end{definition}

\begin{definition}
  If $\aPS_0 \in (\aPSS_1\sSEMI\aPSS_2)$ then
  there are $\aPS_1\in\aPSS_1$ and $\aPS_2\in\aPSS_2$ where:
  \begin{enumerate}
  \item $\aEvs_0 = (\aEvs_1\cup\aEvs_2)$,
  \item $\aEvs_1$ and  $\aEvs_2$ are disjoint,
  \item if $\aEv\in\aEvs_1$ then $\labelingAct_0(\aEv) = \labelingAct_1(\aEv)$, 
  \item if $\aEv\in\aEvs_2$ then $\labelingAct_0(\aEv) = \labelingAct_2(\aEv)$,
  \item if $\aEv\in\aEvs_1$ then $\labelingForm_0(\aEv)$ implies $\labelingForm_1(\aEv)$, 
  \item if $\aEv\in\aEvs_2$ then $\labelingForm_0(\aEv)$ implies $\aTr_1(\labelingForm_2(\aEv))$,
  \item if $\bEv\le_1\aEv$ then $\bEv\le_0\aEv$,
  \item if $\bEv\le_2\aEv$ then $\bEv\le_0\aEv$, and
  \item if $\bEv\in\aEvs_1$ and $\aEv\in\aEvs_2$ then $\bEv\le_0\aEv$.
  \end{enumerate}
\end{definition}

\begin{definition}
  If $\aPS_0 \in (\aForm\sGUARD\aPSS_1)$ then:
  \begin{enumerate}
    \setcounter{enumi}{\value{pomsetGuardCount}}
  \item[1--\thepomsetGuardCount)] As in Definition~\ref{defn:pomset-guard},
  \item $\aForm\land\aTr_0(\aForm)$ implies $\aTr_1(\aForm)$.
  \end{enumerate}
\end{definition}

\begin{definition}
  If $\aPS_0 \in \sSEND\aChan{\aExp}$ then:
  \begin{enumerate}
  \item if $\bEv\in\aEvs_0$ and $\aEv\in\aEvs_0$ then $\bEv=\aEv$,
  \item if $\aEv\in\aEvs_0$ then $\labelingAct_0(\aEv) = \send\aChan{\aVal}$,
  \item if $\aEv\in\aEvs_0$ then $\labelingForm_0(\aEv) = (M = v)$, and
  \item $\aTr_0(\aForm)$ implies $\aForm$.
  \end{enumerate}
\end{definition}

\begin{definition}
  If $\aPS_0 \in \sRECV\aChan{\aVar}$ then:
  \begin{enumerate}
  \item if $\bEv\in\aEvs_0$ and $\aEv\in\aEvs_0$ then $\bEv=\aEv$,
  \item if $\aEv\in\aEvs_0$ then $\labeling_0(\aEv) = \recv\aChan{\aVal}$,
  \item $\aTr_0(\aForm)$ implies $\phi[\aVal/\aVar]$, and
  \item if $\aEvs_0=\emptyset$ then $\aTr_0(\aForm)$ implies $\phi[\bVal/\aVar]$ for any $\bVal$.
  \end{enumerate}
\end{definition}

\subsection{A pomset with preconditions and predicate transformers model of Value-passing Relaxed CCS}

[Introduce relaxed channels, call other channels release/acquire channels, lift relaxed/ra up to actions and events.]

\begin{definition}
  A \emph{family of predicate transformers}
  indexed by subsets of $\aEvs$
  consists of predicate transformers
  $\aTr[\bEvs]$ for each set of events $\bEvs$,
  such that if $(\cEvs\cap\aEvs) \subseteq \bEvs$
  then $\aTr[\cEvs](\aForm)$ implies $\aTr[\bEvs](\aForm)$.
\end{definition}

\begin{definition}
  A pomset with predicate tansformers is a possibly incomplete
  pomset with preconditions, together with a family of predicate transformers $\aTr$
  indexed by subsets of $\aEvs$.
\end{definition}

\begin{definition}
  If $\aPS_0 \in (\aPSS_1\sSEMI\aPSS_2)$ then
  there are $\aPS_1\in\aPSS_1$ and $\aPS_2\in\aPSS_2$ where:
  \begin{enumerate}
  \item $\aEvs_0 = (\aEvs_1\cup\aEvs_2)$,
  \item $\aEvs_1$ and  $\aEvs_2$ are disjoint,
  \item if $\aEv\in\aEvs_1$ then $\labelingAct_0(\aEv) = \labelingAct_1(\aEv)$, 
  \item if $\aEv\in\aEvs_2$ then $\labelingAct_0(\aEv) = \labelingAct_2(\aEv)$,
  \item if $\aEv\in\aEvs_1$ then $\labelingForm_0(\aEv)$ implies $\labelingForm_1(\aEv)$, 
  \item if $\aEv\in\aEvs_2$ then $\labelingForm_0(\aEv)$ implies $\aTr[\downclose_0(e)]_1(\labelingForm_2(\aEv))$,
  \item if $\bEv\le_1\aEv$ then $\bEv\le_0\aEv$, and
  \item if $\bEv\le_2\aEv$ then $\bEv\le_0\aEv$.
  \end{enumerate}
\end{definition}

\begin{definition}
  If $\aPS_0 \in \sSEND\aChan{\aExp}$ then:
  \begin{enumerate}
  \item if $\bEv\in\aEvs_0$ and $\aEv\in\aEvs_0$ then $\bEv=\aEv$,
  \item if $\aEv\in\aEvs_0$ then $\labelingAct_0(\aEv) = \send\aChan{\aVal}$,
  \item if $\aEv\in\aEvs_0$ then $\labelingForm_0(\aEv) = (M = v)$, and
  \item $\aTr[\bEvs]_0(\aForm)$ implies $\aForm$.
  \end{enumerate}
\end{definition}

\begin{definition}
  If $\aPS_0 \in \sRECV\aChan{\aVar}$ then:
  \begin{enumerate}
  \item if $\bEv\in\aEvs_0$ and $\aEv\in\aEvs_0$ then $\bEv=\aEv$,
  \item if $\aEv\in\aEvs_0$ then $\labeling_0(\aEv) = \recv\aChan{\aVal}$,
  \item $\aTr[\bEvs]_0(\aForm)$ implies $\phi[\aVal/\aVar]$, and
  \item $\aTr[\emptyset]_0(\aForm)$ implies $\phi[\bVal/\aVar]$ for any $\bVal$.
  \end{enumerate}
\end{definition}
