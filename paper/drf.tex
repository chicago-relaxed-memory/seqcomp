% \section{Local Data Race Freedom and Sequential Consistency}
\section{LDRF-SC for \PwTmcaTITLE{}}
\label{sec:sc}

\todo{Remove this section?}

\begin{changed}
  In this appendix, we establish a \drfsc{} for \PwTmca{2}.  We prove an
  \emph{external} result, where the notion of \emph{data-race} is independent
  of the semantics itself.  Since every \PwTmca{2} is also a \PwTmca{1}, the
  result also applies there.  Our result is also \emph{local}.  Using
  \citeauthor{Dolan:2018:BDR:3192366.3192421}'s
  [\citeyear{Dolan:2018:BDR:3192366.3192421}] notion of \emph{Local Data Race
    Freedom (LDRF)}.

  We do not address \PwTc{}.  The internal \drfsc{} result for \cXI{}
  \cite{DBLP:phd/ethos/Batty15} does not rely on dependencies and thus
  applies to \PwTc{}.  In internal \drfsc{}, data-races are defined using the
  semantics of the language itself.  Using the notion of dependency defined
  here, it should be possible to prove an stronger external result for
  \cXI{}, similar to that of \cite{DBLP:conf/pldi/LahavVKHD17}---we leave
  this as future work.

  \citet{DBLP:journals/pacmpl/JagadeesanJR20} prove \ldrfsc{} for Pomsets
  with Preconditions (\PwP{}).  \PwTmca{} generalizes \PwP{} to account for
  sequential composition.  Most of the machinery of \ldrfsc{}, however, has
  little to do with sequential semantics.  Thus, we have borrowed heavily
  from the text of \cite{DBLP:journals/pacmpl/JagadeesanJR20}; indeed, we
  have copied directly from the \LaTeX{} source, which is publicly available.
  We indicate substantial changes or additions using a change-bar on the
  right.

  There are several changes:
  \begin{itemize}
  \item \PwP{} imposes several conditions that we have dropped:
    \emph{consistency}, \emph{causal strengthening}, \emph{downset closure}
    (see \textsection\ref{sec:diff}).
  \item \PwP{} allows preconditions that are stronger than the weakest precondition.
  \item \PwP{} imposes \ref{pom-rf-le} ($\rrfx$ implies $\lt$) and thus is
    similar to \PwTmca{1}.  \PwTmca{2} is a weaker model that is new to this
    paper.  %Hence we cannot use the \PwP{} result directly.
  \item \PwP{} did not provide an accurate account of program order for
    merged actions.  We use \reflem{lem:po} to correct this deficiency.
  \end{itemize}
  The first two items require us to define $\semmin{}$ differently, below.
\end{changed}

The result requires that locations are properly initialized.  We assume a
sufficient condition: that programs have the form
``$\aLoc_1\GETS\aVal_1\SEMI \cdots \aLoc_n\GETS\aVal_n\SEMI\aCmd$'' where
every location mentioned in $\aCmd$ is some $\aLoc_i$.  To simplify the
definition of \emph{happens-before}, we ban fences and \RMWs.

To state the theorem, we require several technical definitions.  The reader
unfamiliar with \citep{Dolan:2018:BDR:3192366.3192421} may prefer to skip to
the examples in the proof sketch, referring back as needed.

\bookmark{Definitions}
\begin{changed}
  \paragraph{Program Order}
  Let $\sempomca{2}{}$ be defined by applying the construction of
  \reflem{lem:po} to $\semmca{2}{}$.  We consider only \emph{complete}
  pomsets.  For these, we derive program order on compound events as follows.
  By \reflem{lem:compound:phantom:exists}, if there is a compound event
  $\aEv$, then there is a phantom event $\cEv\in\fmrginv{\aEv}$ such that
  $\labelingForm(\cEv)$ is a tautology.  If there is exactly one tautology,
  we identify $\aEv$ with $\cEv$ in program order.  If there is more than one
  tautology, \reflem{lem:sequential:simple}, below, shows that it suffices to
  pick an arbitrary one---we identify $\aEv$ with the $\cEv\in\fmrginv{\aEv}$
  that is minimal in program order.
\end{changed}
For example, consider \jmm{} causality test case 2, with an added write to $z$:
\begin{gather*}
  \tag{$\ddagger\ddagger$}\label{eq6}
  \begin{gathered}
    \PR{x}{r}\SEMI
    \PW{z}{1}\SEMI
    \PR{x}{s}\SEMI
    \IF{r{=}s}\THEN \PW{y}{1}\FI
    \PAR
    \PW{x}{y}
    \\
    \hbox{\begin{tikzinline}[node distance=1em and 1.5em]
        \phevent{p1}{\DR{x}{1}}{}
        \event{z}{\DW{z}{1}}{right=of p1}
        \phevent{p2}{\DR{x}{0}}{right=of z}
        \event{a3}{\DW{y}{1}}{right=of p2}
        \event{b1}{\DR{y}{1}}{right=3em of a3}
        \event{b2}{\DW{x}{1}}{right=of b1}
        \rf{a3}{b1}
        \po{b1}{b2}
        \pox[out=-15,in=-170]{p1}{z}
        \pox[out=-15,in=-170]{z}{p2}
        \pox[out=-15,in=-170]{p2}{a3}
        \pox[out=-15,in=-170]{b1}{b2}
        \event{a1}{\DR{x}{1}}{above=of z}
        \mrg{p1}[pos=.3]{a1}
        \mrg{p2}[right]{a1}      
        \rf{b2}{a1}
        \pox{a1}{z}
      \end{tikzinline}}
  \end{gathered}
\end{gather*}
% For TC2, we have: 
% \begin{gather*}
%   \PR{x}{r}\SEMI
%   \PR{x}{s}\SEMI
%   \IF{r{=}s}\THEN \PW{y}{1}\FI
%   \PAR
%   \PW{x}{y}
%   \\
%   \hbox{\begin{tikzinline}[node distance=1.5em]
%       \event{a1}{\DR{x}{1}}{}
%       % \event{a2}{\DR{x}{1}}{right=of a1}
%       \event{a3}{\DW{y}{1}}{right=of a1}
%       % \po{a1}{a3}
%       % \po[out=-20,in=-160]{a1}{a3}
%       \event{b1}{\DR{y}{1}}{right=3em of a3}
%       \event{b2}{\DW{x}{1}}{right=of b1}
%       \rf{a3}{b1}
%       \po{b1}{b2}
%       % \rf[out=169,in=11]{b2}{a2}
%       \rf[out=169,in=11]{b2}{a1}
%       \phevent{p2}{\DR{x}{1}}{below=of a1}
%       \phevent{p1}{\DR{x}{1}}{left=of p2}
%       \mrg{p1}[pos=.3]{a1}
%       \mrg{p2}[right]{a1}      
%       \pox[out=-15,in=-170]{b1}[below]{b2}
%       \pox[out=-15,in=-170]{p1}[below]{p2}
%       \pox{p2}[below]{a3}
%     \end{tikzinline}}
%   \\
%   \hbox{\begin{tikzinline}[node distance=1.5em]
%       \event{a1}{\DR{x}{0}}{}
%       % \event{a2}{\DR{x}{1}}{right=of a1}
%       \event{a3}{\DW{y}{1}}{right=of a1}
%       % \po{a1}{a3}
%       % \po[out=-20,in=-160]{a1}{a3}
%       \event{b1}{\DR{y}{1}}{right=3em of a3}
%       \event{b2}{\DW{x}{1}}{right=of b1}
%       \rf{a3}{b1}
%       \po{b1}{b2}
%       % \rf[out=169,in=11]{b2}{a2}
%       %\rf[out=169,in=11]{b2}{a1}
%       \phevent{p2}{\DR{x}{0}}{below=of a1}
%       \phevent{p1}{\DR{x}{0}}{left=of p2}
%       \mrg{p1}[pos=.3]{a1}
%       \mrg{p2}[right]{a1}      
%       \pox[out=-15,in=-170]{b1}[below]{b2}
%       \pox[out=-15,in=-170]{p1}[below]{p2}
%       \pox{p2}[below]{a3}
%     \end{tikzinline}}
% \end{gather*}




\paragraph{Data Race}
Data races are defined using \emph{program} order $(\rpox)$, not
\emph{pomset} order $(\lt)$. %, and thus is stable with respect to augmentation.


Because we ban fences and \RMWs, we can adopt the simplest definition of
\emph{synchronizes\hyp{}with}~($\rsw$): Let $\bEv\xsw\aEv$ exactly when
$\bEv$ fulfills $\aEv$, $\bEv$ is a release, $\aEv$ is an acquire, and
$\lnot(\bEv\xpox\aEv)$.

Let ${\rhb}=({\rpox}\cup{\rsw})^+$ be the \emph{happens-before} relation.

% \begin{changed}
%   Note that ${\rhb}$ is a preorder in general. It is a partial order on
%   \emph{simple} events (\refdef{def:po}).
% \end{changed}

Let $L\subseteq\Loc$ be a set of locations.  We say that $\bEv$ \emph{has an
  $L$-race with} $\aEv$ (notation $\bEv\lrace{L}\aEv$) when
\begin{enumerate*}
\item at least one is relaxed, 
\item at least one is a write,
\item they access the same location in $L$, and
\item they are unordered by $\rhb$: neither $\bEv\xhb\aEv$ nor
  $\aEv\xhb\bEv$.
\end{enumerate*}

\begin{changed}
  \paragraph{Generators}
  We say that $\aPS'\in\downclose{\aPSS}$ if there is some $\aPS\in\aPSS$ such that
  $\aPS$ is \emph{complete} (\refdef{def:pwt:mca:complete}) and $\aPS'$ is a
  \emph{downset} of $\aPS$ (\refdef{def:downset}).
  
  % We say that $\aPS'$ \emph{generates} $\aPS$ if
  % $\aPS$ augments $\aPS'$ (\refdef{def:augment}).

  Let $\aPS$ be \emph{augmentation-minimal} in $\aPSS$ if $\aPS\in\aPSS$ and
  there is no $\aPS{\neq}\aPS'{\in}\aPSS$ such that $\aPS$ augments $\aPS'$.

  Let $\semmin{\aCmd}=\{\aPS\in\downclose{}\sempomca{2}{\aCmd} \mid \aPS$ is
  augmentation-minimal in $\downclose{}\sempomca{2}{\aCmd}\}$.
\end{changed}

\paragraph{Extensions}

We say that $\aPS'$ \emph{$\aCmd$-extends} $\aPS$ if %$\aPS\in\semmin{\aCmd}$,
$\aPS\neq\aPS'\in\semmin{\aCmd}$ and $\aPS$ is a downset of $\aPS'$.

\paragraph{Similarity}
We say that \emph{$\aPS'$ is $\aEv$-similar to $\aPS$} if they differ at most
in
\begin{enumerate*}
\item pomset order adjacent to $\aEv$, 
\item the value associated with event $\aEv$, if it is a read, and
\item the addition and removal of read events $\rpox$-after $\aEv$.
\end{enumerate*}
% We say they are \emph{similar} if they are $\aEv$-similar for some $\aEv$.
% Formally: $\Event'=\Event$, $\labelingForm'=\labelingForm$,
% $\restrict {\le'}{\Event\setminus\{\aEv\}}=\restrict {\le}{\Event\setminus\{\aEv\}}$,
% if $\aEv$ is not a read then $\labelingAct'=\labelingAct$, and if $\aEv$ is a
% read then
% $\restrict{\labelingAct'}{\Event\setminus\{\aEv\}}=\restrict{\labelingAct}{\Event\setminus\{\aEv\}}$
% and $\labelingAct'(\aEv) = \labelingAct(\aEv)[\aVal'/\aVal]$, for some
% $\aVal'$, $\aVal$.

\paragraph{Stability}
We say that $\aPS$ is \emph{$L$-stable in $\aCmd$} if
\begin{enumerate*}  
\item $\aPS\in\semmin{\aCmd}$, 
\item $\aPS$ is $\rpox$-convex (nothing missing in program order),
\item there is no $\aCmd$-extension of $\aPS$ with a \emph{crossing}
  $L$-race: that is, there is no $\bEv\in\Event$, no $\aPS'$
  $\aCmd$-extending $\aPS$, and no $\aEv\in\Event'\setminus\Event$ such that
  $\bEv\lrace{L}\aEv$.
\end{enumerate*}
The empty pomset is $L$-stable.

\paragraph{Sequentiality}
Let ${\pole{L}}={\lt_L}\cup{\rpox}$, where $\lt_L$ is the restriction of $\lt$ to events that access locations in $L$.
We say that $\aPS'$ is \emph{$L$-sequential after $\aPS$} if 
\begin{enumerate*}
\item $\aPS'$ is $\rpox$-convex,
\item $\pole{L}$ is acyclic in $\Event'\setminus\Event$.
\end{enumerate*}

\begin{changed}
  \paragraph{Simplicity}
  We say that $\aPS'$ is \emph{$L$-simple after $\aPS$} if all of the events
  in $\Event'\setminus\Event$ that access locations in $L$ are \emph{simple}
  (\refdef{def:po}).

  \begin{lemma}
    \label{lem:sequential:simple}
    Suppose $\aPS'\in\semmin{\aCmd}$ and $\aPS$ is {$L$-sequential after
      $\aPS$}.
    Let $\aPS''$ be the restriction of $\aPS'$ that is {$L$-simple after
      $\aPS$} (throwing out compound $L$-events after $\aPS$).
    Then $\aPS''\in\semmin{\aCmd}$.
    % projection of $\aPS$ is a $\aPS'\in\semmin{\aCmd}$ such that
    % \begin{enumerate*}
    % \item $\aPS'$ is $L$-sequential and $L$-simple after $\aPS$, and
    % \item  $\aPS'$ is  derived from  $\aPS$  by removing  every compound  event
    %   $\aEv$ that  accesses a location in  $L$, thus changing the remaining
    %   events in $\fmrginvp{\aEv}$ from phantom events to real ones.
    % \end{enumerate*}
  \end{lemma}
  As a negative example, note that \eqref{eq6} is not $L$-sequential---in
  fact there is no execution of the program that results in the simple events
  of \eqref{eq6}: without merging the reads, there would be a dependency
  $\DRP{x}{1} \xpo \DWP{y}{1}$.  $L$-sequential executions of this code must read
  $0$ for $x$:
  \begin{gather*}
    \PR{x}{r}\SEMI
    \PW{z}{1}\SEMI
    \PR{x}{s}\SEMI
    \IF{r{=}s}\THEN \PW{y}{1}\FI
    \PAR
    \PW{x}{y}
    \\
    \hbox{\begin{tikzinline}[node distance=1em and 1.5em]
        \phevent{p1}{\DR{x}{0}}{}
        \event{z}{\DW{z}{1}}{right=of p1}
        \phevent{p2}{\DR{x}{0}}{right=of z}
        \event{a3}{\DW{y}{1}}{right=of p2}
        \event{b1}{\DR{y}{1}}{right=3em of a3}
        \event{b2}{\DW{x}{1}}{right=of b1}
        \rf{a3}{b1}
        \po{b1}{b2}
        \pox[out=-15,in=-170]{p1}{z}
        \pox[out=-15,in=-170]{z}{p2}
        \pox[out=-15,in=-170]{p2}{a3}
        \pox[out=-15,in=-170]{b1}{b2}
        \event{a1}{\DR{x}{0}}{above=of z}
        \mrg{p1}[pos=.3]{a1}
        \mrg{p2}[right]{a1}      
        \pox{a1}{z}
      \end{tikzinline}}
  \end{gather*}
\end{changed}

\bookmark{Theorem and Proof Sketch}
\begin{theorem}
  Let $\aPS$ be $L$-stable in $\aCmd$.  Let $\aPS'$ be a $\aCmd$-extension of
  $\aPS$ that is $L$-sequential after $\aPS$.  Let $\aPS''$ be a
  $\aCmd$-extension of $\aPS'$ that is $\rpox$-convex, such that no subset of
  $\Event''$ satisfies these criteria.
  Then either (1) $\aPS''$ is $L$-sequential
  \begin{changed}
    and $L$-simple
  \end{changed}
  after $\aPS$ or (2) there is some $\aCmd$-extension $\aPS'''$ of $\aPS'$
  and some $\aEv\in(\Event''\setminus\Event')$ such that (a) $\aPS'''$ is
  $\aEv$-similar to $\aPS''$, (b) $\aPS'''$ is $L$-sequential
  \begin{changed}
    and $L$-simple
  \end{changed}
  after $\aPS$, and (c) $\bEv\lrace{L}\aEv$, for some
  $\bEv\in(\Event''\setminus\Event)$.
\end{theorem}
The theorem provides an inductive characterization of \emph{Sequential
  Consistency for Local
  Data-Race Freedom (SC-LDRF)}: Any extension of a $L$-stable pomset is either
$L$-sequential, or is $\aEv$-similar to a $L$-sequential extension that
includes a race involving $\aEv$.
\begin{proof}[Proof Sketch]
  \begin{changed}
    We show $L$-sequentiality.  $L$-simplicity then follows from
    \reflem{lem:sequential:simple}.
  \end{changed}

  In order to develop a technique to find $\aPS'''$ from $\aPS''$, we analyze
  pomset order in generation-minimal top-level pomsets.  First, we note that
  $\lt_*$ (the transitive reduction $\lt$) can be decomposed into three
  disjoint relations.  Let ${\rppo}=({\lt_*}\cap{\rpox})$ denote
  \emph{preserved} program order, as required by sequential composition and
  conditional.  The other two relations are cross-thread subsets of
  $({\lt_*}\setminus{\rpox})$: $\rrfe$ (reads-from-external) orders writes
  before reads, satisfying \ref{par-le-rf1} and \ref{par-le-rf2}; $\rcae$
  (coherence-after-external) orders read and write accesses before writes,
  satisfying \ref{pom-rf-block}. (Within a thread, \ref{seq-le-delays-rf} and
  \ref{seq-le-rf-rf} induce order that is included in ${\rppo}$.)

  Using this decomposition, we can show the following.
  \begin{lemma}
    Suppose $\aPS''\in\semmin{\aCmd}$ has an external read $\bEv\xrfxpp\aEv$ that is
    maximal in $({\rppo}\cup{\rrfe})$.  Further suppose that there another
    write $\bEv'$ that could fulfill $\aEv$.
    Then there exists an $\aEv$-similar $\aPS'''$ with $\bEv'\xrfxppp\aEv$
    such that $\aPS'''\in\semmin{\aCmd}$.
    % and such that every $\rpox$-following read is also $\le$-following
    % ($\aEv\xpox\bEv$ implies $\aEv\le\bEv$, for every read $\bEv$).
    % Further, suppose there is an $\aEv$-similar $\aPS'''$
    % that satisfies the requirements of fulfillment.  Then
    % $\aPS'''\in\semmin{\aCmd}$.
  \end{lemma}
  The proof of the lemma follows an inductive construction of
  $\semmin{\aCmd}$, starting from a large set with little order, and
  pruning the set as order is added: We begin with all pomsets generated by
  the semantics without imposing the requirements of fulfillment (including
  only $\rppo$).  We then prune reads which cannot be fulfilled, starting
  with those that are minimally ordered.
  % This proof is simplified by
  % precluding local declarations.

  We can prove a similar result for $({\rpox}\cup{\rrfe})$-maximal read
  and write accesses.

  Turning to the proof of the theorem, if $\aPS''$ is $L$-sequential after
  $\aPS$, then the result follows from (1).  Otherwise, there must be a
  $\pole{L}$ cycle in $\aPS''$ involving all of the actions in
  $(\Event''\setminus\Event')$: If there were no such cycle, then $\aPS''$
  would be $L$-sequential; if there were elements outside the cycle, then
  there would be a subset of $\Event''$ that satisfies these criteria.

  If there is a $({\rpox}\cup{\rrfe})$-maximal access, we select one of
  these as $\aEv$.  If $\aEv$ is a write, we reverse the outgoing order in
  $\rcae$; the ability to reverse this order witnesses the race.  If $\aEv$
  is a read, we switch its fulfilling write to a ``newer'' one, updating
  $\rcae$; the ability to switch witnesses the race.  For
  example, for $\aPS''$ on the left below, we choose the $\aPS'''$ on the
  right;  $\aEv$ is the read of $x$, which races with $(\DW{x}{1})$.  % Program order
  \begin{gather*}
    x\GETS 0 \SEMI y\GETS 0 \SEMI  (x \GETS 1  \SEMI y \GETS 1
    \PAR
    \IF{y}\THEN \aReg \GETS x \FI)
    \\[-.5ex]
    \hbox{\begin{tikzinline}[node distance=1.5em and 2em]
        \event{wy0}{\DW{y}{0}}{}
        \event{wx0}{\DW{x}{0}}{below=of wy0}
        \event{wx1}{\DW{x}{1}}{right=3em of wy0}
        \event{wy1}{\DW{y}{1}}{right=of wx1}
        \event{ry1}{\DR{y}{1}}{below=of wx1}
        \event{rx}{\DR{x}{0}}{below=of wy1}
        \rf[bend right]{wx0}{rx}
        \rf{wy1}{ry1}
        \wk[bend left]{wy0}{wy1}
        \pox{wx1}{wy1}
        \pox{ry1}[below]{rx}
        \wk{rx}{wx1}
        \node(ix)[left=of wx0]{};
        \node(iy)[left=of wy0]{};
        \bgoval[yellow!50]{(ix)(iy)}{P}
        \bgoval[pink!50]{(wx0)(wy0)}{P'\setminus P}
        \bgoval[green!10]{(ry1)(wx1)(rx)(wy1)}{P''\setminus P'}
        \pox{wx0}{wy0}
        \pox{wy0}{wx1}
        \pox{wy0}[below]{ry1}
      \end{tikzinline}}
    \qquad
    \hbox{\begin{tikzinline}[node distance=1.5em and 2em]
        \event{wy0}{\DW{y}{0}}{}
        \event{wx0}{\DW{x}{0}}{below=of wy0}
        \event{wx1}{\DW{x}{1}}{right=3em of wy0}
        \event{wy1}{\DW{y}{1}}{right=of wx1}
        \event{ry1}{\DR{y}{1}}{below=of wx1}
        \event{rx}{\DR{x}{1}}{below=of wy1}
        \rf{wx1}{rx}
        \rf{wy1}{ry1}
        \wk[bend left]{wy0}{wy1}
        \pox{wx1}{wy1}
        \pox{ry1}[below]{rx}
        \wk{wx0}{wx1}
        \node(ix)[left=of wx0]{};
        \node(iy)[left=of wy0]{};
        \bgoval[yellow!50]{(ix)(iy)}{P}
        \bgoval[pink!50]{(wx0)(wy0)}{P'\setminus P}
        \bgoval[green!10]{(ry1)(wx1)(rx)(wy1)}{P'''\setminus P'}
        \pox{wx0}{wy0}
        \pox{wy0}{wx1}
        \pox{wy0}[below]{ry1}
      \end{tikzinline}}
  \end{gather*}
  It is important that $\aEv$ be $({\rpox}\cup{\rrfe})$-maximal, not just
  $({\rppo}\cup{\rrfe})$-maximal.  The latter criterion would allow us to
  choose $\aEv$ to be the read of $y$, but then there would be no
  $\aEv$-similar pomset: if an execution reads $0$ for $y$ then there is no
  read of $x$, due to the conditional.

  \begin{changed}
    In the above argument, it is unimportant whether $\aEv$ reads-from an
    internal or an external write; thus the argument applies to \PwTmca{2}
    and \PwTmca{1} as it does for \PwTmca{1}.
  \end{changed}
  
  If there is no $({\rpox}\cup{\rrfe})$-maximal access, then all cross-thread
  order must be from $\rrfe$.  In this case, we select a
  $({\rppo}\cup{\rrfe})$-maximal read, switching its fulfilling write to an
  ``older'' one.  If there are several of these, we choose one that is
  $\rpox$-minimal.  As an example, consider the following; once again, $\aEv$
  is the read of $x$, which races with $(\DW{x}{1})$.
  \begin{gather*}
    x\GETS 0 \SEMI y\GETS 0 \SEMI (\aReg \GETS x  \SEMI y \GETS 1
    \PAR
    \bReg \GETS y \SEMI x \GETS \bReg)
    \\[-.5ex]
    \hbox{\begin{tikzinline}[node distance=1.5em and 2em]
        \event{wx0}{\DW{y}{0}}{}
        \event{ry}{\DR{x}{1}}{right=3em of wx0}
        \event{wx1}{\DW{y}{1}}{right=of ry}
        \event{wy0}{\DW{x}{0}}{below=of wx0}
        \event{rx1}{\DR{y}{1}}{right=3em of wy0}
        \event{wy1}{\DW{x}{1}}{right=of rx1}
        \rf{wx1}{rx1}
        \rf{wy1}{ry}
        \po{rx1}{wy1}
        \pox{ry}{wx1}
        \wk[bend left]{wx0}{wx1}
        \wk[bend right]{wy0}{wy1}
        \node(ix)[left=of wx0]{};
        \node(iy)[left=of wy0]{};
        \bgoval[yellow!50]{(ix)(iy)}{P}
        \bgoval[pink!50]{(wx0)(wy0)}{P'\setminus P}
        \bgoval[green!10]{(ry)(wx1)(rx1)(wy1)}{P''\setminus P'}
        \pox{wy0}{wx0}
        \pox{wx0}{ry}
        \pox{wx0}[below]{rx1}
      \end{tikzinline}}
    \qquad
    \hbox{\begin{tikzinline}[node distance=1.5em and 2em]
        \event{wx0}{\DW{y}{0}}{}
        \event{ry}{\DR{x}{0}}{right=3em of wx0}
        \event{wx1}{\DW{y}{1}}{right=of ry}
        \event{wy0}{\DW{x}{0}}{below=of wx0}
        \event{rx1}{\DR{y}{1}}{right=3em of wy0}
        \event{wy1}{\DW{x}{1}}{right=of rx1}
        \pox{ry}{wx1}
        \wk[bend left]{wx0}{wx1}
        \rf{wx1}{rx1}
        \rf{wy0}{ry}
        \po{rx1}{wy1}
        \wk{ry}{wy1}
        \node(ix)[left=of wx0]{};
        \node(iy)[left=of wy0]{};
        \bgoval[yellow!50]{(ix)(iy)}{P}
        \bgoval[pink!50]{(wx0)(wy0)}{P'\setminus P}
        \bgoval[green!10]{(ry)(wx1)(rx1)(wy1)}{P'''\setminus P'}
        \pox{wy0}{wx0}
        \pox{wx0}{ry}
        \pox{wx0}[below]{rx1}
      \end{tikzinline}}
  \end{gather*}
  This example requires $(\DW{x}{0})$.  Proper initialization ensures the
  existence of such ``older'' writes.
\end{proof}


