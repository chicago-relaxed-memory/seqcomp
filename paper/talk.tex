\documentclass[t,aspectratio=169]{beamer} %tlmgr install translator
% \usepackage{media9}

\usepackage{macros}
\usepackage{talk}

% \includeonlyframes{oota-pomsets}  
\usepackage{pdfcomment}
% \setbeameroption{show notes on second screen}
\newcommand<>{\pdfnote}[1]{%
  \only#2{\marginnote{\pdfcomment[icon=note,hspace=10pt,color=white,hoffset=30cm]{#1}}}%
  \note#2[item]{#1}%
}
\usecolortheme{seahorse}
\usetheme{default}
\setbeamertemplate{navigation symbols}{}%\insertframenumber/\inserttotalframenumber}
\setbeamertemplate{itemize item}[circle]
\setbeamertemplate{itemize subitem}[circle]
\setbeamertemplate{itemize subsubitem}[circle]


\begin{comment}
  
  While developing slides, use
  \includeonlyframes{current}  

  Then latex will only process slides with
  \begin{frame}[label=y]
  \end{frame}  
  
  \begin{frame}
    \frametitle{}
    \begin{itemize}
    \item 
    \end{itemize}
  \end{frame}
  \begin{itemize}
  \item 
  \end{itemize}

  \begin{framecenter}
  \end{framecenter}

  Great Beamer guide here:
  https://www.overleaf.com/learn/latex/Beamer_Presentations:_A_Tutorial_for_Beginners_(Part_4)—Overlay_Specifications

  How to present on mac without blurring:
  Use firefox, or Presentation.app
  https://tex.stackexchange.com/questions/431423/os-x-blurry-beamer-presentations-on-sierra-high-sierra

  With Presentation.app:
  \usepackage{pdfcomment}
  \newcommand{\pdfnote}[1]{\marginnote{\pdfcomment[icon=note]{#1}}}

  To convert from PDF to KeyNOTE: use the automator suggestion here
  https://apple.stackexchange.com/questions/95856/import-multiple-pages-from-a-pdf-as-separate-slides-in-keynote
  I put the automator app here: https://www.dropbox.com/sh/1o36nes45cydmbx/AADPVWpeEBZj2vemVXkGKQXVa?dl=0
  - Copy that to your machine somewhere.
  - In the finder, drop talk.pdf onto "PDF Save Pages as Images.app".
  - This will create a bunch of png files on the desktop
  - Now, open keynote with a blank document.
  - In the finder, select the files, drop onto the leftmost "slides" pane of Keynote.

  Another display option is to use http://dspdfviewer.danny-edel.de

  For Beamer notes:
  \usepackage{pgfpages}
  \setbeameroption{show notes on second screen}

  Skim is not great for this:

  To give a presentation with the Skim reader (http://skim-app.sourceforge.net) %on OSX so
  that you see the notes on your laptop and the slides on the projector, do %the following:
  
  1. Generate just the presentation (hide notes) and save to slides.pdf
  2. Generate only the notes (show only nodes) and save to notes.pdf
  3. With Skim open both slides.pdf and notes.pdf
  4. Click on slides.pdf to bring it to front.
  5. In Skim, under "View -> Presentation Option -> Synchronized Noted Document"
  select notes.pdf.
  6. Now as you move around in slides.pdf the notes.pdf file will follow you.
  7. Arrange windows so that notes.pdf is in full screen mode on your laptop
  and slides.pdf is in presentation mode on the projector.
  You can use the keys "t" and "d" to get different things on the presenter screen
  
  \pause
  \onslide<>

  \visible<>{during}                 \begin{visibleenv}<>during\end{visibleenv}
  \invisible<>{other}                \begin{invisibleenv}<>other \end{invisibleenv}
  
  \onslide<>{during}
  \uncover<>{during}                 \begin{uncoverenv}<>during\end{uncoverenv}
  
  \only<>{during}                    \begin{onlyenv}<>during\end{onlyenv}
  \alt<>{during}{other}              \begin{altenv}<>{during-before}{during-after}{other-before}{other-after}text\end{altenv}
  \temporal<>{before}{during}{after}
  \setbeamercovered{transparent=35}


  \begin{itemize}[<+-| alert@+>]
  \end{itemize}

  \begin{columns}
    \begin{column}{0.5\textwidth}
    \end{column}
    \begin{column}{0.5\textwidth}
    \end{column}
  \end{columns}

\end{comment}

\title{The Leaky Semicolon}
\subtitle{Compositional Semantic Dependencies for Relaxed-Memory Concurrency}
\author{
  \alert{Alan Jeffrey}$^*$
  \and
  \alert{James Riely}$^\dagger$
  \and
  Mark Batty$^\ddagger$
  \\
  Simon Cooksey$^\ddagger$
  \and
  Ilya Kaysin$^{**}$
  \and
  Anton Podkopaev$^{\dagger\dagger}$
}
\date{POPL\\January 2022}
\institute{
  $^*$Roblox
  \qquad
  $^\dagger$DePaul
  \qquad
  $^\ddagger$Kent
  \and
  $^{**}$JetBrains + Cambridge
  \qquad
  $^{\dagger\dagger}$HSE
}


\begin{document}

\begin{comment}
\begin{frame}[label=title]
  \maketitle
\end{frame}

% \begin{frame}
%   \frametitle{Are these programs equal?}
%   \huge
%   \renewcommand{\aLoc}{{\color{blue}x}}
%   \renewcommand{\bLoc}{{\color{green}y}}
%   \renewcommand{\aReg}{{\color{red}r}}
%   \renewcommand{\bReg}{{\color{red}s}}
%   \begin{align*}
%     \only<-2,4->{\phantom{\SEMI\framebox{$\PW{\aLoc}{\PR{\bLoc}{}}$}}}
%     \PR{\aLoc}{\aReg} \SEMI \PW{\bLoc}{\aReg}
%     \only<3>{\SEMI\framebox{$\PR{\bLoc}{\bReg}$}}
%     &\visible<2->{\;\;\neq\;\;
%     \PW{\bLoc}{\aReg} \SEMI  \PR{\aLoc}{\aReg}}
%     \only<3>{\SEMI\framebox{$\PR{\bLoc}{\bReg}$}}
%     \only<-2,4->{\phantom{\SEMI\framebox{$\PW{\aLoc}{\PR{\bLoc}{}}$}}}
%     %     \\[1ex]
%     %     \visible<3->{
%     %     \alt<6>{\PW{\bLoc}{1}\SEMI  \PR{\aLoc}{\aReg}}{\PR{\aLoc}{\aReg} \only<4>{\SEMI\framebox{$\PW{\aLoc}{\PR{\bLoc}{}}$}}\SEMI \PW{\bLoc}{1}}
%     %     &\;\;\alt<4,5>{\neq}{=}\;\;
%     %     \PW{\bLoc}{1} \only<4>{\SEMI\framebox{$\PW{\aLoc}{\PR{\bLoc}{}}$}}\SEMI  \PR{\aLoc}{\aReg}}
%     %     \\[1ex]
%     %     \visible<8->{\PW{\aLoc}{1} \SEMI  \PW{\aLoc}{1}
%     %     &\;\;\neq\;\;
%     %     \PW{\aLoc}{1}}
%   \end{align*}
% \end{frame}
% \begin{frame}
%   \frametitle{Are these programs equal?}
%   \huge
%   \renewcommand{\aLoc}{{\color{blue}x}}
%   \renewcommand{\bLoc}{{\color{green}y}}
%   \renewcommand{\aReg}{{\color{red}r}}
%   \begin{align*}
%     \phantom{\SEMI\framebox{$\PW{\aLoc}{\PR{\bLoc}{}}$}}
%     \PR{\aLoc}{\aReg} \SEMI  \PW{\bLoc}{\aReg}
%     &\visible<2->{\;\;\neq\;\;
%     \PW{\bLoc}{\aReg} \SEMI  \PR{\aLoc}{\aReg}}
%     \phantom{\SEMI\framebox{$\PW{\aLoc}{\PR{\bLoc}{}}$}}
%     \\[1ex]
%     \visible<3->{
%     \alt<6>{\PW{\bLoc}{1}\SEMI  \PR{\aLoc}{\aReg}}{\PR{\aLoc}{\aReg} \only<4>{\SEMI\framebox{$\PW{\aLoc}{\PR{\bLoc}{}}$}}\SEMI \PW{\bLoc}{1}}
%     &\;\;\alt<4,5>{\neq}{=}\;\;
%     \PW{\bLoc}{1} \only<4>{\SEMI\framebox{$\PW{\aLoc}{\PR{\bLoc}{}}$}}\SEMI  \PR{\aLoc}{\aReg}}
%     \\[1ex]
%     \visible<8->{\PW{\aLoc}{1} \SEMI  \PW{\aLoc}{2}
%     &\;\;=\;\;
%     \PW{\aLoc}{2}}
%   \end{align*}
% \end{frame}

% \begin{frame}
%   \huge
%   \begin{gather*}
%     \begin{aligned}
%       r &\in  \text{Registers}\\
%       M &\in  \text{Expressions}\\
%       \visible<2->{x,y,z &\in \text{Shared Variables}\\}
%     \end{aligned}
%     \\
%     \begin{array}{rcccccccc}
%       S&\BNFDEF&\SKIP&\BNFSEP&\LET{r}{M}&\BNFSEP&S_1 \SEMI S_2
%       \\
%       \visible<2->{&\BNFSEP&\PW[]{x}{M}&\BNFSEP&\PR[]{x}{r}&\BNFSEP&S_1 \PAR S_2}
%     \end{array}
%   \end{gather*}
% \end{frame}



\begin{frame}
  \frametitle{Remembering nothing that happened after 1999 \ldots}
  \onslide<2->Develop a model of imperative programming
  \onslide<3->that supports:
  \begin{itemize}
  \item<3-> Shared-memory concurrency 
  \item<4-> Compositional reasoning for sequencing and parallelism
  \item<5-> A few other things:
    \begin{itemize}
    \item Features: pointers, release/acquire/SC access, fences, RMWs
    \item Optimizations: reorderings, if-introduction, redundant read elimination, etc.
    \item Java Causality Test Cases 
    \item DRF-SC, No-Thin-Air, temporal invariant reasoning
    \item Efficient implementation on Arm8
    \item Connection to C11, JavaScript, PTX, etc.
    \item Automatic litmus test checker
    \item Certified proofs
    \end{itemize}
  \end{itemize}
\end{frame}

\begin{frame}
  \frametitle{Sequential Computation\visible<3->{: Preconditions + \alt<9->{Predicate Transformers}{Hoare Logic}!}}
  % \frametitle{\alt<-2>{What does this mean?}{Preconditions!}}
  \huge
  \begin{displaymath}
    \PW{x}{0}
    \SEMI
    \PR{x}{r}
    \SEMI
    \PW{y}{r+1}
  \end{displaymath}
  \vspace{-2ex}
  \only<2>{\sol{Preconditions!}}
  \alt<-8>{
    \begin{displaymatharray}{rll}
      \visible<7->{\alt<7>{\hoare{}{&\PW{x}{0}&}  {x=0}}{\hoare{\TRUE}{&\PW{x}{0}&}  {x=0}}}\\
      \visible<5->{\alt<5>{\phantom{\{x=0\}}\hoare{}{&\PR{x}{r}&}  {r=0}}{\hoare{x=0}  {&\PR{x}{r}&}  {r=0}}}\\
      \visible<3->{\alt<3>{\hoare{}{&\PW{y}{r+1}&}{y=1}}{\hoare{r=0}  {&\PW{y}{r+1}&}{y=1}}}
    \end{displaymatharray}
  }{
    \begin{displaymatharray}{rll}
      \dij{\TRUE}{&\mkern-8mu\PW{x}{0}\mkern-8mu&}  {x=0}\mkern50mu\\
      \dij{(x=0)}{&\mkern-8mu\PR{x}{r}\mkern-8mu&}  {r=0}\\
      \dij{(r=0)}{&\mkern-8mu\PW{y}{r+1}\mkern-8mu&}{y=1}
    \end{displaymatharray}
  }
  \normalsize
  \onslide<3->
  \vspace{1.5cm}
  
  [Hoare 1969] \only<9->{[Dijkstra 1975]}
  \onslide<10->
  \probX{Concurrency?}
  \onslide<11->
  \sol{Ordered Events!}  
\end{frame}

\begin{frame}
  \frametitle{Concurrent Computation\visible<2->{: Pomsets! (aka labeled partial orders)}}
  \huge
  \begin{displaymath}
    \PW{x}{0}
    \SEMI
    \PR{x}{r}
    \SEMI
    \PW{y}{r+1}
    \PAR
    \PW{x}{5}
  \end{displaymath}
  \vspace{-2ex}
  \onslide<2->
  \begin{displaymath}
    \scalebox{2}{    
      \begin{tikzinline}[node distance=2.5ex and 1.5em]
        \event{a}{\DW{x}{0}}{}
        \event{b}{\DR{x}{\alt<-4>{5}{0}}}{right=of a}
        \event{c}{\DW{y}{\alt<-4>{6}{1}}}{right=of b}
        \event{d}{\DW{x}{5}}{below=of a}      
        \visible<3>{\po{a}{b}}
        \visible<3>{\po{b}{c}}
        \visible<3-4>{\rf{d}{b}}
        \visible<4>{\wk{a}{d}}
        \visible<5->{\rfi{a}{b}}
        \visible<5->{\wk{b}{d}}
        % \event{d}{\DW{x}{5}}{right=3em of c}      
        % \visible<1-3>{\po{a}{b}}
        % \po{b}{c}
        % \visible<3-4>{\rf[out=-160,in=-20]{d}{b}}
        % \visible<4-4>{\wk[out=20,in=160]{a}{d}}
        % \visible<5->{\rfi{a}{b}}
        % \visible<5->{\wk[out=20,in=160]{b}{d}}
      \end{tikzinline}}
  \end{displaymath}
  \normalsize

  \normalsize
  \only<-3>{
    \vspace{2cm}
    [Pratt 1985] [Gischer 1988]
    % [Brookes 1993] 
    % ($r$ is a thread-local register)
  }
  \onslide<4->
  \vspace{1cm}
  Complete pomset: $\forall\DR{x}{v}$
  \begin{itemize}
  \item $\exists\DW{x}{v}<\DR{x}{v}$
  \item $\forall\DW{x}{u}$ either $\DW{x}{u}\leq\DW{x}{v}$ or $\DR{x}{v}<\DW{x}{u}$
  \end{itemize}
  \onslide<6->
  \probX{Hardware Reordering?}
\end{frame}
\begin{frame}
  \frametitle{Concurrency?}
  \huge
  \begin{displaymath}
    \PW{x}{0}
    \SEMI
    \PR{x}{r}
    \SEMI
    \PW{y}{r+1}
    \PAR
    \PW{x}{5}
  \end{displaymath}
  \vspace{-2ex}
  \onslide<2->
  \begin{displaymath}
    \scalebox{2}{    
      \begin{tikzinline}[node distance=2.5ex and 1.5em]
        \event{a}{\DW{x}{0}}{}
        \event{b}{\DR{x}{\alt<-4>{5}{0}}}{right=of a}
        \event{c}{\DW{y}{\alt<-4>{6}{1}}}{right=of b}
        \event{d}{\DW{x}{5}}{below=of a}      
        \visible<2-3>{\po{a}{b}}
        \po{b}{c}
        \visible<3-4>{\rf{d}{b}}
        \visible<4>{\wk{a}{d}}
        \visible<5->{\rfi{a}{b}}
        \visible<5->{\wk{b}{d}}
        % \event{d}{\DW{x}{5}}{right=3em of c}      
        % \visible<1-3>{\po{a}{b}}
        % \po{b}{c}
        % \visible<3-4>{\rf[out=-160,in=-20]{d}{b}}
        % \visible<4-4>{\wk[out=20,in=160]{a}{d}}
        % \visible<5->{\rfi{a}{b}}
        % \visible<5->{\wk[out=20,in=160]{b}{d}}
      \end{tikzinline}}
  \end{displaymath}
  \normalsize

  \normalsize
  \only<-3>{
    \vspace{2cm}
    [Gischer 1988] [Brookes 1993] 
    % ($r$ is a thread-local register)
  }
  \onslide<4->
  \vspace{1cm}
  Complete pomset: $\forall\DR{x}{v}$
  \begin{itemize}
  \item $\exists\DW{x}{v}<\DR{x}{v}$
  \item $\forall\DW{x}{u}$ either $\DW{x}{u}\leq\DW{x}{v}$ or $\DR{x}{v}<\DW{x}{u}$
  \end{itemize}
  \onslide<6->
  \probX{Hardware Reordering?}
\end{frame}


\begin{frame}
  \frametitle{\alt<-4>{Hardware Reordering?}{Deordering!}}
  \huge
  \begin{displaymath}
    \PW{x}{0}
    \SEMI
    \PW{x}{1}
    \SEMI
    \PR{y}{r}
    \PAR
    \PW{y}{0}
    \SEMI
    \PW{y}{1}
    \SEMI
    \PR{x}{s}
  \end{displaymath}
  \begin{displaymath}
    \scalebox{2}{    
      \begin{tikzinline}[node distance=2.5ex and 1.5em]
        \event{a}{\DW{x}{0}}{}
        \event{b}{\DW{x}{1}}{right=of a}
        \event{c}{\DR{y}{0}}{right=of b}
        \event{d}{\DW{y}{0}}{below=of a}
        \event{e}{\DW{y}{1}}{right=of d}
        \event{f}{\DR{x}{0}}{right=of e}
        \only<1-4>{
          \po{a}{b}
          \po{d}{e}
        }
        \only<1-4>{
          \po{b}{c}
          \po{e}{f}
        }
        \only<2->{
          \rf{a}{f}
          \wk{f}{b}
        }
        \only<3->{
          \rf{d}{c}
          \wk{c}{e}
        }
        % \rf[out=10,in=170]{a}{f}
        % \rf[out=-170,in=-10]{d}{c}
      \end{tikzinline}}
  \end{displaymath}
  \only<4>{\sol{Deorder!}}

  \onslide<6->
  \probX{SC-DRF?}
  \onslide<7->
  \prob[10]{Thin Air?}
\end{frame}

\begin{frame}
  \frametitle{\alt<-3>{SC-DRF?  Thin Air?}{Syntactic Dependencies!}}
  \huge
  \begin{displaymath}
    \IF{\PR{x}{}}\THEN\PW{y}{1}\FI
    \PAR
    \IF{\PR{y}{}}\THEN\PW{x}{1}\FI
  \end{displaymath}
  \begin{displaymath}
    \scalebox{2}{    
      \begin{tikzinline}[node distance=2.5ex and 1.5em]
        \event{a}{\DR{x}{1}}{}
        \event{b}{\DW{y}{1}}{right=of a}
        \event{c}{\DR{y}{1}}{below=of a}
        \event{d}{\DW{x}{1}}{right=of c}
        \only<4->{
          \po{a}{b}
          \po{c}{d}
        }
        \rf{d}{a}
        \rf{b}{c}
      \end{tikzinline}}
  \end{displaymath}
  \only<2-3>{\sol{Dependencies!}}
  \only<3>{\sol[-10]{Syntactic Dependencies!}}
  \onslide<5->
  \probX{Compiler Reordering?}
\end{frame}

\begin{frame}
  \frametitle{\alt<-2>{Compiler Reordering?}{Semantic Dependencies!}}
  %\frametitle{Putting it together}
  \huge
  \begin{displaymath}
    \PR{x}{r}\SEMI \PW{y}{(r*0)+1}
    \PAR
    \PR{y}{s}\SEMI \PW{x}{s}
  \end{displaymath}
  \begin{displaymath}
    \scalebox{2}{    
      \begin{tikzinline}[node distance=2.5ex and 1.5em]
        \event{a}{\DR{x}{1}}{}
        \event{b}{\DW{y}{1}}{right=of a}
        \event{c}{\DR{y}{1}}{below=of a}
        \event{d}{\DW{x}{1}}{right=of c}
        \only<-3>{
          \po{a}{b}
        }
        \po{c}{d}
        \rf{d}{a}
        \rf{b}{c}
      \end{tikzinline}}
  \end{displaymath}
  \only<2>{\sol{Semantic Dependencies!}}
  \only<5->{\probX{How?}}
\end{frame}

% \begin{frame}
%   \frametitle{Pomsets with Preconditions}
%   \huge
%   \begin{displaymath}
%     \visible<6->{\PR{x}{r}\SEMI}
%     \visible<4->{\PW{y}{(r*0)+1}}
%     \visible<4->{\PAR}
%     \visible<2->{\PR{y}{s}\SEMI}
%     \visible<1->{\PW{x}{s}}
%   \end{displaymath}
%   \begin{displaymath}
%     \scalebox{2}{    
%       \begin{tikzinline}[node distance=2.5ex and 2em]
%         \visible<6->{\event{a}{\DR{x}{1}\reg{r}}{}}
%         \visible<4->{\event{b}{(r*0)+1 = 1\bigmid\DW{y}{1}}{right=of a}}
%         \visible<2->{\event{c}{\DR{y}{1}\reg{s}}{below=of a}}
%         \visible<1->{\event{d}{\visible<3->{1{=}s\limplies} s{=}1\bigmid\DW{x}{1}}{right=of c}}
%         \visible<3->{\po{c}{d}}
%         \visible<6->{\rf{d}{a}}
%         \visible<4->{\rf{b}{c}}
%       \end{tikzinline}}
%   \end{displaymath}
%   \only<8->{\probX{Control Independencies?}}
% \end{frame}
\begin{frame}
  \frametitle{Pomsets with Preconditions \only<3->{and Predicate Transformers}}
  \huge
  \begin{displaymath}
    \PR{y}{s} \mkern80mu\; \visible<2->{\SEMI} \mkern80mu \PW{x}{s}
  \end{displaymath}
  \vspace{-2ex}
  \begin{displaymath}
    \scalebox{1.7}{    
      \begin{tikzinline}[node distance=100mu]
        \event{c}{\TRUE\bigmid\DR{y}{1}\reg{s}}{}
        \alt<-3>
        {\event{d}{s{=}1\bigmid\DW{x}{1}}{right=of c}}
        {\event{d}{1{=}s\limplies s{=}1\bigmid\DW{x}{1}}{right=71.5mu of c}}
        \visible<4->{\po{c}{d}}
      \end{tikzinline}}
  \end{displaymath}
  % \vspace{-1ex}
  \onslide<3->
  \Large
  \begin{displaymath}
    \mkern-20mu
    \begin{array}[t]{c|r}
      \bEvs&\aTr{\bEvs}{\bForm}\\ \hline
      \emptyset & \bForm\\
      \only<4>{\hightight}{\{\DR{y}{1}\reg{s}\}} & (1{=}s) \limplies \bForm                             
    \end{array}
    \mkern150mu
    \begin{array}[t]{c|r}
      \aEvs&\aTr{\aEvs}{\bForm}\\ \hline
      \emptyset & \bForm\\
      \{\DW{x}{1}\} & \bForm                             
    \end{array}
  \end{displaymath}
\end{frame}
\begin{frame}
  \frametitle{Pomsets with Preconditions and Predicate Transformers}
  \huge
  \begin{displaymath}
    \PR{x}{r} \mkern80mu\; \SEMI \mkern10mu \PW{y}{(r*0)+1}
  \end{displaymath}
  \vspace{-2ex}
  \begin{displaymath}
    \scalebox{1.7}{    
      \begin{tikzinline}[node distance=60mu]
        \event{a}{\TRUE\bigmid\DR{x}{1}\reg{s}}{}
        \event{b}{(r*0)+1 = 1\bigmid\DW{y}{1}}{right=of a}
      \end{tikzinline}}
  \end{displaymath}
  % \vspace{-1ex}
  \Large

  \begin{displaymath}
    \mkern-20mu
    \begin{array}[t]{c|r}
      \bEvs&\aTr{\bEvs}{\bForm}\\ \hline
      \hightight{\emptyset} & \bForm\\
      \{\DR{x}{1}\reg{r}\} & (1{=}r) \limplies \bForm                             
    \end{array}
    \mkern150mu
    \begin{array}[t]{c|r}
      \aEvs&\aTr{\aEvs}{\bForm}\\ \hline
      \emptyset & \bForm\\
      \{\DW{y}{1}\} & \bForm                             
    \end{array}
  \end{displaymath}
\end{frame}

\begin{frame}
  \frametitle{Pomsets with Preconditions and Predicate Transformers}
  \huge
  \begin{displaymath}
    \PR{x}{r}\SEMI \PW{y}{(r*0)+1}
    \PAR
    \PR{y}{s}\SEMI \PW{x}{s}
  \end{displaymath}
  \begin{displaymath}
    \scalebox{2}{    
      \begin{tikzinline}[node distance=2.5ex and 1.5em]
        \event{a}{\only<-2>{\TRUE\bigmid}\DR{x}{1}}{}
        \event{b}{\only<-2>{(r*0)+1 = 1\bigmid}\DW{y}{1}}{right=of a}
        \event{c}{\only<-2>{\TRUE\bigmid}\DR{y}{1}}{below=of a}
        \event{d}{\only<-2>{1{=}s\limplies s{=}1\bigmid}\DW{x}{1}}{right=of c}
        \po{c}{d}
        \only<2->{\rf{d}{a}}
        \only<2->{\rf{b}{c}}
      \end{tikzinline}}
  \end{displaymath}
  \only<4->{\sol{\begin{tabular}{c}Logic=Thread order\\Pomset=Global order\end{tabular}}}
  \only<5->{\probX{Control Independencies?}}
\end{frame}
% \begin{frame}
%   \frametitle{Pomsets with Predicate Transformers}
%   \huge
%   \begin{displaymath}
%     \PR{x}{r}\SEMI \PW{y}{(r*0)+1}
%   \end{displaymath}
%   \vspace{-2ex}
%   \onslide<2->
%   \begin{displaymath}
%     \scalebox{1.7}{    
%       \begin{tikzinline}[node distance=2ex and 1.2em]
%         \event{a}{\DR{x}{1}\reg{r}}{}
%         \event{b}{(r*0)+1 = 1\bigmid\DW{y}{1}}{right=of a}
%         % \event{c}{\DR{y}{1}\reg{s}}{below=of a}
%         % \event{d}{\visible<3->{1{=}s\limplies} s{=}1\bigmid\DW{x}{1}}{right=of c}
%         % \visible<3->{\po{c}{d}}
%       \end{tikzinline}}
%   \end{displaymath}
%   % \vspace{-1ex}
%   \onslide<2->
%   \Large
%   \begin{displaymath}
%     \begin{array}[t]{c|r}
%       \bEvs&\aTr{\bEvs}{\bForm}\\ \hline
%       \emptyset & \bForm\\
%       \{\DR{x}{1}\reg{r}\} & (1{=}r) \limplies \bForm
%       % \DR{x}{1}\reg{r}&\multicolumn{1}{c}{\aTr{}{\bForm}}\\ \hline
%       % \xmark & \bForm\\
%       % \cmark & (1{=}r) \limplies \bForm
%     \end{array}
%   \end{displaymath}
% \end{frame}

\begin{frame}
  \frametitle{\alt<-2>{Control Independencies?}{Merging!}}
  \huge
  \begin{displaymath}
    \PR{x}{r}
    \SEMI
    \IF{r\neq 0}\THEN\PW{y}{1}\FI
    \SEMI
    \IF{r = 0}\THEN\PW{y}{1}\FI
  \end{displaymath}
  \begin{displaymath}
    \scalebox{2}{    
      \begin{tikzinline}[node distance=2.5ex and 1.5em]
        \event{a}{\DR{x}{1}\reg{r}}{}
        \alt<3->{
          \event{b}{r\neq 0 \lor r=0\bigmid\DW{y}{1}}{right=of a}
        }{
          \event{b}{r\neq 0\bigmid\DW{y}{1}}{right=of a}
          \event{c}{r=0\bigmid\DW{y}{1}}{right=of b}
        }
      \end{tikzinline}}
  \end{displaymath}
  \only<2>{\sol{Merging!}}
  \only<4>{\probX{Associative?}}
\end{frame}

\begin{frame}
  \frametitle{Associative! (Left to Right)}
  \huge
  \begin{displaymath}
    \visible<2->{(}
    \PR{x}{r}
    \visible<2->{\mkern18mu\;\SEMI\mkern18mu}
    \PR{y}{s}
    \visible<2->{)}
    \visible<4->{\mkern20mu\;\SEMI\mkern20mu}
    \IF{s} \THEN \PW{z}{r{*}(s{-}1)} \FI
  \end{displaymath}
  \vspace{-4ex}
  \begin{displaymath}
    \scalebox{1.4}{    
      \begin{tikzinline}
        \event{rx0}{\DR{x}{1}\reg{r}}{}
        \event{ry0}{\DR{y}{1}\reg{s}}{right=50mu of rx0}
        \alt<4->
        {\event{wz0}{(1{=}s) \limplies (s{\neq}0) \land (r{*}(s{-}1)){=}0 \bigmid \DW{z}{0}}{right=30mu of ry0}
          \po{ry0}{wz0}}
        {\event{wz0}{(s{\neq}0) \land (r{*}(s{-}1)){=}0 \bigmid \DW{z}{0}}{right=60mu of ry0}}
      \end{tikzinline}}
  \end{displaymath}
  % \vspace{-1ex}
  \normalsize
  \smallskip

  \only<1-2>{
    \begin{math}
      \mkern-10mu
      \begin{array}[t]{c|r}
        \cEvs&\aTr{\cEvs}{\bForm}\\ \hline
        \only<2>{\hightight}{\emptyset} & \bForm\\
        \{\DR{x}{1}\reg{r}\} & (1{=}r) \limplies \bForm                             
      \end{array}
      \mkern20mu
      \begin{array}[t]{c|r}
        \bEvs&\aTr{\bEvs}{\bForm}\\ \hline
        \emptyset & \bForm\\
        \{\DR{y}{1}\reg{s}\} & (1{=}s) \limplies \bForm                             
      \end{array}
      \mkern50mu
      \begin{array}[t]{c|r}
        \aEvs&\aTr{\aEvs}{\bForm}\\ \hline
        \emptyset & \bForm\\
        \{\DW{y}{1}\} & \bForm                             
      \end{array}
    \end{math}
  }
  \only<3-4>{
    \begin{math}
      \mkern45mu
      \begin{array}[t]{c|r}
        \bEvs&\aTr{\bEvs}{\bForm}\\ \hline
        \emptyset
             & \bForm\\
        \{\DR{x}{1}\reg{r}\}
             & (1{=}r) \limplies \bForm\\
        \only<4>{\hightight}{\{\DR{y}{1}\reg{s}\}}
             & (1{=}s) \limplies \bForm\\
        \{\DR{x}{1}\reg{r},\DR{y}{1}\reg{s}\}
             & (1{=}r) \limplies(1{=}s) \limplies \bForm
      \end{array}
      \mkern70mu
      \begin{array}[t]{c|r}
        \aEvs&\aTr{\aEvs}{\bForm}\\ \hline
        \emptyset & \bForm\\
        \{\DW{y}{1}\} & \bForm                             
      \end{array}
    \end{math}
  }
  \only<5>{
    \begin{math}
      \mkern40mu
      \begin{array}[t]{c|r}
        \bEvs&\aTr{\bEvs}{\bForm}\\ \hline
        \emptyset,\{\DW{y}{1}\}
             & \bForm\\
        \{\DR{x}{1}\reg{r}\},\{\DR{x}{1}\reg{r},\DW{y}{1}\}
             & (1{=}r) \limplies \bForm\\
        \{\DR{y}{1}\reg{s}\},\{\DR{y}{1}\reg{s},\DW{y}{1}\}
             & (1{=}s) \limplies \bForm\\
        \{\DR{x}{1}\reg{r},\DR{y}{1}\reg{s}\},\{\DR{x}{1}\reg{r},\DR{y}{1}\reg{s},\DW{y}{1}\}
             & (1{=}r) \limplies(1{=}s) \limplies \bForm
      \end{array}
    \end{math}
  }
\end{frame}
\begin{frame}
  \frametitle{Associative! (Right to Left)}
  \huge
  \begin{displaymath}
    \mkern8mu
    \PR{x}{r}
    \visible<4->{\mkern15mu\;\SEMI\mkern15mu}
    \visible<2->{(}
    \PR{y}{s}
    \visible<2->{\mkern23mu\;\SEMI\mkern23mu}
    \IF{s} \THEN \PW{z}{r{*}(s{-}1)} \FI
    \visible<2->{)}
  \end{displaymath}
  \vspace{-4ex}
  \begin{displaymath}
    \scalebox{1.4}{    
      \begin{tikzinline}
        \event{rx0}{\DR{x}{1}\reg{r}}{}
        \event{ry0}{\DR{y}{1}\reg{s}}{right=50mu of rx0}
        \alt<2->
        {\event{wz0}{(1{=}s) \limplies (s{\neq}0) \land (r{*}(s{-}1)){=}0 \bigmid \DW{z}{0}}{right=30mu of ry0}
          \po{ry0}{wz0}}
        {\event{wz0}{(s{\neq}0) \land (r{*}(s{-}1)){=}0 \bigmid \DW{z}{0}}{right=60mu of ry0}}
      \end{tikzinline}}
  \end{displaymath}
  % \vspace{-1ex}
  \normalsize
  \smallskip

  \only<1-2>{
    \begin{math}
      \mkern-10mu
      \begin{array}[t]{c|r}
        \cEvs&\aTr{\cEvs}{\bForm}\\ \hline
        \emptyset & \bForm\\
        \{\DR{x}{1}\reg{r}\} & (1{=}r) \limplies \bForm                             
      \end{array}
      \mkern20mu
      \begin{array}[t]{c|r}
        \bEvs&\aTr{\bEvs}{\bForm}\\ \hline
        \emptyset & \bForm\\
        \only<2>{\hightight}{\{\DR{y}{1}\reg{s}\}} & (1{=}s) \limplies \bForm                             
      \end{array}
      \mkern50mu
      \begin{array}[t]{c|r}
        \aEvs&\aTr{\aEvs}{\bForm}\\ \hline
        \emptyset & \bForm\\
        \{\DW{y}{1}\} & \bForm                             
      \end{array}
    \end{math}
  }
  \only<3-4>{
    \begin{math}
      \mkern-10mu
      \begin{array}[t]{c|r}
        \cEvs&\aTr{\cEvs}{\bForm}\\ \hline
        \only<4>{\hightight}{\emptyset} & \bForm\\
        \{\DR{x}{1}\reg{r}\} & (1{=}r) \limplies \bForm                             
      \end{array}
      \mkern70mu
      \begin{array}[t]{c|r}
        \bEvs&\aTr{\bEvs}{\bForm}\\ \hline
        \emptyset,\{\DW{y}{1}\} & \bForm\\
        \{\DR{y}{1}\reg{s}\}, \{\DR{y}{1}\reg{s},\DW{y}{1}\} & (1{=}s) \limplies \bForm                             
      \end{array}
    \end{math}
  }
  \only<5>{
    \begin{math}
      \mkern40mu
      \begin{array}[t]{c|r}
        \bEvs&\aTr{\bEvs}{\bForm}\\ \hline
        \emptyset,\{\DW{y}{1}\}
             & \bForm\\
        \{\DR{x}{1}\reg{r}\},\{\DR{x}{1}\reg{r},\DW{y}{1}\}
             & (1{=}r) \limplies \bForm\\
        \{\DR{y}{1}\reg{s}\},\{\DR{y}{1}\reg{s},\DW{y}{1}\}
             & (1{=}s) \limplies \bForm\\
        \{\DR{x}{1}\reg{r},\DR{y}{1}\reg{s}\},\{\DR{x}{1}\reg{r},\DR{y}{1}\reg{s},\DW{y}{1}\}
             & (1{=}r) \limplies(1{=}s) \limplies \bForm
      \end{array}
    \end{math}
  }
\end{frame}

%\imagepage{sounds/fireworks.webp}

\begin{frame}
  \frametitle{Victory!}
  \begin{itemize}
  \item<+->No.  Where about certified proofs?
  \item<+->Ilya has proofs.
  \item<+->Victory!
  \item<+->\probX{Real Hardware?}  
  \end{itemize}
\end{frame}

\begin{frame}
  \frametitle{Multicopy Atomic Architectures}
  \Huge
  \begin{displaymath}
    \aCmd_1 \SEMI \aCmd_2
  \end{displaymath}
  \   \smallskip

  
  \large
  Let order include (in addition to dependency):
  \begin{itemize}
  \item Coherence  ($\DW{x}{} \xpo \DW{x}{}$)
  \item Release/Acquire Access/Fences ($\DW{x}{} \xpo \DW[\mREL]{y}{}$, $\DR[\mACQ]{x}{} \xpo \DW{y}{}$)
  \item SC Access/Fences ($\DW[\mSC]{x}{} \xpo \DR[\mSC]{y}{}$)
  \end{itemize}
  \onslide<2->
   Arm8 optimal for: Relaxed$^{\textstyle\cmark}$\!  Release$^{\textstyle\cmark}$\! Acquiring$^{\textstyle\xmark}$

  \onslide<3->
  Arm8 some work:\,  Relaxed$^{\textstyle\cmark}$\!  Release$^{\textstyle\cmark}$\! Acquiring$^{\textstyle\cmark}$

  \onslide<4->
  \probX{Non-MCA?}  
\end{frame}

\begin{frame}
  \frametitle{Dependencies for RC11}
  \large
  Replace No-TAR axiom
  %\onslide<2->
  \probPos[.5cm]{$\textsf{sb}\cup\textsf{rf}$ is acyclic}
  \onslide<2->
  \solPos[-1.5cm]{$\textsf{sdep}\cup\textsf{rf}$ is acyclic}
 
  \vfill
  \vfill
  \vfill
  \vfill
  \vfill
  \vfill
  \vfill
  \vfill
  \vfill
  
  \hbox{ } \hfill \textsf{sdep} = pomset order
\end{frame}
% \begin{frame}
%   \frametitle{Merging?}
%   \huge
%   \begin{displaymath}
%     \PR{x}{r}
%     \SEMI
%     \IF{r\neq 0}\THEN\PW{y}{1}\FI
%     \SEMI
%     \IF{r = 0}\THEN\PW{y}{1}\FI
%   \end{displaymath}
%   \begin{displaymath}
%     \scalebox{2}{    
%       \begin{tikzinline}[node distance=2.5ex and 1.5em]
%         \event{a}{\DR{x}{1}\reg{r}}{}
%         \event{b}{r\neq 0 \lor r=0\bigmid\DW{y}{1}}{right=of a}
%         \only<2->{
%           \phevent{c}{r\neq 0\bigmid\DW{y}{1}}{below right=2.5ex and -4em of b}
%           \phevent{d}{r=0\bigmid\DW{y}{1}}{below left=2.5ex and -4em of b}
%         }
%       \end{tikzinline}}
%   \end{displaymath}
% \end{frame}
\begin{frame}
  \frametitle{Merging and Program Order?}
  \huge
  \begin{displaymath}
    \IF{r}\THEN
    \PW{x}{1}
    \SEMI
    \PW{y}{1}
    \ELSE
    \PW{y}{1}
    \SEMI
    \PW{x}{1}
    \FI
  \end{displaymath}
  \only<1>{
    \vspace{-3ex}
    \begin{displaymath}
      \scalebox{1.6}{    
        \begin{tikzinline}[node distance=2.5ex and 4em]
          \event{d}{\DW{x}{1}}{}
          \event{e}{\DW{y}{1}}{right=of d}
          \pox[out=15,in=165]{d}{e}
          \pox[out=-165,in=-15]{e}{d}
        \end{tikzinline}}
    \end{displaymath}
  }
  \only<2->{
    \vspace{-2ex}
    \begin{displaymath}
      \scalebox{1.6}{    
        \begin{tikzinline}[node distance=1ex and 1.5em]
          % \phevent{d1}{\makebox[5mm]{$\temporal<3>{r{\neq}0}{\TRUE}{\FALSE}$}\bigmid\DW{x}{1}}{}
          % \phevent{e1}{\makebox[5mm]{$\temporal<3>{r{\neq}0}{\TRUE}{\FALSE}$}\bigmid\DW{y}{1}}{right=of d1}
          % \phevent{d2}{\makebox[5mm]{$\temporal<3>{r{=}0}{\FALSE}{\TRUE}$}\bigmid\DW{x}{1}}{right=of e1}
          % \phevent{e2}{\makebox[5mm]{$\temporal<3>{r{=}0}{\FALSE}{\TRUE}$}\bigmid\DW{y}{1}}{right=of d2}
          % \pheventModal{d1}{r{\neq}0\bigmid\DW{x}{1}}{}{\temporal<3>{gray!10}{green!10}{red!10}}
          % \pheventModal{e1}{r{\neq}0\bigmid\DW{y}{1}}{right=of d1}{\temporal<3>{gray!10}{green!10}{red!10}}
          % \pheventModal{d2}{r{=}0   \bigmid\DW{x}{1}}{right=of e1}{\temporal<3>{gray!10}{red!10}{green!10}}
          % \pheventModal{e2}{r{=}0   \bigmid\DW{y}{1}}{right=of d2}{\temporal<3>{gray!10}{red!10}{green!10}}
          \temporal<3>{
            \phevent{d1}{r{\neq}0\bigmid\DW{x}{1}}{}
            \phevent{e1}{r{\neq}0\bigmid\DW{y}{1}}{right=of d1}
            \phevent{d2}{r{=}0   \bigmid\DW{x}{1}}{right=of e1}
            \phevent{e2}{r{=}0   \bigmid\DW{y}{1}}{right=of d2}
          }{
            \pheventGreen{d1}{r{\neq}0\bigmid\DW{x}{1}}{}
            \pheventGreen{e1}{r{\neq}0\bigmid\DW{y}{1}}{right=of d1}
            \pheventRed{d2}{r{=}0   \bigmid\DW{x}{1}}{right=of e1}
            \pheventRed{e2}{r{=}0   \bigmid\DW{y}{1}}{right=of d2}
          }{
            \pheventRed{d1}{r{\neq}0\bigmid\DW{x}{1}}{}
            \pheventRed{e1}{r{\neq}0\bigmid\DW{y}{1}}{right=of d1}
            \pheventGreen{d2}{r{=}0   \bigmid\DW{x}{1}}{right=of e1}
            \pheventGreen{e2}{r{=}0   \bigmid\DW{y}{1}}{right=of d2}
          }
          \event{d}{\DW{x}{1}}{above=of e1}
          \event{e}{\DW{y}{1}}{above=of d2}
          \pox[out=-15,in=-165]{d1}{e1}
          \pox[out=-165,in=-15]{e2}{d2}
          \mrg{e1}{e}
          \mrg{e2}{e}
          \mrg{d1}{d}
          \mrg{d2}{d}
          % \event{e}{(v{=}r\lor w{=}r)\limplies (r{=}0\lor r{=}1)\bigmid\DW{x}{1}}{}
          % \event{r}{\DR{y}{v}}{left=of e}
          % \phevent{e1}{v{=}r\limplies r{=}0\bigmid\DW{x}{1}}{below left=1em and -7em of e}
          % \phevent{e2}{v{=}r\limplies r{=}1\bigmid\DW{x}{1}}{below right=1em and -7em of e}
          % \pox{r}[pos=0,below]{e1}
          % \pox[out=-15,in=-167]{e1}[below]{e2}
          % \mrg{e1}{e}
          % \mrg{e2}{e}
          % \event{i}{\DW{y}{w}}{left=of r}
          % \pox[out=-15,in=-167]{i}[below]{r}
          % \pox{r}{e}
        \end{tikzinline}}    
    \end{displaymath}
  }
  \onslide<4->
\end{frame}
\end{comment}


% \begin{frame}
%   %\probX{Power?}  
%   \probX{Control Independencies?}
%   %\sol{\begin{tabular}{c}Logic=Local order\\Pomset=Other order\end{tabular}}
%   %\sol{\begin{tabular}{c}Concurrency is easy!\\Sequentiality is hard!\end{tabular}}
%   % \only<1>{\sol{Concurrency is easy!}}
%   % \only<2>{\prob{Sequentiality is hard!}}
%   %\probX{Non-MCA?}  
% \end{frame}



\begin{frame}
  %\frametitle{Pomsets with Predicate Transformers}
  \frametitle{Sequential composition for an optimized concurrent language}
  \Large

  \begin{itemize}
  \item<+-> Dependency = Pomset order = Order seen by other threads
  \item<+-> Very compositional for MCA
    \normalsize
    \begin{itemize}
    \item Solution for Out-Of-Thin-Air models (C, JavaScript)
    \end{itemize}
  \item<+-> Denotational semantics
    \normalsize
    \begin{itemize}
    \item \textcolor{blue}{Tractable} notion of refinement for peephole optimization
    \item \textcolor{blue}{Understandable} by human beings
    \end{itemize}
  \item<+-> Main complexity
    \normalsize
    \begin{itemize}
    \item Calculating dependencies
    \item Merging and unmerging
    \item In conditionals, single execution uses both sides
    \end{itemize}    
  \item<+-> More in the paper
    \normalsize
    \begin{itemize}
    \item RMWs, address calculation, Indirect dependencies, If-introduction,
      etc
    %\item Associativity: Skolemization, Inconsistent preconditions
    \end{itemize}    

  \end{itemize}
  % Details\ldots
  % \normalsize
  % \begin{itemize}
  % \item Indirect dependencies, Read-Read independency
  % \item Associativity: Skolemization, Inconsistent preconditions
  % \item Compiler optimizations: If-introduction, redundant read elimination, etc
  % \item Release/acquire, SC access, fences, RMWs, address calculation
  % \end{itemize}
  \only<+->{\solPosR[1cm]{Concurrency is easy!}}
  \only<+->{\probPosR[-1cm]{Sequentiality is hard!}}
\end{frame}
\end{document}



\begin{frame}
  \frametitle{Temporal Reasoning!}
  \Large
  \begin{columns}
    \begin{column}{.48\textwidth}      
      \begin{displaymath}
        \IF{\PR{x}{}}\THEN\PW{y}{1}\FI
        \PAR
        \IF{\PR{y}{}}\THEN\PW{x}{1}\FI
      \end{displaymath}
      \begin{displaymath}
        \scalebox{1.6}{    
          \begin{tikzinline}[node distance=2.5ex and 1.5em]
            \event{a}{\DR{x}{1}}{}
            \event{b}{\DW{y}{1}}{right=of a}
            \event{c}{\DR{y}{1}}{below=of a}
            \event{d}{\DW{x}{1}}{right=of c}
            \po{a}{b}
            \po{c}{d}
            \rf{d}{a}
            \rf{b}{c}
          \end{tikzinline}}
      \end{displaymath}
    \end{column}
    \pause
    \begin{column}{.48\textwidth}      
      \begin{displaymath}
        [\DW{x}{1}\Rightarrow\once\DR{y}{1}]
        \land
        [\DW{y}{1}\Rightarrow\once\DR{x}{1}]
      \end{displaymath}
      \pause
      \begin{displaymath}
        [\DR{x}{1}\Rightarrow\once\DW{x}{1}]
        \land
        [\DR{y}{1}\Rightarrow\once\DW{y}{1}]
      \end{displaymath}
      \pause
      \begin{displaymath}
        \DW{x}{1}\Rightarrow\once\DW{x}{1}
      \end{displaymath}
      \pause
      \begin{displaymath}
        \llap{Coinduction: }
        \aPS \models (\afo \Rightarrow\once{\afo}) \Rightarrow\lnot \afo
      \end{displaymath}
    \end{column}
  \end{columns}
\end{frame}

\begin{frame}
  \frametitle{Bait and switch}
  \Large
  \begin{gather*}
    \PW{y}{\PR{x}{}}
    \PAR
    \hightight{\PR{y}{r}} \SEMI \IF{b}\THEN  \hightight{\PW{x}{r}} \SEMI \PW{z}{r} \ELSE \hightight{\PW{x}{1}} \FI
    \PAR
    \PW{b}{1}
    \\
    \visible<2->{
      \scalebox{1.4}{    
        \hbox{\begin{tikzinline}[node distance=2.5ex and 1.5em]
            \event{rx}{\DR{x}{1}}{}
            \event{wy}{\DW{y}{1}}{right=of rx}
            \po{rx}{wy}
            \event{ry}{\DR{y}{1}}{right=3em of wy} 
            \event{wx}{\DW{x}{1}}{right=of ry}
            \event{wz}{\DW{z}{1}}{right=of wx}
            \event{rb}{\DR{b}{1}}{right=of wz}
            \event{wb1}{\DW{b}{1}}{right=3em of rb}
            \alt<-2>{\po[dotted]{ry}{wx}}{\po{ry}{wx}}
            \rf{wb1}{rb}
            \rf{wy}{ry}
            \rf[out=-160,in=-20]{wx}{rx}
            \po{rb}{wz}
            \po[out=30,in=150]{ry}{wz}
            \begin{pgfonlayer}{background}
              % \node at ([shift={(45:.7)}]ry) [fill=yellow,rotate=45,align=left] {$\phantom{\smash{xxxxxxxxx}}$};
              \node at ([shift={(0:.7)}]ry) [fill=yellow,align=left] {$\phantom{\smash{xxxxx}}$};
            \end{pgfonlayer}
          \end{tikzinline}}}}
    \\
    \visible<3->{
      [\once\DW{y}{1} \Rightarrow \once\DR{x}{1}]
      \land
      [\notonce\DW{z}{1} \Rightarrow (\once\DR{y}{1} \land \always(\DW{x}{1} \Rightarrow \once\DR{y}{1}))]}
  \end{gather*}  

  \onslide<4->
  \vspace{1cm}
  \normalsize
  Allow: \;Breaks safety [Lochbihler 2013] \\
  \onslide<5->
  Forbid: Breaks optimization [Cho et al PLDI 2021]\\
  \hspace{1em} (Read introduction + if introduction)
\end{frame}

% \imagepage{images/what-if.jpg}{}

% \begin{frame}
%   \frametitle{Speculative models}
%   \Large
%   \begin{gather*}
%     \PW{y}{\PR{x}{}}
%     \PAR
%     \PR{y}{r} \SEMI \IF{b}\THEN  \PW{x}{r} \SEMI \PW{z}{r} \ELSE \PW{x}{1} \FI
%     \PAR
%     \PW{b}{1}
%   \end{gather*}  
%   \normalsize
%   \pause

%   Many speculative models:
%   \begin{itemize}
%   \item Java Memory Model [Pugh \emph{et al.} 2005]
%   \item Speculative operational semantics [Jagadeesan, Pitcher and Riely 2010]
%   \item Event structures with AE-justification [Jeffrey and Riely 2016]
%   \item Promising semantics [Kang \emph{et al.} 2017]
%   \item \dots
%   \end{itemize}
%   \pause
%   All fall for bait and switch.

% \end{frame}

\begin{frame}
  \frametitle{Contribution}
  A model of imperative programming that supports:
  \begin{itemize}
  \item shared-memory concurrency %(with data races)
    % \item<+-> data races
  \item hardware reorderings
  \item compiler reorderings
  \item temporal invariant reasoning
  \item other things:
    \begin{itemize}
    \item java causality test cases
    \item if-introduction (aka, case analysis)
    \item redundant read elimination
    \item implementation on x86-64 and Arm8
    \item address calculation
    \item release/acquire and SC access
    \item fences and RMWs
    \end{itemize}
  \end{itemize}
\end{frame}



% \begin{frame}
%   \frametitle{Other things!}

%   See the paper for\dots
%   \begin{itemize}
%   \item java causality test cases
%   \item if-introduction (aka, case analysis)
%   \item redundant read elimination
%   \item implementation on x86-64 and Arm8
%   \item address calculation
%   \item release/acquire and SC access
%   \item fences and RMWs
%   \end{itemize}

% \end{frame}

% \begin{frame}
%   \frametitle{Conclusions}

%   A memory model which supports
%   \begin{itemize}
%   \item shared-memory concurrency (with data races)
%   \item hardware reorderings
%   \item compiler reorderings
%   \item temporal invariant reasoning
%   \item other things
%   \end{itemize}

% \end{frame}
\end{document}

% Local Variables:
% mode: latex
% TeX-master: t
% End:

Hi, I'm James.  In this paper, we explore the interaction of sequential
composition and mutable state in the context of optimized
concurrent languages, such as C, Java and Javascript.

Consider a simple language for straight-line code.
We have commands that do nothing, assign a register, and do things sequentially.
In this context, we can all agree on the meaning of the semicolon: S1 executes before, and S2 executes after.
The semicolon provides a clear temporal barrier between the actions of S1 and S2.

Things change when we add concurrency, with commands that write to shared memory, read from shared memory, and do things concurrently.
When threads can access the same memory at the same time, the meaning of the semicolon has to change.
There is no longer a clear distinction between before and after.
As Marino and coauthors note, this amorphous, shifting semicolon creates many difficulties for programmers.
They argue that sequentiality is an important abstraction, which should be protected.
And I agree.
But, unfortunately, saving the sacred semicolon comes at a cost:
A cost in computing time, with knock-on costs in power consumption and the resulting environmental effects. 
It's a cost that implementers of highly-optimized concurrent code may be unwilling pay.

In this paper, we explore the opposite extreme:  Is there a tractable model of sequential composition that captures all of the messiness of relaxed memory models?




It's very easy to get it wrong.
Annoyingly, this is a rather mature area, so you have to do a lot of work to
get a paper published.  Check it out!  And thanks for listening!





I'm James.  I'm going to briefly introduce our attempt to provide a
denotational account of sequential composition.  I know what you're thinking:
wasn't that done in the 70's?  Well yes, but the context has changed: In this
paper we consider sequential composition in an optimized language with
shared-memory concurrency, like C, Java or JavaScript.  
One way to understand meaning is to look at which things are equated and which aren't.

Let's start by looking at two programs that are clearly different.
The program on the left loads x into a register, then stores the value of register in y.
The program on the right performs these operations in the opposite order, storing the prior value of the register before performing the load.
These programs are distinguished by a sequential context that subsequently loads y.

If instead the store takes a constant, then the order of the operations is not observable sequentially, and the programs are equated.
