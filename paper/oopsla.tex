\section{Differences with ``Pomsets with Preconditions''}
\label{sec:diff}

\myparagraph{Substitution}

\jjr{} uses substitution rather than Skolemizing.  Indeed our use of
Skolemization is motivated by disjunction closure for predicate transformers,
which do not appear in \jjr{}.  In \reffig{fig:sem}, 
we gave the semantics of read for nonempty pomsets as:
\begin{enumerate}
\item[{\labeltext[\textsc{r}4a]{(\textsc{r}4a)}{read-tau-dep-oopsla}}]
  if $(\aEvs\cap\bEvs)\neq\emptyset$ then
  \begin{math}
    \aTr{\bEvs}{\bForm} \rimplies
    \aVal{=}\aReg
    \limplies \bForm
  \end{math},    
\item[{\labeltext[\textsc{r}4b]{(\textsc{r}4b)}{read-tau-ind-oopsla}}]
  if $(\aEvs\cap\bEvs)=\emptyset$ then
  \begin{math}
    \aTr{\bEvs}{\bForm} \rimplies
    \PBR{\aVal{=}\aReg \lor \aLoc{=}\aReg} \limplies
    \bForm.
  \end{math}
\end{enumerate}
In \jjr{}, the definition is roughly as follows:
% (adding the case for $\ref{L6}$, which was missing):
\begin{enumerate}
\item[{\labeltext[\textsc{r}4a$'$]{(\textsc{r}4a$'$)}{read-tau-dep-oopsla-sub}}]
  if $(\aEvs\cap\bEvs)\neq\emptyset$ then
  \begin{math}
    \aTr{\bEvs}{\bForm} \rimplies
    \bForm[\aVal/\aReg][\aVal/\aLoc]
    % \aVal{=}\aReg
    % \limplies \bForm[\aReg/\aLoc]
  \end{math},    
\item[{\labeltext[\textsc{r}4b$'$]{(\textsc{r}4b$'$)}{read-tau-ind-oopsla-sub}}]
  if $(\aEvs\cap\bEvs)=\emptyset$ then
  \begin{math}
    \aTr{\bEvs}{\bForm} \rimplies
    \bForm[\aVal/\aReg][\aVal/\aLoc]\land\bForm[\aLoc/\aReg]
  \end{math}
\end{enumerate}
The use of conjunction in \ref{read-tau-ind-oopsla-sub} causes disjunction closure to fail
because the predicate transformer
% $\aTr{}{\bForm}=\bForm[\aVal/\aReg][\aVal/\aLoc]\land\bForm[\aLoc/\aReg]$ does not distribute through
% disjunction:
% \begin{math}
%   \aTr{}{\bForm_1\lor \bForm_2}=
%   (\bForm_1\lor \bForm_2)[\aVal/\aReg][\aVal/\aLoc]\land(\bForm_1\lor \bForm_2)[\aLoc/\aReg]
%   \neq
%   (\bForm_1[\aVal/\aReg][\aVal/\aLoc]\land\bForm_1[\aLoc/\aReg]) \lor
%   (\bForm_2[\aVal/\aReg][\aVal/\aLoc]\land\bForm_2[\aLoc/\aReg])
%   = \aTr{}{\bForm_1} \lor \aTr{}{\bForm_2}
% \end{math}
$\aTr{}{\bForm}=\bForm'\land\bForm''$ does not distribute through
disjunction, even assuming that the prime operations do:\footnote{%
  \begin{math}
    (\bForm_1\lor \bForm_2)'=(\bForm_1'\lor \bForm_2')
  \end{math}
  and
  \begin{math}
    (\bForm_1\lor \bForm_2)''=(\bForm_1''\lor \bForm_2'')
  \end{math}.
}
\begin{math}
  \aTr{}{\bForm_1\lor \bForm_2}=
  \href{https://www.wolframalpha.com/input/?i=\%28a+or+b\%29+and+\%28c+or+d\%29}{(\bForm_1'\lor \bForm_2')\land(\bForm_1''\lor \bForm_2'')}
  \neq
  \href{https://www.wolframalpha.com/input/?i=\%28a+and+c\%29+or+\%28b+and+d\%29}{(\bForm_1'\land\bForm_1'') \lor (\bForm_2'\land\bForm_2'')}
  = \aTr{}{\bForm_1} \lor \aTr{}{\bForm_2}
\end{math}.
% \begin{math}
%   (\bForm_{1}^{1}\lor \bForm_{1}^{2}) \land (\bForm_{2}^{1}\lor \bForm_{2}^{2})
%   \neq
%   (\bForm_{1}^{1}\land\bForm_{2}^{1}) \lor (\bForm_{1}^{2}\land\bForm_{1}^{2}).
% \end{math}
See also \textsection\ref{sec:ex:assoc}.

The substitutions collapse $\aLoc$ and $\aReg$, allowing local invariant
reasoning (\xLIR{}), as required by causality test case 1, discussed at the end of
\textsection\ref{sec:ex:control}.  Without Skolemizing it is necessary to
substitute $[\aLoc/\aReg]$, since the reverse substitution $[\aReg/\aLoc]$ is
useless when $\aReg$ is bound---compare with
\textsection\ref{sec:substitutions}.  As discussed below (\ref{p:downset}),
including this substitution affects the interaction of \xLIR{} and downset
closure.

Removing the substitution of $[x/r]$ in the independent case has a technical
advantage: we no longer require \emph{extended} expressions (which include
memory references), since substitutions no longer introduce memory
references.

\begin{scope}
  The substitution $[x/r]$ does not work with Skolemization, even for the
  dependent case, since we lose the unique marker for each read.  In effect,
  this forces all reads of a location to see the same values.
  % To be concrete, the candidate
  % definition would modify \ref{L4} to be:
  % \begin{enumerate}
  % \item[\ref{L4})]
  %   $\aTr{\bEvs}{\bForm} \rimplies \aVal{=}\aLoc\limplies\bForm[\aLoc/\aReg]$.
  %   % \item[\ref{L5})]
  %   %   $\aTr{\cEvs}{\bForm} \rimplies (\aVal{=}\aLoc\lor\TRUE)\limplies\bForm[\aLoc/\aReg]$. %, when $\aEvs\neq\emptyset$,
  %   % \item[\ref{L6})] 
  %   %   $\aTr{\dEvs}{\bForm}\; \rimplies \bForm$, when $\aEvs=\emptyset$.
  % \end{enumerate}
  Using this definition, consider the following:
  \begin{gather*}
    \PR{x}{r}\SEMI
    \PR{x}{s}\SEMI
    \IF{r{<}s}\THEN \PW{y}{1}\FI 
    \\[-1ex]
    \hbox{\begin{tikzinline}[node distance=0.5em and 1.5em]
        \event{a1}{\DR{x}{1}}{}
        \event{a2}{\DR{x}{2}}{right=of a1}
        \event{a3}{1{=}x\limplies 2{=}x\limplies x{<} x\mid\DW{y}{1}}{right=of a2}
        \po[out=20,in=160]{a1}{a3}
        \po{a2}{a3}
      \end{tikzinline}}
  \end{gather*}
  Although the execution seems reasonable, the precondition on the write is
  not a tautology.
\end{scope}


% There, item \ref{loadpre-kappa2}  of $\sLOADPRE{}{}{}$ is written 
% \begin{enumerate}
% \item[] %[\ref{loadpre-kappa2})]
%   if $\aEv\in\aEvs_2\setminus\aEvs_1$ then either \\
%   $\labelingForm(\aEv) \rimplies \labelingForm_2(\aEv)[\aLoc/\aReg][\aVal/\aLoc]$ and $(\exists\bEv\in\aEvs_1)\bEv{<}\aEv$, or \\
%   $\labelingForm(\aEv) \rimplies \labelingForm_2(\aEv)[\aLoc/\aReg][\aVal/\aLoc] \land \labelingForm_2(\aEv)[\aLoc/\aReg]$.
% \end{enumerate}


% [Skolemization ensures disjunction closure, which is necessary
% for associativity. Show example.]

\myparagraph[p:downset]{Downset closure}

\jjr{} enforces downset closure in the prefixing rule.  Even without this,
downset closure would be different for the two semantics, due to the use of
substitution in \jjr{}.  Consider the final pomset in the last example of
\textsection\ref{sec:downset} under the semantics of this paper, which elides
the middle read event:
\begin{align*}
  \begin{gathered}[t]
    \PW{x}{0} 
    \SEMI\PR{x}{r} 
    \SEMI\IF{r{\geq}0}\THEN \PW{y}{1} \FI
    \\
    \hbox{\begin{tikzinline}[node distance=.5em and 1.5em]
        \event{a0}{\DW{x}{0}}{}
        % \event{a1}{\DR{x}{1}}{right=of a0}
        \event{a2}{r{\geq}0\mid\DW{y}{1}}{right=3em of a1}      
        % \wk{a0}{a1}
      \end{tikzinline}}    
  \end{gathered}
\end{align*}
In \jjr{}, the substitution $[x/r]$ is performed by the middle read
regardless of whether it is included in the pomset, with the subsequent
substitution of $[0/x]$ by the preceding write, we have $[x/r][0/x]$, which
is $[0/r][0/x]$, resulting in:
\begin{align*}
  \begin{gathered}[t]
    \hbox{\begin{tikzinline}[node distance=.5em and 1.5em]
        \event{a0}{\DW{x}{0}}{}
        % \event{a1}{\DR{x}{1}}{right=of a0}
        \event{a2}{0{\geq}0\mid\DW{y}{1}}{right=3em of a1}      
        % \wk{a0}{a1}
      \end{tikzinline}}    
  \end{gathered}
\end{align*}


\myparagraph{Consistency}
\jjr{} imposes \emph{consistency}, which requires that for every pomset
$\aPS$, $\bigwedge_{\aEv}\labelingForm(\aEv)$ is satisfiable.  
\begin{scope}
  Associativity requires that we allow pomsets with inconsistent
  preconditions.  Consider a variant of the example from \textsection\ref{sec:semca}.
  \begin{scope}
    \footnotesize
    \begin{align*}
      \begin{gathered}
        \IF{\aExp}\THEN\PW{x}{1}\FI
        \\
        \hbox{\begin{tikzinline}[node distance=1em]
            \event{a}{\aExp\mid\DW{x}{1}}{}
          \end{tikzinline}}
      \end{gathered}
      &&
      \begin{gathered}
        \IF{\BANG\aExp}\THEN\PW{x}{1}\FI
        \\
        \hbox{\begin{tikzinline}[node distance=1em]
            \event{a}{\lnot\aExp\mid\DW{x}{1}}{}
          \end{tikzinline}}
      \end{gathered}
      &&
      \begin{gathered}
        \IF{\aExp}\THEN\PW{y}{1}\FI
        \\
        \hbox{\begin{tikzinline}[node distance=1em]
            \event{a}{\aExp\mid\DW{y}{1}}{}
          \end{tikzinline}}
      \end{gathered}
      &&
      \begin{gathered}
        \IF{\BANG\aExp}\THEN\PW{y}{1}\FI
        \\
        \hbox{\begin{tikzinline}[node distance=1em]
            \event{a}{\lnot\aExp\mid\DW{y}{1}}{}
          \end{tikzinline}}
      \end{gathered}
    \end{align*}
  \end{scope}
  Associating left and right, we have:
  \begin{scope}
    \footnotesize
    \begin{align*}
      \begin{gathered}
        \IF{\aExp}\THEN\PW{x}{1}\FI
        \SEMI
        \IF{\BANG\aExp}\THEN\PW{x}{1}\FI
        \\
        \hbox{\begin{tikzinline}[node distance=1em]
            \event{a}{\DW{x}{1}}{}
          \end{tikzinline}}
      \end{gathered}
      &&
      \begin{gathered}
        \IF{\aExp}\THEN\PW{y}{1}\FI
        \SEMI
        \IF{\BANG\aExp}\THEN\PW{y}{1}\FI
        \\
        \hbox{\begin{tikzinline}[node distance=1em]
            \event{a}{\DW{y}{1}}{}
          \end{tikzinline}}
      \end{gathered}
    \end{align*}
  \end{scope}  
  Associating into the middle, instead, we require:
  \begin{scope}
    \footnotesize
    \begin{align*}
      \begin{gathered}
        \IF{\aExp}\THEN\PW{x}{1}\FI
        \\
        \hbox{\begin{tikzinline}[node distance=1em]
            \event{a}{\aExp\mid\DW{x}{1}}{}
          \end{tikzinline}}
      \end{gathered}
      &&
      \begin{gathered}
        \IF{\BANG\aExp}\THEN\PW{x}{1}\FI
        \SEMI
        \IF{\aExp}\THEN\PW{y}{1}\FI
        \\
        \hbox{\begin{tikzinline}[node distance=1em]
            \event{a}{\lnot\aExp\mid\DW{x}{1}}{}
            \event{b}{\aExp\mid\DW{y}{1}}{right=of a}
          \end{tikzinline}}
      \end{gathered}
      &&
      \begin{gathered}
        \IF{\BANG\aExp}\THEN\PW{y}{1}\FI
        \\
        \hbox{\begin{tikzinline}[node distance=1em]
            \event{a}{\lnot\aExp\mid\DW{y}{1}}{}
          \end{tikzinline}}
      \end{gathered}
    \end{align*}
  \end{scope}
  Joining left and right, we have:
  \begin{scope}
    \footnotesize
    \begin{align*}
      \begin{gathered}
        \IF{\aExp}\THEN\PW{x}{1}\FI
        \SEMI
        \IF{\BANG\aExp}\THEN\PW{x}{1}\FI
        \SEMI
        \IF{\aExp}\THEN\PW{y}{1}\FI
        \SEMI
        \IF{\BANG\aExp}\THEN\PW{y}{1}\FI
        \\
        \hbox{\begin{tikzinline}[node distance=1em]
            \event{a}{\DW{x}{1}}{}
            \event{b}{\DW{y}{1}}{right=of a}
          \end{tikzinline}}
      \end{gathered}
    \end{align*}
  \end{scope}  
\end{scope}

\myparagraph{Causal Strengthening}
% \labeltext[]{Causal Strengthening}{xCausal}
\jjr{} imposes \emph{causal strengthening}, which requires for every pomset
$\aPS$, if $\bEv\le\aEv$ then $\labelingForm(\aEv) \rimplies \labelingForm(\bEv)$. 
\begin{scope}
  Associativity requires that we allow pomsets without causal strengthening.
  Consider the following.
  \begin{align*}
    \begin{gathered}
      \IF{\aExp}\THEN\PR{x}{r}\FI
      \\
      \hbox{\begin{tikzinline}[node distance=1em]
          \event{a}{\aExp\mid\DR{x}{1}}{}
        \end{tikzinline}}
    \end{gathered}
    &&
    \begin{gathered}
      \PW{y}{r}
      \\
      \hbox{\begin{tikzinline}[node distance=1em]
          \event{a}{r{=}1\mid\DW{y}{1}}{}
        \end{tikzinline}}
    \end{gathered}
    &&
    \begin{gathered}
      \IF{\BANG\aExp}\THEN\PR{x}{s}\FI
      \\
      \hbox{\begin{tikzinline}[node distance=1em]
          \event{a}{\lnot\aExp\mid\DR{x}{1}}{}
        \end{tikzinline}}
    \end{gathered}
  \end{align*}
  Associating left, with causal strengthening:
  \begin{align*}
    \begin{gathered}
      \IF{\aExp}\THEN\PR{x}{r}\FI
      \SEMI
      \PW{y}{r}
      \\
      \hbox{\begin{tikzinline}[node distance=1em]
          \event{a}{\aExp\mid\DR{x}{1}}{}
          \event{b}{\aExp\mid\DW{y}{1}}{right=of a}
          \po{a}{b}
        \end{tikzinline}}
    \end{gathered}
    &&
    \begin{gathered}
      \IF{\BANG\aExp}\THEN\PR{x}{s}\FI
      \\
      \hbox{\begin{tikzinline}[node distance=1em]
          \event{a}{\lnot\aExp\mid\DR{x}{1}}{}
        \end{tikzinline}}
    \end{gathered}
  \end{align*}
  Finally, merging:
  \begin{align*}
    \begin{gathered}
      \IF{\aExp}\THEN\PR{x}{r}\FI
      \SEMI
      \PW{y}{r}
      \SEMI
      \IF{\BANG\aExp}\THEN\PR{x}{s}\FI
      \\
      \hbox{\begin{tikzinline}[node distance=1em]
          \event{a}{\DR{x}{1}}{}
          \event{b}{\aExp\mid\DW{y}{1}}{right=of a}
          \po{a}{b}
        \end{tikzinline}}
    \end{gathered}
  \end{align*}
  Instead, associating right:
  \begin{align*}
    \begin{gathered}
      \IF{\aExp}\THEN\PR{x}{r}\FI
      \\
      \hbox{\begin{tikzinline}[node distance=1em]
          \event{a}{\aExp\mid\DR{x}{1}}{}
        \end{tikzinline}}
    \end{gathered}
    &&
    \begin{gathered}
      \PW{y}{r}
      \SEMI
      \IF{\BANG\aExp}\THEN\PR{x}{s}\FI
      \\
      \hbox{\begin{tikzinline}[node distance=1em]
          \event{a}{\lnot\aExp\mid\DR{x}{1}}{}
          \event{b}{r{=}1\mid\DW{y}{1}}{left=of a}
        \end{tikzinline}}
    \end{gathered}
  \end{align*}
  Merging:
  \begin{align*}
    \begin{gathered}
      \IF{\aExp}\THEN\PR{x}{r}\FI
      \SEMI
      \PW{y}{r}
      \SEMI
      \IF{\BANG\aExp}\THEN\PR{x}{s}\FI
      \\
      \hbox{\begin{tikzinline}[node distance=1em]
          \event{a}{\DR{x}{1}}{}
          \event{b}{\DW{y}{1}}{right=of a}
          \po{a}{b}
        \end{tikzinline}}
    \end{gathered}
  \end{align*}
  With causal strengthening, the precondition of $\DW{y}{1}$ depends upon how
  we associate.  This is not an issue in \jjr{}, which always associates to
  the right.
\end{scope}

% \myparagraph{Causal Strengthening and Address Dependencies}
% \labeltext[]{Causal Strengthening and Address Dependencies}{xADDRxRRD}

\begin{scope}  
  One use of causal strengthening is to ensure that address dependencies do
  not introduce thin air reads.  Associating to the right, the intermediate
  state of the example in \textsection\ref{sec:addr} is:
  \begin{align*}
    \begin{gathered}[t]
      \PR{\REF{r}}{s}
      \SEMI
      \PW{x}{s}
      \\
      \hbox{\begin{tikzinline}[node distance=.5em and 1.5em]
          \event{a2}{r\EQ2\mid\DR{\REF{2}}{1}}{}
          \event{a3}{(r\EQ2\limplies 1\EQ s) \limplies s\EQ1\mid\DW{x}{1}}{right=of a2}
          \po{a2}{a3}
        \end{tikzinline}}
    \end{gathered}
  \end{align*}
  In \jjr{}, we have, instead:
  \begin{gather*}
    % \begin{gathered}[t]
    %   \PW{x}{s}
    %   \\
    %   \hbox{\begin{tikzinline}[node distance=.5em and 1.5em]
    %     \event{b}{s\EQ1\mid\DW{x}{1}}{}
    %   \end{tikzinline}}
    % \end{gathered}
    % \\
    \begin{gathered}
      % \PR{y}{r}\SEMI
      \PR{\REF{r}}{s}\SEMI \PW{x}{s}
      \\
      \hbox{\begin{tikzinline}[node distance=.5em and 1.5em]
          % \event{a1}{\DR{y}{2}}{}
          \event{a2}{r\EQ2\mid\DR{\REF{2}}{1}}{}%right=of a1}
          \event{a3}{r\EQ2\land\REF{2}\EQ1\mid\DW{x}{1}}{right=of a2}
          \po{a2}{a3}
        \end{tikzinline}}
    \end{gathered}
  \end{gather*}
  Without causal strengthening, the precondition of $\DWP{x}{1}$ would be
  simply $\REF{2}\EQ1$.  The treatment in this paper, using implication
  rather than conjunction, is more precise.
\end{scope}

\myparagraph{Internal Acquiring Reads}

The proof of compilation to Arm in \jjr{} assumes that all internal reads can
be eliminated.
% Shortly after publication, \citet{anton}
% noticed a shortcoming of the implementation on \armeight{} in
% \jjr{\textsection 7}.  The proof given there assumes that all internal reads
% can be dropped.
However, this is not the case for acquiring reds.  For example, \jjr{}
disallows the following execution, where the final values of $x$ is $2$ and
the final value of $y$ is $2$.  This execution is allowed by \armeight{} and
\tso{}.
\begin{gather*}
  \PW{x}{2}\SEMI 
  \PR[\mACQ]{x}{r}\SEMI
  \PR{y}{s} \PAR
  \PW{y}{2}\SEMI
  \PW[\mREL]{x}{1}
  \\
  \hbox{\begin{tikzinline}[node distance=1.5em]
      \event{a}{\DW{x}{2}}{}
      \raevent{b}{\DR[\mACQ]{x}{2}}{right=of a}
      \event{c}{\DR{y}{0}}{right=of b}
      \event{d}{\DW{y}{2}}{right=2.5em of c}
      \raevent{e}{\DW[\mREL]{x}{1}}{right=of d}
      \rf{a}{b}
      \sync{b}{c}
      \wk{c}{d}
      \sync{d}{e}
      \wk[out=-165,in=-15]{e}{a}
      % \rfi{a}{b}
      % \bob{b}{c}
      % \fre{c}{d}
      % \bob{d}{e}
      % \coe[out=-165,in=-15]{e}{a}
    \end{tikzinline}}
\end{gather*}
We discussed two approaches to this problem in \textsection\ref{sec:arm}.
% The solution we have adopted is to allow an acquiring read to be downgraded
% to a relaxed read when it is preceded (sequentially) by a relaxed write that
% could fulfill it.  This solution allows executions that are not allowed under
% \armeight{} since we do not insist that the local relaxed write is actually
% read from.  This may seem counterintuitive, but we don't see a local way to
% be more precise.

% As a result, we use a different proof strategy for \armeight{}
% implementation, which does not rely on read elimination.  The proof idea uses
% a recent alternative characterization of \armeight{}
% \citep{alglave-git-alternate,arm-reference-manual}. %,armed-cats}.

\myparagraph{Redundant Read Elimination}

Contrary to the claim, redundant read elimination fails for \jjr{}.
We discussed redundant read elimination in \textsection\ref{sec:semreg}.
Consider JMM Causality Test Case 2, which we discussed there.
\begin{gather*}
  \PR{x}{r}\SEMI
  \PR{x}{s}\SEMI
  \IF{r{=}s}\THEN \PW{y}{1}\FI
  \PAR
  \PW{x}{y}
  \\
  \hbox{\begin{tikzinline}[node distance=1.5em]
      \event{a1}{\DR{x}{1}}{}
      \event{a2}{\DR{x}{1}}{right=of a1}
      \event{a3}{\DW{y}{1}}{right=of a2}
      \event{b1}{\DR{y}{1}}{right=3em of a3}
      \event{b2}{\DW{x}{1}}{right=of b1}
      \rf{a3}{b1}
      \po{b1}{b2}
      \rf[out=169,in=11]{b2}{a2}
      \rf[out=169,in=11]{b2}{a1}
    \end{tikzinline}}
\end{gather*}
Under the semantics of \jjr{}, we have
\begin{gather*}
  \PR{x}{r}\SEMI
  \PR{x}{s}\SEMI
  \IF{r{=}s}\THEN \PW{y}{1}\FI
  \\
  \hbox{\begin{tikzinline}[node distance=1.5em]
      \event{a1}{\DR{x}{1}}{}
      \event{a2}{\DR{x}{1}}{right=of a1}
      \event{a3}{1\EQ1\land1\EQ x \land x\EQ1 \land x=x\mid\DW{y}{1}}{right=of a2}
    \end{tikzinline}}
\end{gather*}
The precondition of $\DWP{y}{1}$ is \emph{not} a tautology, and therefore
redundant read elimination fails.
(It is a tautology in
\begin{math}
  \PR{x}{r}\SEMI
  \LET{s}{r}\SEMI
  \IF{r{=}s}\THEN \PW{y}{1}\FI
\end{math}.)
\jjr{\textsection3.1} incorrectly stated that the precondition of
$\DWP{y}{1}$ was $1\EQ1\land x\EQ x$.  

\myparagraph{Parallel Composition}

In \jjr{\textsection2.4}, parallel composition is defined allowing coalescing
of events.  Here we have forbidden coalescing.  This difference appears to be
arbitrary.  In \jjr{}, however, there is a mistake in the handling of
termination actions.  The predicates should be joined using $\land$, not
$\lor$.

\myparagraph{Read-Modify-Write Actions}

In \jjr{}, the atomicity axioms \ref{pom-rmw-atomic} erroneously applies only to
overlapping writes, not overlapping reads.  The difficulty can be seen in
\refex{ex:rmw-33}.

In addition, \jjr{} uses $\sLOAD{}{}$ instead of $\sLOADP{}{}$ when
calculating of dependency for \RMW{}s.  For a discussion, see the example at
the end of \textsection\ref{sec:rmw}.

\myparagraph{Definition of Data Race}

The definition of data race is wrong in \jjr{}.  It should require that that
at least one action is relaxed.

\section{A Note on Mixed-Mode Data Races}

In preparing this paper, we came across the following example, which appears
to invalidate Theorem 4.1 of \cite{DBLP:conf/ppopp/DongolJR19}.
\begin{gather}
  \nonumber
  \PW{x}{1}\SEMI
  \PW[\mREL]{y}{1}\SEMI
  \PR[\mACQ]{x}{r}
  \PAR
  \IF{\PR[\mACQ]{y}{}}\THEN \PW[\mREL]{x}{2}\FI
  \\
  \tag{\P}
  \label{mix1}
  \hbox{\begin{tikzinline}[node distance=1.5em]
      \event{a1}{\DW{x}{1}}{}
      \raevent{a2}{\DW[\mREL]{y}{1}}{right=of a1}
      \raevent{a3}{\DR[\mACQ]{x}{1}}{right=of a2}
      \raevent{b1}{\DR[\mACQ]{y}{1}}{right=3em of a3}
      % \raevent{b1}{\DR[\mACQ]{y}{1}}{below=of a1}
      \raevent{b2}{\DW[\mREL]{x}{2}}{right=of b1}
      \sync{a1}{a2}
      \rf[out=20,in=160]{a1}{a3}
      \rf[out=20,in=160]{a2}{b1}
      \wk[out=-20,in=-160]{a3}{b2}
      \sync{b1}{b2}
      % \node(ai)[left=3em of a1]{};
      % \bgoval[yellow!50]{(ai)}{P}
      % \bgoval[pink!50]{(a1)(a2)(b1)(b2)}{P'\setminus P}
      % \bgoval[green!10]{(a3)}{P'''\setminus P'}
    \end{tikzinline}}
  \\
  \nonumber
  %\label{mix2}
  \hbox{\begin{tikzinline}[node distance=1.5em]
      \event{a1}{\DW{x}{1}}{}
      \raevent{a2}{\DW[\mREL]{y}{1}}{right=of a1}
      \raevent{a3}{\DR[\mACQ]{x}{2}}{right=of a2}
      \raevent{b1}{\DR[\mACQ]{y}{1}}{right=3em of a3}
      \raevent{b2}{\DW[\mREL]{x}{2}}{right=of b1}
      \sync{a1}{a2}
      \rf[out=20,in=160]{a2}{b1}
      \rf[out=160,in=20]{b2}{a3}
      \sync{b1}{b2}
    \end{tikzinline}}
\end{gather}
The program is data-race free.  The two executions shown are the only
top-level executions that include $\DWP[\mREL]{x}{2}$.

Theorem 4.1 of \cite{DBLP:conf/ppopp/DongolJR19} is stated by extending
execution sequences.  In the terminology of
\cite{DBLP:conf/ppopp/DongolJR19}, a read is \emph{$L$-weak} if it is
sequentially stale.  Let
\begin{math}
  \rho=\DWP{x}{1}\allowbreak
  \DWP[\mREL]{y}{1}\allowbreak
  \DRP[\mACQ]{y}{1}\allowbreak
  \DWP[\mREL]{x}{2}
\end{math}
be a sequence and
\begin{math}
  \alpha=\DRP[\mACQ]{x}{1}.
\end{math}
$\rho$ is $L$-sequential and $\alpha$ is $L$-weak in $\rho\alpha$.  But there
is no execution of this program that includes a data race, contradicting the
theorem.  The error seems to be in Lemma A.4 of
\cite{DBLP:conf/ppopp/DongolJR19}, which states that if $\alpha$ is $L$-weak
after an $L$-sequential $\rho$, then $\alpha$ must be in a data race.  That
is clearly false here, since $\DRP[\mACQ]{x}{1}$ is stale, but the program is
data race free.

In proving the SC-LDRF result in \jjr{\textsection8}, we noted that our proof
technique is more robust than that of \cite{DBLP:conf/ppopp/DongolJR19},
because it limits the prefixes that must be considered.  In \eqref{mix1}, the
induction hypothesis requires that we add $\DRP[\mACQ]{x}{1}$ before
$\DWP[\mREL]{x}{2}$ since $\DRP[\mACQ]{x}{1}\xwk\DWP[\mREL]{x}{2}$.  In
particular,
\begin{gather*}
  \hbox{\begin{tikzinline}[node distance=1.5em]
      \event{a1}{\DW{x}{1}}{}
      \raevent{a2}{\DW[\mREL]{y}{1}}{right=of a1}
      % \raevent{a3}{\DR[\mACQ]{x}{1}}{right=of a2}
      \raevent{b1}{\DR[\mACQ]{y}{1}}{right=3em of a3}
      % \raevent{b1}{\DR[\mACQ]{y}{1}}{below=of a1}
      \raevent{b2}{\DW[\mREL]{x}{2}}{right=of b1}
      \sync{a1}{a2}
      % \rf[out=20,in=160]{a1}{a3}
      \rf[out=20,in=160]{a2}{b1}
      % \wk[out=-20,in=-160]{a3}{b2}
      \sync{b1}{b2}
      % \node(ai)[left=3em of a1]{};
      % \bgoval[yellow!50]{(ai)}{P}
      % \bgoval[pink!50]{(a1)(a2)(b1)(b2)}{P'\setminus P}
      % \bgoval[green!10]{(a3)}{P'''\setminus P'}
    \end{tikzinline}}
\end{gather*}
is not a downset of \eqref{mix1}, because
$\DRP[\mACQ]{x}{1}\xwk\DWP[\mREL]{x}{2}$.  As we noted in \jjr{\textsection8},
this affects the inductive order in which we move across pomsets, but does
not affect the set of pomsets that are considered.  In particular,
\begin{gather*}
  \hbox{\begin{tikzinline}[node distance=1.5em]
      \event{a1}{\DW{x}{1}}{}
      \raevent{a2}{\DW[\mREL]{y}{1}}{right=of a1}
      % \raevent{a3}{\DR[\mACQ]{x}{1}}{right=of a2}
      \raevent{b1}{\DR[\mACQ]{y}{1}}{right=3em of a3}
      % \raevent{b1}{\DR[\mACQ]{y}{1}}{below=of a1}
      % \raevent{b2}{\DW[\mREL]{x}{2}}{right=of b1}
      \sync{a1}{a2}
      % \rf[out=20,in=160]{a1}{a3}
      \rf[out=20,in=160]{a2}{b1}
      % \wk[out=-20,in=-160]{a3}{b2}
      % \sync{b1}{b2}
      % \node(ai)[left=3em of a1]{};
      % \bgoval[yellow!50]{(ai)}{P}
      % \bgoval[pink!50]{(a1)(a2)(b1)(b2)}{P'\setminus P}
      % \bgoval[green!10]{(a3)}{P'''\setminus P'}
    \end{tikzinline}}
\end{gather*}
is a downset of \eqref{mix1}.

\section{Additional RMW Examples}
\begin{example}
  This definition ensures atomicity, disallowing executions such as
  \cite[Ex.~3.2]{DBLP:journals/pacmpl/PodkopaevLV19}:
  \begin{gather*}
    % \taglabel{RMW1}
    \begin{gathered}
      \PW{x}{0}\SEMI \PFADD[\mRLX][\mRLX]{x}{s}{1}
      \PAR
      \PW{x}{2}\SEMI \PR{x}{s}
      \\
      \hbox{\begin{tikzinline}[node distance=1.5em]
          \event{a2}{\DR{x}{0}}{}
          \event{a1}{\DW{x}{0}}{left=of a2}
          \rf{a1}{a2}
          \event{a3}{\DW{x}{2}}{right=of a2}
          \wk{a2}{a3}
          \event{b2}{\DW{x}{1}}{right=of a3}
          \event{b3}{\DR{x}{1}}{right=of b2}
          \rmw[out=-15,in=-165]{a2}[below]{b2}
          \wk{a3}{b2}
          \rf{b2}{b3}
          \liftrmw[out=165,in=15]{a3}{a2}
        \end{tikzinline}}
    \end{gathered}
  \end{gather*}
  By \ref{pom-rmw-atomic1}, since $\DWP{x}{2}\xwk\DWP{x}{1}$, it must be that
  $\DWP{x}{2}\xwk\DRP{x}{0}$, creating a cycle.
\end{example}

\begin{example}
  \label{ex:rmw-33}
  Two successful \RMW{}s cannot see the same write:
  \begin{gather*}
    \begin{gathered}
      \PW{x}{0}\SEMI (\FADD^{\mRLX,\mRLX}(\aLoc,1)\PAR \FADD^{\mRLX,\mRLX}(\aLoc,1))
      \\
      \hbox{\begin{tikzinline}[node distance=1.5em]
          \event{i}{\DW{x}{0}}{}
          \event{a1}{a{:}\DR{x}{0}}{right=3em of i}
          \event{a2}{b{:}\DW{x}{1}}{right=of a1}
          \event{b1}{c{:}\DR{x}{0}}{right=3em of a2}
          \event{b2}{d{:}\DW{x}{1}}{right=of b1}
          \rmw{a1}{a2}
          \rmw{b1}{b2}
          \rf{i}{a1}
          \rf[out=15,in=165]{i}{b1}
          \wk[out=-15,in=-165]{a1}{b2}
          \liftrmw[out=-15,in=-165]{a2}{b1}
          % \wk{a1}{b2}
          \wk{b1}{a2}
        \end{tikzinline}}
    \end{gathered}
  \end{gather*}
  The order from read-to-write is required by fulfillment.  
  Apply \ref{pom-rmw-atomic1} of the second \RMW{} to $a\xwk d$, we have that $a\xwk c$.  Subsequently
  applying \ref{pom-rmw-atomic2} of the first \RMW{}, we have $b \xwk c$, creating a cycle.
\end{example}

\begin{example}
  By using two actions rather than one, the definition allows examples such as the
  following, which is allowed by \armeight{} 
  \cite[Ex.~3.10]{DBLP:journals/pacmpl/PodkopaevLV19}:
  \begin{gather*}
    % \taglabel{RMW2}
    \begin{gathered}
      \PR{y}{r}\SEMI
      \PW{z}{r}
      \PAR
      \PR{z}{r}\SEMI
      \PW{x}{0}\SEMI
      \PFADD[\mRLX][\mREL]{x}{s}{1} \SEMI
      \PW{y}{s}{+}1
      \\[-1ex]
      \hbox{\begin{tikzinline}[node distance=1.5em]
          \event{a1}{\DR{y}{1}}{}
          \event{a2}{\DW{z}{1}}{right=of a1}
          \po{a1}{a2}
          \event{b1}{\DR{z}{1}}{right=3em of a2}
          \rf{a2}{b1}
          \event{b2}{\DW{x}{0}}{right=of b1}
          \event{b3}{\DR{x}{0}}{right=of b2}
          \rf{b2}{b3}
          \event{b4}{\DWRel{x}{1}}{right=2em of b3}
          \rmw{b3}{b4}
          \event{b5}{\DW{y}{1}}{right=of b4}
          \sync[out=-15,in=-165]{b1}{b4}
          \po[out=-20,in=-160]{b3}{b5}
          \rf[out=170,in=10]{b5}{a1}
        \end{tikzinline}}
    \end{gathered}
  \end{gather*}
\end{example}

\begin{example}
  Consider the \textsc{cdrf} example from \cite{DBLP:conf/pldi/LeeCPCHLV20}:
  \begin{gather*}
    \begin{gathered}
      \begin{aligned}
        &\PFADD[\mACQ][\mREL]{x}{r}{1}\SEMI \IF{r{=}0}\THEN \PW{y}{1} \FI
        \\\PAR\;\;&
        \PFADD[\mACQ][\mREL]{x}{r}{1}\SEMI \IF{r{=}0}\THEN \IF{y}\THEN \PW{x}{0} \FI \FI
      \end{aligned}
      \\
      \hbox{\footnotesize\begin{tikzinline}[node distance=1.5em]
          \raevent{a1}{\DR[\mACQ]{x}{0}}{}
          \raevent{a1b}{\DW[\mREL]{x}{1}}{right=of a1}
          \event{a2}{\DW{y}{1}}{right=of a1b}
          \raevent{b0}{\DR[\mACQ]{x}{0}}{right=3em of a2}
          \raevent{b0b}{\DW[\mREL]{x}{1}}{right=of b0}
          \event{b1}{\DR{y}{1}}{right=of b0b}
          \event{b2}{\DW{x}{0}}{right=of b1}
          \rmw{a1}{a1b}
          \rmw{b0}{b0b}
          \rf[out=-13,in=-163]{a2}{b1}
          \sync[out=20,in=160]{a1}{a2}
          \sync[out=20,in=160]{b0}{b1}
          \po{b1}{b2}
          \rf[out=-165,in=-12]{b2}{a1}
        \end{tikzinline}}
    \end{gathered}
  \end{gather*}
\end{example}
\begin{example}
  Consider this example from \cite[\textsection C]{DBLP:conf/pldi/LeeCPCHLV20}:
  \begin{gather*}
    \begin{gathered}
      \begin{aligned}
        &\PCAS[\mRLX][\mRLX]{x}{r}{0}{1}\SEMI \IF{r{\leq}1}\THEN \PW{y}{1} \FI
        \\\PAR\;\;&
        \PCAS[\mRLX][\mRLX]{x}{r}{0}{2}\SEMI \IF{r{=}0}\THEN \IF{y}\THEN \PW{x}{0} \FI \FI
      \end{aligned}
      \\
      \hbox{\footnotesize\begin{tikzinline}[node distance=1.5em]
          \event{a1}{\DR{x}{0}}{}
          \event{a1b}{\DW{x}{1}}{right=of a1}
          \event{a2}{\DW{y}{1}}{right=of a1b}
          \event{b0}{\DR{x}{0}}{right=3em of a2}
          \event{b0b}{\DW{x}{2}}{right=of b0}
          \event{b1}{\DR{y}{1}}{right=of b0b}
          \event{b2}{\DW{x}{0}}{right=of b1}
          \rmw{a1}{a1b}
          \rmw{b0}{b0b}
          \rf[out=-13,in=-163]{a2}{b1}
          \po[out=20,in=160]{a1}{a2}
          \po[out=20,in=160]{b0}{b1}
          \po{b1}{b2}
          \rf[out=-165,in=-12]{b2}{a1}
        \end{tikzinline}}
    \end{gathered}
  \end{gather*}
\end{example}








\endinput




Precondition of $\DWP{y}{1}$ is $(r{=}s)$ in
\begin{math}
  \sem{\IF{r{=}s}\THEN \PW{y}{1}\FI}.
\end{math}
Predicate transformers for $\emptyset$ in $\sem{\PR{x}{r}}$ and $\sem{\PR{x}{s}}$ are
\begin{align*}
  \PREDP{(r{=}1 \lor r{=}x)\limplies\bForm[r/x]},
  \\
  \PREDP{(s{=}1 \lor s{=}x)\limplies\bForm[s/x]}.
\end{align*}
Combining the transformers, we have
\begin{displaymath}
  \PREDP{(r{=}1 \lor r{=}x)\limplies(s{=}1 \lor s{=}r)\limplies\bForm[s/x]}.
\end{displaymath}
Applying this to $(r{=}s)$, we have
\begin{displaymath}
  \PREDP{(r{=}1 \lor r{=}x)\limplies (s{=}1 \lor s{=}r)\limplies (r{=}s)},
\end{displaymath}
which is not a tautology.

Same problem occurs \jjr{}, where we have:
\begin{align*}
  \PREDP{\bForm[v/x,r] \land \bForm[x/r]},
  \\
  \PREDP{\bForm[v/x,s] \land \bForm[x/s]}.
\end{align*}
Combining the transformers, we have
\begin{displaymath}
  \PREDP{\bForm[v/x,r,s] \land \bForm [v/x,r][x/s] \land \bForm[x/r][v/x,s] \land \bForm[x/r,s]}.
\end{displaymath}
Applying this to $(r{=}s)$, we have
\begin{displaymath}
  \PREDP{v{=}v \land v{=}x \land x{=}v \land x{=}x},
\end{displaymath}
which is not a tautology.

The semantics here allows this by coalescing:
\begin{gather*}
  \PR{x}{r}\SEMI
  \PR{x}{s}\SEMI
  \IF{r{=}s}\THEN \PW{y}{1}\FI
  \PAR
  \PW{x}{y}
  \\
  \hbox{\begin{tikzinline}[node distance=1.5em]
      \event{a1}{\DR{x}{1}}{}
      \event{a3}{\DW{y}{1}}{right=of a1}
      \event{b1}{\DR{y}{1}}{right=3em of a3}
      \event{b2}{\DW{x}{1}}{right=of b1}
      \rf{a3}{b1}
      \po{b1}{b2}
      \rf[out=169,in=11]{b2}{a1}
    \end{tikzinline}}
\end{gather*}

In \jjr{\textsection2.6} the semantics of read is defined as follows:
\begin{align*}
  \sem{\PR[\amode]{\aLoc}{\aReg}\SEMI \aCmd} & \eqdef \textstyle\bigcup_\aVal\;
  (\DRmode\aLoc\aVal) \prefix \sem{\aCmd} [\aLoc/\aReg]
\end{align*}
The definition of prefixing$((\aForm \mid \aAct) \prefix \aPSS)$ has several clauses.
The most relevant are as follows, where $\bEv$ is the new event labeled with
$(\aForm \mid \aAct)$ and $\aEv$ is an event from $\aPSS$:
\begin{description}
\item[{\labeltextsc[P4c]{(P4c)}{4c}}]
  If $\bEv$ reads $\aVal$ from $\aLoc$ then either $\aEv=\bEv$ or
  $\labelingForm'(\aEv) \rimplies \labelingForm(\aEv)[\aVal/\aLoc]$.
\item[{\labeltextsc[P5a]{(P5a)}{5a}}]\labeltextsc[P5]{}{5}%
  If $\bEv$ reads and $\aEv$ writes then either $\labelingForm'(\aEv) \rimplies \labelingForm(\aEv)$ or $\bEv\le'\aEv$.
  % \item[{\labeltextsc[P5b]{(P5b)}{5b}}]
  %   If $\bEv$ and $\aEv$ are in conflict then $\bEv\le'\aEv$.
\end{description}

We have discovered two issues with this definition.

The first issue concerns the substitution $[\aLoc/\aReg]$.  It should be
$[\aReg/\aLoc]$.  We noticed this error while developing the alternative
characterization presented here.  The error causes redundant read elimination
to fail in \jjr{}.  As a result, common subexpression elimination also fails.
The problem can be seen in \ref{TC2}.
\begin{gather*}
  \taglabel{TC2}
  \PR{x}{r}\SEMI
  \PR{x}{s}\SEMI
  \IF{r{=}s}\THEN \PW{y}{1}\FI
  \PAR
  \PW{x}{y}
\end{gather*}
% In \jjr{\textsection3.1},
We claimed that \ref{TC2} allowed the following
execution:
\begin{gather*}
  \hbox{\begin{tikzinline}[node distance=1.5em]
      \event{a1}{\DR{x}{1}}{}
      \event{a2}{\DR{x}{1}}{right=of a1}
      \event{a3}{\DW{y}{1}}{right=of a2}
      % \po{a2}{a3}
      % \po[out=15,in=165]{a1}{a3}
      \event{b1}{\DR{y}{1}}{right=3em of a3}
      \event{b2}{\DW{x}{1}}{right=of b1}
      \rf{a3}{b1}
      \po{b1}{b2}
      \rf[out=169,in=11]{b2}{a2}
      \rf[out=169,in=11]{b2}{a1}
    \end{tikzinline}}
\end{gather*}
But this execution is not possible using the semantics of \jjr{}:
$\DWP{y}{1}$ has precondition $r{=}s$ in
\begin{math}
  \sem{\IF{r{=}s}\THEN \PW{y}{1}\FI}.
\end{math}
Given the lack of order in the execution, the precondition of $\DWP{y}{1}$
must entail $r{=}1\land r{=}x$ in 
\begin{math}
  \sem{\PR{x}{s}\SEMI
    \IF{r{=}s}\THEN \PW{y}{1}\FI}.
\end{math}
\ref{4c} imposes $r{=}1$, and \ref{5a} imposes $r{=}x$.  Adding the second
read, the precondition of $\DWP{y}{1}$ must entail both $1{=}1\land 1{=}x$
and also $x{=}1\land x{=}x$.  This can be simplified to $x{=}1$.  This leaves
a requirement that must be satisfied by a preceding write.  Since the
preceding write is the initialization to $0$, the requirement cannot be
satisfied, and the execution is impossible.\footnote{In \jjr{} we ignore the
  middle terms, mistakenly simplifying this to $1{=}1\land x{=}x$.
  Correcting the error, the attempted execution is:
  \begin{gather*}
    \hbox{\begin{tikzinline}[node distance=1.5em]
        \event{a1}{\DR{x}{1}}{}
        \event{a2}{\DR{x}{1}}{right=of a1}
        \event{a3}{\DW{y}{1}}{right=of a2}
        \po{a2}{a3}
        \po[out=-20,in=-160]{a1}{a3}
        \event{b1}{\DR{y}{1}}{right=3em of a3}
        \event{b2}{\DW{x}{1}}{right=of b1}
        \rf{a3}{b1}
        \po{b1}{b2}
        \rf[out=169,in=11]{b2}{a2}
        \rf[out=169,in=11]{b2}{a1}
      \end{tikzinline}}
  \end{gather*}}

The substitution $[\aLoc/\aReg]$ leaves the obligation on $\aLoc$ to be
fulfilled by the preceding write.  Thus, the read does not update the
\emph{value} of $\aLoc$ in subsequent predicates.  The substitution
$[\aReg/\aLoc]$, instead, does update the value of $\aLoc$, thus removing any
obligation on $\aLoc$ for preceding code.

In order to write this, we must update the definition of prefixing reads to
include the register.  Then \ref{4c} becomes:
\begin{description}
\item[\textsc{(p4c)}] If $\bEv$ reads $\aVal$ from $\aLoc$ then either
  $\aEv=\bEv$ or $\labelingForm'(\aEv) \rimplies \labelingForm(\aEv)[\aVal/\aReg]$.
\end{description}

We can then reason with \ref{TC2} as follows: $\DWP{y}{1}$ has precondition
$r{=}s$ in
\begin{math}
  \sem{\IF{r{=}s}\THEN \PW{y}{1}\FI}.
\end{math}
To avoid introducing order in the execution, the precondition of $\DWP{y}{1}$
must entail $r{=}1\land r{=}s$ in 
\begin{math}
  \sem{\PR{x}{s}\SEMI
    \IF{r{=}s}\THEN \PW{y}{1}\FI}.
\end{math}
\ref{4c} imposes $r{=}1$, and \ref{5a} imposes $r{=}x$.  Adding the second
read, the precondition of $\DWP{y}{1}$ must entail both $1{=}1\land 1{=}x$
and also $x{=}1\land x{=}x$.  This can be simplified to $x{=}1$.  This leaves
a requirement that must be satisfied by a preceding write.


With read elimination, the rule for relaxed reads is as follows:
\begin{align*}
  \sem{\PR{\aLoc}{\aReg} \SEMI \aCmd} &\eqdef
  \sem{\aCmd}[\aLoc/\aReg]
  \cup
  \textstyle\bigcup_\aVal\;
  \DRP{\aLoc}{\aVal} \prefix_{\aReg} %\Rdis{\aLoc}{\aVal}
  \sem{\aCmd}[\aReg/\aLoc]
\end{align*}
It is interesting to note that the substitution is $[\aLoc/\aReg]$ on
eliminated reads, and $[\aReg/\aLoc]$ on non-eliminated reads.  Intuitively,
the subsequent value of $\aLoc$ is fixed by an explicit read, but not for an
eliminated read.  In the latter case, the value is fixed by some preceding
action.  The preceding action may itself be a read. This gives rise to some
fear that we might introduce thin-air reads, since we do not enforce
read-read coherence.  But this is not the case.  Consider the following example:
\begin{gather*}
  \PR{x}{r}\SEMI
  \PR{x}{s}\SEMI
  \PW{y}{s}
  \PAR
  \PW{x}{y}
  \\
  \hbox{\begin{tikzinline}[node distance=1.5em]
      \event{a1}{\DR{x}{1}}{}
      \event{a2}{\DR{x}{1}}{right=of a1}
      \event{a3}{\DW{y}{1}}{right=of a2}
      % \po{a2}{a3}
      \po[out=-20,in=-160]{a1}{a3}
      \event{b1}{\DR{y}{1}}{right=3em of a3}
      \event{b2}{\DW{x}{1}}{right=of b1}
      \rf{a3}{b1}
      \po{b1}{b2}
      \rf[out=169,in=11]{b2}{a2}
      \rf[out=169,in=11]{b2}{a1}
    \end{tikzinline}}
  \\
  \hbox{\begin{tikzinline}[node distance=1.5em]
      \event{a1}{\DR{x}{1}}{}
      \internal{a2}{\DR{x}{1}}{right=of a1}
      \event{a3}{\DW{y}{1}}{right=of a2}
      % \po{a2}{a3}
      \po[out=-20,in=-160]{a1}{a3}
      \event{b1}{\DR{y}{1}}{right=3em of a3}
      \event{b2}{\DW{x}{1}}{right=of b1}
      \rf{a3}{b1}
      \po{b1}{b2}
      % \rf[out=169,in=11]{b2}{a2}
      \rf[out=169,in=11]{b2}{a1}
    \end{tikzinline}}
\end{gather*}
But this is not a problem, since fulfillment requires that $\DWP{x}{1}$
precede both reads of $x$.



